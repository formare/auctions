
\newcommand{\emp}{\emptyset}
\newcommand{\N}{\mathbb{N}}
\newcommand{\U}{\bigcup}
\newcommand{\Z}{\mathbb{Z}}
\newcommand{\C}{\subseteq}
\newcommand{\pair}[2]{\left( {#1}, {#2} \right) }
\newcommand{\powset}[1]{2^{#1}}
\newcommand{\strip}[1]{e_{#1}}
\newcommand{\relcomp}{\bullet}
\newcommand{\funccomp}{\circ}
\newcommand{\Classes}[2]{{#1}/{#2}}
\newcommand{\peel}[1]{\}\{_{#1}}
\newcommand{\toClass}[1]{\pi_{#1}}
\newcommand{\comp}[2]{$\left( #1, #2 \right)$-compatible}
\newcommand{\new}[2]{\mathcal{P}_{#1,#2}}
\newcommand{\id}[1]{\mathcal{I}_{#1}}
\newcommand{\restrict}[2]{{\left. #1 \right|}_{#2}}
\newcommand{\quotient}[3]{\frac{\hphantom{#2}#1\hphantom{#3}}{#2\hphantom{#1}#3}}
\newcommand{\cartprod}{\times}
\newcommand{\sdiff}{\backslash}
\newcommand{\im}[2]{#1 \left[ #2 \right]}
\newcommand{\iter}[2]{{#1}^{\left({#2} \right)}}
\newcommand{\inv}[1]{\iter{#1}{-1}}
\newcommand{\query}[1]{\marginnote{\raggedright\footnotesize\itshape\hrule\smallskip{#1}\smallskip\hrule}}
%to print a margin note on a side: \query{...} 
%to print it on the other side of page: \reversemarginpar\query{ ... }\normalmarginpar
% \renewcommand{\query}[1]{} % Uncomment to turn off marginnotes
\newcommand{\rnote}[1]{\query{#1}}
\newcommand{\lnote}[1]{\reversemarginpar\query{#1}\normalmarginpar}
\newcommand{\alter}[3]{{#1}_{#2}^{#3}}
\newcommand{\me}{\Name{}~\Surname{}}
\newcommand{\I}{Isabelle}
\newcommand{\M}{Mizar}
\newcommand{\Phdtitle}{A simplified framework for first-order languages and its formalization in \M{}}
\newcommand{\Name}{Marco~B.}
\newcommand{\Surname}{Caminati}
\newcommand{\wwwMbc}{http://caminati.net.tf/}
\newcommand{\paste}{\lhd}
\newcommand{\mapsFromTo}[2]{{#2}^{#1}}
\newcommand{\R}{\mathbb{R}}
\newcommand{\per}{}

%LaTeX starts here%%%%%%%%%%%%%%%%%%%%%%%
%%%%%%%%%%%%%%%%%%%%%%%%%%%%%%%%%%%%%%%%%

\documentclass[oneside
% twoside
% ,binding=0.6cm
% ,draft
]
{article}

\usepackage{amsmath}
% \usepackage{amsmidx}
\usepackage{amsthm}
\usepackage{amssymb} %loads amsfonts
% \usepackage{hyperref} % Note that hyperref is much slower than the url package.
\usepackage[bibstyle=alphabetic,citestyle=alphabetic,firstinits=true]{biblatex}
\usepackage[]{marginnote} %mbc
\usepackage[]{url}

\theoremstyle{plain}
\newtheorem{Cor}{Corollary}[subsection]
\newtheorem{Lm}[Cor]{Lemma}
\newtheorem{Prop}[Cor]{Proposition}
\newtheorem{Thm}[Cor]{Theorem}
\theoremstyle{definition}
\newtheorem{Def}[Cor]{Definition}
\newtheorem{Rem}[Cor]{Remark}
\newtheorem{Not}[Cor]{Notation}
\newtheorem{Ex}[Cor]{Example}
\DeclareMathOperator{\rng}{rng}
\DeclareMathOperator{\dom}{dom}
\urldef{\WwwKernel}\url{http://en.wikipedia.org/wiki/Equivalence_class#Invariants}
\bibliography{colin}
\bibliography{mbc}





%Document starts here
%%%%%%%%%%%%%%%%%%%%%%%%%%%%%%%%%%%%%%%%%

\begin{document}


\section{Introduction}

Our purpose is the formalization of \cite[Proposition~2]{mas-04}, whose proof's wordiness hide some substantial difficulties when pouring it into formal symbols.
A major one is that, after showing the independence of a function $t_i^L$ on the $i$-th component of a vector $b$, it is inferred that a function $ t_i $ depending only on the remaining components can be introduced and used instead.
Besides the minor problem of defining how the remaining components can be extracted, the formal way to do so is quotientation by a suitable equivalence relation.%
\footnote{In \M{}, the corresponding gear is already available. 
With respect to \I{}, maybe \cite{slotosch1997higher} will give hints.
}%
A related problem is how the thesis is spelled out: $p_i$ is equated to a branched function whose side conditions read `if buyer $i$ wins' and `if buyer $i$ loses'.
How to render that? Luckily, thanks to the quotientation approach, a natural way will show up. 

Secondarily, the mathematical definition of payoff is never spelled out.
\section{General results}

This section recalls the abstract notions of compatibility (a generalization of morphism, or class invariance, see \WwwKernel{}) and of quotient. 
These, although in varying forms and generality, are standard/mainstream mathematical matters, and are expected to be present in any decent library; any not including those should.
The treatment here is somehow low-level and is leaning towards set theoretical foundations, but I hope that anyhow it will give the reader, besides the writer, a guide towards real formalization.
It is an excerpt of my Ph.D. thesis, which indeed described a \M{} formalization.

\subsection{Notations}

\label{RefNotationZero}
\hfill
\begin{enumerate}
% \item
% $\card{X}$ is the cardinality of the set $X$.
\item
$X \cartprod Y$ is the cartesian product of the sets $X$ and $Y$:
\begin{align*}
X \cartprod Y = & \left\{ \pair{x}{y}: x \in X, y \in Y \right\}.
\end{align*}
\item
$\N$, $\Z$ are the sets of natural numbers (including $0 = \emptyset$) and of the integers, respectively. We also write $\Z^+$ for $\N \sdiff \left\{ 0 \right\}$.
% \rnote{0914: Aggiunto $Z^+$.}
\item
% \lnote{Aggiunta precisazione su $\projL$ versus $\dom$.}
$\dom P$ and $\rng P$ denote the domain and range of a given relation $P$.
% They are synonyms, adopted only when applied to relations, for $\projL \left( P \right)$ and $\projR \left( P \right)$ respectively.
\item
We will use the terms function, map and mapping interchangeably.
\item
\label{RefEq31}
$\mapsFromTo{X}{Y}$ is the set of the maps from $X$ into $Y$.
\item
$\powset{X}$ is the power set of $X$.
\item
$\peel{X}$ is the map: 
\begin{align*}
\left\{ x \right\} \mapsto x \in X.
\end{align*}
\item
$\id{X}$ is the identity map on the set $X$: $\id{X} := \U_{x \in X} \left\{ x \right\} \cartprod \left\{ x \right\}$.
\end{enumerate}

\begin{Not}
Consider a relation $P$ and a set $X$. 
We write
$ \restrict{P}{X}$ for the restriction of $P$ to $X$:
\begin{align*}
\restrict{P}{X} := \left( X \cartprod \rng P \right) \cap P,
\end{align*}
and 
$\im{P}{X}$ for the set of those  elements of $\rng{P}$ corresponding through $P$ to some element of $X$:
\begin{align*}
\im{P}{X} := \rng \left( \restrict{P}{X} \right).
\end{align*}
\end{Not}

\begin{Not}
\label{RefNotationCompositions}
$\relcomp$ is the infix symbol for the composition of relations:
$ \im{\left( Q \relcomp P \right)}{X} = \im{P}{\im{Q}{X}} $.
\\
$\funccomp$ is the infix symbol for the composition of functions:
$ g \funccomp f: x \mapsto g \left( f \left( x \right) \right).$
\end{Not}

\begin{Rem}
Note that $\relcomp$ and $\funccomp$ operate in reverse order: $ f \relcomp g = g \funccomp f$. 
\end{Rem}

\begin{Not}
% \label{RefNotationIter}
If $P$ is a relation such that $\rng P \C \dom P $, we can refer to the $n$-th iteration of $P$ for any given $n \in \N$.
We write it as
\begin{align*}
\iter{P}{n}.
\end{align*}
\end{Not}
\begin{Not}
Given any relation $P$, it is natural to denote with $\inv{P}$ its inverse relation, and with $\iter{P}{-n}$ the relation $ \iter{\inv{P}}{n}$.
\end{Not}

\begin{Not}[`Functional pasting with right-hand precedence{}']
\label{RefDefPaste}
Given relations $Q$, $P$, set
\begin{align*}
Q \paste P := Q \sdiff \left( \dom P \cartprod \left( \rng Q \right) \right) \cup P.
\end{align*}
\end{Not}

\begin{Rem}
Given two functions $f$, $g$: 
\begin{itemize}
\item
$f \paste g$ is a function;
\item
if $f$ and $g$ agree on $\dom f \cap \left( \dom g \right)$, then $ f \paste g$ $= f \cup g $.
\end{itemize}
\end{Rem}

\begin{Not}[Pointwise function alteration]
Given $x,y$ and a relation $P$, denote:
$\alter{P}{x}{y} := P \paste \left( \left\{ x \right\}\cartprod \left\{ y \right\} \right)$. 
\end{Not}

\subsection{Quotients}
\label{RefSectQuotients}
\begin{Def}
\label{RefDefCompatible}
Let $P, Q$ be relations, $f$ be a function. We say that $f$ is \comp{P}{Q} if, given  
$\pair{x}{y} \in \dom f \cartprod \left( \dom f \right) \cap P$, it is  
$\pair{f\left( x \right)}{f \left( y \right)} \in Q$.
\end{Def}

\begin{Def}
% \lnote{20110930: In \M ho limitato questa definizione alle sole relazioni di equivalenza \ldots}
% \rnote{ \ldots probabilmente per lavorare meno, ora non ricordo.}
Given a non empty relation $P$, we consider the map
\begin{align*}
\toClass{P} : \dom P \ni x \mapsto \im{P}{\left\{ x \right\}} \in \powset{\rng P}.
\end{align*}
Given a set $X$ and a relation $P$ such that $X = \dom P$, we set
\begin{align*}
\Classes{X}{P} := \rng \left( \toClass {P} \right).
\end{align*}
\end{Def}

\begin{Rem}
If $P$ is an equivalence relation over $X$, $\Classes{X}{P}$ is the set of the equivalence classes of $P$ (hence a partition of $X$), and $\toClass{P}$ maps each element of the domain of $P$ to the unique equivalence class including it.
\end{Rem}

\begin{Def}[Quotient of a relation]
\label{RefDefQuotient}
% \lnote{Anche qui: in \M $P$ e $Q$ erano relazioni di equivalenza non vuote}
% \rnote{Mi sono limitato a conservare la richiesta di non vuotezza}
Let $O, P, Q$ be relations, with $P$ and $Q$ non empty. 
The quotient of $O$ by $\pair{P}{Q}$ is defined as:
\begin{align*}
\quotient{O}{P}{Q} := \left\{
\pair{p}{q} \in \rng \left( \toClass{P} \right) \cartprod \left( \rng \left( \toClass{Q} \right) \right) : 
% \exists x, y \st \quad \pair{x}{y} \in 
p \cartprod q \cap O \neq \emp 
\right\}.
\end{align*}
\end{Def}

\begin{Prop}
\label{RefThmCompatibleFunction}
% \lnote{Probabilmente si potrebbero indebolire le richieste su $E$, $F$, $f$.}
Let  $E, F$ be non empty equivalence relations.
\\
If $f \in$ 
$\mapsFromTo{\dom E}{\left( \dom F \right)}$ 
is \comp{E}{F}, then
\begin{align*}
\quotient{f}{E}{F} \in \mapsFromTo{\rng  \toClass{E} }{ \left( \rng  \toClass{F} \right)}.
\end{align*}
\end{Prop}

\begin{proof}
Set 
$
g := \quotient{f}{E}{F}
$.
Since $g \C \rng \toClass{E} \cartprod \rng \toClass{F}$ by \ref{RefDefQuotient}, it is $\rng g \C \rng \toClass{F}$, hence we are left with two points to prove:
\begin{enumerate}
\item
$ g $ is a function.
\item
$ \rng \toClass{E} \C \dom g $.
\end{enumerate}
The two corresponding proofs are given.
\begin{enumerate}
\item
Consider sets $X$, $Y_1$, $Y_2$ such that 
$ \left\{ \pair {X} {Y_1}, \pair{X} {Y_2} \right\} \C g$. 
The goal is to show $Y_1 = Y_2$.
By \ref{RefDefQuotient}, consider $x_1, x_2, y_1, y_2$ such that 
$\pair{x_1}{y_1} \in X \cartprod Y_1 \cap f$ and
$\pair{x_2}{y_2} \in X \cartprod Y_2 \cap f$.
Since $X$ is an equivalence class of $E$, this implies $\pair{x_1}{x_2} \in E$ which in turn, by \ref{RefDefCompatible}, gives $ \pair{y_1}{y_2} \in F$.
Hence $y_1$ and $y_2$ must belong to the same equivalence class of $F$, which gives $Y_1 = Y_2$.
\item
Let $X \in \rng \toClass{E}$. 
$X$ being an equivalence class of the non empty equivalence relation $E$, there is $x \in X \C \dom E$.
Set
\begin{align}
\label{RefEq32}
y := & f \left( x \right) \in \dom F
\\
\notag
Y := & \toClass{F} \left( y \right) \in \rng F.
\end{align}
Since $\pair {x} {y} \in f$ by \eqref{RefEq32}, and $y \in Y$, we draw $\pair {X}{Y} \in g$ by \ref{RefDefQuotient}.
\end{enumerate}
\end{proof}



% \end{document}



























\section{Application to auctions}

Let $I$ be a set (of indices), $R$ be an ordered ring (e.g., $\R{}$), and $x, p \in \mapsFromTo{
\left( 
\mapsFromTo{I}{R}
\right)
}{R}
$ satisfying 
\begin{align}
\label{RefWeakDom}
r \per x \left( b \right) - p \left( b \right) \leq r \per x \left( \alter{b}{i}{r} \right) - 
p \left( \alter {b}{i}{r} \right)
\end{align}
for any $b \in \mapsFromTo{I}{R}, r \in R$.

\begin{Def}
Consider the map 
$ \strip{i} := \mapsFromTo{I}{R} \ni b \mapsto 
\pair{b \sdiff \left( \left\{ i \right\} \cartprod \rng b \right)}{ x \left( b \right)} 
\in \mapsFromTo{I \sdiff {i}}{R} \cartprod \rng {x}, 
$
and denote with $E_i$ its kernel%
\footnote{See \WwwKernel{}.}%
% Consider the kernel of $\strip{i}$, 
, i.e., the equivalence relation given by $ b \sim b'$ iff 
$
b \sdiff \left( \left\{ i \right\} \cartprod \rng b \right)
=
b' \sdiff \left( \left\{ i \right\} \cartprod \rng b' \right)
$
and 
$ x \left( b \right) = x \left( b' \right)$.
% call $E_i$ its restriction to $D:=\im{\inv{x}}{\left\{ 0_R \right\}}$. 
\end{Def}

\begin{Lm}
\label{RefLmComp}
\eqref{RefWeakDom} implies that $p$ is \comp{E_i}{\id{R}}{}.
\end{Lm}

\begin{proof}
It is to be shown that, given $b$, $b'$ satisfying $ x \left( b \right) = x \left( b' \right)$ and  
$ b \sdiff \left(\left\{ i \right\} \cartprod \rng b \right) = b' \sdiff \left( \left\{ i \right\} \cartprod \rng b' \right) , $ 
we have 
$ p \left( b \right) = p \left( b' \right)$. 
This is easily done by applying \eqref{RefWeakDom} twice to get 
\begin{align*}
b'\left( i \right) \per x \left( b \right) - p \left( b \right) 
& \leq 
b'\left( i \right) \per x \left( b' \right) - p \left( b' \right)
\\ 
b\left( i \right) \per x \left( b' \right) - p \left( b' \right) 
& \leq 
b\left( i \right) \per x \left( b \right) - p \left( b \right)
\end{align*}
which, thanks to $x \left( b \right) = x \left( b' \right)$, reduce to
$ -p \left( b \right) \leq -p \left( b' \right) \leq -p \left( b \right).$
\end{proof}


% \begin{Def}
% $$
% \end{Def}


By \ref{RefLmComp} and \ref{RefThmCompatibleFunction}, $\quotient{p}{E_i}{\id{R}} \in
\mapsFromTo
 {\left(\rng{\toClass{E_i}} \right)}  
{ \left( \rng{\toClass{\left( \id{R}  \right) }} \right) },
$ 
so that 
$ 
% \mapsFromTo{I \sdiff \left\{ i \right\}}{R} \ni b \mapsto
\new{p}{i} := 
\toClass{
\left( 
\inv{\strip{i}}
 \right)
} \relcomp \quotient{p}{E_i}{\id{R}} \relcomp \peel{R}
$
maps $ \mapsFromTo{I \sdiff \left\{ i \right\}}{R} \cartprod \rng {x} $ to $R$ and $ p \left( b \right) = \new{p}{i} \left( \strip{i} \left( b \right) \right)$.
% We refer to this mapping as $ \mathcal{Q}_{p,i}$.
% $ \toClass{E_i} \relcomp \new{p}{i} \relcomp \peel{R}$ maps $ \mapsFromTo{I}{R}$ to $R$.

Above, we have shown that, if $i$ changes his bid, the price he pays does not change as long as the allocation of the good to $i$ does not change. 
This is a good part of \cite[Proposition~2]{mas-04}.
It remains to show that the winning price also has the property of being the sum of the losing price and of the second price.
This, of course, imposes requirements on $x$. 
A first requirement is that a bid vector whose $i$-th component is the second price is a sort of accumulation point:

\begin{Lm}
Consider $b \in \mapsFromTo{I}{R}, r \in R$, and assume that for any $\delta > 0_R$ there are $\delta > \epsilon_1, \epsilon_2 > 0_R$ such that
\begin{enumerate}
\item
$ \alter{b}{i}{b \left( i \right) - \epsilon_1} \in \im{\inv x}{\left\{ 0 \right\}},$ and
\item
$ \alter{b}{i}{b \left( i \right) + \epsilon_2} \in \im{\inv x}{\left\{ r \right\}} .$
\end{enumerate}
Then \eqref{RefWeakDom} implies that
\begin{align*}
\new{p}{i} \left( b \sdiff \left( \left\{ i \right\} \cartprod \rng b \right), r \right) 
=
r \per b \left( i \right) + 
\new{p}{i} \left( b \sdiff \left( \left\{ i \right\} \cartprod \rng b \right), 0_R \right).
\end{align*}
\end{Lm}

\begin{proof}
\ldots
\end{proof}

Now, the wanted completion of \cite[Proposition~2]{mas-04} is easily obtained by realizing that, upon requiring efficiency and truthfulness, the above accumulation properties are satisfied by any bid vector $b$ whose $i$-th component is the second price. 

\printbibliography

\end{document}

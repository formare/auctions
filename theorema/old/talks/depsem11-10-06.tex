\documentclass{beamer}
\usepackage{code}
\usepackage{tikz} % for TikZ and PGF, graphics packages compatible with beamer
\usetikzlibrary{intersections} % from 10/03/09 CVS build
\usepackage{beamerthemesplit}
\def\sTeX{s\TeX{}}
\def\theorema{{\rm Theorema}} %%%\sf TH$\sf\exists$OREM$\sf\forall$}}
\def\mathematica{{\rm Mathematica}}
\def\isabelle{{\rm Isabelle}}
\def\coq{{\rm Coq}}
\def\leo{{\rm Leo}}

\defbeamertemplate*{footline}{infolines theme}
{
\hspace*{2ex}    \insertframenumber{} / \inserttotalframenumber\hspace*{2ex} 
%   \insertpagenumber{} / \insertpresentationendpage \hspace*{2ex}
  \vskip1ex%
}

\usepackage{color}
\usepackage{bm}
\usepackage{txfonts}
\def\mcolor#1#2{\rule{0ex}{0ex}\color{#1}#2\color{black}{}}
%\setbeamercolor{title}{fg=red!80!black,bg=red!20!white}
\makeatletter % code block to allow custom labels to be cross-ref'ed; see comp.text.tex "customized display labels cross-ref'd"
\def\mycaption#1{\centerline{\textbf{Table 1.} #1}}
\newcommand{\displabel}[1]{(#1)\hfill}
\def\Defstrictfi{Def$_\strictfi$}
\def\DefD{Def$_D$}
\def\DefK{Def$_{\mathcal{K}}$}
\def\DefS{Def$_{\mathcal{S}}$}
\def\Deft{Def$_{\mathbf{t}}$}
\def\Defs{Def$_{\mathbf{s}}$}
\def\squeeze#1{\SperrRest#1\endlistxxx}
\def\sperrdist{0.1ex}
\def\endlistxxx{\endlistxxx}
\def\SperrRest{\afterassignment\SperrZeichen\let\next= }
\def\SperrZeichen{\ifx\next\endlistxxx\let\next\relax\kern\sperrdist
                  \else\next\kern-\sperrdist\let\next\SperrRest\fi\next}

\newcommand{\IffDef}{:\Leftrightarrow}
\def\Quant#1#2{\begin{array}[t]{@{}c@{}}#1 \\[-1ex] \scriptstyle #2 \end{array}\;}
\def\QuantCond#1#2#3{\begin{array}[t]{@{}c@{}}#1 \\[-1ex] \scriptstyle #2 \\[-1ex]\scriptstyle #3 \end{array}\;}
\def\ForAll#1{\Quant{\forall}{#1}}
\def\ForAllCond#1#2{\QuantCond{\forall}{#1}{#2}}
\def\Exists#1{\Quant{\exists}{#1}}
\def\ExistsCond#1#2{\QuantCond{\exists}{#1}{#2}}
\def\SetOf#1#2#3{\big\{#1\begin{array}[t]{@{}c@{}}| \\[-2pt] \scriptstyle #2 \end{array}#3\big\}}
\newcommand{\NN}{\mathbb{N}}

\newenvironment{casedistinction}{\left\{\begin{array}{l@{\hspace{2pt}\Leftarrow\hspace{2pt}}l}}{\end{array}\right.}

\newenvironment{flist}{\begin{array}{@{}ll@{}}}{\end{array}}

\newcommand{\lfEnv}[2]{#2 & \mbox{``#1''}}
\def\lf#1#2{\begin{displaymath}\mbox{(#1)\quad} #2\end{displaymath}}

\newenvironment{Env}[2]{\vspace{0.3ex}\par\noindent\textbf{#1}[``#2'',
  \begin{math}\begin{flist}}{\end{flist}]\end{math}\vspace{0.3ex}}

\newenvironment{Envany}[3]{\vspace{0.3ex}\par\noindent\textbf{#1}[``#2'',
  any[$#3$], %\\[0.5ex] \makebox[2mm]{}
  \begin{math}\begin{flist}}{\end{flist}]%
  \end{math}\vspace{0.3ex}}

\newenvironment{Envanylong}[3]{\vspace{0.3ex}\par\noindent\textbf{#1}[``#2'',
  any[$#3$], \\[0.5ex] \makebox[5mm]{}
  \begin{math}\begin{flist}}{\end{flist}]%
  \end{math}\vspace{0.3ex}}

\newenvironment{Envanylongvar}[3]{\vspace{0.3ex}\par\noindent\textbf{#1}[``#2'',
  any[$#3$], \\[0.5ex] %\makebox[5mm]{}
  \hspace*{-4ex}\begin{math}\begin{flist}}{\end{flist}]%
  \end{math}\vspace{0.3ex}}

\newenvironment{Envanywith}[4]{\vspace{0.3ex}\par\noindent \textbf{#1}[``#2'', any[$#3$], %
  with[$#4$], \\[0.5ex]\makebox[5mm]{}
  \begin{math}\begin{flist}}{\end{flist}]%
  \end{math}\vspace{0.3ex}\rmfamily}

\newenvironment{Envanywithlong}[4]{\vspace{0.3ex}\par\noindent \textbf{#1}[``#2'', any[$#3$], \\
  \makebox[3mm]{}with[$#4$], \\[0.5ex]\makebox[5mm]{}
  \begin{math}\begin{flist}}{\end{flist}]%
  \end{math}\vspace{0.3ex}\rmfamily}

\newenvironment{Envanybound}[4]{\vspace{0.3ex}\par\noindent \textbf{#1}[``#2'', any[$#3$], %
  bound[$#4$], \\[0.5ex]\makebox[5mm]{}
  \begin{math}\begin{flist}}{\end{flist}]%
  \end{math}\vspace{0.3ex}\rmfamily}

\newenvironment{Envanyboundmult}[4]{\vspace{0.3ex}\par\noindent \textbf{#1}[``#2'', any[$#3$], %
  bound[$#4$],
  \begin{displaymath}}{\end{displaymath}\vspace{0.3ex}\rmfamily}

\newenvironment{disp}
%   {\begin{list}{}%
%     {\renewcommand{\makelabel}{\displabel}%
%      \let\olditem=\item%
%      \renewcommand*{\item}[1][]{%<<add test for empty arg (ifmtarg.sty)
%         \protected@edef\@currentlabel{##1}%
%         \olditem[##1]}
%      \setlength{\leftmargin}{.5in}
%      \setlength{\rightmargin}{.5in}}}%
%   {\end{list}}
{\begin{description}\setlength{\leftmargin}{5in}\setlength{\rightmargin}{10in}}
{\end{description}}
  \makeatother
\begin{document}
\title{Pillage Games and Formal Proofs\\ Past Work, Future Plans}

\author[M.~Kerber, C.~Rowat]{\begin{tabular}{cc}Manfred Kerber$^1$ & Colin Rowat$^2$\\
\multicolumn{2}{c}{University of Birmingham}\\
Computer Science & Economics
\end{tabular}\rule[-5ex]{0ex}{5ex}
  $^1${\tt www.cs.bham.ac.uk/\~{}mmk}\\
  $^2${\tt www.socscistaff.bham.ac.uk/rowat}\\
}

\date{6 October 2011}

\begin{frame}
\titlepage
\end{frame}

\section{Overview}
\begin{frame}
\frametitle{Overview}
\mcolor{red}{Motivation:}
\begin{itemize}
\item Proofs in economics use typically undergraduate level proofs
\item Proofs in economics are error prone (just as in other theoretical fields)
\item Formalization should be achievable
\item Automation (or minimization of user interactions) as goal
\end{itemize}\pause

\mcolor{blue}{Outline}
\begin{itemize}
\item Basic Theory
\item Pseudo Algorithm
\item Examples
\item A Lemmas and Theorema
\item Plans
\item Summary
\end{itemize}
\end{frame}

\section{Basic Theory}
\begin{frame}
\frametitle{Power Function}
$\mathcal{X} \equiv \left\{ \left\{ x_i \right\}_{i \in I} \left|
    x_i \ge 0, \sum_{i \in I} x_i = 1 \right.\right\}$., the following
axioms can be defined. 
 A \mcolor{blue}{power function} $\pi$ satisfies\bigskip

\begin{description}[WC]
  \item[\mcolor{red}{WC}]\qquad if $C \subset C' \subseteq I$ then $\pi \left( C, \bm{x} \right) \le \pi \left( C', \bm{x} \right) \forall \bm{x} \in \mathcal{X}$;\pause

  \item[\mcolor{red}{WR}]\qquad if $y_i \ge x_i \forall i \in C \subseteq I$ then $\pi \left( C, \bm{y} \right) \ge \pi \left( C, \bm{x} \right)$; \pause and

  \item[\mcolor{red}{SR}]\qquad if $\emptyset \ne C \subseteq I$ and $y_i > x_i \forall i \in C$ then $\pi \left( C, \bm{y} \right) > \pi \left( C, \bm{x} \right)$.
\end{description}
\end{frame}

\begin{frame}
\frametitle{The Same in Theorema (WC)}
\begin{description}[WC]
  \item[\mcolor{red}{WC}]\qquad if $C \subset C' \subseteq I$ then $\pi \left( C, \bm{x} \right) \le \pi \left( C', \bm{x} \right) \forall \bm{x} \in \mathcal{X}$
\end{description}\bigskip

\mcolor{blue}{
\begin{Envanybound}{Definition}{WC}{\pi,n}{\text{allocation}_n[x]}
  \text{WC}[\pi,n]\IffDef n\in\NN\land\ForAllCond{C1,C2}{C1\subset C2\land C2\subseteq I[n]}{\ForAll{x}{\pi [C2,x]\geq \pi [C1,x]]}}
\end{Envanybound}}
\end{frame}

\begin{frame}
\frametitle{The Same in Theorema (WR)}
\begin{description}[WR]
  \item[\mcolor{red}{WR}]\qquad if $y_i \ge x_i \forall i \in C \subseteq I$ then $\pi \left( C, \bm{y} \right) \ge \pi \left( C, \bm{x} \right)$\bigskip

\end{description}
\mcolor{blue}{\begin{Envanybound}{Definition}{WR}{\pi,n}{\text{allocation}_n[x], \text{allocation}_n[y]}
  \text{WR}[\pi,n]\IffDef n\in\NN\land(\ForAllCond{C}{C\subseteq I[n]}{\ForAll{x,y}{\big((\ForAll{i\in C}{y_i\geq x_i})\implies \pi [C,y]\geq \pi [C,x]\big)}})
\end{Envanybound}}
\end{frame}

\begin{frame}
\frametitle{The Same in Theorema (SR)}
\begin{description}[SR]
  \item[\mcolor{red}{SR}]\qquad if $\emptyset \ne C \subseteq I$ and $y_i > x_i \forall i \in C$ then $\pi \left( C, \bm{y} \right) > \pi \left( C, \bm{x} \right)$.
\end{description}

\mcolor{blue}{\begin{Envanybound}{Definition}{SR}{\pi,n}{\text{allocation}_n[x], \text{allocation}_n[y]}
  \text{SR}[\pi,n]\IffDef n\in\NN\land(\ForAllCond{C}{C\subseteq I[n]\land C\neq\emptyset}{\ForAll{x,y}{\big((\ForAll{i\in C}{y_i > x_i})\!\implies\! \pi [C,y] > \pi [C,x]\big)}})
\end{Envanybound}}


 \end{frame}

\begin{frame}
\frametitle{Properties}
Other important properties that power functions may have:
\begin{description}[AN]
  \item[\mcolor{red}{AN}]\quad if $\sigma: I \rightarrow I$ is a 1:1 onto function permuting the agent set,\\\quad $i \in C \Leftrightarrow \sigma \left( i \right) \in C'$, and $x_i = x'_{\sigma \left( i \right)}$ then $\pi \left( C, \bm{x} \right) = \pi \left( C', \bm{x}' \right)$.
  \item[\mcolor{red}{CX}]\quad $\pi \left( C, \bm{x} \right)$ is continuous in $\bm{x}$.
  \item[\mcolor{red}{RE}]\quad if $i \notin C$ and $\pi \left( \left\{ i \right\}, \bm{x} \right) > 0$ then $\pi \left( C \cup \left\{ i \right\}, \bm{x} \right) > \pi \left( C, \bm{x} \right)$.
\end{description}
\end{frame}

\begin{frame}
\frametitle{Domination}
\begin{description}
\item[\mcolor{red}{\Defstrictfi}]
An allocation $\bm{y}$ \mcolor{blue}{dominates} an allocation $\bm{x}$, written
$\bm{y} \strictfi \bm{x}$, iff
\(
   \pi \left( W, \bm{x} \right) > \pi \left( L, \bm{x} \right);
\)
where $W \equiv \left\{ i \left| y_i > x_i \right. \right\}$ and $L \equiv \left\{ i \left| x_i > y_i \right. \right\}$.
$W$ = win set \& $L$ lose set.\bigskip

\item[\mcolor{red}{\DefD}]
For $\mathcal{Y} \subset \mathcal{X}$, let
$D \left( \mathcal{Y} \right) \equiv \left\{ \bm{x} \in \mathcal{X} \left| \exists \bm{y} \in \mathcal{Y} \textrm{ s.t. } \bm{y} \strictfi \bm{x} \right. \right\}$
be the \mcolor{blue}{dominion} of $\mathcal{Y}$.  $U \left( \mathcal{Y} \right) = \mathcal{X} \backslash D \left( \mathcal{Y} \right)$, the set of allocations undominated by any allocation in $\mathcal{Y}$.
\end{description}
\end{frame}

\begin{frame}
\frametitle{Core and stable set}
\begin{description}
\item[\mcolor{red}{\DefK}] The \mcolor{blue}{core}, $\mathcal{K}$, is the set of undominated allocations,
$U \left( \mathcal{X} \right)$.  \bigskip

\item[\mcolor{red}{\DefS}]
A set of allocations, $\mathcal{S} \subseteq \mathcal{X}$, is a \mcolor{blue}{stable set} iff it satisfies\bigskip

\def\arraystretch{1.2}
\begin{tabular}{p{0.3\textwidth}p{0.3\textwidth}p{0.2\textwidth}}
 \mcolor{red}{internal stability}{}, &
 \mcolor{blue}{$\mathcal{S} \cap D \left( \mathcal{S} \right) = \emptyset$} & (IS)\\
 \mcolor{red}{external stability}{}, &
 \mcolor{blue}{$\mathcal{S} \cup D \left( \mathcal{S} \right) = \mathcal{X}$} & (ES)
\end{tabular}
\end{description}

The conditions combine to yield $\mathcal{S} = \mathcal{X} \backslash
D \left( \mathcal{S} \right)$.  The core necessarily belongs to any
existing stable set.
\end{frame}

\section{Example}
\begin{frame}
  \frametitle{Wealth Is Power}
\begin{minipage}{0.5\textwidth}
\[\mcolor{blue}{\text{WIP}\pi[C,x]:=\sum _{i\in C} x_i}\]
\end{minipage}\pause
\begin{minipage}{0.45\textwidth}
  \begin{center}
  \input{wipn3.tkz}
  \end{center}
\end{minipage}\pause

\vspace*{-1cm}
\begin{minipage}[b]{0.45\textwidth}
Stable Set: $S=$ \\$\left\{\begin{array}{ll}(0,0,1),(0,1,0),(1,0,0),\\(0,{1\over 2},{1\over 2}),({1\over 2},0,{1\over 2}),({1\over 2},{1\over 2},0),\\
({1\over 4},{1\over 4},{1\over 2}),({1\over 4},{1\over 2},{1\over 4}),({1\over 2},{1\over 4},{1\over 4}), \end{array}\right\}$
\end{minipage}

\end{frame}

\section{Pseudo Algorithm}
\long\def\IF{{\bf if}\ }
\def\ELSE{{\bf else}\ }
\def\ENDIF{{\bf end if}}
\def\STATE{{\bf then}\ }
\def\a#1{\rule{#1}{0ex}}
\begin{frame}\def\arraystretch{0.7}
\frametitle{The stable set in $n=3$ with AN, CX, and RE}
\begin{tabular}{rl}
1 &    \IF{$\pi \left( \left\{ i \right\}, \bm{t}^i \right) < \pi \left( \left\{ j, k \right\}, \bm{t}^i \right)$}\\
2 &     \a{2ex} \STATE $\mathcal{S} = \mathcal{D}_1 \backslash \mathcal{D}_0$\\
3 &     \a{2ex} \ELSE\\
4 &     \a{4ex} \IF{$R^i = \emptyset$}\\
5 &     \a{6ex}   \STATE \textbf{return} ``no stable set exists''\\
6 &     \a{6ex} \ELSE\\
7 &     \a{8ex}   \IF{$\pi \left( \left\{ j \right\}, \bm{s}^{jk} \right) \ge \pi \left( \left\{ i, k \right\}, \bm{s}^{jk} \right)$}\\
8 &     \a{10ex}     \STATE $\mathcal{S} = \mathcal{D}_1 \cup \left\{ \mathcal{S}^i \right\}_{i=1}^3$\\
9 &     \a{10ex}   \ELSE\\
10 &    \a{12ex}      $\mathcal{S} = \mathcal{D}_0 \cup \left\{ \mathcal{S}^i \right\}_{i=1}^3 \cup \mathcal{P}$\\
11 &    \a{8ex}    \ENDIF\\
12 &    \a{4ex}  \ENDIF\\
13 &    \ENDIF\\
14 &    \textbf{return} $\mathcal{S}$\\


%1: &   \IF\quad  {$\pi \left( \left\{ i \right\}, \bm{t}^i \right) \ge \pi \left( \left\{ j, k \right\}, \bm{t}^i \right)$}\quad  \STATE\\
%2: &    \a{2ex} $\mathcal{S}_0 = \mathcal{D}_0$ \\
%3: &    \a{2ex} \IF\quad {$M^i = \emptyset$} \quad   \STATE\\
%4: &    \a{4ex} \textbf{return} ``no stable set exists''\\
%5: &    \a{2ex}      \ELSE\\
%6: &    \a{4ex} $\mathcal{S}_1 = U^2 \left( \mathcal{S}_0 \right) = \mathcal{S}_0 \cup \bigcup_{i=1}^3 \mathcal{S}^i$\\
%7: &    \a{4ex} \IF\quad{$\mathcal{S}_1 \cup D \left( \mathcal{S}_1 \right) \ne \mathcal{X}$}
%          \STATE\\
%8: &    \a{6ex} $\mathcal{S} = \mathcal{S}_2 = U^2 \left( \mathcal{S}_1 \right) = \mathcal{S}_1 \cup \mathcal{P}$\\
%9: &    \a{4ex}  \ELSE\\
%10: &   \a{6ex}   $\mathcal{S} = \mathcal{S}_1$\\
%11: &   \a{4ex}  \ENDIF\\
%12: &   \a{2ex}  \ENDIF\\
%13: &   \ELSE\\
%14: &   \a{2ex}$\mathcal{S} = \mathcal{D}_1 \backslash \mathcal{D}_0$\\
%15: &   \ENDIF\\
%16: &   \textbf{return} $\mathcal{S}$
\end{tabular}
\end{frame}

% \def\squeezetext#1{\text{\squeeze{#1}}}
% \section{Pseudo Algorithm}
% \begin{frame}
% \frametitle{Pseudo Algorithm}
% \mcolor{blue}{
%   \begin{Envanylongvar}{Algorithm}{StableSet2}{\pi}
%   \squeezetext{stableSet}[\pi ]:=\\
%   \begin{casedistinction}
%     \begin{casedistinction}
%       \squeezetext{``no stable''} & \squeezetext{empty}[M[1,\pi]] \\
%       \parbox{67mm}{$\squeezetext{where}[\mathcal{S}=\squeezetext{dyadicSet}[0,3]\cup \underset{i=1,\ldots ,3}{\cup }S[i,\pi ],$\\
%         $\begin{casedistinction}
%           \mathcal{S}\cup P[\pi] & \neg\squeezetext{fullSet}[\mathcal{S}\cup D[\mathcal{S},\pi,3]] \\
%           \mathcal{S} & \squeezetext{fullSet}[\mathcal{S}\cup D[\mathcal{S},\pi,3]] \\
%           \squeezetext{``unknown X''} & \squeezetext{otherwise}
%         \end{casedistinction}]$} & \neg\squeezetext{empty}[M[1,\pi]] \\
%       \squeezetext{``unknown M''} & \squeezetext{otherwise}
%     \end{casedistinction} & (*) \\
%     \squeezetext{dyadicSet}[1,3]\backslash \squeezetext{dyadicSet}[0,3] & \squeezetext{otherwise}
%   \end{casedistinction}
% \end{Envanylongvar}}\vspace{5pt}
% \noindent with \mcolor{blue}{$(*)$} to be replaced
% by \mcolor{blue}{$\pi [\{1\},t[1,3]]\geq \pi[\{2,3\},t[1,3]]$}.

% \end{frame}


\section{Examples}

\begin{frame}
  \frametitle{Wealth Is Power}
\begin{minipage}{0.5\textwidth}
\[\mcolor{blue}{\text{WIP}\pi[C,x]:=\sum _{i\in C} x_i}\]
\end{minipage}
\begin{minipage}{0.45\textwidth}
  \begin{center}
  \input{wipn3.tkz}
  \end{center}
\end{minipage}

\vspace*{-1cm}
\begin{minipage}[b]{0.45\textwidth}
Stable Set: $S=$ \\$\left\{\begin{array}{ll}(0,0,1),(0,1,0),(1,0,0),\\(0,{1\over 2},{1\over 2}),({1\over 2},0,{1\over 2}),({1\over 2},{1\over 2},0),\\
({1\over 4},{1\over 4},{1\over 2}),({1\over 4},{1\over 2},{1\over 4}),({1\over 2},{1\over 4},{1\over 4}), \end{array}\right\}$
\end{minipage}

\end{frame}



\begin{frame}
  \frametitle{Strength In Numbers with $\nu > 1$}
\begin{minipage}{0.5\textwidth}
\[\mcolor{blue}{\text{SIN}\pi_\nu[C,x]:=\sum _{i\in C} \left(x_i+\nu \right)}\]
\hspace*{5ex} with \(\nu>1\)
\end{minipage}\pause
\begin{minipage}{0.45\textwidth}
  \begin{center}
   \input{sinn31v.tkz}
  \end{center}
\end{minipage}\pause

\vspace*{-1cm}
\begin{minipage}[b]{0.45\textwidth}
Stable Set: $S=$\\
$\left\{(0,{1\over 2},{1\over 2}), ({1\over 2},0,{1\over 2}), 
         ({1\over 2},{1\over 2},0)\right\}$
\end{minipage}

\end{frame}

\begin{frame}
\frametitle{Strength In Numbers with $0<\nu<1$}

\begin{minipage}{0.45\textwidth}
\[\mcolor{blue}{\text{SIN}\pi_\nu[C,x]:=\sum _{i\in C} \left(x_i+\nu \right)} \]
\hspace*{5ex} with \(0<\nu<1\)\bigskip

no stable set exists

\end{minipage}
\begin{minipage}{0.45\textwidth}

  \centering
  \input{sinn30v1.tkz}
\end{minipage}
\end{frame}


% \def\bb#1{\hbox to 4.8ex{\hspace*{\fill}#1\hspace*{\fill}}}
% \def\t{\bb{$\times$}}
% \begin{frame}
% \small
% \frametitle{Some Explicit Dependencies of Statements}\tabcolsep0pt
% \begin{tabular}{c||c|c|c|c|c|c|c|c|c|c}
%      & \bb{WR} & \bb{SR} & \bb{WC} & \bb{AN} & \bb{CX} & \bb{RE} & \bb{Def1} & \bb{Thm}  & \bb{Lem} & \bb{Ext}\\\hline\hline
% Lem1 & \t &    &    &    &    &    &      &      &      &\\\hline
% Lem2 &    &    &    & \t &    &    &      &      &      &\\\hline
% Thm1 &    &    &    &    &    &    &      &      &      &\t\\\hline
% Lem3 &    & \t & \t & \t &    &    &      &      &      &\\\hline
% Lem4 &    &    &    &    &    &    &  \t  & 1    &      &\\\hline
% Lem5 &    &    & \t &    &    &    &  \t  &      &      &\\\hline
% Thm2 &    &    &    & \t &    &    &      & 1    & 4,5  &\t\\\hline
% Thm3 &    &    &    & \t &    &    &      &      & 2    &\\\hline
% Cor1 &    &    &    &    &    &    &      & 3    &      &\t\\\hline
% Lem6 &    & \t & \t & \t &    &    &      &      &      &\\\hline
% Lem7 & \t & \t &    &    &    &    &      &      &      &\\\hline
% Cor2 &    &    &    &    &    &    &      &      & 7    &\\\hline
% \end{tabular}
% \end{frame}

% Lem8 &    & \t & \t & \t &    &    &      &      & 7    &
% Lem9 &    & \t & \t & \t &    & \t &      &      &      &\\\hline
% Lem10&    &    & \t & \t &    & \t &      &      & 7    &\\\hline
% Thm4 &    &    &    & \t &    & \t &      &      & 8,10 &\t\\\hline
% Lem11&    & \t &    & \t & \t &    &      & 4    &      &\\\hline
% Thm5 &    &    &    & \t & \t & \t &      & 4    & 11   &\\\hline
% Thm6 &    &    &    &    &    &    &  \t  &      &      &\t\\\hline
% Thm7 &    &    &    & \t & \t & \t &      & 4,5,6&      &\t\\\hline
% Cor3 &    &    &    & \t & \t & \t &      &      & 7    &\\\hline
% Thm8 &    &    &    & \t & \t &    &      &      &      &\t\\\hline
% Thm9 &    &    &    & \t & \t &    &  \t  &      &      &\t\\\hline
% Lem12&    &    &    & \t &    &    &      &      &      &\t\\\hline
% Thm10&    &    &    & \t &    &    &      &      &      &\t\\\hline
% Cor4 &    &    &    & \t &    &    &      & 10   &      &\t\\\hline
% Lem13& \t & \t &    &    & \t &    &      &      & 1    &\\\hline
% Thm11&    &    &    &    & \t &    &      &      & 13   &\t\\\hline
% Thm12&    &    &    & \t & \t & \t &  \t  & 4    & 5,10,11&\t\\\hline
% Lem14&    &    &    &    &    &    &      &      & 4    &\\\hline
% Thm13& \t & \t & \t & \t &    &    &      &      &      &\\\hline
% Lem15&    &    &    &    &    &    &      &      &      &\\\hline
% Lem16&    &    &    & \t & \t &    &      &      & 3,6  &\\\hline
% Lem17&    &    &    & \t & \t &    &      &      &      &\t\\\hline
% Lem18&    &    &    &    &    &    &  \t  &      &      &\\\hline
% \end{tabular}
% \end{frame}

\section{Proof of a Lemma}
\begin{frame}
\frametitle{Proof of a Lemma}
(One Lemma of 14 lemmas, 12 theorems, and 4 corollaries)
 
% \begin{lemma}
%   Any power function, $\pi \left( C, \bm{x} \right)$, can be represented by another, $\pi' \left( C, \left\{ x_i \right\}_{i \in C} \right)$, which depends only on the resource holdings of its coalition members.
% \end{lemma}

% % \mcolor{blue}{\begin{Envanywithlong}{Lemma}{powerfunction-independent}{\pi,n,C,x,y}{\text{allocation}_n[x] \land  \text{allocation}_n[y]\land C\subseteq  I[n]\land  \text{powerfunction}[\pi ,n]}
% %   \ForAll{i\in C}{(x_i=y_i)\implies (\pi[C,x]=\pi[C,y])}
% % \end{Envanywithlong}}
% \end{frame}


% \begin{frame}
% \frametitle{Another Example}
\begin{lemma}
  When $n = 3$:\quad
1. $\mathcal{K} = \emptyset$ implies $\bm{t}^i \in D \left( \bm{s}^{jk} \right)$\quad
  for distinct $i, j, k \in I$.
\end{lemma}
\begin{proof}
  \begin{enumerate}
  \item As $\mathcal{K} = \emptyset$, no agent can defend its holdings
    against both others, so that $\pi \left( \left\{ i \right\},
      \bm{t}^i \right) < \pi \left( \left\{ j, k \right\}, \bm{t}^i
    \right)$ for distinct $i, j$ and $k$.  As $\left\{ j, k \right\}$
    prefers $\bm{s}^{jk}$ to $\bm{t}^i$, this ensures that
    $\bm{s}^{jk} \strictfi \bm{t}^i$.
  \end{enumerate}
\end{proof}
\end{frame}

% \begin{frame}
% \frametitle{An Example (Cont'd)}
% Make this proof more formal

% \begin{itemize}
% \item AN: use $1$, $2$, and $3$ instead of $i$,$j$, and $k$.  

% \item In $\bm{t}^1$, $2$ and $3$ together are more powerful than $1$
%   on its own: $\mathcal{K} = \emptyset$
%   means that $\bm{t}^1\notin\mathcal{K}$, that is, there exists an
%   $\bm{x}$ such that $\bm{x} \strictfi\bm{t}^1$. 
%   For $\bm{x}=(x_1,x_2,x_3)$  distinguish 3 cases:
% \begin{description}
% \item[Case 1:] $x_1,x_2\neq 0$. Since $\bm{t}^1\notin\mathcal{K}$ we
%   have $\pi(\{2,3\},\bm{t}^1)>\pi(\{1\},\bm{t}^1)$, hence we get
%   $\bm{s}^{23}\strictfi\bm{t}^1$.

% \item[Case 2:] Without loss of generality $x_2>x_3=0$.
% With axiom WC we have $\pi(\{2,3\},\bm{t}^1)>\pi(\{2\},\bm{t}^1)$.

% \item[Case 3:] $x_2=x_3=0$. This would mean $\bm{x}=\bm{t}^1$, which
%   cannot be.
% \end{description}
% \end{itemize}

% \end{frame}

\section{Plans}
\begin{frame}
\frametitle{Representation and Proof}
\begin{itemize}
   \item \mcolor{blue}{\sTeX} a semantic version of \TeX. (Do not use macros of the
type \texttt{$\backslash$mathbb\{N\}} but of type \texttt{$\backslash$naturalNumbers}).
   \item Prove all theorems in \mcolor{blue}{\theorema} and \mcolor{blue}{\isabelle}.
   \item Extract \mcolor{blue}{challenge problems} for higher order theorem proving
(to be represented in TPTP).\pause
   \item Extend study to \mcolor{blue}{games not satsifying additional restrictions}.\pause
   \item Test \mcolor{blue}{higher order theorem provers} (\leo).
   \item \mcolor{blue}{Extract proof tactics}.
   \item \mcolor{blue}{Reuse proofs tactics}.
   \item Extract \mcolor{blue}{computational content}.
   \item Guide proofs by computational \mcolor{blue}{models}.\pause
   \item \mcolor{blue}{Exploit results}.
\end{itemize}
\end{frame}

\section{Summary}
\begin{frame}
\frametitle{Summary (Part I)}

The pseudo algorithm:
\begin{itemize}
\item \mcolor{blue}{Non-computational in several aspects}
\item \mcolor{blue}{Evaluation by a mixture of reasoning and
    computing}.  Can compute the stable set of WIP, SIN, assumed the
  corresponding lemmas are available.
\item \mcolor{blue}{Plan:} Extend the computational part, e.g.,
  represent infinite set in a finite way. Use underlying Mathematica
  to compute solutions of equations.
\end{itemize}
\end{frame}


\begin{frame}
\frametitle{Summary (Part II)}

\begin{itemize}
\item Axiomatic approach in theoretical economics valuable (eliminate errors, even without full proof)
\item Good field with non-trivial but not very deep mathematics.
\item \mcolor{blue}{Formalisation} in Theorema is easy and fast even for beginners.
\item \mcolor{blue}{Automation} at least partially possible. Reasoning requires more expert knowledge and work. 
\item Theorema offers \mcolor{blue}{mixture of reasoning and
    computation}.  Very useful for determining stable sets.
\end{itemize}
\end{frame}
\end{document}

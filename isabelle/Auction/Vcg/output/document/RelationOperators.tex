%
\begin{isabellebody}%
\def\isabellecontext{RelationOperators}%
%
\isamarkupheader{Additional operators on relations, going beyond Relations.thy,
  and properties of these operators%
}
\isamarkuptrue%
%
\isadelimtheory
%
\endisadelimtheory
%
\isatagtheory
\isacommand{theory}\isamarkupfalse%
\ RelationOperators\isanewline
\isakeyword{imports}\isanewline
\ \ Main\isanewline
\ \ SetUtils\isanewline
\ \ {\isachardoublequoteopen}{\isachartilde}{\isachartilde}{\isacharslash}src{\isacharslash}HOL{\isacharslash}Library{\isacharslash}Code{\isacharunderscore}Target{\isacharunderscore}Nat{\isachardoublequoteclose}\isanewline
\isanewline
\isakeyword{begin}%
\endisatagtheory
{\isafoldtheory}%
%
\isadelimtheory
%
\endisadelimtheory
%
\isamarkupsection{evaluating a relation as a function%
}
\isamarkuptrue%
%
\begin{isamarkuptext}%
If an input has a unique image element under a given relation, return that element; 
  otherwise return a fallback value.%
\end{isamarkuptext}%
\isamarkuptrue%
\isacommand{fun}\isamarkupfalse%
\ eval{\isacharunderscore}rel{\isacharunderscore}or\ {\isacharcolon}{\isacharcolon}\ {\isachardoublequoteopen}{\isacharparenleft}{\isacharprime}a\ {\isasymtimes}\ {\isacharprime}b{\isacharparenright}\ set\ {\isasymRightarrow}\ {\isacharprime}a\ {\isasymRightarrow}\ {\isacharprime}b\ {\isasymRightarrow}\ {\isacharprime}b{\isachardoublequoteclose}\isanewline
\ \ \isakeyword{where}\ {\isachardoublequoteopen}eval{\isacharunderscore}rel{\isacharunderscore}or\ R\ a\ z\ {\isacharequal}\ {\isacharparenleft}let\ im\ {\isacharequal}\ R\ {\isacharbackquote}{\isacharbackquote}\ {\isacharbraceleft}a{\isacharbraceright}\ in\ if\ card\ im\ {\isacharequal}\ {\isadigit{1}}\ then\ the{\isacharunderscore}elem\ im\ else\ z{\isacharparenright}{\isachardoublequoteclose}%
\begin{isamarkuptext}%
right-uniqueness of a relation: the image of a \isa{trivial} set (i.e.\ an empty or
  singleton set) under the relation is trivial again. 
This is the set-theoretical way of characterizing functions, as opposed to \isa{{\isasymlambda}} functions.%
\end{isamarkuptext}%
\isamarkuptrue%
\isacommand{definition}\isamarkupfalse%
\ runiq\ {\isacharcolon}{\isacharcolon}\ {\isachardoublequoteopen}{\isacharparenleft}{\isacharprime}a\ {\isasymtimes}\ {\isacharprime}b{\isacharparenright}\ set\ {\isasymRightarrow}\ bool{\isachardoublequoteclose}\ \isanewline
\ \ \isakeyword{where}\ {\isachardoublequoteopen}runiq\ R\ {\isacharequal}\ {\isacharparenleft}{\isasymforall}\ X\ {\isachardot}\ trivial\ X\ {\isasymlongrightarrow}\ trivial\ {\isacharparenleft}R\ {\isacharbackquote}{\isacharbackquote}\ X{\isacharparenright}{\isacharparenright}{\isachardoublequoteclose}%
\isamarkupsection{restriction%
}
\isamarkuptrue%
%
\begin{isamarkuptext}%
restriction of a relation to a set (usually resulting in a relation with a smaller domain)%
\end{isamarkuptext}%
\isamarkuptrue%
\isacommand{definition}\isamarkupfalse%
\ restrict\ {\isacharcolon}{\isacharcolon}\ {\isachardoublequoteopen}{\isacharparenleft}{\isacharprime}a\ {\isasymtimes}\ {\isacharprime}b{\isacharparenright}\ set\ {\isasymRightarrow}\ {\isacharprime}a\ set\ {\isasymRightarrow}\ {\isacharparenleft}{\isacharprime}a\ {\isasymtimes}\ {\isacharprime}b{\isacharparenright}\ set{\isachardoublequoteclose}\ {\isacharparenleft}\isakeyword{infix}\ {\isachardoublequoteopen}{\isacharbar}{\isacharbar}{\isachardoublequoteclose}\ {\isadigit{7}}{\isadigit{5}}{\isacharparenright}\isanewline
\ \ \isakeyword{where}\ {\isachardoublequoteopen}R\ {\isacharbar}{\isacharbar}\ X\ {\isacharequal}\ {\isacharparenleft}X\ {\isasymtimes}\ Range\ R{\isacharparenright}\ {\isasyminter}\ R{\isachardoublequoteclose}%
\begin{isamarkuptext}%
extensional characterization of the pairs within a restricted relation%
\end{isamarkuptext}%
\isamarkuptrue%
\isacommand{lemma}\isamarkupfalse%
\ restrict{\isacharunderscore}ext{\isacharcolon}\ {\isachardoublequoteopen}R\ {\isacharbar}{\isacharbar}\ X\ {\isacharequal}\ {\isacharbraceleft}{\isacharparenleft}x{\isacharcomma}\ y{\isacharparenright}\ {\isacharbar}\ x\ y\ {\isachardot}\ x\ {\isasymin}\ X\ {\isasymand}\ {\isacharparenleft}x{\isacharcomma}\ y{\isacharparenright}\ {\isasymin}\ R{\isacharbraceright}{\isachardoublequoteclose}\isanewline
%
\isadelimproof
\ \ \ \ \ \ %
\endisadelimproof
%
\isatagproof
\isacommand{unfolding}\isamarkupfalse%
\ restrict{\isacharunderscore}def\ \isacommand{using}\isamarkupfalse%
\ Range{\isacharunderscore}iff\ \isacommand{by}\isamarkupfalse%
\ blast%
\endisatagproof
{\isafoldproof}%
%
\isadelimproof
%
\endisadelimproof
%
\begin{isamarkuptext}%
alternative statement of \isa{{\isacharquery}R\ {\isacharbar}{\isacharbar}\ {\isacharquery}X\ {\isacharequal}\ {\isacharbraceleft}{\isacharparenleft}x{\isacharcomma}\ y{\isacharparenright}\ {\isacharbar}x\ y{\isachardot}\ x\ {\isasymin}\ {\isacharquery}X\ {\isasymand}\ {\isacharparenleft}x{\isacharcomma}\ y{\isacharparenright}\ {\isasymin}\ {\isacharquery}R{\isacharbraceright}} without explicitly naming the pair's components%
\end{isamarkuptext}%
\isamarkuptrue%
\isacommand{lemma}\isamarkupfalse%
\ restrict{\isacharunderscore}ext{\isacharprime}{\isacharcolon}\ {\isachardoublequoteopen}R\ {\isacharbar}{\isacharbar}\ X\ {\isacharequal}\ {\isacharbraceleft}p\ {\isachardot}\ fst\ p\ {\isasymin}\ X\ {\isasymand}\ p\ {\isasymin}\ R{\isacharbraceright}{\isachardoublequoteclose}\isanewline
%
\isadelimproof
%
\endisadelimproof
%
\isatagproof
\isacommand{proof}\isamarkupfalse%
\ {\isacharminus}\isanewline
\ \ \isacommand{have}\isamarkupfalse%
\ {\isachardoublequoteopen}R\ {\isacharbar}{\isacharbar}\ X\ {\isacharequal}\ {\isacharbraceleft}{\isacharparenleft}x{\isacharcomma}\ y{\isacharparenright}\ {\isacharbar}\ x\ y\ {\isachardot}\ x\ {\isasymin}\ X\ {\isasymand}\ {\isacharparenleft}x{\isacharcomma}\ y{\isacharparenright}\ {\isasymin}\ R{\isacharbraceright}{\isachardoublequoteclose}\ \isacommand{by}\isamarkupfalse%
\ {\isacharparenleft}rule\ restrict{\isacharunderscore}ext{\isacharparenright}\isanewline
\ \ \isacommand{also}\isamarkupfalse%
\ \isacommand{have}\isamarkupfalse%
\ {\isachardoublequoteopen}{\isasymdots}\ {\isacharequal}\ {\isacharbraceleft}\ p\ {\isachardot}\ fst\ p\ {\isasymin}\ X\ {\isasymand}\ p\ {\isasymin}\ R\ {\isacharbraceright}{\isachardoublequoteclose}\ \isacommand{by}\isamarkupfalse%
\ force\isanewline
\ \ \isacommand{finally}\isamarkupfalse%
\ \isacommand{show}\isamarkupfalse%
\ {\isacharquery}thesis\ \isacommand{{\isachardot}}\isamarkupfalse%
\isanewline
\isacommand{qed}\isamarkupfalse%
%
\endisatagproof
{\isafoldproof}%
%
\isadelimproof
%
\endisadelimproof
%
\begin{isamarkuptext}%
Restricting a relation to the empty set yields the empty set.%
\end{isamarkuptext}%
\isamarkuptrue%
\isacommand{lemma}\isamarkupfalse%
\ restrict{\isacharunderscore}empty{\isacharcolon}\ {\isachardoublequoteopen}P\ {\isacharbar}{\isacharbar}\ {\isacharbraceleft}{\isacharbraceright}\ {\isacharequal}\ {\isacharbraceleft}{\isacharbraceright}{\isachardoublequoteclose}\ \isanewline
%
\isadelimproof
\ \ \ \ \ \ %
\endisadelimproof
%
\isatagproof
\isacommand{unfolding}\isamarkupfalse%
\ restrict{\isacharunderscore}def\ \isacommand{by}\isamarkupfalse%
\ simp%
\endisatagproof
{\isafoldproof}%
%
\isadelimproof
%
\endisadelimproof
%
\begin{isamarkuptext}%
A restriction is a subrelation of the original relation.%
\end{isamarkuptext}%
\isamarkuptrue%
\isacommand{lemma}\isamarkupfalse%
\ restriction{\isacharunderscore}is{\isacharunderscore}subrel{\isacharcolon}\ {\isachardoublequoteopen}P\ {\isacharbar}{\isacharbar}\ X\ {\isasymsubseteq}\ P{\isachardoublequoteclose}\ \isanewline
%
\isadelimproof
\ \ \ \ \ \ %
\endisadelimproof
%
\isatagproof
\isacommand{using}\isamarkupfalse%
\ restrict{\isacharunderscore}def\ \isacommand{by}\isamarkupfalse%
\ blast%
\endisatagproof
{\isafoldproof}%
%
\isadelimproof
%
\endisadelimproof
%
\begin{isamarkuptext}%
Restricting a relation only has an effect within its domain.%
\end{isamarkuptext}%
\isamarkuptrue%
\isacommand{lemma}\isamarkupfalse%
\ restriction{\isacharunderscore}within{\isacharunderscore}domain{\isacharcolon}\ {\isachardoublequoteopen}P\ {\isacharbar}{\isacharbar}\ X\ {\isacharequal}\ P\ {\isacharbar}{\isacharbar}\ {\isacharparenleft}X\ {\isasyminter}\ {\isacharparenleft}Domain\ P{\isacharparenright}{\isacharparenright}{\isachardoublequoteclose}\ \isanewline
%
\isadelimproof
\ \ \ \ \ \ %
\endisadelimproof
%
\isatagproof
\isacommand{unfolding}\isamarkupfalse%
\ restrict{\isacharunderscore}def\ \isacommand{by}\isamarkupfalse%
\ fast%
\endisatagproof
{\isafoldproof}%
%
\isadelimproof
%
\endisadelimproof
%
\begin{isamarkuptext}%
alternative characterization of the restriction of a relation to a singleton set%
\end{isamarkuptext}%
\isamarkuptrue%
\isacommand{lemma}\isamarkupfalse%
\ restrict{\isacharunderscore}to{\isacharunderscore}singleton{\isacharcolon}\ {\isachardoublequoteopen}P\ {\isacharbar}{\isacharbar}\ {\isacharbraceleft}x{\isacharbraceright}\ {\isacharequal}\ {\isacharbraceleft}x{\isacharbraceright}\ {\isasymtimes}\ {\isacharparenleft}P\ {\isacharbackquote}{\isacharbackquote}\ {\isacharbraceleft}x{\isacharbraceright}{\isacharparenright}{\isachardoublequoteclose}\ \isanewline
%
\isadelimproof
\ \ \ \ \ \ %
\endisadelimproof
%
\isatagproof
\isacommand{unfolding}\isamarkupfalse%
\ restrict{\isacharunderscore}def\ \isacommand{by}\isamarkupfalse%
\ fast%
\endisatagproof
{\isafoldproof}%
%
\isadelimproof
%
\endisadelimproof
%
\isamarkupsection{relation outside some set%
}
\isamarkuptrue%
%
\begin{isamarkuptext}%
For a set-theoretical relation \isa{R} and an ``exclusion'' set \isa{X}, return those
  tuples of \isa{R} whose first component is not in \isa{X}.  In other words, exclude \isa{X}
  from the domain of \isa{R}.%
\end{isamarkuptext}%
\isamarkuptrue%
\isacommand{definition}\isamarkupfalse%
\ Outside\ {\isacharcolon}{\isacharcolon}\ {\isachardoublequoteopen}{\isacharparenleft}{\isacharprime}a\ {\isasymtimes}\ {\isacharprime}b{\isacharparenright}\ set\ {\isasymRightarrow}\ {\isacharprime}a\ set\ {\isasymRightarrow}\ {\isacharparenleft}{\isacharprime}a\ {\isasymtimes}\ {\isacharprime}b{\isacharparenright}\ set{\isachardoublequoteclose}\ {\isacharparenleft}\isakeyword{infix}\ {\isachardoublequoteopen}outside{\isachardoublequoteclose}\ {\isadigit{7}}{\isadigit{5}}{\isacharparenright}\isanewline
\ \ \ \isakeyword{where}\ {\isachardoublequoteopen}R\ outside\ X\ {\isacharequal}\ R\ {\isacharminus}\ {\isacharparenleft}X\ {\isasymtimes}\ Range\ R{\isacharparenright}{\isachardoublequoteclose}%
\begin{isamarkuptext}%
Considering a relation outside some set \isa{X} reduces its domain by \isa{X}.%
\end{isamarkuptext}%
\isamarkuptrue%
\isacommand{lemma}\isamarkupfalse%
\ outside{\isacharunderscore}reduces{\isacharunderscore}domain{\isacharcolon}\ {\isachardoublequoteopen}Domain\ {\isacharparenleft}P\ outside\ X{\isacharparenright}\ {\isacharequal}\ {\isacharparenleft}Domain\ P{\isacharparenright}\ {\isacharminus}\ X{\isachardoublequoteclose}\isanewline
%
\isadelimproof
\ \ \ \ \ \ %
\endisadelimproof
%
\isatagproof
\isacommand{unfolding}\isamarkupfalse%
\ Outside{\isacharunderscore}def\ \isacommand{by}\isamarkupfalse%
\ fast%
\endisatagproof
{\isafoldproof}%
%
\isadelimproof
%
\endisadelimproof
%
\begin{isamarkuptext}%
Considering a relation outside a singleton set \isa{{\isacharbraceleft}x{\isacharbraceright}} reduces its domain by 
  \isa{x}.%
\end{isamarkuptext}%
\isamarkuptrue%
\isacommand{corollary}\isamarkupfalse%
\ Domain{\isacharunderscore}outside{\isacharunderscore}singleton{\isacharcolon}\isanewline
\ \ \isakeyword{assumes}\ {\isachardoublequoteopen}Domain\ R\ {\isacharequal}\ insert\ x\ A{\isachardoublequoteclose}\isanewline
\ \ \ \ \ \ \isakeyword{and}\ {\isachardoublequoteopen}x\ {\isasymnotin}\ A{\isachardoublequoteclose}\isanewline
\ \ \isakeyword{shows}\ {\isachardoublequoteopen}Domain\ {\isacharparenleft}R\ outside\ {\isacharbraceleft}x{\isacharbraceright}{\isacharparenright}\ {\isacharequal}\ A{\isachardoublequoteclose}\isanewline
%
\isadelimproof
\ \ %
\endisadelimproof
%
\isatagproof
\isacommand{using}\isamarkupfalse%
\ assms\ outside{\isacharunderscore}reduces{\isacharunderscore}domain\ \isacommand{by}\isamarkupfalse%
\ {\isacharparenleft}metis\ Diff{\isacharunderscore}insert{\isacharunderscore}absorb{\isacharparenright}%
\endisatagproof
{\isafoldproof}%
%
\isadelimproof
%
\endisadelimproof
%
\begin{isamarkuptext}%
For any set, a relation equals the union of its restriction to that set and its
  pairs outside that set.%
\end{isamarkuptext}%
\isamarkuptrue%
\isacommand{lemma}\isamarkupfalse%
\ outside{\isacharunderscore}union{\isacharunderscore}restrict{\isacharcolon}\ {\isachardoublequoteopen}P\ {\isacharequal}\ {\isacharparenleft}P\ outside\ X{\isacharparenright}\ {\isasymunion}\ {\isacharparenleft}P\ {\isacharbar}{\isacharbar}\ X{\isacharparenright}{\isachardoublequoteclose}\isanewline
%
\isadelimproof
\ \ \ \ \ \ %
\endisadelimproof
%
\isatagproof
\isacommand{unfolding}\isamarkupfalse%
\ Outside{\isacharunderscore}def\ restrict{\isacharunderscore}def\ \isacommand{by}\isamarkupfalse%
\ fast%
\endisatagproof
{\isafoldproof}%
%
\isadelimproof
%
\endisadelimproof
%
\begin{isamarkuptext}%
The range of a relation \isa{R} outside some exclusion set \isa{X} is a 
  subset of the image of the domain of \isa{R}, minus \isa{X}, under \isa{R}.%
\end{isamarkuptext}%
\isamarkuptrue%
\isacommand{lemma}\isamarkupfalse%
\ Range{\isacharunderscore}outside{\isacharunderscore}sub{\isacharunderscore}Image{\isacharunderscore}Domain{\isacharcolon}\ {\isachardoublequoteopen}Range\ {\isacharparenleft}R\ outside\ X{\isacharparenright}\ {\isasymsubseteq}\ R\ {\isacharbackquote}{\isacharbackquote}\ {\isacharparenleft}Domain\ R\ {\isacharminus}\ X{\isacharparenright}{\isachardoublequoteclose}\isanewline
%
\isadelimproof
\ \ \ \ \ \ %
\endisadelimproof
%
\isatagproof
\isacommand{using}\isamarkupfalse%
\ Outside{\isacharunderscore}def\ Image{\isacharunderscore}def\ Domain{\isacharunderscore}def\ Range{\isacharunderscore}def\ \isacommand{by}\isamarkupfalse%
\ blast%
\endisatagproof
{\isafoldproof}%
%
\isadelimproof
%
\endisadelimproof
%
\begin{isamarkuptext}%
Considering a relation outside some set does not enlarge its range.%
\end{isamarkuptext}%
\isamarkuptrue%
\isacommand{lemma}\isamarkupfalse%
\ Range{\isacharunderscore}outside{\isacharunderscore}sub{\isacharcolon}\isanewline
\ \ \isakeyword{assumes}\ {\isachardoublequoteopen}Range\ R\ {\isasymsubseteq}\ Y{\isachardoublequoteclose}\isanewline
\ \ \isakeyword{shows}\ {\isachardoublequoteopen}Range\ {\isacharparenleft}R\ outside\ X{\isacharparenright}\ {\isasymsubseteq}\ Y{\isachardoublequoteclose}\isanewline
%
\isadelimproof
\ \ %
\endisadelimproof
%
\isatagproof
\isacommand{using}\isamarkupfalse%
\ assms\ outside{\isacharunderscore}union{\isacharunderscore}restrict\ \isacommand{by}\isamarkupfalse%
\ {\isacharparenleft}metis\ Range{\isacharunderscore}mono\ inf{\isacharunderscore}sup{\isacharunderscore}ord{\isacharparenleft}{\isadigit{3}}{\isacharparenright}\ subset{\isacharunderscore}trans{\isacharparenright}%
\endisatagproof
{\isafoldproof}%
%
\isadelimproof
%
\endisadelimproof
%
\isamarkupsection{flipping pairs of relations%
}
\isamarkuptrue%
%
\begin{isamarkuptext}%
flipping a pair: exchanging first and second component%
\end{isamarkuptext}%
\isamarkuptrue%
\isacommand{definition}\isamarkupfalse%
\ flip\ \isakeyword{where}\ {\isachardoublequoteopen}flip\ tup\ {\isacharequal}\ {\isacharparenleft}snd\ tup{\isacharcomma}\ fst\ tup{\isacharparenright}{\isachardoublequoteclose}%
\begin{isamarkuptext}%
Flipped pairs can be found in the converse relation.%
\end{isamarkuptext}%
\isamarkuptrue%
\isacommand{lemma}\isamarkupfalse%
\ flip{\isacharunderscore}in{\isacharunderscore}conv{\isacharcolon}\isanewline
\ \ \isakeyword{assumes}\ {\isachardoublequoteopen}tup\ {\isasymin}\ R{\isachardoublequoteclose}\isanewline
\ \ \isakeyword{shows}\ {\isachardoublequoteopen}flip\ tup\ {\isasymin}\ R{\isasyminverse}{\isachardoublequoteclose}\isanewline
%
\isadelimproof
\ \ %
\endisadelimproof
%
\isatagproof
\isacommand{using}\isamarkupfalse%
\ assms\ \isacommand{unfolding}\isamarkupfalse%
\ flip{\isacharunderscore}def\ \isacommand{by}\isamarkupfalse%
\ simp%
\endisatagproof
{\isafoldproof}%
%
\isadelimproof
%
\endisadelimproof
%
\begin{isamarkuptext}%
Flipping a pair twice doesn't change it.%
\end{isamarkuptext}%
\isamarkuptrue%
\isacommand{lemma}\isamarkupfalse%
\ flip{\isacharunderscore}flip{\isacharcolon}\ {\isachardoublequoteopen}flip\ {\isacharparenleft}flip\ tup{\isacharparenright}\ {\isacharequal}\ tup{\isachardoublequoteclose}\isanewline
%
\isadelimproof
\ \ %
\endisadelimproof
%
\isatagproof
\isacommand{by}\isamarkupfalse%
\ {\isacharparenleft}metis\ flip{\isacharunderscore}def\ fst{\isacharunderscore}conv\ snd{\isacharunderscore}conv\ surjective{\isacharunderscore}pairing{\isacharparenright}%
\endisatagproof
{\isafoldproof}%
%
\isadelimproof
%
\endisadelimproof
%
\begin{isamarkuptext}%
Flipping all pairs in a relation yields the converse relation.%
\end{isamarkuptext}%
\isamarkuptrue%
\isacommand{lemma}\isamarkupfalse%
\ flip{\isacharunderscore}conv{\isacharcolon}\ {\isachardoublequoteopen}flip\ {\isacharbackquote}\ R\ {\isacharequal}\ R{\isasyminverse}{\isachardoublequoteclose}\isanewline
%
\isadelimproof
%
\endisadelimproof
%
\isatagproof
\isacommand{proof}\isamarkupfalse%
\ {\isacharminus}\isanewline
\ \ \isacommand{have}\isamarkupfalse%
\ {\isachardoublequoteopen}flip\ {\isacharbackquote}\ R\ {\isacharequal}\ {\isacharbraceleft}\ flip\ tup\ {\isacharbar}\ tup\ {\isachardot}\ tup\ {\isasymin}\ R\ {\isacharbraceright}{\isachardoublequoteclose}\ \isacommand{by}\isamarkupfalse%
\ {\isacharparenleft}metis\ image{\isacharunderscore}Collect{\isacharunderscore}mem{\isacharparenright}\isanewline
\ \ \isacommand{also}\isamarkupfalse%
\ \isacommand{have}\isamarkupfalse%
\ {\isachardoublequoteopen}{\isasymdots}\ {\isacharequal}\ {\isacharbraceleft}\ tup\ {\isachardot}\ tup\ {\isasymin}\ R{\isasyminverse}\ {\isacharbraceright}{\isachardoublequoteclose}\ \isacommand{using}\isamarkupfalse%
\ flip{\isacharunderscore}in{\isacharunderscore}conv\ \isacommand{by}\isamarkupfalse%
\ {\isacharparenleft}metis\ converse{\isacharunderscore}converse\ flip{\isacharunderscore}flip{\isacharparenright}\isanewline
\ \ \isacommand{also}\isamarkupfalse%
\ \isacommand{have}\isamarkupfalse%
\ {\isachardoublequoteopen}{\isasymdots}\ {\isacharequal}\ R{\isasyminverse}{\isachardoublequoteclose}\ \isacommand{by}\isamarkupfalse%
\ simp\isanewline
\ \ \isacommand{finally}\isamarkupfalse%
\ \isacommand{show}\isamarkupfalse%
\ {\isacharquery}thesis\ \isacommand{{\isachardot}}\isamarkupfalse%
\isanewline
\isacommand{qed}\isamarkupfalse%
%
\endisatagproof
{\isafoldproof}%
%
\isadelimproof
%
\endisadelimproof
%
\isamarkupsection{evaluation as a function%
}
\isamarkuptrue%
%
\begin{isamarkuptext}%
Evaluates a relation \isa{R} for a single argument, as if it were a function.
  This will only work if \isa{R} is right-unique, i.e. if the image is always a singleton set.%
\end{isamarkuptext}%
\isamarkuptrue%
\isacommand{fun}\isamarkupfalse%
\ eval{\isacharunderscore}rel\ {\isacharcolon}{\isacharcolon}\ {\isachardoublequoteopen}{\isacharparenleft}{\isacharprime}a\ {\isasymtimes}\ {\isacharprime}b{\isacharparenright}\ set\ {\isasymRightarrow}\ {\isacharprime}a\ {\isasymRightarrow}\ {\isacharprime}b{\isachardoublequoteclose}\ {\isacharparenleft}\isakeyword{infix}\ {\isachardoublequoteopen}{\isacharcomma}{\isacharcomma}{\isachardoublequoteclose}\ {\isadigit{7}}{\isadigit{5}}{\isacharparenright}\ \isanewline
\ \ \ \ \isakeyword{where}\ {\isachardoublequoteopen}R\ {\isacharcomma}{\isacharcomma}\ a\ {\isacharequal}\ the{\isacharunderscore}elem\ {\isacharparenleft}R\ {\isacharbackquote}{\isacharbackquote}\ {\isacharbraceleft}a{\isacharbraceright}{\isacharparenright}{\isachardoublequoteclose}%
\isamarkupsection{paste%
}
\isamarkuptrue%
%
\begin{isamarkuptext}%
the union of two binary relations \isa{P} and \isa{Q}, where pairs from \isa{Q}
  override pairs from \isa{P} when their first components coincide.
This is particularly useful when P, Q are \isa{runiq}, and one wants to preserve that property.%
\end{isamarkuptext}%
\isamarkuptrue%
\isacommand{definition}\isamarkupfalse%
\ paste\ {\isacharparenleft}\isakeyword{infix}\ {\isachardoublequoteopen}{\isacharplus}{\isacharasterisk}{\isachardoublequoteclose}\ {\isadigit{7}}{\isadigit{5}}{\isacharparenright}\isanewline
\ \ \ \isakeyword{where}\ {\isachardoublequoteopen}P\ {\isacharplus}{\isacharasterisk}\ Q\ {\isacharequal}\ {\isacharparenleft}P\ outside\ Domain\ Q{\isacharparenright}\ {\isasymunion}\ Q{\isachardoublequoteclose}%
\begin{isamarkuptext}%
If a relation \isa{P} is a subrelation of another relation \isa{Q} on \isa{Q}'s
  domain, pasting \isa{Q} on \isa{P} is the same as forming their union.%
\end{isamarkuptext}%
\isamarkuptrue%
\isacommand{lemma}\isamarkupfalse%
\ paste{\isacharunderscore}subrel{\isacharcolon}\ \isanewline
\ \ \ \isakeyword{assumes}\ {\isachardoublequoteopen}P\ {\isacharbar}{\isacharbar}\ Domain\ Q\ {\isasymsubseteq}\ Q{\isachardoublequoteclose}\ \isanewline
\ \ \ \isakeyword{shows}\ {\isachardoublequoteopen}P\ {\isacharplus}{\isacharasterisk}\ Q\ {\isacharequal}\ P\ {\isasymunion}\ Q{\isachardoublequoteclose}\isanewline
%
\isadelimproof
\ \ \ %
\endisadelimproof
%
\isatagproof
\isacommand{unfolding}\isamarkupfalse%
\ paste{\isacharunderscore}def\ \isacommand{using}\isamarkupfalse%
\ assms\ outside{\isacharunderscore}union{\isacharunderscore}restrict\ \isacommand{by}\isamarkupfalse%
\ blast%
\endisatagproof
{\isafoldproof}%
%
\isadelimproof
%
\endisadelimproof
%
\begin{isamarkuptext}%
Pasting two relations with disjoint domains is the same as forming their union.%
\end{isamarkuptext}%
\isamarkuptrue%
\isacommand{lemma}\isamarkupfalse%
\ paste{\isacharunderscore}disj{\isacharunderscore}domains{\isacharcolon}\ \isanewline
\ \ \ \isakeyword{assumes}\ {\isachardoublequoteopen}Domain\ P\ {\isasyminter}\ Domain\ Q\ {\isacharequal}\ {\isacharbraceleft}{\isacharbraceright}{\isachardoublequoteclose}\ \isanewline
\ \ \ \isakeyword{shows}\ {\isachardoublequoteopen}P\ {\isacharplus}{\isacharasterisk}\ Q\ {\isacharequal}\ P\ {\isasymunion}\ Q{\isachardoublequoteclose}\isanewline
%
\isadelimproof
\ \ \ %
\endisadelimproof
%
\isatagproof
\isacommand{unfolding}\isamarkupfalse%
\ paste{\isacharunderscore}def\ Outside{\isacharunderscore}def\ \isacommand{using}\isamarkupfalse%
\ assms\ \isacommand{by}\isamarkupfalse%
\ fast%
\endisatagproof
{\isafoldproof}%
%
\isadelimproof
%
\endisadelimproof
%
\begin{isamarkuptext}%
A relation \isa{P} is equivalent to pasting its restriction to some set \isa{X} on 
  \isa{P\ outside\ X}.%
\end{isamarkuptext}%
\isamarkuptrue%
\isacommand{lemma}\isamarkupfalse%
\ paste{\isacharunderscore}outside{\isacharunderscore}restrict{\isacharcolon}\ {\isachardoublequoteopen}P\ {\isacharequal}\ {\isacharparenleft}P\ outside\ X{\isacharparenright}\ {\isacharplus}{\isacharasterisk}\ {\isacharparenleft}P\ {\isacharbar}{\isacharbar}\ X{\isacharparenright}{\isachardoublequoteclose}\isanewline
%
\isadelimproof
%
\endisadelimproof
%
\isatagproof
\isacommand{proof}\isamarkupfalse%
\ {\isacharminus}\isanewline
\ \ \isacommand{have}\isamarkupfalse%
\ {\isachardoublequoteopen}Domain\ {\isacharparenleft}P\ outside\ X{\isacharparenright}\ {\isasyminter}\ Domain\ {\isacharparenleft}P\ {\isacharbar}{\isacharbar}\ X{\isacharparenright}\ {\isacharequal}\ {\isacharbraceleft}{\isacharbraceright}{\isachardoublequoteclose}\isanewline
\ \ \ \ \isacommand{unfolding}\isamarkupfalse%
\ Outside{\isacharunderscore}def\ restrict{\isacharunderscore}def\ \isacommand{by}\isamarkupfalse%
\ fast\isanewline
\ \ \isacommand{moreover}\isamarkupfalse%
\ \isacommand{have}\isamarkupfalse%
\ {\isachardoublequoteopen}P\ {\isacharequal}\ P\ outside\ X\ {\isasymunion}\ P\ {\isacharbar}{\isacharbar}\ X{\isachardoublequoteclose}\ \isacommand{by}\isamarkupfalse%
\ {\isacharparenleft}rule\ outside{\isacharunderscore}union{\isacharunderscore}restrict{\isacharparenright}\isanewline
\ \ \isacommand{ultimately}\isamarkupfalse%
\ \isacommand{show}\isamarkupfalse%
\ {\isacharquery}thesis\ \isacommand{using}\isamarkupfalse%
\ paste{\isacharunderscore}disj{\isacharunderscore}domains\ \isacommand{by}\isamarkupfalse%
\ metis\isanewline
\isacommand{qed}\isamarkupfalse%
%
\endisatagproof
{\isafoldproof}%
%
\isadelimproof
%
\endisadelimproof
%
\begin{isamarkuptext}%
The domain of two pasted relations equals the union of their domains.%
\end{isamarkuptext}%
\isamarkuptrue%
\isacommand{lemma}\isamarkupfalse%
\ paste{\isacharunderscore}Domain{\isacharcolon}\ {\isachardoublequoteopen}Domain{\isacharparenleft}P\ {\isacharplus}{\isacharasterisk}\ Q{\isacharparenright}{\isacharequal}Domain\ P{\isasymunion}Domain\ Q{\isachardoublequoteclose}%
\isadelimproof
\ %
\endisadelimproof
%
\isatagproof
\isacommand{unfolding}\isamarkupfalse%
\ paste{\isacharunderscore}def\ Outside{\isacharunderscore}def\ \isacommand{by}\isamarkupfalse%
\ blast%
\endisatagproof
{\isafoldproof}%
%
\isadelimproof
%
\endisadelimproof
%
\begin{isamarkuptext}%
Pasting two relations yields a subrelation of their union.%
\end{isamarkuptext}%
\isamarkuptrue%
\isacommand{lemma}\isamarkupfalse%
\ paste{\isacharunderscore}sub{\isacharunderscore}Un{\isacharcolon}\ {\isachardoublequoteopen}P\ {\isacharplus}{\isacharasterisk}\ Q\ {\isasymsubseteq}\ P\ {\isasymunion}\ Q{\isachardoublequoteclose}\ \isanewline
%
\isadelimproof
\ \ %
\endisadelimproof
%
\isatagproof
\isacommand{unfolding}\isamarkupfalse%
\ paste{\isacharunderscore}def\ Outside{\isacharunderscore}def\ \isacommand{by}\isamarkupfalse%
\ fast%
\endisatagproof
{\isafoldproof}%
%
\isadelimproof
%
\endisadelimproof
%
\begin{isamarkuptext}%
The range of two pasted relations is a subset of the union of their ranges.%
\end{isamarkuptext}%
\isamarkuptrue%
\isacommand{lemma}\isamarkupfalse%
\ paste{\isacharunderscore}Range{\isacharcolon}\ {\isachardoublequoteopen}Range\ {\isacharparenleft}P\ {\isacharplus}{\isacharasterisk}\ Q{\isacharparenright}\ {\isasymsubseteq}\ Range\ P\ {\isasymunion}\ Range\ Q{\isachardoublequoteclose}\isanewline
%
\isadelimproof
\ \ %
\endisadelimproof
%
\isatagproof
\isacommand{using}\isamarkupfalse%
\ paste{\isacharunderscore}sub{\isacharunderscore}Un\ \isacommand{by}\isamarkupfalse%
\ blast%
\endisatagproof
{\isafoldproof}%
%
\isadelimproof
\isanewline
%
\endisadelimproof
%
\isadelimtheory
%
\endisadelimtheory
%
\isatagtheory
\isacommand{end}\isamarkupfalse%
%
\endisatagtheory
{\isafoldtheory}%
%
\isadelimtheory
%
\endisadelimtheory
\end{isabellebody}%
%%% Local Variables:
%%% mode: latex
%%% TeX-master: "root"
%%% End:

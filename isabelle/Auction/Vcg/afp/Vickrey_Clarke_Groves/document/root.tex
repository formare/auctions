\documentclass[11pt,a4paper]{article}
\usepackage{isabelle,isabellesym}
\usepackage{url}
% further packages required for unusual symbols (see also
% isabellesym.sty), use only when needed

%\usepackage{amssymb}
  %for \<leadsto>, \<box>, \<diamond>, \<sqsupset>, \<mho>, \<Join>,
  %\<lhd>, \<lesssim>, \<greatersim>, \<lessapprox>, \<greaterapprox>,
  %\<triangleq>, \<yen>, \<lozenge>

%\usepackage{eurosym}
  %for \<euro>

%\usepackage[only,bigsqcap]{stmaryrd}
  %for \<Sqinter>

%\usepackage{eufrak}
  %for \<AA> ... \<ZZ>, \<aa> ... \<zz> (also included in amssymb)

%\usepackage{textcomp}
  %for \<onequarter>, \<onehalf>, \<threequarters>, \<degree>, \<cent>,
  %\<currency>

% this should be the last package used
\usepackage{pdfsetup}

% urls in roman style, theory text in math-similar italics
\urlstyle{rm}
\isabellestyle{it}

% for uniform font size
%\renewcommand{\isastyle}{\isastyleminor}


\begin{document}

\title{Vickrey-Clarke-Groves (VCG) Auctions}
\author{M. B. Caminati\footnote{School of Computer Science, University of Birmingham, UK}\addtocounter{footnote}{-1}
  \and M. Kerber\footnotemark
  \and C. Lange\footnote{Fraunhofer IAIS and University of Bonn, Germany, and School of Computer Science, University of Birmingham, UK}
  \and C. Rowat\footnote{Department of Economics, University of Birmingham, UK}}

\maketitle

\begin{abstract}
A VCG auction (named after their inventors Vickrey, Clarke, and
Groves) is a generalization of the single-good, second price Vickrey
auction to the case of a combinatorial auction (multiple goods, from
which any participant can bid on each possible combination). We
formalize in this entry VCG auctions, including tie-breaking and prove
that the functions for the allocation and the price determination are
well-defined. Furthermore we show that the allocation function
allocates goods only to participants, only goods in the auction are
allocated, and no good is allocated twice. We also show that the price
function is non-negative. These properties also hold for the
automatically extracted Scala code.
\end{abstract}


\tableofcontents

\section{Introduction}
An auction mechanism is mathematically represented through a pair of
functions $(a, p)$: the first describes how some given goods at stake
are allocated among the bidders (also called participants or agents),
while the second specifies how much each bidder pays following this
allocation.  Each possible output of this pair of functions is
referred to as an outcome of the auction. Both functions take the same
argument, which is another function, commonly called a bid vector $b$;
it describes how much each bidder values the possible outcomes of the
auction. This valuation is usually expressed through money.  In this
setting, some common questions are the study of the quantitative and
qualitative properties of a given auction mechanism (e.g., whether it
maximizes some relevant quantity, such as revenue, or whether it is
efficient, that is, whether it allocates the item to the bidder who
values it most), and the study of the algorithms running it (in
particular, their correctness).

A VCG auction (named after their inventors Vickrey, Clarke, and
Groves) is a generalization of the single-good, second price Vickrey
auction to the case of a combinatorial auction (multiple goods, from
which any participant can bid on each possible combination). We
formalize in this entry VCG auctions, including tie-breaking and prove
that the functions $a$ and $p$ are well-defined. Furthermore we show
that the allocation function $a$ allocates goods only to participants,
only goods in the auction are allocated, and no good is allocated
twice. Furthermore we show that the price function $p$ is
non-negative. These properties also hold for the automatically
extracted Scala code. For further details on the formalization, see
\cite{ec15}. For background information on VCG auctions, see \cite{cramton}.

  
The following files are part of the Auction Theory Toolbox
(ATT)~\cite{github} developed in the ForMaRE project~\cite{formare}.
The theories \texttt{CombinatorialAuction.thy},
\texttt{StrictCombinatorialAuction.thy} and
\texttt{UniformTieBreaking.thy} contain the relevant definitions and
theorems; \texttt{CombinatorialAuctionExamples.thy} and
\texttt{CombinatorialAuctionCodeExtraction.thy} present simple helper
definitions to run them on given examples and to export
them to the Scala language, respectively; \texttt{FirstPrice.thy}
shows how easy it is to adapt the definitions to the first price
combinatorial auction.  The remaining theories contain more general
mathematical definitions and theorems.


\subsection{Rationale for developing set theory as replacing one bidder in a second price auction}

Throughout the whole ATT, there is a duality in the way mathematical
notions are modeled: either through objects typical of lambda calculus
and HOL (lambda-abstracted functions and lists, for example) or
through objects typical of set theory (for example, relations,
intersection, union, set difference, Cartesian product).

This is possible because inside HOL, it is possible to model a
simply-typed set theory which, although quite restrained if compared
to, e.g., ZFC, is powerful enough for many standard mathematical
purposes.

ATT freely adopts one approach, the other, or a mixture thereof, depending on technical and
expressive convenience.
A technical discussion of this topic can be found in~\cite{cicm2014}.

\subsection{Bridging}

One of the differences between the approaches of functional
definitions on the one hand and classical (often set-theoretical)
definitions on the other hand is that, commonly (although not always),
the first approach is better suited to produce Isabelle/HOL
definitions which are computable (typically, inductive definitions);
while the definitions from the second approach are often more general
(e.g., encompassing infinite sets), closer to pen-and-paper
mathematics, but also not computable.  This means that many theorems
are proved with respect to definitions of the second type, while in
the end we want them to apply to definitions of the first type,
because we want our theorems to hold for the code we will be actually
running.  Hence, bridging theorems are needed, showing that, for the
limited portions of objects for which we state both kinds of
definitions, they are the same.

\subsection{Main theorems}

The main theorems about VCG auctions are:
\begin{description}
  \item[the definiteness theorem:] our definitions grant that there is exactly one solution; this is 
    ensured by \texttt{vcgaDefiniteness}.
  \item[PairwiseDisjointAllocations:] no good is allocated to more than one participant.
  \item[onlyGoodsAreAllocated:] only the actually available goods are allocated.
  \item[the adequacy theorem:] the solution provided by our algorithm is indeed the one prescribed by
    standard pen-and-paper definition.
  \item[NonnegPrices:] no participant ends up paying a negative price 
    (e.g., no participant receives money at the end of the auction).
  \item[Bridging theorems:] as discussed above, such theorems permit to apply the theorems in this 
    list to the executable code Isabelle generates.
\end{description}

\subsection{Scala code extraction}

Isabelle permits to generate, from our definition of VCG, Scala code
to run any VCG auction.  Use
\texttt{CombinatorialAuctionCodeExtraction.thy} for this. This code is
in the form of Scala functions which can be evaluated on any input
(e.g., a bidvector) to return the resulting allocation and prices.

To deploy such functions use the provided Scala
wrapper (taking care of the output and including sample inputs).  In
order to do so, you can evaluate inside Isabelle/JEdit the file
\texttt{CombinatorialAuctionCodeExtraction.thy} (position the cursor on its
last line and wait for Isabelle/JEdit to end all its processing).
This will result in the file
\texttt{/dev/shm/VCG-withoutWrapper.scala}, which can be automatically
appended to the wrapper by running the shell script at
the end of \texttt{CombinatorialAuctionCodeExtraction.thy}. For details of how to run the Scala code see
\url{http://www.cs.bham.ac.uk/research/projects/formare/vcg.php}.


% sane default for proof documents
\parindent 0pt\parskip 0.5ex

% generated text of all theories
%
\begin{isabellebody}%
\def\isabellecontext{SetUtils}%
%
\isamarkupheader{Additional material that we would have expected in Set.thy%
}
\isamarkuptrue%
%
\isadelimtheory
%
\endisadelimtheory
%
\isatagtheory
\isacommand{theory}\isamarkupfalse%
\ SetUtils\isanewline
\isakeyword{imports}\isanewline
\ \ Main\isanewline
\isanewline
\isakeyword{begin}%
\endisatagtheory
{\isafoldtheory}%
%
\isadelimtheory
%
\endisadelimtheory
%
\isamarkupsection{Equality%
}
\isamarkuptrue%
%
\begin{isamarkuptext}%
An inference (introduction) rule that combines \isa{{\isasymlbrakk}{\isacharquery}A\ {\isasymsubseteq}\ {\isacharquery}B{\isacharsemicolon}\ {\isacharquery}B\ {\isasymsubseteq}\ {\isacharquery}A{\isasymrbrakk}\ {\isasymLongrightarrow}\ {\isacharquery}A\ {\isacharequal}\ {\isacharquery}B} and \isa{{\isacharparenleft}{\isasymAnd}x{\isachardot}\ x\ {\isasymin}\ {\isacharquery}A\ {\isasymLongrightarrow}\ x\ {\isasymin}\ {\isacharquery}B{\isacharparenright}\ {\isasymLongrightarrow}\ {\isacharquery}A\ {\isasymsubseteq}\ {\isacharquery}B} to a single step%
\end{isamarkuptext}%
\isamarkuptrue%
\isacommand{lemma}\isamarkupfalse%
\ equalitySubsetI{\isacharcolon}\ {\isachardoublequoteopen}{\isacharparenleft}{\isasymAnd}x\ {\isachardot}\ x\ {\isasymin}\ A\ {\isasymLongrightarrow}\ x\ {\isasymin}\ B{\isacharparenright}\ {\isasymLongrightarrow}\ {\isacharparenleft}{\isasymAnd}x\ {\isachardot}\ x\ {\isasymin}\ B\ {\isasymLongrightarrow}\ x\ {\isasymin}\ A{\isacharparenright}\ {\isasymLongrightarrow}\ A\ {\isacharequal}\ B{\isachardoublequoteclose}\ \isanewline
%
\isadelimproof
\ \ \ \ \ \ %
\endisadelimproof
%
\isatagproof
\isacommand{by}\isamarkupfalse%
\ blast%
\endisatagproof
{\isafoldproof}%
%
\isadelimproof
%
\endisadelimproof
%
\isamarkupsection{Trivial sets%
}
\isamarkuptrue%
%
\begin{isamarkuptext}%
A trivial set (i.e. singleton or empty), as in Mizar%
\end{isamarkuptext}%
\isamarkuptrue%
\isacommand{definition}\isamarkupfalse%
\ trivial\ \isakeyword{where}\ {\isachardoublequoteopen}trivial\ x\ {\isacharequal}\ {\isacharparenleft}x\ {\isasymsubseteq}\ {\isacharbraceleft}the{\isacharunderscore}elem\ x{\isacharbraceright}{\isacharparenright}{\isachardoublequoteclose}%
\begin{isamarkuptext}%
The empty set is trivial.%
\end{isamarkuptext}%
\isamarkuptrue%
\isacommand{lemma}\isamarkupfalse%
\ trivial{\isacharunderscore}empty{\isacharcolon}\ {\isachardoublequoteopen}trivial\ {\isacharbraceleft}{\isacharbraceright}{\isachardoublequoteclose}\ \isanewline
%
\isadelimproof
\ \ \ \ \ \ %
\endisadelimproof
%
\isatagproof
\isacommand{unfolding}\isamarkupfalse%
\ trivial{\isacharunderscore}def\ \isacommand{by}\isamarkupfalse%
\ {\isacharparenleft}rule\ empty{\isacharunderscore}subsetI{\isacharparenright}%
\endisatagproof
{\isafoldproof}%
%
\isadelimproof
%
\endisadelimproof
%
\begin{isamarkuptext}%
A singleton set is trivial.%
\end{isamarkuptext}%
\isamarkuptrue%
\isacommand{lemma}\isamarkupfalse%
\ trivial{\isacharunderscore}singleton{\isacharcolon}\ {\isachardoublequoteopen}trivial\ {\isacharbraceleft}x{\isacharbraceright}{\isachardoublequoteclose}\ \isanewline
%
\isadelimproof
\ \ \ \ \ \ %
\endisadelimproof
%
\isatagproof
\isacommand{unfolding}\isamarkupfalse%
\ trivial{\isacharunderscore}def\ \isacommand{by}\isamarkupfalse%
\ simp%
\endisatagproof
{\isafoldproof}%
%
\isadelimproof
%
\endisadelimproof
%
\begin{isamarkuptext}%
If a trivial set has a singleton subset, the latter is unique.%
\end{isamarkuptext}%
\isamarkuptrue%
\isacommand{lemma}\isamarkupfalse%
\ singleton{\isacharunderscore}sub{\isacharunderscore}trivial{\isacharunderscore}uniq{\isacharcolon}\isanewline
\ \ \ \ \ \ \isakeyword{fixes}\ \ \ x\ X\isanewline
\ \ \ \ \ \ \isakeyword{assumes}\ {\isachardoublequoteopen}{\isacharbraceleft}x{\isacharbraceright}\ {\isasymsubseteq}\ X{\isachardoublequoteclose}\ \isakeyword{and}\ {\isachardoublequoteopen}trivial\ X{\isachardoublequoteclose}\isanewline
\ \ \ \ \ \ \isakeyword{shows}\ \ \ {\isachardoublequoteopen}x\ {\isacharequal}\ the{\isacharunderscore}elem\ X{\isachardoublequoteclose}\isanewline
%
\isadelimproof
\isanewline
\ \ \ \ \ \ %
\endisadelimproof
%
\isatagproof
\isacommand{using}\isamarkupfalse%
\ assms\ \isacommand{unfolding}\isamarkupfalse%
\ trivial{\isacharunderscore}def\ \isacommand{by}\isamarkupfalse%
\ fast%
\endisatagproof
{\isafoldproof}%
%
\isadelimproof
%
\endisadelimproof
%
\begin{isamarkuptext}%
Any subset of a trivial set is trivial.%
\end{isamarkuptext}%
\isamarkuptrue%
\isacommand{lemma}\isamarkupfalse%
\ trivial{\isacharunderscore}subset{\isacharcolon}\ \isakeyword{fixes}\ X\ Y\ \isakeyword{assumes}\ {\isachardoublequoteopen}trivial\ Y{\isachardoublequoteclose}\ \isakeyword{assumes}\ {\isachardoublequoteopen}X\ {\isasymsubseteq}\ Y{\isachardoublequoteclose}\ \isanewline
\ \ \ \ \ \ \ \ \ \ \ \ \ \ \ \ \ \ \ \ \ \ \isakeyword{shows}\ {\isachardoublequoteopen}trivial\ X{\isachardoublequoteclose}\isanewline
%
\isadelimproof
\isanewline
\ \ \ \ \ \ \ \ %
\endisadelimproof
%
\isatagproof
\isacommand{using}\isamarkupfalse%
\ assms\ \isacommand{unfolding}\isamarkupfalse%
\ trivial{\isacharunderscore}def\ \isanewline
\ \ \ \ \ \ \ \ \isacommand{by}\isamarkupfalse%
\ {\isacharparenleft}metis\ {\isacharparenleft}full{\isacharunderscore}types{\isacharparenright}\ subset{\isacharunderscore}empty\ subset{\isacharunderscore}insertI{\isadigit{2}}\ subset{\isacharunderscore}singletonD{\isacharparenright}%
\endisatagproof
{\isafoldproof}%
%
\isadelimproof
%
\endisadelimproof
%
\begin{isamarkuptext}%
There are no two different elements in a trivial set.%
\end{isamarkuptext}%
\isamarkuptrue%
\isacommand{lemma}\isamarkupfalse%
\ trivial{\isacharunderscore}imp{\isacharunderscore}no{\isacharunderscore}distinct{\isacharcolon}\isanewline
\ \ \isakeyword{assumes}\ triv{\isacharcolon}\ {\isachardoublequoteopen}trivial\ X{\isachardoublequoteclose}\ \isakeyword{and}\ x{\isacharcolon}\ {\isachardoublequoteopen}x\ {\isasymin}\ X{\isachardoublequoteclose}\ \isakeyword{and}\ y{\isacharcolon}\ {\isachardoublequoteopen}y\ {\isasymin}\ X{\isachardoublequoteclose}\isanewline
\ \ \isakeyword{shows}\ \ \ {\isachardoublequoteopen}x\ {\isacharequal}\ y{\isachardoublequoteclose}\isanewline
%
\isadelimproof
\isanewline
\ \ %
\endisadelimproof
%
\isatagproof
\isacommand{using}\isamarkupfalse%
\ assms\ \isacommand{by}\isamarkupfalse%
\ {\isacharparenleft}metis\ empty{\isacharunderscore}subsetI\ insert{\isacharunderscore}subset\ singleton{\isacharunderscore}sub{\isacharunderscore}trivial{\isacharunderscore}uniq{\isacharparenright}%
\endisatagproof
{\isafoldproof}%
%
\isadelimproof
%
\endisadelimproof
%
\isamarkupsection{The image of a set under a function%
}
\isamarkuptrue%
%
\begin{isamarkuptext}%
an equivalent notation for the image of a set, using set comprehension%
\end{isamarkuptext}%
\isamarkuptrue%
\isacommand{lemma}\isamarkupfalse%
\ image{\isacharunderscore}Collect{\isacharunderscore}mem{\isacharcolon}\ {\isachardoublequoteopen}{\isacharbraceleft}\ f\ x\ {\isacharbar}\ x\ {\isachardot}\ x\ {\isasymin}\ S\ {\isacharbraceright}\ {\isacharequal}\ f\ {\isacharbackquote}\ S{\isachardoublequoteclose}\ \isanewline
%
\isadelimproof
\ \ \ \ \ \ %
\endisadelimproof
%
\isatagproof
\isacommand{by}\isamarkupfalse%
\ auto%
\endisatagproof
{\isafoldproof}%
%
\isadelimproof
%
\endisadelimproof
%
\isamarkupsection{Big Union%
}
\isamarkuptrue%
%
\begin{isamarkuptext}%
An element is in the union of a family of sets if it is in one of the family's member sets.%
\end{isamarkuptext}%
\isamarkuptrue%
\isacommand{lemma}\isamarkupfalse%
\ Union{\isacharunderscore}member{\isacharcolon}\ {\isachardoublequoteopen}{\isacharparenleft}{\isasymexists}\ S\ {\isasymin}\ F\ {\isachardot}\ x\ {\isasymin}\ S{\isacharparenright}\ {\isasymlongleftrightarrow}\ x\ {\isasymin}\ {\isasymUnion}\ F{\isachardoublequoteclose}\ \isanewline
%
\isadelimproof
\ \ \ \ \ \ %
\endisadelimproof
%
\isatagproof
\isacommand{by}\isamarkupfalse%
\ blast%
\endisatagproof
{\isafoldproof}%
%
\isadelimproof
%
\endisadelimproof
%
\isamarkupsection{Miscellaneous%
}
\isamarkuptrue%
\isacommand{lemma}\isamarkupfalse%
\ trivial{\isacharunderscore}subset{\isacharunderscore}non{\isacharunderscore}empty{\isacharcolon}\ \isakeyword{assumes}\ {\isachardoublequoteopen}trivial\ t{\isachardoublequoteclose}\ {\isachardoublequoteopen}t\ {\isasyminter}\ X\ {\isasymnoteq}\ {\isacharbraceleft}{\isacharbraceright}{\isachardoublequoteclose}\ \isanewline
\ \ \ \ \ \ \ \ \ \ \ \ \isakeyword{shows}\ \ \ {\isachardoublequoteopen}t\ {\isasymsubseteq}\ X{\isachardoublequoteclose}\ \isanewline
%
\isadelimproof
\ \ \ \ \ \ %
\endisadelimproof
%
\isatagproof
\isacommand{using}\isamarkupfalse%
\ trivial{\isacharunderscore}def\ assms\ in{\isacharunderscore}mono\ \isacommand{by}\isamarkupfalse%
\ fast%
\endisatagproof
{\isafoldproof}%
%
\isadelimproof
\isanewline
%
\endisadelimproof
\isanewline
\isacommand{lemma}\isamarkupfalse%
\ trivial{\isacharunderscore}implies{\isacharunderscore}finite{\isacharcolon}\ \isakeyword{assumes}\ {\isachardoublequoteopen}trivial\ X{\isachardoublequoteclose}\ \isanewline
\ \ \ \ \ \ \ \ \ \ \ \ \isakeyword{shows}\ \ \ {\isachardoublequoteopen}finite\ X{\isachardoublequoteclose}\ \isanewline
%
\isadelimproof
\ \ \ \ \ \ %
\endisadelimproof
%
\isatagproof
\isacommand{using}\isamarkupfalse%
\ assms\ \isacommand{by}\isamarkupfalse%
\ {\isacharparenleft}metis\ finite{\isachardot}simps\ subset{\isacharunderscore}singletonD\ trivial{\isacharunderscore}def{\isacharparenright}%
\endisatagproof
{\isafoldproof}%
%
\isadelimproof
\isanewline
%
\endisadelimproof
\isanewline
\isanewline
\isacommand{lemma}\isamarkupfalse%
\ lm{\isadigit{0}}{\isadigit{1}}{\isacharcolon}\ \isakeyword{assumes}\ {\isachardoublequoteopen}trivial\ {\isacharparenleft}A\ {\isasymtimes}\ B{\isacharparenright}{\isachardoublequoteclose}\ \isanewline
\ \ \ \ \ \ \ \ \ \ \ \ \ \ \isakeyword{shows}\ \ \ {\isachardoublequoteopen}{\isacharparenleft}finite\ {\isacharparenleft}A{\isasymtimes}B{\isacharparenright}\ {\isacharampersand}\ card\ A\ {\isacharasterisk}\ {\isacharparenleft}card\ B{\isacharparenright}\ {\isasymle}\ {\isadigit{1}}{\isacharparenright}{\isachardoublequoteclose}\ \isanewline
%
\isadelimproof
\ \ \ \ \ \ %
\endisadelimproof
%
\isatagproof
\isacommand{using}\isamarkupfalse%
\ trivial{\isacharunderscore}def\ assms\ One{\isacharunderscore}nat{\isacharunderscore}def\ card{\isacharunderscore}cartesian{\isacharunderscore}product\ card{\isacharunderscore}empty\ card{\isacharunderscore}insert{\isacharunderscore}disjoint\isanewline
\ \ \ \ \ \ \ \ \ \ \ \ empty{\isacharunderscore}iff\ finite{\isachardot}emptyI\ le{\isadigit{0}}\ trivial{\isacharunderscore}implies{\isacharunderscore}finite\ order{\isacharunderscore}refl\ subset{\isacharunderscore}singletonD\ \isacommand{by}\isamarkupfalse%
\ {\isacharparenleft}metis{\isacharparenleft}no{\isacharunderscore}types{\isacharparenright}{\isacharparenright}%
\endisatagproof
{\isafoldproof}%
%
\isadelimproof
\isanewline
%
\endisadelimproof
\isanewline
\isacommand{lemma}\isamarkupfalse%
\ lm{\isadigit{0}}{\isadigit{2}}{\isacharcolon}\ \isakeyword{assumes}\ {\isachardoublequoteopen}finite\ X{\isachardoublequoteclose}\ \isanewline
\ \ \ \ \ \ \ \ \ \ \ \ \isakeyword{shows}\ \ \ {\isachardoublequoteopen}trivial\ X{\isacharequal}{\isacharparenleft}card\ X\ {\isasymle}\ {\isadigit{1}}{\isacharparenright}{\isachardoublequoteclose}\ \isanewline
%
\isadelimproof
\ \ \ \ \ \ %
\endisadelimproof
%
\isatagproof
\isacommand{using}\isamarkupfalse%
\ assms\ One{\isacharunderscore}nat{\isacharunderscore}def\ card{\isacharunderscore}empty\ card{\isacharunderscore}insert{\isacharunderscore}if\ card{\isacharunderscore}mono\ card{\isacharunderscore}seteq\ empty{\isacharunderscore}iff\ \isanewline
\ \ \ \ \ \ \ \ \ \ \ \ empty{\isacharunderscore}subsetI\ finite{\isachardot}cases\ finite{\isachardot}emptyI\ finite{\isacharunderscore}insert\ insert{\isacharunderscore}mono\ \isanewline
\ \ \ \ \ \ \ \ \ \ \ \ trivial{\isacharunderscore}def\ trivial{\isacharunderscore}singleton\isanewline
\ \ \ \ \ \ \isacommand{by}\isamarkupfalse%
\ {\isacharparenleft}metis{\isacharparenleft}no{\isacharunderscore}types{\isacharparenright}{\isacharparenright}%
\endisatagproof
{\isafoldproof}%
%
\isadelimproof
\isanewline
%
\endisadelimproof
\isanewline
\isacommand{lemma}\isamarkupfalse%
\ lm{\isadigit{0}}{\isadigit{3}}{\isacharcolon}\ \isakeyword{shows}\ {\isachardoublequoteopen}trivial\ {\isacharbraceleft}x{\isacharbraceright}{\isachardoublequoteclose}\ \isanewline
%
\isadelimproof
\ \ \ \ \ \ %
\endisadelimproof
%
\isatagproof
\isacommand{by}\isamarkupfalse%
\ {\isacharparenleft}metis\ order{\isacharunderscore}refl\ the{\isacharunderscore}elem{\isacharunderscore}eq\ trivial{\isacharunderscore}def{\isacharparenright}%
\endisatagproof
{\isafoldproof}%
%
\isadelimproof
\isanewline
%
\endisadelimproof
\isanewline
\isacommand{lemma}\isamarkupfalse%
\ lm{\isadigit{0}}{\isadigit{4}}{\isacharcolon}\ \isakeyword{assumes}\ {\isachardoublequoteopen}trivial\ X{\isachardoublequoteclose}\ {\isachardoublequoteopen}{\isacharbraceleft}x{\isacharbraceright}\ {\isasymsubseteq}\ X{\isachardoublequoteclose}\ \isanewline
\ \ \ \ \ \ \ \ \ \ \ \ \isakeyword{shows}\ \ \ {\isachardoublequoteopen}{\isacharbraceleft}x{\isacharbraceright}\ {\isacharequal}\ X{\isachardoublequoteclose}\ \isanewline
%
\isadelimproof
\ \ \ \ \ \ %
\endisadelimproof
%
\isatagproof
\isacommand{using}\isamarkupfalse%
\ singleton{\isacharunderscore}sub{\isacharunderscore}trivial{\isacharunderscore}uniq\ assms\ \isacommand{by}\isamarkupfalse%
\ {\isacharparenleft}metis\ subset{\isacharunderscore}antisym\ trivial{\isacharunderscore}def{\isacharparenright}%
\endisatagproof
{\isafoldproof}%
%
\isadelimproof
\isanewline
%
\endisadelimproof
\isanewline
\isacommand{lemma}\isamarkupfalse%
\ lm{\isadigit{0}}{\isadigit{5}}{\isacharcolon}\ \isakeyword{assumes}\ {\isachardoublequoteopen}{\isasymnot}\ trivial\ X{\isachardoublequoteclose}\ {\isachardoublequoteopen}trivial\ T{\isachardoublequoteclose}\ \isanewline
\ \ \ \ \ \ \ \ \ \ \ \ \isakeyword{shows}\ \ \ {\isachardoublequoteopen}X\ {\isacharminus}\ T\ \ {\isasymnoteq}\ \ {\isacharbraceleft}{\isacharbraceright}{\isachardoublequoteclose}\isanewline
%
\isadelimproof
\ \ \ \ \ \ %
\endisadelimproof
%
\isatagproof
\isacommand{using}\isamarkupfalse%
\ assms\ \isacommand{by}\isamarkupfalse%
\ {\isacharparenleft}metis\ Diff{\isacharunderscore}iff\ empty{\isacharunderscore}iff\ subsetI\ trivial{\isacharunderscore}subset{\isacharparenright}%
\endisatagproof
{\isafoldproof}%
%
\isadelimproof
\isanewline
%
\endisadelimproof
\isanewline
\isacommand{lemma}\isamarkupfalse%
\ lm{\isadigit{0}}{\isadigit{6}}{\isacharcolon}\ \isakeyword{assumes}\ {\isachardoublequoteopen}{\isacharparenleft}finite\ {\isacharparenleft}A\ {\isasymtimes}\ B{\isacharparenright}\ \ {\isacharampersand}\ \ card\ A\ {\isacharasterisk}\ {\isacharparenleft}card\ B{\isacharparenright}\ {\isasymle}\ {\isadigit{1}}{\isacharparenright}{\isachardoublequoteclose}\ \isanewline
\ \ \ \ \ \ \ \ \ \ \ \ \ \ \isakeyword{shows}\ \ \ {\isachardoublequoteopen}trivial\ {\isacharparenleft}A\ {\isasymtimes}\ B{\isacharparenright}{\isachardoublequoteclose}\ \isanewline
%
\isadelimproof
\ \ \ \ \ \ %
\endisadelimproof
%
\isatagproof
\isacommand{unfolding}\isamarkupfalse%
\ trivial{\isacharunderscore}def\ \isacommand{using}\isamarkupfalse%
\ trivial{\isacharunderscore}def\ assms\ \isacommand{by}\isamarkupfalse%
\ {\isacharparenleft}metis\ card{\isacharunderscore}cartesian{\isacharunderscore}product\ lm{\isadigit{0}}{\isadigit{2}}{\isacharparenright}%
\endisatagproof
{\isafoldproof}%
%
\isadelimproof
\isanewline
%
\endisadelimproof
\isanewline
\isacommand{lemma}\isamarkupfalse%
\ lm{\isadigit{0}}{\isadigit{7}}{\isacharcolon}\ {\isachardoublequoteopen}trivial\ {\isacharparenleft}A\ {\isasymtimes}\ B{\isacharparenright}\ {\isacharequal}\ {\isacharparenleft}finite\ {\isacharparenleft}A\ {\isasymtimes}\ B{\isacharparenright}\ {\isacharampersand}\ card\ A\ {\isacharasterisk}\ {\isacharparenleft}card\ B{\isacharparenright}\ {\isasymle}\ {\isadigit{1}}{\isacharparenright}{\isachardoublequoteclose}\ \isanewline
%
\isadelimproof
\ \ \ \ \ \ %
\endisadelimproof
%
\isatagproof
\isacommand{using}\isamarkupfalse%
\ lm{\isadigit{0}}{\isadigit{1}}\ lm{\isadigit{0}}{\isadigit{6}}\ \isacommand{by}\isamarkupfalse%
\ blast%
\endisatagproof
{\isafoldproof}%
%
\isadelimproof
\isanewline
%
\endisadelimproof
\isanewline
\isacommand{lemma}\isamarkupfalse%
\ trivial{\isacharunderscore}empty{\isacharunderscore}or{\isacharunderscore}singleton{\isacharcolon}\ {\isachardoublequoteopen}trivial\ X\ {\isacharequal}\ {\isacharparenleft}X\ {\isacharequal}\ {\isacharbraceleft}{\isacharbraceright}\ {\isasymor}\ X\ {\isacharequal}\ {\isacharbraceleft}the{\isacharunderscore}elem\ X{\isacharbraceright}{\isacharparenright}{\isachardoublequoteclose}\ \isanewline
%
\isadelimproof
\ \ \ \ \ \ %
\endisadelimproof
%
\isatagproof
\isacommand{by}\isamarkupfalse%
\ {\isacharparenleft}metis\ subset{\isacharunderscore}singletonD\ trivial{\isacharunderscore}def\ trivial{\isacharunderscore}empty\ trivial{\isacharunderscore}singleton{\isacharparenright}%
\endisatagproof
{\isafoldproof}%
%
\isadelimproof
\isanewline
%
\endisadelimproof
\isanewline
\isacommand{lemma}\isamarkupfalse%
\ trivial{\isacharunderscore}cartesian{\isacharcolon}\ \isakeyword{assumes}\ {\isachardoublequoteopen}trivial\ X{\isachardoublequoteclose}\ {\isachardoublequoteopen}trivial\ Y{\isachardoublequoteclose}\ \isanewline
\ \ \ \ \ \ \ \ \ \ \ \ \isakeyword{shows}\ \ \ {\isachardoublequoteopen}trivial\ {\isacharparenleft}X\ {\isasymtimes}\ Y{\isacharparenright}{\isachardoublequoteclose}\isanewline
%
\isadelimproof
\ \ \ \ \ \ %
\endisadelimproof
%
\isatagproof
\isacommand{using}\isamarkupfalse%
\ assms\ lm{\isadigit{0}}{\isadigit{7}}\ One{\isacharunderscore}nat{\isacharunderscore}def\ Sigma{\isacharunderscore}empty{\isadigit{1}}\ Sigma{\isacharunderscore}empty{\isadigit{2}}\ card{\isacharunderscore}empty\ card{\isacharunderscore}insert{\isacharunderscore}if\isanewline
\ \ \ \ \ \ \ \ \ \ \ \ finite{\isacharunderscore}SigmaI\ trivial{\isacharunderscore}implies{\isacharunderscore}finite\ nat{\isacharunderscore}{\isadigit{1}}{\isacharunderscore}eq{\isacharunderscore}mult{\isacharunderscore}iff\ order{\isacharunderscore}refl\ subset{\isacharunderscore}singletonD\ trivial{\isacharunderscore}def\ trivial{\isacharunderscore}empty\isanewline
\ \ \ \ \ \ \isacommand{by}\isamarkupfalse%
\ {\isacharparenleft}metis\ {\isacharparenleft}full{\isacharunderscore}types{\isacharparenright}{\isacharparenright}%
\endisatagproof
{\isafoldproof}%
%
\isadelimproof
\isanewline
%
\endisadelimproof
\isanewline
\isacommand{lemma}\isamarkupfalse%
\ trivial{\isacharunderscore}same{\isacharcolon}\ {\isachardoublequoteopen}trivial\ X\ {\isacharequal}\ {\isacharparenleft}{\isasymforall}x{\isadigit{1}}\ {\isasymin}\ X{\isachardot}\ {\isasymforall}x{\isadigit{2}}\ {\isasymin}\ X{\isachardot}\ x{\isadigit{1}}\ {\isacharequal}\ x{\isadigit{2}}{\isacharparenright}{\isachardoublequoteclose}\ \isanewline
%
\isadelimproof
\ \ \ \ \ \ %
\endisadelimproof
%
\isatagproof
\isacommand{using}\isamarkupfalse%
\ trivial{\isacharunderscore}def\ trivial{\isacharunderscore}imp{\isacharunderscore}no{\isacharunderscore}distinct\ ex{\isacharunderscore}in{\isacharunderscore}conv\ insertCI\ subsetI\ subset{\isacharunderscore}singletonD\isanewline
\ \ \ \ \ \ \ \ \ \ \ \ trivial{\isacharunderscore}singleton\ \isanewline
\ \ \ \ \ \ \isacommand{by}\isamarkupfalse%
\ {\isacharparenleft}metis\ {\isacharparenleft}no{\isacharunderscore}types{\isacharcomma}\ hide{\isacharunderscore}lams{\isacharparenright}{\isacharparenright}%
\endisatagproof
{\isafoldproof}%
%
\isadelimproof
\isanewline
%
\endisadelimproof
\isanewline
\isacommand{lemma}\isamarkupfalse%
\ lm{\isadigit{0}}{\isadigit{8}}{\isacharcolon}\ \isakeyword{assumes}\ {\isachardoublequoteopen}{\isacharparenleft}Pow\ X\ {\isasymsubseteq}\ {\isacharbraceleft}{\isacharbraceleft}{\isacharbraceright}{\isacharcomma}X{\isacharbraceright}{\isacharparenright}{\isachardoublequoteclose}\ \isanewline
\ \ \ \ \ \ \ \ \ \ \ \ \ \ \isakeyword{shows}\ \ {\isachardoublequoteopen}trivial\ X{\isachardoublequoteclose}\ \isanewline
%
\isadelimproof
\ \ \ \ \ \ %
\endisadelimproof
%
\isatagproof
\isacommand{unfolding}\isamarkupfalse%
\ trivial{\isacharunderscore}same\ \isacommand{using}\isamarkupfalse%
\ assms\ \isacommand{by}\isamarkupfalse%
\ auto%
\endisatagproof
{\isafoldproof}%
%
\isadelimproof
\isanewline
%
\endisadelimproof
\isanewline
\isacommand{lemma}\isamarkupfalse%
\ lm{\isadigit{0}}{\isadigit{9}}{\isacharcolon}\ \isakeyword{assumes}\ {\isachardoublequoteopen}trivial\ X{\isachardoublequoteclose}\ \isanewline
\ \ \ \ \ \ \ \ \ \ \ \ \ \ \isakeyword{shows}\ {\isachardoublequoteopen}{\isacharparenleft}Pow\ X\ {\isasymsubseteq}\ {\isacharbraceleft}{\isacharbraceleft}{\isacharbraceright}{\isacharcomma}X{\isacharbraceright}{\isacharparenright}{\isachardoublequoteclose}\ \isanewline
%
\isadelimproof
\ \ \ \ \ \ %
\endisadelimproof
%
\isatagproof
\isacommand{using}\isamarkupfalse%
\ assms\ trivial{\isacharunderscore}same\ \isacommand{by}\isamarkupfalse%
\ fast%
\endisatagproof
{\isafoldproof}%
%
\isadelimproof
\isanewline
%
\endisadelimproof
\isanewline
\isacommand{lemma}\isamarkupfalse%
\ lm{\isadigit{1}}{\isadigit{0}}{\isacharcolon}\ {\isachardoublequoteopen}trivial\ X\ {\isacharequal}\ {\isacharparenleft}Pow\ X\ {\isasymsubseteq}\ {\isacharbraceleft}{\isacharbraceleft}{\isacharbraceright}{\isacharcomma}X{\isacharbraceright}{\isacharparenright}{\isachardoublequoteclose}\ \isanewline
%
\isadelimproof
\ \ \ \ \ \ %
\endisadelimproof
%
\isatagproof
\isacommand{using}\isamarkupfalse%
\ lm{\isadigit{0}}{\isadigit{8}}\ lm{\isadigit{0}}{\isadigit{9}}\ \isacommand{by}\isamarkupfalse%
\ metis%
\endisatagproof
{\isafoldproof}%
%
\isadelimproof
\isanewline
%
\endisadelimproof
\isanewline
\isacommand{lemma}\isamarkupfalse%
\ lm{\isadigit{1}}{\isadigit{1}}{\isacharcolon}\ {\isachardoublequoteopen}{\isacharparenleft}{\isacharbraceleft}x{\isacharbraceright}\ {\isasymtimes}\ UNIV{\isacharparenright}\ {\isasyminter}\ P\ {\isacharequal}\ {\isacharbraceleft}x{\isacharbraceright}\ {\isasymtimes}\ {\isacharparenleft}P\ {\isacharbackquote}{\isacharbackquote}\ {\isacharbraceleft}x{\isacharbraceright}{\isacharparenright}{\isachardoublequoteclose}\ \isanewline
%
\isadelimproof
\ \ \ \ \ \ %
\endisadelimproof
%
\isatagproof
\isacommand{by}\isamarkupfalse%
\ fast%
\endisatagproof
{\isafoldproof}%
%
\isadelimproof
\isanewline
%
\endisadelimproof
\isanewline
\isacommand{lemma}\isamarkupfalse%
\ lm{\isadigit{1}}{\isadigit{2}}{\isacharcolon}\ {\isachardoublequoteopen}{\isacharparenleft}x{\isacharcomma}y{\isacharparenright}\ {\isasymin}\ P\ {\isacharequal}\ {\isacharparenleft}y\ {\isasymin}\ P{\isacharbackquote}{\isacharbackquote}{\isacharbraceleft}x{\isacharbraceright}{\isacharparenright}{\isachardoublequoteclose}\ \isanewline
%
\isadelimproof
\ \ \ \ \ \ %
\endisadelimproof
%
\isatagproof
\isacommand{by}\isamarkupfalse%
\ simp%
\endisatagproof
{\isafoldproof}%
%
\isadelimproof
\isanewline
%
\endisadelimproof
\isanewline
\isacommand{lemma}\isamarkupfalse%
\ lm{\isadigit{1}}{\isadigit{3}}{\isacharcolon}\ \isakeyword{assumes}\ {\isachardoublequoteopen}inj{\isacharunderscore}on\ f\ A{\isachardoublequoteclose}\ {\isachardoublequoteopen}inj{\isacharunderscore}on\ f\ B{\isachardoublequoteclose}\ \isanewline
\ \ \ \ \ \ \ \ \ \ \ \ \ \isakeyword{shows}\ \ \ {\isachardoublequoteopen}inj{\isacharunderscore}on\ f\ {\isacharparenleft}A\ {\isasymunion}\ B{\isacharparenright}\ \ {\isacharequal}\ \ {\isacharparenleft}f{\isacharbackquote}{\isacharparenleft}A{\isacharminus}B{\isacharparenright}\ {\isasyminter}\ {\isacharparenleft}f{\isacharbackquote}{\isacharparenleft}B{\isacharminus}A{\isacharparenright}{\isacharparenright}\ {\isacharequal}\ {\isacharbraceleft}{\isacharbraceright}{\isacharparenright}{\isachardoublequoteclose}\isanewline
%
\isadelimproof
\ \ \ \ \ \ %
\endisadelimproof
%
\isatagproof
\isacommand{using}\isamarkupfalse%
\ assms\ inj{\isacharunderscore}on{\isacharunderscore}Un\ \isacommand{by}\isamarkupfalse%
\ {\isacharparenleft}metis{\isacharparenright}%
\endisatagproof
{\isafoldproof}%
%
\isadelimproof
\isanewline
%
\endisadelimproof
\isanewline
\isacommand{lemma}\isamarkupfalse%
\ injection{\isacharunderscore}union{\isacharcolon}\ \isakeyword{assumes}\ {\isachardoublequoteopen}inj{\isacharunderscore}on\ f\ A{\isachardoublequoteclose}\ {\isachardoublequoteopen}inj{\isacharunderscore}on\ f\ B{\isachardoublequoteclose}\ {\isachardoublequoteopen}{\isacharparenleft}f{\isacharbackquote}A{\isacharparenright}\ {\isasyminter}\ {\isacharparenleft}f{\isacharbackquote}B{\isacharparenright}\ {\isacharequal}\ {\isacharbraceleft}{\isacharbraceright}{\isachardoublequoteclose}\ \isanewline
\ \ \ \ \ \ \ \ \ \ \ \ \ \ \isakeyword{shows}\ {\isachardoublequoteopen}inj{\isacharunderscore}on\ f\ {\isacharparenleft}A\ {\isasymunion}\ B{\isacharparenright}{\isachardoublequoteclose}\ \isanewline
%
\isadelimproof
\ \ \ \ \ \ %
\endisadelimproof
%
\isatagproof
\isacommand{using}\isamarkupfalse%
\ assms\ lm{\isadigit{1}}{\isadigit{3}}\ \isacommand{by}\isamarkupfalse%
\ fast%
\endisatagproof
{\isafoldproof}%
%
\isadelimproof
\isanewline
%
\endisadelimproof
\isanewline
\isacommand{lemma}\isamarkupfalse%
\ lm{\isadigit{1}}{\isadigit{4}}{\isacharcolon}\ {\isachardoublequoteopen}{\isacharparenleft}Pow\ X\ {\isacharequal}\ {\isacharbraceleft}X{\isacharbraceright}{\isacharparenright}\ {\isacharequal}\ {\isacharparenleft}X{\isacharequal}{\isacharbraceleft}{\isacharbraceright}{\isacharparenright}{\isachardoublequoteclose}\ \isanewline
%
\isadelimproof
\ \ \ \ \ \ %
\endisadelimproof
%
\isatagproof
\isacommand{by}\isamarkupfalse%
\ auto%
\endisatagproof
{\isafoldproof}%
%
\isadelimproof
\isanewline
%
\endisadelimproof
%
\isadelimtheory
\isanewline
%
\endisadelimtheory
%
\isatagtheory
\isacommand{end}\isamarkupfalse%
%
\endisatagtheory
{\isafoldtheory}%
%
\isadelimtheory
%
\endisadelimtheory
\end{isabellebody}%
%%% Local Variables:
%%% mode: latex
%%% TeX-master: "root"
%%% End:


%
\begin{isabellebody}%
\def\isabellecontext{Partitions}%
%
\isamarkupheader{Partitions of sets%
}
\isamarkuptrue%
%
\isadelimtheory
%
\endisadelimtheory
%
\isatagtheory
\isacommand{theory}\isamarkupfalse%
\ Partitions\isanewline
\isakeyword{imports}\isanewline
\ \ Main\isanewline
\ \ SetUtils\isanewline
\isanewline
\isakeyword{begin}%
\endisatagtheory
{\isafoldtheory}%
%
\isadelimtheory
%
\endisadelimtheory
%
\begin{isamarkuptext}%
We define the set of all partitions of a set (\isa{all{\isacharunderscore}partitions}) in textbook style, as well as a computable function \isa{all{\isacharunderscore}partitions{\isacharunderscore}list} to algorithmically compute this set (then represented as a list).  This function is suitable for code generation.  We prove the equivalence of the two definition in order to ensure that the generated code correctly implements the original textbook-style definition.  For further background on the overall approach, see Caminati, Kerber, Lange, Rowat: \href{http://arxiv.org/abs/1308.1779}{Proving soundness of combinatorial Vickrey auctions and generating verified executable code}, 2013.%
\end{isamarkuptext}%
\isamarkuptrue%
%
\begin{isamarkuptext}%
\isa{P} is a family of non-overlapping  sets.%
\end{isamarkuptext}%
\isamarkuptrue%
\isacommand{definition}\isamarkupfalse%
\ is{\isacharunderscore}non{\isacharunderscore}overlapping\ \isanewline
\ \ \ \ \ \ \ \ \ \ \ \isakeyword{where}\ {\isachardoublequoteopen}is{\isacharunderscore}non{\isacharunderscore}overlapping\ P\ {\isacharequal}\ {\isacharparenleft}{\isasymforall}\ X{\isasymin}P\ {\isachardot}\ {\isasymforall}\ Y{\isasymin}\ P\ {\isachardot}\ {\isacharparenleft}X\ {\isasyminter}\ Y\ {\isasymnoteq}\ {\isacharbraceleft}{\isacharbraceright}\ {\isasymlongleftrightarrow}\ X\ {\isacharequal}\ Y{\isacharparenright}{\isacharparenright}{\isachardoublequoteclose}%
\begin{isamarkuptext}%
A subfamily of a non-overlapping family is also a non-overlapping family%
\end{isamarkuptext}%
\isamarkuptrue%
\isacommand{lemma}\isamarkupfalse%
\ subset{\isacharunderscore}is{\isacharunderscore}non{\isacharunderscore}overlapping{\isacharcolon}\isanewline
\ \ \isakeyword{assumes}\ subset{\isacharcolon}\ {\isachardoublequoteopen}P\ {\isasymsubseteq}\ Q{\isachardoublequoteclose}\ \isakeyword{and}\ \isanewline
\ \ \ \ \ \ \ \ \ \ non{\isacharunderscore}overlapping{\isacharcolon}\ {\isachardoublequoteopen}is{\isacharunderscore}non{\isacharunderscore}overlapping\ Q{\isachardoublequoteclose}\isanewline
\ \ \isakeyword{shows}\ {\isachardoublequoteopen}is{\isacharunderscore}non{\isacharunderscore}overlapping\ P{\isachardoublequoteclose}\isanewline
%
\isadelimproof
\isanewline
%
\endisadelimproof
%
\isatagproof
\isacommand{proof}\isamarkupfalse%
\ {\isacharminus}\isanewline
\ \ \isacommand{{\isacharbraceleft}}\isamarkupfalse%
\isanewline
\ \ \ \ \isacommand{fix}\isamarkupfalse%
\ X\ Y\ \isacommand{assume}\isamarkupfalse%
\ {\isachardoublequoteopen}X\ {\isasymin}\ P\ {\isasymand}\ Y\ {\isasymin}\ P{\isachardoublequoteclose}\isanewline
\ \ \ \ \isacommand{then}\isamarkupfalse%
\ \isacommand{have}\isamarkupfalse%
\ {\isachardoublequoteopen}X\ {\isasymin}\ Q\ {\isasymand}\ Y\ {\isasymin}\ Q{\isachardoublequoteclose}\ \isacommand{using}\isamarkupfalse%
\ subset\ \isacommand{by}\isamarkupfalse%
\ fast\isanewline
\ \ \ \ \isacommand{then}\isamarkupfalse%
\ \isacommand{have}\isamarkupfalse%
\ {\isachardoublequoteopen}X\ {\isasyminter}\ Y\ {\isasymnoteq}\ {\isacharbraceleft}{\isacharbraceright}\ {\isasymlongleftrightarrow}\ X\ {\isacharequal}\ Y{\isachardoublequoteclose}\ \isacommand{using}\isamarkupfalse%
\ non{\isacharunderscore}overlapping\ \isacommand{unfolding}\isamarkupfalse%
\ is{\isacharunderscore}non{\isacharunderscore}overlapping{\isacharunderscore}def\ \isacommand{by}\isamarkupfalse%
\ force\isanewline
\ \ \isacommand{{\isacharbraceright}}\isamarkupfalse%
\isanewline
\ \ \isacommand{then}\isamarkupfalse%
\ \isacommand{show}\isamarkupfalse%
\ {\isacharquery}thesis\ \isacommand{unfolding}\isamarkupfalse%
\ is{\isacharunderscore}non{\isacharunderscore}overlapping{\isacharunderscore}def\ \isacommand{by}\isamarkupfalse%
\ force\isanewline
\isacommand{qed}\isamarkupfalse%
%
\endisatagproof
{\isafoldproof}%
%
\isadelimproof
%
\endisadelimproof
%
\begin{isamarkuptext}%
The family that results from removing one element from an equivalence class of a non-overlapping family is not otherwise a member of the family.%
\end{isamarkuptext}%
\isamarkuptrue%
\isacommand{lemma}\isamarkupfalse%
\ remove{\isacharunderscore}from{\isacharunderscore}eq{\isacharunderscore}class{\isacharunderscore}preserves{\isacharunderscore}disjoint{\isacharcolon}\isanewline
\ \ \ \ \ \ \isakeyword{fixes}\ elem{\isacharcolon}{\isacharcolon}{\isacharprime}a\isanewline
\ \ \ \ \ \ \ \ \isakeyword{and}\ X{\isacharcolon}{\isacharcolon}{\isachardoublequoteopen}{\isacharprime}a\ set{\isachardoublequoteclose}\isanewline
\ \ \ \ \ \ \ \ \isakeyword{and}\ P{\isacharcolon}{\isacharcolon}{\isachardoublequoteopen}{\isacharprime}a\ set\ set{\isachardoublequoteclose}\isanewline
\ \ \ \ \ \ \isakeyword{assumes}\ non{\isacharunderscore}overlapping{\isacharcolon}\ {\isachardoublequoteopen}is{\isacharunderscore}non{\isacharunderscore}overlapping\ P{\isachardoublequoteclose}\isanewline
\ \ \ \ \ \ \ \ \isakeyword{and}\ eq{\isacharunderscore}class{\isacharcolon}\ {\isachardoublequoteopen}X\ {\isasymin}\ P{\isachardoublequoteclose}\isanewline
\ \ \ \ \ \ \ \ \isakeyword{and}\ elem{\isacharcolon}\ {\isachardoublequoteopen}elem\ {\isasymin}\ X{\isachardoublequoteclose}\isanewline
\ \ \ \ \ \ \isakeyword{shows}\ {\isachardoublequoteopen}X\ {\isacharminus}\ {\isacharbraceleft}elem{\isacharbraceright}\ {\isasymnotin}\ P{\isachardoublequoteclose}\isanewline
%
\isadelimproof
\ \ %
\endisadelimproof
%
\isatagproof
\isacommand{using}\isamarkupfalse%
\ assms\ Int{\isacharunderscore}Diff\ is{\isacharunderscore}non{\isacharunderscore}overlapping{\isacharunderscore}def\ Diff{\isacharunderscore}disjoint\ Diff{\isacharunderscore}eq{\isacharunderscore}empty{\isacharunderscore}iff\ \isanewline
\ \ \ \ \ \ \ \ Int{\isacharunderscore}absorb{\isadigit{2}}\ insert{\isacharunderscore}Diff{\isacharunderscore}if\ insert{\isacharunderscore}not{\isacharunderscore}empty\ \isacommand{by}\isamarkupfalse%
\ {\isacharparenleft}metis{\isacharparenright}%
\endisatagproof
{\isafoldproof}%
%
\isadelimproof
%
\endisadelimproof
%
\begin{isamarkuptext}%
Inserting into a non-overlapping family \isa{P} a set \isa{X}, which is disjoint with the set 
  partitioned by \isa{P}, yields another non-overlapping family.%
\end{isamarkuptext}%
\isamarkuptrue%
\isacommand{lemma}\isamarkupfalse%
\ non{\isacharunderscore}overlapping{\isacharunderscore}extension{\isadigit{1}}{\isacharcolon}\isanewline
\ \ \isakeyword{fixes}\ P{\isacharcolon}{\isacharcolon}{\isachardoublequoteopen}{\isacharprime}a\ set\ set{\isachardoublequoteclose}\isanewline
\ \ \ \ \isakeyword{and}\ X{\isacharcolon}{\isacharcolon}{\isachardoublequoteopen}{\isacharprime}a\ set{\isachardoublequoteclose}\isanewline
\ \ \isakeyword{assumes}\ partition{\isacharcolon}\ {\isachardoublequoteopen}is{\isacharunderscore}non{\isacharunderscore}overlapping\ P{\isachardoublequoteclose}\isanewline
\ \ \ \ \ \ \ \isakeyword{and}\ disjoint{\isacharcolon}\ {\isachardoublequoteopen}X\ {\isasyminter}\ {\isasymUnion}\ P\ {\isacharequal}\ {\isacharbraceleft}{\isacharbraceright}{\isachardoublequoteclose}\ \isanewline
\ \ \ \ \ \ \isakeyword{and}\ non{\isacharunderscore}empty{\isacharcolon}\ {\isachardoublequoteopen}X\ {\isasymnoteq}\ {\isacharbraceleft}{\isacharbraceright}{\isachardoublequoteclose}\isanewline
\ \ \isakeyword{shows}\ {\isachardoublequoteopen}is{\isacharunderscore}non{\isacharunderscore}overlapping\ {\isacharparenleft}insert\ X\ P{\isacharparenright}{\isachardoublequoteclose}\isanewline
%
\isadelimproof
%
\endisadelimproof
%
\isatagproof
\isacommand{proof}\isamarkupfalse%
\ {\isacharminus}\isanewline
\ \ \isacommand{{\isacharbraceleft}}\isamarkupfalse%
\isanewline
\ \ \ \ \isacommand{fix}\isamarkupfalse%
\ Y{\isacharcolon}{\isacharcolon}{\isachardoublequoteopen}{\isacharprime}a\ set{\isachardoublequoteclose}\ \isakeyword{and}\ Z{\isacharcolon}{\isacharcolon}{\isachardoublequoteopen}{\isacharprime}a\ set{\isachardoublequoteclose}\isanewline
\ \ \ \ \isacommand{assume}\isamarkupfalse%
\ Y{\isacharunderscore}Z{\isacharunderscore}in{\isacharunderscore}ext{\isacharunderscore}P{\isacharcolon}\ {\isachardoublequoteopen}Y\ {\isasymin}\ insert\ X\ P\ {\isasymand}\ Z\ {\isasymin}\ insert\ X\ P{\isachardoublequoteclose}\isanewline
\ \ \ \ \isacommand{have}\isamarkupfalse%
\ {\isachardoublequoteopen}Y\ {\isasyminter}\ Z\ {\isasymnoteq}\ {\isacharbraceleft}{\isacharbraceright}\ {\isasymlongleftrightarrow}\ Y\ {\isacharequal}\ Z{\isachardoublequoteclose}\isanewline
\ \ \ \ \isacommand{proof}\isamarkupfalse%
\isanewline
\ \ \ \ \ \ \isacommand{assume}\isamarkupfalse%
\ {\isachardoublequoteopen}Y\ {\isasyminter}\ Z\ {\isasymnoteq}\ {\isacharbraceleft}{\isacharbraceright}{\isachardoublequoteclose}\isanewline
\ \ \ \ \ \ \isacommand{then}\isamarkupfalse%
\ \isacommand{show}\isamarkupfalse%
\ {\isachardoublequoteopen}Y\ {\isacharequal}\ Z{\isachardoublequoteclose}\isanewline
\ \ \ \ \ \ \ \ \isacommand{using}\isamarkupfalse%
\ Y{\isacharunderscore}Z{\isacharunderscore}in{\isacharunderscore}ext{\isacharunderscore}P\ partition\ disjoint\isanewline
\ \ \ \ \ \ \ \ \isacommand{unfolding}\isamarkupfalse%
\ is{\isacharunderscore}non{\isacharunderscore}overlapping{\isacharunderscore}def\isanewline
\ \ \ \ \ \ \ \ \isacommand{by}\isamarkupfalse%
\ fast\isanewline
\ \ \ \ \isacommand{next}\isamarkupfalse%
\isanewline
\ \ \ \ \ \ \isacommand{assume}\isamarkupfalse%
\ {\isachardoublequoteopen}Y\ {\isacharequal}\ Z{\isachardoublequoteclose}\isanewline
\ \ \ \ \ \ \isacommand{then}\isamarkupfalse%
\ \isacommand{show}\isamarkupfalse%
\ {\isachardoublequoteopen}Y\ {\isasyminter}\ Z\ {\isasymnoteq}\ {\isacharbraceleft}{\isacharbraceright}{\isachardoublequoteclose}\isanewline
\ \ \ \ \ \ \ \ \isacommand{using}\isamarkupfalse%
\ Y{\isacharunderscore}Z{\isacharunderscore}in{\isacharunderscore}ext{\isacharunderscore}P\ partition\ non{\isacharunderscore}empty\isanewline
\ \ \ \ \ \ \ \ \isacommand{unfolding}\isamarkupfalse%
\ is{\isacharunderscore}non{\isacharunderscore}overlapping{\isacharunderscore}def\isanewline
\ \ \ \ \ \ \ \ \isacommand{by}\isamarkupfalse%
\ auto\isanewline
\ \ \ \ \isacommand{qed}\isamarkupfalse%
\isanewline
\ \ \isacommand{{\isacharbraceright}}\isamarkupfalse%
\isanewline
\ \ \isacommand{then}\isamarkupfalse%
\ \isacommand{show}\isamarkupfalse%
\ {\isacharquery}thesis\ \isacommand{unfolding}\isamarkupfalse%
\ is{\isacharunderscore}non{\isacharunderscore}overlapping{\isacharunderscore}def\ \isacommand{by}\isamarkupfalse%
\ force\isanewline
\isacommand{qed}\isamarkupfalse%
%
\endisatagproof
{\isafoldproof}%
%
\isadelimproof
%
\endisadelimproof
%
\begin{isamarkuptext}%
An element of a non-overlapping family has no intersection with any other of its elements.%
\end{isamarkuptext}%
\isamarkuptrue%
\isacommand{lemma}\isamarkupfalse%
\ disj{\isacharunderscore}eq{\isacharunderscore}classes{\isacharcolon}\isanewline
\ \ \isakeyword{fixes}\ P{\isacharcolon}{\isacharcolon}{\isachardoublequoteopen}{\isacharprime}a\ set\ set{\isachardoublequoteclose}\isanewline
\ \ \ \ \isakeyword{and}\ X{\isacharcolon}{\isacharcolon}{\isachardoublequoteopen}{\isacharprime}a\ set{\isachardoublequoteclose}\isanewline
\ \ \isakeyword{assumes}\ {\isachardoublequoteopen}is{\isacharunderscore}non{\isacharunderscore}overlapping\ P{\isachardoublequoteclose}\isanewline
\ \ \ \ \ \ \isakeyword{and}\ {\isachardoublequoteopen}X\ {\isasymin}\ P{\isachardoublequoteclose}\isanewline
\ \ \isakeyword{shows}\ {\isachardoublequoteopen}X\ {\isasyminter}\ {\isasymUnion}\ {\isacharparenleft}P\ {\isacharminus}\ {\isacharbraceleft}X{\isacharbraceright}{\isacharparenright}\ {\isacharequal}\ {\isacharbraceleft}{\isacharbraceright}{\isachardoublequoteclose}\ \isanewline
%
\isadelimproof
%
\endisadelimproof
%
\isatagproof
\isacommand{proof}\isamarkupfalse%
\ {\isacharminus}\isanewline
\ \ \isacommand{{\isacharbraceleft}}\isamarkupfalse%
\isanewline
\ \ \ \ \isacommand{fix}\isamarkupfalse%
\ x{\isacharcolon}{\isacharcolon}{\isacharprime}a\isanewline
\ \ \ \ \isacommand{assume}\isamarkupfalse%
\ x{\isacharunderscore}in{\isacharunderscore}two{\isacharunderscore}eq{\isacharunderscore}classes{\isacharcolon}\ {\isachardoublequoteopen}x\ {\isasymin}\ X\ {\isasyminter}\ {\isasymUnion}\ {\isacharparenleft}P\ {\isacharminus}\ {\isacharbraceleft}X{\isacharbraceright}{\isacharparenright}{\isachardoublequoteclose}\isanewline
\ \ \ \ \isacommand{then}\isamarkupfalse%
\ \isacommand{obtain}\isamarkupfalse%
\ Y\ \isakeyword{where}\ other{\isacharunderscore}eq{\isacharunderscore}class{\isacharcolon}\ {\isachardoublequoteopen}Y\ {\isasymin}\ P\ {\isacharminus}\ {\isacharbraceleft}X{\isacharbraceright}\ {\isasymand}\ x\ {\isasymin}\ Y{\isachardoublequoteclose}\ \isacommand{by}\isamarkupfalse%
\ blast\isanewline
\ \ \ \ \isacommand{have}\isamarkupfalse%
\ {\isachardoublequoteopen}x\ {\isasymin}\ X\ {\isasyminter}\ Y\ {\isasymand}\ Y\ {\isasymin}\ P{\isachardoublequoteclose}\isanewline
\ \ \ \ \ \ \isacommand{using}\isamarkupfalse%
\ x{\isacharunderscore}in{\isacharunderscore}two{\isacharunderscore}eq{\isacharunderscore}classes\ other{\isacharunderscore}eq{\isacharunderscore}class\ \isacommand{by}\isamarkupfalse%
\ force\isanewline
\ \ \ \ \isacommand{then}\isamarkupfalse%
\ \isacommand{have}\isamarkupfalse%
\ {\isachardoublequoteopen}X\ {\isacharequal}\ Y{\isachardoublequoteclose}\ \isacommand{using}\isamarkupfalse%
\ assms\ is{\isacharunderscore}non{\isacharunderscore}overlapping{\isacharunderscore}def\ \isacommand{by}\isamarkupfalse%
\ fast\isanewline
\ \ \ \ \isacommand{then}\isamarkupfalse%
\ \isacommand{have}\isamarkupfalse%
\ {\isachardoublequoteopen}x\ {\isasymin}\ {\isacharbraceleft}{\isacharbraceright}{\isachardoublequoteclose}\ \isacommand{using}\isamarkupfalse%
\ other{\isacharunderscore}eq{\isacharunderscore}class\ \isacommand{by}\isamarkupfalse%
\ fast\isanewline
\ \ \isacommand{{\isacharbraceright}}\isamarkupfalse%
\isanewline
\ \ \isacommand{then}\isamarkupfalse%
\ \isacommand{show}\isamarkupfalse%
\ {\isacharquery}thesis\ \isacommand{by}\isamarkupfalse%
\ blast\isanewline
\isacommand{qed}\isamarkupfalse%
%
\endisatagproof
{\isafoldproof}%
%
\isadelimproof
%
\endisadelimproof
%
\begin{isamarkuptext}%
The empty set is not element of a non-overlapping family.%
\end{isamarkuptext}%
\isamarkuptrue%
\isacommand{lemma}\isamarkupfalse%
\ no{\isacharunderscore}empty{\isacharunderscore}in{\isacharunderscore}non{\isacharunderscore}overlapping{\isacharcolon}\isanewline
\ \ \isakeyword{assumes}\ {\isachardoublequoteopen}is{\isacharunderscore}non{\isacharunderscore}overlapping\ p{\isachardoublequoteclose}\isanewline
\ \ \isakeyword{shows}\ {\isachardoublequoteopen}{\isacharbraceleft}{\isacharbraceright}\ {\isasymnotin}\ p{\isachardoublequoteclose}\isanewline
%
\isadelimproof
\isanewline
\ \ %
\endisadelimproof
%
\isatagproof
\isacommand{using}\isamarkupfalse%
\ assms\ is{\isacharunderscore}non{\isacharunderscore}overlapping{\isacharunderscore}def\ \isacommand{by}\isamarkupfalse%
\ fast%
\endisatagproof
{\isafoldproof}%
%
\isadelimproof
%
\endisadelimproof
%
\begin{isamarkuptext}%
\isa{P} is a partition of the set \isa{A}. The infix notation takes the form ``noun-verb-object''%
\end{isamarkuptext}%
\isamarkuptrue%
\isacommand{definition}\isamarkupfalse%
\ is{\isacharunderscore}partition{\isacharunderscore}of\ {\isacharparenleft}\isakeyword{infix}\ {\isachardoublequoteopen}partitions{\isachardoublequoteclose}\ {\isadigit{7}}{\isadigit{5}}{\isacharparenright}\isanewline
\ \ \ \ \ \ \ \ \ \ \ \isakeyword{where}\ {\isachardoublequoteopen}is{\isacharunderscore}partition{\isacharunderscore}of\ P\ A\ {\isacharequal}\ {\isacharparenleft}{\isasymUnion}\ P\ {\isacharequal}\ A\ {\isasymand}\ is{\isacharunderscore}non{\isacharunderscore}overlapping\ P{\isacharparenright}{\isachardoublequoteclose}%
\begin{isamarkuptext}%
No partition of a non-empty set is empty.%
\end{isamarkuptext}%
\isamarkuptrue%
\isacommand{lemma}\isamarkupfalse%
\ non{\isacharunderscore}empty{\isacharunderscore}imp{\isacharunderscore}non{\isacharunderscore}empty{\isacharunderscore}partition{\isacharcolon}\isanewline
\ \ \isakeyword{assumes}\ {\isachardoublequoteopen}A\ {\isasymnoteq}\ {\isacharbraceleft}{\isacharbraceright}{\isachardoublequoteclose}\isanewline
\ \ \ \ \ \ \isakeyword{and}\ {\isachardoublequoteopen}P\ partitions\ A{\isachardoublequoteclose}\isanewline
\ \ \isakeyword{shows}\ {\isachardoublequoteopen}P\ {\isasymnoteq}\ {\isacharbraceleft}{\isacharbraceright}{\isachardoublequoteclose}\isanewline
%
\isadelimproof
\ \ %
\endisadelimproof
%
\isatagproof
\isacommand{using}\isamarkupfalse%
\ assms\ \isacommand{unfolding}\isamarkupfalse%
\ is{\isacharunderscore}partition{\isacharunderscore}of{\isacharunderscore}def\ \isacommand{by}\isamarkupfalse%
\ fast%
\endisatagproof
{\isafoldproof}%
%
\isadelimproof
%
\endisadelimproof
%
\begin{isamarkuptext}%
Every element of a partitioned set ends up in one element in the partition.%
\end{isamarkuptext}%
\isamarkuptrue%
\isacommand{lemma}\isamarkupfalse%
\ elem{\isacharunderscore}in{\isacharunderscore}partition{\isacharcolon}\isanewline
\ \ \isakeyword{assumes}\ in{\isacharunderscore}set{\isacharcolon}\ {\isachardoublequoteopen}x\ {\isasymin}\ A{\isachardoublequoteclose}\isanewline
\ \ \ \ \ \ \isakeyword{and}\ part{\isacharcolon}\ {\isachardoublequoteopen}P\ partitions\ A{\isachardoublequoteclose}\isanewline
\ \ \isakeyword{obtains}\ X\ \isakeyword{where}\ {\isachardoublequoteopen}x\ {\isasymin}\ X{\isachardoublequoteclose}\ \isakeyword{and}\ {\isachardoublequoteopen}X\ {\isasymin}\ P{\isachardoublequoteclose}\isanewline
%
\isadelimproof
\ \ %
\endisadelimproof
%
\isatagproof
\isacommand{using}\isamarkupfalse%
\ part\ in{\isacharunderscore}set\ \isacommand{unfolding}\isamarkupfalse%
\ is{\isacharunderscore}partition{\isacharunderscore}of{\isacharunderscore}def\ is{\isacharunderscore}non{\isacharunderscore}overlapping{\isacharunderscore}def\ \isacommand{by}\isamarkupfalse%
\ {\isacharparenleft}auto\ simp\ add{\isacharcolon}\ UnionE{\isacharparenright}%
\endisatagproof
{\isafoldproof}%
%
\isadelimproof
%
\endisadelimproof
%
\begin{isamarkuptext}%
Every element of the difference of a set \isa{A} and another set \isa{B} ends up in 
  an element of a partition of \isa{A}, but not in an element of the partition of \isa{{\isacharbraceleft}B{\isacharbraceright}}.%
\end{isamarkuptext}%
\isamarkuptrue%
\isacommand{lemma}\isamarkupfalse%
\ diff{\isacharunderscore}elem{\isacharunderscore}in{\isacharunderscore}partition{\isacharcolon}\isanewline
\ \ \isakeyword{assumes}\ x{\isacharcolon}\ {\isachardoublequoteopen}x\ {\isasymin}\ A\ {\isacharminus}\ B{\isachardoublequoteclose}\isanewline
\ \ \ \ \ \ \isakeyword{and}\ part{\isacharcolon}\ {\isachardoublequoteopen}P\ partitions\ A{\isachardoublequoteclose}\isanewline
\ \ \isakeyword{shows}\ {\isachardoublequoteopen}{\isasymexists}\ S\ {\isasymin}\ P\ {\isacharminus}\ {\isacharbraceleft}\ B\ {\isacharbraceright}\ {\isachardot}\ x\ {\isasymin}\ S{\isachardoublequoteclose}\isanewline
%
\isadelimproof
\isanewline
%
\endisadelimproof
%
\isatagproof
\isacommand{proof}\isamarkupfalse%
\ {\isacharminus}\isanewline
\ \ \isacommand{from}\isamarkupfalse%
\ part\ x\ \isacommand{obtain}\isamarkupfalse%
\ X\ \isakeyword{where}\ {\isachardoublequoteopen}x\ {\isasymin}\ X{\isachardoublequoteclose}\ \isakeyword{and}\ {\isachardoublequoteopen}X\ {\isasymin}\ P{\isachardoublequoteclose}\isanewline
\ \ \ \ \isacommand{by}\isamarkupfalse%
\ {\isacharparenleft}metis\ Diff{\isacharunderscore}iff\ elem{\isacharunderscore}in{\isacharunderscore}partition{\isacharparenright}\isanewline
\ \ \isacommand{with}\isamarkupfalse%
\ x\ \isacommand{have}\isamarkupfalse%
\ {\isachardoublequoteopen}X\ {\isasymnoteq}\ B{\isachardoublequoteclose}\ \isacommand{by}\isamarkupfalse%
\ fast\isanewline
\ \ \isacommand{with}\isamarkupfalse%
\ {\isacharbackquoteopen}x\ {\isasymin}\ X{\isacharbackquoteclose}\ {\isacharbackquoteopen}X\ {\isasymin}\ P{\isacharbackquoteclose}\ \isacommand{show}\isamarkupfalse%
\ {\isacharquery}thesis\ \isacommand{by}\isamarkupfalse%
\ blast\isanewline
\isacommand{qed}\isamarkupfalse%
%
\endisatagproof
{\isafoldproof}%
%
\isadelimproof
%
\endisadelimproof
%
\begin{isamarkuptext}%
Every element of a partitioned set ends up in exactly one set.%
\end{isamarkuptext}%
\isamarkuptrue%
\isacommand{lemma}\isamarkupfalse%
\ elem{\isacharunderscore}in{\isacharunderscore}uniq{\isacharunderscore}set{\isacharcolon}\isanewline
\ \ \isakeyword{assumes}\ in{\isacharunderscore}set{\isacharcolon}\ {\isachardoublequoteopen}x\ {\isasymin}\ A{\isachardoublequoteclose}\isanewline
\ \ \ \ \ \ \isakeyword{and}\ part{\isacharcolon}\ {\isachardoublequoteopen}P\ partitions\ A{\isachardoublequoteclose}\isanewline
\ \ \isakeyword{shows}\ {\isachardoublequoteopen}{\isasymexists}{\isacharbang}\ X\ {\isasymin}\ P\ {\isachardot}\ x\ {\isasymin}\ X{\isachardoublequoteclose}\isanewline
%
\isadelimproof
%
\endisadelimproof
%
\isatagproof
\isacommand{proof}\isamarkupfalse%
\ {\isacharminus}\isanewline
\ \ \isacommand{from}\isamarkupfalse%
\ assms\ \isacommand{obtain}\isamarkupfalse%
\ X\ \isakeyword{where}\ {\isacharasterisk}{\isacharcolon}\ {\isachardoublequoteopen}X\ {\isasymin}\ P\ {\isasymand}\ x\ {\isasymin}\ X{\isachardoublequoteclose}\isanewline
\ \ \ \ \isacommand{by}\isamarkupfalse%
\ {\isacharparenleft}rule\ elem{\isacharunderscore}in{\isacharunderscore}partition{\isacharparenright}\ blast\isanewline
\ \ \isacommand{moreover}\isamarkupfalse%
\ \isacommand{{\isacharbraceleft}}\isamarkupfalse%
\isanewline
\ \ \ \ \isacommand{fix}\isamarkupfalse%
\ Y\ \isacommand{assume}\isamarkupfalse%
\ {\isachardoublequoteopen}Y\ {\isasymin}\ P\ {\isasymand}\ x\ {\isasymin}\ Y{\isachardoublequoteclose}\isanewline
\ \ \ \ \isacommand{then}\isamarkupfalse%
\ \isacommand{have}\isamarkupfalse%
\ {\isachardoublequoteopen}Y\ {\isacharequal}\ X{\isachardoublequoteclose}\isanewline
\ \ \ \ \ \ \isacommand{using}\isamarkupfalse%
\ part\ in{\isacharunderscore}set\ {\isacharasterisk}\isanewline
\ \ \ \ \ \ \isacommand{unfolding}\isamarkupfalse%
\ is{\isacharunderscore}partition{\isacharunderscore}of{\isacharunderscore}def\ is{\isacharunderscore}non{\isacharunderscore}overlapping{\isacharunderscore}def\isanewline
\ \ \ \ \ \ \isacommand{by}\isamarkupfalse%
\ {\isacharparenleft}metis\ disjoint{\isacharunderscore}iff{\isacharunderscore}not{\isacharunderscore}equal{\isacharparenright}\isanewline
\ \ \isacommand{{\isacharbraceright}}\isamarkupfalse%
\isanewline
\ \ \isacommand{ultimately}\isamarkupfalse%
\ \isacommand{show}\isamarkupfalse%
\ {\isacharquery}thesis\ \isacommand{by}\isamarkupfalse%
\ {\isacharparenleft}rule\ ex{\isadigit{1}}I{\isacharparenright}\isanewline
\isacommand{qed}\isamarkupfalse%
%
\endisatagproof
{\isafoldproof}%
%
\isadelimproof
%
\endisadelimproof
%
\begin{isamarkuptext}%
A non-empty set ``is'' a partition of itself.%
\end{isamarkuptext}%
\isamarkuptrue%
\isacommand{lemma}\isamarkupfalse%
\ set{\isacharunderscore}partitions{\isacharunderscore}itself{\isacharcolon}\isanewline
\ \ \isakeyword{assumes}\ {\isachardoublequoteopen}A\ {\isasymnoteq}\ {\isacharbraceleft}{\isacharbraceright}{\isachardoublequoteclose}\isanewline
\ \ \isakeyword{shows}\ {\isachardoublequoteopen}{\isacharbraceleft}A{\isacharbraceright}\ partitions\ A{\isachardoublequoteclose}%
\isadelimproof
\ %
\endisadelimproof
%
\isatagproof
\isacommand{unfolding}\isamarkupfalse%
\ is{\isacharunderscore}partition{\isacharunderscore}of{\isacharunderscore}def\ is{\isacharunderscore}non{\isacharunderscore}overlapping{\isacharunderscore}def\isanewline
\isanewline
\isacommand{proof}\isamarkupfalse%
\isanewline
\ \ \isacommand{show}\isamarkupfalse%
\ {\isachardoublequoteopen}{\isasymUnion}\ {\isacharbraceleft}A{\isacharbraceright}\ {\isacharequal}\ A{\isachardoublequoteclose}\ \isacommand{by}\isamarkupfalse%
\ simp\isanewline
\ \ \isacommand{{\isacharbraceleft}}\isamarkupfalse%
\isanewline
\ \ \ \ \isacommand{fix}\isamarkupfalse%
\ X\ Y\isanewline
\ \ \ \ \isacommand{assume}\isamarkupfalse%
\ {\isachardoublequoteopen}X\ {\isasymin}\ {\isacharbraceleft}A{\isacharbraceright}{\isachardoublequoteclose}\isanewline
\ \ \ \ \isacommand{then}\isamarkupfalse%
\ \isacommand{have}\isamarkupfalse%
\ {\isachardoublequoteopen}X\ {\isacharequal}\ A{\isachardoublequoteclose}\ \isacommand{by}\isamarkupfalse%
\ {\isacharparenleft}rule\ singletonD{\isacharparenright}\isanewline
\ \ \ \ \isacommand{assume}\isamarkupfalse%
\ {\isachardoublequoteopen}Y\ {\isasymin}\ {\isacharbraceleft}A{\isacharbraceright}{\isachardoublequoteclose}\isanewline
\ \ \ \ \isacommand{then}\isamarkupfalse%
\ \isacommand{have}\isamarkupfalse%
\ {\isachardoublequoteopen}Y\ {\isacharequal}\ A{\isachardoublequoteclose}\ \isacommand{by}\isamarkupfalse%
\ {\isacharparenleft}rule\ singletonD{\isacharparenright}\isanewline
\ \ \ \ \isacommand{from}\isamarkupfalse%
\ {\isacharbackquoteopen}X\ {\isacharequal}\ A{\isacharbackquoteclose}\ {\isacharbackquoteopen}Y\ {\isacharequal}\ A{\isacharbackquoteclose}\ \isacommand{have}\isamarkupfalse%
\ {\isachardoublequoteopen}X\ {\isasyminter}\ Y\ {\isasymnoteq}\ {\isacharbraceleft}{\isacharbraceright}\ {\isasymlongleftrightarrow}\ X\ {\isacharequal}\ Y{\isachardoublequoteclose}\ \isacommand{using}\isamarkupfalse%
\ assms\ \isacommand{by}\isamarkupfalse%
\ simp\isanewline
\ \ \isacommand{{\isacharbraceright}}\isamarkupfalse%
\isanewline
\ \ \isacommand{then}\isamarkupfalse%
\ \isacommand{show}\isamarkupfalse%
\ {\isachardoublequoteopen}{\isasymforall}\ X\ {\isasymin}\ {\isacharbraceleft}A{\isacharbraceright}\ {\isachardot}\ {\isasymforall}\ Y\ {\isasymin}\ {\isacharbraceleft}A{\isacharbraceright}\ {\isachardot}\ X\ {\isasyminter}\ Y\ {\isasymnoteq}\ {\isacharbraceleft}{\isacharbraceright}\ {\isasymlongleftrightarrow}\ X\ {\isacharequal}\ Y{\isachardoublequoteclose}\ \isacommand{by}\isamarkupfalse%
\ force\isanewline
\isacommand{qed}\isamarkupfalse%
%
\endisatagproof
{\isafoldproof}%
%
\isadelimproof
%
\endisadelimproof
%
\begin{isamarkuptext}%
The empty set is a partition of the empty set.%
\end{isamarkuptext}%
\isamarkuptrue%
\isacommand{lemma}\isamarkupfalse%
\ emptyset{\isacharunderscore}part{\isacharunderscore}emptyset{\isadigit{1}}{\isacharcolon}\isanewline
\ \ \isakeyword{shows}\ {\isachardoublequoteopen}{\isacharbraceleft}{\isacharbraceright}\ partitions\ {\isacharbraceleft}{\isacharbraceright}{\isachardoublequoteclose}\ \isanewline
%
\isadelimproof
\ \ %
\endisadelimproof
%
\isatagproof
\isacommand{unfolding}\isamarkupfalse%
\ is{\isacharunderscore}partition{\isacharunderscore}of{\isacharunderscore}def\ is{\isacharunderscore}non{\isacharunderscore}overlapping{\isacharunderscore}def\ \isacommand{by}\isamarkupfalse%
\ fast%
\endisatagproof
{\isafoldproof}%
%
\isadelimproof
%
\endisadelimproof
%
\begin{isamarkuptext}%
Any partition of the empty set is empty.%
\end{isamarkuptext}%
\isamarkuptrue%
\isacommand{lemma}\isamarkupfalse%
\ emptyset{\isacharunderscore}part{\isacharunderscore}emptyset{\isadigit{2}}{\isacharcolon}\isanewline
\ \ \isakeyword{assumes}\ {\isachardoublequoteopen}P\ partitions\ {\isacharbraceleft}{\isacharbraceright}{\isachardoublequoteclose}\isanewline
\ \ \isakeyword{shows}\ {\isachardoublequoteopen}P\ {\isacharequal}\ {\isacharbraceleft}{\isacharbraceright}{\isachardoublequoteclose}\isanewline
%
\isadelimproof
\ \ %
\endisadelimproof
%
\isatagproof
\isacommand{using}\isamarkupfalse%
\ assms\ is{\isacharunderscore}non{\isacharunderscore}overlapping{\isacharunderscore}def\ is{\isacharunderscore}partition{\isacharunderscore}of{\isacharunderscore}def\ \isacommand{by}\isamarkupfalse%
\ fast%
\endisatagproof
{\isafoldproof}%
%
\isadelimproof
%
\endisadelimproof
%
\begin{isamarkuptext}%
Classical set-theoretical definition of ``all partitions of a set \isa{A}''%
\end{isamarkuptext}%
\isamarkuptrue%
\isacommand{definition}\isamarkupfalse%
\ all{\isacharunderscore}partitions\ \isakeyword{where}\ \isanewline
{\isachardoublequoteopen}all{\isacharunderscore}partitions\ A\ {\isacharequal}\ {\isacharbraceleft}P\ {\isachardot}\ P\ partitions\ A{\isacharbraceright}{\isachardoublequoteclose}%
\begin{isamarkuptext}%
The set of all partitions of the empty set only contains the empty set.
  We need this to prove the base case of \isa{all{\isacharunderscore}partitions{\isacharunderscore}paper{\isacharunderscore}equiv{\isacharunderscore}alg}.%
\end{isamarkuptext}%
\isamarkuptrue%
\isacommand{lemma}\isamarkupfalse%
\ emptyset{\isacharunderscore}part{\isacharunderscore}emptyset{\isadigit{3}}{\isacharcolon}\isanewline
\ \ \isakeyword{shows}\ {\isachardoublequoteopen}all{\isacharunderscore}partitions\ {\isacharbraceleft}{\isacharbraceright}\ {\isacharequal}\ {\isacharbraceleft}{\isacharbraceleft}{\isacharbraceright}{\isacharbraceright}{\isachardoublequoteclose}\isanewline
%
\isadelimproof
\ \ %
\endisadelimproof
%
\isatagproof
\isacommand{unfolding}\isamarkupfalse%
\ all{\isacharunderscore}partitions{\isacharunderscore}def\ \isacommand{using}\isamarkupfalse%
\ emptyset{\isacharunderscore}part{\isacharunderscore}emptyset{\isadigit{1}}\ emptyset{\isacharunderscore}part{\isacharunderscore}emptyset{\isadigit{2}}\ \isacommand{by}\isamarkupfalse%
\ fast%
\endisatagproof
{\isafoldproof}%
%
\isadelimproof
%
\endisadelimproof
%
\begin{isamarkuptext}%
inserts an element new_el into a specified set S inside a given family of sets%
\end{isamarkuptext}%
\isamarkuptrue%
\isacommand{definition}\isamarkupfalse%
\ insert{\isacharunderscore}into{\isacharunderscore}member\ {\isacharcolon}{\isacharcolon}\ {\isachardoublequoteopen}{\isacharprime}a\ {\isasymRightarrow}\ {\isacharprime}a\ set\ set\ {\isasymRightarrow}\ {\isacharprime}a\ set\ {\isasymRightarrow}\ {\isacharprime}a\ set\ set{\isachardoublequoteclose}\isanewline
\ \ \ \isakeyword{where}\ {\isachardoublequoteopen}insert{\isacharunderscore}into{\isacharunderscore}member\ new{\isacharunderscore}el\ Sets\ S\ {\isacharequal}\ insert\ {\isacharparenleft}S\ {\isasymunion}\ {\isacharbraceleft}new{\isacharunderscore}el{\isacharbraceright}{\isacharparenright}\ {\isacharparenleft}Sets\ {\isacharminus}\ {\isacharbraceleft}S{\isacharbraceright}{\isacharparenright}{\isachardoublequoteclose}%
\begin{isamarkuptext}%
Using \isa{insert{\isacharunderscore}into{\isacharunderscore}member} to insert a fresh element, which is not a member of the
  set \isa{S} being partitioned, into a non-overlapping family of sets yields another
  non-overlapping family.%
\end{isamarkuptext}%
\isamarkuptrue%
\isacommand{lemma}\isamarkupfalse%
\ non{\isacharunderscore}overlapping{\isacharunderscore}extension{\isadigit{2}}{\isacharcolon}\isanewline
\ \ \isakeyword{fixes}\ new{\isacharunderscore}el{\isacharcolon}{\isacharcolon}{\isacharprime}a\isanewline
\ \ \ \ \isakeyword{and}\ P{\isacharcolon}{\isacharcolon}{\isachardoublequoteopen}{\isacharprime}a\ set\ set{\isachardoublequoteclose}\isanewline
\ \ \ \ \isakeyword{and}\ X{\isacharcolon}{\isacharcolon}{\isachardoublequoteopen}{\isacharprime}a\ set{\isachardoublequoteclose}\isanewline
\ \ \isakeyword{assumes}\ non{\isacharunderscore}overlapping{\isacharcolon}\ {\isachardoublequoteopen}is{\isacharunderscore}non{\isacharunderscore}overlapping\ P{\isachardoublequoteclose}\isanewline
\ \ \ \ \ \ \isakeyword{and}\ class{\isacharunderscore}element{\isacharcolon}\ {\isachardoublequoteopen}X\ {\isasymin}\ P{\isachardoublequoteclose}\isanewline
\ \ \ \ \ \ \isakeyword{and}\ new{\isacharcolon}\ {\isachardoublequoteopen}new{\isacharunderscore}el\ {\isasymnotin}\ {\isasymUnion}\ P{\isachardoublequoteclose}\isanewline
\ \ \isakeyword{shows}\ {\isachardoublequoteopen}is{\isacharunderscore}non{\isacharunderscore}overlapping\ {\isacharparenleft}insert{\isacharunderscore}into{\isacharunderscore}member\ new{\isacharunderscore}el\ P\ X{\isacharparenright}{\isachardoublequoteclose}\isanewline
%
\isadelimproof
%
\endisadelimproof
%
\isatagproof
\isacommand{proof}\isamarkupfalse%
\ {\isacharminus}\isanewline
\ \ \isacommand{let}\isamarkupfalse%
\ {\isacharquery}Y\ {\isacharequal}\ {\isachardoublequoteopen}insert\ new{\isacharunderscore}el\ X{\isachardoublequoteclose}\isanewline
\ \ \isacommand{have}\isamarkupfalse%
\ rest{\isacharunderscore}is{\isacharunderscore}non{\isacharunderscore}overlapping{\isacharcolon}\ {\isachardoublequoteopen}is{\isacharunderscore}non{\isacharunderscore}overlapping\ {\isacharparenleft}P\ {\isacharminus}\ {\isacharbraceleft}X{\isacharbraceright}{\isacharparenright}{\isachardoublequoteclose}\isanewline
\ \ \ \ \isacommand{using}\isamarkupfalse%
\ non{\isacharunderscore}overlapping\ subset{\isacharunderscore}is{\isacharunderscore}non{\isacharunderscore}overlapping\ \isacommand{by}\isamarkupfalse%
\ blast\isanewline
\ \ \isacommand{have}\isamarkupfalse%
\ {\isacharasterisk}{\isacharcolon}\ {\isachardoublequoteopen}X\ {\isasyminter}\ {\isasymUnion}\ {\isacharparenleft}P\ {\isacharminus}\ {\isacharbraceleft}X{\isacharbraceright}{\isacharparenright}\ {\isacharequal}\ {\isacharbraceleft}{\isacharbraceright}{\isachardoublequoteclose}\isanewline
\ \ \ \isacommand{using}\isamarkupfalse%
\ non{\isacharunderscore}overlapping\ class{\isacharunderscore}element\ \isacommand{by}\isamarkupfalse%
\ {\isacharparenleft}rule\ disj{\isacharunderscore}eq{\isacharunderscore}classes{\isacharparenright}\isanewline
\ \ \isacommand{from}\isamarkupfalse%
\ {\isacharasterisk}\ \isacommand{have}\isamarkupfalse%
\ non{\isacharunderscore}empty{\isacharcolon}\ {\isachardoublequoteopen}{\isacharquery}Y\ {\isasymnoteq}\ {\isacharbraceleft}{\isacharbraceright}{\isachardoublequoteclose}\ \isacommand{by}\isamarkupfalse%
\ blast\isanewline
\ \ \isacommand{from}\isamarkupfalse%
\ {\isacharasterisk}\ \isacommand{have}\isamarkupfalse%
\ disjoint{\isacharcolon}\ {\isachardoublequoteopen}{\isacharquery}Y\ {\isasyminter}\ {\isasymUnion}\ {\isacharparenleft}P\ {\isacharminus}\ {\isacharbraceleft}X{\isacharbraceright}{\isacharparenright}\ {\isacharequal}\ {\isacharbraceleft}{\isacharbraceright}{\isachardoublequoteclose}\ \isacommand{using}\isamarkupfalse%
\ new\ \isacommand{by}\isamarkupfalse%
\ force\isanewline
\ \ \isacommand{have}\isamarkupfalse%
\ {\isachardoublequoteopen}is{\isacharunderscore}non{\isacharunderscore}overlapping\ {\isacharparenleft}insert\ {\isacharquery}Y\ {\isacharparenleft}P\ {\isacharminus}\ {\isacharbraceleft}X{\isacharbraceright}{\isacharparenright}{\isacharparenright}{\isachardoublequoteclose}\isanewline
\ \ \ \ \isacommand{using}\isamarkupfalse%
\ rest{\isacharunderscore}is{\isacharunderscore}non{\isacharunderscore}overlapping\ disjoint\ non{\isacharunderscore}empty\ \isacommand{by}\isamarkupfalse%
\ {\isacharparenleft}rule\ non{\isacharunderscore}overlapping{\isacharunderscore}extension{\isadigit{1}}{\isacharparenright}\isanewline
\ \ \isacommand{then}\isamarkupfalse%
\ \isacommand{show}\isamarkupfalse%
\ {\isacharquery}thesis\ \isacommand{unfolding}\isamarkupfalse%
\ insert{\isacharunderscore}into{\isacharunderscore}member{\isacharunderscore}def\ \isacommand{by}\isamarkupfalse%
\ simp\isanewline
\isacommand{qed}\isamarkupfalse%
%
\endisatagproof
{\isafoldproof}%
%
\isadelimproof
%
\endisadelimproof
%
\begin{isamarkuptext}%
inserts an element into a specified set inside the given list of sets --
   the list variant of \isa{insert{\isacharunderscore}into{\isacharunderscore}member}

   The rationale for this variant and for everything that depends on it is:
   While it is possible to computationally enumerate ``all partitions of a set'' as an
   \isa{{\isacharprime}a\ set\ set\ set}, we need a list representation to apply further computational
   functions to partitions.  Because of the way we construct partitions (using functions
   such as \isa{all{\isacharunderscore}coarser{\isacharunderscore}partitions{\isacharunderscore}with} below) it is not sufficient to simply use 
   \isa{{\isacharprime}a\ set\ set\ list}, but we need \isa{{\isacharprime}a\ set\ list\ list}.  This is because it is hard 
   to impossible to convert a set to a list, whereas it is easy to convert a \isa{list} to a \isa{set}.%
\end{isamarkuptext}%
\isamarkuptrue%
\isacommand{definition}\isamarkupfalse%
\ insert{\isacharunderscore}into{\isacharunderscore}member{\isacharunderscore}list\ {\isacharcolon}{\isacharcolon}\ {\isachardoublequoteopen}{\isacharprime}a\ {\isasymRightarrow}\ {\isacharprime}a\ set\ list\ {\isasymRightarrow}\ {\isacharprime}a\ set\ {\isasymRightarrow}\ {\isacharprime}a\ set\ list{\isachardoublequoteclose}\isanewline
\ \ \isakeyword{where}\ {\isachardoublequoteopen}insert{\isacharunderscore}into{\isacharunderscore}member{\isacharunderscore}list\ new{\isacharunderscore}el\ Sets\ S\ {\isacharequal}\ {\isacharparenleft}S\ {\isasymunion}\ {\isacharbraceleft}new{\isacharunderscore}el{\isacharbraceright}{\isacharparenright}\ {\isacharhash}\ {\isacharparenleft}remove{\isadigit{1}}\ S\ Sets{\isacharparenright}{\isachardoublequoteclose}%
\begin{isamarkuptext}%
\isa{insert{\isacharunderscore}into{\isacharunderscore}member{\isacharunderscore}list} and \isa{insert{\isacharunderscore}into{\isacharunderscore}member} are equivalent
  (as in returning the same set).%
\end{isamarkuptext}%
\isamarkuptrue%
\isacommand{lemma}\isamarkupfalse%
\ insert{\isacharunderscore}into{\isacharunderscore}member{\isacharunderscore}list{\isacharunderscore}equivalence{\isacharcolon}\isanewline
\ \ \isakeyword{fixes}\ new{\isacharunderscore}el{\isacharcolon}{\isacharcolon}{\isacharprime}a\isanewline
\ \ \ \ \isakeyword{and}\ Sets{\isacharcolon}{\isacharcolon}{\isachardoublequoteopen}{\isacharprime}a\ set\ list{\isachardoublequoteclose}\isanewline
\ \ \ \ \isakeyword{and}\ S{\isacharcolon}{\isacharcolon}{\isachardoublequoteopen}{\isacharprime}a\ set{\isachardoublequoteclose}\isanewline
\ \ \isakeyword{assumes}\ {\isachardoublequoteopen}distinct\ Sets{\isachardoublequoteclose}\isanewline
\ \ \isakeyword{shows}\ {\isachardoublequoteopen}set\ {\isacharparenleft}insert{\isacharunderscore}into{\isacharunderscore}member{\isacharunderscore}list\ new{\isacharunderscore}el\ Sets\ S{\isacharparenright}\ {\isacharequal}\ insert{\isacharunderscore}into{\isacharunderscore}member\ new{\isacharunderscore}el\ {\isacharparenleft}set\ Sets{\isacharparenright}\ S{\isachardoublequoteclose}\isanewline
%
\isadelimproof
\ \ %
\endisadelimproof
%
\isatagproof
\isacommand{unfolding}\isamarkupfalse%
\ insert{\isacharunderscore}into{\isacharunderscore}member{\isacharunderscore}list{\isacharunderscore}def\ insert{\isacharunderscore}into{\isacharunderscore}member{\isacharunderscore}def\ \isacommand{using}\isamarkupfalse%
\ assms\ \isacommand{by}\isamarkupfalse%
\ simp%
\endisatagproof
{\isafoldproof}%
%
\isadelimproof
%
\endisadelimproof
%
\begin{isamarkuptext}%
an alternative characterization of the set partitioned by a partition obtained by 
  inserting an element into an equivalence class of a given partition (if \isa{P}
  \emph{is} a partition)%
\end{isamarkuptext}%
\isamarkuptrue%
\isacommand{lemma}\isamarkupfalse%
\ insert{\isacharunderscore}into{\isacharunderscore}member{\isacharunderscore}partition{\isadigit{1}}{\isacharcolon}\isanewline
\ \ \isakeyword{fixes}\ elem{\isacharcolon}{\isacharcolon}{\isacharprime}a\isanewline
\ \ \ \ \isakeyword{and}\ P{\isacharcolon}{\isacharcolon}{\isachardoublequoteopen}{\isacharprime}a\ set\ set{\isachardoublequoteclose}\isanewline
\ \ \ \ \isakeyword{and}\ set{\isacharcolon}{\isacharcolon}{\isachardoublequoteopen}{\isacharprime}a\ set{\isachardoublequoteclose}\isanewline
\ \ \isakeyword{shows}\ {\isachardoublequoteopen}{\isasymUnion}\ insert{\isacharunderscore}into{\isacharunderscore}member\ elem\ P\ set\ {\isacharequal}\ {\isasymUnion}\ insert\ {\isacharparenleft}set\ {\isasymunion}\ {\isacharbraceleft}elem{\isacharbraceright}{\isacharparenright}\ {\isacharparenleft}P\ {\isacharminus}\ {\isacharbraceleft}set{\isacharbraceright}{\isacharparenright}{\isachardoublequoteclose}\isanewline
%
\isadelimproof
\isanewline
\ \ \ \ %
\endisadelimproof
%
\isatagproof
\isacommand{unfolding}\isamarkupfalse%
\ insert{\isacharunderscore}into{\isacharunderscore}member{\isacharunderscore}def\isanewline
\ \ \ \ \isacommand{by}\isamarkupfalse%
\ fast%
\endisatagproof
{\isafoldproof}%
%
\isadelimproof
%
\endisadelimproof
%
\begin{isamarkuptext}%
Assuming that \isa{P} is a partition of a set \isa{S}, and \isa{new{\isacharunderscore}el\ {\isasymnotin}\ S}, the function defined below yields
  all possible partitions of \isa{S\ {\isasymunion}\ {\isacharbraceleft}new{\isacharunderscore}el{\isacharbraceright}} that are coarser than \isa{P}
  (i.e.\ not splitting classes that already exist in \isa{P}).  These comprise one partition 
  with a class \isa{{\isacharbraceleft}new{\isacharunderscore}el{\isacharbraceright}} and all other classes unchanged,
  as well as all partitions obtained by inserting \isa{new{\isacharunderscore}el} into one class of \isa{P} at a time. While we use the definition to build coarser partitions of an existing partition P, the definition itself does not require P to be a partition.%
\end{isamarkuptext}%
\isamarkuptrue%
\isacommand{definition}\isamarkupfalse%
\ coarser{\isacharunderscore}partitions{\isacharunderscore}with\ {\isacharcolon}{\isacharcolon}{\isachardoublequoteopen}{\isacharprime}a\ {\isasymRightarrow}\ {\isacharprime}a\ set\ set\ {\isasymRightarrow}\ {\isacharprime}a\ set\ set\ set{\isachardoublequoteclose}\isanewline
\ \ \isakeyword{where}\ {\isachardoublequoteopen}coarser{\isacharunderscore}partitions{\isacharunderscore}with\ new{\isacharunderscore}el\ P\ {\isacharequal}\ \isanewline
\ \ \ \ insert\isanewline
\ \ \ \ {\isacharparenleft}{\isacharasterisk}\ Let\ P\ be\ a\ partition\ of\ a\ set\ Set{\isacharcomma}\isanewline
\ \ \ \ \ and\ suppose\ new{\isacharunderscore}el\ {\isasymnotin}\ Set{\isacharcomma}\ i{\isachardot}e{\isachardot}\ {\isacharbraceleft}new{\isacharunderscore}el{\isacharbraceright}\ {\isasymnotin}\ P{\isacharcomma}\isanewline
\ \ \ \ \ then\ the\ following\ constructs\ a\ partition\ of\ {\isacharprime}Set\ {\isasymunion}\ {\isacharbraceleft}new{\isacharunderscore}el{\isacharbraceright}{\isacharprime}\ obtained\ by\isanewline
\ \ \ \ \ inserting\ a\ new\ class\ {\isacharbraceleft}new{\isacharunderscore}el{\isacharbraceright}\ and\ leaving\ all\ previous\ classes\ unchanged{\isachardot}\ {\isacharasterisk}{\isacharparenright}\isanewline
\ \ \ \ {\isacharparenleft}insert\ {\isacharbraceleft}new{\isacharunderscore}el{\isacharbraceright}\ P{\isacharparenright}\isanewline
\ \ {\isacharparenleft}{\isacharasterisk}\ Let\ P\ be\ a\ partition\ of\ a\ set\ Set{\isacharcomma}\isanewline
\ \ \ \ \ and\ suppose\ new{\isacharunderscore}el\ {\isasymnotin}\ Set{\isacharcomma}\isanewline
\ \ \ \ \ then\ the\ following\ constructs\isanewline
\ \ \ \ \ the\ set\ of\ those\ partitions\ of\ {\isacharprime}Set\ {\isasymunion}\ {\isacharbraceleft}new{\isacharunderscore}el{\isacharbraceright}{\isacharprime}\ obtained\ by\isanewline
\ \ \ \ \ inserting\ new{\isacharunderscore}el\ into\ one\ class\ of\ P\ at\ a\ time{\isachardot}\ {\isacharasterisk}{\isacharparenright}\isanewline
\ \ \ \ {\isacharparenleft}{\isacharparenleft}insert{\isacharunderscore}into{\isacharunderscore}member\ new{\isacharunderscore}el\ P{\isacharparenright}\ {\isacharbackquote}\ P{\isacharparenright}{\isachardoublequoteclose}%
\begin{isamarkuptext}%
the list variant of \isa{coarser{\isacharunderscore}partitions{\isacharunderscore}with}%
\end{isamarkuptext}%
\isamarkuptrue%
\isacommand{definition}\isamarkupfalse%
\ coarser{\isacharunderscore}partitions{\isacharunderscore}with{\isacharunderscore}list\ {\isacharcolon}{\isacharcolon}{\isachardoublequoteopen}{\isacharprime}a\ {\isasymRightarrow}\ {\isacharprime}a\ set\ list\ {\isasymRightarrow}\ {\isacharprime}a\ set\ list\ list{\isachardoublequoteclose}\isanewline
\ \ \isakeyword{where}\ {\isachardoublequoteopen}coarser{\isacharunderscore}partitions{\isacharunderscore}with{\isacharunderscore}list\ new{\isacharunderscore}el\ P\ {\isacharequal}\ \isanewline
\ \ {\isacharparenleft}{\isacharasterisk}\ Let\ P\ be\ a\ partition\ of\ a\ set\ Set{\isacharcomma}\isanewline
\ \ \ \ \ and\ suppose\ new{\isacharunderscore}el\ {\isasymnotin}\ Set{\isacharcomma}\ i{\isachardot}e{\isachardot}\ {\isacharbraceleft}new{\isacharunderscore}el{\isacharbraceright}\ {\isasymnotin}\ set\ P{\isacharcomma}\isanewline
\ \ \ \ \ then\ the\ following\ constructs\ a\ partition\ of\ {\isacharprime}Set\ {\isasymunion}\ {\isacharbraceleft}new{\isacharunderscore}el{\isacharbraceright}{\isacharprime}\ obtained\ by\isanewline
\ \ \ \ \ inserting\ a\ new\ class\ {\isacharbraceleft}new{\isacharunderscore}el{\isacharbraceright}\ and\ leaving\ all\ previous\ classes\ unchanged{\isachardot}\ {\isacharasterisk}{\isacharparenright}\isanewline
\ \ \ \ {\isacharparenleft}{\isacharbraceleft}new{\isacharunderscore}el{\isacharbraceright}\ {\isacharhash}\ P{\isacharparenright}\isanewline
\ \ \ \ {\isacharhash}\isanewline
\ \ {\isacharparenleft}{\isacharasterisk}\ Let\ P\ be\ a\ partition\ of\ a\ set\ Set{\isacharcomma}\isanewline
\ \ \ \ \ and\ suppose\ new{\isacharunderscore}el\ {\isasymnotin}\ Set{\isacharcomma}\isanewline
\ \ \ \ \ then\ the\ following\ constructs\isanewline
\ \ \ \ \ the\ set\ of\ those\ partitions\ of\ {\isacharprime}Set\ {\isasymunion}\ {\isacharbraceleft}new{\isacharunderscore}el{\isacharbraceright}{\isacharprime}\ obtained\ by\isanewline
\ \ \ \ \ inserting\ new{\isacharunderscore}el\ into\ one\ class\ of\ P\ at\ a\ time{\isachardot}\ {\isacharasterisk}{\isacharparenright}\isanewline
\ \ \ \ {\isacharparenleft}map\ {\isacharparenleft}{\isacharparenleft}insert{\isacharunderscore}into{\isacharunderscore}member{\isacharunderscore}list\ new{\isacharunderscore}el\ P{\isacharparenright}{\isacharparenright}\ P{\isacharparenright}{\isachardoublequoteclose}%
\begin{isamarkuptext}%
\isa{coarser{\isacharunderscore}partitions{\isacharunderscore}with{\isacharunderscore}list} and \isa{coarser{\isacharunderscore}partitions{\isacharunderscore}with} are equivalent.%
\end{isamarkuptext}%
\isamarkuptrue%
\isacommand{lemma}\isamarkupfalse%
\ coarser{\isacharunderscore}partitions{\isacharunderscore}with{\isacharunderscore}list{\isacharunderscore}equivalence{\isacharcolon}\isanewline
\ \ \isakeyword{assumes}\ {\isachardoublequoteopen}distinct\ P{\isachardoublequoteclose}\isanewline
\ \ \isakeyword{shows}\ {\isachardoublequoteopen}set\ {\isacharparenleft}map\ set\ {\isacharparenleft}coarser{\isacharunderscore}partitions{\isacharunderscore}with{\isacharunderscore}list\ new{\isacharunderscore}el\ P{\isacharparenright}{\isacharparenright}\ {\isacharequal}\ \isanewline
\ \ \ \ \ \ \ \ \ coarser{\isacharunderscore}partitions{\isacharunderscore}with\ new{\isacharunderscore}el\ {\isacharparenleft}set\ P{\isacharparenright}{\isachardoublequoteclose}\isanewline
%
\isadelimproof
%
\endisadelimproof
%
\isatagproof
\isacommand{proof}\isamarkupfalse%
\ {\isacharminus}\isanewline
\ \ \isacommand{have}\isamarkupfalse%
\ {\isachardoublequoteopen}set\ {\isacharparenleft}map\ set\ {\isacharparenleft}coarser{\isacharunderscore}partitions{\isacharunderscore}with{\isacharunderscore}list\ new{\isacharunderscore}el\ P{\isacharparenright}{\isacharparenright}\ {\isacharequal}\ set\ {\isacharparenleft}map\ set\ {\isacharparenleft}{\isacharparenleft}{\isacharbraceleft}new{\isacharunderscore}el{\isacharbraceright}\ {\isacharhash}\ P{\isacharparenright}\ {\isacharhash}\ {\isacharparenleft}map\ {\isacharparenleft}{\isacharparenleft}insert{\isacharunderscore}into{\isacharunderscore}member{\isacharunderscore}list\ new{\isacharunderscore}el\ P{\isacharparenright}{\isacharparenright}\ P{\isacharparenright}{\isacharparenright}{\isacharparenright}{\isachardoublequoteclose}\isanewline
\ \ \ \ \isacommand{unfolding}\isamarkupfalse%
\ coarser{\isacharunderscore}partitions{\isacharunderscore}with{\isacharunderscore}list{\isacharunderscore}def\ \isacommand{{\isachardot}{\isachardot}}\isamarkupfalse%
\isanewline
\ \ \isacommand{also}\isamarkupfalse%
\ \isacommand{have}\isamarkupfalse%
\ {\isachardoublequoteopen}{\isasymdots}\ {\isacharequal}\ insert\ {\isacharparenleft}insert\ {\isacharbraceleft}new{\isacharunderscore}el{\isacharbraceright}\ {\isacharparenleft}set\ P{\isacharparenright}{\isacharparenright}\ {\isacharparenleft}{\isacharparenleft}set\ {\isasymcirc}\ {\isacharparenleft}insert{\isacharunderscore}into{\isacharunderscore}member{\isacharunderscore}list\ new{\isacharunderscore}el\ P{\isacharparenright}{\isacharparenright}\ {\isacharbackquote}\ set\ P{\isacharparenright}{\isachardoublequoteclose}\isanewline
\ \ \ \ \isacommand{by}\isamarkupfalse%
\ simp\isanewline
\ \ \isacommand{also}\isamarkupfalse%
\ \isacommand{have}\isamarkupfalse%
\ {\isachardoublequoteopen}{\isasymdots}\ {\isacharequal}\ insert\ {\isacharparenleft}insert\ {\isacharbraceleft}new{\isacharunderscore}el{\isacharbraceright}\ {\isacharparenleft}set\ P{\isacharparenright}{\isacharparenright}\ {\isacharparenleft}{\isacharparenleft}insert{\isacharunderscore}into{\isacharunderscore}member\ new{\isacharunderscore}el\ {\isacharparenleft}set\ P{\isacharparenright}{\isacharparenright}\ {\isacharbackquote}\ set\ P{\isacharparenright}{\isachardoublequoteclose}\isanewline
\ \ \ \ \isacommand{using}\isamarkupfalse%
\ assms\ insert{\isacharunderscore}into{\isacharunderscore}member{\isacharunderscore}list{\isacharunderscore}equivalence\ \isacommand{by}\isamarkupfalse%
\ {\isacharparenleft}metis\ comp{\isacharunderscore}apply{\isacharparenright}\isanewline
\ \ \isacommand{finally}\isamarkupfalse%
\ \isacommand{show}\isamarkupfalse%
\ {\isacharquery}thesis\ \isacommand{unfolding}\isamarkupfalse%
\ coarser{\isacharunderscore}partitions{\isacharunderscore}with{\isacharunderscore}def\ \isacommand{{\isachardot}}\isamarkupfalse%
\isanewline
\isacommand{qed}\isamarkupfalse%
%
\endisatagproof
{\isafoldproof}%
%
\isadelimproof
%
\endisadelimproof
%
\begin{isamarkuptext}%
Any member of the set of coarser partitions of a given partition, obtained by inserting 
  a given fresh element into each of its classes, is non_overlapping.%
\end{isamarkuptext}%
\isamarkuptrue%
\isacommand{lemma}\isamarkupfalse%
\ non{\isacharunderscore}overlapping{\isacharunderscore}extension{\isadigit{3}}{\isacharcolon}\isanewline
\ \ \isakeyword{fixes}\ elem{\isacharcolon}{\isacharcolon}{\isacharprime}a\isanewline
\ \ \ \ \isakeyword{and}\ P{\isacharcolon}{\isacharcolon}{\isachardoublequoteopen}{\isacharprime}a\ set\ set{\isachardoublequoteclose}\isanewline
\ \ \ \ \isakeyword{and}\ Q{\isacharcolon}{\isacharcolon}{\isachardoublequoteopen}{\isacharprime}a\ set\ set{\isachardoublequoteclose}\isanewline
\ \ \isakeyword{assumes}\ P{\isacharunderscore}non{\isacharunderscore}overlapping{\isacharcolon}\ {\isachardoublequoteopen}is{\isacharunderscore}non{\isacharunderscore}overlapping\ P{\isachardoublequoteclose}\isanewline
\ \ \ \ \ \ \isakeyword{and}\ new{\isacharunderscore}elem{\isacharcolon}\ {\isachardoublequoteopen}elem\ {\isasymnotin}\ {\isasymUnion}\ P{\isachardoublequoteclose}\isanewline
\ \ \ \ \ \ \isakeyword{and}\ Q{\isacharunderscore}coarser{\isacharcolon}\ {\isachardoublequoteopen}Q\ {\isasymin}\ coarser{\isacharunderscore}partitions{\isacharunderscore}with\ elem\ P{\isachardoublequoteclose}\isanewline
\ \ \isakeyword{shows}\ {\isachardoublequoteopen}is{\isacharunderscore}non{\isacharunderscore}overlapping\ Q{\isachardoublequoteclose}\isanewline
%
\isadelimproof
%
\endisadelimproof
%
\isatagproof
\isacommand{proof}\isamarkupfalse%
\ {\isacharminus}\isanewline
\ \ \isacommand{let}\isamarkupfalse%
\ {\isacharquery}q\ {\isacharequal}\ {\isachardoublequoteopen}insert\ {\isacharbraceleft}elem{\isacharbraceright}\ P{\isachardoublequoteclose}\isanewline
\ \ \isacommand{have}\isamarkupfalse%
\ Q{\isacharunderscore}coarser{\isacharunderscore}unfolded{\isacharcolon}\ {\isachardoublequoteopen}Q\ {\isasymin}\ insert\ {\isacharquery}q\ {\isacharparenleft}insert{\isacharunderscore}into{\isacharunderscore}member\ elem\ P\ {\isacharbackquote}\ P{\isacharparenright}{\isachardoublequoteclose}\ \isanewline
\ \ \ \ \isacommand{using}\isamarkupfalse%
\ Q{\isacharunderscore}coarser\ \isanewline
\ \ \ \ \isacommand{unfolding}\isamarkupfalse%
\ coarser{\isacharunderscore}partitions{\isacharunderscore}with{\isacharunderscore}def\isanewline
\ \ \ \ \isacommand{by}\isamarkupfalse%
\ fast\isanewline
\ \ \isacommand{show}\isamarkupfalse%
\ {\isacharquery}thesis\isanewline
\ \ \isacommand{proof}\isamarkupfalse%
\ {\isacharparenleft}cases\ {\isachardoublequoteopen}Q\ {\isacharequal}\ {\isacharquery}q{\isachardoublequoteclose}{\isacharparenright}\isanewline
\ \ \ \ \isacommand{case}\isamarkupfalse%
\ True\isanewline
\ \ \ \ \isacommand{then}\isamarkupfalse%
\ \isacommand{show}\isamarkupfalse%
\ {\isacharquery}thesis\isanewline
\ \ \ \ \ \ \isacommand{using}\isamarkupfalse%
\ P{\isacharunderscore}non{\isacharunderscore}overlapping\ new{\isacharunderscore}elem\ non{\isacharunderscore}overlapping{\isacharunderscore}extension{\isadigit{1}}\isanewline
\ \ \ \ \ \ \isacommand{by}\isamarkupfalse%
\ fastforce\isanewline
\ \ \isacommand{next}\isamarkupfalse%
\isanewline
\ \ \ \ \isacommand{case}\isamarkupfalse%
\ False\isanewline
\ \ \ \ \isacommand{then}\isamarkupfalse%
\ \isacommand{have}\isamarkupfalse%
\ {\isachardoublequoteopen}Q\ {\isasymin}\ {\isacharparenleft}insert{\isacharunderscore}into{\isacharunderscore}member\ elem\ P{\isacharparenright}\ {\isacharbackquote}\ P{\isachardoublequoteclose}\ \isacommand{using}\isamarkupfalse%
\ Q{\isacharunderscore}coarser{\isacharunderscore}unfolded\ \isacommand{by}\isamarkupfalse%
\ fastforce\isanewline
\ \ \ \ \isacommand{then}\isamarkupfalse%
\ \isacommand{show}\isamarkupfalse%
\ {\isacharquery}thesis\ \isacommand{using}\isamarkupfalse%
\ non{\isacharunderscore}overlapping{\isacharunderscore}extension{\isadigit{2}}\ P{\isacharunderscore}non{\isacharunderscore}overlapping\ new{\isacharunderscore}elem\ \isacommand{by}\isamarkupfalse%
\ fast\isanewline
\ \ \isacommand{qed}\isamarkupfalse%
\isanewline
\isacommand{qed}\isamarkupfalse%
%
\endisatagproof
{\isafoldproof}%
%
\isadelimproof
%
\endisadelimproof
%
\begin{isamarkuptext}%
Let \isa{P} be a partition of a set \isa{S}, and \isa{elem} an element (which may or may not be
  in \isa{S} already).  Then, any member of \isa{coarser{\isacharunderscore}partitions{\isacharunderscore}with\ elem\ P} is a set of sets
  whose union is \isa{S\ {\isasymunion}\ {\isacharbraceleft}elem{\isacharbraceright}}, i.e.\ it satisfies one of the necessary criteria for being a partition of \isa{S\ {\isasymunion}\ {\isacharbraceleft}elem{\isacharbraceright}}.%
\end{isamarkuptext}%
\isamarkuptrue%
\isacommand{lemma}\isamarkupfalse%
\ coarser{\isacharunderscore}partitions{\isacharunderscore}covers{\isacharcolon}\isanewline
\ \ \isakeyword{fixes}\ elem{\isacharcolon}{\isacharcolon}{\isacharprime}a\isanewline
\ \ \ \ \isakeyword{and}\ P{\isacharcolon}{\isacharcolon}{\isachardoublequoteopen}{\isacharprime}a\ set\ set{\isachardoublequoteclose}\isanewline
\ \ \ \ \isakeyword{and}\ Q{\isacharcolon}{\isacharcolon}{\isachardoublequoteopen}{\isacharprime}a\ set\ set{\isachardoublequoteclose}\isanewline
\ \ \isakeyword{assumes}\ {\isachardoublequoteopen}Q\ {\isasymin}\ coarser{\isacharunderscore}partitions{\isacharunderscore}with\ elem\ P{\isachardoublequoteclose}\isanewline
\ \ \isakeyword{shows}\ {\isachardoublequoteopen}{\isasymUnion}\ Q\ {\isacharequal}\ insert\ elem\ {\isacharparenleft}{\isasymUnion}\ P{\isacharparenright}{\isachardoublequoteclose}\isanewline
%
\isadelimproof
%
\endisadelimproof
%
\isatagproof
\isacommand{proof}\isamarkupfalse%
\ {\isacharminus}\isanewline
\ \ \isacommand{let}\isamarkupfalse%
\ {\isacharquery}S\ {\isacharequal}\ {\isachardoublequoteopen}{\isasymUnion}\ P{\isachardoublequoteclose}\isanewline
\ \ \isacommand{have}\isamarkupfalse%
\ Q{\isacharunderscore}cases{\isacharcolon}\ {\isachardoublequoteopen}Q\ {\isasymin}\ {\isacharparenleft}insert{\isacharunderscore}into{\isacharunderscore}member\ elem\ P{\isacharparenright}\ {\isacharbackquote}\ P\ {\isasymor}\ Q\ {\isacharequal}\ insert\ {\isacharbraceleft}elem{\isacharbraceright}\ P{\isachardoublequoteclose}\isanewline
\ \ \ \ \isacommand{using}\isamarkupfalse%
\ assms\ \isacommand{unfolding}\isamarkupfalse%
\ coarser{\isacharunderscore}partitions{\isacharunderscore}with{\isacharunderscore}def\ \isacommand{by}\isamarkupfalse%
\ fast\isanewline
\ \ \isacommand{{\isacharbraceleft}}\isamarkupfalse%
\isanewline
\ \ \ \ \isacommand{fix}\isamarkupfalse%
\ eq{\isacharunderscore}class\ \isacommand{assume}\isamarkupfalse%
\ eq{\isacharunderscore}class{\isacharunderscore}in{\isacharunderscore}P{\isacharcolon}\ {\isachardoublequoteopen}eq{\isacharunderscore}class\ {\isasymin}\ P{\isachardoublequoteclose}\isanewline
\ \ \ \ \isacommand{have}\isamarkupfalse%
\ {\isachardoublequoteopen}{\isasymUnion}\ insert\ {\isacharparenleft}eq{\isacharunderscore}class\ {\isasymunion}\ {\isacharbraceleft}elem{\isacharbraceright}{\isacharparenright}\ {\isacharparenleft}P\ {\isacharminus}\ {\isacharbraceleft}eq{\isacharunderscore}class{\isacharbraceright}{\isacharparenright}\ {\isacharequal}\ {\isacharquery}S\ {\isasymunion}\ {\isacharparenleft}eq{\isacharunderscore}class\ {\isasymunion}\ {\isacharbraceleft}elem{\isacharbraceright}{\isacharparenright}{\isachardoublequoteclose}\isanewline
\ \ \ \ \ \ \isacommand{using}\isamarkupfalse%
\ insert{\isacharunderscore}into{\isacharunderscore}member{\isacharunderscore}partition{\isadigit{1}}\isanewline
\ \ \ \ \ \ \isacommand{by}\isamarkupfalse%
\ {\isacharparenleft}metis\ Sup{\isacharunderscore}insert\ Un{\isacharunderscore}commute\ Un{\isacharunderscore}empty{\isacharunderscore}right\ Un{\isacharunderscore}insert{\isacharunderscore}right\ insert{\isacharunderscore}Diff{\isacharunderscore}single{\isacharparenright}\isanewline
\ \ \ \ \isacommand{with}\isamarkupfalse%
\ eq{\isacharunderscore}class{\isacharunderscore}in{\isacharunderscore}P\ \isacommand{have}\isamarkupfalse%
\ {\isachardoublequoteopen}{\isasymUnion}\ insert\ {\isacharparenleft}eq{\isacharunderscore}class\ {\isasymunion}\ {\isacharbraceleft}elem{\isacharbraceright}{\isacharparenright}\ {\isacharparenleft}P\ {\isacharminus}\ {\isacharbraceleft}eq{\isacharunderscore}class{\isacharbraceright}{\isacharparenright}\ {\isacharequal}\ {\isacharquery}S\ {\isasymunion}\ {\isacharbraceleft}elem{\isacharbraceright}{\isachardoublequoteclose}\ \isacommand{by}\isamarkupfalse%
\ blast\isanewline
\ \ \ \ \isacommand{then}\isamarkupfalse%
\ \isacommand{have}\isamarkupfalse%
\ {\isachardoublequoteopen}{\isasymUnion}\ insert{\isacharunderscore}into{\isacharunderscore}member\ elem\ P\ eq{\isacharunderscore}class\ {\isacharequal}\ {\isacharquery}S\ {\isasymunion}\ {\isacharbraceleft}elem{\isacharbraceright}{\isachardoublequoteclose}\isanewline
\ \ \ \ \ \ \isacommand{using}\isamarkupfalse%
\ insert{\isacharunderscore}into{\isacharunderscore}member{\isacharunderscore}partition{\isadigit{1}}\isanewline
\ \ \ \ \ \ \isacommand{by}\isamarkupfalse%
\ {\isacharparenleft}rule\ subst{\isacharparenright}\isanewline
\ \ \isacommand{{\isacharbraceright}}\isamarkupfalse%
\isanewline
\ \ \isacommand{then}\isamarkupfalse%
\ \isacommand{show}\isamarkupfalse%
\ {\isacharquery}thesis\ \isacommand{using}\isamarkupfalse%
\ Q{\isacharunderscore}cases\ \isacommand{by}\isamarkupfalse%
\ blast\isanewline
\isacommand{qed}\isamarkupfalse%
%
\endisatagproof
{\isafoldproof}%
%
\isadelimproof
%
\endisadelimproof
%
\begin{isamarkuptext}%
Removes the element \isa{elem} from every set in \isa{P}, and removes from \isa{P} any
  remaining empty sets.  This function is intended to be applied to partitions, i.e. \isa{elem}
  occurs in at most one set.  \isa{partition{\isacharunderscore}without\ e} reverses \isa{coarser{\isacharunderscore}partitions{\isacharunderscore}with\ e}.
\isa{coarser{\isacharunderscore}partitions{\isacharunderscore}with} is one-to-many, while this is one-to-one, so we can think of a tree relation,
where coarser partitions of a set \isa{S\ {\isasymunion}\ {\isacharbraceleft}elem{\isacharbraceright}} are child nodes of one partition of \isa{S}.%
\end{isamarkuptext}%
\isamarkuptrue%
\isacommand{definition}\isamarkupfalse%
\ partition{\isacharunderscore}without\ {\isacharcolon}{\isacharcolon}\ {\isachardoublequoteopen}{\isacharprime}a\ {\isasymRightarrow}\ {\isacharprime}a\ set\ set\ {\isasymRightarrow}\ {\isacharprime}a\ set\ set{\isachardoublequoteclose}\isanewline
\ \ \isakeyword{where}\ {\isachardoublequoteopen}partition{\isacharunderscore}without\ elem\ P\ {\isacharequal}\ {\isacharparenleft}{\isasymlambda}X\ {\isachardot}\ X\ {\isacharminus}\ {\isacharbraceleft}elem{\isacharbraceright}{\isacharparenright}\ {\isacharbackquote}\ P\ {\isacharminus}\ {\isacharbraceleft}{\isacharbraceleft}{\isacharbraceright}{\isacharbraceright}{\isachardoublequoteclose}%
\begin{isamarkuptext}%
alternative characterization of the set partitioned by the partition obtained 
  by removing an element from a given partition using \isa{partition{\isacharunderscore}without}%
\end{isamarkuptext}%
\isamarkuptrue%
\isacommand{lemma}\isamarkupfalse%
\ partition{\isacharunderscore}without{\isacharunderscore}covers{\isacharcolon}\isanewline
\ \ \isakeyword{fixes}\ elem{\isacharcolon}{\isacharcolon}{\isacharprime}a\isanewline
\ \ \ \ \isakeyword{and}\ P{\isacharcolon}{\isacharcolon}{\isachardoublequoteopen}{\isacharprime}a\ set\ set{\isachardoublequoteclose}\isanewline
\ \ \isakeyword{shows}\ {\isachardoublequoteopen}{\isasymUnion}\ partition{\isacharunderscore}without\ elem\ P\ {\isacharequal}\ {\isacharparenleft}{\isasymUnion}\ P{\isacharparenright}\ {\isacharminus}\ {\isacharbraceleft}elem{\isacharbraceright}{\isachardoublequoteclose}\isanewline
%
\isadelimproof
%
\endisadelimproof
%
\isatagproof
\isacommand{proof}\isamarkupfalse%
\ {\isacharminus}\isanewline
\ \ \isacommand{have}\isamarkupfalse%
\ {\isachardoublequoteopen}{\isasymUnion}\ partition{\isacharunderscore}without\ elem\ P\ {\isacharequal}\ {\isasymUnion}\ {\isacharparenleft}{\isacharparenleft}{\isasymlambda}x\ {\isachardot}\ x\ {\isacharminus}\ {\isacharbraceleft}elem{\isacharbraceright}{\isacharparenright}\ {\isacharbackquote}\ P\ {\isacharminus}\ {\isacharbraceleft}{\isacharbraceleft}{\isacharbraceright}{\isacharbraceright}{\isacharparenright}{\isachardoublequoteclose}\isanewline
\ \ \ \ \isacommand{unfolding}\isamarkupfalse%
\ partition{\isacharunderscore}without{\isacharunderscore}def\ \isacommand{by}\isamarkupfalse%
\ fast\isanewline
\ \ \isacommand{also}\isamarkupfalse%
\ \isacommand{have}\isamarkupfalse%
\ {\isachardoublequoteopen}{\isasymdots}\ {\isacharequal}\ {\isasymUnion}\ P\ {\isacharminus}\ {\isacharbraceleft}elem{\isacharbraceright}{\isachardoublequoteclose}\ \isacommand{by}\isamarkupfalse%
\ blast\isanewline
\ \ \isacommand{finally}\isamarkupfalse%
\ \isacommand{show}\isamarkupfalse%
\ {\isacharquery}thesis\ \isacommand{{\isachardot}}\isamarkupfalse%
\isanewline
\isacommand{qed}\isamarkupfalse%
%
\endisatagproof
{\isafoldproof}%
%
\isadelimproof
%
\endisadelimproof
%
\begin{isamarkuptext}%
Any class of the partition obtained by removing an element \isa{elem} from an
  original partition \isa{P} using \isa{partition{\isacharunderscore}without} equals some
  class of \isa{P}, reduced by \isa{elem}.%
\end{isamarkuptext}%
\isamarkuptrue%
\isacommand{lemma}\isamarkupfalse%
\ super{\isacharunderscore}class{\isacharcolon}\isanewline
\ \ \isakeyword{assumes}\ {\isachardoublequoteopen}X\ {\isasymin}\ partition{\isacharunderscore}without\ elem\ P{\isachardoublequoteclose}\isanewline
\ \ \isakeyword{obtains}\ Z\ \isakeyword{where}\ {\isachardoublequoteopen}Z\ {\isasymin}\ P{\isachardoublequoteclose}\ \isakeyword{and}\ {\isachardoublequoteopen}X\ {\isacharequal}\ Z\ {\isacharminus}\ {\isacharbraceleft}elem{\isacharbraceright}{\isachardoublequoteclose}\isanewline
%
\isadelimproof
%
\endisadelimproof
%
\isatagproof
\isacommand{proof}\isamarkupfalse%
\ {\isacharminus}\isanewline
\ \ \isacommand{from}\isamarkupfalse%
\ assms\ \isacommand{have}\isamarkupfalse%
\ {\isachardoublequoteopen}X\ {\isasymin}\ {\isacharparenleft}{\isasymlambda}X\ {\isachardot}\ X\ {\isacharminus}\ {\isacharbraceleft}elem{\isacharbraceright}{\isacharparenright}\ {\isacharbackquote}\ P\ {\isacharminus}\ {\isacharbraceleft}{\isacharbraceleft}{\isacharbraceright}{\isacharbraceright}{\isachardoublequoteclose}\ \isacommand{unfolding}\isamarkupfalse%
\ partition{\isacharunderscore}without{\isacharunderscore}def\ \isacommand{{\isachardot}}\isamarkupfalse%
\isanewline
\ \ \isacommand{then}\isamarkupfalse%
\ \isacommand{obtain}\isamarkupfalse%
\ Z\ \isakeyword{where}\ Z{\isacharunderscore}in{\isacharunderscore}P{\isacharcolon}\ {\isachardoublequoteopen}Z\ {\isasymin}\ P{\isachardoublequoteclose}\ \isakeyword{and}\ Z{\isacharunderscore}sup{\isacharcolon}\ {\isachardoublequoteopen}X\ {\isacharequal}\ Z\ {\isacharminus}\ {\isacharbraceleft}elem{\isacharbraceright}{\isachardoublequoteclose}\isanewline
\ \ \ \ \isacommand{by}\isamarkupfalse%
\ {\isacharparenleft}metis\ {\isacharparenleft}lifting{\isacharparenright}\ Diff{\isacharunderscore}iff\ image{\isacharunderscore}iff{\isacharparenright}\isanewline
\ \ \isacommand{then}\isamarkupfalse%
\ \isacommand{show}\isamarkupfalse%
\ {\isacharquery}thesis\ \isacommand{{\isachardot}{\isachardot}}\isamarkupfalse%
\isanewline
\isacommand{qed}\isamarkupfalse%
%
\endisatagproof
{\isafoldproof}%
%
\isadelimproof
%
\endisadelimproof
%
\begin{isamarkuptext}%
The class of sets obtained by removing an element from a non-overlapping class is another
  non-overlapping clas.%
\end{isamarkuptext}%
\isamarkuptrue%
\isacommand{lemma}\isamarkupfalse%
\ non{\isacharunderscore}overlapping{\isacharunderscore}without{\isacharunderscore}is{\isacharunderscore}non{\isacharunderscore}overlapping{\isacharcolon}\isanewline
\ \ \isakeyword{fixes}\ elem{\isacharcolon}{\isacharcolon}{\isacharprime}a\isanewline
\ \ \ \ \isakeyword{and}\ P{\isacharcolon}{\isacharcolon}{\isachardoublequoteopen}{\isacharprime}a\ set\ set{\isachardoublequoteclose}\isanewline
\ \ \isakeyword{assumes}\ {\isachardoublequoteopen}is{\isacharunderscore}non{\isacharunderscore}overlapping\ P{\isachardoublequoteclose}\isanewline
\ \ \isakeyword{shows}\ {\isachardoublequoteopen}is{\isacharunderscore}non{\isacharunderscore}overlapping\ {\isacharparenleft}partition{\isacharunderscore}without\ elem\ P{\isacharparenright}{\isachardoublequoteclose}\ {\isacharparenleft}\isakeyword{is}\ {\isachardoublequoteopen}is{\isacharunderscore}non{\isacharunderscore}overlapping\ {\isacharquery}Q{\isachardoublequoteclose}{\isacharparenright}\isanewline
%
\isadelimproof
%
\endisadelimproof
%
\isatagproof
\isacommand{proof}\isamarkupfalse%
\ {\isacharminus}\ \ \ \isanewline
\ \ \isacommand{have}\isamarkupfalse%
\ {\isachardoublequoteopen}{\isasymforall}\ X{\isadigit{1}}\ {\isasymin}\ {\isacharquery}Q{\isachardot}\ {\isasymforall}\ X{\isadigit{2}}\ {\isasymin}\ {\isacharquery}Q{\isachardot}\ X{\isadigit{1}}\ {\isasyminter}\ X{\isadigit{2}}\ {\isasymnoteq}\ {\isacharbraceleft}{\isacharbraceright}\ {\isasymlongleftrightarrow}\ X{\isadigit{1}}\ {\isacharequal}\ X{\isadigit{2}}{\isachardoublequoteclose}\isanewline
\ \ \isacommand{proof}\isamarkupfalse%
\ \isanewline
\ \ \ \ \isacommand{fix}\isamarkupfalse%
\ X{\isadigit{1}}\ \isacommand{assume}\isamarkupfalse%
\ X{\isadigit{1}}{\isacharunderscore}in{\isacharunderscore}Q{\isacharcolon}\ {\isachardoublequoteopen}X{\isadigit{1}}\ {\isasymin}\ {\isacharquery}Q{\isachardoublequoteclose}\isanewline
\ \ \ \ \isacommand{then}\isamarkupfalse%
\ \isacommand{obtain}\isamarkupfalse%
\ Z{\isadigit{1}}\ \isakeyword{where}\ Z{\isadigit{1}}{\isacharunderscore}in{\isacharunderscore}P{\isacharcolon}\ {\isachardoublequoteopen}Z{\isadigit{1}}\ {\isasymin}\ P{\isachardoublequoteclose}\ \isakeyword{and}\ Z{\isadigit{1}}{\isacharunderscore}sup{\isacharcolon}\ {\isachardoublequoteopen}X{\isadigit{1}}\ {\isacharequal}\ Z{\isadigit{1}}\ {\isacharminus}\ {\isacharbraceleft}elem{\isacharbraceright}{\isachardoublequoteclose}\isanewline
\ \ \ \ \ \ \isacommand{by}\isamarkupfalse%
\ {\isacharparenleft}rule\ super{\isacharunderscore}class{\isacharparenright}\isanewline
\ \ \ \ \isacommand{have}\isamarkupfalse%
\ X{\isadigit{1}}{\isacharunderscore}non{\isacharunderscore}empty{\isacharcolon}\ {\isachardoublequoteopen}X{\isadigit{1}}\ {\isasymnoteq}\ {\isacharbraceleft}{\isacharbraceright}{\isachardoublequoteclose}\ \isacommand{using}\isamarkupfalse%
\ X{\isadigit{1}}{\isacharunderscore}in{\isacharunderscore}Q\ partition{\isacharunderscore}without{\isacharunderscore}def\ \isacommand{by}\isamarkupfalse%
\ fast\isanewline
\ \ \ \ \isacommand{show}\isamarkupfalse%
\ {\isachardoublequoteopen}{\isasymforall}\ X{\isadigit{2}}\ {\isasymin}\ {\isacharquery}Q{\isachardot}\ X{\isadigit{1}}\ {\isasyminter}\ X{\isadigit{2}}\ {\isasymnoteq}\ {\isacharbraceleft}{\isacharbraceright}\ {\isasymlongleftrightarrow}\ X{\isadigit{1}}\ {\isacharequal}\ X{\isadigit{2}}{\isachardoublequoteclose}\ \isanewline
\ \ \ \ \isacommand{proof}\isamarkupfalse%
\isanewline
\ \ \ \ \ \ \isacommand{fix}\isamarkupfalse%
\ X{\isadigit{2}}\ \isacommand{assume}\isamarkupfalse%
\ {\isachardoublequoteopen}X{\isadigit{2}}\ {\isasymin}\ {\isacharquery}Q{\isachardoublequoteclose}\isanewline
\ \ \ \ \ \ \isacommand{then}\isamarkupfalse%
\ \isacommand{obtain}\isamarkupfalse%
\ Z{\isadigit{2}}\ \isakeyword{where}\ Z{\isadigit{2}}{\isacharunderscore}in{\isacharunderscore}P{\isacharcolon}\ {\isachardoublequoteopen}Z{\isadigit{2}}\ {\isasymin}\ P{\isachardoublequoteclose}\ \isakeyword{and}\ Z{\isadigit{2}}{\isacharunderscore}sup{\isacharcolon}\ {\isachardoublequoteopen}X{\isadigit{2}}\ {\isacharequal}\ Z{\isadigit{2}}\ {\isacharminus}\ {\isacharbraceleft}elem{\isacharbraceright}{\isachardoublequoteclose}\isanewline
\ \ \ \ \ \ \ \ \isacommand{by}\isamarkupfalse%
\ {\isacharparenleft}rule\ super{\isacharunderscore}class{\isacharparenright}\isanewline
\ \ \ \ \ \ \isacommand{have}\isamarkupfalse%
\ {\isachardoublequoteopen}X{\isadigit{1}}\ {\isasyminter}\ X{\isadigit{2}}\ {\isasymnoteq}\ {\isacharbraceleft}{\isacharbraceright}\ {\isasymlongrightarrow}\ X{\isadigit{1}}\ {\isacharequal}\ X{\isadigit{2}}{\isachardoublequoteclose}\isanewline
\ \ \ \ \ \ \isacommand{proof}\isamarkupfalse%
\isanewline
\ \ \ \ \ \ \ \ \isacommand{assume}\isamarkupfalse%
\ {\isachardoublequoteopen}X{\isadigit{1}}\ {\isasyminter}\ X{\isadigit{2}}\ {\isasymnoteq}\ {\isacharbraceleft}{\isacharbraceright}{\isachardoublequoteclose}\isanewline
\ \ \ \ \ \ \ \ \isacommand{then}\isamarkupfalse%
\ \isacommand{have}\isamarkupfalse%
\ {\isachardoublequoteopen}Z{\isadigit{1}}\ {\isasyminter}\ Z{\isadigit{2}}\ {\isasymnoteq}\ {\isacharbraceleft}{\isacharbraceright}{\isachardoublequoteclose}\ \isacommand{using}\isamarkupfalse%
\ Z{\isadigit{1}}{\isacharunderscore}sup\ Z{\isadigit{2}}{\isacharunderscore}sup\ \isacommand{by}\isamarkupfalse%
\ fast\isanewline
\ \ \ \ \ \ \ \ \isacommand{then}\isamarkupfalse%
\ \isacommand{have}\isamarkupfalse%
\ {\isachardoublequoteopen}Z{\isadigit{1}}\ {\isacharequal}\ Z{\isadigit{2}}{\isachardoublequoteclose}\ \isacommand{using}\isamarkupfalse%
\ Z{\isadigit{1}}{\isacharunderscore}in{\isacharunderscore}P\ Z{\isadigit{2}}{\isacharunderscore}in{\isacharunderscore}P\ assms\ \isacommand{unfolding}\isamarkupfalse%
\ is{\isacharunderscore}non{\isacharunderscore}overlapping{\isacharunderscore}def\ \isacommand{by}\isamarkupfalse%
\ fast\isanewline
\ \ \ \ \ \ \ \ \isacommand{then}\isamarkupfalse%
\ \isacommand{show}\isamarkupfalse%
\ {\isachardoublequoteopen}X{\isadigit{1}}\ {\isacharequal}\ X{\isadigit{2}}{\isachardoublequoteclose}\ \isacommand{using}\isamarkupfalse%
\ Z{\isadigit{1}}{\isacharunderscore}sup\ Z{\isadigit{2}}{\isacharunderscore}sup\ \isacommand{by}\isamarkupfalse%
\ fast\isanewline
\ \ \ \ \ \ \isacommand{qed}\isamarkupfalse%
\isanewline
\ \ \ \ \ \ \isacommand{moreover}\isamarkupfalse%
\ \isacommand{have}\isamarkupfalse%
\ {\isachardoublequoteopen}X{\isadigit{1}}\ {\isacharequal}\ X{\isadigit{2}}\ {\isasymlongrightarrow}\ X{\isadigit{1}}\ {\isasyminter}\ X{\isadigit{2}}\ {\isasymnoteq}\ {\isacharbraceleft}{\isacharbraceright}{\isachardoublequoteclose}\ \isacommand{using}\isamarkupfalse%
\ X{\isadigit{1}}{\isacharunderscore}non{\isacharunderscore}empty\ \isacommand{by}\isamarkupfalse%
\ auto\isanewline
\ \ \ \ \ \ \isacommand{ultimately}\isamarkupfalse%
\ \isacommand{show}\isamarkupfalse%
\ {\isachardoublequoteopen}{\isacharparenleft}X{\isadigit{1}}\ {\isasyminter}\ X{\isadigit{2}}\ {\isasymnoteq}\ {\isacharbraceleft}{\isacharbraceright}{\isacharparenright}\ {\isasymlongleftrightarrow}\ X{\isadigit{1}}\ {\isacharequal}\ X{\isadigit{2}}{\isachardoublequoteclose}\ \isacommand{by}\isamarkupfalse%
\ blast\isanewline
\ \ \ \ \isacommand{qed}\isamarkupfalse%
\isanewline
\ \ \isacommand{qed}\isamarkupfalse%
\isanewline
\ \ \isacommand{then}\isamarkupfalse%
\ \isacommand{show}\isamarkupfalse%
\ {\isacharquery}thesis\ \isacommand{unfolding}\isamarkupfalse%
\ is{\isacharunderscore}non{\isacharunderscore}overlapping{\isacharunderscore}def\ \isacommand{{\isachardot}}\isamarkupfalse%
\isanewline
\isacommand{qed}\isamarkupfalse%
%
\endisatagproof
{\isafoldproof}%
%
\isadelimproof
%
\endisadelimproof
%
\begin{isamarkuptext}%
\isa{coarser{\isacharunderscore}partitions{\isacharunderscore}with\ elem} is the ``inverse'' of 
  \isa{partition{\isacharunderscore}without\ elem}.%
\end{isamarkuptext}%
\isamarkuptrue%
\isacommand{lemma}\isamarkupfalse%
\ coarser{\isacharunderscore}partitions{\isacharunderscore}inv{\isacharunderscore}without{\isacharcolon}\isanewline
\ \ \isakeyword{fixes}\ elem{\isacharcolon}{\isacharcolon}{\isacharprime}a\isanewline
\ \ \ \ \isakeyword{and}\ P{\isacharcolon}{\isacharcolon}{\isachardoublequoteopen}{\isacharprime}a\ set\ set{\isachardoublequoteclose}\isanewline
\ \ \isakeyword{assumes}\ non{\isacharunderscore}overlapping{\isacharcolon}\ {\isachardoublequoteopen}is{\isacharunderscore}non{\isacharunderscore}overlapping\ P{\isachardoublequoteclose}\isanewline
\ \ \ \ \ \ \isakeyword{and}\ elem{\isacharcolon}\ {\isachardoublequoteopen}elem\ {\isasymin}\ {\isasymUnion}\ P{\isachardoublequoteclose}\ \isanewline
\ \ \isakeyword{shows}\ {\isachardoublequoteopen}P\ {\isasymin}\ coarser{\isacharunderscore}partitions{\isacharunderscore}with\ elem\ {\isacharparenleft}partition{\isacharunderscore}without\ elem\ P{\isacharparenright}{\isachardoublequoteclose}\isanewline
\ \ \ \ {\isacharparenleft}\isakeyword{is}\ {\isachardoublequoteopen}P\ {\isasymin}\ coarser{\isacharunderscore}partitions{\isacharunderscore}with\ elem\ {\isacharquery}Q{\isachardoublequoteclose}{\isacharparenright}\isanewline
%
\isadelimproof
%
\endisadelimproof
%
\isatagproof
\isacommand{proof}\isamarkupfalse%
\ {\isacharminus}\isanewline
\ \ \isacommand{let}\isamarkupfalse%
\ {\isacharquery}remove{\isacharunderscore}elem\ {\isacharequal}\ {\isachardoublequoteopen}{\isasymlambda}X\ {\isachardot}\ X\ {\isacharminus}\ {\isacharbraceleft}elem{\isacharbraceright}{\isachardoublequoteclose}\ \isanewline
\ \ \isacommand{obtain}\isamarkupfalse%
\ Y\ \isanewline
\ \ \ \ \isakeyword{where}\ elem{\isacharunderscore}eq{\isacharunderscore}class{\isacharcolon}\ {\isachardoublequoteopen}elem\ {\isasymin}\ Y{\isachardoublequoteclose}\ \isakeyword{and}\ elem{\isacharunderscore}eq{\isacharunderscore}class{\isacharprime}{\isacharcolon}\ {\isachardoublequoteopen}Y\ {\isasymin}\ P{\isachardoublequoteclose}\ \isacommand{using}\isamarkupfalse%
\ elem\ \isacommand{{\isachardot}{\isachardot}}\isamarkupfalse%
\isanewline
\ \ \isacommand{let}\isamarkupfalse%
\ {\isacharquery}elem{\isacharunderscore}neq{\isacharunderscore}classes\ {\isacharequal}\ {\isachardoublequoteopen}P\ {\isacharminus}\ {\isacharbraceleft}Y{\isacharbraceright}{\isachardoublequoteclose}\ \isanewline
\ \ \isacommand{have}\isamarkupfalse%
\ P{\isacharunderscore}wrt{\isacharunderscore}elem{\isacharcolon}\ {\isachardoublequoteopen}P\ {\isacharequal}\ {\isacharquery}elem{\isacharunderscore}neq{\isacharunderscore}classes\ {\isasymunion}\ {\isacharbraceleft}Y{\isacharbraceright}{\isachardoublequoteclose}\ \isacommand{using}\isamarkupfalse%
\ elem{\isacharunderscore}eq{\isacharunderscore}class{\isacharprime}\ \isacommand{by}\isamarkupfalse%
\ blast\isanewline
\ \ \isacommand{let}\isamarkupfalse%
\ {\isacharquery}elem{\isacharunderscore}eq\ {\isacharequal}\ {\isachardoublequoteopen}Y\ {\isacharminus}\ {\isacharbraceleft}elem{\isacharbraceright}{\isachardoublequoteclose}\ \isanewline
\ \ \isacommand{have}\isamarkupfalse%
\ Y{\isacharunderscore}elem{\isacharunderscore}eq{\isacharcolon}\ {\isachardoublequoteopen}{\isacharquery}remove{\isacharunderscore}elem\ {\isacharbackquote}\ {\isacharbraceleft}Y{\isacharbraceright}\ {\isacharequal}\ {\isacharbraceleft}{\isacharquery}elem{\isacharunderscore}eq{\isacharbraceright}{\isachardoublequoteclose}\ \isacommand{by}\isamarkupfalse%
\ fast\isanewline
\ \ \isanewline
\ \ \isacommand{have}\isamarkupfalse%
\ elem{\isacharunderscore}neq{\isacharunderscore}classes{\isacharunderscore}part{\isacharcolon}\ {\isachardoublequoteopen}is{\isacharunderscore}non{\isacharunderscore}overlapping\ {\isacharquery}elem{\isacharunderscore}neq{\isacharunderscore}classes{\isachardoublequoteclose}\isanewline
\ \ \ \ \isacommand{using}\isamarkupfalse%
\ subset{\isacharunderscore}is{\isacharunderscore}non{\isacharunderscore}overlapping\ non{\isacharunderscore}overlapping\isanewline
\ \ \ \ \isacommand{by}\isamarkupfalse%
\ blast\isanewline
\ \ \isacommand{have}\isamarkupfalse%
\ elem{\isacharunderscore}eq{\isacharunderscore}wrt{\isacharunderscore}P{\isacharcolon}\ {\isachardoublequoteopen}{\isacharquery}elem{\isacharunderscore}eq\ {\isasymin}\ {\isacharquery}remove{\isacharunderscore}elem\ {\isacharbackquote}\ P{\isachardoublequoteclose}\ \isacommand{using}\isamarkupfalse%
\ elem{\isacharunderscore}eq{\isacharunderscore}class{\isacharprime}\ \isacommand{by}\isamarkupfalse%
\ blast\isanewline
\ \ \isanewline
\ \ \isacommand{{\isacharbraceleft}}\isamarkupfalse%
\ \isanewline
\ \ \ \ \isacommand{fix}\isamarkupfalse%
\ W\ \isacommand{assume}\isamarkupfalse%
\ W{\isacharunderscore}eq{\isacharunderscore}class{\isacharcolon}\ {\isachardoublequoteopen}W\ {\isasymin}\ {\isacharquery}elem{\isacharunderscore}neq{\isacharunderscore}classes{\isachardoublequoteclose}\isanewline
\ \ \ \ \isacommand{then}\isamarkupfalse%
\ \isacommand{have}\isamarkupfalse%
\ {\isachardoublequoteopen}elem\ {\isasymnotin}\ W{\isachardoublequoteclose}\isanewline
\ \ \ \ \ \ \isacommand{using}\isamarkupfalse%
\ elem{\isacharunderscore}eq{\isacharunderscore}class\ elem{\isacharunderscore}eq{\isacharunderscore}class{\isacharprime}\ non{\isacharunderscore}overlapping\ is{\isacharunderscore}non{\isacharunderscore}overlapping{\isacharunderscore}def\isanewline
\ \ \ \ \ \ \isacommand{by}\isamarkupfalse%
\ fast\isanewline
\ \ \ \ \isacommand{then}\isamarkupfalse%
\ \isacommand{have}\isamarkupfalse%
\ {\isachardoublequoteopen}{\isacharquery}remove{\isacharunderscore}elem\ W\ {\isacharequal}\ W{\isachardoublequoteclose}\ \isacommand{by}\isamarkupfalse%
\ simp\isanewline
\ \ \isacommand{{\isacharbraceright}}\isamarkupfalse%
\isanewline
\ \ \isacommand{then}\isamarkupfalse%
\ \isacommand{have}\isamarkupfalse%
\ elem{\isacharunderscore}neq{\isacharunderscore}classes{\isacharunderscore}id{\isacharcolon}\ {\isachardoublequoteopen}{\isacharquery}remove{\isacharunderscore}elem\ {\isacharbackquote}\ {\isacharquery}elem{\isacharunderscore}neq{\isacharunderscore}classes\ {\isacharequal}\ {\isacharquery}elem{\isacharunderscore}neq{\isacharunderscore}classes{\isachardoublequoteclose}\ \isacommand{by}\isamarkupfalse%
\ fastforce\isanewline
\isanewline
\ \ \isacommand{have}\isamarkupfalse%
\ Q{\isacharunderscore}unfolded{\isacharcolon}\ {\isachardoublequoteopen}{\isacharquery}Q\ {\isacharequal}\ {\isacharquery}remove{\isacharunderscore}elem\ {\isacharbackquote}\ P\ {\isacharminus}\ {\isacharbraceleft}{\isacharbraceleft}{\isacharbraceright}{\isacharbraceright}{\isachardoublequoteclose}\isanewline
\ \ \ \ \isacommand{unfolding}\isamarkupfalse%
\ partition{\isacharunderscore}without{\isacharunderscore}def\isanewline
\ \ \ \ \isacommand{using}\isamarkupfalse%
\ image{\isacharunderscore}Collect{\isacharunderscore}mem\isanewline
\ \ \ \ \isacommand{by}\isamarkupfalse%
\ blast\isanewline
\ \ \isacommand{also}\isamarkupfalse%
\ \isacommand{have}\isamarkupfalse%
\ {\isachardoublequoteopen}{\isasymdots}\ {\isacharequal}\ {\isacharquery}remove{\isacharunderscore}elem\ {\isacharbackquote}\ {\isacharparenleft}{\isacharquery}elem{\isacharunderscore}neq{\isacharunderscore}classes\ {\isasymunion}\ {\isacharbraceleft}Y{\isacharbraceright}{\isacharparenright}\ {\isacharminus}\ {\isacharbraceleft}{\isacharbraceleft}{\isacharbraceright}{\isacharbraceright}{\isachardoublequoteclose}\ \isacommand{using}\isamarkupfalse%
\ P{\isacharunderscore}wrt{\isacharunderscore}elem\ \isacommand{by}\isamarkupfalse%
\ presburger\isanewline
\ \ \isacommand{also}\isamarkupfalse%
\ \isacommand{have}\isamarkupfalse%
\ {\isachardoublequoteopen}{\isasymdots}\ {\isacharequal}\ {\isacharquery}elem{\isacharunderscore}neq{\isacharunderscore}classes\ {\isasymunion}\ {\isacharbraceleft}{\isacharquery}elem{\isacharunderscore}eq{\isacharbraceright}\ {\isacharminus}\ {\isacharbraceleft}{\isacharbraceleft}{\isacharbraceright}{\isacharbraceright}{\isachardoublequoteclose}\isanewline
\ \ \ \ \isacommand{using}\isamarkupfalse%
\ Y{\isacharunderscore}elem{\isacharunderscore}eq\ elem{\isacharunderscore}neq{\isacharunderscore}classes{\isacharunderscore}id\ image{\isacharunderscore}Un\ \isacommand{by}\isamarkupfalse%
\ metis\isanewline
\ \ \isacommand{finally}\isamarkupfalse%
\ \isacommand{have}\isamarkupfalse%
\ Q{\isacharunderscore}wrt{\isacharunderscore}elem{\isacharcolon}\ {\isachardoublequoteopen}{\isacharquery}Q\ {\isacharequal}\ {\isacharquery}elem{\isacharunderscore}neq{\isacharunderscore}classes\ {\isasymunion}\ {\isacharbraceleft}{\isacharquery}elem{\isacharunderscore}eq{\isacharbraceright}\ {\isacharminus}\ {\isacharbraceleft}{\isacharbraceleft}{\isacharbraceright}{\isacharbraceright}{\isachardoublequoteclose}\ \isacommand{{\isachardot}}\isamarkupfalse%
\isanewline
\isanewline
\ \ \isacommand{have}\isamarkupfalse%
\ {\isachardoublequoteopen}{\isacharquery}elem{\isacharunderscore}eq\ {\isacharequal}\ {\isacharbraceleft}{\isacharbraceright}\ {\isasymor}\ {\isacharquery}elem{\isacharunderscore}eq\ {\isasymnotin}\ P{\isachardoublequoteclose}\isanewline
\ \ \ \ \isacommand{using}\isamarkupfalse%
\ elem{\isacharunderscore}eq{\isacharunderscore}class\ elem{\isacharunderscore}eq{\isacharunderscore}class{\isacharprime}\ non{\isacharunderscore}overlapping\ Diff{\isacharunderscore}Int{\isacharunderscore}distrib{\isadigit{2}}\ Diff{\isacharunderscore}iff\ empty{\isacharunderscore}Diff\ insert{\isacharunderscore}iff\isanewline
\isacommand{unfolding}\isamarkupfalse%
\ is{\isacharunderscore}non{\isacharunderscore}overlapping{\isacharunderscore}def\ \isacommand{by}\isamarkupfalse%
\ metis\isanewline
\ \ \isacommand{then}\isamarkupfalse%
\ \isacommand{have}\isamarkupfalse%
\ {\isachardoublequoteopen}{\isacharquery}elem{\isacharunderscore}eq\ {\isasymnotin}\ P{\isachardoublequoteclose}\isanewline
\ \ \ \ \isacommand{using}\isamarkupfalse%
\ non{\isacharunderscore}overlapping\ no{\isacharunderscore}empty{\isacharunderscore}in{\isacharunderscore}non{\isacharunderscore}overlapping\isanewline
\ \ \ \ \isacommand{by}\isamarkupfalse%
\ metis\isanewline
\ \ \isacommand{then}\isamarkupfalse%
\ \isacommand{have}\isamarkupfalse%
\ elem{\isacharunderscore}neq{\isacharunderscore}classes{\isacharcolon}\ {\isachardoublequoteopen}{\isacharquery}elem{\isacharunderscore}neq{\isacharunderscore}classes\ {\isacharminus}\ {\isacharbraceleft}{\isacharquery}elem{\isacharunderscore}eq{\isacharbraceright}\ {\isacharequal}\ {\isacharquery}elem{\isacharunderscore}neq{\isacharunderscore}classes{\isachardoublequoteclose}\ \isacommand{by}\isamarkupfalse%
\ fastforce\isanewline
\isanewline
\ \ \isacommand{show}\isamarkupfalse%
\ {\isacharquery}thesis\isanewline
\ \ \isacommand{proof}\isamarkupfalse%
\ cases\isanewline
\ \ \ \ \isacommand{assume}\isamarkupfalse%
\ {\isachardoublequoteopen}{\isacharquery}elem{\isacharunderscore}eq\ {\isasymnotin}\ {\isacharquery}Q{\isachardoublequoteclose}\ \isanewline
\ \ \ \ \isacommand{then}\isamarkupfalse%
\ \isacommand{have}\isamarkupfalse%
\ {\isachardoublequoteopen}{\isacharquery}elem{\isacharunderscore}eq\ {\isasymin}\ {\isacharbraceleft}{\isacharbraceleft}{\isacharbraceright}{\isacharbraceright}{\isachardoublequoteclose}\isanewline
\ \ \ \ \ \ \isacommand{using}\isamarkupfalse%
\ elem{\isacharunderscore}eq{\isacharunderscore}wrt{\isacharunderscore}P\ Q{\isacharunderscore}unfolded\isanewline
\ \ \ \ \ \ \isacommand{by}\isamarkupfalse%
\ {\isacharparenleft}metis\ DiffI{\isacharparenright}\isanewline
\ \ \ \ \isacommand{then}\isamarkupfalse%
\ \isacommand{have}\isamarkupfalse%
\ Y{\isacharunderscore}singleton{\isacharcolon}\ {\isachardoublequoteopen}Y\ {\isacharequal}\ {\isacharbraceleft}elem{\isacharbraceright}{\isachardoublequoteclose}\ \isacommand{using}\isamarkupfalse%
\ elem{\isacharunderscore}eq{\isacharunderscore}class\ \isacommand{by}\isamarkupfalse%
\ fast\isanewline
\ \ \ \ \isacommand{then}\isamarkupfalse%
\ \isacommand{have}\isamarkupfalse%
\ {\isachardoublequoteopen}{\isacharquery}Q\ {\isacharequal}\ {\isacharquery}elem{\isacharunderscore}neq{\isacharunderscore}classes\ {\isacharminus}\ {\isacharbraceleft}{\isacharbraceleft}{\isacharbraceright}{\isacharbraceright}{\isachardoublequoteclose}\isanewline
\ \ \ \ \ \ \isacommand{using}\isamarkupfalse%
\ Q{\isacharunderscore}wrt{\isacharunderscore}elem\isanewline
\ \ \ \ \ \ \isacommand{by}\isamarkupfalse%
\ force\isanewline
\ \ \ \ \isacommand{then}\isamarkupfalse%
\ \isacommand{have}\isamarkupfalse%
\ {\isachardoublequoteopen}{\isacharquery}Q\ {\isacharequal}\ {\isacharquery}elem{\isacharunderscore}neq{\isacharunderscore}classes{\isachardoublequoteclose}\isanewline
\ \ \ \ \ \ \isacommand{using}\isamarkupfalse%
\ no{\isacharunderscore}empty{\isacharunderscore}in{\isacharunderscore}non{\isacharunderscore}overlapping\ elem{\isacharunderscore}neq{\isacharunderscore}classes{\isacharunderscore}part\isanewline
\ \ \ \ \ \ \isacommand{by}\isamarkupfalse%
\ blast\isanewline
\ \ \ \ \isacommand{then}\isamarkupfalse%
\ \isacommand{have}\isamarkupfalse%
\ {\isachardoublequoteopen}insert\ {\isacharbraceleft}elem{\isacharbraceright}\ {\isacharquery}Q\ {\isacharequal}\ P{\isachardoublequoteclose}\isanewline
\ \ \ \ \ \ \isacommand{using}\isamarkupfalse%
\ Y{\isacharunderscore}singleton\ elem{\isacharunderscore}eq{\isacharunderscore}class{\isacharprime}\isanewline
\ \ \ \ \ \ \isacommand{by}\isamarkupfalse%
\ fast\isanewline
\ \ \ \ \isacommand{then}\isamarkupfalse%
\ \isacommand{show}\isamarkupfalse%
\ {\isacharquery}thesis\ \isacommand{unfolding}\isamarkupfalse%
\ coarser{\isacharunderscore}partitions{\isacharunderscore}with{\isacharunderscore}def\ \isacommand{by}\isamarkupfalse%
\ auto\isanewline
\ \ \isacommand{next}\isamarkupfalse%
\isanewline
\ \ \ \ \isacommand{assume}\isamarkupfalse%
\ True{\isacharcolon}\ {\isachardoublequoteopen}{\isasymnot}\ {\isacharquery}elem{\isacharunderscore}eq\ {\isasymnotin}\ {\isacharquery}Q{\isachardoublequoteclose}\isanewline
\ \ \ \ \isacommand{hence}\isamarkupfalse%
\ Y{\isacharprime}{\isacharcolon}\ {\isachardoublequoteopen}{\isacharquery}elem{\isacharunderscore}neq{\isacharunderscore}classes\ {\isasymunion}\ {\isacharbraceleft}{\isacharquery}elem{\isacharunderscore}eq{\isacharbraceright}\ {\isacharminus}\ {\isacharbraceleft}{\isacharbraceleft}{\isacharbraceright}{\isacharbraceright}\ {\isacharequal}\ {\isacharquery}elem{\isacharunderscore}neq{\isacharunderscore}classes\ {\isasymunion}\ {\isacharbraceleft}{\isacharquery}elem{\isacharunderscore}eq{\isacharbraceright}{\isachardoublequoteclose}\isanewline
\ \ \ \ \ \ \isacommand{using}\isamarkupfalse%
\ no{\isacharunderscore}empty{\isacharunderscore}in{\isacharunderscore}non{\isacharunderscore}overlapping\ non{\isacharunderscore}overlapping\ non{\isacharunderscore}overlapping{\isacharunderscore}without{\isacharunderscore}is{\isacharunderscore}non{\isacharunderscore}overlapping\isanewline
\ \ \ \ \ \ \isacommand{by}\isamarkupfalse%
\ force\isanewline
\ \ \ \ \isacommand{have}\isamarkupfalse%
\ {\isachardoublequoteopen}insert{\isacharunderscore}into{\isacharunderscore}member\ elem\ {\isacharparenleft}{\isacharbraceleft}{\isacharquery}elem{\isacharunderscore}eq{\isacharbraceright}\ {\isasymunion}\ {\isacharquery}elem{\isacharunderscore}neq{\isacharunderscore}classes{\isacharparenright}\ {\isacharquery}elem{\isacharunderscore}eq\ {\isacharequal}\ insert\ {\isacharparenleft}{\isacharquery}elem{\isacharunderscore}eq\ {\isasymunion}\ {\isacharbraceleft}elem{\isacharbraceright}{\isacharparenright}\ {\isacharparenleft}{\isacharparenleft}{\isacharbraceleft}{\isacharquery}elem{\isacharunderscore}eq{\isacharbraceright}\ {\isasymunion}\ {\isacharquery}elem{\isacharunderscore}neq{\isacharunderscore}classes{\isacharparenright}\ {\isacharminus}\ {\isacharbraceleft}{\isacharquery}elem{\isacharunderscore}eq{\isacharbraceright}{\isacharparenright}{\isachardoublequoteclose}\isanewline
\ \ \ \ \ \ \isacommand{unfolding}\isamarkupfalse%
\ insert{\isacharunderscore}into{\isacharunderscore}member{\isacharunderscore}def\ \isacommand{{\isachardot}{\isachardot}}\isamarkupfalse%
\isanewline
\ \ \ \ \isacommand{also}\isamarkupfalse%
\ \isacommand{have}\isamarkupfalse%
\ {\isachardoublequoteopen}{\isasymdots}\ {\isacharequal}\ {\isacharparenleft}{\isacharbraceleft}{\isacharbraceright}\ {\isasymunion}\ {\isacharquery}elem{\isacharunderscore}neq{\isacharunderscore}classes{\isacharparenright}\ {\isasymunion}\ {\isacharbraceleft}{\isacharquery}elem{\isacharunderscore}eq\ {\isasymunion}\ {\isacharbraceleft}elem{\isacharbraceright}{\isacharbraceright}{\isachardoublequoteclose}\ \isacommand{using}\isamarkupfalse%
\ elem{\isacharunderscore}neq{\isacharunderscore}classes\ \isacommand{by}\isamarkupfalse%
\ force\isanewline
\ \ \ \ \isacommand{also}\isamarkupfalse%
\ \isacommand{have}\isamarkupfalse%
\ {\isachardoublequoteopen}{\isasymdots}\ {\isacharequal}\ {\isacharquery}elem{\isacharunderscore}neq{\isacharunderscore}classes\ {\isasymunion}\ {\isacharbraceleft}Y{\isacharbraceright}{\isachardoublequoteclose}\ \isacommand{using}\isamarkupfalse%
\ elem{\isacharunderscore}eq{\isacharunderscore}class\ \isacommand{by}\isamarkupfalse%
\ blast\isanewline
\ \ \ \ \isacommand{finally}\isamarkupfalse%
\ \isacommand{have}\isamarkupfalse%
\ {\isachardoublequoteopen}insert{\isacharunderscore}into{\isacharunderscore}member\ elem\ {\isacharparenleft}{\isacharbraceleft}{\isacharquery}elem{\isacharunderscore}eq{\isacharbraceright}\ {\isasymunion}\ {\isacharquery}elem{\isacharunderscore}neq{\isacharunderscore}classes{\isacharparenright}\ {\isacharquery}elem{\isacharunderscore}eq\ {\isacharequal}\ {\isacharquery}elem{\isacharunderscore}neq{\isacharunderscore}classes\ {\isasymunion}\ {\isacharbraceleft}Y{\isacharbraceright}{\isachardoublequoteclose}\ \isacommand{{\isachardot}}\isamarkupfalse%
\isanewline
\ \ \ \ \isacommand{then}\isamarkupfalse%
\ \isacommand{have}\isamarkupfalse%
\ {\isachardoublequoteopen}{\isacharquery}elem{\isacharunderscore}neq{\isacharunderscore}classes\ {\isasymunion}\ {\isacharbraceleft}Y{\isacharbraceright}\ {\isacharequal}\ insert{\isacharunderscore}into{\isacharunderscore}member\ elem\ {\isacharquery}Q\ {\isacharquery}elem{\isacharunderscore}eq{\isachardoublequoteclose}\isanewline
\ \ \ \ \ \ \isacommand{using}\isamarkupfalse%
\ Q{\isacharunderscore}wrt{\isacharunderscore}elem\ Y{\isacharprime}\ partition{\isacharunderscore}without{\isacharunderscore}def\isanewline
\ \ \ \ \ \ \isacommand{by}\isamarkupfalse%
\ force\isanewline
\ \ \ \ \isacommand{then}\isamarkupfalse%
\ \isacommand{have}\isamarkupfalse%
\ {\isachardoublequoteopen}{\isacharbraceleft}Y{\isacharbraceright}\ {\isasymunion}\ {\isacharquery}elem{\isacharunderscore}neq{\isacharunderscore}classes\ {\isasymin}\ insert{\isacharunderscore}into{\isacharunderscore}member\ elem\ {\isacharquery}Q\ {\isacharbackquote}\ {\isacharquery}Q{\isachardoublequoteclose}\ \isacommand{using}\isamarkupfalse%
\ True\ \isacommand{by}\isamarkupfalse%
\ blast\isanewline
\ \ \ \ \isacommand{then}\isamarkupfalse%
\ \isacommand{have}\isamarkupfalse%
\ {\isachardoublequoteopen}{\isacharbraceleft}Y{\isacharbraceright}\ {\isasymunion}\ {\isacharquery}elem{\isacharunderscore}neq{\isacharunderscore}classes\ {\isasymin}\ coarser{\isacharunderscore}partitions{\isacharunderscore}with\ elem\ {\isacharquery}Q{\isachardoublequoteclose}\ \isacommand{unfolding}\isamarkupfalse%
\ coarser{\isacharunderscore}partitions{\isacharunderscore}with{\isacharunderscore}def\ \isacommand{by}\isamarkupfalse%
\ simp\isanewline
\ \ \ \ \isacommand{then}\isamarkupfalse%
\ \isacommand{show}\isamarkupfalse%
\ {\isacharquery}thesis\ \isacommand{using}\isamarkupfalse%
\ P{\isacharunderscore}wrt{\isacharunderscore}elem\ \isacommand{by}\isamarkupfalse%
\ simp\isanewline
\ \ \isacommand{qed}\isamarkupfalse%
\isanewline
\isacommand{qed}\isamarkupfalse%
%
\endisatagproof
{\isafoldproof}%
%
\isadelimproof
%
\endisadelimproof
%
\begin{isamarkuptext}%
Given a set \isa{Ps} of partitions, this is intended to compute the set of all coarser
  partitions (given an extension element) of all partitions in \isa{Ps}.%
\end{isamarkuptext}%
\isamarkuptrue%
\isacommand{definition}\isamarkupfalse%
\ all{\isacharunderscore}coarser{\isacharunderscore}partitions{\isacharunderscore}with\ {\isacharcolon}{\isacharcolon}\ {\isachardoublequoteopen}\ {\isacharprime}a\ {\isasymRightarrow}\ {\isacharprime}a\ set\ set\ set\ {\isasymRightarrow}\ {\isacharprime}a\ set\ set\ set{\isachardoublequoteclose}\isanewline
\ \ \ \isakeyword{where}\ {\isachardoublequoteopen}all{\isacharunderscore}coarser{\isacharunderscore}partitions{\isacharunderscore}with\ elem\ Ps\ {\isacharequal}\ {\isasymUnion}\ {\isacharparenleft}coarser{\isacharunderscore}partitions{\isacharunderscore}with\ elem\ {\isacharbackquote}\ Ps{\isacharparenright}{\isachardoublequoteclose}%
\begin{isamarkuptext}%
the list variant of \isa{all{\isacharunderscore}coarser{\isacharunderscore}partitions{\isacharunderscore}with}%
\end{isamarkuptext}%
\isamarkuptrue%
\isacommand{definition}\isamarkupfalse%
\ all{\isacharunderscore}coarser{\isacharunderscore}partitions{\isacharunderscore}with{\isacharunderscore}list\ {\isacharcolon}{\isacharcolon}\ {\isachardoublequoteopen}\ {\isacharprime}a\ {\isasymRightarrow}\ {\isacharprime}a\ set\ list\ list\ {\isasymRightarrow}\ {\isacharprime}a\ set\ list\ list{\isachardoublequoteclose}\isanewline
\ \ \isakeyword{where}\ {\isachardoublequoteopen}all{\isacharunderscore}coarser{\isacharunderscore}partitions{\isacharunderscore}with{\isacharunderscore}list\ elem\ Ps\ {\isacharequal}\ \isanewline
\ \ \ \ \ \ \ \ \ concat\ {\isacharparenleft}map\ {\isacharparenleft}coarser{\isacharunderscore}partitions{\isacharunderscore}with{\isacharunderscore}list\ elem{\isacharparenright}\ Ps{\isacharparenright}{\isachardoublequoteclose}%
\begin{isamarkuptext}%
\isa{all{\isacharunderscore}coarser{\isacharunderscore}partitions{\isacharunderscore}with{\isacharunderscore}list} and \isa{all{\isacharunderscore}coarser{\isacharunderscore}partitions{\isacharunderscore}with} are equivalent.%
\end{isamarkuptext}%
\isamarkuptrue%
\isacommand{lemma}\isamarkupfalse%
\ all{\isacharunderscore}coarser{\isacharunderscore}partitions{\isacharunderscore}with{\isacharunderscore}list{\isacharunderscore}equivalence{\isacharcolon}\isanewline
\ \ \isakeyword{fixes}\ elem{\isacharcolon}{\isacharcolon}{\isacharprime}a\isanewline
\ \ \ \ \isakeyword{and}\ Ps{\isacharcolon}{\isacharcolon}{\isachardoublequoteopen}{\isacharprime}a\ set\ list\ list{\isachardoublequoteclose}\isanewline
\ \ \isakeyword{assumes}\ distinct{\isacharcolon}\ {\isachardoublequoteopen}{\isasymforall}\ P\ {\isasymin}\ set\ Ps\ {\isachardot}\ distinct\ P{\isachardoublequoteclose}\isanewline
\ \ \isakeyword{shows}\ {\isachardoublequoteopen}set\ {\isacharparenleft}map\ set\ {\isacharparenleft}all{\isacharunderscore}coarser{\isacharunderscore}partitions{\isacharunderscore}with{\isacharunderscore}list\ elem\ Ps{\isacharparenright}{\isacharparenright}\ {\isacharequal}\ all{\isacharunderscore}coarser{\isacharunderscore}partitions{\isacharunderscore}with\ elem\ {\isacharparenleft}set\ {\isacharparenleft}map\ set\ Ps{\isacharparenright}{\isacharparenright}{\isachardoublequoteclose}\isanewline
\ \ \ \ {\isacharparenleft}\isakeyword{is}\ {\isachardoublequoteopen}{\isacharquery}list{\isacharunderscore}expr\ {\isacharequal}\ {\isacharquery}set{\isacharunderscore}expr{\isachardoublequoteclose}{\isacharparenright}\isanewline
%
\isadelimproof
%
\endisadelimproof
%
\isatagproof
\isacommand{proof}\isamarkupfalse%
\ {\isacharminus}\isanewline
\ \ \isacommand{have}\isamarkupfalse%
\ {\isachardoublequoteopen}{\isacharquery}list{\isacharunderscore}expr\ {\isacharequal}\ set\ {\isacharparenleft}map\ set\ {\isacharparenleft}concat\ {\isacharparenleft}map\ {\isacharparenleft}coarser{\isacharunderscore}partitions{\isacharunderscore}with{\isacharunderscore}list\ elem{\isacharparenright}\ Ps{\isacharparenright}{\isacharparenright}{\isacharparenright}{\isachardoublequoteclose}\isanewline
\ \ \ \ \isacommand{unfolding}\isamarkupfalse%
\ all{\isacharunderscore}coarser{\isacharunderscore}partitions{\isacharunderscore}with{\isacharunderscore}list{\isacharunderscore}def\ \isacommand{{\isachardot}{\isachardot}}\isamarkupfalse%
\isanewline
\ \ \isacommand{also}\isamarkupfalse%
\ \isacommand{have}\isamarkupfalse%
\ {\isachardoublequoteopen}{\isasymdots}\ {\isacharequal}\ set\ {\isacharbackquote}\ {\isacharparenleft}{\isasymUnion}\ x\ {\isasymin}\ {\isacharparenleft}coarser{\isacharunderscore}partitions{\isacharunderscore}with{\isacharunderscore}list\ elem{\isacharparenright}\ {\isacharbackquote}\ {\isacharparenleft}set\ Ps{\isacharparenright}\ {\isachardot}\ set\ x{\isacharparenright}{\isachardoublequoteclose}\ \isacommand{by}\isamarkupfalse%
\ simp\isanewline
\ \ \isacommand{also}\isamarkupfalse%
\ \isacommand{have}\isamarkupfalse%
\ {\isachardoublequoteopen}{\isasymdots}\ {\isacharequal}\ set\ {\isacharbackquote}\ {\isacharparenleft}{\isasymUnion}\ x\ {\isasymin}\ {\isacharbraceleft}\ coarser{\isacharunderscore}partitions{\isacharunderscore}with{\isacharunderscore}list\ elem\ P\ {\isacharbar}\ P\ {\isachardot}\ P\ {\isasymin}\ set\ Ps\ {\isacharbraceright}\ {\isachardot}\ set\ x{\isacharparenright}{\isachardoublequoteclose}\isanewline
\ \ \ \ \isacommand{by}\isamarkupfalse%
\ {\isacharparenleft}simp\ add{\isacharcolon}\ image{\isacharunderscore}Collect{\isacharunderscore}mem{\isacharparenright}\isanewline
\ \ \isacommand{also}\isamarkupfalse%
\ \isacommand{have}\isamarkupfalse%
\ {\isachardoublequoteopen}{\isasymdots}\ {\isacharequal}\ {\isasymUnion}\ {\isacharbraceleft}\ set\ {\isacharparenleft}map\ set\ {\isacharparenleft}coarser{\isacharunderscore}partitions{\isacharunderscore}with{\isacharunderscore}list\ elem\ P{\isacharparenright}{\isacharparenright}\ {\isacharbar}\ P\ {\isachardot}\ P\ {\isasymin}\ set\ Ps\ {\isacharbraceright}{\isachardoublequoteclose}\ \isacommand{by}\isamarkupfalse%
\ auto\isanewline
\ \ \isacommand{also}\isamarkupfalse%
\ \isacommand{have}\isamarkupfalse%
\ {\isachardoublequoteopen}{\isasymdots}\ {\isacharequal}\ {\isasymUnion}\ {\isacharbraceleft}\ coarser{\isacharunderscore}partitions{\isacharunderscore}with\ elem\ {\isacharparenleft}set\ P{\isacharparenright}\ {\isacharbar}\ P\ {\isachardot}\ P\ {\isasymin}\ set\ Ps\ {\isacharbraceright}{\isachardoublequoteclose}\isanewline
\ \ \ \ \isacommand{using}\isamarkupfalse%
\ distinct\ coarser{\isacharunderscore}partitions{\isacharunderscore}with{\isacharunderscore}list{\isacharunderscore}equivalence\ \isacommand{by}\isamarkupfalse%
\ fast\isanewline
\ \ \isacommand{also}\isamarkupfalse%
\ \isacommand{have}\isamarkupfalse%
\ {\isachardoublequoteopen}{\isasymdots}\ {\isacharequal}\ {\isasymUnion}\ {\isacharparenleft}coarser{\isacharunderscore}partitions{\isacharunderscore}with\ elem\ {\isacharbackquote}\ {\isacharparenleft}set\ {\isacharbackquote}\ {\isacharparenleft}set\ Ps{\isacharparenright}{\isacharparenright}{\isacharparenright}{\isachardoublequoteclose}\ \isacommand{by}\isamarkupfalse%
\ {\isacharparenleft}simp\ add{\isacharcolon}\ image{\isacharunderscore}Collect{\isacharunderscore}mem{\isacharparenright}\isanewline
\ \ \isacommand{also}\isamarkupfalse%
\ \isacommand{have}\isamarkupfalse%
\ {\isachardoublequoteopen}{\isasymdots}\ {\isacharequal}\ {\isasymUnion}\ {\isacharparenleft}coarser{\isacharunderscore}partitions{\isacharunderscore}with\ elem\ {\isacharbackquote}\ {\isacharparenleft}set\ {\isacharparenleft}map\ set\ Ps{\isacharparenright}{\isacharparenright}{\isacharparenright}{\isachardoublequoteclose}\ \isacommand{by}\isamarkupfalse%
\ simp\isanewline
\ \ \isacommand{also}\isamarkupfalse%
\ \isacommand{have}\isamarkupfalse%
\ {\isachardoublequoteopen}{\isasymdots}\ {\isacharequal}\ {\isacharquery}set{\isacharunderscore}expr{\isachardoublequoteclose}\ \isacommand{unfolding}\isamarkupfalse%
\ all{\isacharunderscore}coarser{\isacharunderscore}partitions{\isacharunderscore}with{\isacharunderscore}def\ \isacommand{{\isachardot}{\isachardot}}\isamarkupfalse%
\isanewline
\ \ \isacommand{finally}\isamarkupfalse%
\ \isacommand{show}\isamarkupfalse%
\ {\isacharquery}thesis\ \isacommand{{\isachardot}}\isamarkupfalse%
\isanewline
\isacommand{qed}\isamarkupfalse%
%
\endisatagproof
{\isafoldproof}%
%
\isadelimproof
%
\endisadelimproof
%
\begin{isamarkuptext}%
all partitions of a set (given as list) in form of a set%
\end{isamarkuptext}%
\isamarkuptrue%
\isacommand{fun}\isamarkupfalse%
\ all{\isacharunderscore}partitions{\isacharunderscore}set\ {\isacharcolon}{\isacharcolon}\ {\isachardoublequoteopen}{\isacharprime}a\ list\ {\isasymRightarrow}\ {\isacharprime}a\ set\ set\ set{\isachardoublequoteclose}\isanewline
\ \ \isakeyword{where}\ \isanewline
\ \ \ {\isachardoublequoteopen}all{\isacharunderscore}partitions{\isacharunderscore}set\ {\isacharbrackleft}{\isacharbrackright}\ {\isacharequal}\ {\isacharbraceleft}{\isacharbraceleft}{\isacharbraceright}{\isacharbraceright}{\isachardoublequoteclose}\ {\isacharbar}\isanewline
\ \ \ {\isachardoublequoteopen}all{\isacharunderscore}partitions{\isacharunderscore}set\ {\isacharparenleft}e\ {\isacharhash}\ X{\isacharparenright}\ {\isacharequal}\ all{\isacharunderscore}coarser{\isacharunderscore}partitions{\isacharunderscore}with\ e\ {\isacharparenleft}all{\isacharunderscore}partitions{\isacharunderscore}set\ X{\isacharparenright}{\isachardoublequoteclose}%
\begin{isamarkuptext}%
all partitions of a set (given as list) in form of a list%
\end{isamarkuptext}%
\isamarkuptrue%
\isacommand{fun}\isamarkupfalse%
\ all{\isacharunderscore}partitions{\isacharunderscore}list\ {\isacharcolon}{\isacharcolon}\ {\isachardoublequoteopen}{\isacharprime}a\ list\ {\isasymRightarrow}\ {\isacharprime}a\ set\ list\ list{\isachardoublequoteclose}\isanewline
\ \ \isakeyword{where}\ \isanewline
\ \ \ {\isachardoublequoteopen}all{\isacharunderscore}partitions{\isacharunderscore}list\ {\isacharbrackleft}{\isacharbrackright}\ {\isacharequal}\ {\isacharbrackleft}{\isacharbrackleft}{\isacharbrackright}{\isacharbrackright}{\isachardoublequoteclose}\ {\isacharbar}\isanewline
\ \ \ {\isachardoublequoteopen}all{\isacharunderscore}partitions{\isacharunderscore}list\ {\isacharparenleft}e\ {\isacharhash}\ X{\isacharparenright}\ {\isacharequal}\ all{\isacharunderscore}coarser{\isacharunderscore}partitions{\isacharunderscore}with{\isacharunderscore}list\ e\ {\isacharparenleft}all{\isacharunderscore}partitions{\isacharunderscore}list\ X{\isacharparenright}{\isachardoublequoteclose}%
\begin{isamarkuptext}%
A list of partitions coarser than a given partition in list representation (constructed
  with \isa{coarser{\isacharunderscore}partitions{\isacharunderscore}with} is distinct under certain conditions.%
\end{isamarkuptext}%
\isamarkuptrue%
\isacommand{lemma}\isamarkupfalse%
\ coarser{\isacharunderscore}partitions{\isacharunderscore}with{\isacharunderscore}list{\isacharunderscore}distinct{\isacharcolon}\isanewline
\ \ \isakeyword{fixes}\ ps\isanewline
\ \ \isakeyword{assumes}\ ps{\isacharunderscore}coarser{\isacharcolon}\ {\isachardoublequoteopen}ps\ {\isasymin}\ set\ {\isacharparenleft}coarser{\isacharunderscore}partitions{\isacharunderscore}with{\isacharunderscore}list\ x\ Q{\isacharparenright}{\isachardoublequoteclose}\isanewline
\ \ \ \ \ \ \isakeyword{and}\ distinct{\isacharcolon}\ {\isachardoublequoteopen}distinct\ Q{\isachardoublequoteclose}\isanewline
\ \ \ \ \ \ \isakeyword{and}\ partition{\isacharcolon}\ {\isachardoublequoteopen}is{\isacharunderscore}non{\isacharunderscore}overlapping\ {\isacharparenleft}set\ Q{\isacharparenright}{\isachardoublequoteclose}\isanewline
\ \ \ \ \ \ \isakeyword{and}\ new{\isacharcolon}\ {\isachardoublequoteopen}{\isacharbraceleft}x{\isacharbraceright}\ {\isasymnotin}\ set\ Q{\isachardoublequoteclose}\isanewline
\ \ \isakeyword{shows}\ {\isachardoublequoteopen}distinct\ ps{\isachardoublequoteclose}\isanewline
%
\isadelimproof
%
\endisadelimproof
%
\isatagproof
\isacommand{proof}\isamarkupfalse%
\ {\isacharminus}\isanewline
\ \ \isacommand{have}\isamarkupfalse%
\ {\isachardoublequoteopen}set\ {\isacharparenleft}coarser{\isacharunderscore}partitions{\isacharunderscore}with{\isacharunderscore}list\ x\ Q{\isacharparenright}\ {\isacharequal}\ insert\ {\isacharparenleft}{\isacharbraceleft}x{\isacharbraceright}\ {\isacharhash}\ Q{\isacharparenright}\ {\isacharparenleft}set\ {\isacharparenleft}map\ {\isacharparenleft}insert{\isacharunderscore}into{\isacharunderscore}member{\isacharunderscore}list\ x\ Q{\isacharparenright}\ Q{\isacharparenright}{\isacharparenright}{\isachardoublequoteclose}\isanewline
\ \ \ \ \isacommand{unfolding}\isamarkupfalse%
\ coarser{\isacharunderscore}partitions{\isacharunderscore}with{\isacharunderscore}list{\isacharunderscore}def\ \isacommand{by}\isamarkupfalse%
\ simp\isanewline
\ \ \isacommand{with}\isamarkupfalse%
\ ps{\isacharunderscore}coarser\ \isacommand{have}\isamarkupfalse%
\ {\isachardoublequoteopen}ps\ {\isasymin}\ insert\ {\isacharparenleft}{\isacharbraceleft}x{\isacharbraceright}\ {\isacharhash}\ Q{\isacharparenright}\ {\isacharparenleft}set\ {\isacharparenleft}map\ {\isacharparenleft}{\isacharparenleft}insert{\isacharunderscore}into{\isacharunderscore}member{\isacharunderscore}list\ x\ Q{\isacharparenright}{\isacharparenright}\ Q{\isacharparenright}{\isacharparenright}{\isachardoublequoteclose}\ \isacommand{by}\isamarkupfalse%
\ blast\isanewline
\ \ \isacommand{then}\isamarkupfalse%
\ \isacommand{show}\isamarkupfalse%
\ {\isacharquery}thesis\isanewline
\ \ \isacommand{proof}\isamarkupfalse%
\isanewline
\ \ \ \ \isacommand{assume}\isamarkupfalse%
\ {\isachardoublequoteopen}ps\ {\isacharequal}\ {\isacharbraceleft}x{\isacharbraceright}\ {\isacharhash}\ Q{\isachardoublequoteclose}\isanewline
\ \ \ \ \isacommand{with}\isamarkupfalse%
\ distinct\ \isakeyword{and}\ new\ \isacommand{show}\isamarkupfalse%
\ {\isacharquery}thesis\ \isacommand{by}\isamarkupfalse%
\ simp\isanewline
\ \ \isacommand{next}\isamarkupfalse%
\isanewline
\ \ \ \ \isacommand{assume}\isamarkupfalse%
\ {\isachardoublequoteopen}ps\ {\isasymin}\ set\ {\isacharparenleft}map\ {\isacharparenleft}insert{\isacharunderscore}into{\isacharunderscore}member{\isacharunderscore}list\ x\ Q{\isacharparenright}\ Q{\isacharparenright}{\isachardoublequoteclose}\isanewline
\ \ \ \ \isacommand{then}\isamarkupfalse%
\ \isacommand{obtain}\isamarkupfalse%
\ X\ \isakeyword{where}\ X{\isacharunderscore}in{\isacharunderscore}Q{\isacharcolon}\ {\isachardoublequoteopen}X\ {\isasymin}\ set\ Q{\isachardoublequoteclose}\ \isakeyword{and}\ ps{\isacharunderscore}insert{\isacharcolon}\ {\isachardoublequoteopen}ps\ {\isacharequal}\ insert{\isacharunderscore}into{\isacharunderscore}member{\isacharunderscore}list\ x\ Q\ X{\isachardoublequoteclose}\ \isacommand{by}\isamarkupfalse%
\ auto\isanewline
\ \ \ \ \isacommand{from}\isamarkupfalse%
\ ps{\isacharunderscore}insert\ \isacommand{have}\isamarkupfalse%
\ {\isachardoublequoteopen}ps\ {\isacharequal}\ {\isacharparenleft}X\ {\isasymunion}\ {\isacharbraceleft}x{\isacharbraceright}{\isacharparenright}\ {\isacharhash}\ {\isacharparenleft}remove{\isadigit{1}}\ X\ Q{\isacharparenright}{\isachardoublequoteclose}\ \isacommand{unfolding}\isamarkupfalse%
\ insert{\isacharunderscore}into{\isacharunderscore}member{\isacharunderscore}list{\isacharunderscore}def\ \isacommand{{\isachardot}}\isamarkupfalse%
\isanewline
\ \ \ \ \isacommand{also}\isamarkupfalse%
\ \isacommand{have}\isamarkupfalse%
\ {\isachardoublequoteopen}{\isasymdots}\ {\isacharequal}\ {\isacharparenleft}X\ {\isasymunion}\ {\isacharbraceleft}x{\isacharbraceright}{\isacharparenright}\ {\isacharhash}\ {\isacharparenleft}removeAll\ X\ Q{\isacharparenright}{\isachardoublequoteclose}\ \isacommand{using}\isamarkupfalse%
\ distinct\ \isacommand{by}\isamarkupfalse%
\ {\isacharparenleft}metis\ distinct{\isacharunderscore}remove{\isadigit{1}}{\isacharunderscore}removeAll{\isacharparenright}\isanewline
\ \ \ \ \isacommand{finally}\isamarkupfalse%
\ \isacommand{have}\isamarkupfalse%
\ ps{\isacharunderscore}list{\isacharcolon}\ {\isachardoublequoteopen}ps\ {\isacharequal}\ {\isacharparenleft}X\ {\isasymunion}\ {\isacharbraceleft}x{\isacharbraceright}{\isacharparenright}\ {\isacharhash}\ {\isacharparenleft}removeAll\ X\ Q{\isacharparenright}{\isachardoublequoteclose}\ \isacommand{{\isachardot}}\isamarkupfalse%
\isanewline
\ \ \ \ \isanewline
\ \ \ \ \isacommand{have}\isamarkupfalse%
\ distinct{\isacharunderscore}tl{\isacharcolon}\ {\isachardoublequoteopen}X\ {\isasymunion}\ {\isacharbraceleft}x{\isacharbraceright}\ {\isasymnotin}\ set\ {\isacharparenleft}removeAll\ X\ Q{\isacharparenright}{\isachardoublequoteclose}\isanewline
\ \ \ \ \isacommand{proof}\isamarkupfalse%
\isanewline
\ \ \ \ \ \ \isacommand{from}\isamarkupfalse%
\ partition\ \isacommand{have}\isamarkupfalse%
\ partition{\isacharprime}{\isacharcolon}\ {\isachardoublequoteopen}{\isasymforall}x{\isasymin}set\ Q{\isachardot}\ {\isasymforall}y{\isasymin}set\ Q{\isachardot}\ {\isacharparenleft}x\ {\isasyminter}\ y\ {\isasymnoteq}\ {\isacharbraceleft}{\isacharbraceright}{\isacharparenright}\ {\isacharequal}\ {\isacharparenleft}x\ {\isacharequal}\ y{\isacharparenright}{\isachardoublequoteclose}\ \isacommand{unfolding}\isamarkupfalse%
\ is{\isacharunderscore}non{\isacharunderscore}overlapping{\isacharunderscore}def\ \isacommand{{\isachardot}}\isamarkupfalse%
\isanewline
\ \ \ \ \ \ \isacommand{assume}\isamarkupfalse%
\ {\isachardoublequoteopen}X\ {\isasymunion}\ {\isacharbraceleft}x{\isacharbraceright}\ {\isasymin}\ set\ {\isacharparenleft}removeAll\ X\ Q{\isacharparenright}{\isachardoublequoteclose}\isanewline
\ \ \ \ \ \ \isacommand{with}\isamarkupfalse%
\ X{\isacharunderscore}in{\isacharunderscore}Q\ partition\ \isacommand{show}\isamarkupfalse%
\ False\ \isacommand{by}\isamarkupfalse%
\ {\isacharparenleft}metis\ partition{\isacharprime}\ inf{\isacharunderscore}sup{\isacharunderscore}absorb\ member{\isacharunderscore}remove\ no{\isacharunderscore}empty{\isacharunderscore}in{\isacharunderscore}non{\isacharunderscore}overlapping\ remove{\isacharunderscore}code{\isacharparenleft}{\isadigit{1}}{\isacharparenright}{\isacharparenright}\isanewline
\ \ \ \ \isacommand{qed}\isamarkupfalse%
\isanewline
\ \ \ \ \isacommand{with}\isamarkupfalse%
\ ps{\isacharunderscore}list\ distinct\ \isacommand{show}\isamarkupfalse%
\ {\isacharquery}thesis\ \isacommand{by}\isamarkupfalse%
\ {\isacharparenleft}metis\ {\isacharparenleft}full{\isacharunderscore}types{\isacharparenright}\ distinct{\isachardot}simps{\isacharparenleft}{\isadigit{2}}{\isacharparenright}\ distinct{\isacharunderscore}removeAll{\isacharparenright}\isanewline
\ \ \isacommand{qed}\isamarkupfalse%
\isanewline
\isacommand{qed}\isamarkupfalse%
%
\endisatagproof
{\isafoldproof}%
%
\isadelimproof
%
\endisadelimproof
%
\begin{isamarkuptext}%
The classical definition \isa{all{\isacharunderscore}partitions} and the algorithmic (constructive) definition \isa{all{\isacharunderscore}partitions{\isacharunderscore}list} are equivalent.%
\end{isamarkuptext}%
\isamarkuptrue%
\isacommand{lemma}\isamarkupfalse%
\ all{\isacharunderscore}partitions{\isacharunderscore}equivalence{\isacharprime}{\isacharcolon}\isanewline
\ \ \isakeyword{fixes}\ xs{\isacharcolon}{\isacharcolon}{\isachardoublequoteopen}{\isacharprime}a\ list{\isachardoublequoteclose}\isanewline
\ \ \isakeyword{shows}\ {\isachardoublequoteopen}distinct\ xs\ {\isasymLongrightarrow}\ \isanewline
\ \ \ \ \ \ \ \ \ {\isacharparenleft}{\isacharparenleft}set\ {\isacharparenleft}map\ set\ {\isacharparenleft}all{\isacharunderscore}partitions{\isacharunderscore}list\ xs{\isacharparenright}{\isacharparenright}\ {\isacharequal}\ \isanewline
\ \ \ \ \ \ \ \ \ \ all{\isacharunderscore}partitions\ {\isacharparenleft}set\ xs{\isacharparenright}{\isacharparenright}\ {\isasymand}\ {\isacharparenleft}{\isasymforall}\ ps\ {\isasymin}\ set\ {\isacharparenleft}all{\isacharunderscore}partitions{\isacharunderscore}list\ xs{\isacharparenright}\ {\isachardot}\ distinct\ ps{\isacharparenright}{\isacharparenright}{\isachardoublequoteclose}\isanewline
%
\isadelimproof
%
\endisadelimproof
%
\isatagproof
\isacommand{proof}\isamarkupfalse%
\ {\isacharparenleft}induct\ xs{\isacharparenright}\isanewline
\ \ \isacommand{case}\isamarkupfalse%
\ Nil\isanewline
\ \ \isacommand{have}\isamarkupfalse%
\ {\isachardoublequoteopen}set\ {\isacharparenleft}map\ set\ {\isacharparenleft}all{\isacharunderscore}partitions{\isacharunderscore}list\ {\isacharbrackleft}{\isacharbrackright}{\isacharparenright}{\isacharparenright}\ {\isacharequal}\ all{\isacharunderscore}partitions\ {\isacharparenleft}set\ {\isacharbrackleft}{\isacharbrackright}{\isacharparenright}{\isachardoublequoteclose}\isanewline
\ \ \ \ \isacommand{unfolding}\isamarkupfalse%
\ List{\isachardot}set{\isacharunderscore}simps{\isacharparenleft}{\isadigit{1}}{\isacharparenright}\ emptyset{\isacharunderscore}part{\isacharunderscore}emptyset{\isadigit{3}}\ \isacommand{by}\isamarkupfalse%
\ simp\isanewline
\ \ \ \ \isanewline
\ \ \isacommand{moreover}\isamarkupfalse%
\ \isacommand{have}\isamarkupfalse%
\ {\isachardoublequoteopen}{\isasymforall}\ ps\ {\isasymin}\ set\ {\isacharparenleft}all{\isacharunderscore}partitions{\isacharunderscore}list\ {\isacharbrackleft}{\isacharbrackright}{\isacharparenright}\ {\isachardot}\ distinct\ ps{\isachardoublequoteclose}\ \isacommand{by}\isamarkupfalse%
\ fastforce\isanewline
\ \ \isacommand{ultimately}\isamarkupfalse%
\ \isacommand{show}\isamarkupfalse%
\ {\isacharquery}case\ \isacommand{{\isachardot}{\isachardot}}\isamarkupfalse%
\isanewline
\isacommand{next}\isamarkupfalse%
\isanewline
\ \ \isacommand{case}\isamarkupfalse%
\ {\isacharparenleft}Cons\ x\ xs{\isacharparenright}\isanewline
\ \ \isacommand{from}\isamarkupfalse%
\ Cons{\isachardot}prems\ Cons{\isachardot}hyps\isanewline
\ \ \ \ \isacommand{have}\isamarkupfalse%
\ hyp{\isacharunderscore}equiv{\isacharcolon}\ {\isachardoublequoteopen}set\ {\isacharparenleft}map\ set\ {\isacharparenleft}all{\isacharunderscore}partitions{\isacharunderscore}list\ xs{\isacharparenright}{\isacharparenright}\ {\isacharequal}\ all{\isacharunderscore}partitions\ {\isacharparenleft}set\ xs{\isacharparenright}{\isachardoublequoteclose}\ \isacommand{by}\isamarkupfalse%
\ simp\isanewline
\ \ \isacommand{from}\isamarkupfalse%
\ Cons{\isachardot}prems\ Cons{\isachardot}hyps\isanewline
\ \ \ \ \isacommand{have}\isamarkupfalse%
\ hyp{\isacharunderscore}distinct{\isacharcolon}\ {\isachardoublequoteopen}{\isasymforall}\ ps\ {\isasymin}\ set\ {\isacharparenleft}all{\isacharunderscore}partitions{\isacharunderscore}list\ xs{\isacharparenright}\ {\isachardot}\ distinct\ ps{\isachardoublequoteclose}\ \isacommand{by}\isamarkupfalse%
\ simp\isanewline
\isanewline
\ \ \isacommand{have}\isamarkupfalse%
\ distinct{\isacharunderscore}xs{\isacharcolon}\ {\isachardoublequoteopen}distinct\ xs{\isachardoublequoteclose}\ \isacommand{using}\isamarkupfalse%
\ Cons{\isachardot}prems\ \isacommand{by}\isamarkupfalse%
\ simp\isanewline
\ \ \isacommand{have}\isamarkupfalse%
\ x{\isacharunderscore}notin{\isacharunderscore}xs{\isacharcolon}\ {\isachardoublequoteopen}x\ {\isasymnotin}\ set\ xs{\isachardoublequoteclose}\ \isacommand{using}\isamarkupfalse%
\ Cons{\isachardot}prems\ \isacommand{by}\isamarkupfalse%
\ simp\isanewline
\ \ \isanewline
\ \ \isacommand{have}\isamarkupfalse%
\ {\isachardoublequoteopen}set\ {\isacharparenleft}map\ set\ {\isacharparenleft}all{\isacharunderscore}partitions{\isacharunderscore}list\ {\isacharparenleft}x\ {\isacharhash}\ xs{\isacharparenright}{\isacharparenright}{\isacharparenright}\ {\isacharequal}\ all{\isacharunderscore}partitions\ {\isacharparenleft}set\ {\isacharparenleft}x\ {\isacharhash}\ xs{\isacharparenright}{\isacharparenright}{\isachardoublequoteclose}\isanewline
\ \ \isacommand{proof}\isamarkupfalse%
\ {\isacharparenleft}rule\ equalitySubsetI{\isacharparenright}\ \isanewline
\ \ \ \ \isacommand{fix}\isamarkupfalse%
\ P{\isacharcolon}{\isacharcolon}{\isachardoublequoteopen}{\isacharprime}a\ set\ set{\isachardoublequoteclose}\ \isanewline
\ \ \ \ \isacommand{let}\isamarkupfalse%
\ {\isacharquery}P{\isacharunderscore}without{\isacharunderscore}x\ {\isacharequal}\ {\isachardoublequoteopen}partition{\isacharunderscore}without\ x\ P{\isachardoublequoteclose}\isanewline
\ \ \ \ \isacommand{have}\isamarkupfalse%
\ P{\isacharunderscore}partitions{\isacharunderscore}exc{\isacharunderscore}x{\isacharcolon}\ {\isachardoublequoteopen}{\isasymUnion}\ {\isacharquery}P{\isacharunderscore}without{\isacharunderscore}x\ {\isacharequal}\ {\isasymUnion}\ P\ {\isacharminus}\ {\isacharbraceleft}x{\isacharbraceright}{\isachardoublequoteclose}\ \isacommand{using}\isamarkupfalse%
\ partition{\isacharunderscore}without{\isacharunderscore}covers\ \isacommand{{\isachardot}}\isamarkupfalse%
\isanewline
\isanewline
\ \ \ \ \isacommand{assume}\isamarkupfalse%
\ {\isachardoublequoteopen}P\ {\isasymin}\ all{\isacharunderscore}partitions\ {\isacharparenleft}set\ {\isacharparenleft}x\ {\isacharhash}\ xs{\isacharparenright}{\isacharparenright}{\isachardoublequoteclose}\isanewline
\ \ \ \ \isacommand{then}\isamarkupfalse%
\ \isacommand{have}\isamarkupfalse%
\ is{\isacharunderscore}partition{\isacharunderscore}of{\isacharcolon}\ {\isachardoublequoteopen}P\ partitions\ {\isacharparenleft}set\ {\isacharparenleft}x\ {\isacharhash}\ xs{\isacharparenright}{\isacharparenright}{\isachardoublequoteclose}\ \isacommand{unfolding}\isamarkupfalse%
\ all{\isacharunderscore}partitions{\isacharunderscore}def\ \isacommand{{\isachardot}{\isachardot}}\isamarkupfalse%
\isanewline
\ \ \ \ \isacommand{then}\isamarkupfalse%
\ \isacommand{have}\isamarkupfalse%
\ is{\isacharunderscore}non{\isacharunderscore}overlapping{\isacharcolon}\ {\isachardoublequoteopen}is{\isacharunderscore}non{\isacharunderscore}overlapping\ P{\isachardoublequoteclose}\ \isacommand{unfolding}\isamarkupfalse%
\ is{\isacharunderscore}partition{\isacharunderscore}of{\isacharunderscore}def\ \isacommand{by}\isamarkupfalse%
\ simp\isanewline
\ \ \ \ \isacommand{from}\isamarkupfalse%
\ is{\isacharunderscore}partition{\isacharunderscore}of\ \isacommand{have}\isamarkupfalse%
\ P{\isacharunderscore}covers{\isacharcolon}\ {\isachardoublequoteopen}{\isasymUnion}\ P\ {\isacharequal}\ set\ {\isacharparenleft}x\ {\isacharhash}\ xs{\isacharparenright}{\isachardoublequoteclose}\ \isacommand{unfolding}\isamarkupfalse%
\ is{\isacharunderscore}partition{\isacharunderscore}of{\isacharunderscore}def\ \isacommand{by}\isamarkupfalse%
\ simp\isanewline
\isanewline
\ \ \ \ \isacommand{have}\isamarkupfalse%
\ {\isachardoublequoteopen}{\isacharquery}P{\isacharunderscore}without{\isacharunderscore}x\ partitions\ {\isacharparenleft}set\ xs{\isacharparenright}{\isachardoublequoteclose}\isanewline
\ \ \ \ \ \ \isacommand{unfolding}\isamarkupfalse%
\ is{\isacharunderscore}partition{\isacharunderscore}of{\isacharunderscore}def\isanewline
\ \ \ \ \ \ \isacommand{using}\isamarkupfalse%
\ is{\isacharunderscore}non{\isacharunderscore}overlapping\ non{\isacharunderscore}overlapping{\isacharunderscore}without{\isacharunderscore}is{\isacharunderscore}non{\isacharunderscore}overlapping\ partition{\isacharunderscore}without{\isacharunderscore}covers\ P{\isacharunderscore}covers\ x{\isacharunderscore}notin{\isacharunderscore}xs\isanewline
\ \ \ \ \ \ \isacommand{by}\isamarkupfalse%
\ {\isacharparenleft}metis\ Diff{\isacharunderscore}insert{\isacharunderscore}absorb\ List{\isachardot}set{\isacharunderscore}simps{\isacharparenleft}{\isadigit{2}}{\isacharparenright}{\isacharparenright}\isanewline
\ \ \ \ \isacommand{with}\isamarkupfalse%
\ hyp{\isacharunderscore}equiv\ \isacommand{have}\isamarkupfalse%
\ p{\isacharunderscore}list{\isacharcolon}\ {\isachardoublequoteopen}{\isacharquery}P{\isacharunderscore}without{\isacharunderscore}x\ {\isasymin}\ set\ {\isacharparenleft}map\ set\ {\isacharparenleft}all{\isacharunderscore}partitions{\isacharunderscore}list\ xs{\isacharparenright}{\isacharparenright}{\isachardoublequoteclose}\isanewline
\ \ \ \ \ \ \isacommand{unfolding}\isamarkupfalse%
\ all{\isacharunderscore}partitions{\isacharunderscore}def\ \isacommand{by}\isamarkupfalse%
\ fast\isanewline
\ \ \ \ \isacommand{have}\isamarkupfalse%
\ {\isachardoublequoteopen}P\ {\isasymin}\ coarser{\isacharunderscore}partitions{\isacharunderscore}with\ x\ {\isacharquery}P{\isacharunderscore}without{\isacharunderscore}x{\isachardoublequoteclose}\isanewline
\ \ \ \ \ \ \isacommand{using}\isamarkupfalse%
\ coarser{\isacharunderscore}partitions{\isacharunderscore}inv{\isacharunderscore}without\ is{\isacharunderscore}non{\isacharunderscore}overlapping\ P{\isacharunderscore}covers\isanewline
\ \ \ \ \ \ \isacommand{by}\isamarkupfalse%
\ {\isacharparenleft}metis\ List{\isachardot}set{\isacharunderscore}simps{\isacharparenleft}{\isadigit{2}}{\isacharparenright}\ insertI{\isadigit{1}}{\isacharparenright}\isanewline
\ \ \ \ \isacommand{then}\isamarkupfalse%
\ \isacommand{have}\isamarkupfalse%
\ {\isachardoublequoteopen}P\ {\isasymin}\ {\isasymUnion}\ {\isacharparenleft}coarser{\isacharunderscore}partitions{\isacharunderscore}with\ x\ {\isacharbackquote}\ set\ {\isacharparenleft}map\ set\ {\isacharparenleft}all{\isacharunderscore}partitions{\isacharunderscore}list\ xs{\isacharparenright}{\isacharparenright}{\isacharparenright}{\isachardoublequoteclose}\isanewline
\ \ \ \ \ \ \isacommand{using}\isamarkupfalse%
\ p{\isacharunderscore}list\ \isacommand{by}\isamarkupfalse%
\ blast\isanewline
\ \ \ \ \isacommand{then}\isamarkupfalse%
\ \isacommand{have}\isamarkupfalse%
\ {\isachardoublequoteopen}P\ {\isasymin}\ all{\isacharunderscore}coarser{\isacharunderscore}partitions{\isacharunderscore}with\ x\ {\isacharparenleft}set\ {\isacharparenleft}map\ set\ {\isacharparenleft}all{\isacharunderscore}partitions{\isacharunderscore}list\ xs{\isacharparenright}{\isacharparenright}{\isacharparenright}{\isachardoublequoteclose}\isanewline
\ \ \ \ \ \ \isacommand{unfolding}\isamarkupfalse%
\ all{\isacharunderscore}coarser{\isacharunderscore}partitions{\isacharunderscore}with{\isacharunderscore}def\ \isacommand{by}\isamarkupfalse%
\ fast\isanewline
\ \ \ \ \isacommand{then}\isamarkupfalse%
\ \isacommand{show}\isamarkupfalse%
\ {\isachardoublequoteopen}P\ {\isasymin}\ set\ {\isacharparenleft}map\ set\ {\isacharparenleft}all{\isacharunderscore}partitions{\isacharunderscore}list\ {\isacharparenleft}x\ {\isacharhash}\ xs{\isacharparenright}{\isacharparenright}{\isacharparenright}{\isachardoublequoteclose}\isanewline
\ \ \ \ \ \ \isacommand{using}\isamarkupfalse%
\ all{\isacharunderscore}coarser{\isacharunderscore}partitions{\isacharunderscore}with{\isacharunderscore}list{\isacharunderscore}equivalence\ hyp{\isacharunderscore}distinct\isanewline
\ \ \ \ \ \ \isacommand{by}\isamarkupfalse%
\ {\isacharparenleft}metis\ all{\isacharunderscore}partitions{\isacharunderscore}list{\isachardot}simps{\isacharparenleft}{\isadigit{2}}{\isacharparenright}{\isacharparenright}\isanewline
\ \ \isacommand{next}\isamarkupfalse%
\ \isanewline
\ \ \ \ \isacommand{fix}\isamarkupfalse%
\ P{\isacharcolon}{\isacharcolon}{\isachardoublequoteopen}{\isacharprime}a\ set\ set{\isachardoublequoteclose}\ \isanewline
\ \ \ \ \isacommand{assume}\isamarkupfalse%
\ P{\isacharcolon}\ {\isachardoublequoteopen}P\ {\isasymin}\ set\ {\isacharparenleft}map\ set\ {\isacharparenleft}all{\isacharunderscore}partitions{\isacharunderscore}list\ {\isacharparenleft}x\ {\isacharhash}\ xs{\isacharparenright}{\isacharparenright}{\isacharparenright}{\isachardoublequoteclose}\isanewline
\isanewline
\ \ \ \ \isacommand{have}\isamarkupfalse%
\ {\isachardoublequoteopen}set\ {\isacharparenleft}map\ set\ {\isacharparenleft}all{\isacharunderscore}partitions{\isacharunderscore}list\ {\isacharparenleft}x\ {\isacharhash}\ xs{\isacharparenright}{\isacharparenright}{\isacharparenright}\ {\isacharequal}\ set\ {\isacharparenleft}map\ set\ {\isacharparenleft}all{\isacharunderscore}coarser{\isacharunderscore}partitions{\isacharunderscore}with{\isacharunderscore}list\ x\ {\isacharparenleft}all{\isacharunderscore}partitions{\isacharunderscore}list\ xs{\isacharparenright}{\isacharparenright}{\isacharparenright}{\isachardoublequoteclose}\isanewline
\ \ \ \ \ \ \isacommand{by}\isamarkupfalse%
\ simp\isanewline
\ \ \ \ \isacommand{also}\isamarkupfalse%
\ \isacommand{have}\isamarkupfalse%
\ {\isachardoublequoteopen}{\isasymdots}\ {\isacharequal}\ all{\isacharunderscore}coarser{\isacharunderscore}partitions{\isacharunderscore}with\ x\ {\isacharparenleft}set\ {\isacharparenleft}map\ set\ {\isacharparenleft}all{\isacharunderscore}partitions{\isacharunderscore}list\ xs{\isacharparenright}{\isacharparenright}{\isacharparenright}{\isachardoublequoteclose}\isanewline
\ \ \ \ \ \ \isacommand{using}\isamarkupfalse%
\ distinct{\isacharunderscore}xs\ hyp{\isacharunderscore}distinct\ all{\isacharunderscore}coarser{\isacharunderscore}partitions{\isacharunderscore}with{\isacharunderscore}list{\isacharunderscore}equivalence\ \isacommand{by}\isamarkupfalse%
\ fast\isanewline
\ \ \ \ \isacommand{also}\isamarkupfalse%
\ \isacommand{have}\isamarkupfalse%
\ {\isachardoublequoteopen}{\isasymdots}\ {\isacharequal}\ all{\isacharunderscore}coarser{\isacharunderscore}partitions{\isacharunderscore}with\ x\ {\isacharparenleft}all{\isacharunderscore}partitions\ {\isacharparenleft}set\ xs{\isacharparenright}{\isacharparenright}{\isachardoublequoteclose}\isanewline
\ \ \ \ \ \ \isacommand{using}\isamarkupfalse%
\ distinct{\isacharunderscore}xs\ hyp{\isacharunderscore}equiv\ \isacommand{by}\isamarkupfalse%
\ auto\isanewline
\ \ \ \ \isacommand{finally}\isamarkupfalse%
\ \isacommand{have}\isamarkupfalse%
\ P{\isacharunderscore}set{\isacharcolon}\ {\isachardoublequoteopen}set\ {\isacharparenleft}map\ set\ {\isacharparenleft}all{\isacharunderscore}partitions{\isacharunderscore}list\ {\isacharparenleft}x\ {\isacharhash}\ xs{\isacharparenright}{\isacharparenright}{\isacharparenright}\ {\isacharequal}\ all{\isacharunderscore}coarser{\isacharunderscore}partitions{\isacharunderscore}with\ x\ {\isacharparenleft}all{\isacharunderscore}partitions\ {\isacharparenleft}set\ xs{\isacharparenright}{\isacharparenright}{\isachardoublequoteclose}\ \isacommand{{\isachardot}}\isamarkupfalse%
\isanewline
\isanewline
\ \ \ \ \isacommand{with}\isamarkupfalse%
\ P\ \isacommand{have}\isamarkupfalse%
\ {\isachardoublequoteopen}P\ {\isasymin}\ all{\isacharunderscore}coarser{\isacharunderscore}partitions{\isacharunderscore}with\ x\ {\isacharparenleft}all{\isacharunderscore}partitions\ {\isacharparenleft}set\ xs{\isacharparenright}{\isacharparenright}{\isachardoublequoteclose}\ \isacommand{by}\isamarkupfalse%
\ fast\isanewline
\ \ \ \ \isacommand{then}\isamarkupfalse%
\ \isacommand{have}\isamarkupfalse%
\ {\isachardoublequoteopen}P\ {\isasymin}\ {\isasymUnion}\ {\isacharparenleft}coarser{\isacharunderscore}partitions{\isacharunderscore}with\ x\ {\isacharbackquote}\ {\isacharparenleft}all{\isacharunderscore}partitions\ {\isacharparenleft}set\ xs{\isacharparenright}{\isacharparenright}{\isacharparenright}{\isachardoublequoteclose}\isanewline
\ \ \ \ \ \ \isacommand{unfolding}\isamarkupfalse%
\ all{\isacharunderscore}coarser{\isacharunderscore}partitions{\isacharunderscore}with{\isacharunderscore}def\ \isacommand{{\isachardot}}\isamarkupfalse%
\isanewline
\ \ \ \ \isacommand{then}\isamarkupfalse%
\ \isacommand{obtain}\isamarkupfalse%
\ Y\isanewline
\ \ \ \ \ \ \isakeyword{where}\ P{\isacharunderscore}in{\isacharunderscore}Y{\isacharcolon}\ {\isachardoublequoteopen}P\ {\isasymin}\ Y{\isachardoublequoteclose}\isanewline
\ \ \ \ \ \ \ \ \isakeyword{and}\ Y{\isacharunderscore}coarser{\isacharcolon}\ {\isachardoublequoteopen}Y\ {\isasymin}\ coarser{\isacharunderscore}partitions{\isacharunderscore}with\ x\ {\isacharbackquote}\ {\isacharparenleft}all{\isacharunderscore}partitions\ {\isacharparenleft}set\ xs{\isacharparenright}{\isacharparenright}{\isachardoublequoteclose}\ \isacommand{{\isachardot}{\isachardot}}\isamarkupfalse%
\isanewline
\ \ \ \ \isacommand{from}\isamarkupfalse%
\ Y{\isacharunderscore}coarser\ \isacommand{obtain}\isamarkupfalse%
\ Q\isanewline
\ \ \ \ \ \ \isakeyword{where}\ Q{\isacharunderscore}part{\isacharunderscore}xs{\isacharcolon}\ {\isachardoublequoteopen}Q\ {\isasymin}\ all{\isacharunderscore}partitions\ {\isacharparenleft}set\ xs{\isacharparenright}{\isachardoublequoteclose}\isanewline
\ \ \ \ \ \ \ \ \isakeyword{and}\ Y{\isacharunderscore}coarser{\isacharprime}{\isacharcolon}\ {\isachardoublequoteopen}Y\ {\isacharequal}\ coarser{\isacharunderscore}partitions{\isacharunderscore}with\ x\ Q{\isachardoublequoteclose}\ \isacommand{{\isachardot}{\isachardot}}\isamarkupfalse%
\isanewline
\ \ \ \ \isacommand{from}\isamarkupfalse%
\ P{\isacharunderscore}in{\isacharunderscore}Y\ Y{\isacharunderscore}coarser{\isacharprime}\ \isacommand{have}\isamarkupfalse%
\ P{\isacharunderscore}wrt{\isacharunderscore}Q{\isacharcolon}\ {\isachardoublequoteopen}P\ {\isasymin}\ coarser{\isacharunderscore}partitions{\isacharunderscore}with\ x\ Q{\isachardoublequoteclose}\ \isacommand{by}\isamarkupfalse%
\ fast\isanewline
\ \ \ \ \isacommand{then}\isamarkupfalse%
\ \isacommand{have}\isamarkupfalse%
\ {\isachardoublequoteopen}Q\ {\isasymin}\ all{\isacharunderscore}partitions\ {\isacharparenleft}set\ xs{\isacharparenright}{\isachardoublequoteclose}\ \isacommand{using}\isamarkupfalse%
\ Q{\isacharunderscore}part{\isacharunderscore}xs\ \isacommand{by}\isamarkupfalse%
\ simp\isanewline
\ \ \ \ \isacommand{then}\isamarkupfalse%
\ \isacommand{have}\isamarkupfalse%
\ {\isachardoublequoteopen}Q\ partitions\ {\isacharparenleft}set\ xs{\isacharparenright}{\isachardoublequoteclose}\ \isacommand{unfolding}\isamarkupfalse%
\ all{\isacharunderscore}partitions{\isacharunderscore}def\ \isacommand{{\isachardot}{\isachardot}}\isamarkupfalse%
\isanewline
\ \ \ \ \isacommand{then}\isamarkupfalse%
\ \isacommand{have}\isamarkupfalse%
\ {\isachardoublequoteopen}is{\isacharunderscore}non{\isacharunderscore}overlapping\ Q{\isachardoublequoteclose}\ \isakeyword{and}\ Q{\isacharunderscore}covers{\isacharcolon}\ {\isachardoublequoteopen}{\isasymUnion}\ Q\ {\isacharequal}\ set\ xs{\isachardoublequoteclose}\isanewline
\ \ \ \ \ \ \isacommand{unfolding}\isamarkupfalse%
\ is{\isacharunderscore}partition{\isacharunderscore}of{\isacharunderscore}def\ \isacommand{by}\isamarkupfalse%
\ simp{\isacharunderscore}all\isanewline
\ \ \ \ \isacommand{then}\isamarkupfalse%
\ \isacommand{have}\isamarkupfalse%
\ P{\isacharunderscore}partition{\isacharcolon}\ {\isachardoublequoteopen}is{\isacharunderscore}non{\isacharunderscore}overlapping\ P{\isachardoublequoteclose}\isanewline
\ \ \ \ \ \ \isacommand{using}\isamarkupfalse%
\ non{\isacharunderscore}overlapping{\isacharunderscore}extension{\isadigit{3}}\ P{\isacharunderscore}wrt{\isacharunderscore}Q\ x{\isacharunderscore}notin{\isacharunderscore}xs\ \isacommand{by}\isamarkupfalse%
\ fast\isanewline
\ \ \ \ \isacommand{have}\isamarkupfalse%
\ {\isachardoublequoteopen}{\isasymUnion}\ P\ {\isacharequal}\ set\ xs\ {\isasymunion}\ {\isacharbraceleft}x{\isacharbraceright}{\isachardoublequoteclose}\isanewline
\ \ \ \ \ \ \isacommand{using}\isamarkupfalse%
\ Q{\isacharunderscore}covers\ P{\isacharunderscore}in{\isacharunderscore}Y\ Y{\isacharunderscore}coarser{\isacharprime}\ coarser{\isacharunderscore}partitions{\isacharunderscore}covers\ \isacommand{by}\isamarkupfalse%
\ fast\isanewline
\ \ \ \ \isacommand{then}\isamarkupfalse%
\ \isacommand{have}\isamarkupfalse%
\ {\isachardoublequoteopen}{\isasymUnion}\ P\ {\isacharequal}\ set\ {\isacharparenleft}x\ {\isacharhash}\ xs{\isacharparenright}{\isachardoublequoteclose}\isanewline
\ \ \ \ \ \ \isacommand{using}\isamarkupfalse%
\ x{\isacharunderscore}notin{\isacharunderscore}xs\ P{\isacharunderscore}wrt{\isacharunderscore}Q\ Q{\isacharunderscore}covers\isanewline
\ \ \ \ \ \ \isacommand{by}\isamarkupfalse%
\ {\isacharparenleft}metis\ List{\isachardot}set{\isacharunderscore}simps{\isacharparenleft}{\isadigit{2}}{\isacharparenright}\ insert{\isacharunderscore}is{\isacharunderscore}Un\ sup{\isacharunderscore}commute{\isacharparenright}\isanewline
\ \ \ \ \isacommand{then}\isamarkupfalse%
\ \isacommand{have}\isamarkupfalse%
\ {\isachardoublequoteopen}P\ partitions\ {\isacharparenleft}set\ {\isacharparenleft}x\ {\isacharhash}\ xs{\isacharparenright}{\isacharparenright}{\isachardoublequoteclose}\isanewline
\ \ \ \ \ \ \isacommand{using}\isamarkupfalse%
\ P{\isacharunderscore}partition\ \isacommand{unfolding}\isamarkupfalse%
\ is{\isacharunderscore}partition{\isacharunderscore}of{\isacharunderscore}def\ \isacommand{by}\isamarkupfalse%
\ blast\isanewline
\ \ \ \ \isacommand{then}\isamarkupfalse%
\ \isacommand{show}\isamarkupfalse%
\ {\isachardoublequoteopen}P\ {\isasymin}\ all{\isacharunderscore}partitions\ {\isacharparenleft}set\ {\isacharparenleft}x\ {\isacharhash}\ xs{\isacharparenright}{\isacharparenright}{\isachardoublequoteclose}\ \isacommand{unfolding}\isamarkupfalse%
\ all{\isacharunderscore}partitions{\isacharunderscore}def\ \isacommand{{\isachardot}{\isachardot}}\isamarkupfalse%
\isanewline
\ \ \isacommand{qed}\isamarkupfalse%
\isanewline
\ \ \isacommand{moreover}\isamarkupfalse%
\ \isacommand{have}\isamarkupfalse%
\ {\isachardoublequoteopen}{\isasymforall}\ ps\ {\isasymin}\ set\ {\isacharparenleft}all{\isacharunderscore}partitions{\isacharunderscore}list\ {\isacharparenleft}x\ {\isacharhash}\ xs{\isacharparenright}{\isacharparenright}\ {\isachardot}\ distinct\ ps{\isachardoublequoteclose}\isanewline
\ \ \isacommand{proof}\isamarkupfalse%
\isanewline
\ \ \ \ \isacommand{fix}\isamarkupfalse%
\ ps{\isacharcolon}{\isacharcolon}{\isachardoublequoteopen}{\isacharprime}a\ set\ list{\isachardoublequoteclose}\ \isacommand{assume}\isamarkupfalse%
\ ps{\isacharunderscore}part{\isacharcolon}\ {\isachardoublequoteopen}ps\ {\isasymin}\ set\ {\isacharparenleft}all{\isacharunderscore}partitions{\isacharunderscore}list\ {\isacharparenleft}x\ {\isacharhash}\ xs{\isacharparenright}{\isacharparenright}{\isachardoublequoteclose}\isanewline
\isanewline
\ \ \ \ \isacommand{have}\isamarkupfalse%
\ {\isachardoublequoteopen}set\ {\isacharparenleft}all{\isacharunderscore}partitions{\isacharunderscore}list\ {\isacharparenleft}x\ {\isacharhash}\ xs{\isacharparenright}{\isacharparenright}\ {\isacharequal}\ set\ {\isacharparenleft}all{\isacharunderscore}coarser{\isacharunderscore}partitions{\isacharunderscore}with{\isacharunderscore}list\ x\ {\isacharparenleft}all{\isacharunderscore}partitions{\isacharunderscore}list\ xs{\isacharparenright}{\isacharparenright}{\isachardoublequoteclose}\isanewline
\ \ \ \ \ \ \isacommand{by}\isamarkupfalse%
\ simp\isanewline
\ \ \ \ \isacommand{also}\isamarkupfalse%
\ \isacommand{have}\isamarkupfalse%
\ {\isachardoublequoteopen}{\isasymdots}\ {\isacharequal}\ set\ {\isacharparenleft}concat\ {\isacharparenleft}map\ {\isacharparenleft}coarser{\isacharunderscore}partitions{\isacharunderscore}with{\isacharunderscore}list\ x{\isacharparenright}\ {\isacharparenleft}all{\isacharunderscore}partitions{\isacharunderscore}list\ xs{\isacharparenright}{\isacharparenright}{\isacharparenright}{\isachardoublequoteclose}\isanewline
\ \ \ \ \ \ \isacommand{unfolding}\isamarkupfalse%
\ all{\isacharunderscore}coarser{\isacharunderscore}partitions{\isacharunderscore}with{\isacharunderscore}list{\isacharunderscore}def\ \isacommand{{\isachardot}{\isachardot}}\isamarkupfalse%
\isanewline
\ \ \ \ \isacommand{also}\isamarkupfalse%
\ \isacommand{have}\isamarkupfalse%
\ {\isachardoublequoteopen}{\isasymdots}\ {\isacharequal}\ {\isasymUnion}\ {\isacharparenleft}{\isacharparenleft}set\ {\isasymcirc}\ {\isacharparenleft}coarser{\isacharunderscore}partitions{\isacharunderscore}with{\isacharunderscore}list\ x{\isacharparenright}{\isacharparenright}\ {\isacharbackquote}\ {\isacharparenleft}set\ {\isacharparenleft}all{\isacharunderscore}partitions{\isacharunderscore}list\ xs{\isacharparenright}{\isacharparenright}{\isacharparenright}{\isachardoublequoteclose}\isanewline
\ \ \ \ \ \ \isacommand{by}\isamarkupfalse%
\ simp\isanewline
\ \ \ \ \isacommand{finally}\isamarkupfalse%
\ \isacommand{have}\isamarkupfalse%
\ all{\isacharunderscore}parts{\isacharunderscore}unfolded{\isacharcolon}\ {\isachardoublequoteopen}set\ {\isacharparenleft}all{\isacharunderscore}partitions{\isacharunderscore}list\ {\isacharparenleft}x\ {\isacharhash}\ xs{\isacharparenright}{\isacharparenright}\ {\isacharequal}\ {\isasymUnion}\ {\isacharparenleft}{\isacharparenleft}set\ {\isasymcirc}\ {\isacharparenleft}coarser{\isacharunderscore}partitions{\isacharunderscore}with{\isacharunderscore}list\ x{\isacharparenright}{\isacharparenright}\ {\isacharbackquote}\ {\isacharparenleft}set\ {\isacharparenleft}all{\isacharunderscore}partitions{\isacharunderscore}list\ xs{\isacharparenright}{\isacharparenright}{\isacharparenright}{\isachardoublequoteclose}\ \isacommand{{\isachardot}}\isamarkupfalse%
\isanewline
\ \ \ \ \isanewline
\isanewline
\ \ \ \ \isacommand{with}\isamarkupfalse%
\ ps{\isacharunderscore}part\ \isacommand{obtain}\isamarkupfalse%
\ qs\isanewline
\ \ \ \ \ \ \isakeyword{where}\ qs{\isacharcolon}\ {\isachardoublequoteopen}qs\ {\isasymin}\ set\ {\isacharparenleft}all{\isacharunderscore}partitions{\isacharunderscore}list\ xs{\isacharparenright}{\isachardoublequoteclose}\isanewline
\ \ \ \ \ \ \ \ \isakeyword{and}\ ps{\isacharunderscore}coarser{\isacharcolon}\ {\isachardoublequoteopen}ps\ {\isasymin}\ set\ {\isacharparenleft}coarser{\isacharunderscore}partitions{\isacharunderscore}with{\isacharunderscore}list\ x\ qs{\isacharparenright}{\isachardoublequoteclose}\isanewline
\ \ \ \ \ \ \isacommand{using}\isamarkupfalse%
\ UnionE\ comp{\isacharunderscore}def\ imageE\ \isacommand{by}\isamarkupfalse%
\ auto\isanewline
\isanewline
\ \ \ \ \isacommand{from}\isamarkupfalse%
\ qs\ \isacommand{have}\isamarkupfalse%
\ {\isachardoublequoteopen}set\ qs\ {\isasymin}\ set\ {\isacharparenleft}map\ set\ {\isacharparenleft}all{\isacharunderscore}partitions{\isacharunderscore}list\ {\isacharparenleft}xs{\isacharparenright}{\isacharparenright}{\isacharparenright}{\isachardoublequoteclose}\ \isacommand{by}\isamarkupfalse%
\ simp\isanewline
\ \ \ \ \isacommand{with}\isamarkupfalse%
\ distinct{\isacharunderscore}xs\ hyp{\isacharunderscore}equiv\ \isacommand{have}\isamarkupfalse%
\ qs{\isacharunderscore}hyp{\isacharcolon}\ {\isachardoublequoteopen}set\ qs\ {\isasymin}\ all{\isacharunderscore}partitions\ {\isacharparenleft}set\ xs{\isacharparenright}{\isachardoublequoteclose}\ \isacommand{by}\isamarkupfalse%
\ fast\isanewline
\ \ \ \ \isacommand{then}\isamarkupfalse%
\ \isacommand{have}\isamarkupfalse%
\ qs{\isacharunderscore}part{\isacharcolon}\ {\isachardoublequoteopen}is{\isacharunderscore}non{\isacharunderscore}overlapping\ {\isacharparenleft}set\ qs{\isacharparenright}{\isachardoublequoteclose}\isanewline
\ \ \ \ \ \ \isacommand{using}\isamarkupfalse%
\ all{\isacharunderscore}partitions{\isacharunderscore}def\ is{\isacharunderscore}partition{\isacharunderscore}of{\isacharunderscore}def\isanewline
\ \ \ \ \ \ \isacommand{by}\isamarkupfalse%
\ {\isacharparenleft}metis\ mem{\isacharunderscore}Collect{\isacharunderscore}eq{\isacharparenright}\isanewline
\ \ \ \ \isacommand{then}\isamarkupfalse%
\ \isacommand{have}\isamarkupfalse%
\ distinct{\isacharunderscore}qs{\isacharcolon}\ {\isachardoublequoteopen}distinct\ qs{\isachardoublequoteclose}\isanewline
\ \ \ \ \ \ \isacommand{using}\isamarkupfalse%
\ qs\ distinct{\isacharunderscore}xs\ hyp{\isacharunderscore}distinct\ \isacommand{by}\isamarkupfalse%
\ fast\isanewline
\ \ \ \ \isanewline
\ \ \ \ \isacommand{from}\isamarkupfalse%
\ Cons{\isachardot}prems\ \isacommand{have}\isamarkupfalse%
\ {\isachardoublequoteopen}x\ {\isasymnotin}\ set\ xs{\isachardoublequoteclose}\ \isacommand{by}\isamarkupfalse%
\ simp\isanewline
\ \ \ \ \isacommand{then}\isamarkupfalse%
\ \isacommand{have}\isamarkupfalse%
\ new{\isacharcolon}\ {\isachardoublequoteopen}{\isacharbraceleft}x{\isacharbraceright}\ {\isasymnotin}\ set\ qs{\isachardoublequoteclose}\isanewline
\ \ \ \ \ \ \isacommand{using}\isamarkupfalse%
\ qs{\isacharunderscore}hyp\isanewline
\ \ \ \ \ \ \isacommand{unfolding}\isamarkupfalse%
\ all{\isacharunderscore}partitions{\isacharunderscore}def\ is{\isacharunderscore}partition{\isacharunderscore}of{\isacharunderscore}def\isanewline
\ \ \ \ \ \ \isacommand{by}\isamarkupfalse%
\ {\isacharparenleft}metis\ {\isacharparenleft}lifting{\isacharcomma}\ mono{\isacharunderscore}tags{\isacharparenright}\ UnionI\ insertI{\isadigit{1}}\ mem{\isacharunderscore}Collect{\isacharunderscore}eq{\isacharparenright}\isanewline
\isanewline
\ \ \ \ \isacommand{from}\isamarkupfalse%
\ ps{\isacharunderscore}coarser\ distinct{\isacharunderscore}qs\ qs{\isacharunderscore}part\ new\isanewline
\ \ \ \ \ \ \isacommand{show}\isamarkupfalse%
\ {\isachardoublequoteopen}distinct\ ps{\isachardoublequoteclose}\ \isacommand{by}\isamarkupfalse%
\ {\isacharparenleft}rule\ coarser{\isacharunderscore}partitions{\isacharunderscore}with{\isacharunderscore}list{\isacharunderscore}distinct{\isacharparenright}\isanewline
\ \ \isacommand{qed}\isamarkupfalse%
\isanewline
\ \ \isacommand{ultimately}\isamarkupfalse%
\ \isacommand{show}\isamarkupfalse%
\ {\isacharquery}case\ \isacommand{{\isachardot}{\isachardot}}\isamarkupfalse%
\isanewline
\isacommand{qed}\isamarkupfalse%
%
\endisatagproof
{\isafoldproof}%
%
\isadelimproof
%
\endisadelimproof
%
\begin{isamarkuptext}%
The classical definition \isa{all{\isacharunderscore}partitions} and the algorithmic (constructive) definition \isa{all{\isacharunderscore}partitions{\isacharunderscore}list} are equivalent.  This is a front-end theorem derived from
  \isa{distinct\ {\isacharquery}xs\ {\isasymLongrightarrow}\ set\ {\isacharparenleft}map\ set\ {\isacharparenleft}all{\isacharunderscore}partitions{\isacharunderscore}list\ {\isacharquery}xs{\isacharparenright}{\isacharparenright}\ {\isacharequal}\ all{\isacharunderscore}partitions\ {\isacharparenleft}set\ {\isacharquery}xs{\isacharparenright}\ {\isasymand}\ {\isacharparenleft}{\isasymforall}ps{\isasymin}set\ {\isacharparenleft}all{\isacharunderscore}partitions{\isacharunderscore}list\ {\isacharquery}xs{\isacharparenright}{\isachardot}\ distinct\ ps{\isacharparenright}}; it does not make the auxiliary statement about partitions
  being distinct lists.%
\end{isamarkuptext}%
\isamarkuptrue%
\isacommand{theorem}\isamarkupfalse%
\ all{\isacharunderscore}partitions{\isacharunderscore}paper{\isacharunderscore}equiv{\isacharunderscore}alg{\isacharcolon}\isanewline
\ \ \isakeyword{fixes}\ xs{\isacharcolon}{\isacharcolon}{\isachardoublequoteopen}{\isacharprime}a\ list{\isachardoublequoteclose}\isanewline
\ \ \isakeyword{shows}\ {\isachardoublequoteopen}distinct\ xs\ {\isasymLongrightarrow}\ set\ {\isacharparenleft}map\ set\ {\isacharparenleft}all{\isacharunderscore}partitions{\isacharunderscore}list\ xs{\isacharparenright}{\isacharparenright}\ {\isacharequal}\ all{\isacharunderscore}partitions\ {\isacharparenleft}set\ xs{\isacharparenright}{\isachardoublequoteclose}\isanewline
%
\isadelimproof
\ \ %
\endisadelimproof
%
\isatagproof
\isacommand{using}\isamarkupfalse%
\ all{\isacharunderscore}partitions{\isacharunderscore}equivalence{\isacharprime}\ \isacommand{by}\isamarkupfalse%
\ blast%
\endisatagproof
{\isafoldproof}%
%
\isadelimproof
%
\endisadelimproof
%
\begin{isamarkuptext}%
The function that we will be using in practice to compute all partitions of a set,
  a set-oriented front-end to \isa{all{\isacharunderscore}partitions{\isacharunderscore}list}%
\end{isamarkuptext}%
\isamarkuptrue%
\isacommand{definition}\isamarkupfalse%
\ all{\isacharunderscore}partitions{\isacharunderscore}alg\ {\isacharcolon}{\isacharcolon}\ {\isachardoublequoteopen}{\isacharprime}a{\isasymColon}linorder\ set\ {\isasymRightarrow}\ {\isacharprime}a\ set\ list\ list{\isachardoublequoteclose}\isanewline
\ \ \isakeyword{where}\ {\isachardoublequoteopen}all{\isacharunderscore}partitions{\isacharunderscore}alg\ X\ {\isacharequal}\ all{\isacharunderscore}partitions{\isacharunderscore}list\ {\isacharparenleft}sorted{\isacharunderscore}list{\isacharunderscore}of{\isacharunderscore}set\ X{\isacharparenright}{\isachardoublequoteclose}\isanewline
%
\isadelimtheory
\isanewline
%
\endisadelimtheory
%
\isatagtheory
\isacommand{end}\isamarkupfalse%
%
\endisatagtheory
{\isafoldtheory}%
%
\isadelimtheory
%
\endisadelimtheory
\end{isabellebody}%
%%% Local Variables:
%%% mode: latex
%%% TeX-master: "root"
%%% End:


%
\begin{isabellebody}%
\def\isabellecontext{Code{\isacharunderscore}Abstract{\isacharunderscore}Nat}%
%
\isamarkupheader{Avoidance of pattern matching on natural numbers%
}
\isamarkuptrue%
%
\isadelimtheory
%
\endisadelimtheory
%
\isatagtheory
\isacommand{theory}\isamarkupfalse%
\ Code{\isacharunderscore}Abstract{\isacharunderscore}Nat\isanewline
\isakeyword{imports}\ Main\isanewline
\isakeyword{begin}%
\endisatagtheory
{\isafoldtheory}%
%
\isadelimtheory
%
\endisadelimtheory
%
\begin{isamarkuptext}%
When natural numbers are implemented in another than the
  conventional inductive \isa{{\isadigit{0}}}/\isa{Suc} representation,
  it is necessary to avoid all pattern matching on natural numbers
  altogether.  This is accomplished by this theory (up to a certain
  extent).%
\end{isamarkuptext}%
\isamarkuptrue%
%
\isamarkupsubsection{Case analysis%
}
\isamarkuptrue%
%
\begin{isamarkuptext}%
Case analysis on natural numbers is rephrased using a conditional
  expression:%
\end{isamarkuptext}%
\isamarkuptrue%
\isacommand{lemma}\isamarkupfalse%
\ {\isacharbrackleft}code{\isacharcomma}\ code{\isacharunderscore}unfold{\isacharbrackright}{\isacharcolon}\isanewline
\ \ {\isachardoublequoteopen}case{\isacharunderscore}nat\ {\isacharequal}\ {\isacharparenleft}{\isasymlambda}f\ g\ n{\isachardot}\ if\ n\ {\isacharequal}\ {\isadigit{0}}\ then\ f\ else\ g\ {\isacharparenleft}n\ {\isacharminus}\ {\isadigit{1}}{\isacharparenright}{\isacharparenright}{\isachardoublequoteclose}\isanewline
%
\isadelimproof
\ \ %
\endisadelimproof
%
\isatagproof
\isacommand{by}\isamarkupfalse%
\ {\isacharparenleft}auto\ simp\ add{\isacharcolon}\ fun{\isacharunderscore}eq{\isacharunderscore}iff\ dest{\isacharbang}{\isacharcolon}\ gr{\isadigit{0}}{\isacharunderscore}implies{\isacharunderscore}Suc{\isacharparenright}%
\endisatagproof
{\isafoldproof}%
%
\isadelimproof
%
\endisadelimproof
%
\isamarkupsubsection{Preprocessors%
}
\isamarkuptrue%
%
\begin{isamarkuptext}%
The term \isa{Suc\ n} is no longer a valid pattern.  Therefore,
  all occurrences of this term in a position where a pattern is
  expected (i.e.~on the left-hand side of a code equation) must be
  eliminated.  This can be accomplished -- as far as possible -- by
  applying the following transformation rule:%
\end{isamarkuptext}%
\isamarkuptrue%
\isacommand{lemma}\isamarkupfalse%
\ Suc{\isacharunderscore}if{\isacharunderscore}eq{\isacharcolon}\isanewline
\ \ \isakeyword{assumes}\ {\isachardoublequoteopen}{\isasymAnd}n{\isachardot}\ f\ {\isacharparenleft}Suc\ n{\isacharparenright}\ {\isasymequiv}\ h\ n{\isachardoublequoteclose}\isanewline
\ \ \isakeyword{assumes}\ {\isachardoublequoteopen}f\ {\isadigit{0}}\ {\isasymequiv}\ g{\isachardoublequoteclose}\isanewline
\ \ \isakeyword{shows}\ {\isachardoublequoteopen}f\ n\ {\isasymequiv}\ if\ n\ {\isacharequal}\ {\isadigit{0}}\ then\ g\ else\ h\ {\isacharparenleft}n\ {\isacharminus}\ {\isadigit{1}}{\isacharparenright}{\isachardoublequoteclose}\isanewline
%
\isadelimproof
\ \ %
\endisadelimproof
%
\isatagproof
\isacommand{by}\isamarkupfalse%
\ {\isacharparenleft}rule\ eq{\isacharunderscore}reflection{\isacharparenright}\ {\isacharparenleft}cases\ n{\isacharcomma}\ insert\ assms{\isacharcomma}\ simp{\isacharunderscore}all{\isacharparenright}%
\endisatagproof
{\isafoldproof}%
%
\isadelimproof
%
\endisadelimproof
%
\begin{isamarkuptext}%
The rule above is built into a preprocessor that is plugged into
  the code generator.%
\end{isamarkuptext}%
\isamarkuptrue%
%
\isadelimML
%
\endisadelimML
%
\isatagML
\isacommand{setup}\isamarkupfalse%
\ {\isacharverbatimopen}\isanewline
let\isanewline
\isanewline
val\ Suc{\isacharunderscore}if{\isacharunderscore}eq\ {\isacharequal}\ Thm{\isachardot}incr{\isacharunderscore}indexes\ {\isadigit{1}}\ %
\isaantiq
thm\ Suc{\isacharunderscore}if{\isacharunderscore}eq{}%
\endisaantiq
{\isacharsemicolon}\isanewline
\isanewline
fun\ remove{\isacharunderscore}suc\ ctxt\ thms\ {\isacharequal}\isanewline
\ \ let\isanewline
\ \ \ \ val\ thy\ {\isacharequal}\ Proof{\isacharunderscore}Context{\isachardot}theory{\isacharunderscore}of\ ctxt{\isacharsemicolon}\isanewline
\ \ \ \ val\ vname\ {\isacharequal}\ singleton\ {\isacharparenleft}Name{\isachardot}variant{\isacharunderscore}list\ {\isacharparenleft}map\ fst\isanewline
\ \ \ \ \ \ {\isacharparenleft}fold\ {\isacharparenleft}Term{\isachardot}add{\isacharunderscore}var{\isacharunderscore}names\ o\ Thm{\isachardot}full{\isacharunderscore}prop{\isacharunderscore}of{\isacharparenright}\ thms\ {\isacharbrackleft}{\isacharbrackright}{\isacharparenright}{\isacharparenright}{\isacharparenright}\ {\isachardoublequote}n{\isachardoublequote}{\isacharsemicolon}\isanewline
\ \ \ \ val\ cv\ {\isacharequal}\ cterm{\isacharunderscore}of\ thy\ {\isacharparenleft}Var\ {\isacharparenleft}{\isacharparenleft}vname{\isacharcomma}\ {\isadigit{0}}{\isacharparenright}{\isacharcomma}\ HOLogic{\isachardot}natT{\isacharparenright}{\isacharparenright}{\isacharsemicolon}\isanewline
\ \ \ \ val\ lhs{\isacharunderscore}of\ {\isacharequal}\ snd\ o\ Thm{\isachardot}dest{\isacharunderscore}comb\ o\ fst\ o\ Thm{\isachardot}dest{\isacharunderscore}comb\ o\ cprop{\isacharunderscore}of{\isacharsemicolon}\isanewline
\ \ \ \ val\ rhs{\isacharunderscore}of\ {\isacharequal}\ snd\ o\ Thm{\isachardot}dest{\isacharunderscore}comb\ o\ cprop{\isacharunderscore}of{\isacharsemicolon}\isanewline
\ \ \ \ fun\ find{\isacharunderscore}vars\ ct\ {\isacharequal}\ {\isacharparenleft}case\ term{\isacharunderscore}of\ ct\ of\isanewline
\ \ \ \ \ \ \ \ {\isacharparenleft}Const\ {\isacharparenleft}%
\isaantiq
const{\isacharunderscore}name\ Suc{}%
\endisaantiq
{\isacharcomma}\ {\isacharunderscore}{\isacharparenright}\ {\isachardollar}\ Var\ {\isacharunderscore}{\isacharparenright}\ {\isacharequal}{\isachargreater}\ {\isacharbrackleft}{\isacharparenleft}cv{\isacharcomma}\ snd\ {\isacharparenleft}Thm{\isachardot}dest{\isacharunderscore}comb\ ct{\isacharparenright}{\isacharparenright}{\isacharbrackright}\isanewline
\ \ \ \ \ \ {\isacharbar}\ {\isacharunderscore}\ {\isachardollar}\ {\isacharunderscore}\ {\isacharequal}{\isachargreater}\isanewline
\ \ \ \ \ \ \ \ let\ val\ {\isacharparenleft}ct{\isadigit{1}}{\isacharcomma}\ ct{\isadigit{2}}{\isacharparenright}\ {\isacharequal}\ Thm{\isachardot}dest{\isacharunderscore}comb\ ct\isanewline
\ \ \ \ \ \ \ \ in\ \isanewline
\ \ \ \ \ \ \ \ \ \ map\ {\isacharparenleft}apfst\ {\isacharparenleft}fn\ ct\ {\isacharequal}{\isachargreater}\ Thm{\isachardot}apply\ ct\ ct{\isadigit{2}}{\isacharparenright}{\isacharparenright}\ {\isacharparenleft}find{\isacharunderscore}vars\ ct{\isadigit{1}}{\isacharparenright}\ {\isacharat}\isanewline
\ \ \ \ \ \ \ \ \ \ map\ {\isacharparenleft}apfst\ {\isacharparenleft}Thm{\isachardot}apply\ ct{\isadigit{1}}{\isacharparenright}{\isacharparenright}\ {\isacharparenleft}find{\isacharunderscore}vars\ ct{\isadigit{2}}{\isacharparenright}\isanewline
\ \ \ \ \ \ \ \ end\isanewline
\ \ \ \ \ \ {\isacharbar}\ {\isacharunderscore}\ {\isacharequal}{\isachargreater}\ {\isacharbrackleft}{\isacharbrackright}{\isacharparenright}{\isacharsemicolon}\isanewline
\ \ \ \ val\ eqs\ {\isacharequal}\ maps\isanewline
\ \ \ \ \ \ {\isacharparenleft}fn\ thm\ {\isacharequal}{\isachargreater}\ map\ {\isacharparenleft}pair\ thm{\isacharparenright}\ {\isacharparenleft}find{\isacharunderscore}vars\ {\isacharparenleft}lhs{\isacharunderscore}of\ thm{\isacharparenright}{\isacharparenright}{\isacharparenright}\ thms{\isacharsemicolon}\isanewline
\ \ \ \ fun\ mk{\isacharunderscore}thms\ {\isacharparenleft}thm{\isacharcomma}\ {\isacharparenleft}ct{\isacharcomma}\ cv{\isacharprime}{\isacharparenright}{\isacharparenright}\ {\isacharequal}\isanewline
\ \ \ \ \ \ let\isanewline
\ \ \ \ \ \ \ \ val\ thm{\isacharprime}\ {\isacharequal}\isanewline
\ \ \ \ \ \ \ \ \ \ Thm{\isachardot}implies{\isacharunderscore}elim\isanewline
\ \ \ \ \ \ \ \ \ \ \ {\isacharparenleft}Conv{\isachardot}fconv{\isacharunderscore}rule\ {\isacharparenleft}Thm{\isachardot}beta{\isacharunderscore}conversion\ true{\isacharparenright}\isanewline
\ \ \ \ \ \ \ \ \ \ \ \ \ {\isacharparenleft}Drule{\isachardot}instantiate{\isacharprime}\isanewline
\ \ \ \ \ \ \ \ \ \ \ \ \ \ \ {\isacharbrackleft}SOME\ {\isacharparenleft}ctyp{\isacharunderscore}of{\isacharunderscore}term\ ct{\isacharparenright}{\isacharbrackright}\ {\isacharbrackleft}SOME\ {\isacharparenleft}Thm{\isachardot}lambda\ cv\ ct{\isacharparenright}{\isacharcomma}\isanewline
\ \ \ \ \ \ \ \ \ \ \ \ \ \ \ \ \ SOME\ {\isacharparenleft}Thm{\isachardot}lambda\ cv{\isacharprime}\ {\isacharparenleft}rhs{\isacharunderscore}of\ thm{\isacharparenright}{\isacharparenright}{\isacharcomma}\ NONE{\isacharcomma}\ SOME\ cv{\isacharprime}{\isacharbrackright}\isanewline
\ \ \ \ \ \ \ \ \ \ \ \ \ \ \ Suc{\isacharunderscore}if{\isacharunderscore}eq{\isacharparenright}{\isacharparenright}\ {\isacharparenleft}Thm{\isachardot}forall{\isacharunderscore}intr\ cv{\isacharprime}\ thm{\isacharparenright}\isanewline
\ \ \ \ \ \ in\isanewline
\ \ \ \ \ \ \ \ case\ map{\isacharunderscore}filter\ {\isacharparenleft}fn\ thm{\isacharprime}{\isacharprime}\ {\isacharequal}{\isachargreater}\isanewline
\ \ \ \ \ \ \ \ \ \ \ \ SOME\ {\isacharparenleft}thm{\isacharprime}{\isacharprime}{\isacharcomma}\ singleton\isanewline
\ \ \ \ \ \ \ \ \ \ \ \ \ \ {\isacharparenleft}Variable{\isachardot}trade\ {\isacharparenleft}K\ {\isacharparenleft}fn\ {\isacharbrackleft}thm{\isacharprime}{\isacharprime}{\isacharprime}{\isacharbrackright}\ {\isacharequal}{\isachargreater}\ {\isacharbrackleft}thm{\isacharprime}{\isacharprime}{\isacharprime}\ RS\ thm{\isacharprime}{\isacharbrackright}{\isacharparenright}{\isacharparenright}\isanewline
\ \ \ \ \ \ \ \ \ \ \ \ \ \ \ \ {\isacharparenleft}Variable{\isachardot}global{\isacharunderscore}thm{\isacharunderscore}context\ thm{\isacharprime}{\isacharprime}{\isacharparenright}{\isacharparenright}\ thm{\isacharprime}{\isacharprime}{\isacharparenright}\isanewline
\ \ \ \ \ \ \ \ \ \ handle\ THM\ {\isacharunderscore}\ {\isacharequal}{\isachargreater}\ NONE{\isacharparenright}\ thms\ of\isanewline
\ \ \ \ \ \ \ \ \ \ \ \ {\isacharbrackleft}{\isacharbrackright}\ {\isacharequal}{\isachargreater}\ NONE\isanewline
\ \ \ \ \ \ \ \ \ \ {\isacharbar}\ thmps\ {\isacharequal}{\isachargreater}\isanewline
\ \ \ \ \ \ \ \ \ \ \ \ \ \ let\ val\ {\isacharparenleft}thms{\isadigit{1}}{\isacharcomma}\ thms{\isadigit{2}}{\isacharparenright}\ {\isacharequal}\ split{\isacharunderscore}list\ thmps\isanewline
\ \ \ \ \ \ \ \ \ \ \ \ \ \ in\ SOME\ {\isacharparenleft}subtract\ Thm{\isachardot}eq{\isacharunderscore}thm\ {\isacharparenleft}thm\ {\isacharcolon}{\isacharcolon}\ thms{\isadigit{1}}{\isacharparenright}\ thms\ {\isacharat}\ thms{\isadigit{2}}{\isacharparenright}\ end\isanewline
\ \ \ \ \ \ end\isanewline
\ \ in\ get{\isacharunderscore}first\ mk{\isacharunderscore}thms\ eqs\ end{\isacharsemicolon}\isanewline
\isanewline
fun\ eqn{\isacharunderscore}suc{\isacharunderscore}base{\isacharunderscore}preproc\ thy\ thms\ {\isacharequal}\isanewline
\ \ let\isanewline
\ \ \ \ val\ dest\ {\isacharequal}\ fst\ o\ Logic{\isachardot}dest{\isacharunderscore}equals\ o\ prop{\isacharunderscore}of{\isacharsemicolon}\isanewline
\ \ \ \ val\ contains{\isacharunderscore}suc\ {\isacharequal}\ exists{\isacharunderscore}Const\ {\isacharparenleft}fn\ {\isacharparenleft}c{\isacharcomma}\ {\isacharunderscore}{\isacharparenright}\ {\isacharequal}{\isachargreater}\ c\ {\isacharequal}\ %
\isaantiq
const{\isacharunderscore}name\ Suc{}%
\endisaantiq
{\isacharparenright}{\isacharsemicolon}\isanewline
\ \ in\isanewline
\ \ \ \ if\ forall\ {\isacharparenleft}can\ dest{\isacharparenright}\ thms\ andalso\ exists\ {\isacharparenleft}contains{\isacharunderscore}suc\ o\ dest{\isacharparenright}\ thms\isanewline
\ \ \ \ \ \ then\ thms\ {\isacharbar}{\isachargreater}\ perhaps{\isacharunderscore}loop\ {\isacharparenleft}remove{\isacharunderscore}suc\ thy{\isacharparenright}\ {\isacharbar}{\isachargreater}\ {\isacharparenleft}Option{\isachardot}map\ o\ map{\isacharparenright}\ Drule{\isachardot}zero{\isacharunderscore}var{\isacharunderscore}indexes\isanewline
\ \ \ \ \ \ \ else\ NONE\isanewline
\ \ end{\isacharsemicolon}\isanewline
\isanewline
val\ eqn{\isacharunderscore}suc{\isacharunderscore}preproc\ {\isacharequal}\ Code{\isacharunderscore}Preproc{\isachardot}simple{\isacharunderscore}functrans\ eqn{\isacharunderscore}suc{\isacharunderscore}base{\isacharunderscore}preproc{\isacharsemicolon}\isanewline
\isanewline
in\isanewline
\isanewline
\ \ Code{\isacharunderscore}Preproc{\isachardot}add{\isacharunderscore}functrans\ {\isacharparenleft}{\isachardoublequote}eqn{\isacharunderscore}Suc{\isachardoublequote}{\isacharcomma}\ eqn{\isacharunderscore}suc{\isacharunderscore}preproc{\isacharparenright}\isanewline
\isanewline
end{\isacharsemicolon}\isanewline
{\isacharverbatimclose}%
\endisatagML
{\isafoldML}%
%
\isadelimML
%
\endisadelimML
\isanewline
%
\isadelimtheory
\isanewline
%
\endisadelimtheory
%
\isatagtheory
\isacommand{end}\isamarkupfalse%
%
\endisatagtheory
{\isafoldtheory}%
%
\isadelimtheory
%
\endisadelimtheory
\end{isabellebody}%
%%% Local Variables:
%%% mode: latex
%%% TeX-master: "root"
%%% End:


%
\begin{isabellebody}%
\def\isabellecontext{Code{\isacharunderscore}Target{\isacharunderscore}Nat}%
%
\isamarkupheader{Implementation of natural numbers by target-language integers%
}
\isamarkuptrue%
%
\isadelimtheory
%
\endisadelimtheory
%
\isatagtheory
\isacommand{theory}\isamarkupfalse%
\ Code{\isacharunderscore}Target{\isacharunderscore}Nat\isanewline
\isakeyword{imports}\ Code{\isacharunderscore}Abstract{\isacharunderscore}Nat\isanewline
\isakeyword{begin}%
\endisatagtheory
{\isafoldtheory}%
%
\isadelimtheory
%
\endisadelimtheory
%
\isamarkupsubsection{Implementation for \isa{nat}%
}
\isamarkuptrue%
\isacommand{context}\isamarkupfalse%
\isanewline
\isakeyword{includes}\ natural{\isachardot}lifting\ integer{\isachardot}lifting\isanewline
\isakeyword{begin}\isanewline
\isanewline
\isacommand{lift{\isacharunderscore}definition}\isamarkupfalse%
\ Nat\ {\isacharcolon}{\isacharcolon}\ {\isachardoublequoteopen}integer\ {\isasymRightarrow}\ nat{\isachardoublequoteclose}\isanewline
\ \ \isakeyword{is}\ nat\isanewline
%
\isadelimproof
\ \ %
\endisadelimproof
%
\isatagproof
\isacommand{{\isachardot}}\isamarkupfalse%
%
\endisatagproof
{\isafoldproof}%
%
\isadelimproof
\isanewline
%
\endisadelimproof
\isanewline
\isacommand{lemma}\isamarkupfalse%
\ {\isacharbrackleft}code{\isacharunderscore}post{\isacharbrackright}{\isacharcolon}\isanewline
\ \ {\isachardoublequoteopen}Nat\ {\isadigit{0}}\ {\isacharequal}\ {\isadigit{0}}{\isachardoublequoteclose}\isanewline
\ \ {\isachardoublequoteopen}Nat\ {\isadigit{1}}\ {\isacharequal}\ {\isadigit{1}}{\isachardoublequoteclose}\isanewline
\ \ {\isachardoublequoteopen}Nat\ {\isacharparenleft}numeral\ k{\isacharparenright}\ {\isacharequal}\ numeral\ k{\isachardoublequoteclose}\isanewline
%
\isadelimproof
\ \ %
\endisadelimproof
%
\isatagproof
\isacommand{by}\isamarkupfalse%
\ {\isacharparenleft}transfer{\isacharcomma}\ simp{\isacharparenright}{\isacharplus}%
\endisatagproof
{\isafoldproof}%
%
\isadelimproof
\isanewline
%
\endisadelimproof
\isanewline
\isacommand{lemma}\isamarkupfalse%
\ {\isacharbrackleft}code{\isacharunderscore}abbrev{\isacharbrackright}{\isacharcolon}\isanewline
\ \ {\isachardoublequoteopen}integer{\isacharunderscore}of{\isacharunderscore}nat\ {\isacharequal}\ of{\isacharunderscore}nat{\isachardoublequoteclose}\isanewline
%
\isadelimproof
\ \ %
\endisadelimproof
%
\isatagproof
\isacommand{by}\isamarkupfalse%
\ transfer\ rule%
\endisatagproof
{\isafoldproof}%
%
\isadelimproof
\isanewline
%
\endisadelimproof
\isanewline
\isacommand{lemma}\isamarkupfalse%
\ {\isacharbrackleft}code{\isacharunderscore}unfold{\isacharbrackright}{\isacharcolon}\isanewline
\ \ {\isachardoublequoteopen}Int{\isachardot}nat\ {\isacharparenleft}int{\isacharunderscore}of{\isacharunderscore}integer\ k{\isacharparenright}\ {\isacharequal}\ nat{\isacharunderscore}of{\isacharunderscore}integer\ k{\isachardoublequoteclose}\isanewline
%
\isadelimproof
\ \ %
\endisadelimproof
%
\isatagproof
\isacommand{by}\isamarkupfalse%
\ transfer\ rule%
\endisatagproof
{\isafoldproof}%
%
\isadelimproof
\isanewline
%
\endisadelimproof
\isanewline
\isacommand{lemma}\isamarkupfalse%
\ {\isacharbrackleft}code\ abstype{\isacharbrackright}{\isacharcolon}\isanewline
\ \ {\isachardoublequoteopen}Code{\isacharunderscore}Target{\isacharunderscore}Nat{\isachardot}Nat\ {\isacharparenleft}integer{\isacharunderscore}of{\isacharunderscore}nat\ n{\isacharparenright}\ {\isacharequal}\ n{\isachardoublequoteclose}\isanewline
%
\isadelimproof
\ \ %
\endisadelimproof
%
\isatagproof
\isacommand{by}\isamarkupfalse%
\ transfer\ simp%
\endisatagproof
{\isafoldproof}%
%
\isadelimproof
\isanewline
%
\endisadelimproof
\isanewline
\isacommand{lemma}\isamarkupfalse%
\ {\isacharbrackleft}code\ abstract{\isacharbrackright}{\isacharcolon}\isanewline
\ \ {\isachardoublequoteopen}integer{\isacharunderscore}of{\isacharunderscore}nat\ {\isacharparenleft}nat{\isacharunderscore}of{\isacharunderscore}integer\ k{\isacharparenright}\ {\isacharequal}\ max\ {\isadigit{0}}\ k{\isachardoublequoteclose}\isanewline
%
\isadelimproof
\ \ %
\endisadelimproof
%
\isatagproof
\isacommand{by}\isamarkupfalse%
\ transfer\ auto%
\endisatagproof
{\isafoldproof}%
%
\isadelimproof
\isanewline
%
\endisadelimproof
\isanewline
\isacommand{lemma}\isamarkupfalse%
\ {\isacharbrackleft}code{\isacharunderscore}abbrev{\isacharbrackright}{\isacharcolon}\isanewline
\ \ {\isachardoublequoteopen}nat{\isacharunderscore}of{\isacharunderscore}integer\ {\isacharparenleft}numeral\ k{\isacharparenright}\ {\isacharequal}\ nat{\isacharunderscore}of{\isacharunderscore}num\ k{\isachardoublequoteclose}\isanewline
%
\isadelimproof
\ \ %
\endisadelimproof
%
\isatagproof
\isacommand{by}\isamarkupfalse%
\ transfer\ {\isacharparenleft}simp\ add{\isacharcolon}\ nat{\isacharunderscore}of{\isacharunderscore}num{\isacharunderscore}numeral{\isacharparenright}%
\endisatagproof
{\isafoldproof}%
%
\isadelimproof
\isanewline
%
\endisadelimproof
\isanewline
\isacommand{lemma}\isamarkupfalse%
\ {\isacharbrackleft}code\ abstract{\isacharbrackright}{\isacharcolon}\isanewline
\ \ {\isachardoublequoteopen}integer{\isacharunderscore}of{\isacharunderscore}nat\ {\isacharparenleft}nat{\isacharunderscore}of{\isacharunderscore}num\ n{\isacharparenright}\ {\isacharequal}\ integer{\isacharunderscore}of{\isacharunderscore}num\ n{\isachardoublequoteclose}\isanewline
%
\isadelimproof
\ \ %
\endisadelimproof
%
\isatagproof
\isacommand{by}\isamarkupfalse%
\ transfer\ {\isacharparenleft}simp\ add{\isacharcolon}\ nat{\isacharunderscore}of{\isacharunderscore}num{\isacharunderscore}numeral{\isacharparenright}%
\endisatagproof
{\isafoldproof}%
%
\isadelimproof
\isanewline
%
\endisadelimproof
\isanewline
\isacommand{lemma}\isamarkupfalse%
\ {\isacharbrackleft}code\ abstract{\isacharbrackright}{\isacharcolon}\isanewline
\ \ {\isachardoublequoteopen}integer{\isacharunderscore}of{\isacharunderscore}nat\ {\isadigit{0}}\ {\isacharequal}\ {\isadigit{0}}{\isachardoublequoteclose}\isanewline
%
\isadelimproof
\ \ %
\endisadelimproof
%
\isatagproof
\isacommand{by}\isamarkupfalse%
\ transfer\ simp%
\endisatagproof
{\isafoldproof}%
%
\isadelimproof
\isanewline
%
\endisadelimproof
\isanewline
\isacommand{lemma}\isamarkupfalse%
\ {\isacharbrackleft}code\ abstract{\isacharbrackright}{\isacharcolon}\isanewline
\ \ {\isachardoublequoteopen}integer{\isacharunderscore}of{\isacharunderscore}nat\ {\isadigit{1}}\ {\isacharequal}\ {\isadigit{1}}{\isachardoublequoteclose}\isanewline
%
\isadelimproof
\ \ %
\endisadelimproof
%
\isatagproof
\isacommand{by}\isamarkupfalse%
\ transfer\ simp%
\endisatagproof
{\isafoldproof}%
%
\isadelimproof
\isanewline
%
\endisadelimproof
\isanewline
\isacommand{lemma}\isamarkupfalse%
\ {\isacharbrackleft}code{\isacharbrackright}{\isacharcolon}\isanewline
\ \ {\isachardoublequoteopen}Suc\ n\ {\isacharequal}\ n\ {\isacharplus}\ {\isadigit{1}}{\isachardoublequoteclose}\isanewline
%
\isadelimproof
\ \ %
\endisadelimproof
%
\isatagproof
\isacommand{by}\isamarkupfalse%
\ simp%
\endisatagproof
{\isafoldproof}%
%
\isadelimproof
\isanewline
%
\endisadelimproof
\isanewline
\isacommand{lemma}\isamarkupfalse%
\ {\isacharbrackleft}code\ abstract{\isacharbrackright}{\isacharcolon}\isanewline
\ \ {\isachardoublequoteopen}integer{\isacharunderscore}of{\isacharunderscore}nat\ {\isacharparenleft}m\ {\isacharplus}\ n{\isacharparenright}\ {\isacharequal}\ of{\isacharunderscore}nat\ m\ {\isacharplus}\ of{\isacharunderscore}nat\ n{\isachardoublequoteclose}\isanewline
%
\isadelimproof
\ \ %
\endisadelimproof
%
\isatagproof
\isacommand{by}\isamarkupfalse%
\ transfer\ simp%
\endisatagproof
{\isafoldproof}%
%
\isadelimproof
\isanewline
%
\endisadelimproof
\isanewline
\isacommand{lemma}\isamarkupfalse%
\ {\isacharbrackleft}code\ abstract{\isacharbrackright}{\isacharcolon}\isanewline
\ \ {\isachardoublequoteopen}integer{\isacharunderscore}of{\isacharunderscore}nat\ {\isacharparenleft}m\ {\isacharminus}\ n{\isacharparenright}\ {\isacharequal}\ max\ {\isadigit{0}}\ {\isacharparenleft}of{\isacharunderscore}nat\ m\ {\isacharminus}\ of{\isacharunderscore}nat\ n{\isacharparenright}{\isachardoublequoteclose}\isanewline
%
\isadelimproof
\ \ %
\endisadelimproof
%
\isatagproof
\isacommand{by}\isamarkupfalse%
\ transfer\ simp%
\endisatagproof
{\isafoldproof}%
%
\isadelimproof
\isanewline
%
\endisadelimproof
\isanewline
\isacommand{lemma}\isamarkupfalse%
\ {\isacharbrackleft}code\ abstract{\isacharbrackright}{\isacharcolon}\isanewline
\ \ {\isachardoublequoteopen}integer{\isacharunderscore}of{\isacharunderscore}nat\ {\isacharparenleft}m\ {\isacharasterisk}\ n{\isacharparenright}\ {\isacharequal}\ of{\isacharunderscore}nat\ m\ {\isacharasterisk}\ of{\isacharunderscore}nat\ n{\isachardoublequoteclose}\isanewline
%
\isadelimproof
\ \ %
\endisadelimproof
%
\isatagproof
\isacommand{by}\isamarkupfalse%
\ transfer\ {\isacharparenleft}simp\ add{\isacharcolon}\ of{\isacharunderscore}nat{\isacharunderscore}mult{\isacharparenright}%
\endisatagproof
{\isafoldproof}%
%
\isadelimproof
\isanewline
%
\endisadelimproof
\isanewline
\isacommand{lemma}\isamarkupfalse%
\ {\isacharbrackleft}code\ abstract{\isacharbrackright}{\isacharcolon}\isanewline
\ \ {\isachardoublequoteopen}integer{\isacharunderscore}of{\isacharunderscore}nat\ {\isacharparenleft}m\ div\ n{\isacharparenright}\ {\isacharequal}\ of{\isacharunderscore}nat\ m\ div\ of{\isacharunderscore}nat\ n{\isachardoublequoteclose}\isanewline
%
\isadelimproof
\ \ %
\endisadelimproof
%
\isatagproof
\isacommand{by}\isamarkupfalse%
\ transfer\ {\isacharparenleft}simp\ add{\isacharcolon}\ zdiv{\isacharunderscore}int{\isacharparenright}%
\endisatagproof
{\isafoldproof}%
%
\isadelimproof
\isanewline
%
\endisadelimproof
\isanewline
\isacommand{lemma}\isamarkupfalse%
\ {\isacharbrackleft}code\ abstract{\isacharbrackright}{\isacharcolon}\isanewline
\ \ {\isachardoublequoteopen}integer{\isacharunderscore}of{\isacharunderscore}nat\ {\isacharparenleft}m\ mod\ n{\isacharparenright}\ {\isacharequal}\ of{\isacharunderscore}nat\ m\ mod\ of{\isacharunderscore}nat\ n{\isachardoublequoteclose}\isanewline
%
\isadelimproof
\ \ %
\endisadelimproof
%
\isatagproof
\isacommand{by}\isamarkupfalse%
\ transfer\ {\isacharparenleft}simp\ add{\isacharcolon}\ zmod{\isacharunderscore}int{\isacharparenright}%
\endisatagproof
{\isafoldproof}%
%
\isadelimproof
\isanewline
%
\endisadelimproof
\isanewline
\isacommand{lemma}\isamarkupfalse%
\ {\isacharbrackleft}code{\isacharbrackright}{\isacharcolon}\isanewline
\ \ {\isachardoublequoteopen}Divides{\isachardot}divmod{\isacharunderscore}nat\ m\ n\ {\isacharequal}\ {\isacharparenleft}m\ div\ n{\isacharcomma}\ m\ mod\ n{\isacharparenright}{\isachardoublequoteclose}\isanewline
%
\isadelimproof
\ \ %
\endisadelimproof
%
\isatagproof
\isacommand{by}\isamarkupfalse%
\ {\isacharparenleft}fact\ divmod{\isacharunderscore}nat{\isacharunderscore}div{\isacharunderscore}mod{\isacharparenright}%
\endisatagproof
{\isafoldproof}%
%
\isadelimproof
\isanewline
%
\endisadelimproof
\isanewline
\isacommand{lemma}\isamarkupfalse%
\ {\isacharbrackleft}code{\isacharbrackright}{\isacharcolon}\isanewline
\ \ {\isachardoublequoteopen}HOL{\isachardot}equal\ m\ n\ {\isacharequal}\ HOL{\isachardot}equal\ {\isacharparenleft}of{\isacharunderscore}nat\ m\ {\isacharcolon}{\isacharcolon}\ integer{\isacharparenright}\ {\isacharparenleft}of{\isacharunderscore}nat\ n{\isacharparenright}{\isachardoublequoteclose}\isanewline
%
\isadelimproof
\ \ %
\endisadelimproof
%
\isatagproof
\isacommand{by}\isamarkupfalse%
\ transfer\ {\isacharparenleft}simp\ add{\isacharcolon}\ equal{\isacharparenright}%
\endisatagproof
{\isafoldproof}%
%
\isadelimproof
\isanewline
%
\endisadelimproof
\isanewline
\isacommand{lemma}\isamarkupfalse%
\ {\isacharbrackleft}code{\isacharbrackright}{\isacharcolon}\isanewline
\ \ {\isachardoublequoteopen}m\ {\isasymle}\ n\ {\isasymlongleftrightarrow}\ {\isacharparenleft}of{\isacharunderscore}nat\ m\ {\isacharcolon}{\isacharcolon}\ integer{\isacharparenright}\ {\isasymle}\ of{\isacharunderscore}nat\ n{\isachardoublequoteclose}\isanewline
%
\isadelimproof
\ \ %
\endisadelimproof
%
\isatagproof
\isacommand{by}\isamarkupfalse%
\ simp%
\endisatagproof
{\isafoldproof}%
%
\isadelimproof
\isanewline
%
\endisadelimproof
\isanewline
\isacommand{lemma}\isamarkupfalse%
\ {\isacharbrackleft}code{\isacharbrackright}{\isacharcolon}\isanewline
\ \ {\isachardoublequoteopen}m\ {\isacharless}\ n\ {\isasymlongleftrightarrow}\ {\isacharparenleft}of{\isacharunderscore}nat\ m\ {\isacharcolon}{\isacharcolon}\ integer{\isacharparenright}\ {\isacharless}\ of{\isacharunderscore}nat\ n{\isachardoublequoteclose}\isanewline
%
\isadelimproof
\ \ %
\endisadelimproof
%
\isatagproof
\isacommand{by}\isamarkupfalse%
\ simp%
\endisatagproof
{\isafoldproof}%
%
\isadelimproof
\isanewline
%
\endisadelimproof
\isanewline
\isacommand{lemma}\isamarkupfalse%
\ num{\isacharunderscore}of{\isacharunderscore}nat{\isacharunderscore}code\ {\isacharbrackleft}code{\isacharbrackright}{\isacharcolon}\isanewline
\ \ {\isachardoublequoteopen}num{\isacharunderscore}of{\isacharunderscore}nat\ {\isacharequal}\ num{\isacharunderscore}of{\isacharunderscore}integer\ {\isasymcirc}\ of{\isacharunderscore}nat{\isachardoublequoteclose}\isanewline
%
\isadelimproof
\ \ %
\endisadelimproof
%
\isatagproof
\isacommand{by}\isamarkupfalse%
\ transfer\ {\isacharparenleft}simp\ add{\isacharcolon}\ fun{\isacharunderscore}eq{\isacharunderscore}iff{\isacharparenright}%
\endisatagproof
{\isafoldproof}%
%
\isadelimproof
\isanewline
%
\endisadelimproof
\isanewline
\isacommand{end}\isamarkupfalse%
\isanewline
\isanewline
\isacommand{lemma}\isamarkupfalse%
\ {\isacharparenleft}\isakeyword{in}\ semiring{\isacharunderscore}{\isadigit{1}}{\isacharparenright}\ of{\isacharunderscore}nat{\isacharunderscore}code{\isacharunderscore}if{\isacharcolon}\isanewline
\ \ {\isachardoublequoteopen}of{\isacharunderscore}nat\ n\ {\isacharequal}\ {\isacharparenleft}if\ n\ {\isacharequal}\ {\isadigit{0}}\ then\ {\isadigit{0}}\isanewline
\ \ \ \ \ else\ let\isanewline
\ \ \ \ \ \ \ {\isacharparenleft}m{\isacharcomma}\ q{\isacharparenright}\ {\isacharequal}\ divmod{\isacharunderscore}nat\ n\ {\isadigit{2}}{\isacharsemicolon}\isanewline
\ \ \ \ \ \ \ m{\isacharprime}\ {\isacharequal}\ {\isadigit{2}}\ {\isacharasterisk}\ of{\isacharunderscore}nat\ m\isanewline
\ \ \ \ \ in\ if\ q\ {\isacharequal}\ {\isadigit{0}}\ then\ m{\isacharprime}\ else\ m{\isacharprime}\ {\isacharplus}\ {\isadigit{1}}{\isacharparenright}{\isachardoublequoteclose}\isanewline
%
\isadelimproof
%
\endisadelimproof
%
\isatagproof
\isacommand{proof}\isamarkupfalse%
\ {\isacharminus}\isanewline
\ \ \isacommand{from}\isamarkupfalse%
\ mod{\isacharunderscore}div{\isacharunderscore}equality\ \isacommand{have}\isamarkupfalse%
\ {\isacharasterisk}{\isacharcolon}\ {\isachardoublequoteopen}of{\isacharunderscore}nat\ n\ {\isacharequal}\ of{\isacharunderscore}nat\ {\isacharparenleft}n\ div\ {\isadigit{2}}\ {\isacharasterisk}\ {\isadigit{2}}\ {\isacharplus}\ n\ mod\ {\isadigit{2}}{\isacharparenright}{\isachardoublequoteclose}\ \isacommand{by}\isamarkupfalse%
\ simp\isanewline
\ \ \isacommand{show}\isamarkupfalse%
\ {\isacharquery}thesis\isanewline
\ \ \ \ \isacommand{by}\isamarkupfalse%
\ {\isacharparenleft}simp\ add{\isacharcolon}\ Let{\isacharunderscore}def\ divmod{\isacharunderscore}nat{\isacharunderscore}div{\isacharunderscore}mod\ of{\isacharunderscore}nat{\isacharunderscore}add\ {\isacharbrackleft}symmetric{\isacharbrackright}{\isacharparenright}\isanewline
\ \ \ \ \ \ {\isacharparenleft}simp\ add{\isacharcolon}\ {\isacharasterisk}\ mult{\isachardot}commute\ of{\isacharunderscore}nat{\isacharunderscore}mult\ add{\isachardot}commute{\isacharparenright}\isanewline
\isacommand{qed}\isamarkupfalse%
%
\endisatagproof
{\isafoldproof}%
%
\isadelimproof
\isanewline
%
\endisadelimproof
\isanewline
\isacommand{declare}\isamarkupfalse%
\ of{\isacharunderscore}nat{\isacharunderscore}code{\isacharunderscore}if\ {\isacharbrackleft}code{\isacharbrackright}\isanewline
\isanewline
\isacommand{definition}\isamarkupfalse%
\ int{\isacharunderscore}of{\isacharunderscore}nat\ {\isacharcolon}{\isacharcolon}\ {\isachardoublequoteopen}nat\ {\isasymRightarrow}\ int{\isachardoublequoteclose}\ \isakeyword{where}\isanewline
\ \ {\isacharbrackleft}code{\isacharunderscore}abbrev{\isacharbrackright}{\isacharcolon}\ {\isachardoublequoteopen}int{\isacharunderscore}of{\isacharunderscore}nat\ {\isacharequal}\ of{\isacharunderscore}nat{\isachardoublequoteclose}\isanewline
\isanewline
\isacommand{lemma}\isamarkupfalse%
\ {\isacharbrackleft}code{\isacharbrackright}{\isacharcolon}\isanewline
\ \ {\isachardoublequoteopen}int{\isacharunderscore}of{\isacharunderscore}nat\ n\ {\isacharequal}\ int{\isacharunderscore}of{\isacharunderscore}integer\ {\isacharparenleft}of{\isacharunderscore}nat\ n{\isacharparenright}{\isachardoublequoteclose}\isanewline
%
\isadelimproof
\ \ %
\endisadelimproof
%
\isatagproof
\isacommand{by}\isamarkupfalse%
\ {\isacharparenleft}simp\ add{\isacharcolon}\ int{\isacharunderscore}of{\isacharunderscore}nat{\isacharunderscore}def{\isacharparenright}%
\endisatagproof
{\isafoldproof}%
%
\isadelimproof
\isanewline
%
\endisadelimproof
\isanewline
\isacommand{lemma}\isamarkupfalse%
\ {\isacharbrackleft}code\ abstract{\isacharbrackright}{\isacharcolon}\isanewline
\ \ {\isachardoublequoteopen}integer{\isacharunderscore}of{\isacharunderscore}nat\ {\isacharparenleft}nat\ k{\isacharparenright}\ {\isacharequal}\ max\ {\isadigit{0}}\ {\isacharparenleft}integer{\isacharunderscore}of{\isacharunderscore}int\ k{\isacharparenright}{\isachardoublequoteclose}\isanewline
\ \ \isacommand{including}\isamarkupfalse%
\ integer{\isachardot}lifting%
\isadelimproof
\ %
\endisadelimproof
%
\isatagproof
\isacommand{by}\isamarkupfalse%
\ transfer\ auto%
\endisatagproof
{\isafoldproof}%
%
\isadelimproof
%
\endisadelimproof
\isanewline
\isanewline
\isacommand{lemma}\isamarkupfalse%
\ term{\isacharunderscore}of{\isacharunderscore}nat{\isacharunderscore}code\ {\isacharbrackleft}code{\isacharbrackright}{\isacharcolon}\isanewline
\ \ %
\isamarkupcmt{Use \isa{nat{\isacharunderscore}of{\isacharunderscore}integer} in term reconstruction
        instead of \isa{Code{\isacharunderscore}Target{\isacharunderscore}Nat{\isachardot}Nat} such that reconstructed
        terms can be fed back to the code generator%
}
\isanewline
\ \ {\isachardoublequoteopen}term{\isacharunderscore}of{\isacharunderscore}class{\isachardot}term{\isacharunderscore}of\ n\ {\isacharequal}\isanewline
\ \ \ Code{\isacharunderscore}Evaluation{\isachardot}App\isanewline
\ \ \ \ \ {\isacharparenleft}Code{\isacharunderscore}Evaluation{\isachardot}Const\ {\isacharparenleft}STR\ {\isacharprime}{\isacharprime}Code{\isacharunderscore}Numeral{\isachardot}nat{\isacharunderscore}of{\isacharunderscore}integer{\isacharprime}{\isacharprime}{\isacharparenright}\isanewline
\ \ \ \ \ \ \ \ {\isacharparenleft}typerep{\isachardot}Typerep\ {\isacharparenleft}STR\ {\isacharprime}{\isacharprime}fun{\isacharprime}{\isacharprime}{\isacharparenright}\isanewline
\ \ \ \ \ \ \ \ \ \ \ {\isacharbrackleft}typerep{\isachardot}Typerep\ {\isacharparenleft}STR\ {\isacharprime}{\isacharprime}Code{\isacharunderscore}Numeral{\isachardot}integer{\isacharprime}{\isacharprime}{\isacharparenright}\ {\isacharbrackleft}{\isacharbrackright}{\isacharcomma}\isanewline
\ \ \ \ \ \ \ \ \ typerep{\isachardot}Typerep\ {\isacharparenleft}STR\ {\isacharprime}{\isacharprime}Nat{\isachardot}nat{\isacharprime}{\isacharprime}{\isacharparenright}\ {\isacharbrackleft}{\isacharbrackright}{\isacharbrackright}{\isacharparenright}{\isacharparenright}\isanewline
\ \ \ \ \ {\isacharparenleft}term{\isacharunderscore}of{\isacharunderscore}class{\isachardot}term{\isacharunderscore}of\ {\isacharparenleft}integer{\isacharunderscore}of{\isacharunderscore}nat\ n{\isacharparenright}{\isacharparenright}{\isachardoublequoteclose}\isanewline
%
\isadelimproof
\ \ %
\endisadelimproof
%
\isatagproof
\isacommand{by}\isamarkupfalse%
\ {\isacharparenleft}simp\ add{\isacharcolon}\ term{\isacharunderscore}of{\isacharunderscore}anything{\isacharparenright}%
\endisatagproof
{\isafoldproof}%
%
\isadelimproof
\isanewline
%
\endisadelimproof
\isanewline
\isacommand{lemma}\isamarkupfalse%
\ nat{\isacharunderscore}of{\isacharunderscore}integer{\isacharunderscore}code{\isacharunderscore}post\ {\isacharbrackleft}code{\isacharunderscore}post{\isacharbrackright}{\isacharcolon}\isanewline
\ \ {\isachardoublequoteopen}nat{\isacharunderscore}of{\isacharunderscore}integer\ {\isadigit{0}}\ {\isacharequal}\ {\isadigit{0}}{\isachardoublequoteclose}\isanewline
\ \ {\isachardoublequoteopen}nat{\isacharunderscore}of{\isacharunderscore}integer\ {\isadigit{1}}\ {\isacharequal}\ {\isadigit{1}}{\isachardoublequoteclose}\isanewline
\ \ {\isachardoublequoteopen}nat{\isacharunderscore}of{\isacharunderscore}integer\ {\isacharparenleft}numeral\ k{\isacharparenright}\ {\isacharequal}\ numeral\ k{\isachardoublequoteclose}\isanewline
\ \ \isacommand{including}\isamarkupfalse%
\ integer{\isachardot}lifting%
\isadelimproof
\ %
\endisadelimproof
%
\isatagproof
\isacommand{by}\isamarkupfalse%
\ {\isacharparenleft}transfer{\isacharcomma}\ simp{\isacharparenright}{\isacharplus}%
\endisatagproof
{\isafoldproof}%
%
\isadelimproof
%
\endisadelimproof
\isanewline
\isanewline
\isacommand{code{\isacharunderscore}identifier}\isamarkupfalse%
\isanewline
\ \ \isakeyword{code{\isacharunderscore}module}\ Code{\isacharunderscore}Target{\isacharunderscore}Nat\ {\isasymrightharpoonup}\isanewline
\ \ \ \ {\isacharparenleft}SML{\isacharparenright}\ Arith\ \isakeyword{and}\ {\isacharparenleft}OCaml{\isacharparenright}\ Arith\ \isakeyword{and}\ {\isacharparenleft}Haskell{\isacharparenright}\ Arith\isanewline
%
\isadelimtheory
\isanewline
%
\endisadelimtheory
%
\isatagtheory
\isacommand{end}\isamarkupfalse%
%
\endisatagtheory
{\isafoldtheory}%
%
\isadelimtheory
%
\endisadelimtheory
\end{isabellebody}%
%%% Local Variables:
%%% mode: latex
%%% TeX-master: "root"
%%% End:


%
\begin{isabellebody}%
\def\isabellecontext{RelationOperators}%
%
\isamarkupheader{Additional operators on relations, going beyond Relations.thy,
  and properties of these operators%
}
\isamarkuptrue%
%
\isadelimtheory
%
\endisadelimtheory
%
\isatagtheory
\isacommand{theory}\isamarkupfalse%
\ RelationOperators\isanewline
\isakeyword{imports}\isanewline
\ \ Main\isanewline
\ \ SetUtils\isanewline
\ \ {\isachardoublequoteopen}{\isachartilde}{\isachartilde}{\isacharslash}src{\isacharslash}HOL{\isacharslash}Library{\isacharslash}Code{\isacharunderscore}Target{\isacharunderscore}Nat{\isachardoublequoteclose}\isanewline
\isanewline
\isakeyword{begin}%
\endisatagtheory
{\isafoldtheory}%
%
\isadelimtheory
%
\endisadelimtheory
%
\isamarkupsection{evaluating a relation as a function%
}
\isamarkuptrue%
%
\begin{isamarkuptext}%
If an input has a unique image element under a given relation, return that element; 
  otherwise return a fallback value.%
\end{isamarkuptext}%
\isamarkuptrue%
\isacommand{fun}\isamarkupfalse%
\ eval{\isacharunderscore}rel{\isacharunderscore}or\ {\isacharcolon}{\isacharcolon}\ {\isachardoublequoteopen}{\isacharparenleft}{\isacharprime}a\ {\isasymtimes}\ {\isacharprime}b{\isacharparenright}\ set\ {\isasymRightarrow}\ {\isacharprime}a\ {\isasymRightarrow}\ {\isacharprime}b\ {\isasymRightarrow}\ {\isacharprime}b{\isachardoublequoteclose}\isanewline
\ \ \isakeyword{where}\ {\isachardoublequoteopen}eval{\isacharunderscore}rel{\isacharunderscore}or\ R\ a\ z\ {\isacharequal}\ {\isacharparenleft}let\ im\ {\isacharequal}\ R\ {\isacharbackquote}{\isacharbackquote}\ {\isacharbraceleft}a{\isacharbraceright}\ in\ if\ card\ im\ {\isacharequal}\ {\isadigit{1}}\ then\ the{\isacharunderscore}elem\ im\ else\ z{\isacharparenright}{\isachardoublequoteclose}%
\begin{isamarkuptext}%
right-uniqueness of a relation: the image of a \isa{trivial} set (i.e.\ an empty or
  singleton set) under the relation is trivial again. 
This is the set-theoretical way of characterizing functions, as opposed to \isa{{\isasymlambda}} functions.%
\end{isamarkuptext}%
\isamarkuptrue%
\isacommand{definition}\isamarkupfalse%
\ runiq\ {\isacharcolon}{\isacharcolon}\ {\isachardoublequoteopen}{\isacharparenleft}{\isacharprime}a\ {\isasymtimes}\ {\isacharprime}b{\isacharparenright}\ set\ {\isasymRightarrow}\ bool{\isachardoublequoteclose}\ \isanewline
\ \ \isakeyword{where}\ {\isachardoublequoteopen}runiq\ R\ {\isacharequal}\ {\isacharparenleft}{\isasymforall}\ X\ {\isachardot}\ trivial\ X\ {\isasymlongrightarrow}\ trivial\ {\isacharparenleft}R\ {\isacharbackquote}{\isacharbackquote}\ X{\isacharparenright}{\isacharparenright}{\isachardoublequoteclose}%
\isamarkupsection{restriction%
}
\isamarkuptrue%
%
\begin{isamarkuptext}%
restriction of a relation to a set (usually resulting in a relation with a smaller domain)%
\end{isamarkuptext}%
\isamarkuptrue%
\isacommand{definition}\isamarkupfalse%
\ restrict\ {\isacharcolon}{\isacharcolon}\ {\isachardoublequoteopen}{\isacharparenleft}{\isacharprime}a\ {\isasymtimes}\ {\isacharprime}b{\isacharparenright}\ set\ {\isasymRightarrow}\ {\isacharprime}a\ set\ {\isasymRightarrow}\ {\isacharparenleft}{\isacharprime}a\ {\isasymtimes}\ {\isacharprime}b{\isacharparenright}\ set{\isachardoublequoteclose}\ {\isacharparenleft}\isakeyword{infix}\ {\isachardoublequoteopen}{\isacharbar}{\isacharbar}{\isachardoublequoteclose}\ {\isadigit{7}}{\isadigit{5}}{\isacharparenright}\isanewline
\ \ \isakeyword{where}\ {\isachardoublequoteopen}R\ {\isacharbar}{\isacharbar}\ X\ {\isacharequal}\ {\isacharparenleft}X\ {\isasymtimes}\ Range\ R{\isacharparenright}\ {\isasyminter}\ R{\isachardoublequoteclose}%
\begin{isamarkuptext}%
extensional characterization of the pairs within a restricted relation%
\end{isamarkuptext}%
\isamarkuptrue%
\isacommand{lemma}\isamarkupfalse%
\ restrict{\isacharunderscore}ext{\isacharcolon}\ {\isachardoublequoteopen}R\ {\isacharbar}{\isacharbar}\ X\ {\isacharequal}\ {\isacharbraceleft}{\isacharparenleft}x{\isacharcomma}\ y{\isacharparenright}\ {\isacharbar}\ x\ y\ {\isachardot}\ x\ {\isasymin}\ X\ {\isasymand}\ {\isacharparenleft}x{\isacharcomma}\ y{\isacharparenright}\ {\isasymin}\ R{\isacharbraceright}{\isachardoublequoteclose}\isanewline
%
\isadelimproof
\ \ \ \ \ \ %
\endisadelimproof
%
\isatagproof
\isacommand{unfolding}\isamarkupfalse%
\ restrict{\isacharunderscore}def\ \isacommand{using}\isamarkupfalse%
\ Range{\isacharunderscore}iff\ \isacommand{by}\isamarkupfalse%
\ blast%
\endisatagproof
{\isafoldproof}%
%
\isadelimproof
%
\endisadelimproof
%
\begin{isamarkuptext}%
alternative statement of \isa{{\isacharquery}R\ {\isacharbar}{\isacharbar}\ {\isacharquery}X\ {\isacharequal}\ {\isacharbraceleft}{\isacharparenleft}x{\isacharcomma}\ y{\isacharparenright}\ {\isacharbar}x\ y{\isachardot}\ x\ {\isasymin}\ {\isacharquery}X\ {\isasymand}\ {\isacharparenleft}x{\isacharcomma}\ y{\isacharparenright}\ {\isasymin}\ {\isacharquery}R{\isacharbraceright}} without explicitly naming the pair's components%
\end{isamarkuptext}%
\isamarkuptrue%
\isacommand{lemma}\isamarkupfalse%
\ restrict{\isacharunderscore}ext{\isacharprime}{\isacharcolon}\ {\isachardoublequoteopen}R\ {\isacharbar}{\isacharbar}\ X\ {\isacharequal}\ {\isacharbraceleft}p\ {\isachardot}\ fst\ p\ {\isasymin}\ X\ {\isasymand}\ p\ {\isasymin}\ R{\isacharbraceright}{\isachardoublequoteclose}\isanewline
%
\isadelimproof
%
\endisadelimproof
%
\isatagproof
\isacommand{proof}\isamarkupfalse%
\ {\isacharminus}\isanewline
\ \ \isacommand{have}\isamarkupfalse%
\ {\isachardoublequoteopen}R\ {\isacharbar}{\isacharbar}\ X\ {\isacharequal}\ {\isacharbraceleft}{\isacharparenleft}x{\isacharcomma}\ y{\isacharparenright}\ {\isacharbar}\ x\ y\ {\isachardot}\ x\ {\isasymin}\ X\ {\isasymand}\ {\isacharparenleft}x{\isacharcomma}\ y{\isacharparenright}\ {\isasymin}\ R{\isacharbraceright}{\isachardoublequoteclose}\ \isacommand{by}\isamarkupfalse%
\ {\isacharparenleft}rule\ restrict{\isacharunderscore}ext{\isacharparenright}\isanewline
\ \ \isacommand{also}\isamarkupfalse%
\ \isacommand{have}\isamarkupfalse%
\ {\isachardoublequoteopen}{\isasymdots}\ {\isacharequal}\ {\isacharbraceleft}\ p\ {\isachardot}\ fst\ p\ {\isasymin}\ X\ {\isasymand}\ p\ {\isasymin}\ R\ {\isacharbraceright}{\isachardoublequoteclose}\ \isacommand{by}\isamarkupfalse%
\ force\isanewline
\ \ \isacommand{finally}\isamarkupfalse%
\ \isacommand{show}\isamarkupfalse%
\ {\isacharquery}thesis\ \isacommand{{\isachardot}}\isamarkupfalse%
\isanewline
\isacommand{qed}\isamarkupfalse%
%
\endisatagproof
{\isafoldproof}%
%
\isadelimproof
%
\endisadelimproof
%
\begin{isamarkuptext}%
Restricting a relation to the empty set yields the empty set.%
\end{isamarkuptext}%
\isamarkuptrue%
\isacommand{lemma}\isamarkupfalse%
\ restrict{\isacharunderscore}empty{\isacharcolon}\ {\isachardoublequoteopen}P\ {\isacharbar}{\isacharbar}\ {\isacharbraceleft}{\isacharbraceright}\ {\isacharequal}\ {\isacharbraceleft}{\isacharbraceright}{\isachardoublequoteclose}\ \isanewline
%
\isadelimproof
\ \ \ \ \ \ %
\endisadelimproof
%
\isatagproof
\isacommand{unfolding}\isamarkupfalse%
\ restrict{\isacharunderscore}def\ \isacommand{by}\isamarkupfalse%
\ simp%
\endisatagproof
{\isafoldproof}%
%
\isadelimproof
%
\endisadelimproof
%
\begin{isamarkuptext}%
A restriction is a subrelation of the original relation.%
\end{isamarkuptext}%
\isamarkuptrue%
\isacommand{lemma}\isamarkupfalse%
\ restriction{\isacharunderscore}is{\isacharunderscore}subrel{\isacharcolon}\ {\isachardoublequoteopen}P\ {\isacharbar}{\isacharbar}\ X\ {\isasymsubseteq}\ P{\isachardoublequoteclose}\ \isanewline
%
\isadelimproof
\ \ \ \ \ \ %
\endisadelimproof
%
\isatagproof
\isacommand{using}\isamarkupfalse%
\ restrict{\isacharunderscore}def\ \isacommand{by}\isamarkupfalse%
\ blast%
\endisatagproof
{\isafoldproof}%
%
\isadelimproof
%
\endisadelimproof
%
\begin{isamarkuptext}%
Restricting a relation only has an effect within its domain.%
\end{isamarkuptext}%
\isamarkuptrue%
\isacommand{lemma}\isamarkupfalse%
\ restriction{\isacharunderscore}within{\isacharunderscore}domain{\isacharcolon}\ {\isachardoublequoteopen}P\ {\isacharbar}{\isacharbar}\ X\ {\isacharequal}\ P\ {\isacharbar}{\isacharbar}\ {\isacharparenleft}X\ {\isasyminter}\ {\isacharparenleft}Domain\ P{\isacharparenright}{\isacharparenright}{\isachardoublequoteclose}\ \isanewline
%
\isadelimproof
\ \ \ \ \ \ %
\endisadelimproof
%
\isatagproof
\isacommand{unfolding}\isamarkupfalse%
\ restrict{\isacharunderscore}def\ \isacommand{by}\isamarkupfalse%
\ fast%
\endisatagproof
{\isafoldproof}%
%
\isadelimproof
%
\endisadelimproof
%
\begin{isamarkuptext}%
alternative characterization of the restriction of a relation to a singleton set%
\end{isamarkuptext}%
\isamarkuptrue%
\isacommand{lemma}\isamarkupfalse%
\ restrict{\isacharunderscore}to{\isacharunderscore}singleton{\isacharcolon}\ {\isachardoublequoteopen}P\ {\isacharbar}{\isacharbar}\ {\isacharbraceleft}x{\isacharbraceright}\ {\isacharequal}\ {\isacharbraceleft}x{\isacharbraceright}\ {\isasymtimes}\ {\isacharparenleft}P\ {\isacharbackquote}{\isacharbackquote}\ {\isacharbraceleft}x{\isacharbraceright}{\isacharparenright}{\isachardoublequoteclose}\ \isanewline
%
\isadelimproof
\ \ \ \ \ \ %
\endisadelimproof
%
\isatagproof
\isacommand{unfolding}\isamarkupfalse%
\ restrict{\isacharunderscore}def\ \isacommand{by}\isamarkupfalse%
\ fast%
\endisatagproof
{\isafoldproof}%
%
\isadelimproof
%
\endisadelimproof
%
\isamarkupsection{relation outside some set%
}
\isamarkuptrue%
%
\begin{isamarkuptext}%
For a set-theoretical relation \isa{R} and an ``exclusion'' set \isa{X}, return those
  tuples of \isa{R} whose first component is not in \isa{X}.  In other words, exclude \isa{X}
  from the domain of \isa{R}.%
\end{isamarkuptext}%
\isamarkuptrue%
\isacommand{definition}\isamarkupfalse%
\ Outside\ {\isacharcolon}{\isacharcolon}\ {\isachardoublequoteopen}{\isacharparenleft}{\isacharprime}a\ {\isasymtimes}\ {\isacharprime}b{\isacharparenright}\ set\ {\isasymRightarrow}\ {\isacharprime}a\ set\ {\isasymRightarrow}\ {\isacharparenleft}{\isacharprime}a\ {\isasymtimes}\ {\isacharprime}b{\isacharparenright}\ set{\isachardoublequoteclose}\ {\isacharparenleft}\isakeyword{infix}\ {\isachardoublequoteopen}outside{\isachardoublequoteclose}\ {\isadigit{7}}{\isadigit{5}}{\isacharparenright}\isanewline
\ \ \ \isakeyword{where}\ {\isachardoublequoteopen}R\ outside\ X\ {\isacharequal}\ R\ {\isacharminus}\ {\isacharparenleft}X\ {\isasymtimes}\ Range\ R{\isacharparenright}{\isachardoublequoteclose}%
\begin{isamarkuptext}%
Considering a relation outside some set \isa{X} reduces its domain by \isa{X}.%
\end{isamarkuptext}%
\isamarkuptrue%
\isacommand{lemma}\isamarkupfalse%
\ outside{\isacharunderscore}reduces{\isacharunderscore}domain{\isacharcolon}\ {\isachardoublequoteopen}Domain\ {\isacharparenleft}P\ outside\ X{\isacharparenright}\ {\isacharequal}\ {\isacharparenleft}Domain\ P{\isacharparenright}\ {\isacharminus}\ X{\isachardoublequoteclose}\isanewline
%
\isadelimproof
\ \ \ \ \ \ %
\endisadelimproof
%
\isatagproof
\isacommand{unfolding}\isamarkupfalse%
\ Outside{\isacharunderscore}def\ \isacommand{by}\isamarkupfalse%
\ fast%
\endisatagproof
{\isafoldproof}%
%
\isadelimproof
%
\endisadelimproof
%
\begin{isamarkuptext}%
Considering a relation outside a singleton set \isa{{\isacharbraceleft}x{\isacharbraceright}} reduces its domain by 
  \isa{x}.%
\end{isamarkuptext}%
\isamarkuptrue%
\isacommand{corollary}\isamarkupfalse%
\ Domain{\isacharunderscore}outside{\isacharunderscore}singleton{\isacharcolon}\isanewline
\ \ \isakeyword{assumes}\ {\isachardoublequoteopen}Domain\ R\ {\isacharequal}\ insert\ x\ A{\isachardoublequoteclose}\isanewline
\ \ \ \ \ \ \isakeyword{and}\ {\isachardoublequoteopen}x\ {\isasymnotin}\ A{\isachardoublequoteclose}\isanewline
\ \ \isakeyword{shows}\ {\isachardoublequoteopen}Domain\ {\isacharparenleft}R\ outside\ {\isacharbraceleft}x{\isacharbraceright}{\isacharparenright}\ {\isacharequal}\ A{\isachardoublequoteclose}\isanewline
%
\isadelimproof
\ \ %
\endisadelimproof
%
\isatagproof
\isacommand{using}\isamarkupfalse%
\ assms\ outside{\isacharunderscore}reduces{\isacharunderscore}domain\ \isacommand{by}\isamarkupfalse%
\ {\isacharparenleft}metis\ Diff{\isacharunderscore}insert{\isacharunderscore}absorb{\isacharparenright}%
\endisatagproof
{\isafoldproof}%
%
\isadelimproof
%
\endisadelimproof
%
\begin{isamarkuptext}%
For any set, a relation equals the union of its restriction to that set and its
  pairs outside that set.%
\end{isamarkuptext}%
\isamarkuptrue%
\isacommand{lemma}\isamarkupfalse%
\ outside{\isacharunderscore}union{\isacharunderscore}restrict{\isacharcolon}\ {\isachardoublequoteopen}P\ {\isacharequal}\ {\isacharparenleft}P\ outside\ X{\isacharparenright}\ {\isasymunion}\ {\isacharparenleft}P\ {\isacharbar}{\isacharbar}\ X{\isacharparenright}{\isachardoublequoteclose}\isanewline
%
\isadelimproof
\ \ \ \ \ \ %
\endisadelimproof
%
\isatagproof
\isacommand{unfolding}\isamarkupfalse%
\ Outside{\isacharunderscore}def\ restrict{\isacharunderscore}def\ \isacommand{by}\isamarkupfalse%
\ fast%
\endisatagproof
{\isafoldproof}%
%
\isadelimproof
%
\endisadelimproof
%
\begin{isamarkuptext}%
The range of a relation \isa{R} outside some exclusion set \isa{X} is a 
  subset of the image of the domain of \isa{R}, minus \isa{X}, under \isa{R}.%
\end{isamarkuptext}%
\isamarkuptrue%
\isacommand{lemma}\isamarkupfalse%
\ Range{\isacharunderscore}outside{\isacharunderscore}sub{\isacharunderscore}Image{\isacharunderscore}Domain{\isacharcolon}\ {\isachardoublequoteopen}Range\ {\isacharparenleft}R\ outside\ X{\isacharparenright}\ {\isasymsubseteq}\ R\ {\isacharbackquote}{\isacharbackquote}\ {\isacharparenleft}Domain\ R\ {\isacharminus}\ X{\isacharparenright}{\isachardoublequoteclose}\isanewline
%
\isadelimproof
\ \ \ \ \ \ %
\endisadelimproof
%
\isatagproof
\isacommand{using}\isamarkupfalse%
\ Outside{\isacharunderscore}def\ Image{\isacharunderscore}def\ Domain{\isacharunderscore}def\ Range{\isacharunderscore}def\ \isacommand{by}\isamarkupfalse%
\ blast%
\endisatagproof
{\isafoldproof}%
%
\isadelimproof
%
\endisadelimproof
%
\begin{isamarkuptext}%
Considering a relation outside some set does not enlarge its range.%
\end{isamarkuptext}%
\isamarkuptrue%
\isacommand{lemma}\isamarkupfalse%
\ Range{\isacharunderscore}outside{\isacharunderscore}sub{\isacharcolon}\isanewline
\ \ \isakeyword{assumes}\ {\isachardoublequoteopen}Range\ R\ {\isasymsubseteq}\ Y{\isachardoublequoteclose}\isanewline
\ \ \isakeyword{shows}\ {\isachardoublequoteopen}Range\ {\isacharparenleft}R\ outside\ X{\isacharparenright}\ {\isasymsubseteq}\ Y{\isachardoublequoteclose}\isanewline
%
\isadelimproof
\ \ %
\endisadelimproof
%
\isatagproof
\isacommand{using}\isamarkupfalse%
\ assms\ outside{\isacharunderscore}union{\isacharunderscore}restrict\ \isacommand{by}\isamarkupfalse%
\ {\isacharparenleft}metis\ Range{\isacharunderscore}mono\ inf{\isacharunderscore}sup{\isacharunderscore}ord{\isacharparenleft}{\isadigit{3}}{\isacharparenright}\ subset{\isacharunderscore}trans{\isacharparenright}%
\endisatagproof
{\isafoldproof}%
%
\isadelimproof
%
\endisadelimproof
%
\isamarkupsection{flipping pairs of relations%
}
\isamarkuptrue%
%
\begin{isamarkuptext}%
flipping a pair: exchanging first and second component%
\end{isamarkuptext}%
\isamarkuptrue%
\isacommand{definition}\isamarkupfalse%
\ flip\ \isakeyword{where}\ {\isachardoublequoteopen}flip\ tup\ {\isacharequal}\ {\isacharparenleft}snd\ tup{\isacharcomma}\ fst\ tup{\isacharparenright}{\isachardoublequoteclose}%
\begin{isamarkuptext}%
Flipped pairs can be found in the converse relation.%
\end{isamarkuptext}%
\isamarkuptrue%
\isacommand{lemma}\isamarkupfalse%
\ flip{\isacharunderscore}in{\isacharunderscore}conv{\isacharcolon}\isanewline
\ \ \isakeyword{assumes}\ {\isachardoublequoteopen}tup\ {\isasymin}\ R{\isachardoublequoteclose}\isanewline
\ \ \isakeyword{shows}\ {\isachardoublequoteopen}flip\ tup\ {\isasymin}\ R{\isasyminverse}{\isachardoublequoteclose}\isanewline
%
\isadelimproof
\ \ %
\endisadelimproof
%
\isatagproof
\isacommand{using}\isamarkupfalse%
\ assms\ \isacommand{unfolding}\isamarkupfalse%
\ flip{\isacharunderscore}def\ \isacommand{by}\isamarkupfalse%
\ simp%
\endisatagproof
{\isafoldproof}%
%
\isadelimproof
%
\endisadelimproof
%
\begin{isamarkuptext}%
Flipping a pair twice doesn't change it.%
\end{isamarkuptext}%
\isamarkuptrue%
\isacommand{lemma}\isamarkupfalse%
\ flip{\isacharunderscore}flip{\isacharcolon}\ {\isachardoublequoteopen}flip\ {\isacharparenleft}flip\ tup{\isacharparenright}\ {\isacharequal}\ tup{\isachardoublequoteclose}\isanewline
%
\isadelimproof
\ \ %
\endisadelimproof
%
\isatagproof
\isacommand{by}\isamarkupfalse%
\ {\isacharparenleft}metis\ flip{\isacharunderscore}def\ fst{\isacharunderscore}conv\ snd{\isacharunderscore}conv\ surjective{\isacharunderscore}pairing{\isacharparenright}%
\endisatagproof
{\isafoldproof}%
%
\isadelimproof
%
\endisadelimproof
%
\begin{isamarkuptext}%
Flipping all pairs in a relation yields the converse relation.%
\end{isamarkuptext}%
\isamarkuptrue%
\isacommand{lemma}\isamarkupfalse%
\ flip{\isacharunderscore}conv{\isacharcolon}\ {\isachardoublequoteopen}flip\ {\isacharbackquote}\ R\ {\isacharequal}\ R{\isasyminverse}{\isachardoublequoteclose}\isanewline
%
\isadelimproof
%
\endisadelimproof
%
\isatagproof
\isacommand{proof}\isamarkupfalse%
\ {\isacharminus}\isanewline
\ \ \isacommand{have}\isamarkupfalse%
\ {\isachardoublequoteopen}flip\ {\isacharbackquote}\ R\ {\isacharequal}\ {\isacharbraceleft}\ flip\ tup\ {\isacharbar}\ tup\ {\isachardot}\ tup\ {\isasymin}\ R\ {\isacharbraceright}{\isachardoublequoteclose}\ \isacommand{by}\isamarkupfalse%
\ {\isacharparenleft}metis\ image{\isacharunderscore}Collect{\isacharunderscore}mem{\isacharparenright}\isanewline
\ \ \isacommand{also}\isamarkupfalse%
\ \isacommand{have}\isamarkupfalse%
\ {\isachardoublequoteopen}{\isasymdots}\ {\isacharequal}\ {\isacharbraceleft}\ tup\ {\isachardot}\ tup\ {\isasymin}\ R{\isasyminverse}\ {\isacharbraceright}{\isachardoublequoteclose}\ \isacommand{using}\isamarkupfalse%
\ flip{\isacharunderscore}in{\isacharunderscore}conv\ \isacommand{by}\isamarkupfalse%
\ {\isacharparenleft}metis\ converse{\isacharunderscore}converse\ flip{\isacharunderscore}flip{\isacharparenright}\isanewline
\ \ \isacommand{also}\isamarkupfalse%
\ \isacommand{have}\isamarkupfalse%
\ {\isachardoublequoteopen}{\isasymdots}\ {\isacharequal}\ R{\isasyminverse}{\isachardoublequoteclose}\ \isacommand{by}\isamarkupfalse%
\ simp\isanewline
\ \ \isacommand{finally}\isamarkupfalse%
\ \isacommand{show}\isamarkupfalse%
\ {\isacharquery}thesis\ \isacommand{{\isachardot}}\isamarkupfalse%
\isanewline
\isacommand{qed}\isamarkupfalse%
%
\endisatagproof
{\isafoldproof}%
%
\isadelimproof
%
\endisadelimproof
%
\isamarkupsection{evaluation as a function%
}
\isamarkuptrue%
%
\begin{isamarkuptext}%
Evaluates a relation \isa{R} for a single argument, as if it were a function.
  This will only work if \isa{R} is right-unique, i.e. if the image is always a singleton set.%
\end{isamarkuptext}%
\isamarkuptrue%
\isacommand{fun}\isamarkupfalse%
\ eval{\isacharunderscore}rel\ {\isacharcolon}{\isacharcolon}\ {\isachardoublequoteopen}{\isacharparenleft}{\isacharprime}a\ {\isasymtimes}\ {\isacharprime}b{\isacharparenright}\ set\ {\isasymRightarrow}\ {\isacharprime}a\ {\isasymRightarrow}\ {\isacharprime}b{\isachardoublequoteclose}\ {\isacharparenleft}\isakeyword{infix}\ {\isachardoublequoteopen}{\isacharcomma}{\isacharcomma}{\isachardoublequoteclose}\ {\isadigit{7}}{\isadigit{5}}{\isacharparenright}\ \isanewline
\ \ \ \ \isakeyword{where}\ {\isachardoublequoteopen}R\ {\isacharcomma}{\isacharcomma}\ a\ {\isacharequal}\ the{\isacharunderscore}elem\ {\isacharparenleft}R\ {\isacharbackquote}{\isacharbackquote}\ {\isacharbraceleft}a{\isacharbraceright}{\isacharparenright}{\isachardoublequoteclose}%
\isamarkupsection{paste%
}
\isamarkuptrue%
%
\begin{isamarkuptext}%
the union of two binary relations \isa{P} and \isa{Q}, where pairs from \isa{Q}
  override pairs from \isa{P} when their first components coincide.
This is particularly useful when P, Q are \isa{runiq}, and one wants to preserve that property.%
\end{isamarkuptext}%
\isamarkuptrue%
\isacommand{definition}\isamarkupfalse%
\ paste\ {\isacharparenleft}\isakeyword{infix}\ {\isachardoublequoteopen}{\isacharplus}{\isacharasterisk}{\isachardoublequoteclose}\ {\isadigit{7}}{\isadigit{5}}{\isacharparenright}\isanewline
\ \ \ \isakeyword{where}\ {\isachardoublequoteopen}P\ {\isacharplus}{\isacharasterisk}\ Q\ {\isacharequal}\ {\isacharparenleft}P\ outside\ Domain\ Q{\isacharparenright}\ {\isasymunion}\ Q{\isachardoublequoteclose}%
\begin{isamarkuptext}%
If a relation \isa{P} is a subrelation of another relation \isa{Q} on \isa{Q}'s
  domain, pasting \isa{Q} on \isa{P} is the same as forming their union.%
\end{isamarkuptext}%
\isamarkuptrue%
\isacommand{lemma}\isamarkupfalse%
\ paste{\isacharunderscore}subrel{\isacharcolon}\ \isanewline
\ \ \ \isakeyword{assumes}\ {\isachardoublequoteopen}P\ {\isacharbar}{\isacharbar}\ Domain\ Q\ {\isasymsubseteq}\ Q{\isachardoublequoteclose}\ \isanewline
\ \ \ \isakeyword{shows}\ {\isachardoublequoteopen}P\ {\isacharplus}{\isacharasterisk}\ Q\ {\isacharequal}\ P\ {\isasymunion}\ Q{\isachardoublequoteclose}\isanewline
%
\isadelimproof
\ \ \ %
\endisadelimproof
%
\isatagproof
\isacommand{unfolding}\isamarkupfalse%
\ paste{\isacharunderscore}def\ \isacommand{using}\isamarkupfalse%
\ assms\ outside{\isacharunderscore}union{\isacharunderscore}restrict\ \isacommand{by}\isamarkupfalse%
\ blast%
\endisatagproof
{\isafoldproof}%
%
\isadelimproof
%
\endisadelimproof
%
\begin{isamarkuptext}%
Pasting two relations with disjoint domains is the same as forming their union.%
\end{isamarkuptext}%
\isamarkuptrue%
\isacommand{lemma}\isamarkupfalse%
\ paste{\isacharunderscore}disj{\isacharunderscore}domains{\isacharcolon}\ \isanewline
\ \ \ \isakeyword{assumes}\ {\isachardoublequoteopen}Domain\ P\ {\isasyminter}\ Domain\ Q\ {\isacharequal}\ {\isacharbraceleft}{\isacharbraceright}{\isachardoublequoteclose}\ \isanewline
\ \ \ \isakeyword{shows}\ {\isachardoublequoteopen}P\ {\isacharplus}{\isacharasterisk}\ Q\ {\isacharequal}\ P\ {\isasymunion}\ Q{\isachardoublequoteclose}\isanewline
%
\isadelimproof
\ \ \ %
\endisadelimproof
%
\isatagproof
\isacommand{unfolding}\isamarkupfalse%
\ paste{\isacharunderscore}def\ Outside{\isacharunderscore}def\ \isacommand{using}\isamarkupfalse%
\ assms\ \isacommand{by}\isamarkupfalse%
\ fast%
\endisatagproof
{\isafoldproof}%
%
\isadelimproof
%
\endisadelimproof
%
\begin{isamarkuptext}%
A relation \isa{P} is equivalent to pasting its restriction to some set \isa{X} on 
  \isa{P\ outside\ X}.%
\end{isamarkuptext}%
\isamarkuptrue%
\isacommand{lemma}\isamarkupfalse%
\ paste{\isacharunderscore}outside{\isacharunderscore}restrict{\isacharcolon}\ {\isachardoublequoteopen}P\ {\isacharequal}\ {\isacharparenleft}P\ outside\ X{\isacharparenright}\ {\isacharplus}{\isacharasterisk}\ {\isacharparenleft}P\ {\isacharbar}{\isacharbar}\ X{\isacharparenright}{\isachardoublequoteclose}\isanewline
%
\isadelimproof
%
\endisadelimproof
%
\isatagproof
\isacommand{proof}\isamarkupfalse%
\ {\isacharminus}\isanewline
\ \ \isacommand{have}\isamarkupfalse%
\ {\isachardoublequoteopen}Domain\ {\isacharparenleft}P\ outside\ X{\isacharparenright}\ {\isasyminter}\ Domain\ {\isacharparenleft}P\ {\isacharbar}{\isacharbar}\ X{\isacharparenright}\ {\isacharequal}\ {\isacharbraceleft}{\isacharbraceright}{\isachardoublequoteclose}\isanewline
\ \ \ \ \isacommand{unfolding}\isamarkupfalse%
\ Outside{\isacharunderscore}def\ restrict{\isacharunderscore}def\ \isacommand{by}\isamarkupfalse%
\ fast\isanewline
\ \ \isacommand{moreover}\isamarkupfalse%
\ \isacommand{have}\isamarkupfalse%
\ {\isachardoublequoteopen}P\ {\isacharequal}\ P\ outside\ X\ {\isasymunion}\ P\ {\isacharbar}{\isacharbar}\ X{\isachardoublequoteclose}\ \isacommand{by}\isamarkupfalse%
\ {\isacharparenleft}rule\ outside{\isacharunderscore}union{\isacharunderscore}restrict{\isacharparenright}\isanewline
\ \ \isacommand{ultimately}\isamarkupfalse%
\ \isacommand{show}\isamarkupfalse%
\ {\isacharquery}thesis\ \isacommand{using}\isamarkupfalse%
\ paste{\isacharunderscore}disj{\isacharunderscore}domains\ \isacommand{by}\isamarkupfalse%
\ metis\isanewline
\isacommand{qed}\isamarkupfalse%
%
\endisatagproof
{\isafoldproof}%
%
\isadelimproof
%
\endisadelimproof
%
\begin{isamarkuptext}%
The domain of two pasted relations equals the union of their domains.%
\end{isamarkuptext}%
\isamarkuptrue%
\isacommand{lemma}\isamarkupfalse%
\ paste{\isacharunderscore}Domain{\isacharcolon}\ {\isachardoublequoteopen}Domain{\isacharparenleft}P\ {\isacharplus}{\isacharasterisk}\ Q{\isacharparenright}{\isacharequal}Domain\ P{\isasymunion}Domain\ Q{\isachardoublequoteclose}%
\isadelimproof
\ %
\endisadelimproof
%
\isatagproof
\isacommand{unfolding}\isamarkupfalse%
\ paste{\isacharunderscore}def\ Outside{\isacharunderscore}def\ \isacommand{by}\isamarkupfalse%
\ blast%
\endisatagproof
{\isafoldproof}%
%
\isadelimproof
%
\endisadelimproof
%
\begin{isamarkuptext}%
Pasting two relations yields a subrelation of their union.%
\end{isamarkuptext}%
\isamarkuptrue%
\isacommand{lemma}\isamarkupfalse%
\ paste{\isacharunderscore}sub{\isacharunderscore}Un{\isacharcolon}\ {\isachardoublequoteopen}P\ {\isacharplus}{\isacharasterisk}\ Q\ {\isasymsubseteq}\ P\ {\isasymunion}\ Q{\isachardoublequoteclose}\ \isanewline
%
\isadelimproof
\ \ %
\endisadelimproof
%
\isatagproof
\isacommand{unfolding}\isamarkupfalse%
\ paste{\isacharunderscore}def\ Outside{\isacharunderscore}def\ \isacommand{by}\isamarkupfalse%
\ fast%
\endisatagproof
{\isafoldproof}%
%
\isadelimproof
%
\endisadelimproof
%
\begin{isamarkuptext}%
The range of two pasted relations is a subset of the union of their ranges.%
\end{isamarkuptext}%
\isamarkuptrue%
\isacommand{lemma}\isamarkupfalse%
\ paste{\isacharunderscore}Range{\isacharcolon}\ {\isachardoublequoteopen}Range\ {\isacharparenleft}P\ {\isacharplus}{\isacharasterisk}\ Q{\isacharparenright}\ {\isasymsubseteq}\ Range\ P\ {\isasymunion}\ Range\ Q{\isachardoublequoteclose}\isanewline
%
\isadelimproof
\ \ %
\endisadelimproof
%
\isatagproof
\isacommand{using}\isamarkupfalse%
\ paste{\isacharunderscore}sub{\isacharunderscore}Un\ \isacommand{by}\isamarkupfalse%
\ blast%
\endisatagproof
{\isafoldproof}%
%
\isadelimproof
\isanewline
%
\endisadelimproof
%
\isadelimtheory
%
\endisadelimtheory
%
\isatagtheory
\isacommand{end}\isamarkupfalse%
%
\endisatagtheory
{\isafoldtheory}%
%
\isadelimtheory
%
\endisadelimtheory
\end{isabellebody}%
%%% Local Variables:
%%% mode: latex
%%% TeX-master: "root"
%%% End:


%
\begin{isabellebody}%
\def\isabellecontext{RelationProperties}%
%
\isamarkupheader{Additional properties of relations, and operators on relations,
  as they have been defined by Relations.thy%
}
\isamarkuptrue%
%
\isadelimtheory
%
\endisadelimtheory
%
\isatagtheory
\isacommand{theory}\isamarkupfalse%
\ RelationProperties\isanewline
\isakeyword{imports}\isanewline
\ \ Main\isanewline
\ \ RelationOperators\isanewline
\ \ SetUtils\isanewline
\ \ Conditionally{\isacharunderscore}Complete{\isacharunderscore}Lattices\ \isanewline
\isanewline
\isakeyword{begin}%
\endisatagtheory
{\isafoldtheory}%
%
\isadelimtheory
%
\endisadelimtheory
%
\isamarkupsection{right-uniqueness%
}
\isamarkuptrue%
\isacommand{lemma}\isamarkupfalse%
\ injflip{\isacharcolon}\ {\isachardoublequoteopen}inj{\isacharunderscore}on\ flip\ A{\isachardoublequoteclose}%
\isadelimproof
\ %
\endisadelimproof
%
\isatagproof
\isacommand{by}\isamarkupfalse%
\ {\isacharparenleft}metis\ flip{\isacharunderscore}flip\ inj{\isacharunderscore}on{\isacharunderscore}def{\isacharparenright}%
\endisatagproof
{\isafoldproof}%
%
\isadelimproof
%
\endisadelimproof
\isanewline
\isanewline
\isacommand{lemma}\isamarkupfalse%
\ lm{\isadigit{0}}{\isadigit{0}}{\isadigit{3}}{\isacharcolon}\ {\isachardoublequoteopen}card\ P\ {\isacharequal}\ card\ {\isacharparenleft}P{\isacharcircum}{\isacharminus}{\isadigit{1}}{\isacharparenright}{\isachardoublequoteclose}%
\isadelimproof
\ %
\endisadelimproof
%
\isatagproof
\isacommand{using}\isamarkupfalse%
\ assms\ card{\isacharunderscore}image\ flip{\isacharunderscore}conv\ injflip\ \isacommand{by}\isamarkupfalse%
\ metis%
\endisatagproof
{\isafoldproof}%
%
\isadelimproof
%
\endisadelimproof
\isanewline
\isanewline
\isacommand{lemma}\isamarkupfalse%
\ nn{\isadigit{5}}{\isadigit{6}}{\isacharcolon}\ {\isachardoublequoteopen}card\ X{\isacharequal}{\isadigit{1}}\ {\isacharequal}\ {\isacharparenleft}X{\isacharequal}{\isacharbraceleft}the{\isacharunderscore}elem\ X{\isacharbraceright}{\isacharparenright}{\isachardoublequoteclose}\ \isanewline
%
\isadelimproof
%
\endisadelimproof
%
\isatagproof
\isacommand{by}\isamarkupfalse%
\ {\isacharparenleft}metis\ One{\isacharunderscore}nat{\isacharunderscore}def\ card{\isacharunderscore}Suc{\isacharunderscore}eq\ card{\isacharunderscore}empty\ empty{\isacharunderscore}iff\ the{\isacharunderscore}elem{\isacharunderscore}eq{\isacharparenright}%
\endisatagproof
{\isafoldproof}%
%
\isadelimproof
\isanewline
%
\endisadelimproof
\isanewline
\isacommand{lemma}\isamarkupfalse%
\ lm{\isadigit{0}}{\isadigit{0}}{\isadigit{7}}b{\isacharcolon}\ {\isachardoublequoteopen}trivial\ X\ {\isacharequal}\ {\isacharparenleft}X{\isacharequal}{\isacharbraceleft}{\isacharbraceright}\ {\isasymor}\ card\ X{\isacharequal}{\isadigit{1}}{\isacharparenright}{\isachardoublequoteclose}%
\isadelimproof
\ %
\endisadelimproof
%
\isatagproof
\isacommand{using}\isamarkupfalse%
\ \isanewline
nn{\isadigit{5}}{\isadigit{6}}\ order{\isacharunderscore}refl\ subset{\isacharunderscore}singletonD\ trivial{\isacharunderscore}def\ trivial{\isacharunderscore}empty\ \isacommand{by}\isamarkupfalse%
\ {\isacharparenleft}metis{\isacharparenleft}no{\isacharunderscore}types{\isacharparenright}{\isacharparenright}%
\endisatagproof
{\isafoldproof}%
%
\isadelimproof
%
\endisadelimproof
\isanewline
\isanewline
\isacommand{lemma}\isamarkupfalse%
\ lm{\isadigit{0}}{\isadigit{0}}{\isadigit{4}}{\isacharcolon}\ {\isachardoublequoteopen}trivial\ P\ {\isacharequal}\ trivial\ {\isacharparenleft}P{\isacharcircum}{\isacharminus}{\isadigit{1}}{\isacharparenright}{\isachardoublequoteclose}%
\isadelimproof
\ %
\endisadelimproof
%
\isatagproof
\isacommand{using}\isamarkupfalse%
\ trivial{\isacharunderscore}def\ subset{\isacharunderscore}singletonD\ \isanewline
subset{\isacharunderscore}refl\ subset{\isacharunderscore}insertI\ nn{\isadigit{5}}{\isadigit{6}}\ converse{\isacharunderscore}inject\ converse{\isacharunderscore}empty\ lm{\isadigit{0}}{\isadigit{0}}{\isadigit{3}}\ \isacommand{by}\isamarkupfalse%
\ metis%
\endisatagproof
{\isafoldproof}%
%
\isadelimproof
%
\endisadelimproof
\isanewline
\isanewline
\isacommand{lemma}\isamarkupfalse%
\ lll{\isadigit{8}}{\isadigit{5}}{\isacharcolon}\ {\isachardoublequoteopen}Range\ {\isacharparenleft}P{\isacharbar}{\isacharbar}X{\isacharparenright}\ {\isacharequal}\ P{\isacharbackquote}{\isacharbackquote}X{\isachardoublequoteclose}%
\isadelimproof
\ %
\endisadelimproof
%
\isatagproof
\isacommand{unfolding}\isamarkupfalse%
\ restrict{\isacharunderscore}def\ \isacommand{by}\isamarkupfalse%
\ blast%
\endisatagproof
{\isafoldproof}%
%
\isadelimproof
%
\endisadelimproof
\isanewline
\isacommand{lemma}\isamarkupfalse%
\ lll{\isadigit{0}}{\isadigit{2}}{\isacharcolon}\ \ {\isachardoublequoteopen}{\isacharparenleft}P\ {\isacharbar}{\isacharbar}\ X{\isacharparenright}\ {\isacharbar}{\isacharbar}\ Y\ {\isacharequal}\ P\ {\isacharbar}{\isacharbar}\ {\isacharparenleft}X\ {\isasyminter}\ Y{\isacharparenright}{\isachardoublequoteclose}\ \isanewline
%
\isadelimproof
\isanewline
%
\endisadelimproof
%
\isatagproof
\isacommand{unfolding}\isamarkupfalse%
\ restrict{\isacharunderscore}def\ \isacommand{by}\isamarkupfalse%
\ fast%
\endisatagproof
{\isafoldproof}%
%
\isadelimproof
\isanewline
%
\endisadelimproof
\isacommand{lemma}\isamarkupfalse%
\ ll{\isadigit{4}}{\isadigit{1}}{\isacharcolon}\ {\isachardoublequoteopen}Domain\ {\isacharparenleft}R{\isacharbar}{\isacharbar}X{\isacharparenright}\ {\isacharequal}\ Domain\ R\ {\isasyminter}\ X{\isachardoublequoteclose}%
\isadelimproof
\ %
\endisadelimproof
%
\isatagproof
\isacommand{using}\isamarkupfalse%
\ restrict{\isacharunderscore}def\ \isacommand{by}\isamarkupfalse%
\ fastforce%
\endisatagproof
{\isafoldproof}%
%
\isadelimproof
%
\endisadelimproof
%
\begin{isamarkuptext}%
A subrelation of a right-unique relation is right-unique.%
\end{isamarkuptext}%
\isamarkuptrue%
\isacommand{lemma}\isamarkupfalse%
\ subrel{\isacharunderscore}runiq{\isacharcolon}\ \isakeyword{assumes}\ {\isachardoublequoteopen}runiq\ Q{\isachardoublequoteclose}\ {\isachardoublequoteopen}P\ {\isasymsubseteq}\ Q{\isachardoublequoteclose}\ \isakeyword{shows}\ {\isachardoublequoteopen}runiq\ P{\isachardoublequoteclose}\ \isanewline
%
\isadelimproof
%
\endisadelimproof
%
\isatagproof
\isacommand{using}\isamarkupfalse%
\ assms\ runiq{\isacharunderscore}def\ \isacommand{by}\isamarkupfalse%
\ {\isacharparenleft}metis\ Image{\isacharunderscore}mono\ subsetI\ trivial{\isacharunderscore}subset{\isacharparenright}%
\endisatagproof
{\isafoldproof}%
%
\isadelimproof
\isanewline
%
\endisadelimproof
\isanewline
\isacommand{lemma}\isamarkupfalse%
\ lll{\isadigit{3}}{\isadigit{1}}{\isacharcolon}\ \isakeyword{assumes}\ {\isachardoublequoteopen}runiq\ P{\isachardoublequoteclose}\ \isakeyword{shows}\ {\isachardoublequoteopen}inj{\isacharunderscore}on\ fst\ P{\isachardoublequoteclose}\ \isanewline
%
\isadelimproof
%
\endisadelimproof
%
\isatagproof
\isacommand{unfolding}\isamarkupfalse%
\ inj{\isacharunderscore}on{\isacharunderscore}def\ \isacommand{using}\isamarkupfalse%
\ assms\ runiq{\isacharunderscore}def\ trivial{\isacharunderscore}def\ trivial{\isacharunderscore}imp{\isacharunderscore}no{\isacharunderscore}distinct\ \isanewline
the{\isacharunderscore}elem{\isacharunderscore}eq\ surjective{\isacharunderscore}pairing\ subsetI\ Image{\isacharunderscore}singleton{\isacharunderscore}iff\ \isacommand{by}\isamarkupfalse%
\ {\isacharparenleft}metis{\isacharparenleft}no{\isacharunderscore}types{\isacharparenright}{\isacharparenright}%
\endisatagproof
{\isafoldproof}%
%
\isadelimproof
%
\endisadelimproof
%
\begin{isamarkuptext}%
alternative characterisation of right-uniqueness: the image of a singleton set is
   \isa{trivial}, i.e.\ an empty or singleton set.%
\end{isamarkuptext}%
\isamarkuptrue%
\isacommand{lemma}\isamarkupfalse%
\ runiq{\isacharunderscore}alt{\isacharcolon}\ {\isachardoublequoteopen}runiq\ R\ {\isasymlongleftrightarrow}\ {\isacharparenleft}{\isasymforall}\ x\ {\isachardot}\ trivial\ {\isacharparenleft}R\ {\isacharbackquote}{\isacharbackquote}\ {\isacharbraceleft}x{\isacharbraceright}{\isacharparenright}{\isacharparenright}{\isachardoublequoteclose}\ \isanewline
%
\isadelimproof
%
\endisadelimproof
%
\isatagproof
\isacommand{unfolding}\isamarkupfalse%
\ runiq{\isacharunderscore}def\ \isacommand{using}\isamarkupfalse%
\ Image{\isacharunderscore}empty\ lm{\isadigit{0}}{\isadigit{0}}{\isadigit{7}}\ the{\isacharunderscore}elem{\isacharunderscore}eq\ \isacommand{by}\isamarkupfalse%
\ {\isacharparenleft}metis{\isacharparenleft}no{\isacharunderscore}types{\isacharparenright}{\isacharparenright}%
\endisatagproof
{\isafoldproof}%
%
\isadelimproof
%
\endisadelimproof
%
\begin{isamarkuptext}%
an alternative definition of right-uniqueness in terms of \isa{op\ {\isacharcomma}{\isacharcomma}}%
\end{isamarkuptext}%
\isamarkuptrue%
\isacommand{lemma}\isamarkupfalse%
\ runiq{\isacharunderscore}wrt{\isacharunderscore}eval{\isacharunderscore}rel{\isacharcolon}\ {\isachardoublequoteopen}runiq\ R\ {\isacharequal}\ {\isacharparenleft}{\isasymforall}x\ {\isachardot}\ R\ {\isacharbackquote}{\isacharbackquote}\ {\isacharbraceleft}x{\isacharbraceright}\ {\isasymsubseteq}\ {\isacharbraceleft}R\ {\isacharcomma}{\isacharcomma}\ x{\isacharbraceright}{\isacharparenright}{\isachardoublequoteclose}%
\isadelimproof
\ %
\endisadelimproof
%
\isatagproof
\isacommand{by}\isamarkupfalse%
\ {\isacharparenleft}metis\ eval{\isacharunderscore}rel{\isachardot}simps\ runiq{\isacharunderscore}alt\ trivial{\isacharunderscore}def{\isacharparenright}%
\endisatagproof
{\isafoldproof}%
%
\isadelimproof
%
\endisadelimproof
\isanewline
\isacommand{lemma}\isamarkupfalse%
\ l{\isadigit{3}}{\isadigit{1}}{\isacharcolon}\ \isakeyword{assumes}\ {\isachardoublequoteopen}runiq\ f{\isachardoublequoteclose}\ \isakeyword{assumes}\ {\isachardoublequoteopen}{\isacharparenleft}x{\isacharcomma}y{\isacharparenright}{\isasymin}f{\isachardoublequoteclose}\ \isakeyword{shows}\ {\isachardoublequoteopen}y{\isacharequal}f{\isacharcomma}{\isacharcomma}x{\isachardoublequoteclose}%
\isadelimproof
\ %
\endisadelimproof
%
\isatagproof
\isacommand{using}\isamarkupfalse%
\ \isanewline
assms\ runiq{\isacharunderscore}wrt{\isacharunderscore}eval{\isacharunderscore}rel\ subset{\isacharunderscore}singletonD\ Image{\isacharunderscore}singleton{\isacharunderscore}iff\ equals{\isadigit{0}}D\ singletonE\ \isacommand{by}\isamarkupfalse%
\ fast%
\endisatagproof
{\isafoldproof}%
%
\isadelimproof
%
\endisadelimproof
\isanewline
\isacommand{lemma}\isamarkupfalse%
\ runiq{\isacharunderscore}basic{\isacharcolon}\ {\isachardoublequoteopen}runiq\ R\ {\isasymlongleftrightarrow}\ {\isacharparenleft}{\isasymforall}\ x\ y\ y{\isacharprime}\ {\isachardot}\ {\isacharparenleft}x{\isacharcomma}\ y{\isacharparenright}\ {\isasymin}\ R\ {\isasymand}\ {\isacharparenleft}x{\isacharcomma}\ y{\isacharprime}{\isacharparenright}\ {\isasymin}\ R\ {\isasymlongrightarrow}\ y\ {\isacharequal}\ y{\isacharprime}{\isacharparenright}{\isachardoublequoteclose}\ \isanewline
%
\isadelimproof
%
\endisadelimproof
%
\isatagproof
\isacommand{unfolding}\isamarkupfalse%
\ runiq{\isacharunderscore}alt\ lm{\isadigit{0}}{\isadigit{1}}\ \isacommand{by}\isamarkupfalse%
\ blast%
\endisatagproof
{\isafoldproof}%
%
\isadelimproof
\isanewline
%
\endisadelimproof
\isanewline
\isacommand{lemma}\isamarkupfalse%
\ ll{\isadigit{7}}{\isadigit{1}}{\isacharcolon}\ \isakeyword{assumes}\ {\isachardoublequoteopen}runiq\ f{\isachardoublequoteclose}\ \isakeyword{shows}\ {\isachardoublequoteopen}f{\isacharbackquote}{\isacharbackquote}{\isacharparenleft}f{\isacharcircum}{\isacharminus}{\isadigit{1}}{\isacharbackquote}{\isacharbackquote}Y{\isacharparenright}\ {\isasymsubseteq}\ Y{\isachardoublequoteclose}\ \isanewline
%
\isadelimproof
%
\endisadelimproof
%
\isatagproof
\isacommand{using}\isamarkupfalse%
\ assms\ runiq{\isacharunderscore}basic\ ImageE\ converse{\isacharunderscore}iff\ subsetI\ \isacommand{by}\isamarkupfalse%
\ {\isacharparenleft}metis{\isacharparenleft}no{\isacharunderscore}types{\isacharparenright}{\isacharparenright}%
\endisatagproof
{\isafoldproof}%
%
\isadelimproof
\isanewline
%
\endisadelimproof
\isanewline
\isacommand{lemma}\isamarkupfalse%
\ ll{\isadigit{6}}{\isadigit{8}}{\isacharcolon}\ \isakeyword{assumes}\ {\isachardoublequoteopen}runiq\ f{\isachardoublequoteclose}\ {\isachardoublequoteopen}y{\isadigit{1}}\ {\isasymin}\ Range\ f{\isachardoublequoteclose}\ \isakeyword{shows}\ \isanewline
{\isachardoublequoteopen}{\isacharparenleft}f{\isacharcircum}{\isacharminus}{\isadigit{1}}\ {\isacharbackquote}{\isacharbackquote}\ {\isacharbraceleft}y{\isadigit{1}}{\isacharbraceright}\ {\isasyminter}\ f{\isacharcircum}{\isacharminus}{\isadigit{1}}\ {\isacharbackquote}{\isacharbackquote}\ {\isacharbraceleft}y{\isadigit{2}}{\isacharbraceright}\ {\isasymnoteq}\ {\isacharbraceleft}{\isacharbraceright}{\isacharparenright}\ {\isacharequal}\ {\isacharparenleft}f{\isacharcircum}{\isacharminus}{\isadigit{1}}{\isacharbackquote}{\isacharbackquote}{\isacharbraceleft}y{\isadigit{1}}{\isacharbraceright}{\isacharequal}f{\isacharcircum}{\isacharminus}{\isadigit{1}}{\isacharbackquote}{\isacharbackquote}{\isacharbraceleft}y{\isadigit{2}}{\isacharbraceright}{\isacharparenright}{\isachardoublequoteclose}\isanewline
%
\isadelimproof
%
\endisadelimproof
%
\isatagproof
\isacommand{using}\isamarkupfalse%
\ assms\ ll{\isadigit{7}}{\isadigit{1}}\ \isacommand{by}\isamarkupfalse%
\ fast%
\endisatagproof
{\isafoldproof}%
%
\isadelimproof
\isanewline
%
\endisadelimproof
\isanewline
\isanewline
\isacommand{lemma}\isamarkupfalse%
\ converse{\isacharunderscore}Image{\isacharcolon}\ \isanewline
\ \ \isakeyword{assumes}\ runiq{\isacharcolon}\ {\isachardoublequoteopen}runiq\ R{\isachardoublequoteclose}\isanewline
\ \ \ \ \ \ \isakeyword{and}\ runiq{\isacharunderscore}conv{\isacharcolon}\ {\isachardoublequoteopen}runiq\ {\isacharparenleft}R{\isacharcircum}{\isacharminus}{\isadigit{1}}{\isacharparenright}{\isachardoublequoteclose}\isanewline
\isakeyword{shows}\ {\isachardoublequoteopen}{\isacharparenleft}R{\isacharcircum}{\isacharminus}{\isadigit{1}}{\isacharparenright}\ {\isacharbackquote}{\isacharbackquote}\ R\ {\isacharbackquote}{\isacharbackquote}\ X\ {\isasymsubseteq}\ X{\isachardoublequoteclose}%
\isadelimproof
\ %
\endisadelimproof
%
\isatagproof
\isacommand{using}\isamarkupfalse%
\ assms\ \isacommand{by}\isamarkupfalse%
\ {\isacharparenleft}metis\ converse{\isacharunderscore}converse\ ll{\isadigit{7}}{\isadigit{1}}{\isacharparenright}%
\endisatagproof
{\isafoldproof}%
%
\isadelimproof
%
\endisadelimproof
\isanewline
\isanewline
\isacommand{lemma}\isamarkupfalse%
\ lll{\isadigit{3}}{\isadigit{2}}{\isacharcolon}\ \isakeyword{assumes}\ {\isachardoublequoteopen}inj{\isacharunderscore}on\ fst\ P{\isachardoublequoteclose}\ \isakeyword{shows}\ {\isachardoublequoteopen}runiq\ P{\isachardoublequoteclose}%
\isadelimproof
\ %
\endisadelimproof
%
\isatagproof
\isacommand{unfolding}\isamarkupfalse%
\ runiq{\isacharunderscore}basic\ \isanewline
\isacommand{using}\isamarkupfalse%
\ assms\ fst{\isacharunderscore}conv\ inj{\isacharunderscore}on{\isacharunderscore}def\ old{\isachardot}prod{\isachardot}inject\ \isacommand{by}\isamarkupfalse%
\ {\isacharparenleft}metis{\isacharparenleft}no{\isacharunderscore}types{\isacharparenright}{\isacharparenright}%
\endisatagproof
{\isafoldproof}%
%
\isadelimproof
%
\endisadelimproof
\isanewline
\isanewline
\isacommand{lemma}\isamarkupfalse%
\ lll{\isadigit{3}}{\isadigit{3}}{\isacharcolon}\ {\isachardoublequoteopen}runiq\ P{\isacharequal}inj{\isacharunderscore}on\ fst\ P{\isachardoublequoteclose}%
\isadelimproof
\ %
\endisadelimproof
%
\isatagproof
\isacommand{using}\isamarkupfalse%
\ lll{\isadigit{3}}{\isadigit{1}}\ lll{\isadigit{3}}{\isadigit{2}}\ \isacommand{by}\isamarkupfalse%
\ blast%
\endisatagproof
{\isafoldproof}%
%
\isadelimproof
%
\endisadelimproof
\isanewline
\isanewline
\isanewline
\isacommand{lemma}\isamarkupfalse%
\ disj{\isacharunderscore}Un{\isacharunderscore}runiq{\isacharcolon}\ \isakeyword{assumes}\ {\isachardoublequoteopen}runiq\ P{\isachardoublequoteclose}\ {\isachardoublequoteopen}runiq\ Q{\isachardoublequoteclose}\ {\isachardoublequoteopen}Domain\ P\ {\isasyminter}\ {\isacharparenleft}Domain\ Q{\isacharparenright}\ {\isacharequal}\ {\isacharbraceleft}{\isacharbraceright}{\isachardoublequoteclose}\ \isakeyword{shows}\ {\isachardoublequoteopen}runiq\ {\isacharparenleft}P\ Un\ Q{\isacharparenright}{\isachardoublequoteclose}\ \isanewline
%
\isadelimproof
%
\endisadelimproof
%
\isatagproof
\isacommand{using}\isamarkupfalse%
\ assms\ lll{\isadigit{3}}{\isadigit{3}}\ fst{\isacharunderscore}eq{\isacharunderscore}Domain\ lm{\isadigit{0}}{\isadigit{1}}{\isadigit{0}}b\ \isacommand{by}\isamarkupfalse%
\ metis%
\endisatagproof
{\isafoldproof}%
%
\isadelimproof
\isanewline
%
\endisadelimproof
\isanewline
\isacommand{lemma}\isamarkupfalse%
\ runiq{\isacharunderscore}paste{\isadigit{1}}{\isacharcolon}\ \isakeyword{assumes}\ {\isachardoublequoteopen}runiq\ Q{\isachardoublequoteclose}\ {\isachardoublequoteopen}runiq\ {\isacharparenleft}P\ outside\ Domain\ Q{\isacharparenright}{\isachardoublequoteclose}\ \isakeyword{shows}\ {\isachardoublequoteopen}runiq\ {\isacharparenleft}P\ {\isacharplus}{\isacharasterisk}\ Q{\isacharparenright}{\isachardoublequoteclose}\isanewline
%
\isadelimproof
%
\endisadelimproof
%
\isatagproof
\isacommand{unfolding}\isamarkupfalse%
\ paste{\isacharunderscore}def\ \isacommand{using}\isamarkupfalse%
\ assms\ disj{\isacharunderscore}Un{\isacharunderscore}runiq\ Diff{\isacharunderscore}disjoint\ Un{\isacharunderscore}commute\ outside{\isacharunderscore}reduces{\isacharunderscore}domain\isanewline
\isacommand{by}\isamarkupfalse%
\ {\isacharparenleft}metis\ {\isacharparenleft}poly{\isacharunderscore}guards{\isacharunderscore}query{\isacharparenright}{\isacharparenright}%
\endisatagproof
{\isafoldproof}%
%
\isadelimproof
\isanewline
%
\endisadelimproof
\isanewline
\isacommand{corollary}\isamarkupfalse%
\ runiq{\isacharunderscore}paste{\isadigit{2}}{\isacharcolon}\ \isakeyword{assumes}\ {\isachardoublequoteopen}runiq\ Q{\isachardoublequoteclose}\ {\isachardoublequoteopen}runiq\ P{\isachardoublequoteclose}\ \isakeyword{shows}\ {\isachardoublequoteopen}runiq\ {\isacharparenleft}P\ {\isacharplus}{\isacharasterisk}\ Q{\isacharparenright}{\isachardoublequoteclose}\isanewline
%
\isadelimproof
%
\endisadelimproof
%
\isatagproof
\isacommand{using}\isamarkupfalse%
\ assms\ runiq{\isacharunderscore}paste{\isadigit{1}}\ subrel{\isacharunderscore}runiq\ Diff{\isacharunderscore}subset\ Outside{\isacharunderscore}def\ \isacommand{by}\isamarkupfalse%
\ {\isacharparenleft}metis{\isacharparenright}%
\endisatagproof
{\isafoldproof}%
%
\isadelimproof
\isanewline
%
\endisadelimproof
\isanewline
\isacommand{lemma}\isamarkupfalse%
\ l{\isadigit{1}}{\isadigit{4}}{\isacharcolon}\ {\isachardoublequoteopen}runiq\ {\isacharbraceleft}{\isacharparenleft}x{\isacharcomma}f\ x{\isacharparenright}{\isacharbar}\ x{\isachardot}\ P\ x{\isacharbraceright}{\isachardoublequoteclose}%
\isadelimproof
\ %
\endisadelimproof
%
\isatagproof
\isacommand{unfolding}\isamarkupfalse%
\ runiq{\isacharunderscore}basic\ \isacommand{by}\isamarkupfalse%
\ fast%
\endisatagproof
{\isafoldproof}%
%
\isadelimproof
%
\endisadelimproof
\isanewline
\isanewline
\isacommand{lemma}\isamarkupfalse%
\ runiq{\isacharunderscore}alt{\isadigit{2}}{\isacharcolon}\ {\isachardoublequoteopen}runiq\ R\ {\isacharequal}\ {\isacharparenleft}{\isasymforall}\ x\ {\isasymin}\ Domain\ R{\isachardot}\ trivial\ {\isacharparenleft}R\ {\isacharbackquote}{\isacharbackquote}\ {\isacharbraceleft}x{\isacharbraceright}{\isacharparenright}{\isacharparenright}{\isachardoublequoteclose}\ \isanewline
%
\isadelimproof
%
\endisadelimproof
%
\isatagproof
\isacommand{by}\isamarkupfalse%
\ {\isacharparenleft}metis\ Domain{\isachardot}DomainI\ Image{\isacharunderscore}singleton{\isacharunderscore}iff\ lm{\isadigit{0}}{\isadigit{1}}\ runiq{\isacharunderscore}alt{\isacharparenright}%
\endisatagproof
{\isafoldproof}%
%
\isadelimproof
\isanewline
%
\endisadelimproof
\isanewline
\isacommand{lemma}\isamarkupfalse%
\ lm{\isadigit{0}}{\isadigit{1}}{\isadigit{3}}{\isacharcolon}\ \isakeyword{assumes}\ {\isachardoublequoteopen}x\ {\isasymin}\ Domain\ R{\isachardoublequoteclose}\ {\isachardoublequoteopen}runiq\ R{\isachardoublequoteclose}\ \isakeyword{shows}\ {\isachardoublequoteopen}card\ {\isacharparenleft}R{\isacharbackquote}{\isacharbackquote}{\isacharbraceleft}x{\isacharbraceright}{\isacharparenright}{\isacharequal}{\isadigit{1}}{\isachardoublequoteclose}\isanewline
%
\isadelimproof
%
\endisadelimproof
%
\isatagproof
\isacommand{using}\isamarkupfalse%
\ assms\ runiq{\isacharunderscore}alt{\isadigit{2}}\ lm{\isadigit{0}}{\isadigit{0}}{\isadigit{7}}b\ \isacommand{by}\isamarkupfalse%
\ {\isacharparenleft}metis\ DomainE\ Image{\isacharunderscore}singleton{\isacharunderscore}iff\ empty{\isacharunderscore}iff{\isacharparenright}%
\endisatagproof
{\isafoldproof}%
%
\isadelimproof
%
\endisadelimproof
%
\begin{isamarkuptext}%
The image of a singleton set under a right-unique relation is a singleton set.%
\end{isamarkuptext}%
\isamarkuptrue%
\isacommand{lemma}\isamarkupfalse%
\ Image{\isacharunderscore}runiq{\isacharunderscore}eq{\isacharunderscore}eval{\isacharcolon}\ \isakeyword{assumes}\ {\isachardoublequoteopen}x\ {\isasymin}\ Domain\ R{\isachardoublequoteclose}\ {\isachardoublequoteopen}runiq\ R{\isachardoublequoteclose}\ \isakeyword{shows}\ {\isachardoublequoteopen}R\ {\isacharbackquote}{\isacharbackquote}\ {\isacharbraceleft}x{\isacharbraceright}\ {\isacharequal}\ {\isacharbraceleft}R\ {\isacharcomma}{\isacharcomma}\ x{\isacharbraceright}{\isachardoublequoteclose}\ \isanewline
%
\isadelimproof
%
\endisadelimproof
%
\isatagproof
\isacommand{using}\isamarkupfalse%
\ assms\ lm{\isadigit{0}}{\isadigit{1}}{\isadigit{3}}\ \isacommand{by}\isamarkupfalse%
\ {\isacharparenleft}metis\ eval{\isacharunderscore}rel{\isachardot}simps\ nn{\isadigit{5}}{\isadigit{6}}{\isacharparenright}%
\endisatagproof
{\isafoldproof}%
%
\isadelimproof
%
\endisadelimproof
%
\begin{isamarkuptext}%
the image of a singleton set under a right-unique relation is
   \isa{trivial}, i.e.\ an empty or singleton set.%
\end{isamarkuptext}%
\isamarkuptrue%
%
\begin{isamarkuptext}%
If all images of singleton sets under a relation are
   \isa{trivial}, i.e.\ an empty or singleton set, the relation is right-unique.%
\end{isamarkuptext}%
\isamarkuptrue%
\isacommand{lemma}\isamarkupfalse%
\ Image{\isacharunderscore}within{\isacharunderscore}runiq{\isacharunderscore}domain{\isacharcolon}\isanewline
\ \ \isakeyword{fixes}\ x\ R\isanewline
\ \ \isakeyword{assumes}\ {\isachardoublequoteopen}runiq\ R{\isachardoublequoteclose}\isanewline
\ \ \isakeyword{shows}\ {\isachardoublequoteopen}x\ {\isasymin}\ Domain\ R\ {\isasymlongleftrightarrow}\ {\isacharparenleft}{\isasymexists}\ y\ {\isachardot}\ R\ {\isacharbackquote}{\isacharbackquote}\ {\isacharbraceleft}x{\isacharbraceright}\ {\isacharequal}\ {\isacharbraceleft}y{\isacharbraceright}{\isacharparenright}{\isachardoublequoteclose}%
\isadelimproof
\ %
\endisadelimproof
%
\isatagproof
\isacommand{using}\isamarkupfalse%
\ assms\ Image{\isacharunderscore}runiq{\isacharunderscore}eq{\isacharunderscore}eval\ \isacommand{by}\isamarkupfalse%
\ fast%
\endisatagproof
{\isafoldproof}%
%
\isadelimproof
%
\endisadelimproof
\isanewline
\isanewline
\isacommand{lemma}\isamarkupfalse%
\ runiq{\isacharunderscore}imp{\isacharunderscore}singleton{\isacharunderscore}image{\isacharprime}{\isacharcolon}\isanewline
\ \ \isakeyword{assumes}\ runiq{\isacharcolon}\ {\isachardoublequoteopen}runiq\ R{\isachardoublequoteclose}\isanewline
\ \ \ \ \ \ \isakeyword{and}\ dom{\isacharcolon}\ {\isachardoublequoteopen}x\ {\isasymin}\ Domain\ R{\isachardoublequoteclose}\isanewline
\ \ \isakeyword{shows}\ {\isachardoublequoteopen}the{\isacharunderscore}elem\ {\isacharparenleft}R\ {\isacharbackquote}{\isacharbackquote}\ {\isacharbraceleft}x{\isacharbraceright}{\isacharparenright}\ {\isacharequal}\ {\isacharparenleft}THE\ y\ {\isachardot}\ {\isacharparenleft}x{\isacharcomma}\ y{\isacharparenright}\ {\isasymin}\ R{\isacharparenright}{\isachardoublequoteclose}\ {\isacharparenleft}\isakeyword{is}\ {\isachardoublequoteopen}the{\isacharunderscore}elem\ {\isacharparenleft}R\ {\isacharbackquote}{\isacharbackquote}\ {\isacharbraceleft}x{\isacharbraceright}{\isacharparenright}\ {\isacharequal}\ {\isacharquery}y{\isachardoublequoteclose}{\isacharparenright}\isanewline
%
\isadelimproof
%
\endisadelimproof
%
\isatagproof
\isacommand{unfolding}\isamarkupfalse%
\ the{\isacharunderscore}elem{\isacharunderscore}def\isanewline
\isacommand{using}\isamarkupfalse%
\ assms\ Image{\isacharunderscore}singleton{\isacharunderscore}iff\ Image{\isacharunderscore}within{\isacharunderscore}runiq{\isacharunderscore}domain\ singletonD\ singletonI\ \isacommand{by}\isamarkupfalse%
\ {\isacharparenleft}metis{\isacharparenright}%
\endisatagproof
{\isafoldproof}%
%
\isadelimproof
\isanewline
%
\endisadelimproof
\isanewline
\isacommand{lemma}\isamarkupfalse%
\ runiq{\isacharunderscore}conv{\isacharunderscore}imp{\isacharunderscore}singleton{\isacharunderscore}preimage{\isacharprime}{\isacharcolon}\isanewline
\ \ \isakeyword{assumes}\ runiq{\isacharunderscore}conv{\isacharcolon}\ {\isachardoublequoteopen}runiq\ {\isacharparenleft}R{\isasyminverse}{\isacharparenright}{\isachardoublequoteclose}\isanewline
\ \ \ \ \ \ \isakeyword{and}\ ran{\isacharcolon}\ {\isachardoublequoteopen}y\ {\isasymin}\ Range\ R{\isachardoublequoteclose}\isanewline
\ \ \isakeyword{shows}\ {\isachardoublequoteopen}the{\isacharunderscore}elem\ {\isacharparenleft}{\isacharparenleft}R{\isasyminverse}{\isacharparenright}\ {\isacharbackquote}{\isacharbackquote}\ {\isacharbraceleft}y{\isacharbraceright}{\isacharparenright}\ {\isacharequal}\ {\isacharparenleft}THE\ x\ {\isachardot}\ {\isacharparenleft}x{\isacharcomma}\ y{\isacharparenright}\ {\isasymin}\ R{\isacharparenright}{\isachardoublequoteclose}\ \isanewline
%
\isadelimproof
\isanewline
%
\endisadelimproof
%
\isatagproof
\isacommand{proof}\isamarkupfalse%
\ {\isacharminus}\isanewline
\ \ \isacommand{from}\isamarkupfalse%
\ ran\ \isacommand{have}\isamarkupfalse%
\ dom{\isacharcolon}\ {\isachardoublequoteopen}y\ {\isasymin}\ Domain\ {\isacharparenleft}R{\isasyminverse}{\isacharparenright}{\isachardoublequoteclose}\ \isacommand{by}\isamarkupfalse%
\ simp\isanewline
\ \ \isacommand{with}\isamarkupfalse%
\ runiq{\isacharunderscore}conv\ \isacommand{have}\isamarkupfalse%
\ {\isachardoublequoteopen}the{\isacharunderscore}elem\ {\isacharparenleft}{\isacharparenleft}R{\isasyminverse}{\isacharparenright}\ {\isacharbackquote}{\isacharbackquote}\ {\isacharbraceleft}y{\isacharbraceright}{\isacharparenright}\ {\isacharequal}\ {\isacharparenleft}THE\ x\ {\isachardot}\ {\isacharparenleft}y{\isacharcomma}\ x{\isacharparenright}\ {\isasymin}\ {\isacharparenleft}R{\isasyminverse}{\isacharparenright}{\isacharparenright}{\isachardoublequoteclose}\ \isacommand{by}\isamarkupfalse%
\ {\isacharparenleft}rule\ runiq{\isacharunderscore}imp{\isacharunderscore}singleton{\isacharunderscore}image{\isacharprime}{\isacharparenright}\isanewline
\ \ \isacommand{also}\isamarkupfalse%
\ \isacommand{have}\isamarkupfalse%
\ {\isachardoublequoteopen}{\isasymdots}\ {\isacharequal}\ {\isacharparenleft}THE\ x\ {\isachardot}\ {\isacharparenleft}x{\isacharcomma}\ y{\isacharparenright}\ {\isasymin}\ R{\isacharparenright}{\isachardoublequoteclose}\ \isacommand{by}\isamarkupfalse%
\ simp\isanewline
\ \ \isacommand{finally}\isamarkupfalse%
\ \isacommand{show}\isamarkupfalse%
\ {\isacharquery}thesis\ \isacommand{{\isachardot}}\isamarkupfalse%
\isanewline
\isacommand{qed}\isamarkupfalse%
%
\endisatagproof
{\isafoldproof}%
%
\isadelimproof
%
\endisadelimproof
%
\begin{isamarkuptext}%
another alternative definition of right-uniqueness in terms of \isa{op\ {\isacharcomma}{\isacharcomma}}%
\end{isamarkuptext}%
\isamarkuptrue%
\isacommand{lemma}\isamarkupfalse%
\ runiq{\isacharunderscore}wrt{\isacharunderscore}eval{\isacharunderscore}rel{\isacharprime}{\isacharcolon}\isanewline
\ \ \isakeyword{fixes}\ R\ {\isacharcolon}{\isacharcolon}\ {\isachardoublequoteopen}{\isacharparenleft}{\isacharprime}a\ {\isasymtimes}\ {\isacharprime}b{\isacharparenright}\ set{\isachardoublequoteclose}\isanewline
\ \ \isakeyword{shows}\ {\isachardoublequoteopen}runiq\ R\ {\isasymlongleftrightarrow}\ {\isacharparenleft}{\isasymforall}x\ {\isasymin}\ Domain\ R\ {\isachardot}\ R\ {\isacharbackquote}{\isacharbackquote}\ {\isacharbraceleft}x{\isacharbraceright}\ {\isacharequal}\ {\isacharbraceleft}R\ {\isacharcomma}{\isacharcomma}\ x{\isacharbraceright}{\isacharparenright}{\isachardoublequoteclose}%
\isadelimproof
\ %
\endisadelimproof
%
\isatagproof
\isacommand{unfolding}\isamarkupfalse%
\ runiq{\isacharunderscore}wrt{\isacharunderscore}eval{\isacharunderscore}rel\ \isacommand{by}\isamarkupfalse%
\ fast%
\endisatagproof
{\isafoldproof}%
%
\isadelimproof
%
\endisadelimproof
\isanewline
\isanewline
\isacommand{lemma}\isamarkupfalse%
\ runiq{\isacharunderscore}wrt{\isacharunderscore}ex{\isadigit{1}}{\isacharcolon}\isanewline
\ \ {\isachardoublequoteopen}runiq\ R\ {\isasymlongleftrightarrow}\ {\isacharparenleft}{\isasymforall}\ a\ {\isasymin}\ Domain\ R\ {\isachardot}\ {\isasymexists}{\isacharbang}\ b\ {\isachardot}\ {\isacharparenleft}a{\isacharcomma}\ b{\isacharparenright}\ {\isasymin}\ R{\isacharparenright}{\isachardoublequoteclose}\isanewline
%
\isadelimproof
%
\endisadelimproof
%
\isatagproof
\isacommand{using}\isamarkupfalse%
\ runiq{\isacharunderscore}basic\ \isacommand{by}\isamarkupfalse%
\ {\isacharparenleft}metis\ Domain{\isachardot}DomainI\ Domain{\isachardot}cases{\isacharparenright}%
\endisatagproof
{\isafoldproof}%
%
\isadelimproof
\isanewline
%
\endisadelimproof
\isanewline
\isacommand{lemma}\isamarkupfalse%
\ runiq{\isacharunderscore}imp{\isacharunderscore}THE{\isacharunderscore}right{\isacharunderscore}comp{\isacharcolon}\isanewline
\ \ \isakeyword{fixes}\ a\ \isakeyword{and}\ b\isanewline
\ \ \isakeyword{assumes}\ runiq{\isacharcolon}\ {\isachardoublequoteopen}runiq\ R{\isachardoublequoteclose}\isanewline
\ \ \ \ \ \ \isakeyword{and}\ aRb{\isacharcolon}\ {\isachardoublequoteopen}{\isacharparenleft}a{\isacharcomma}\ b{\isacharparenright}\ {\isasymin}\ R{\isachardoublequoteclose}\isanewline
\ \ \isakeyword{shows}\ {\isachardoublequoteopen}b\ {\isacharequal}\ {\isacharparenleft}THE\ b\ {\isachardot}\ {\isacharparenleft}a{\isacharcomma}\ b{\isacharparenright}\ {\isasymin}\ R{\isacharparenright}{\isachardoublequoteclose}%
\isadelimproof
\ %
\endisadelimproof
%
\isatagproof
\isacommand{using}\isamarkupfalse%
\ assms\ \isacommand{by}\isamarkupfalse%
\ {\isacharparenleft}metis\ runiq{\isacharunderscore}basic\ the{\isacharunderscore}equality{\isacharparenright}%
\endisatagproof
{\isafoldproof}%
%
\isadelimproof
%
\endisadelimproof
\isanewline
\isanewline
\isacommand{lemma}\isamarkupfalse%
\ runiq{\isacharunderscore}imp{\isacharunderscore}THE{\isacharunderscore}right{\isacharunderscore}comp{\isacharprime}{\isacharcolon}\isanewline
\ \ \isakeyword{assumes}\ runiq{\isacharcolon}\ {\isachardoublequoteopen}runiq\ R{\isachardoublequoteclose}\isanewline
\ \ \ \ \ \ \isakeyword{and}\ in{\isacharunderscore}Domain{\isacharcolon}\ {\isachardoublequoteopen}a\ {\isasymin}\ Domain\ R{\isachardoublequoteclose}\isanewline
\ \ \isakeyword{shows}\ {\isachardoublequoteopen}{\isacharparenleft}a{\isacharcomma}\ THE\ b{\isachardot}\ {\isacharparenleft}a{\isacharcomma}\ b{\isacharparenright}\ {\isasymin}\ R{\isacharparenright}\ {\isasymin}\ R{\isachardoublequoteclose}\isanewline
%
\isadelimproof
%
\endisadelimproof
%
\isatagproof
\isacommand{proof}\isamarkupfalse%
\ {\isacharminus}\isanewline
\ \ \isacommand{from}\isamarkupfalse%
\ in{\isacharunderscore}Domain\ \isacommand{obtain}\isamarkupfalse%
\ b\ \isakeyword{where}\ {\isacharasterisk}{\isacharcolon}\ {\isachardoublequoteopen}{\isacharparenleft}a{\isacharcomma}\ b{\isacharparenright}\ {\isasymin}\ R{\isachardoublequoteclose}\ \isacommand{by}\isamarkupfalse%
\ force\isanewline
\ \ \isacommand{with}\isamarkupfalse%
\ runiq\ \isacommand{have}\isamarkupfalse%
\ {\isachardoublequoteopen}b\ {\isacharequal}\ {\isacharparenleft}THE\ b\ {\isachardot}\ {\isacharparenleft}a{\isacharcomma}\ b{\isacharparenright}\ {\isasymin}\ R{\isacharparenright}{\isachardoublequoteclose}\ \isacommand{by}\isamarkupfalse%
\ {\isacharparenleft}rule\ runiq{\isacharunderscore}imp{\isacharunderscore}THE{\isacharunderscore}right{\isacharunderscore}comp{\isacharparenright}\isanewline
\ \ \isacommand{with}\isamarkupfalse%
\ {\isacharasterisk}\ \isacommand{show}\isamarkupfalse%
\ {\isacharquery}thesis\ \isacommand{by}\isamarkupfalse%
\ simp\isanewline
\isacommand{qed}\isamarkupfalse%
%
\endisatagproof
{\isafoldproof}%
%
\isadelimproof
\isanewline
%
\endisadelimproof
\isanewline
\isacommand{lemma}\isamarkupfalse%
\ THE{\isacharunderscore}right{\isacharunderscore}comp{\isacharunderscore}imp{\isacharunderscore}runiq{\isacharcolon}\isanewline
\ \ \isakeyword{assumes}\ {\isachardoublequoteopen}{\isasymforall}\ a\ b\ {\isachardot}\ {\isacharparenleft}a{\isacharcomma}\ b{\isacharparenright}\ {\isasymin}\ R\ {\isasymlongrightarrow}\ b\ {\isacharequal}\ {\isacharparenleft}THE\ b\ {\isachardot}\ {\isacharparenleft}a{\isacharcomma}\ b{\isacharparenright}\ {\isasymin}\ R{\isacharparenright}{\isachardoublequoteclose}\isanewline
\ \ \isakeyword{shows}\ {\isachardoublequoteopen}runiq\ R{\isachardoublequoteclose}\isanewline
%
\isadelimproof
%
\endisadelimproof
%
\isatagproof
\isacommand{using}\isamarkupfalse%
\ assms\ DomainE\ runiq{\isacharunderscore}wrt{\isacharunderscore}ex{\isadigit{1}}\ \isacommand{by}\isamarkupfalse%
\ metis%
\endisatagproof
{\isafoldproof}%
%
\isadelimproof
%
\endisadelimproof
%
\begin{isamarkuptext}%
another alternative definition of right-uniqueness in terms of \isa{The}%
\end{isamarkuptext}%
\isamarkuptrue%
\isacommand{lemma}\isamarkupfalse%
\ runiq{\isacharunderscore}wrt{\isacharunderscore}THE{\isacharcolon}\isanewline
\ \ {\isachardoublequoteopen}runiq\ R\ {\isasymlongleftrightarrow}\ {\isacharparenleft}{\isasymforall}\ a\ b\ {\isachardot}\ {\isacharparenleft}a{\isacharcomma}\ b{\isacharparenright}\ {\isasymin}\ R\ {\isasymlongrightarrow}\ b\ {\isacharequal}\ {\isacharparenleft}THE\ b\ {\isachardot}\ {\isacharparenleft}a{\isacharcomma}\ b{\isacharparenright}\ {\isasymin}\ R{\isacharparenright}{\isacharparenright}{\isachardoublequoteclose}\isanewline
%
\isadelimproof
%
\endisadelimproof
%
\isatagproof
\isacommand{proof}\isamarkupfalse%
\isanewline
\ \ \isacommand{assume}\isamarkupfalse%
\ {\isachardoublequoteopen}runiq\ R{\isachardoublequoteclose}\isanewline
\ \ \isacommand{then}\isamarkupfalse%
\ \isacommand{show}\isamarkupfalse%
\ {\isachardoublequoteopen}{\isasymforall}\ a\ b\ {\isachardot}\ {\isacharparenleft}a{\isacharcomma}\ b{\isacharparenright}\ {\isasymin}\ R\ {\isasymlongrightarrow}\ b\ {\isacharequal}\ {\isacharparenleft}THE\ b\ {\isachardot}\ {\isacharparenleft}a{\isacharcomma}\ b{\isacharparenright}\ {\isasymin}\ R{\isacharparenright}{\isachardoublequoteclose}\ \isacommand{by}\isamarkupfalse%
\ {\isacharparenleft}metis\ runiq{\isacharunderscore}imp{\isacharunderscore}THE{\isacharunderscore}right{\isacharunderscore}comp{\isacharparenright}\isanewline
\isacommand{next}\isamarkupfalse%
\isanewline
\ \ \isacommand{assume}\isamarkupfalse%
\ {\isachardoublequoteopen}{\isasymforall}\ a\ b\ {\isachardot}\ {\isacharparenleft}a{\isacharcomma}\ b{\isacharparenright}\ {\isasymin}\ R\ {\isasymlongrightarrow}\ b\ {\isacharequal}\ {\isacharparenleft}THE\ b\ {\isachardot}\ {\isacharparenleft}a{\isacharcomma}\ b{\isacharparenright}\ {\isasymin}\ R{\isacharparenright}{\isachardoublequoteclose}\isanewline
\ \ \isacommand{then}\isamarkupfalse%
\ \isacommand{show}\isamarkupfalse%
\ {\isachardoublequoteopen}runiq\ R{\isachardoublequoteclose}\ \isacommand{by}\isamarkupfalse%
\ {\isacharparenleft}rule\ THE{\isacharunderscore}right{\isacharunderscore}comp{\isacharunderscore}imp{\isacharunderscore}runiq{\isacharparenright}\isanewline
\isacommand{qed}\isamarkupfalse%
%
\endisatagproof
{\isafoldproof}%
%
\isadelimproof
\isanewline
%
\endisadelimproof
\isanewline
\isacommand{lemma}\isamarkupfalse%
\ runiq{\isacharunderscore}conv{\isacharunderscore}imp{\isacharunderscore}THE{\isacharunderscore}left{\isacharunderscore}comp{\isacharcolon}\isanewline
\ \ \isakeyword{assumes}\ runiq{\isacharunderscore}conv{\isacharcolon}\ {\isachardoublequoteopen}runiq\ {\isacharparenleft}R{\isasyminverse}{\isacharparenright}{\isachardoublequoteclose}\ \isakeyword{and}\ aRb{\isacharcolon}\ {\isachardoublequoteopen}{\isacharparenleft}a{\isacharcomma}\ b{\isacharparenright}\ {\isasymin}\ R{\isachardoublequoteclose}\isanewline
\ \ \isakeyword{shows}\ {\isachardoublequoteopen}a\ {\isacharequal}\ {\isacharparenleft}THE\ a\ {\isachardot}\ {\isacharparenleft}a{\isacharcomma}\ b{\isacharparenright}\ {\isasymin}\ R{\isacharparenright}{\isachardoublequoteclose}\isanewline
%
\isadelimproof
%
\endisadelimproof
%
\isatagproof
\isacommand{proof}\isamarkupfalse%
\ {\isacharminus}\isanewline
\ \ \isacommand{from}\isamarkupfalse%
\ aRb\ \isacommand{have}\isamarkupfalse%
\ {\isachardoublequoteopen}{\isacharparenleft}b{\isacharcomma}\ a{\isacharparenright}\ {\isasymin}\ R{\isasyminverse}{\isachardoublequoteclose}\ \isacommand{by}\isamarkupfalse%
\ simp\isanewline
\ \ \isacommand{with}\isamarkupfalse%
\ runiq{\isacharunderscore}conv\ \isacommand{have}\isamarkupfalse%
\ {\isachardoublequoteopen}a\ {\isacharequal}\ {\isacharparenleft}THE\ a\ {\isachardot}\ {\isacharparenleft}b{\isacharcomma}\ a{\isacharparenright}\ {\isasymin}\ R{\isasyminverse}{\isacharparenright}{\isachardoublequoteclose}\ \isacommand{by}\isamarkupfalse%
\ {\isacharparenleft}rule\ runiq{\isacharunderscore}imp{\isacharunderscore}THE{\isacharunderscore}right{\isacharunderscore}comp{\isacharparenright}\isanewline
\ \ \isacommand{then}\isamarkupfalse%
\ \isacommand{show}\isamarkupfalse%
\ {\isacharquery}thesis\ \isacommand{by}\isamarkupfalse%
\ fastforce\isanewline
\isacommand{qed}\isamarkupfalse%
%
\endisatagproof
{\isafoldproof}%
%
\isadelimproof
\isanewline
%
\endisadelimproof
\isanewline
\isacommand{lemma}\isamarkupfalse%
\ runiq{\isacharunderscore}conv{\isacharunderscore}imp{\isacharunderscore}THE{\isacharunderscore}left{\isacharunderscore}comp{\isacharprime}{\isacharcolon}\isanewline
\ \ \isakeyword{assumes}\ runiq{\isacharunderscore}conv{\isacharcolon}\ {\isachardoublequoteopen}runiq\ {\isacharparenleft}R{\isasyminverse}{\isacharparenright}{\isachardoublequoteclose}\isanewline
\ \ \ \ \ \ \isakeyword{and}\ in{\isacharunderscore}Range{\isacharcolon}\ {\isachardoublequoteopen}b\ {\isasymin}\ Range\ R{\isachardoublequoteclose}\isanewline
\ \ \isakeyword{shows}\ {\isachardoublequoteopen}{\isacharparenleft}THE\ a{\isachardot}\ {\isacharparenleft}a{\isacharcomma}\ b{\isacharparenright}\ {\isasymin}\ R{\isacharcomma}\ b{\isacharparenright}\ {\isasymin}\ R{\isachardoublequoteclose}\isanewline
%
\isadelimproof
%
\endisadelimproof
%
\isatagproof
\isacommand{proof}\isamarkupfalse%
\ {\isacharminus}\isanewline
\ \ \isacommand{from}\isamarkupfalse%
\ in{\isacharunderscore}Range\ \isacommand{obtain}\isamarkupfalse%
\ a\ \isakeyword{where}\ {\isacharasterisk}{\isacharcolon}\ {\isachardoublequoteopen}{\isacharparenleft}a{\isacharcomma}\ b{\isacharparenright}\ {\isasymin}\ R{\isachardoublequoteclose}\ \isacommand{by}\isamarkupfalse%
\ force\isanewline
\ \ \isacommand{with}\isamarkupfalse%
\ runiq{\isacharunderscore}conv\ \isacommand{have}\isamarkupfalse%
\ {\isachardoublequoteopen}a\ {\isacharequal}\ {\isacharparenleft}THE\ a\ {\isachardot}\ {\isacharparenleft}a{\isacharcomma}\ b{\isacharparenright}\ {\isasymin}\ R{\isacharparenright}{\isachardoublequoteclose}\ \isacommand{by}\isamarkupfalse%
\ {\isacharparenleft}rule\ runiq{\isacharunderscore}conv{\isacharunderscore}imp{\isacharunderscore}THE{\isacharunderscore}left{\isacharunderscore}comp{\isacharparenright}\isanewline
\ \ \isacommand{with}\isamarkupfalse%
\ {\isacharasterisk}\ \isacommand{show}\isamarkupfalse%
\ {\isacharquery}thesis\ \isacommand{by}\isamarkupfalse%
\ simp\isanewline
\isacommand{qed}\isamarkupfalse%
%
\endisatagproof
{\isafoldproof}%
%
\isadelimproof
\isanewline
%
\endisadelimproof
\isanewline
\isacommand{lemma}\isamarkupfalse%
\ THE{\isacharunderscore}left{\isacharunderscore}comp{\isacharunderscore}imp{\isacharunderscore}runiq{\isacharunderscore}conv{\isacharcolon}\isanewline
\ \ \isakeyword{assumes}\ {\isachardoublequoteopen}{\isasymforall}\ a\ b\ {\isachardot}\ {\isacharparenleft}a{\isacharcomma}\ b{\isacharparenright}\ {\isasymin}\ R\ {\isasymlongrightarrow}\ a\ {\isacharequal}\ {\isacharparenleft}THE\ a\ {\isachardot}\ {\isacharparenleft}a{\isacharcomma}\ b{\isacharparenright}\ {\isasymin}\ R{\isacharparenright}{\isachardoublequoteclose}\isanewline
\ \ \isakeyword{shows}\ {\isachardoublequoteopen}runiq\ {\isacharparenleft}R{\isasyminverse}{\isacharparenright}{\isachardoublequoteclose}\isanewline
%
\isadelimproof
%
\endisadelimproof
%
\isatagproof
\isacommand{proof}\isamarkupfalse%
\ {\isacharminus}\isanewline
\ \ \isacommand{from}\isamarkupfalse%
\ assms\ \isacommand{have}\isamarkupfalse%
\ {\isachardoublequoteopen}{\isasymforall}\ b\ a\ {\isachardot}\ {\isacharparenleft}b{\isacharcomma}\ a{\isacharparenright}\ {\isasymin}\ R{\isasyminverse}\ {\isasymlongrightarrow}\ a\ {\isacharequal}\ {\isacharparenleft}THE\ a\ {\isachardot}\ {\isacharparenleft}b{\isacharcomma}\ a{\isacharparenright}\ {\isasymin}\ R{\isasyminverse}{\isacharparenright}{\isachardoublequoteclose}\ \isacommand{by}\isamarkupfalse%
\ auto\isanewline
\ \ \isacommand{then}\isamarkupfalse%
\ \isacommand{show}\isamarkupfalse%
\ {\isacharquery}thesis\ \isacommand{by}\isamarkupfalse%
\ {\isacharparenleft}rule\ THE{\isacharunderscore}right{\isacharunderscore}comp{\isacharunderscore}imp{\isacharunderscore}runiq{\isacharparenright}\isanewline
\isacommand{qed}\isamarkupfalse%
%
\endisatagproof
{\isafoldproof}%
%
\isadelimproof
\isanewline
%
\endisadelimproof
\isanewline
\isacommand{lemma}\isamarkupfalse%
\ runiq{\isacharunderscore}conv{\isacharunderscore}wrt{\isacharunderscore}THE{\isacharcolon}\isanewline
\ \ {\isachardoublequoteopen}runiq\ {\isacharparenleft}R{\isasyminverse}{\isacharparenright}\ {\isasymlongleftrightarrow}\ {\isacharparenleft}{\isasymforall}\ a\ b\ {\isachardot}\ {\isacharparenleft}a{\isacharcomma}\ b{\isacharparenright}\ {\isasymin}\ R\ {\isasymlongrightarrow}\ a\ {\isacharequal}\ {\isacharparenleft}THE\ a\ {\isachardot}\ {\isacharparenleft}a{\isacharcomma}\ b{\isacharparenright}\ {\isasymin}\ R{\isacharparenright}{\isacharparenright}{\isachardoublequoteclose}\isanewline
%
\isadelimproof
%
\endisadelimproof
%
\isatagproof
\isacommand{proof}\isamarkupfalse%
\ {\isacharminus}\isanewline
\ \ \isacommand{have}\isamarkupfalse%
\ {\isachardoublequoteopen}runiq\ {\isacharparenleft}R{\isasyminverse}{\isacharparenright}\ {\isasymlongleftrightarrow}\ {\isacharparenleft}{\isasymforall}\ a\ b\ {\isachardot}\ {\isacharparenleft}a{\isacharcomma}\ b{\isacharparenright}\ {\isasymin}\ R{\isasyminverse}\ {\isasymlongrightarrow}\ b\ {\isacharequal}\ {\isacharparenleft}THE\ b\ {\isachardot}\ {\isacharparenleft}a{\isacharcomma}\ b{\isacharparenright}\ {\isasymin}\ R{\isasyminverse}{\isacharparenright}{\isacharparenright}{\isachardoublequoteclose}\ \isacommand{by}\isamarkupfalse%
\ {\isacharparenleft}rule\ runiq{\isacharunderscore}wrt{\isacharunderscore}THE{\isacharparenright}\isanewline
\ \ \isacommand{also}\isamarkupfalse%
\ \isacommand{have}\isamarkupfalse%
\ {\isachardoublequoteopen}{\isasymdots}\ {\isasymlongleftrightarrow}\ {\isacharparenleft}{\isasymforall}\ a\ b\ {\isachardot}\ {\isacharparenleft}a{\isacharcomma}\ b{\isacharparenright}\ {\isasymin}\ R\ {\isasymlongrightarrow}\ a\ {\isacharequal}\ {\isacharparenleft}THE\ a\ {\isachardot}\ {\isacharparenleft}a{\isacharcomma}\ b{\isacharparenright}\ {\isasymin}\ R{\isacharparenright}{\isacharparenright}{\isachardoublequoteclose}\ \isacommand{by}\isamarkupfalse%
\ auto\isanewline
\ \ \isacommand{finally}\isamarkupfalse%
\ \isacommand{show}\isamarkupfalse%
\ {\isacharquery}thesis\ \isacommand{{\isachardot}}\isamarkupfalse%
\isanewline
\isacommand{qed}\isamarkupfalse%
%
\endisatagproof
{\isafoldproof}%
%
\isadelimproof
\isanewline
%
\endisadelimproof
\isanewline
\isacommand{lemma}\isamarkupfalse%
\ lm{\isadigit{0}}{\isadigit{2}}{\isadigit{2}}{\isacharcolon}\ \ \isakeyword{assumes}\ {\isachardoublequoteopen}trivial\ f{\isachardoublequoteclose}\ \isakeyword{shows}\ {\isachardoublequoteopen}runiq\ f{\isachardoublequoteclose}%
\isadelimproof
\ %
\endisadelimproof
%
\isatagproof
\isacommand{using}\isamarkupfalse%
\ assms\ \isacommand{by}\isamarkupfalse%
\ {\isacharparenleft}metis\ {\isacharparenleft}erased{\isacharcomma}\ hide{\isacharunderscore}lams{\isacharparenright}\ lm{\isadigit{0}}{\isadigit{1}}\ runiq{\isacharunderscore}basic\ snd{\isacharunderscore}conv{\isacharparenright}%
\endisatagproof
{\isafoldproof}%
%
\isadelimproof
%
\endisadelimproof
%
\begin{isamarkuptext}%
A singleton relation is right-unique.%
\end{isamarkuptext}%
\isamarkuptrue%
\isacommand{corollary}\isamarkupfalse%
\ runiq{\isacharunderscore}singleton{\isacharunderscore}rel{\isacharcolon}\ {\isachardoublequoteopen}runiq\ {\isacharbraceleft}{\isacharparenleft}x{\isacharcomma}\ y{\isacharparenright}{\isacharbraceright}{\isachardoublequoteclose}\ {\isacharparenleft}\isakeyword{is}\ {\isachardoublequoteopen}runiq\ {\isacharquery}R{\isachardoublequoteclose}{\isacharparenright}\isanewline
%
\isadelimproof
%
\endisadelimproof
%
\isatagproof
\isacommand{using}\isamarkupfalse%
\ trivial{\isacharunderscore}singleton\ lm{\isadigit{0}}{\isadigit{2}}{\isadigit{2}}\ \isacommand{by}\isamarkupfalse%
\ fast%
\endisatagproof
{\isafoldproof}%
%
\isadelimproof
%
\endisadelimproof
%
\begin{isamarkuptext}%
The empty relation is right-unique%
\end{isamarkuptext}%
\isamarkuptrue%
\isacommand{lemma}\isamarkupfalse%
\ runiq{\isacharunderscore}emptyrel{\isacharcolon}\ {\isachardoublequoteopen}runiq\ {\isacharbraceleft}{\isacharbraceright}{\isachardoublequoteclose}%
\isadelimproof
\ %
\endisadelimproof
%
\isatagproof
\isacommand{using}\isamarkupfalse%
\ trivial{\isacharunderscore}empty\ lm{\isadigit{0}}{\isadigit{2}}{\isadigit{2}}\ \isacommand{by}\isamarkupfalse%
\ blast%
\endisatagproof
{\isafoldproof}%
%
\isadelimproof
%
\endisadelimproof
%
\begin{isamarkuptext}%
alternative characterisation of the fact that, if a relation \isa{R} is right-unique,
  its evaluation \isa{R\ {\isacharcomma}{\isacharcomma}\ x} on some argument \isa{x} in its domain, occurs in \isa{R}'s
  range.%
\end{isamarkuptext}%
\isamarkuptrue%
\isacommand{lemma}\isamarkupfalse%
\ eval{\isacharunderscore}runiq{\isacharunderscore}rel{\isacharcolon}\isanewline
\ \ \isakeyword{assumes}\ domain{\isacharcolon}\ {\isachardoublequoteopen}x\ {\isasymin}\ Domain\ R{\isachardoublequoteclose}\isanewline
\ \ \ \ \ \ \isakeyword{and}\ runiq{\isacharcolon}\ {\isachardoublequoteopen}runiq\ R{\isachardoublequoteclose}\ \isanewline
\ \ \isakeyword{shows}\ {\isachardoublequoteopen}{\isacharparenleft}x{\isacharcomma}\ R{\isacharcomma}{\isacharcomma}x{\isacharparenright}\ {\isasymin}\ R{\isachardoublequoteclose}\isanewline
%
\isadelimproof
%
\endisadelimproof
%
\isatagproof
\isacommand{using}\isamarkupfalse%
\ assms\ \isacommand{by}\isamarkupfalse%
\ {\isacharparenleft}metis\ l{\isadigit{3}}{\isadigit{1}}\ runiq{\isacharunderscore}wrt{\isacharunderscore}ex{\isadigit{1}}{\isacharparenright}%
\endisatagproof
{\isafoldproof}%
%
\isadelimproof
%
\endisadelimproof
%
\begin{isamarkuptext}%
Evaluating a right-unique relation as a function on the relation's domain yields an
  element from its range.%
\end{isamarkuptext}%
\isamarkuptrue%
\isacommand{lemma}\isamarkupfalse%
\ eval{\isacharunderscore}runiq{\isacharunderscore}in{\isacharunderscore}Range{\isacharcolon}\isanewline
\ \ \isakeyword{assumes}\ {\isachardoublequoteopen}runiq\ R{\isachardoublequoteclose}\isanewline
\ \ \ \ \ \ \isakeyword{and}\ {\isachardoublequoteopen}a\ {\isasymin}\ Domain\ R{\isachardoublequoteclose}\isanewline
\ \ \isakeyword{shows}\ {\isachardoublequoteopen}R\ {\isacharcomma}{\isacharcomma}\ a\ {\isasymin}\ Range\ R{\isachardoublequoteclose}\isanewline
%
\isadelimproof
%
\endisadelimproof
%
\isatagproof
\isacommand{using}\isamarkupfalse%
\ assms\ \isacommand{by}\isamarkupfalse%
\ {\isacharparenleft}metis\ Range{\isacharunderscore}iff\ eval{\isacharunderscore}runiq{\isacharunderscore}rel{\isacharparenright}%
\endisatagproof
{\isafoldproof}%
%
\isadelimproof
%
\endisadelimproof
%
\begin{isamarkuptext}%
right-uniqueness of a restricted relation expressed using basic set theory%
\end{isamarkuptext}%
\isamarkuptrue%
\isacommand{lemma}\isamarkupfalse%
\ runiq{\isacharunderscore}restrict{\isacharcolon}\ {\isachardoublequoteopen}runiq\ {\isacharparenleft}R\ {\isacharbar}{\isacharbar}\ X{\isacharparenright}\ {\isasymlongleftrightarrow}\ {\isacharparenleft}{\isasymforall}\ x\ {\isasymin}\ X\ {\isachardot}\ {\isasymforall}\ y\ y{\isacharprime}\ {\isachardot}\ {\isacharparenleft}x{\isacharcomma}\ y{\isacharparenright}\ {\isasymin}\ R\ {\isasymand}\ {\isacharparenleft}x{\isacharcomma}\ y{\isacharprime}{\isacharparenright}\ {\isasymin}\ R\ {\isasymlongrightarrow}\ y\ {\isacharequal}\ y{\isacharprime}{\isacharparenright}{\isachardoublequoteclose}\isanewline
%
\isadelimproof
%
\endisadelimproof
%
\isatagproof
\isacommand{proof}\isamarkupfalse%
\ {\isacharminus}\isanewline
\ \ \isacommand{have}\isamarkupfalse%
\ {\isachardoublequoteopen}runiq\ {\isacharparenleft}R\ {\isacharbar}{\isacharbar}\ X{\isacharparenright}\ {\isasymlongleftrightarrow}\ {\isacharparenleft}{\isasymforall}\ x\ y\ y{\isacharprime}\ {\isachardot}\ {\isacharparenleft}x{\isacharcomma}\ y{\isacharparenright}\ {\isasymin}\ R\ {\isacharbar}{\isacharbar}\ X\ {\isasymand}\ {\isacharparenleft}x{\isacharcomma}\ y{\isacharprime}{\isacharparenright}\ {\isasymin}\ R\ {\isacharbar}{\isacharbar}\ X\ {\isasymlongrightarrow}\ y\ {\isacharequal}\ y{\isacharprime}{\isacharparenright}{\isachardoublequoteclose}\isanewline
\ \ \ \ \isacommand{by}\isamarkupfalse%
\ {\isacharparenleft}rule\ runiq{\isacharunderscore}basic{\isacharparenright}\isanewline
\ \ \isacommand{also}\isamarkupfalse%
\ \isacommand{have}\isamarkupfalse%
\ {\isachardoublequoteopen}{\isasymdots}\ {\isasymlongleftrightarrow}\ {\isacharparenleft}{\isasymforall}\ x\ y\ y{\isacharprime}\ {\isachardot}\ {\isacharparenleft}x{\isacharcomma}\ y{\isacharparenright}\ {\isasymin}\ {\isacharbraceleft}\ p\ {\isachardot}\ fst\ p\ {\isasymin}\ X\ {\isasymand}\ p\ {\isasymin}\ R\ {\isacharbraceright}\ {\isasymand}\ {\isacharparenleft}x{\isacharcomma}\ y{\isacharprime}{\isacharparenright}\ {\isasymin}\ {\isacharbraceleft}\ p\ {\isachardot}\ fst\ p\ {\isasymin}\ X\ {\isasymand}\ p\ {\isasymin}\ R\ {\isacharbraceright}\ {\isasymlongrightarrow}\ y\ {\isacharequal}\ y{\isacharprime}{\isacharparenright}{\isachardoublequoteclose}\isanewline
\ \ \ \ \isacommand{using}\isamarkupfalse%
\ restrict{\isacharunderscore}ext{\isacharprime}\ \isacommand{by}\isamarkupfalse%
\ blast\isanewline
\ \ \isacommand{also}\isamarkupfalse%
\ \isacommand{have}\isamarkupfalse%
\ {\isachardoublequoteopen}{\isasymdots}\ {\isasymlongleftrightarrow}\ {\isacharparenleft}{\isasymforall}\ x\ {\isasymin}\ X\ {\isachardot}\ {\isasymforall}\ y\ y{\isacharprime}\ {\isachardot}\ {\isacharparenleft}x{\isacharcomma}\ y{\isacharparenright}\ {\isasymin}\ R\ {\isasymand}\ {\isacharparenleft}x{\isacharcomma}\ y{\isacharprime}{\isacharparenright}\ {\isasymin}\ R\ {\isasymlongrightarrow}\ y\ {\isacharequal}\ y{\isacharprime}{\isacharparenright}{\isachardoublequoteclose}\ \isacommand{by}\isamarkupfalse%
\ auto\isanewline
\ \ \isacommand{finally}\isamarkupfalse%
\ \isacommand{show}\isamarkupfalse%
\ {\isacharquery}thesis\ \isacommand{{\isachardot}}\isamarkupfalse%
\isanewline
\isacommand{qed}\isamarkupfalse%
%
\endisatagproof
{\isafoldproof}%
%
\isadelimproof
%
\endisadelimproof
%
\isamarkupsubsection{paste%
}
\isamarkuptrue%
%
\begin{isamarkuptext}%
Pasting a singleton relation on some other right-unique relation \isa{R} yields a
  right-unique relation if the single element of the singleton's domain is not yet in the 
  domain of \isa{R}.%
\end{isamarkuptext}%
\isamarkuptrue%
\isacommand{lemma}\isamarkupfalse%
\ runiq{\isacharunderscore}paste{\isadigit{3}}{\isacharcolon}\isanewline
\ \ \isakeyword{assumes}\ {\isachardoublequoteopen}runiq\ R{\isachardoublequoteclose}\isanewline
\ \ \ \ \ \ \isakeyword{and}\ {\isachardoublequoteopen}x\ {\isasymnotin}\ Domain\ R{\isachardoublequoteclose}\ \isanewline
\ \ \isakeyword{shows}\ {\isachardoublequoteopen}runiq\ {\isacharparenleft}R\ {\isacharplus}{\isacharasterisk}\ {\isacharbraceleft}{\isacharparenleft}x{\isacharcomma}\ y{\isacharparenright}{\isacharbraceright}{\isacharparenright}{\isachardoublequoteclose}\isanewline
%
\isadelimproof
%
\endisadelimproof
%
\isatagproof
\isacommand{using}\isamarkupfalse%
\ assms\ runiq{\isacharunderscore}paste{\isadigit{2}}\ runiq{\isacharunderscore}singleton{\isacharunderscore}rel\ \isacommand{by}\isamarkupfalse%
\ metis%
\endisatagproof
{\isafoldproof}%
%
\isadelimproof
%
\endisadelimproof
%
\isamarkupsubsection{difference%
}
\isamarkuptrue%
%
\begin{isamarkuptext}%
Removing one pair from a right-unique relation still leaves it right-unique.%
\end{isamarkuptext}%
\isamarkuptrue%
\isacommand{lemma}\isamarkupfalse%
\ runiq{\isacharunderscore}except{\isacharcolon}\isanewline
\ \ \isakeyword{assumes}\ {\isachardoublequoteopen}runiq\ R{\isachardoublequoteclose}\isanewline
\ \ \isakeyword{shows}\ {\isachardoublequoteopen}runiq\ {\isacharparenleft}R\ {\isacharminus}\ {\isacharbraceleft}tup{\isacharbraceright}{\isacharparenright}{\isachardoublequoteclose}\isanewline
%
\isadelimproof
%
\endisadelimproof
%
\isatagproof
\isacommand{using}\isamarkupfalse%
\ assms\isanewline
\isacommand{by}\isamarkupfalse%
\ {\isacharparenleft}rule\ subrel{\isacharunderscore}runiq{\isacharparenright}\ fast%
\endisatagproof
{\isafoldproof}%
%
\isadelimproof
\isanewline
%
\endisadelimproof
\isanewline
\isacommand{lemma}\isamarkupfalse%
\ runiq{\isacharunderscore}Diff{\isacharunderscore}singleton{\isacharunderscore}Domain{\isacharcolon}\isanewline
\ \ \isakeyword{assumes}\ runiq{\isacharcolon}\ {\isachardoublequoteopen}runiq\ R{\isachardoublequoteclose}\isanewline
\ \ \ \ \ \ \isakeyword{and}\ in{\isacharunderscore}rel{\isacharcolon}\ {\isachardoublequoteopen}{\isacharparenleft}x{\isacharcomma}\ y{\isacharparenright}\ {\isasymin}\ R{\isachardoublequoteclose}\isanewline
\ \ \isakeyword{shows}\ {\isachardoublequoteopen}x\ {\isasymnotin}\ Domain\ {\isacharparenleft}R\ {\isacharminus}\ {\isacharbraceleft}{\isacharparenleft}x{\isacharcomma}\ y{\isacharparenright}{\isacharbraceright}{\isacharparenright}{\isachardoublequoteclose}\isanewline
%
\isadelimproof
\isanewline
%
\endisadelimproof
%
\isatagproof
\isacommand{using}\isamarkupfalse%
\ assms\ DomainE\ Domain{\isacharunderscore}Un{\isacharunderscore}eq\ UnI{\isadigit{1}}\ Un{\isacharunderscore}Diff{\isacharunderscore}Int\ member{\isacharunderscore}remove\ remove{\isacharunderscore}def\ runiq{\isacharunderscore}wrt{\isacharunderscore}ex{\isadigit{1}}\isanewline
\isacommand{by}\isamarkupfalse%
\ metis%
\endisatagproof
{\isafoldproof}%
%
\isadelimproof
%
\endisadelimproof
%
\isamarkupsubsection{converse%
}
\isamarkuptrue%
%
\begin{isamarkuptext}%
The inverse image of the image of a singleton set under some relation is the same
  singleton set, if both the relation and its converse are right-unique and the singleton set
  is in the relation's domain.%
\end{isamarkuptext}%
\isamarkuptrue%
\isacommand{lemma}\isamarkupfalse%
\ converse{\isacharunderscore}Image{\isacharunderscore}singleton{\isacharunderscore}Domain{\isacharcolon}\isanewline
\ \ \isakeyword{assumes}\ runiq{\isacharcolon}\ {\isachardoublequoteopen}runiq\ R{\isachardoublequoteclose}\isanewline
\ \ \ \ \ \ \isakeyword{and}\ runiq{\isacharunderscore}conv{\isacharcolon}\ {\isachardoublequoteopen}runiq\ {\isacharparenleft}R{\isasyminverse}{\isacharparenright}{\isachardoublequoteclose}\isanewline
\ \ \ \ \ \ \isakeyword{and}\ domain{\isacharcolon}\ {\isachardoublequoteopen}x\ {\isasymin}\ Domain\ R{\isachardoublequoteclose}\isanewline
\isakeyword{shows}\ {\isachardoublequoteopen}R{\isasyminverse}\ {\isacharbackquote}{\isacharbackquote}\ R\ {\isacharbackquote}{\isacharbackquote}\ {\isacharbraceleft}x{\isacharbraceright}\ {\isacharequal}\ {\isacharbraceleft}x{\isacharbraceright}{\isachardoublequoteclose}\isanewline
%
\isadelimproof
%
\endisadelimproof
%
\isatagproof
\isacommand{proof}\isamarkupfalse%
\ {\isacharminus}\isanewline
\ \ \isacommand{have}\isamarkupfalse%
\ sup{\isacharcolon}\ {\isachardoublequoteopen}{\isacharbraceleft}x{\isacharbraceright}\ {\isasymsubseteq}\ R{\isasyminverse}\ {\isacharbackquote}{\isacharbackquote}\ R\ {\isacharbackquote}{\isacharbackquote}\ {\isacharbraceleft}x{\isacharbraceright}{\isachardoublequoteclose}\ \isacommand{using}\isamarkupfalse%
\ domain\ \isacommand{by}\isamarkupfalse%
\ fast\isanewline
\ \ \isacommand{have}\isamarkupfalse%
\ {\isachardoublequoteopen}trivial\ {\isacharparenleft}R\ {\isacharbackquote}{\isacharbackquote}\ {\isacharbraceleft}x{\isacharbraceright}{\isacharparenright}{\isachardoublequoteclose}\ \isacommand{using}\isamarkupfalse%
\ runiq\ domain\ \isacommand{by}\isamarkupfalse%
\ {\isacharparenleft}metis\ runiq{\isacharunderscore}def\ trivial{\isacharunderscore}singleton{\isacharparenright}\isanewline
\ \ \isacommand{then}\isamarkupfalse%
\ \isacommand{have}\isamarkupfalse%
\ {\isachardoublequoteopen}trivial\ {\isacharparenleft}R{\isasyminverse}\ {\isacharbackquote}{\isacharbackquote}\ R\ {\isacharbackquote}{\isacharbackquote}\ {\isacharbraceleft}x{\isacharbraceright}{\isacharparenright}{\isachardoublequoteclose}\isanewline
\ \ \ \ \isacommand{using}\isamarkupfalse%
\ assms\ runiq{\isacharunderscore}def\ \isacommand{by}\isamarkupfalse%
\ blast\isanewline
\ \ \isacommand{then}\isamarkupfalse%
\ \isacommand{show}\isamarkupfalse%
\ {\isacharquery}thesis\isanewline
\ \ \ \ \isacommand{using}\isamarkupfalse%
\ sup\ \isacommand{by}\isamarkupfalse%
\ {\isacharparenleft}metis\ singleton{\isacharunderscore}sub{\isacharunderscore}trivial{\isacharunderscore}uniq\ subset{\isacharunderscore}antisym\ trivial{\isacharunderscore}def{\isacharparenright}\isanewline
\isacommand{qed}\isamarkupfalse%
%
\endisatagproof
{\isafoldproof}%
%
\isadelimproof
%
\endisadelimproof
%
\begin{isamarkuptext}%
The inverse image of the image of a singleton set under some relation is the same
  singleton set or empty, if both the relation and its converse are right-unique.%
\end{isamarkuptext}%
\isamarkuptrue%
\isacommand{corollary}\isamarkupfalse%
\ converse{\isacharunderscore}Image{\isacharunderscore}singleton{\isacharcolon}\isanewline
\ \ \isakeyword{assumes}\ {\isachardoublequoteopen}runiq\ R{\isachardoublequoteclose}\isanewline
\ \ \ \ \ \ \isakeyword{and}\ {\isachardoublequoteopen}runiq\ {\isacharparenleft}R{\isasyminverse}{\isacharparenright}{\isachardoublequoteclose}\isanewline
\ \ \isakeyword{shows}\ {\isachardoublequoteopen}R{\isasyminverse}\ {\isacharbackquote}{\isacharbackquote}\ R\ {\isacharbackquote}{\isacharbackquote}\ {\isacharbraceleft}x{\isacharbraceright}\ {\isasymsubseteq}\ {\isacharbraceleft}x{\isacharbraceright}{\isachardoublequoteclose}\isanewline
%
\isadelimproof
%
\endisadelimproof
%
\isatagproof
\isacommand{using}\isamarkupfalse%
\ assms\ converse{\isacharunderscore}Image{\isacharunderscore}singleton{\isacharunderscore}Domain\ \isacommand{by}\isamarkupfalse%
\ fast%
\endisatagproof
{\isafoldproof}%
%
\isadelimproof
%
\endisadelimproof
%
\begin{isamarkuptext}%
The inverse image of the image of a set under some relation is a subset of that set,
  if both the relation and its converse are right-unique.%
\end{isamarkuptext}%
\isamarkuptrue%
\isacommand{lemma}\isamarkupfalse%
\ disj{\isacharunderscore}Domain{\isacharunderscore}imp{\isacharunderscore}disj{\isacharunderscore}Image{\isacharcolon}\ \isakeyword{assumes}\ {\isachardoublequoteopen}Domain\ R\ {\isasyminter}\ X\ {\isasyminter}\ Y\ {\isacharequal}\ {\isacharbraceleft}{\isacharbraceright}{\isachardoublequoteclose}\ \isanewline
\ \ \isakeyword{assumes}\ {\isachardoublequoteopen}runiq\ R{\isachardoublequoteclose}\isanewline
\ \ \ \ \ \ \isakeyword{and}\ {\isachardoublequoteopen}runiq\ {\isacharparenleft}R{\isasyminverse}{\isacharparenright}{\isachardoublequoteclose}\isanewline
\ \ \isakeyword{shows}\ {\isachardoublequoteopen}R\ {\isacharbackquote}{\isacharbackquote}\ X\ {\isasyminter}\ R\ {\isacharbackquote}{\isacharbackquote}\ Y\ {\isacharequal}\ {\isacharbraceleft}{\isacharbraceright}{\isachardoublequoteclose}\ \isanewline
%
\isadelimproof
%
\endisadelimproof
%
\isatagproof
\isacommand{using}\isamarkupfalse%
\ assms\ \isacommand{unfolding}\isamarkupfalse%
\ runiq{\isacharunderscore}basic\ \isacommand{by}\isamarkupfalse%
\ blast%
\endisatagproof
{\isafoldproof}%
%
\isadelimproof
\isanewline
%
\endisadelimproof
\isanewline
\isacommand{lemma}\isamarkupfalse%
\ runiq{\isacharunderscore}imp{\isacharunderscore}Dom{\isacharunderscore}rel{\isacharunderscore}Range{\isacharcolon}\isanewline
\ \ \isakeyword{assumes}\ {\isachardoublequoteopen}x\ {\isasymin}\ Domain\ R{\isachardoublequoteclose}\isanewline
\ \ \ \ \ \ \isakeyword{and}\ {\isachardoublequoteopen}runiq\ R{\isachardoublequoteclose}\isanewline
\ \ \isakeyword{shows}\ {\isachardoublequoteopen}{\isacharparenleft}THE\ y\ {\isachardot}\ {\isacharparenleft}x{\isacharcomma}\ y{\isacharparenright}\ {\isasymin}\ R{\isacharparenright}\ {\isasymin}\ Range\ R{\isachardoublequoteclose}\isanewline
%
\isadelimproof
%
\endisadelimproof
%
\isatagproof
\isacommand{using}\isamarkupfalse%
\ assms\isanewline
\isacommand{by}\isamarkupfalse%
\ {\isacharparenleft}metis\ Range{\isachardot}intros\ runiq{\isacharunderscore}imp{\isacharunderscore}THE{\isacharunderscore}right{\isacharunderscore}comp\ runiq{\isacharunderscore}wrt{\isacharunderscore}ex{\isadigit{1}}{\isacharparenright}%
\endisatagproof
{\isafoldproof}%
%
\isadelimproof
\isanewline
%
\endisadelimproof
\isanewline
\isacommand{lemma}\isamarkupfalse%
\ runiq{\isacharunderscore}conv{\isacharunderscore}imp{\isacharunderscore}Range{\isacharunderscore}rel{\isacharunderscore}Dom{\isacharcolon}\isanewline
\ \ \isakeyword{assumes}\ y{\isacharunderscore}Range{\isacharcolon}\ {\isachardoublequoteopen}y\ {\isasymin}\ Range\ R{\isachardoublequoteclose}\isanewline
\ \ \ \ \ \ \isakeyword{and}\ runiq{\isacharunderscore}conv{\isacharcolon}\ {\isachardoublequoteopen}runiq\ {\isacharparenleft}R{\isasyminverse}{\isacharparenright}{\isachardoublequoteclose}\isanewline
\ \ \isakeyword{shows}\ {\isachardoublequoteopen}{\isacharparenleft}THE\ x\ {\isachardot}\ {\isacharparenleft}x{\isacharcomma}\ y{\isacharparenright}\ {\isasymin}\ R{\isacharparenright}\ {\isasymin}\ Domain\ R{\isachardoublequoteclose}\isanewline
%
\isadelimproof
%
\endisadelimproof
%
\isatagproof
\isacommand{proof}\isamarkupfalse%
\ {\isacharminus}\isanewline
\ \ \isacommand{from}\isamarkupfalse%
\ y{\isacharunderscore}Range\ \isacommand{have}\isamarkupfalse%
\ {\isachardoublequoteopen}y\ {\isasymin}\ Domain\ {\isacharparenleft}R{\isasyminverse}{\isacharparenright}{\isachardoublequoteclose}\ \isacommand{by}\isamarkupfalse%
\ simp\isanewline
\ \ \isacommand{then}\isamarkupfalse%
\ \isacommand{have}\isamarkupfalse%
\ {\isachardoublequoteopen}{\isacharparenleft}THE\ x\ {\isachardot}\ {\isacharparenleft}y{\isacharcomma}\ x{\isacharparenright}\ {\isasymin}\ R{\isasyminverse}{\isacharparenright}\ {\isasymin}\ Range\ {\isacharparenleft}R{\isasyminverse}{\isacharparenright}{\isachardoublequoteclose}\ \isacommand{using}\isamarkupfalse%
\ runiq{\isacharunderscore}conv\ \isacommand{by}\isamarkupfalse%
\ {\isacharparenleft}rule\ runiq{\isacharunderscore}imp{\isacharunderscore}Dom{\isacharunderscore}rel{\isacharunderscore}Range{\isacharparenright}\isanewline
\ \ \isacommand{then}\isamarkupfalse%
\ \isacommand{show}\isamarkupfalse%
\ {\isacharquery}thesis\ \isacommand{by}\isamarkupfalse%
\ simp\isanewline
\isacommand{qed}\isamarkupfalse%
%
\endisatagproof
{\isafoldproof}%
%
\isadelimproof
%
\endisadelimproof
%
\begin{isamarkuptext}%
The converse relation of two pasted relations is right-unique, if 
  the relations have disjoint domains and ranges, and if their converses are both
  right-unique.%
\end{isamarkuptext}%
\isamarkuptrue%
\isacommand{lemma}\isamarkupfalse%
\ runiq{\isacharunderscore}converse{\isacharunderscore}paste{\isacharcolon}\ \isanewline
\ \ \isakeyword{assumes}\ runiq{\isacharunderscore}P{\isacharunderscore}conv{\isacharcolon}\ {\isachardoublequoteopen}runiq\ {\isacharparenleft}P{\isasyminverse}{\isacharparenright}{\isachardoublequoteclose}\isanewline
\ \ \ \ \ \ \isakeyword{and}\ runiq{\isacharunderscore}Q{\isacharunderscore}conv{\isacharcolon}\ {\isachardoublequoteopen}runiq\ {\isacharparenleft}Q{\isasyminverse}{\isacharparenright}{\isachardoublequoteclose}\isanewline
\ \ \ \ \ \ \isakeyword{and}\ disj{\isacharunderscore}D{\isacharcolon}\ {\isachardoublequoteopen}Domain\ P\ {\isasyminter}\ Domain\ Q\ {\isacharequal}\ {\isacharbraceleft}{\isacharbraceright}{\isachardoublequoteclose}\isanewline
\ \ \ \ \ \ \isakeyword{and}\ disj{\isacharunderscore}R{\isacharcolon}\ {\isachardoublequoteopen}Range\ P\ {\isasyminter}\ Range\ Q\ {\isacharequal}\ {\isacharbraceleft}{\isacharbraceright}{\isachardoublequoteclose}\isanewline
\ \ \isakeyword{shows}\ {\isachardoublequoteopen}runiq\ {\isacharparenleft}{\isacharparenleft}P\ {\isacharplus}{\isacharasterisk}\ Q{\isacharparenright}{\isasyminverse}{\isacharparenright}{\isachardoublequoteclose}\isanewline
%
\isadelimproof
%
\endisadelimproof
%
\isatagproof
\isacommand{proof}\isamarkupfalse%
\ {\isacharminus}\isanewline
\ \ \isacommand{have}\isamarkupfalse%
\ {\isachardoublequoteopen}P\ {\isacharplus}{\isacharasterisk}\ Q\ {\isacharequal}\ P\ {\isasymunion}\ Q{\isachardoublequoteclose}\ \isacommand{using}\isamarkupfalse%
\ disj{\isacharunderscore}D\ \isacommand{by}\isamarkupfalse%
\ {\isacharparenleft}rule\ paste{\isacharunderscore}disj{\isacharunderscore}domains{\isacharparenright}\isanewline
\ \ \isacommand{then}\isamarkupfalse%
\ \isacommand{have}\isamarkupfalse%
\ {\isachardoublequoteopen}{\isacharparenleft}P\ {\isacharplus}{\isacharasterisk}\ Q{\isacharparenright}{\isasyminverse}\ {\isacharequal}\ P{\isasyminverse}\ {\isasymunion}\ Q{\isasyminverse}{\isachardoublequoteclose}\ \isacommand{by}\isamarkupfalse%
\ auto\isanewline
\ \ \isacommand{also}\isamarkupfalse%
\ \isacommand{have}\isamarkupfalse%
\ {\isachardoublequoteopen}{\isasymdots}\ {\isacharequal}\ P{\isasyminverse}\ {\isacharplus}{\isacharasterisk}\ Q{\isasyminverse}{\isachardoublequoteclose}\ \isacommand{using}\isamarkupfalse%
\ disj{\isacharunderscore}R\ paste{\isacharunderscore}disj{\isacharunderscore}domains\ Domain{\isacharunderscore}converse\ \isacommand{by}\isamarkupfalse%
\ metis\isanewline
\ \ \isacommand{finally}\isamarkupfalse%
\ \isacommand{show}\isamarkupfalse%
\ {\isacharquery}thesis\ \isacommand{using}\isamarkupfalse%
\ runiq{\isacharunderscore}P{\isacharunderscore}conv\ runiq{\isacharunderscore}Q{\isacharunderscore}conv\ runiq{\isacharunderscore}paste{\isadigit{2}}\ \isacommand{by}\isamarkupfalse%
\ auto\isanewline
\isacommand{qed}\isamarkupfalse%
%
\endisatagproof
{\isafoldproof}%
%
\isadelimproof
%
\endisadelimproof
%
\begin{isamarkuptext}%
The converse relation of a singleton relation pasted on some other relation \isa{R} is right-unique,
  if the singleton pair is not in \isa{Domain\ R\ {\isasymtimes}\ Range\ R}, and if \isa{R{\isasyminverse}} is right-unique.%
\end{isamarkuptext}%
\isamarkuptrue%
\isacommand{lemma}\isamarkupfalse%
\ runiq{\isacharunderscore}converse{\isacharunderscore}paste{\isacharunderscore}singleton{\isacharcolon}\isanewline
\ \ \isakeyword{assumes}\ runiq{\isacharcolon}\ {\isachardoublequoteopen}runiq\ {\isacharparenleft}R{\isasyminverse}{\isacharparenright}{\isachardoublequoteclose}\ \isanewline
\ \ \ \ \ \ \isakeyword{and}\ y{\isacharunderscore}notin{\isacharunderscore}R{\isacharcolon}\ {\isachardoublequoteopen}y\ {\isasymnotin}\ Range\ R{\isachardoublequoteclose}\isanewline
\ \ \ \ \ \ \isakeyword{and}\ x{\isacharunderscore}notin{\isacharunderscore}D{\isacharcolon}\ {\isachardoublequoteopen}x\ {\isasymnotin}\ Domain\ R{\isachardoublequoteclose}\isanewline
\ \ \isakeyword{shows}\ {\isachardoublequoteopen}runiq\ {\isacharparenleft}{\isacharparenleft}R\ {\isacharplus}{\isacharasterisk}\ {\isacharbraceleft}{\isacharparenleft}x{\isacharcomma}y{\isacharparenright}{\isacharbraceright}{\isacharparenright}{\isasyminverse}{\isacharparenright}{\isachardoublequoteclose}\isanewline
%
\isadelimproof
%
\endisadelimproof
%
\isatagproof
\isacommand{proof}\isamarkupfalse%
\ {\isacharminus}\isanewline
\ \ \isacommand{have}\isamarkupfalse%
\ {\isachardoublequoteopen}{\isacharbraceleft}{\isacharparenleft}x{\isacharcomma}y{\isacharparenright}{\isacharbraceright}{\isasyminverse}\ {\isacharequal}\ {\isacharbraceleft}{\isacharparenleft}y{\isacharcomma}x{\isacharparenright}{\isacharbraceright}{\isachardoublequoteclose}\ \isacommand{by}\isamarkupfalse%
\ fastforce\isanewline
\ \ \isacommand{then}\isamarkupfalse%
\ \isacommand{have}\isamarkupfalse%
\ {\isachardoublequoteopen}runiq\ {\isacharparenleft}{\isacharbraceleft}{\isacharparenleft}x{\isacharcomma}y{\isacharparenright}{\isacharbraceright}{\isasyminverse}{\isacharparenright}{\isachardoublequoteclose}\ \isacommand{using}\isamarkupfalse%
\ runiq{\isacharunderscore}singleton{\isacharunderscore}rel\ \isacommand{by}\isamarkupfalse%
\ metis\isanewline
\ \ \isacommand{moreover}\isamarkupfalse%
\ \isacommand{have}\isamarkupfalse%
\ {\isachardoublequoteopen}Domain\ R\ {\isasyminter}\ Domain\ {\isacharbraceleft}{\isacharparenleft}x{\isacharcomma}y{\isacharparenright}{\isacharbraceright}\ {\isacharequal}\ {\isacharbraceleft}{\isacharbraceright}{\isachardoublequoteclose}\ \isakeyword{and}\ {\isachardoublequoteopen}Range\ R\ {\isasyminter}\ {\isacharparenleft}Range\ {\isacharbraceleft}{\isacharparenleft}x{\isacharcomma}y{\isacharparenright}{\isacharbraceright}{\isacharparenright}{\isacharequal}{\isacharbraceleft}{\isacharbraceright}{\isachardoublequoteclose}\isanewline
\ \ \ \ \isacommand{using}\isamarkupfalse%
\ y{\isacharunderscore}notin{\isacharunderscore}R\ x{\isacharunderscore}notin{\isacharunderscore}D\ \isacommand{by}\isamarkupfalse%
\ simp{\isacharunderscore}all\isanewline
\ \ \isacommand{ultimately}\isamarkupfalse%
\ \isacommand{show}\isamarkupfalse%
\ {\isacharquery}thesis\ \isacommand{using}\isamarkupfalse%
\ runiq\ runiq{\isacharunderscore}converse{\isacharunderscore}paste\ \isacommand{by}\isamarkupfalse%
\ blast\isanewline
\isacommand{qed}\isamarkupfalse%
%
\endisatagproof
{\isafoldproof}%
%
\isadelimproof
%
\endisadelimproof
%
\begin{isamarkuptext}%
If a relation is known to be right-unique, it is easier to know when we can evaluate it
  like a function, using \isa{eval{\isacharunderscore}rel{\isacharunderscore}or}.%
\end{isamarkuptext}%
\isamarkuptrue%
\isacommand{lemma}\isamarkupfalse%
\ eval{\isacharunderscore}runiq{\isacharunderscore}rel{\isacharunderscore}or{\isacharcolon}\isanewline
\ \ \isakeyword{assumes}\ {\isachardoublequoteopen}runiq\ R{\isachardoublequoteclose}\isanewline
\ \ \isakeyword{shows}\ {\isachardoublequoteopen}eval{\isacharunderscore}rel{\isacharunderscore}or\ R\ a\ z\ {\isacharequal}\ {\isacharparenleft}if\ a\ {\isasymin}\ Domain\ R\ then\ the{\isacharunderscore}elem\ {\isacharparenleft}R\ {\isacharbackquote}{\isacharbackquote}\ {\isacharbraceleft}a{\isacharbraceright}{\isacharparenright}\ else\ z{\isacharparenright}{\isachardoublequoteclose}\isanewline
%
\isadelimproof
%
\endisadelimproof
%
\isatagproof
\isacommand{proof}\isamarkupfalse%
\ {\isacharminus}\isanewline
\ \ \isacommand{from}\isamarkupfalse%
\ assms\ \isacommand{have}\isamarkupfalse%
\ {\isachardoublequoteopen}card\ {\isacharparenleft}R\ {\isacharbackquote}{\isacharbackquote}\ {\isacharbraceleft}a{\isacharbraceright}{\isacharparenright}\ {\isacharequal}\ {\isadigit{1}}\ {\isasymlongleftrightarrow}\ a\ {\isasymin}\ Domain\ R{\isachardoublequoteclose}\isanewline
\ \ \ \ \isanewline
\ \ \ \ \isacommand{using}\isamarkupfalse%
\ Image{\isacharunderscore}within{\isacharunderscore}runiq{\isacharunderscore}domain\ card{\isacharunderscore}Suc{\isacharunderscore}eq\ card{\isacharunderscore}empty\ ex{\isacharunderscore}in{\isacharunderscore}conv\ One{\isacharunderscore}nat{\isacharunderscore}def\ \isacommand{by}\isamarkupfalse%
\ metis\isanewline
\ \ \isacommand{then}\isamarkupfalse%
\ \isacommand{show}\isamarkupfalse%
\ {\isacharquery}thesis\ \isacommand{by}\isamarkupfalse%
\ force\isanewline
\isacommand{qed}\isamarkupfalse%
%
\endisatagproof
{\isafoldproof}%
%
\isadelimproof
%
\endisadelimproof
%
\isamarkupsection{injectivity%
}
\isamarkuptrue%
%
\begin{isamarkuptext}%
A relation \isa{R} is injective on its domain iff any two domain elements having the same image
  are equal.  This definition on its own is of limited utility, as it does not assume that \isa{R}
  is a function, i.e.\ right-unique.%
\end{isamarkuptext}%
\isamarkuptrue%
\isacommand{definition}\isamarkupfalse%
\ injective\ {\isacharcolon}{\isacharcolon}\ {\isachardoublequoteopen}{\isacharparenleft}{\isacharprime}a\ {\isasymtimes}\ {\isacharprime}b{\isacharparenright}\ set\ {\isasymRightarrow}\ bool{\isachardoublequoteclose}\isanewline
\isakeyword{where}\ {\isachardoublequoteopen}injective\ R\ {\isasymlongleftrightarrow}\ {\isacharparenleft}{\isasymforall}\ a\ {\isasymin}\ Domain\ R\ {\isachardot}\ {\isasymforall}\ b\ {\isasymin}\ Domain\ R\ {\isachardot}\ R\ {\isacharbackquote}{\isacharbackquote}\ {\isacharbraceleft}a{\isacharbraceright}\ {\isacharequal}\ R\ {\isacharbackquote}{\isacharbackquote}\ {\isacharbraceleft}b{\isacharbraceright}\ {\isasymlongrightarrow}\ a\ {\isacharequal}\ b{\isacharparenright}{\isachardoublequoteclose}%
\begin{isamarkuptext}%
If both a relation and its converse are right-unique, it is injective on its domain.%
\end{isamarkuptext}%
\isamarkuptrue%
\isacommand{lemma}\isamarkupfalse%
\ runiq{\isacharunderscore}and{\isacharunderscore}conv{\isacharunderscore}imp{\isacharunderscore}injective{\isacharcolon}\ \isanewline
\ \ \isakeyword{assumes}\ runiq{\isacharcolon}\ {\isachardoublequoteopen}runiq\ R{\isachardoublequoteclose}\isanewline
\ \ \ \ \ \ \isakeyword{and}\ runiq{\isacharunderscore}conv{\isacharcolon}\ {\isachardoublequoteopen}runiq\ {\isacharparenleft}R\ {\isasyminverse}{\isacharparenright}{\isachardoublequoteclose}\isanewline
\ \ \isakeyword{shows}\ {\isachardoublequoteopen}injective\ R{\isachardoublequoteclose}\isanewline
%
\isadelimproof
%
\endisadelimproof
%
\isatagproof
\isacommand{proof}\isamarkupfalse%
\ {\isacharminus}\isanewline
\ \ \isacommand{{\isacharbraceleft}}\isamarkupfalse%
\isanewline
\ \ \ \ \isacommand{fix}\isamarkupfalse%
\ a\ \isacommand{assume}\isamarkupfalse%
\ a{\isacharunderscore}Dom{\isacharcolon}\ {\isachardoublequoteopen}a\ {\isasymin}\ Domain\ R{\isachardoublequoteclose}\isanewline
\ \ \ \ \isacommand{fix}\isamarkupfalse%
\ b\ \isacommand{assume}\isamarkupfalse%
\ b{\isacharunderscore}Dom{\isacharcolon}\ {\isachardoublequoteopen}b\ {\isasymin}\ Domain\ R{\isachardoublequoteclose}\isanewline
\ \ \ \ \isacommand{have}\isamarkupfalse%
\ {\isachardoublequoteopen}R\ {\isacharbackquote}{\isacharbackquote}\ {\isacharbraceleft}a{\isacharbraceright}\ {\isacharequal}\ R\ {\isacharbackquote}{\isacharbackquote}\ {\isacharbraceleft}b{\isacharbraceright}\ {\isasymlongrightarrow}\ a\ {\isacharequal}\ b{\isachardoublequoteclose}\isanewline
\ \ \ \ \isacommand{proof}\isamarkupfalse%
\isanewline
\ \ \ \ \ \ \isacommand{assume}\isamarkupfalse%
\ eq{\isacharunderscore}Im{\isacharcolon}\ {\isachardoublequoteopen}R\ {\isacharbackquote}{\isacharbackquote}\ {\isacharbraceleft}a{\isacharbraceright}\ {\isacharequal}\ R\ {\isacharbackquote}{\isacharbackquote}\ {\isacharbraceleft}b{\isacharbraceright}{\isachardoublequoteclose}\isanewline
\ \ \ \ \ \ \isacommand{from}\isamarkupfalse%
\ runiq\ a{\isacharunderscore}Dom\ \isacommand{obtain}\isamarkupfalse%
\ Ra\ \isakeyword{where}\ Ra{\isacharcolon}\ {\isachardoublequoteopen}R\ {\isacharbackquote}{\isacharbackquote}\ {\isacharbraceleft}a{\isacharbraceright}\ {\isacharequal}\ {\isacharbraceleft}Ra{\isacharbraceright}{\isachardoublequoteclose}\ \isacommand{by}\isamarkupfalse%
\ {\isacharparenleft}metis\ Image{\isacharunderscore}runiq{\isacharunderscore}eq{\isacharunderscore}eval{\isacharparenright}\isanewline
\ \ \ \ \ \ \isacommand{from}\isamarkupfalse%
\ runiq\ b{\isacharunderscore}Dom\ \isacommand{obtain}\isamarkupfalse%
\ Rb\ \isakeyword{where}\ Rb{\isacharcolon}\ {\isachardoublequoteopen}R\ {\isacharbackquote}{\isacharbackquote}\ {\isacharbraceleft}b{\isacharbraceright}\ {\isacharequal}\ {\isacharbraceleft}Rb{\isacharbraceright}{\isachardoublequoteclose}\ \isacommand{by}\isamarkupfalse%
\ {\isacharparenleft}metis\ Image{\isacharunderscore}runiq{\isacharunderscore}eq{\isacharunderscore}eval{\isacharparenright}\isanewline
\ \ \ \ \ \ \isacommand{from}\isamarkupfalse%
\ eq{\isacharunderscore}Im\ Ra\ Rb\ \isacommand{have}\isamarkupfalse%
\ eq{\isacharunderscore}Im{\isacharprime}{\isacharcolon}\ {\isachardoublequoteopen}Ra\ {\isacharequal}\ Rb{\isachardoublequoteclose}\ \isacommand{by}\isamarkupfalse%
\ simp\isanewline
\ \ \ \ \ \ \isacommand{from}\isamarkupfalse%
\ eq{\isacharunderscore}Im{\isacharprime}\ Ra\ a{\isacharunderscore}Dom\ runiq{\isacharunderscore}conv\ \isacommand{have}\isamarkupfalse%
\ a{\isacharprime}{\isacharcolon}\ {\isachardoublequoteopen}{\isacharparenleft}R\ {\isasyminverse}{\isacharparenright}\ {\isacharbackquote}{\isacharbackquote}\ {\isacharbraceleft}Ra{\isacharbraceright}\ {\isacharequal}\ {\isacharbraceleft}a{\isacharbraceright}{\isachardoublequoteclose}\isanewline
\ \ \ \ \ \ \ \ \isacommand{using}\isamarkupfalse%
\ converse{\isacharunderscore}Image{\isacharunderscore}singleton{\isacharunderscore}Domain\ runiq\ \isacommand{by}\isamarkupfalse%
\ metis\isanewline
\ \ \ \ \ \ \isacommand{from}\isamarkupfalse%
\ eq{\isacharunderscore}Im{\isacharprime}\ Rb\ b{\isacharunderscore}Dom\ runiq{\isacharunderscore}conv\ \isacommand{have}\isamarkupfalse%
\ b{\isacharprime}{\isacharcolon}\ {\isachardoublequoteopen}{\isacharparenleft}R\ {\isasyminverse}{\isacharparenright}\ {\isacharbackquote}{\isacharbackquote}\ {\isacharbraceleft}Rb{\isacharbraceright}\ {\isacharequal}\ {\isacharbraceleft}b{\isacharbraceright}{\isachardoublequoteclose}\isanewline
\ \ \ \ \ \ \ \ \isacommand{using}\isamarkupfalse%
\ converse{\isacharunderscore}Image{\isacharunderscore}singleton{\isacharunderscore}Domain\ runiq\ \isacommand{by}\isamarkupfalse%
\ metis\isanewline
\ \ \ \ \ \ \isacommand{from}\isamarkupfalse%
\ eq{\isacharunderscore}Im{\isacharprime}\ a{\isacharprime}\ b{\isacharprime}\ \isacommand{show}\isamarkupfalse%
\ {\isachardoublequoteopen}a\ {\isacharequal}\ b{\isachardoublequoteclose}\ \isacommand{by}\isamarkupfalse%
\ simp\isanewline
\ \ \ \ \isacommand{qed}\isamarkupfalse%
\isanewline
\ \ \isacommand{{\isacharbraceright}}\isamarkupfalse%
\isanewline
\ \ \isacommand{then}\isamarkupfalse%
\ \isacommand{show}\isamarkupfalse%
\ {\isacharquery}thesis\ \isacommand{unfolding}\isamarkupfalse%
\ injective{\isacharunderscore}def\ \isacommand{by}\isamarkupfalse%
\ blast\isanewline
\isacommand{qed}\isamarkupfalse%
%
\endisatagproof
{\isafoldproof}%
%
\isadelimproof
%
\endisadelimproof
%
\begin{isamarkuptext}%
the set of all injective functions from \isa{X} to \isa{Y}.%
\end{isamarkuptext}%
\isamarkuptrue%
\isacommand{definition}\isamarkupfalse%
\ injections\ {\isacharcolon}{\isacharcolon}\ {\isachardoublequoteopen}{\isacharprime}a\ set\ {\isasymRightarrow}\ {\isacharprime}b\ set\ {\isasymRightarrow}\ {\isacharparenleft}{\isacharprime}a\ {\isasymtimes}\ {\isacharprime}b{\isacharparenright}\ set\ set{\isachardoublequoteclose}\isanewline
\isakeyword{where}\ {\isachardoublequoteopen}injections\ X\ Y\ {\isacharequal}\ {\isacharbraceleft}R\ {\isachardot}\ Domain\ R\ {\isacharequal}\ X\ {\isasymand}\ Range\ R\ {\isasymsubseteq}\ Y\ {\isasymand}\ runiq\ R\ {\isasymand}\ runiq\ {\isacharparenleft}R{\isasyminverse}{\isacharparenright}{\isacharbraceright}{\isachardoublequoteclose}%
\begin{isamarkuptext}%
introduction rule that establishes the injectivity of a relation%
\end{isamarkuptext}%
\isamarkuptrue%
\isacommand{lemma}\isamarkupfalse%
\ injectionsI{\isacharcolon}\isanewline
\ \ \isakeyword{fixes}\ R{\isacharcolon}{\isacharcolon}{\isachardoublequoteopen}{\isacharparenleft}{\isacharprime}a\ {\isasymtimes}\ {\isacharprime}b{\isacharparenright}\ set{\isachardoublequoteclose}\isanewline
\ \ \isakeyword{assumes}\ {\isachardoublequoteopen}Domain\ R\ {\isacharequal}\ X{\isachardoublequoteclose}\isanewline
\ \ \ \ \ \ \isakeyword{and}\ {\isachardoublequoteopen}Range\ R\ {\isasymsubseteq}\ Y{\isachardoublequoteclose}\isanewline
\ \ \ \ \ \ \isakeyword{and}\ {\isachardoublequoteopen}runiq\ R{\isachardoublequoteclose}\isanewline
\ \ \ \ \ \ \isakeyword{and}\ {\isachardoublequoteopen}runiq\ {\isacharparenleft}R{\isasyminverse}{\isacharparenright}{\isachardoublequoteclose}\isanewline
\ \ \isakeyword{shows}\ {\isachardoublequoteopen}R\ {\isasymin}\ injections\ X\ Y{\isachardoublequoteclose}\isanewline
%
\isadelimproof
%
\endisadelimproof
%
\isatagproof
\isacommand{using}\isamarkupfalse%
\ assms\ \isacommand{unfolding}\isamarkupfalse%
\ injections{\isacharunderscore}def\ \isacommand{using}\isamarkupfalse%
\ CollectI\ \isacommand{by}\isamarkupfalse%
\ blast%
\endisatagproof
{\isafoldproof}%
%
\isadelimproof
%
\endisadelimproof
%
\begin{isamarkuptext}%
the set of all injective partial functions (including total ones) from \isa{X} to \isa{Y}.%
\end{isamarkuptext}%
\isamarkuptrue%
\isacommand{definition}\isamarkupfalse%
\ partial{\isacharunderscore}injections\ {\isacharcolon}{\isacharcolon}\ {\isachardoublequoteopen}{\isacharprime}a\ set\ {\isasymRightarrow}\ {\isacharprime}b\ set\ {\isasymRightarrow}\ {\isacharparenleft}{\isacharprime}a\ {\isasymtimes}\ {\isacharprime}b{\isacharparenright}\ set\ set{\isachardoublequoteclose}\isanewline
\isakeyword{where}\ {\isachardoublequoteopen}partial{\isacharunderscore}injections\ X\ Y\ {\isacharequal}\ {\isacharbraceleft}R\ {\isachardot}\ Domain\ R\ {\isasymsubseteq}\ X\ {\isasymand}\ Range\ R\ {\isasymsubseteq}\ Y\ {\isasymand}\ runiq\ R\ {\isasymand}\ runiq\ {\isacharparenleft}R{\isasyminverse}{\isacharparenright}{\isacharbraceright}{\isachardoublequoteclose}%
\begin{isamarkuptext}%
Given a relation \isa{R}, an element \isa{x} of the relation's domain type and
  a set \isa{Y} of the relation's range type, this function constructs the list of all 
  superrelations of \isa{R} that extend \isa{R} by a pair \isa{{\isacharparenleft}x{\isacharcomma}\ y{\isacharparenright}} for some
  \isa{y} not yet covered by \isa{R}.%
\end{isamarkuptext}%
\isamarkuptrue%
\isacommand{fun}\isamarkupfalse%
\ sup{\isacharunderscore}rels{\isacharunderscore}from{\isacharunderscore}alg\ {\isacharcolon}{\isacharcolon}\ {\isachardoublequoteopen}{\isacharparenleft}{\isacharprime}a\ {\isasymtimes}\ {\isacharprime}b{\isasymColon}linorder{\isacharparenright}\ set\ {\isasymRightarrow}\ {\isacharprime}a\ {\isasymRightarrow}\ {\isacharprime}b\ set\ {\isasymRightarrow}\ {\isacharparenleft}{\isacharprime}a\ {\isasymtimes}\ {\isacharprime}b{\isacharparenright}\ set\ list{\isachardoublequoteclose}\isanewline
\isakeyword{where}\ \isanewline
{\isachardoublequoteopen}sup{\isacharunderscore}rels{\isacharunderscore}from{\isacharunderscore}alg\ R\ x\ Y\ {\isacharequal}\ {\isacharbrackleft}\ R\ {\isacharplus}{\isacharasterisk}\ {\isacharbraceleft}{\isacharparenleft}x{\isacharcomma}y{\isacharparenright}{\isacharbraceright}\ {\isachardot}\ y\ {\isasymleftarrow}\ sorted{\isacharunderscore}list{\isacharunderscore}of{\isacharunderscore}set\ {\isacharparenleft}Y\ {\isacharminus}\ Range\ R{\isacharparenright}\ {\isacharbrackright}{\isachardoublequoteclose}%
\begin{isamarkuptext}%
set-based variant of \isa{sup{\isacharunderscore}rels{\isacharunderscore}from{\isacharunderscore}alg}%
\end{isamarkuptext}%
\isamarkuptrue%
\isacommand{definition}\isamarkupfalse%
\ sup{\isacharunderscore}rels{\isacharunderscore}from\ {\isacharcolon}{\isacharcolon}\ {\isachardoublequoteopen}{\isacharparenleft}{\isacharprime}a\ {\isasymtimes}\ {\isacharprime}b{\isacharparenright}\ set\ {\isasymRightarrow}\ {\isacharprime}a\ {\isasymRightarrow}\ {\isacharprime}b\ set\ {\isasymRightarrow}\ {\isacharparenleft}{\isacharprime}a\ {\isasymtimes}\ {\isacharprime}b{\isacharparenright}\ set\ set{\isachardoublequoteclose}\isanewline
\isakeyword{where}\ {\isachardoublequoteopen}sup{\isacharunderscore}rels{\isacharunderscore}from\ R\ x\ Y\ {\isacharequal}\ {\isacharbraceleft}\ R\ {\isacharplus}{\isacharasterisk}\ {\isacharbraceleft}{\isacharparenleft}x{\isacharcomma}\ y{\isacharparenright}{\isacharbraceright}\ {\isacharbar}\ y\ {\isachardot}\ y\ {\isasymin}\ Y\ {\isacharminus}\ Range\ R\ {\isacharbraceright}{\isachardoublequoteclose}%
\begin{isamarkuptext}%
On finite sets, \isa{sup{\isacharunderscore}rels{\isacharunderscore}from{\isacharunderscore}alg} and \isa{sup{\isacharunderscore}rels{\isacharunderscore}from} are equivalent.%
\end{isamarkuptext}%
\isamarkuptrue%
\isacommand{lemma}\isamarkupfalse%
\ sup{\isacharunderscore}rels{\isacharunderscore}from{\isacharunderscore}paper{\isacharunderscore}equiv{\isacharunderscore}alg{\isacharcolon}\isanewline
\ \ \isakeyword{assumes}\ {\isachardoublequoteopen}finite\ Y{\isachardoublequoteclose}\isanewline
\ \ \isakeyword{shows}\ {\isachardoublequoteopen}set\ {\isacharparenleft}sup{\isacharunderscore}rels{\isacharunderscore}from{\isacharunderscore}alg\ R\ x\ Y{\isacharparenright}\ {\isacharequal}\ sup{\isacharunderscore}rels{\isacharunderscore}from\ R\ x\ Y{\isachardoublequoteclose}\isanewline
%
\isadelimproof
%
\endisadelimproof
%
\isatagproof
\isacommand{proof}\isamarkupfalse%
\ {\isacharminus}\isanewline
\ \ \isacommand{have}\isamarkupfalse%
\ {\isachardoublequoteopen}distinct\ {\isacharparenleft}sorted{\isacharunderscore}list{\isacharunderscore}of{\isacharunderscore}set\ {\isacharparenleft}Y\ {\isacharminus}\ Range\ R{\isacharparenright}{\isacharparenright}{\isachardoublequoteclose}\ \isacommand{using}\isamarkupfalse%
\ assms\ \isacommand{by}\isamarkupfalse%
\ simp\isanewline
\ \ \isacommand{then}\isamarkupfalse%
\ \isacommand{have}\isamarkupfalse%
\ {\isachardoublequoteopen}set\ {\isacharbrackleft}\ R\ {\isacharplus}{\isacharasterisk}\ {\isacharbraceleft}{\isacharparenleft}x{\isacharcomma}y{\isacharparenright}{\isacharbraceright}\ {\isachardot}\ y\ {\isasymleftarrow}\ sorted{\isacharunderscore}list{\isacharunderscore}of{\isacharunderscore}set\ {\isacharparenleft}Y\ {\isacharminus}\ Range\ R{\isacharparenright}\ {\isacharbrackright}\ {\isacharequal}\ {\isacharbraceleft}\ R\ {\isacharplus}{\isacharasterisk}\ {\isacharbraceleft}{\isacharparenleft}x{\isacharcomma}y{\isacharparenright}{\isacharbraceright}\ {\isacharbar}\ y\ {\isachardot}\ y\ {\isasymin}\ set\ {\isacharparenleft}sorted{\isacharunderscore}list{\isacharunderscore}of{\isacharunderscore}set\ {\isacharparenleft}Y\ {\isacharminus}\ Range\ R{\isacharparenright}{\isacharparenright}\ {\isacharbraceright}{\isachardoublequoteclose}\ \ \ \ \isacommand{by}\isamarkupfalse%
\ auto\isanewline
\ \ \isacommand{moreover}\isamarkupfalse%
\ \isacommand{have}\isamarkupfalse%
\ {\isachardoublequoteopen}set\ {\isacharparenleft}sorted{\isacharunderscore}list{\isacharunderscore}of{\isacharunderscore}set\ {\isacharparenleft}Y\ {\isacharminus}\ Range\ R{\isacharparenright}{\isacharparenright}\ {\isacharequal}\ Y\ {\isacharminus}\ Range\ R{\isachardoublequoteclose}\ \isacommand{using}\isamarkupfalse%
\ assms\ \isacommand{by}\isamarkupfalse%
\ simp\isanewline
\ \ \isacommand{ultimately}\isamarkupfalse%
\ \isacommand{show}\isamarkupfalse%
\ {\isacharquery}thesis\ \isacommand{unfolding}\isamarkupfalse%
\ sup{\isacharunderscore}rels{\isacharunderscore}from{\isacharunderscore}def\ \isacommand{by}\isamarkupfalse%
\ simp\isanewline
\isacommand{qed}\isamarkupfalse%
%
\endisatagproof
{\isafoldproof}%
%
\isadelimproof
%
\endisadelimproof
%
\begin{isamarkuptext}%
the list of all injective functions (represented as relations) from one set 
  (represented as a list) to another set%
\end{isamarkuptext}%
\isamarkuptrue%
\isacommand{fun}\isamarkupfalse%
\ injections{\isacharunderscore}alg\ {\isacharcolon}{\isacharcolon}\ {\isachardoublequoteopen}{\isacharprime}a\ list\ {\isasymRightarrow}\ {\isacharprime}b{\isasymColon}linorder\ set\ {\isasymRightarrow}\ {\isacharparenleft}{\isacharprime}a\ {\isasymtimes}\ {\isacharprime}b{\isacharparenright}\ set\ list{\isachardoublequoteclose}\isanewline
\isakeyword{where}\ {\isachardoublequoteopen}injections{\isacharunderscore}alg\ {\isacharbrackleft}{\isacharbrackright}\ Y\ {\isacharequal}\ {\isacharbrackleft}{\isacharbraceleft}{\isacharbraceright}{\isacharbrackright}{\isachardoublequoteclose}\ {\isacharbar}\isanewline
\ \ \ \ \ \ {\isachardoublequoteopen}injections{\isacharunderscore}alg\ {\isacharparenleft}x\ {\isacharhash}\ xs{\isacharparenright}\ Y\ {\isacharequal}\ concat\ {\isacharbrackleft}\ {\isacharbrackleft}\ R\ {\isacharplus}{\isacharasterisk}\ {\isacharbraceleft}{\isacharparenleft}x{\isacharcomma}y{\isacharparenright}{\isacharbraceright}\ {\isachardot}\ y\ {\isasymleftarrow}\ sorted{\isacharunderscore}list{\isacharunderscore}of{\isacharunderscore}set\ {\isacharparenleft}Y\ {\isacharminus}\ Range\ R{\isacharparenright}\ {\isacharbrackright}\isanewline
\ \ \ \ \ \ {\isachardot}\ R\ {\isasymleftarrow}\ injections{\isacharunderscore}alg\ xs\ Y\ {\isacharbrackright}{\isachardoublequoteclose}%
\begin{isamarkuptext}%
the set-theoretic variant of the recursive rule of \isa{injections{\isacharunderscore}alg}%
\end{isamarkuptext}%
\isamarkuptrue%
\isacommand{lemma}\isamarkupfalse%
\ injections{\isacharunderscore}paste{\isacharcolon}\isanewline
\ \ \isakeyword{assumes}\ new{\isacharcolon}\ {\isachardoublequoteopen}x\ {\isasymnotin}\ A{\isachardoublequoteclose}\isanewline
\ \ \isakeyword{shows}\ {\isachardoublequoteopen}injections\ {\isacharparenleft}insert\ x\ A{\isacharparenright}\ Y\ {\isacharequal}\ {\isacharparenleft}{\isasymUnion}\ {\isacharbraceleft}\ sup{\isacharunderscore}rels{\isacharunderscore}from\ P\ x\ Y\ {\isacharbar}\ P\ {\isachardot}\ P\ {\isasymin}\ injections\ A\ Y\ {\isacharbraceright}{\isacharparenright}{\isachardoublequoteclose}\isanewline
%
\isadelimproof
%
\endisadelimproof
%
\isatagproof
\isacommand{proof}\isamarkupfalse%
\ {\isacharparenleft}rule\ equalitySubsetI{\isacharparenright}\isanewline
\ \ \isacommand{fix}\isamarkupfalse%
\ R\isanewline
\ \ \isacommand{assume}\isamarkupfalse%
\ {\isachardoublequoteopen}R\ {\isasymin}\ injections\ {\isacharparenleft}insert\ x\ A{\isacharparenright}\ Y{\isachardoublequoteclose}\isanewline
\ \ \isacommand{then}\isamarkupfalse%
\ \isacommand{have}\isamarkupfalse%
\ injections{\isacharunderscore}unfolded{\isacharcolon}\ {\isachardoublequoteopen}Domain\ R\ {\isacharequal}\ insert\ x\ A\ {\isasymand}\ Range\ R\ {\isasymsubseteq}\ Y\ {\isasymand}\ runiq\ R\ {\isasymand}\ runiq\ {\isacharparenleft}R{\isasyminverse}{\isacharparenright}{\isachardoublequoteclose}\isanewline
\ \ \ \ \isacommand{unfolding}\isamarkupfalse%
\ injections{\isacharunderscore}def\ \isacommand{by}\isamarkupfalse%
\ simp\isanewline
\ \ \isacommand{then}\isamarkupfalse%
\ \isacommand{have}\isamarkupfalse%
\ Domain{\isacharcolon}\ {\isachardoublequoteopen}Domain\ R\ {\isacharequal}\ insert\ x\ A{\isachardoublequoteclose}\isanewline
\ \ \ \ \ \ \ \ \isakeyword{and}\ Range{\isacharcolon}\ {\isachardoublequoteopen}Range\ R\ {\isasymsubseteq}\ Y{\isachardoublequoteclose}\isanewline
\ \ \ \ \ \ \ \ \isakeyword{and}\ runiq{\isacharcolon}\ {\isachardoublequoteopen}runiq\ R{\isachardoublequoteclose}\isanewline
\ \ \ \ \ \ \ \ \isakeyword{and}\ runiq{\isacharunderscore}conv{\isacharcolon}\ {\isachardoublequoteopen}runiq\ {\isacharparenleft}R{\isasyminverse}{\isacharparenright}{\isachardoublequoteclose}\ \isacommand{by}\isamarkupfalse%
\ simp{\isacharunderscore}all\isanewline
\isanewline
\ \ \isacommand{let}\isamarkupfalse%
\ {\isacharquery}P\ {\isacharequal}\ {\isachardoublequoteopen}R\ outside\ {\isacharbraceleft}x{\isacharbraceright}{\isachardoublequoteclose}\isanewline
\ \ \isacommand{have}\isamarkupfalse%
\ subrel{\isacharcolon}\ {\isachardoublequoteopen}{\isacharquery}P\ {\isasymsubseteq}\ R{\isachardoublequoteclose}\ \isacommand{unfolding}\isamarkupfalse%
\ Outside{\isacharunderscore}def\ \isacommand{by}\isamarkupfalse%
\ fast\isanewline
\ \ \isacommand{have}\isamarkupfalse%
\ subrel{\isacharunderscore}conv{\isacharcolon}\ {\isachardoublequoteopen}{\isacharquery}P{\isasyminverse}\ {\isasymsubseteq}\ R{\isasyminverse}{\isachardoublequoteclose}\ \isacommand{using}\isamarkupfalse%
\ subrel\ \isacommand{by}\isamarkupfalse%
\ blast\isanewline
\isanewline
\ \ \isanewline
\ \ \isacommand{from}\isamarkupfalse%
\ Domain\ new\ \isacommand{have}\isamarkupfalse%
\ Domain{\isacharunderscore}pre{\isacharcolon}\ {\isachardoublequoteopen}Domain\ {\isacharquery}P\ {\isacharequal}\ A{\isachardoublequoteclose}\ \isacommand{by}\isamarkupfalse%
\ {\isacharparenleft}rule\ Domain{\isacharunderscore}outside{\isacharunderscore}singleton{\isacharparenright}\isanewline
\ \ \isacommand{have}\isamarkupfalse%
\ P{\isacharunderscore}inj{\isacharcolon}\ {\isachardoublequoteopen}{\isacharquery}P\ {\isasymin}\ injections\ A\ Y{\isachardoublequoteclose}\isanewline
\ \ \isacommand{proof}\isamarkupfalse%
\ {\isacharparenleft}rule\ injectionsI{\isacharparenright}\isanewline
\ \ \ \ \isacommand{show}\isamarkupfalse%
\ {\isachardoublequoteopen}Domain\ {\isacharquery}P\ {\isacharequal}\ A{\isachardoublequoteclose}\ \isacommand{by}\isamarkupfalse%
\ {\isacharparenleft}rule\ Domain{\isacharunderscore}pre{\isacharparenright}\isanewline
\ \ \ \ \isacommand{show}\isamarkupfalse%
\ {\isachardoublequoteopen}Range\ {\isacharquery}P\ {\isasymsubseteq}\ Y{\isachardoublequoteclose}\ \isacommand{using}\isamarkupfalse%
\ Range\ \isacommand{by}\isamarkupfalse%
\ {\isacharparenleft}rule\ Range{\isacharunderscore}outside{\isacharunderscore}sub{\isacharparenright}\isanewline
\ \ \ \ \isacommand{show}\isamarkupfalse%
\ {\isachardoublequoteopen}runiq\ {\isacharquery}P{\isachardoublequoteclose}\ \isacommand{using}\isamarkupfalse%
\ runiq\ subrel\ \isacommand{by}\isamarkupfalse%
\ {\isacharparenleft}rule\ subrel{\isacharunderscore}runiq{\isacharparenright}\isanewline
\ \ \ \ \isacommand{show}\isamarkupfalse%
\ {\isachardoublequoteopen}runiq\ {\isacharparenleft}{\isacharquery}P{\isasyminverse}{\isacharparenright}{\isachardoublequoteclose}\ \isacommand{using}\isamarkupfalse%
\ runiq{\isacharunderscore}conv\ subrel{\isacharunderscore}conv\ \isacommand{by}\isamarkupfalse%
\ {\isacharparenleft}rule\ subrel{\isacharunderscore}runiq{\isacharparenright}\isanewline
\ \ \isacommand{qed}\isamarkupfalse%
\isanewline
\isanewline
\ \ \isacommand{obtain}\isamarkupfalse%
\ y\ \isakeyword{where}\ y{\isacharcolon}\ {\isachardoublequoteopen}R\ {\isacharbackquote}{\isacharbackquote}\ {\isacharbraceleft}x{\isacharbraceright}\ {\isacharequal}\ {\isacharbraceleft}y{\isacharbraceright}{\isachardoublequoteclose}\ \isacommand{using}\isamarkupfalse%
\ Image{\isacharunderscore}runiq{\isacharunderscore}eq{\isacharunderscore}eval\ Domain\ runiq\ \isacommand{by}\isamarkupfalse%
\ {\isacharparenleft}metis\ insertI{\isadigit{1}}{\isacharparenright}\isanewline
\ \ \isacommand{from}\isamarkupfalse%
\ y\ Range\ \isacommand{have}\isamarkupfalse%
\ {\isachardoublequoteopen}y\ {\isasymin}\ Y{\isachardoublequoteclose}\ \isacommand{by}\isamarkupfalse%
\ fast\isanewline
\ \ \isacommand{moreover}\isamarkupfalse%
\ \isacommand{have}\isamarkupfalse%
\ {\isachardoublequoteopen}y\ {\isasymnotin}\ Range\ {\isacharquery}P{\isachardoublequoteclose}\isanewline
\ \ \isacommand{proof}\isamarkupfalse%
\isanewline
\ \ \ \ \isacommand{assume}\isamarkupfalse%
\ assm{\isacharcolon}\ {\isachardoublequoteopen}y\ {\isasymin}\ Range\ {\isacharquery}P{\isachardoublequoteclose}\isanewline
\ \ \ \ \isacommand{then}\isamarkupfalse%
\ \isacommand{obtain}\isamarkupfalse%
\ x{\isacharprime}\ \isakeyword{where}\ x{\isacharprime}{\isacharunderscore}Domain{\isacharcolon}\ {\isachardoublequoteopen}x{\isacharprime}\ {\isasymin}\ Domain\ {\isacharquery}P{\isachardoublequoteclose}\ \isakeyword{and}\ x{\isacharprime}{\isacharunderscore}P{\isacharunderscore}y{\isacharcolon}\ {\isachardoublequoteopen}{\isacharparenleft}x{\isacharprime}{\isacharcomma}\ y{\isacharparenright}\ {\isasymin}\ {\isacharquery}P{\isachardoublequoteclose}\ \isacommand{by}\isamarkupfalse%
\ fast\isanewline
\ \ \ \ \isacommand{have}\isamarkupfalse%
\ x{\isacharprime}{\isacharunderscore}img{\isacharcolon}\ {\isachardoublequoteopen}x{\isacharprime}\ {\isasymin}\ R{\isasyminverse}\ {\isacharbackquote}{\isacharbackquote}\ {\isacharbraceleft}y{\isacharbraceright}{\isachardoublequoteclose}\ \isacommand{using}\isamarkupfalse%
\ subrel\ x{\isacharprime}{\isacharunderscore}P{\isacharunderscore}y\ \isacommand{by}\isamarkupfalse%
\ fast\isanewline
\ \ \ \ \isacommand{have}\isamarkupfalse%
\ x{\isacharunderscore}img{\isacharcolon}\ {\isachardoublequoteopen}x\ {\isasymin}\ R{\isasyminverse}\ {\isacharbackquote}{\isacharbackquote}\ {\isacharbraceleft}y{\isacharbraceright}{\isachardoublequoteclose}\ \isacommand{using}\isamarkupfalse%
\ y\ \isacommand{by}\isamarkupfalse%
\ fast\isanewline
\ \ \ \ \isacommand{have}\isamarkupfalse%
\ {\isachardoublequoteopen}x{\isacharprime}\ {\isasymnoteq}\ x{\isachardoublequoteclose}\isanewline
\ \ \ \ \isacommand{proof}\isamarkupfalse%
\ {\isacharminus}\isanewline
\ \ \ \ \ \ \isacommand{from}\isamarkupfalse%
\ x{\isacharprime}{\isacharunderscore}Domain\ \isacommand{have}\isamarkupfalse%
\ {\isachardoublequoteopen}x{\isacharprime}\ {\isasymin}\ A{\isachardoublequoteclose}\ \isacommand{using}\isamarkupfalse%
\ Domain{\isacharunderscore}pre\ \isacommand{by}\isamarkupfalse%
\ fast\isanewline
\ \ \ \ \ \ \isacommand{with}\isamarkupfalse%
\ new\ \isacommand{show}\isamarkupfalse%
\ {\isacharquery}thesis\ \isacommand{by}\isamarkupfalse%
\ fast\isanewline
\ \ \ \ \isacommand{qed}\isamarkupfalse%
\isanewline
\ \ \ \ \isacommand{have}\isamarkupfalse%
\ {\isachardoublequoteopen}trivial\ {\isacharparenleft}R{\isasyminverse}\ {\isacharbackquote}{\isacharbackquote}\ {\isacharbraceleft}y{\isacharbraceright}{\isacharparenright}{\isachardoublequoteclose}\ \isacommand{using}\isamarkupfalse%
\ runiq{\isacharunderscore}conv\ \isacommand{by}\isamarkupfalse%
\ {\isacharparenleft}metis\ runiq{\isacharunderscore}alt{\isacharparenright}\isanewline
\ \ \ \ \isacommand{then}\isamarkupfalse%
\ \isacommand{have}\isamarkupfalse%
\ {\isachardoublequoteopen}x{\isacharprime}\ {\isacharequal}\ x{\isachardoublequoteclose}\ \isacommand{using}\isamarkupfalse%
\ x{\isacharprime}{\isacharunderscore}img\ x{\isacharunderscore}img\ \isacommand{by}\isamarkupfalse%
\ {\isacharparenleft}rule\ trivial{\isacharunderscore}imp{\isacharunderscore}no{\isacharunderscore}distinct{\isacharparenright}\isanewline
\ \ \ \ \isacommand{with}\isamarkupfalse%
\ {\isacharbackquoteopen}x{\isacharprime}\ {\isasymnoteq}\ x{\isacharbackquoteclose}\ \isacommand{show}\isamarkupfalse%
\ False\ \isacommand{{\isachardot}{\isachardot}}\isamarkupfalse%
\isanewline
\ \ \isacommand{qed}\isamarkupfalse%
\isanewline
\ \ \isacommand{ultimately}\isamarkupfalse%
\ \isacommand{have}\isamarkupfalse%
\ y{\isacharunderscore}in{\isacharcolon}\ {\isachardoublequoteopen}y\ {\isasymin}\ Y\ {\isacharminus}\ Range\ {\isacharquery}P{\isachardoublequoteclose}\ \isacommand{by}\isamarkupfalse%
\ {\isacharparenleft}rule\ DiffI{\isacharparenright}\isanewline
\isanewline
\ \ \isacommand{from}\isamarkupfalse%
\ y\ \isacommand{have}\isamarkupfalse%
\ x{\isacharunderscore}rel{\isacharcolon}\ {\isachardoublequoteopen}R\ {\isacharbar}{\isacharbar}\ {\isacharbraceleft}x{\isacharbraceright}\ {\isacharequal}\ {\isacharbraceleft}{\isacharparenleft}x{\isacharcomma}\ y{\isacharparenright}{\isacharbraceright}{\isachardoublequoteclose}\ \isacommand{unfolding}\isamarkupfalse%
\ restrict{\isacharunderscore}def\ \isacommand{by}\isamarkupfalse%
\ blast\isanewline
\ \ \isacommand{from}\isamarkupfalse%
\ x{\isacharunderscore}rel\ \isacommand{have}\isamarkupfalse%
\ Dom{\isacharunderscore}restrict{\isacharcolon}\ {\isachardoublequoteopen}Domain\ {\isacharparenleft}R\ {\isacharbar}{\isacharbar}\ {\isacharbraceleft}x{\isacharbraceright}{\isacharparenright}\ {\isacharequal}\ {\isacharbraceleft}x{\isacharbraceright}{\isachardoublequoteclose}\ \isacommand{by}\isamarkupfalse%
\ simp\isanewline
\ \ \isacommand{from}\isamarkupfalse%
\ x{\isacharunderscore}rel\ \isacommand{have}\isamarkupfalse%
\ P{\isacharunderscore}paste{\isacharprime}{\isacharcolon}\ {\isachardoublequoteopen}{\isacharquery}P\ {\isacharplus}{\isacharasterisk}\ {\isacharbraceleft}{\isacharparenleft}x{\isacharcomma}\ y{\isacharparenright}{\isacharbraceright}\ {\isacharequal}\ {\isacharquery}P\ {\isasymunion}\ R\ {\isacharbar}{\isacharbar}\ {\isacharbraceleft}x{\isacharbraceright}{\isachardoublequoteclose}\isanewline
\ \ \ \ \isacommand{using}\isamarkupfalse%
\ outside{\isacharunderscore}union{\isacharunderscore}restrict\ paste{\isacharunderscore}outside{\isacharunderscore}restrict\ \isacommand{by}\isamarkupfalse%
\ metis\isanewline
\ \ \isacommand{from}\isamarkupfalse%
\ Dom{\isacharunderscore}restrict\ Domain{\isacharunderscore}pre\ new\ \isacommand{have}\isamarkupfalse%
\ {\isachardoublequoteopen}Domain\ {\isacharquery}P\ {\isasyminter}\ Domain\ {\isacharparenleft}R\ {\isacharbar}{\isacharbar}\ {\isacharbraceleft}x{\isacharbraceright}{\isacharparenright}\ {\isacharequal}\ {\isacharbraceleft}{\isacharbraceright}{\isachardoublequoteclose}\ \isacommand{by}\isamarkupfalse%
\ simp\isanewline
\ \ \isacommand{then}\isamarkupfalse%
\ \isacommand{have}\isamarkupfalse%
\ {\isachardoublequoteopen}{\isacharquery}P\ {\isacharplus}{\isacharasterisk}\ {\isacharparenleft}R\ {\isacharbar}{\isacharbar}\ {\isacharbraceleft}x{\isacharbraceright}{\isacharparenright}\ {\isacharequal}\ {\isacharquery}P\ {\isasymunion}\ {\isacharparenleft}R\ {\isacharbar}{\isacharbar}\ {\isacharbraceleft}x{\isacharbraceright}{\isacharparenright}{\isachardoublequoteclose}\ \isacommand{by}\isamarkupfalse%
\ {\isacharparenleft}rule\ paste{\isacharunderscore}disj{\isacharunderscore}domains{\isacharparenright}\isanewline
\ \ \isacommand{then}\isamarkupfalse%
\ \isacommand{have}\isamarkupfalse%
\ P{\isacharunderscore}paste{\isacharcolon}\ {\isachardoublequoteopen}{\isacharquery}P\ {\isacharplus}{\isacharasterisk}\ {\isacharbraceleft}{\isacharparenleft}x{\isacharcomma}\ y{\isacharparenright}{\isacharbraceright}\ {\isacharequal}\ R{\isachardoublequoteclose}\ \isacommand{using}\isamarkupfalse%
\ P{\isacharunderscore}paste{\isacharprime}\ outside{\isacharunderscore}union{\isacharunderscore}restrict\ \isacommand{by}\isamarkupfalse%
\ blast\isanewline
\isanewline
\ \ \isacommand{from}\isamarkupfalse%
\ P{\isacharunderscore}inj\ y{\isacharunderscore}in\ P{\isacharunderscore}paste\ \isacommand{have}\isamarkupfalse%
\ {\isachardoublequoteopen}{\isasymexists}\ P\ {\isasymin}\ injections\ A\ Y\ {\isachardot}\ {\isasymexists}\ y\ {\isasymin}\ Y\ {\isacharminus}\ Range\ P\ {\isachardot}\ R\ {\isacharequal}\ P\ {\isacharplus}{\isacharasterisk}\ {\isacharbraceleft}{\isacharparenleft}x{\isacharcomma}\ y{\isacharparenright}{\isacharbraceright}{\isachardoublequoteclose}\ \isacommand{by}\isamarkupfalse%
\ blast\isanewline
\ \ \isanewline
\ \ \isacommand{then}\isamarkupfalse%
\ \isacommand{have}\isamarkupfalse%
\ {\isachardoublequoteopen}{\isasymexists}\ Q\ {\isasymin}\ {\isacharbraceleft}\ sup{\isacharunderscore}rels{\isacharunderscore}from\ P\ x\ Y\ {\isacharbar}\ P\ {\isachardot}\ P\ {\isasymin}\ injections\ A\ Y\ {\isacharbraceright}\ {\isachardot}\ R\ {\isasymin}\ Q{\isachardoublequoteclose}\isanewline
\ \ \ \ \isacommand{unfolding}\isamarkupfalse%
\ sup{\isacharunderscore}rels{\isacharunderscore}from{\isacharunderscore}def\ \isacommand{by}\isamarkupfalse%
\ auto\isanewline
\ \ \isacommand{then}\isamarkupfalse%
\ \isacommand{show}\isamarkupfalse%
\ {\isachardoublequoteopen}R\ {\isasymin}\ {\isasymUnion}\ {\isacharbraceleft}\ sup{\isacharunderscore}rels{\isacharunderscore}from\ P\ x\ Y\ {\isacharbar}\ P\ {\isachardot}\ P\ {\isasymin}\ injections\ A\ Y\ {\isacharbraceright}{\isachardoublequoteclose}\isanewline
\ \ \ \ \isacommand{using}\isamarkupfalse%
\ Union{\isacharunderscore}member\ \isacommand{by}\isamarkupfalse%
\ {\isacharparenleft}rule\ rev{\isacharunderscore}iffD{\isadigit{1}}{\isacharparenright}\isanewline
\isacommand{next}\isamarkupfalse%
\isanewline
\ \ \isacommand{fix}\isamarkupfalse%
\ R\isanewline
\ \ \isacommand{assume}\isamarkupfalse%
\ {\isachardoublequoteopen}R\ {\isasymin}\ {\isasymUnion}\ {\isacharbraceleft}\ sup{\isacharunderscore}rels{\isacharunderscore}from\ P\ x\ Y\ {\isacharbar}\ P\ {\isachardot}\ P\ {\isasymin}\ injections\ A\ Y\ {\isacharbraceright}{\isachardoublequoteclose}\isanewline
\ \ \isacommand{then}\isamarkupfalse%
\ \isacommand{have}\isamarkupfalse%
\ {\isachardoublequoteopen}{\isasymexists}\ Q\ {\isasymin}\ {\isacharbraceleft}\ sup{\isacharunderscore}rels{\isacharunderscore}from\ P\ x\ Y\ {\isacharbar}\ P\ {\isachardot}\ P\ {\isasymin}\ injections\ A\ Y\ {\isacharbraceright}\ {\isachardot}\ R\ {\isasymin}\ Q{\isachardoublequoteclose}\isanewline
\ \ \ \ \isacommand{using}\isamarkupfalse%
\ Union{\isacharunderscore}member\ \isacommand{by}\isamarkupfalse%
\ {\isacharparenleft}rule\ rev{\isacharunderscore}iffD{\isadigit{2}}{\isacharparenright}\isanewline
\ \ \isacommand{then}\isamarkupfalse%
\ \isacommand{obtain}\isamarkupfalse%
\ P\ \isakeyword{and}\ y\ \isakeyword{where}\ P{\isacharcolon}\ {\isachardoublequoteopen}P\ {\isasymin}\ injections\ A\ Y{\isachardoublequoteclose}\isanewline
\ \ \ \ \ \ \ \ \ \ \ \ \ \ \ \ \ \ \ \ \ \ \ \ \isakeyword{and}\ y{\isacharcolon}\ {\isachardoublequoteopen}y\ {\isasymin}\ Y\ {\isacharminus}\ Range\ P{\isachardoublequoteclose}\isanewline
\ \ \ \ \ \ \ \ \ \ \ \ \ \ \ \ \ \ \ \ \ \ \ \ \isakeyword{and}\ R{\isacharcolon}\ {\isachardoublequoteopen}R\ {\isacharequal}\ P\ {\isacharplus}{\isacharasterisk}\ {\isacharbraceleft}{\isacharparenleft}x{\isacharcomma}\ y{\isacharparenright}{\isacharbraceright}{\isachardoublequoteclose}\isanewline
\ \ \ \ \isacommand{unfolding}\isamarkupfalse%
\ sup{\isacharunderscore}rels{\isacharunderscore}from{\isacharunderscore}def\ \isacommand{by}\isamarkupfalse%
\ auto\isanewline
\ \ \isacommand{then}\isamarkupfalse%
\ \isacommand{have}\isamarkupfalse%
\ P{\isacharunderscore}unfolded{\isacharcolon}\ {\isachardoublequoteopen}Domain\ P\ {\isacharequal}\ A\ {\isasymand}\ Range\ P\ {\isasymsubseteq}\ Y\ {\isasymand}\ runiq\ P\ {\isasymand}\ runiq\ {\isacharparenleft}P{\isasyminverse}{\isacharparenright}{\isachardoublequoteclose}\isanewline
\ \ \ \ \isacommand{unfolding}\isamarkupfalse%
\ injections{\isacharunderscore}def\ \isacommand{by}\isamarkupfalse%
\ {\isacharparenleft}simp\ add{\isacharcolon}\ CollectE{\isacharparenright}\isanewline
\ \ \isacommand{then}\isamarkupfalse%
\ \isacommand{have}\isamarkupfalse%
\ Domain{\isacharunderscore}pre{\isacharcolon}\ {\isachardoublequoteopen}Domain\ P\ {\isacharequal}\ A{\isachardoublequoteclose}\isanewline
\ \ \ \ \ \ \ \ \isakeyword{and}\ Range{\isacharunderscore}pre{\isacharcolon}\ {\isachardoublequoteopen}Range\ P\ {\isasymsubseteq}\ Y{\isachardoublequoteclose}\isanewline
\ \ \ \ \ \ \ \ \isakeyword{and}\ runiq{\isacharunderscore}pre{\isacharcolon}\ {\isachardoublequoteopen}runiq\ P{\isachardoublequoteclose}\isanewline
\ \ \ \ \ \ \ \ \isakeyword{and}\ runiq{\isacharunderscore}conv{\isacharunderscore}pre{\isacharcolon}\ {\isachardoublequoteopen}runiq\ {\isacharparenleft}P{\isasyminverse}{\isacharparenright}{\isachardoublequoteclose}\ \isacommand{by}\isamarkupfalse%
\ simp{\isacharunderscore}all\isanewline
\isanewline
\ \ \isacommand{show}\isamarkupfalse%
\ {\isachardoublequoteopen}R\ {\isasymin}\ injections\ {\isacharparenleft}insert\ x\ A{\isacharparenright}\ Y{\isachardoublequoteclose}\isanewline
\ \ \isacommand{proof}\isamarkupfalse%
\ {\isacharparenleft}rule\ injectionsI{\isacharparenright}\isanewline
\ \ \ \ \isacommand{show}\isamarkupfalse%
\ {\isachardoublequoteopen}Domain\ R\ {\isacharequal}\ insert\ x\ A{\isachardoublequoteclose}\isanewline
\ \ \ \ \isacommand{proof}\isamarkupfalse%
\ {\isacharminus}\isanewline
\ \ \ \ \ \ \isacommand{have}\isamarkupfalse%
\ {\isachardoublequoteopen}Domain\ R\ {\isacharequal}\ Domain\ P\ {\isasymunion}\ Domain\ {\isacharbraceleft}{\isacharparenleft}x{\isacharcomma}y{\isacharparenright}{\isacharbraceright}{\isachardoublequoteclose}\ \isacommand{using}\isamarkupfalse%
\ paste{\isacharunderscore}Domain\ R\ \isacommand{by}\isamarkupfalse%
\ metis\isanewline
\ \ \ \ \ \ \isacommand{also}\isamarkupfalse%
\ \isacommand{have}\isamarkupfalse%
\ {\isachardoublequoteopen}{\isasymdots}\ {\isacharequal}\ A\ {\isasymunion}\ {\isacharbraceleft}x{\isacharbraceright}{\isachardoublequoteclose}\ \isacommand{using}\isamarkupfalse%
\ Domain{\isacharunderscore}pre\ \isacommand{by}\isamarkupfalse%
\ simp\isanewline
\ \ \ \ \ \ \isacommand{finally}\isamarkupfalse%
\ \isacommand{show}\isamarkupfalse%
\ {\isacharquery}thesis\ \isacommand{by}\isamarkupfalse%
\ auto\isanewline
\ \ \ \ \isacommand{qed}\isamarkupfalse%
\isanewline
\isanewline
\ \ \ \ \isacommand{show}\isamarkupfalse%
\ {\isachardoublequoteopen}Range\ R\ {\isasymsubseteq}\ Y{\isachardoublequoteclose}\isanewline
\ \ \ \ \isacommand{proof}\isamarkupfalse%
\ {\isacharminus}\isanewline
\ \ \ \ \ \ \isacommand{have}\isamarkupfalse%
\ {\isachardoublequoteopen}Range\ R\ {\isasymsubseteq}\ Range\ P\ {\isasymunion}\ Range\ {\isacharbraceleft}{\isacharparenleft}x{\isacharcomma}y{\isacharparenright}{\isacharbraceright}\ {\isasymand}\ Range\ P\ {\isasymunion}\ Range\ {\isacharbraceleft}{\isacharparenleft}x{\isacharcomma}y{\isacharparenright}{\isacharbraceright}\ {\isasymsubseteq}\ Y\ {\isasymunion}\ {\isacharbraceleft}y{\isacharbraceright}{\isachardoublequoteclose}\isanewline
\ \ \ \ \ \ \ \ \isacommand{using}\isamarkupfalse%
\ paste{\isacharunderscore}Range\ R\ Range{\isacharunderscore}pre\ \isacommand{by}\isamarkupfalse%
\ force\isanewline
\ \ \ \ \ \ \isacommand{then}\isamarkupfalse%
\ \isacommand{show}\isamarkupfalse%
\ {\isacharquery}thesis\ \isacommand{using}\isamarkupfalse%
\ y\ \isacommand{by}\isamarkupfalse%
\ auto\isanewline
\ \ \ \ \isacommand{qed}\isamarkupfalse%
\isanewline
\isanewline
\ \ \ \ \isacommand{show}\isamarkupfalse%
\ {\isachardoublequoteopen}runiq\ R{\isachardoublequoteclose}\isanewline
\ \ \ \ \ \ \isacommand{using}\isamarkupfalse%
\ runiq{\isacharunderscore}pre\ R\ runiq{\isacharunderscore}singleton{\isacharunderscore}rel\ runiq{\isacharunderscore}paste{\isadigit{2}}\ \isacommand{by}\isamarkupfalse%
\ fast\isanewline
\isanewline
\ \ \ \ \isacommand{show}\isamarkupfalse%
\ {\isachardoublequoteopen}runiq\ {\isacharparenleft}R{\isasyminverse}{\isacharparenright}{\isachardoublequoteclose}\isanewline
\ \ \ \ \ \ \isacommand{using}\isamarkupfalse%
\ runiq{\isacharunderscore}conv{\isacharunderscore}pre\ R\ y\ new\ \isakeyword{and}\ runiq{\isacharunderscore}converse{\isacharunderscore}paste{\isacharunderscore}singleton\ DiffE\ Domain{\isacharunderscore}pre\isanewline
\ \ \ \ \ \ \isacommand{by}\isamarkupfalse%
\ metis\isanewline
\ \ \isacommand{qed}\isamarkupfalse%
\isanewline
\isacommand{qed}\isamarkupfalse%
%
\endisatagproof
{\isafoldproof}%
%
\isadelimproof
%
\endisadelimproof
%
\begin{isamarkuptext}%
There are finitely many injective function from a finite set to another finite set.%
\end{isamarkuptext}%
\isamarkuptrue%
\isacommand{lemma}\isamarkupfalse%
\ finite{\isacharunderscore}injections{\isacharcolon}\isanewline
\ \ \isakeyword{fixes}\ X{\isacharcolon}{\isacharcolon}{\isachardoublequoteopen}{\isacharprime}a\ set{\isachardoublequoteclose}\isanewline
\ \ \ \ \isakeyword{and}\ Y{\isacharcolon}{\isacharcolon}{\isachardoublequoteopen}{\isacharprime}b\ set{\isachardoublequoteclose}\isanewline
\ \ \isakeyword{assumes}\ {\isachardoublequoteopen}finite\ X{\isachardoublequoteclose}\isanewline
\ \ \ \ \ \ \isakeyword{and}\ {\isachardoublequoteopen}finite\ Y{\isachardoublequoteclose}\isanewline
\ \ \isakeyword{shows}\ {\isachardoublequoteopen}finite\ {\isacharparenleft}injections\ X\ Y{\isacharparenright}{\isachardoublequoteclose}\isanewline
%
\isadelimproof
%
\endisadelimproof
%
\isatagproof
\isacommand{proof}\isamarkupfalse%
\ {\isacharparenleft}rule\ rev{\isacharunderscore}finite{\isacharunderscore}subset{\isacharparenright}\isanewline
\ \ \isacommand{from}\isamarkupfalse%
\ assms\ \isacommand{show}\isamarkupfalse%
\ {\isachardoublequoteopen}finite\ {\isacharparenleft}Pow\ {\isacharparenleft}X\ {\isasymtimes}\ Y{\isacharparenright}{\isacharparenright}{\isachardoublequoteclose}\ \isacommand{by}\isamarkupfalse%
\ simp\isanewline
\ \ \isacommand{moreover}\isamarkupfalse%
\ \isacommand{show}\isamarkupfalse%
\ {\isachardoublequoteopen}injections\ X\ Y\ {\isasymsubseteq}\ Pow\ {\isacharparenleft}X\ {\isasymtimes}\ Y{\isacharparenright}{\isachardoublequoteclose}\isanewline
\ \ \isacommand{proof}\isamarkupfalse%
\isanewline
\ \ \ \ \isacommand{fix}\isamarkupfalse%
\ R\ \isacommand{assume}\isamarkupfalse%
\ {\isachardoublequoteopen}R\ {\isasymin}\ {\isacharparenleft}injections\ X\ Y{\isacharparenright}{\isachardoublequoteclose}\isanewline
\ \ \ \ \isacommand{then}\isamarkupfalse%
\ \isacommand{have}\isamarkupfalse%
\ {\isachardoublequoteopen}Domain\ R\ {\isacharequal}\ X\ {\isasymand}\ Range\ R\ {\isasymsubseteq}\ Y{\isachardoublequoteclose}\ \isacommand{unfolding}\isamarkupfalse%
\ injections{\isacharunderscore}def\ \isacommand{by}\isamarkupfalse%
\ simp\isanewline
\ \ \ \ \isacommand{then}\isamarkupfalse%
\ \isacommand{have}\isamarkupfalse%
\ {\isachardoublequoteopen}R\ {\isasymsubseteq}\ X\ {\isasymtimes}\ Y{\isachardoublequoteclose}\ \isacommand{by}\isamarkupfalse%
\ fast\isanewline
\ \ \ \ \isacommand{then}\isamarkupfalse%
\ \isacommand{show}\isamarkupfalse%
\ {\isachardoublequoteopen}R\ {\isasymin}\ Pow\ {\isacharparenleft}X\ {\isasymtimes}\ Y{\isacharparenright}{\isachardoublequoteclose}\ \isacommand{by}\isamarkupfalse%
\ simp\isanewline
\ \ \isacommand{qed}\isamarkupfalse%
\isanewline
\isacommand{qed}\isamarkupfalse%
%
\endisatagproof
{\isafoldproof}%
%
\isadelimproof
%
\endisadelimproof
%
\begin{isamarkuptext}%
The paper-like definition \isa{injections} and the algorithmic definition 
  \isa{injections{\isacharunderscore}alg} are equivalent.%
\end{isamarkuptext}%
\isamarkuptrue%
\isacommand{theorem}\isamarkupfalse%
\ injections{\isacharunderscore}equiv{\isacharcolon}\isanewline
\ \ \isakeyword{fixes}\ xs{\isacharcolon}{\isacharcolon}{\isachardoublequoteopen}{\isacharprime}a\ list{\isachardoublequoteclose}\isanewline
\ \ \ \ \isakeyword{and}\ Y{\isacharcolon}{\isacharcolon}{\isachardoublequoteopen}{\isacharprime}b{\isasymColon}linorder\ set{\isachardoublequoteclose}\isanewline
\ \ \isakeyword{assumes}\ non{\isacharunderscore}empty{\isacharcolon}\ {\isachardoublequoteopen}card\ Y\ {\isachargreater}\ {\isadigit{0}}{\isachardoublequoteclose}\isanewline
\ \ \isakeyword{shows}\ {\isachardoublequoteopen}distinct\ xs\ {\isasymLongrightarrow}\ {\isacharparenleft}set\ {\isacharparenleft}injections{\isacharunderscore}alg\ xs\ Y{\isacharparenright}{\isacharcolon}{\isacharcolon}{\isacharparenleft}{\isacharprime}a\ {\isasymtimes}\ {\isacharprime}b{\isacharparenright}\ set\ set{\isacharparenright}\ {\isacharequal}\ injections\ {\isacharparenleft}set\ xs{\isacharparenright}\ Y{\isachardoublequoteclose}\isanewline
%
\isadelimproof
%
\endisadelimproof
%
\isatagproof
\isacommand{proof}\isamarkupfalse%
\ {\isacharparenleft}induct\ xs{\isacharparenright}\isanewline
\ \ \isacommand{case}\isamarkupfalse%
\ Nil\isanewline
\ \ \isacommand{have}\isamarkupfalse%
\ {\isachardoublequoteopen}set\ {\isacharparenleft}injections{\isacharunderscore}alg\ {\isacharbrackleft}{\isacharbrackright}\ Y{\isacharparenright}\ {\isacharequal}\ {\isacharbraceleft}{\isacharbraceleft}{\isacharbraceright}{\isacharcolon}{\isacharcolon}{\isacharparenleft}{\isacharprime}a\ {\isasymtimes}\ {\isacharprime}b{\isacharparenright}\ set{\isacharbraceright}{\isachardoublequoteclose}\ \isacommand{by}\isamarkupfalse%
\ simp\isanewline
\ \ \isacommand{also}\isamarkupfalse%
\ \isacommand{have}\isamarkupfalse%
\ {\isachardoublequoteopen}{\isasymdots}\ {\isacharequal}\ injections\ {\isacharparenleft}set\ {\isacharbrackleft}{\isacharbrackright}{\isacharparenright}\ Y{\isachardoublequoteclose}\isanewline
\ \ \isacommand{proof}\isamarkupfalse%
\ {\isacharminus}\isanewline
\ \ \ \ \isacommand{have}\isamarkupfalse%
\ {\isachardoublequoteopen}{\isacharbraceleft}{\isacharbraceleft}{\isacharbraceright}{\isacharbraceright}\ {\isacharequal}\ {\isacharbraceleft}R{\isacharcolon}{\isacharcolon}{\isacharparenleft}{\isacharparenleft}{\isacharprime}a\ {\isasymtimes}\ {\isacharprime}b{\isacharparenright}\ set{\isacharparenright}\ {\isachardot}\ Domain\ R\ {\isacharequal}\ {\isacharbraceleft}{\isacharbraceright}\ {\isasymand}\ Range\ R\ {\isasymsubseteq}\ Y\ {\isasymand}\ runiq\ R\ {\isasymand}\ runiq\ {\isacharparenleft}R{\isasyminverse}{\isacharparenright}{\isacharbraceright}{\isachardoublequoteclose}\ {\isacharparenleft}\isakeyword{is}\ {\isachardoublequoteopen}{\isacharquery}LHS\ {\isacharequal}\ {\isacharquery}RHS{\isachardoublequoteclose}{\isacharparenright}\isanewline
\ \ \ \ \isacommand{proof}\isamarkupfalse%
\isanewline
\ \ \ \ \ \ \isacommand{have}\isamarkupfalse%
\ {\isachardoublequoteopen}Domain\ {\isacharbraceleft}{\isacharbraceright}\ {\isacharequal}\ {\isacharbraceleft}{\isacharbraceright}{\isachardoublequoteclose}\ \isacommand{by}\isamarkupfalse%
\ {\isacharparenleft}rule\ Domain{\isacharunderscore}empty{\isacharparenright}\isanewline
\ \ \ \ \ \ \isacommand{moreover}\isamarkupfalse%
\ \isacommand{have}\isamarkupfalse%
\ {\isachardoublequoteopen}Range\ {\isacharbraceleft}{\isacharbraceright}\ {\isasymsubseteq}\ Y{\isachardoublequoteclose}\ \isacommand{by}\isamarkupfalse%
\ simp\isanewline
\ \ \ \ \ \ \isacommand{moreover}\isamarkupfalse%
\ \isacommand{note}\isamarkupfalse%
\ runiq{\isacharunderscore}emptyrel\isanewline
\ \ \ \ \ \ \isacommand{moreover}\isamarkupfalse%
\ \isacommand{have}\isamarkupfalse%
\ {\isachardoublequoteopen}runiq\ {\isacharparenleft}{\isacharbraceleft}{\isacharbraceright}{\isasyminverse}{\isacharparenright}{\isachardoublequoteclose}\ \isacommand{by}\isamarkupfalse%
\ {\isacharparenleft}simp\ add{\isacharcolon}\ runiq{\isacharunderscore}emptyrel{\isacharparenright}\isanewline
\ \ \ \ \ \ \isacommand{ultimately}\isamarkupfalse%
\ \isacommand{have}\isamarkupfalse%
\ {\isachardoublequoteopen}Domain\ {\isacharbraceleft}{\isacharbraceright}\ {\isacharequal}\ {\isacharbraceleft}{\isacharbraceright}\ {\isasymand}\ Range\ {\isacharbraceleft}{\isacharbraceright}\ {\isasymsubseteq}\ Y\ {\isasymand}\ runiq\ {\isacharbraceleft}{\isacharbraceright}\ {\isasymand}\ runiq\ {\isacharparenleft}{\isacharbraceleft}{\isacharbraceright}{\isasyminverse}{\isacharparenright}{\isachardoublequoteclose}\ \isacommand{by}\isamarkupfalse%
\ blast\isanewline
\ \ \ \ \ \ \isanewline
\ \ \ \ \ \ \isacommand{then}\isamarkupfalse%
\ \isacommand{have}\isamarkupfalse%
\ {\isachardoublequoteopen}{\isacharbraceleft}{\isacharbraceright}\ {\isasymin}\ {\isacharbraceleft}R\ {\isachardot}\ Domain\ R\ {\isacharequal}\ {\isacharbraceleft}{\isacharbraceright}\ {\isasymand}\ Range\ R\ {\isasymsubseteq}\ Y\ {\isasymand}\ runiq\ R\ {\isasymand}\ runiq\ {\isacharparenleft}R{\isasyminverse}{\isacharparenright}{\isacharbraceright}{\isachardoublequoteclose}\ \isacommand{by}\isamarkupfalse%
\ {\isacharparenleft}rule\ CollectI{\isacharparenright}\isanewline
\ \ \ \ \ \ \isacommand{then}\isamarkupfalse%
\ \isacommand{show}\isamarkupfalse%
\ {\isachardoublequoteopen}{\isacharquery}LHS\ {\isasymsubseteq}\ {\isacharquery}RHS{\isachardoublequoteclose}\ \isacommand{using}\isamarkupfalse%
\ empty{\isacharunderscore}subsetI\ insert{\isacharunderscore}subset\ \isacommand{by}\isamarkupfalse%
\ fast\isanewline
\ \ \ \ \isacommand{next}\isamarkupfalse%
\isanewline
\ \ \ \ \ \ \isacommand{show}\isamarkupfalse%
\ {\isachardoublequoteopen}{\isacharquery}RHS\ {\isasymsubseteq}\ {\isacharquery}LHS{\isachardoublequoteclose}\isanewline
\ \ \ \ \ \ \isacommand{proof}\isamarkupfalse%
\isanewline
\ \ \ \ \ \ \ \ \isacommand{fix}\isamarkupfalse%
\ R\isanewline
\ \ \ \ \ \ \ \ \isacommand{assume}\isamarkupfalse%
\ {\isachardoublequoteopen}R\ {\isasymin}\ {\isacharbraceleft}R{\isacharcolon}{\isacharcolon}{\isacharparenleft}{\isacharparenleft}{\isacharprime}a\ {\isasymtimes}\ {\isacharprime}b{\isacharparenright}\ set{\isacharparenright}\ {\isachardot}\ Domain\ R\ {\isacharequal}\ {\isacharbraceleft}{\isacharbraceright}\ {\isasymand}\ Range\ R\ {\isasymsubseteq}\ Y\ {\isasymand}\ runiq\ R\ {\isasymand}\ runiq\ {\isacharparenleft}R{\isasyminverse}{\isacharparenright}{\isacharbraceright}{\isachardoublequoteclose}\isanewline
\ \ \ \ \ \ \ \ \isacommand{then}\isamarkupfalse%
\ \isacommand{show}\isamarkupfalse%
\ {\isachardoublequoteopen}R\ {\isasymin}\ {\isacharbraceleft}{\isacharbraceleft}{\isacharbraceright}{\isacharbraceright}{\isachardoublequoteclose}\ \isacommand{by}\isamarkupfalse%
\ {\isacharparenleft}simp\ add{\isacharcolon}\ Domain{\isacharunderscore}empty{\isacharunderscore}iff{\isacharparenright}\isanewline
\ \ \ \ \ \ \isacommand{qed}\isamarkupfalse%
\isanewline
\ \ \ \ \isacommand{qed}\isamarkupfalse%
\isanewline
\ \ \ \ \isacommand{also}\isamarkupfalse%
\ \isacommand{have}\isamarkupfalse%
\ {\isachardoublequoteopen}{\isasymdots}\ {\isacharequal}\ injections\ {\isacharparenleft}set\ {\isacharbrackleft}{\isacharbrackright}{\isacharparenright}\ Y{\isachardoublequoteclose}\isanewline
\ \ \ \ \ \ \isacommand{unfolding}\isamarkupfalse%
\ injections{\isacharunderscore}def\ \isacommand{by}\isamarkupfalse%
\ simp\isanewline
\ \ \ \ \isacommand{finally}\isamarkupfalse%
\ \isacommand{show}\isamarkupfalse%
\ {\isacharquery}thesis\ \isacommand{{\isachardot}}\isamarkupfalse%
\isanewline
\ \ \isacommand{qed}\isamarkupfalse%
\isanewline
\ \ \isacommand{finally}\isamarkupfalse%
\ \isacommand{show}\isamarkupfalse%
\ {\isacharquery}case\ \isacommand{{\isachardot}}\isamarkupfalse%
\isanewline
\isacommand{next}\isamarkupfalse%
\isanewline
\ \ \isacommand{case}\isamarkupfalse%
\ {\isacharparenleft}Cons\ x\ xs{\isacharparenright}\isanewline
\isanewline
\ \ \isacommand{from}\isamarkupfalse%
\ non{\isacharunderscore}empty\ \isacommand{have}\isamarkupfalse%
\ {\isachardoublequoteopen}finite\ Y{\isachardoublequoteclose}\ \isacommand{by}\isamarkupfalse%
\ {\isacharparenleft}rule\ card{\isacharunderscore}ge{\isacharunderscore}{\isadigit{0}}{\isacharunderscore}finite{\isacharparenright}\isanewline
\ \ \isanewline
\ \ \isanewline
\isanewline
\ \ \isanewline
\ \ \isacommand{have}\isamarkupfalse%
\ {\isachardoublequoteopen}set\ {\isacharparenleft}injections{\isacharunderscore}alg\ {\isacharparenleft}x\ {\isacharhash}\ xs{\isacharparenright}\ Y{\isacharparenright}\ {\isacharequal}\ {\isacharparenleft}{\isasymUnion}\ {\isacharbraceleft}\ set\ {\isacharparenleft}sup{\isacharunderscore}rels{\isacharunderscore}from{\isacharunderscore}alg\ R\ x\ Y{\isacharparenright}\ {\isacharbar}\ R\ {\isachardot}\ R\ {\isasymin}\ injections\ {\isacharparenleft}set\ xs{\isacharparenright}\ Y\ {\isacharbraceright}{\isacharparenright}{\isachardoublequoteclose}\isanewline
\ \ \ \ \isacommand{using}\isamarkupfalse%
\ Cons{\isachardot}hyps\ Cons{\isachardot}prems\ \isacommand{by}\isamarkupfalse%
\ {\isacharparenleft}simp\ add{\isacharcolon}\ image{\isacharunderscore}Collect{\isacharunderscore}mem{\isacharparenright}\isanewline
\ \ \isanewline
\ \ \isanewline
\ \ \isacommand{also}\isamarkupfalse%
\ \isacommand{have}\isamarkupfalse%
\ {\isachardoublequoteopen}{\isasymdots}\ {\isacharequal}\ {\isacharparenleft}{\isasymUnion}\ {\isacharbraceleft}\ sup{\isacharunderscore}rels{\isacharunderscore}from\ R\ x\ Y\ {\isacharbar}\ R\ {\isachardot}\ R\ {\isasymin}\ injections\ {\isacharparenleft}set\ xs{\isacharparenright}\ Y\ {\isacharbraceright}{\isacharparenright}{\isachardoublequoteclose}\isanewline
\ \ \ \ \isacommand{using}\isamarkupfalse%
\ {\isacharbackquoteopen}finite\ Y{\isacharbackquoteclose}\ sup{\isacharunderscore}rels{\isacharunderscore}from{\isacharunderscore}paper{\isacharunderscore}equiv{\isacharunderscore}alg\ \isacommand{by}\isamarkupfalse%
\ fast\isanewline
\ \ \isanewline
\ \ \isacommand{also}\isamarkupfalse%
\ \isacommand{have}\isamarkupfalse%
\ {\isachardoublequoteopen}{\isasymdots}\ {\isacharequal}\ injections\ {\isacharparenleft}set\ {\isacharparenleft}x\ {\isacharhash}\ xs{\isacharparenright}{\isacharparenright}\ Y{\isachardoublequoteclose}\ \isacommand{using}\isamarkupfalse%
\ Cons{\isachardot}prems\ \isacommand{by}\isamarkupfalse%
\ {\isacharparenleft}simp\ add{\isacharcolon}\ injections{\isacharunderscore}paste{\isacharparenright}\isanewline
\ \ \isanewline
\ \ \isanewline
\ \ \isacommand{finally}\isamarkupfalse%
\ \isacommand{show}\isamarkupfalse%
\ {\isacharquery}case\ \isacommand{{\isachardot}}\isamarkupfalse%
\isanewline
\isacommand{qed}\isamarkupfalse%
%
\endisatagproof
{\isafoldproof}%
%
\isadelimproof
\isanewline
%
\endisadelimproof
\isanewline
\isacommand{lemma}\isamarkupfalse%
\ Image{\isacharunderscore}within{\isacharunderscore}domain{\isacharprime}{\isacharcolon}\ \isakeyword{fixes}\ x\ R\ \isakeyword{shows}\ {\isachardoublequoteopen}x\ {\isasymin}\ Domain\ R\ {\isacharequal}\ {\isacharparenleft}R\ {\isacharbackquote}{\isacharbackquote}\ {\isacharbraceleft}x{\isacharbraceright}\ {\isasymnoteq}\ {\isacharbraceleft}{\isacharbraceright}{\isacharparenright}{\isachardoublequoteclose}%
\isadelimproof
\ %
\endisadelimproof
%
\isatagproof
\isacommand{by}\isamarkupfalse%
\ blast%
\endisatagproof
{\isafoldproof}%
%
\isadelimproof
%
\endisadelimproof
\isanewline
%
\isadelimtheory
\isanewline
%
\endisadelimtheory
%
\isatagtheory
\isacommand{end}\isamarkupfalse%
%
\endisatagtheory
{\isafoldtheory}%
%
\isadelimtheory
%
\endisadelimtheory
\end{isabellebody}%
%%% Local Variables:
%%% mode: latex
%%% TeX-master: "root"
%%% End:


%
\begin{isabellebody}%
\def\isabellecontext{Discrete}%
%
\isamarkupheader{Common discrete functions%
}
\isamarkuptrue%
%
\isadelimtheory
%
\endisadelimtheory
%
\isatagtheory
\isacommand{theory}\isamarkupfalse%
\ Discrete\isanewline
\isakeyword{imports}\ Main\isanewline
\isakeyword{begin}%
\endisatagtheory
{\isafoldtheory}%
%
\isadelimtheory
%
\endisadelimtheory
%
\isamarkupsubsection{Discrete logarithm%
}
\isamarkuptrue%
\isacommand{fun}\isamarkupfalse%
\ log\ {\isacharcolon}{\isacharcolon}\ {\isachardoublequoteopen}nat\ {\isasymRightarrow}\ nat{\isachardoublequoteclose}\ \isakeyword{where}\isanewline
\ \ {\isacharbrackleft}simp\ del{\isacharbrackright}{\isacharcolon}\ {\isachardoublequoteopen}log\ n\ {\isacharequal}\ {\isacharparenleft}if\ n\ {\isacharless}\ {\isadigit{2}}\ then\ {\isadigit{0}}\ else\ Suc\ {\isacharparenleft}log\ {\isacharparenleft}n\ div\ {\isadigit{2}}{\isacharparenright}{\isacharparenright}{\isacharparenright}{\isachardoublequoteclose}\isanewline
\isanewline
\isacommand{lemma}\isamarkupfalse%
\ log{\isacharunderscore}zero\ {\isacharbrackleft}simp{\isacharbrackright}{\isacharcolon}\isanewline
\ \ {\isachardoublequoteopen}log\ {\isadigit{0}}\ {\isacharequal}\ {\isadigit{0}}{\isachardoublequoteclose}\isanewline
%
\isadelimproof
\ \ %
\endisadelimproof
%
\isatagproof
\isacommand{by}\isamarkupfalse%
\ {\isacharparenleft}simp\ add{\isacharcolon}\ log{\isachardot}simps{\isacharparenright}%
\endisatagproof
{\isafoldproof}%
%
\isadelimproof
\isanewline
%
\endisadelimproof
\isanewline
\isacommand{lemma}\isamarkupfalse%
\ log{\isacharunderscore}one\ {\isacharbrackleft}simp{\isacharbrackright}{\isacharcolon}\isanewline
\ \ {\isachardoublequoteopen}log\ {\isadigit{1}}\ {\isacharequal}\ {\isadigit{0}}{\isachardoublequoteclose}\isanewline
%
\isadelimproof
\ \ %
\endisadelimproof
%
\isatagproof
\isacommand{by}\isamarkupfalse%
\ {\isacharparenleft}simp\ add{\isacharcolon}\ log{\isachardot}simps{\isacharparenright}%
\endisatagproof
{\isafoldproof}%
%
\isadelimproof
\isanewline
%
\endisadelimproof
\isanewline
\isacommand{lemma}\isamarkupfalse%
\ log{\isacharunderscore}Suc{\isacharunderscore}zero\ {\isacharbrackleft}simp{\isacharbrackright}{\isacharcolon}\isanewline
\ \ {\isachardoublequoteopen}log\ {\isacharparenleft}Suc\ {\isadigit{0}}{\isacharparenright}\ {\isacharequal}\ {\isadigit{0}}{\isachardoublequoteclose}\isanewline
%
\isadelimproof
\ \ %
\endisadelimproof
%
\isatagproof
\isacommand{using}\isamarkupfalse%
\ log{\isacharunderscore}one\ \isacommand{by}\isamarkupfalse%
\ simp%
\endisatagproof
{\isafoldproof}%
%
\isadelimproof
\isanewline
%
\endisadelimproof
\isanewline
\isacommand{lemma}\isamarkupfalse%
\ log{\isacharunderscore}rec{\isacharcolon}\isanewline
\ \ {\isachardoublequoteopen}n\ {\isasymge}\ {\isadigit{2}}\ {\isasymLongrightarrow}\ log\ n\ {\isacharequal}\ Suc\ {\isacharparenleft}log\ {\isacharparenleft}n\ div\ {\isadigit{2}}{\isacharparenright}{\isacharparenright}{\isachardoublequoteclose}\isanewline
%
\isadelimproof
\ \ %
\endisadelimproof
%
\isatagproof
\isacommand{by}\isamarkupfalse%
\ {\isacharparenleft}simp\ add{\isacharcolon}\ log{\isachardot}simps{\isacharparenright}%
\endisatagproof
{\isafoldproof}%
%
\isadelimproof
\isanewline
%
\endisadelimproof
\isanewline
\isacommand{lemma}\isamarkupfalse%
\ log{\isacharunderscore}twice\ {\isacharbrackleft}simp{\isacharbrackright}{\isacharcolon}\isanewline
\ \ {\isachardoublequoteopen}n\ {\isasymnoteq}\ {\isadigit{0}}\ {\isasymLongrightarrow}\ log\ {\isacharparenleft}{\isadigit{2}}\ {\isacharasterisk}\ n{\isacharparenright}\ {\isacharequal}\ Suc\ {\isacharparenleft}log\ n{\isacharparenright}{\isachardoublequoteclose}\isanewline
%
\isadelimproof
\ \ %
\endisadelimproof
%
\isatagproof
\isacommand{by}\isamarkupfalse%
\ {\isacharparenleft}simp\ add{\isacharcolon}\ log{\isacharunderscore}rec{\isacharparenright}%
\endisatagproof
{\isafoldproof}%
%
\isadelimproof
\isanewline
%
\endisadelimproof
\isanewline
\isacommand{lemma}\isamarkupfalse%
\ log{\isacharunderscore}half\ {\isacharbrackleft}simp{\isacharbrackright}{\isacharcolon}\isanewline
\ \ {\isachardoublequoteopen}log\ {\isacharparenleft}n\ div\ {\isadigit{2}}{\isacharparenright}\ {\isacharequal}\ log\ n\ {\isacharminus}\ {\isadigit{1}}{\isachardoublequoteclose}\isanewline
%
\isadelimproof
%
\endisadelimproof
%
\isatagproof
\isacommand{proof}\isamarkupfalse%
\ {\isacharparenleft}cases\ {\isachardoublequoteopen}n\ {\isacharless}\ {\isadigit{2}}{\isachardoublequoteclose}{\isacharparenright}\isanewline
\ \ \isacommand{case}\isamarkupfalse%
\ True\isanewline
\ \ \isacommand{then}\isamarkupfalse%
\ \isacommand{have}\isamarkupfalse%
\ {\isachardoublequoteopen}n\ {\isacharequal}\ {\isadigit{0}}\ {\isasymor}\ n\ {\isacharequal}\ {\isadigit{1}}{\isachardoublequoteclose}\ \isacommand{by}\isamarkupfalse%
\ arith\isanewline
\ \ \isacommand{then}\isamarkupfalse%
\ \isacommand{show}\isamarkupfalse%
\ {\isacharquery}thesis\ \isacommand{by}\isamarkupfalse%
\ {\isacharparenleft}auto\ simp\ del{\isacharcolon}\ One{\isacharunderscore}nat{\isacharunderscore}def{\isacharparenright}\isanewline
\isacommand{next}\isamarkupfalse%
\isanewline
\ \ \isacommand{case}\isamarkupfalse%
\ False\ \isacommand{then}\isamarkupfalse%
\ \isacommand{show}\isamarkupfalse%
\ {\isacharquery}thesis\ \isacommand{by}\isamarkupfalse%
\ {\isacharparenleft}simp\ add{\isacharcolon}\ log{\isacharunderscore}rec{\isacharparenright}\isanewline
\isacommand{qed}\isamarkupfalse%
%
\endisatagproof
{\isafoldproof}%
%
\isadelimproof
\isanewline
%
\endisadelimproof
\isanewline
\isacommand{lemma}\isamarkupfalse%
\ log{\isacharunderscore}exp\ {\isacharbrackleft}simp{\isacharbrackright}{\isacharcolon}\isanewline
\ \ {\isachardoublequoteopen}log\ {\isacharparenleft}{\isadigit{2}}\ {\isacharcircum}\ n{\isacharparenright}\ {\isacharequal}\ n{\isachardoublequoteclose}\isanewline
%
\isadelimproof
\ \ %
\endisadelimproof
%
\isatagproof
\isacommand{by}\isamarkupfalse%
\ {\isacharparenleft}induct\ n{\isacharparenright}\ simp{\isacharunderscore}all%
\endisatagproof
{\isafoldproof}%
%
\isadelimproof
\isanewline
%
\endisadelimproof
\isanewline
\isacommand{lemma}\isamarkupfalse%
\ log{\isacharunderscore}mono{\isacharcolon}\isanewline
\ \ {\isachardoublequoteopen}mono\ log{\isachardoublequoteclose}\isanewline
%
\isadelimproof
%
\endisadelimproof
%
\isatagproof
\isacommand{proof}\isamarkupfalse%
\isanewline
\ \ \isacommand{fix}\isamarkupfalse%
\ m\ n\ {\isacharcolon}{\isacharcolon}\ nat\isanewline
\ \ \isacommand{assume}\isamarkupfalse%
\ {\isachardoublequoteopen}m\ {\isasymle}\ n{\isachardoublequoteclose}\isanewline
\ \ \isacommand{then}\isamarkupfalse%
\ \isacommand{show}\isamarkupfalse%
\ {\isachardoublequoteopen}log\ m\ {\isasymle}\ log\ n{\isachardoublequoteclose}\isanewline
\ \ \isacommand{proof}\isamarkupfalse%
\ {\isacharparenleft}induct\ m\ arbitrary{\isacharcolon}\ n\ rule{\isacharcolon}\ log{\isachardot}induct{\isacharparenright}\isanewline
\ \ \ \ \isacommand{case}\isamarkupfalse%
\ {\isacharparenleft}{\isadigit{1}}\ m{\isacharparenright}\isanewline
\ \ \ \ \isacommand{then}\isamarkupfalse%
\ \isacommand{have}\isamarkupfalse%
\ mn{\isadigit{2}}{\isacharcolon}\ {\isachardoublequoteopen}m\ div\ {\isadigit{2}}\ {\isasymle}\ n\ div\ {\isadigit{2}}{\isachardoublequoteclose}\ \isacommand{by}\isamarkupfalse%
\ arith\isanewline
\ \ \ \ \isacommand{show}\isamarkupfalse%
\ {\isachardoublequoteopen}log\ m\ {\isasymle}\ log\ n{\isachardoublequoteclose}\isanewline
\ \ \ \ \isacommand{proof}\isamarkupfalse%
\ {\isacharparenleft}cases\ {\isachardoublequoteopen}m\ {\isacharless}\ {\isadigit{2}}{\isachardoublequoteclose}{\isacharparenright}\isanewline
\ \ \ \ \ \ \isacommand{case}\isamarkupfalse%
\ True\isanewline
\ \ \ \ \ \ \isacommand{then}\isamarkupfalse%
\ \isacommand{have}\isamarkupfalse%
\ {\isachardoublequoteopen}m\ {\isacharequal}\ {\isadigit{0}}\ {\isasymor}\ m\ {\isacharequal}\ {\isadigit{1}}{\isachardoublequoteclose}\ \isacommand{by}\isamarkupfalse%
\ arith\isanewline
\ \ \ \ \ \ \isacommand{then}\isamarkupfalse%
\ \isacommand{show}\isamarkupfalse%
\ {\isacharquery}thesis\ \isacommand{by}\isamarkupfalse%
\ {\isacharparenleft}auto\ simp\ del{\isacharcolon}\ One{\isacharunderscore}nat{\isacharunderscore}def{\isacharparenright}\isanewline
\ \ \ \ \isacommand{next}\isamarkupfalse%
\isanewline
\ \ \ \ \ \ \isacommand{case}\isamarkupfalse%
\ False\isanewline
\ \ \ \ \ \ \isacommand{with}\isamarkupfalse%
\ mn{\isadigit{2}}\ \isacommand{have}\isamarkupfalse%
\ {\isachardoublequoteopen}m\ {\isasymge}\ {\isadigit{2}}{\isachardoublequoteclose}\ \isakeyword{and}\ {\isachardoublequoteopen}n\ {\isasymge}\ {\isadigit{2}}{\isachardoublequoteclose}\ \isacommand{by}\isamarkupfalse%
\ auto\ arith\isanewline
\ \ \ \ \ \ \isacommand{from}\isamarkupfalse%
\ False\ \isacommand{have}\isamarkupfalse%
\ m{\isadigit{2}}{\isacharunderscore}{\isadigit{0}}{\isacharcolon}\ {\isachardoublequoteopen}m\ div\ {\isadigit{2}}\ {\isasymnoteq}\ {\isadigit{0}}{\isachardoublequoteclose}\ \isacommand{by}\isamarkupfalse%
\ arith\isanewline
\ \ \ \ \ \ \isacommand{with}\isamarkupfalse%
\ mn{\isadigit{2}}\ \isacommand{have}\isamarkupfalse%
\ n{\isadigit{2}}{\isacharunderscore}{\isadigit{0}}{\isacharcolon}\ {\isachardoublequoteopen}n\ div\ {\isadigit{2}}\ {\isasymnoteq}\ {\isadigit{0}}{\isachardoublequoteclose}\ \isacommand{by}\isamarkupfalse%
\ arith\isanewline
\ \ \ \ \ \ \isacommand{from}\isamarkupfalse%
\ False\ {\isachardoublequoteopen}{\isadigit{1}}{\isachardot}hyps{\isachardoublequoteclose}\ mn{\isadigit{2}}\ \isacommand{have}\isamarkupfalse%
\ {\isachardoublequoteopen}log\ {\isacharparenleft}m\ div\ {\isadigit{2}}{\isacharparenright}\ {\isasymle}\ log\ {\isacharparenleft}n\ div\ {\isadigit{2}}{\isacharparenright}{\isachardoublequoteclose}\ \isacommand{by}\isamarkupfalse%
\ blast\isanewline
\ \ \ \ \ \ \isacommand{with}\isamarkupfalse%
\ m{\isadigit{2}}{\isacharunderscore}{\isadigit{0}}\ n{\isadigit{2}}{\isacharunderscore}{\isadigit{0}}\ \isacommand{have}\isamarkupfalse%
\ {\isachardoublequoteopen}log\ {\isacharparenleft}{\isadigit{2}}\ {\isacharasterisk}\ {\isacharparenleft}m\ div\ {\isadigit{2}}{\isacharparenright}{\isacharparenright}\ {\isasymle}\ log\ {\isacharparenleft}{\isadigit{2}}\ {\isacharasterisk}\ {\isacharparenleft}n\ div\ {\isadigit{2}}{\isacharparenright}{\isacharparenright}{\isachardoublequoteclose}\ \isacommand{by}\isamarkupfalse%
\ simp\isanewline
\ \ \ \ \ \ \isacommand{with}\isamarkupfalse%
\ m{\isadigit{2}}{\isacharunderscore}{\isadigit{0}}\ n{\isadigit{2}}{\isacharunderscore}{\isadigit{0}}\ {\isacharbackquoteopen}m\ {\isasymge}\ {\isadigit{2}}{\isacharbackquoteclose}\ {\isacharbackquoteopen}n\ {\isasymge}\ {\isadigit{2}}{\isacharbackquoteclose}\ \isacommand{show}\isamarkupfalse%
\ {\isacharquery}thesis\ \isacommand{by}\isamarkupfalse%
\ {\isacharparenleft}simp\ only{\isacharcolon}\ log{\isacharunderscore}rec\ {\isacharbrackleft}of\ m{\isacharbrackright}\ log{\isacharunderscore}rec\ {\isacharbrackleft}of\ n{\isacharbrackright}{\isacharparenright}\ simp\isanewline
\ \ \ \ \isacommand{qed}\isamarkupfalse%
\isanewline
\ \ \isacommand{qed}\isamarkupfalse%
\isanewline
\isacommand{qed}\isamarkupfalse%
%
\endisatagproof
{\isafoldproof}%
%
\isadelimproof
%
\endisadelimproof
%
\isamarkupsubsection{Discrete square root%
}
\isamarkuptrue%
\isacommand{definition}\isamarkupfalse%
\ sqrt\ {\isacharcolon}{\isacharcolon}\ {\isachardoublequoteopen}nat\ {\isasymRightarrow}\ nat{\isachardoublequoteclose}\isanewline
\isakeyword{where}\isanewline
\ \ {\isachardoublequoteopen}sqrt\ n\ {\isacharequal}\ Max\ {\isacharbraceleft}m{\isachardot}\ m\isactrlsup {\isadigit{2}}\ {\isasymle}\ n{\isacharbraceright}{\isachardoublequoteclose}\isanewline
\isanewline
\isacommand{lemma}\isamarkupfalse%
\ sqrt{\isacharunderscore}aux{\isacharcolon}\isanewline
\ \ \isakeyword{fixes}\ n\ {\isacharcolon}{\isacharcolon}\ nat\isanewline
\ \ \isakeyword{shows}\ {\isachardoublequoteopen}finite\ {\isacharbraceleft}m{\isachardot}\ m\isactrlsup {\isadigit{2}}\ {\isasymle}\ n{\isacharbraceright}{\isachardoublequoteclose}\ \isakeyword{and}\ {\isachardoublequoteopen}{\isacharbraceleft}m{\isachardot}\ m\isactrlsup {\isadigit{2}}\ {\isasymle}\ n{\isacharbraceright}\ {\isasymnoteq}\ {\isacharbraceleft}{\isacharbraceright}{\isachardoublequoteclose}\isanewline
%
\isadelimproof
%
\endisadelimproof
%
\isatagproof
\isacommand{proof}\isamarkupfalse%
\ {\isacharminus}\isanewline
\ \ \isacommand{{\isacharbraceleft}}\isamarkupfalse%
\ \isacommand{fix}\isamarkupfalse%
\ m\isanewline
\ \ \ \ \isacommand{assume}\isamarkupfalse%
\ {\isachardoublequoteopen}m\isactrlsup {\isadigit{2}}\ {\isasymle}\ n{\isachardoublequoteclose}\isanewline
\ \ \ \ \isacommand{then}\isamarkupfalse%
\ \isacommand{have}\isamarkupfalse%
\ {\isachardoublequoteopen}m\ {\isasymle}\ n{\isachardoublequoteclose}\isanewline
\ \ \ \ \ \ \isacommand{by}\isamarkupfalse%
\ {\isacharparenleft}cases\ m{\isacharparenright}\ {\isacharparenleft}simp{\isacharunderscore}all\ add{\isacharcolon}\ power{\isadigit{2}}{\isacharunderscore}eq{\isacharunderscore}square{\isacharparenright}\isanewline
\ \ \isacommand{{\isacharbraceright}}\isamarkupfalse%
\ \isacommand{note}\isamarkupfalse%
\ {\isacharasterisk}{\isacharasterisk}\ {\isacharequal}\ this\isanewline
\ \ \isacommand{then}\isamarkupfalse%
\ \isacommand{have}\isamarkupfalse%
\ {\isachardoublequoteopen}{\isacharbraceleft}m{\isachardot}\ m\isactrlsup {\isadigit{2}}\ {\isasymle}\ n{\isacharbraceright}\ {\isasymsubseteq}\ {\isacharbraceleft}m{\isachardot}\ m\ {\isasymle}\ n{\isacharbraceright}{\isachardoublequoteclose}\ \isacommand{by}\isamarkupfalse%
\ auto\isanewline
\ \ \isacommand{then}\isamarkupfalse%
\ \isacommand{show}\isamarkupfalse%
\ {\isachardoublequoteopen}finite\ {\isacharbraceleft}m{\isachardot}\ m\isactrlsup {\isadigit{2}}\ {\isasymle}\ n{\isacharbraceright}{\isachardoublequoteclose}\ \isacommand{by}\isamarkupfalse%
\ {\isacharparenleft}rule\ finite{\isacharunderscore}subset{\isacharparenright}\ rule\isanewline
\ \ \isacommand{have}\isamarkupfalse%
\ {\isachardoublequoteopen}{\isadigit{0}}\isactrlsup {\isadigit{2}}\ {\isasymle}\ n{\isachardoublequoteclose}\ \isacommand{by}\isamarkupfalse%
\ simp\isanewline
\ \ \isacommand{then}\isamarkupfalse%
\ \isacommand{show}\isamarkupfalse%
\ {\isacharasterisk}{\isacharcolon}\ {\isachardoublequoteopen}{\isacharbraceleft}m{\isachardot}\ m\isactrlsup {\isadigit{2}}\ {\isasymle}\ n{\isacharbraceright}\ {\isasymnoteq}\ {\isacharbraceleft}{\isacharbraceright}{\isachardoublequoteclose}\ \isacommand{by}\isamarkupfalse%
\ blast\isanewline
\isacommand{qed}\isamarkupfalse%
%
\endisatagproof
{\isafoldproof}%
%
\isadelimproof
\isanewline
%
\endisadelimproof
\isanewline
\isacommand{lemma}\isamarkupfalse%
\ {\isacharbrackleft}code{\isacharbrackright}{\isacharcolon}\isanewline
\ \ {\isachardoublequoteopen}sqrt\ n\ {\isacharequal}\ Max\ {\isacharparenleft}Set{\isachardot}filter\ {\isacharparenleft}{\isasymlambda}m{\isachardot}\ m\isactrlsup {\isadigit{2}}\ {\isasymle}\ n{\isacharparenright}\ {\isacharbraceleft}{\isadigit{0}}{\isachardot}{\isachardot}n{\isacharbraceright}{\isacharparenright}{\isachardoublequoteclose}\isanewline
%
\isadelimproof
%
\endisadelimproof
%
\isatagproof
\isacommand{proof}\isamarkupfalse%
\ {\isacharminus}\isanewline
\ \ \isacommand{from}\isamarkupfalse%
\ power{\isadigit{2}}{\isacharunderscore}nat{\isacharunderscore}le{\isacharunderscore}imp{\isacharunderscore}le\ {\isacharbrackleft}of\ {\isacharunderscore}\ n{\isacharbrackright}\ \isacommand{have}\isamarkupfalse%
\ {\isachardoublequoteopen}{\isacharbraceleft}m{\isachardot}\ m\ {\isasymle}\ n\ {\isasymand}\ m\isactrlsup {\isadigit{2}}\ {\isasymle}\ n{\isacharbraceright}\ {\isacharequal}\ {\isacharbraceleft}m{\isachardot}\ m\isactrlsup {\isadigit{2}}\ {\isasymle}\ n{\isacharbraceright}{\isachardoublequoteclose}\ \isacommand{by}\isamarkupfalse%
\ auto\isanewline
\ \ \isacommand{then}\isamarkupfalse%
\ \isacommand{show}\isamarkupfalse%
\ {\isacharquery}thesis\ \isacommand{by}\isamarkupfalse%
\ {\isacharparenleft}simp\ add{\isacharcolon}\ sqrt{\isacharunderscore}def\ Set{\isachardot}filter{\isacharunderscore}def{\isacharparenright}\isanewline
\isacommand{qed}\isamarkupfalse%
%
\endisatagproof
{\isafoldproof}%
%
\isadelimproof
\isanewline
%
\endisadelimproof
\isanewline
\isacommand{lemma}\isamarkupfalse%
\ sqrt{\isacharunderscore}inverse{\isacharunderscore}power{\isadigit{2}}\ {\isacharbrackleft}simp{\isacharbrackright}{\isacharcolon}\isanewline
\ \ {\isachardoublequoteopen}sqrt\ {\isacharparenleft}n\isactrlsup {\isadigit{2}}{\isacharparenright}\ {\isacharequal}\ n{\isachardoublequoteclose}\isanewline
%
\isadelimproof
%
\endisadelimproof
%
\isatagproof
\isacommand{proof}\isamarkupfalse%
\ {\isacharminus}\isanewline
\ \ \isacommand{have}\isamarkupfalse%
\ {\isachardoublequoteopen}{\isacharbraceleft}m{\isachardot}\ m\ {\isasymle}\ n{\isacharbraceright}\ {\isasymnoteq}\ {\isacharbraceleft}{\isacharbraceright}{\isachardoublequoteclose}\ \isacommand{by}\isamarkupfalse%
\ auto\isanewline
\ \ \isacommand{then}\isamarkupfalse%
\ \isacommand{have}\isamarkupfalse%
\ {\isachardoublequoteopen}Max\ {\isacharbraceleft}m{\isachardot}\ m\ {\isasymle}\ n{\isacharbraceright}\ {\isasymle}\ n{\isachardoublequoteclose}\ \isacommand{by}\isamarkupfalse%
\ auto\isanewline
\ \ \isacommand{then}\isamarkupfalse%
\ \isacommand{show}\isamarkupfalse%
\ {\isacharquery}thesis\isanewline
\ \ \ \ \isacommand{by}\isamarkupfalse%
\ {\isacharparenleft}auto\ simp\ add{\isacharcolon}\ sqrt{\isacharunderscore}def\ power{\isadigit{2}}{\isacharunderscore}nat{\isacharunderscore}le{\isacharunderscore}eq{\isacharunderscore}le\ intro{\isacharcolon}\ antisym{\isacharparenright}\isanewline
\isacommand{qed}\isamarkupfalse%
%
\endisatagproof
{\isafoldproof}%
%
\isadelimproof
\isanewline
%
\endisadelimproof
\isanewline
\isacommand{lemma}\isamarkupfalse%
\ mono{\isacharunderscore}sqrt{\isacharcolon}\ {\isachardoublequoteopen}mono\ sqrt{\isachardoublequoteclose}\isanewline
%
\isadelimproof
%
\endisadelimproof
%
\isatagproof
\isacommand{proof}\isamarkupfalse%
\isanewline
\ \ \isacommand{fix}\isamarkupfalse%
\ m\ n\ {\isacharcolon}{\isacharcolon}\ nat\isanewline
\ \ \isacommand{have}\isamarkupfalse%
\ {\isacharasterisk}{\isacharcolon}\ {\isachardoublequoteopen}{\isadigit{0}}\ {\isacharasterisk}\ {\isadigit{0}}\ {\isasymle}\ m{\isachardoublequoteclose}\ \isacommand{by}\isamarkupfalse%
\ simp\isanewline
\ \ \isacommand{assume}\isamarkupfalse%
\ {\isachardoublequoteopen}m\ {\isasymle}\ n{\isachardoublequoteclose}\isanewline
\ \ \isacommand{then}\isamarkupfalse%
\ \isacommand{show}\isamarkupfalse%
\ {\isachardoublequoteopen}sqrt\ m\ {\isasymle}\ sqrt\ n{\isachardoublequoteclose}\isanewline
\ \ \ \ \isacommand{by}\isamarkupfalse%
\ {\isacharparenleft}auto\ intro{\isacharbang}{\isacharcolon}\ Max{\isacharunderscore}mono\ {\isacharbackquoteopen}{\isadigit{0}}\ {\isacharasterisk}\ {\isadigit{0}}\ {\isasymle}\ m{\isacharbackquoteclose}\ finite{\isacharunderscore}less{\isacharunderscore}ub\ simp\ add{\isacharcolon}\ power{\isadigit{2}}{\isacharunderscore}eq{\isacharunderscore}square\ sqrt{\isacharunderscore}def{\isacharparenright}\isanewline
\isacommand{qed}\isamarkupfalse%
%
\endisatagproof
{\isafoldproof}%
%
\isadelimproof
\isanewline
%
\endisadelimproof
\isanewline
\isacommand{lemma}\isamarkupfalse%
\ sqrt{\isacharunderscore}greater{\isacharunderscore}zero{\isacharunderscore}iff\ {\isacharbrackleft}simp{\isacharbrackright}{\isacharcolon}\isanewline
\ \ {\isachardoublequoteopen}sqrt\ n\ {\isachargreater}\ {\isadigit{0}}\ {\isasymlongleftrightarrow}\ n\ {\isachargreater}\ {\isadigit{0}}{\isachardoublequoteclose}\isanewline
%
\isadelimproof
%
\endisadelimproof
%
\isatagproof
\isacommand{proof}\isamarkupfalse%
\ {\isacharminus}\isanewline
\ \ \isacommand{have}\isamarkupfalse%
\ {\isacharasterisk}{\isacharcolon}\ {\isachardoublequoteopen}{\isadigit{0}}\ {\isacharless}\ Max\ {\isacharbraceleft}m{\isachardot}\ m\isactrlsup {\isadigit{2}}\ {\isasymle}\ n{\isacharbraceright}\ {\isasymlongleftrightarrow}\ {\isacharparenleft}{\isasymexists}a{\isasymin}{\isacharbraceleft}m{\isachardot}\ m\isactrlsup {\isadigit{2}}\ {\isasymle}\ n{\isacharbraceright}{\isachardot}\ {\isadigit{0}}\ {\isacharless}\ a{\isacharparenright}{\isachardoublequoteclose}\isanewline
\ \ \ \ \isacommand{by}\isamarkupfalse%
\ {\isacharparenleft}rule\ Max{\isacharunderscore}gr{\isacharunderscore}iff{\isacharparenright}\ {\isacharparenleft}fact\ sqrt{\isacharunderscore}aux{\isacharparenright}{\isacharplus}\isanewline
\ \ \isacommand{show}\isamarkupfalse%
\ {\isacharquery}thesis\isanewline
\ \ \isacommand{proof}\isamarkupfalse%
\isanewline
\ \ \ \ \isacommand{assume}\isamarkupfalse%
\ {\isachardoublequoteopen}{\isadigit{0}}\ {\isacharless}\ sqrt\ n{\isachardoublequoteclose}\isanewline
\ \ \ \ \isacommand{then}\isamarkupfalse%
\ \isacommand{have}\isamarkupfalse%
\ {\isachardoublequoteopen}{\isadigit{0}}\ {\isacharless}\ Max\ {\isacharbraceleft}m{\isachardot}\ m\isactrlsup {\isadigit{2}}\ {\isasymle}\ n{\isacharbraceright}{\isachardoublequoteclose}\ \isacommand{by}\isamarkupfalse%
\ {\isacharparenleft}simp\ add{\isacharcolon}\ sqrt{\isacharunderscore}def{\isacharparenright}\isanewline
\ \ \ \ \isacommand{with}\isamarkupfalse%
\ {\isacharasterisk}\ \isacommand{show}\isamarkupfalse%
\ {\isachardoublequoteopen}{\isadigit{0}}\ {\isacharless}\ n{\isachardoublequoteclose}\ \isacommand{by}\isamarkupfalse%
\ {\isacharparenleft}auto\ dest{\isacharcolon}\ power{\isadigit{2}}{\isacharunderscore}nat{\isacharunderscore}le{\isacharunderscore}imp{\isacharunderscore}le{\isacharparenright}\isanewline
\ \ \isacommand{next}\isamarkupfalse%
\isanewline
\ \ \ \ \isacommand{assume}\isamarkupfalse%
\ {\isachardoublequoteopen}{\isadigit{0}}\ {\isacharless}\ n{\isachardoublequoteclose}\isanewline
\ \ \ \ \isacommand{then}\isamarkupfalse%
\ \isacommand{have}\isamarkupfalse%
\ {\isachardoublequoteopen}{\isadigit{1}}\isactrlsup {\isadigit{2}}\ {\isasymle}\ n\ {\isasymand}\ {\isadigit{0}}\ {\isacharless}\ {\isacharparenleft}{\isadigit{1}}{\isacharcolon}{\isacharcolon}nat{\isacharparenright}{\isachardoublequoteclose}\ \isacommand{by}\isamarkupfalse%
\ simp\isanewline
\ \ \ \ \isacommand{then}\isamarkupfalse%
\ \isacommand{have}\isamarkupfalse%
\ {\isachardoublequoteopen}{\isasymexists}q{\isachardot}\ q\isactrlsup {\isadigit{2}}\ {\isasymle}\ n\ {\isasymand}\ {\isadigit{0}}\ {\isacharless}\ q{\isachardoublequoteclose}\ \isacommand{{\isachardot}{\isachardot}}\isamarkupfalse%
\isanewline
\ \ \ \ \isacommand{with}\isamarkupfalse%
\ {\isacharasterisk}\ \isacommand{have}\isamarkupfalse%
\ {\isachardoublequoteopen}{\isadigit{0}}\ {\isacharless}\ Max\ {\isacharbraceleft}m{\isachardot}\ m\isactrlsup {\isadigit{2}}\ {\isasymle}\ n{\isacharbraceright}{\isachardoublequoteclose}\ \isacommand{by}\isamarkupfalse%
\ blast\isanewline
\ \ \ \ \isacommand{then}\isamarkupfalse%
\ \isacommand{show}\isamarkupfalse%
\ {\isachardoublequoteopen}{\isadigit{0}}\ {\isacharless}\ sqrt\ n{\isachardoublequoteclose}\ \isacommand{by}\isamarkupfalse%
\ \ {\isacharparenleft}simp\ add{\isacharcolon}\ sqrt{\isacharunderscore}def{\isacharparenright}\isanewline
\ \ \isacommand{qed}\isamarkupfalse%
\isanewline
\isacommand{qed}\isamarkupfalse%
%
\endisatagproof
{\isafoldproof}%
%
\isadelimproof
\isanewline
%
\endisadelimproof
\isanewline
\isacommand{lemma}\isamarkupfalse%
\ sqrt{\isacharunderscore}power{\isadigit{2}}{\isacharunderscore}le\ {\isacharbrackleft}simp{\isacharbrackright}{\isacharcolon}\ \isanewline
\ \ {\isachardoublequoteopen}{\isacharparenleft}sqrt\ n{\isacharparenright}\isactrlsup {\isadigit{2}}\ {\isasymle}\ n{\isachardoublequoteclose}\isanewline
%
\isadelimproof
%
\endisadelimproof
%
\isatagproof
\isacommand{proof}\isamarkupfalse%
\ {\isacharparenleft}cases\ {\isachardoublequoteopen}n\ {\isachargreater}\ {\isadigit{0}}{\isachardoublequoteclose}{\isacharparenright}\isanewline
\ \ \isacommand{case}\isamarkupfalse%
\ False\ \isacommand{then}\isamarkupfalse%
\ \isacommand{show}\isamarkupfalse%
\ {\isacharquery}thesis\ \isacommand{by}\isamarkupfalse%
\ {\isacharparenleft}simp\ add{\isacharcolon}\ sqrt{\isacharunderscore}def{\isacharparenright}\isanewline
\isacommand{next}\isamarkupfalse%
\isanewline
\ \ \isacommand{case}\isamarkupfalse%
\ True\ \isacommand{then}\isamarkupfalse%
\ \isacommand{have}\isamarkupfalse%
\ {\isachardoublequoteopen}sqrt\ n\ {\isachargreater}\ {\isadigit{0}}{\isachardoublequoteclose}\ \isacommand{by}\isamarkupfalse%
\ simp\isanewline
\ \ \isacommand{then}\isamarkupfalse%
\ \isacommand{have}\isamarkupfalse%
\ {\isachardoublequoteopen}mono\ {\isacharparenleft}times\ {\isacharparenleft}Max\ {\isacharbraceleft}m{\isachardot}\ m\isactrlsup {\isadigit{2}}\ {\isasymle}\ n{\isacharbraceright}{\isacharparenright}{\isacharparenright}{\isachardoublequoteclose}\ \isacommand{by}\isamarkupfalse%
\ {\isacharparenleft}auto\ intro{\isacharcolon}\ mono{\isacharunderscore}times{\isacharunderscore}nat\ simp\ add{\isacharcolon}\ sqrt{\isacharunderscore}def{\isacharparenright}\isanewline
\ \ \isacommand{then}\isamarkupfalse%
\ \isacommand{have}\isamarkupfalse%
\ {\isacharasterisk}{\isacharcolon}\ {\isachardoublequoteopen}Max\ {\isacharbraceleft}m{\isachardot}\ m\isactrlsup {\isadigit{2}}\ {\isasymle}\ n{\isacharbraceright}\ {\isacharasterisk}\ Max\ {\isacharbraceleft}m{\isachardot}\ m\isactrlsup {\isadigit{2}}\ {\isasymle}\ n{\isacharbraceright}\ {\isacharequal}\ Max\ {\isacharparenleft}times\ {\isacharparenleft}Max\ {\isacharbraceleft}m{\isachardot}\ m\isactrlsup {\isadigit{2}}\ {\isasymle}\ n{\isacharbraceright}{\isacharparenright}\ {\isacharbackquote}\ {\isacharbraceleft}m{\isachardot}\ m\isactrlsup {\isadigit{2}}\ {\isasymle}\ n{\isacharbraceright}{\isacharparenright}{\isachardoublequoteclose}\isanewline
\ \ \ \ \isacommand{using}\isamarkupfalse%
\ sqrt{\isacharunderscore}aux\ {\isacharbrackleft}of\ n{\isacharbrackright}\ \isacommand{by}\isamarkupfalse%
\ {\isacharparenleft}rule\ mono{\isacharunderscore}Max{\isacharunderscore}commute{\isacharparenright}\isanewline
\ \ \isacommand{have}\isamarkupfalse%
\ {\isachardoublequoteopen}Max\ {\isacharparenleft}op\ {\isacharasterisk}\ {\isacharparenleft}Max\ {\isacharbraceleft}m{\isachardot}\ m\ {\isacharasterisk}\ m\ {\isasymle}\ n{\isacharbraceright}{\isacharparenright}\ {\isacharbackquote}\ {\isacharbraceleft}m{\isachardot}\ m\ {\isacharasterisk}\ m\ {\isasymle}\ n{\isacharbraceright}{\isacharparenright}\ {\isasymle}\ n{\isachardoublequoteclose}\isanewline
\ \ \ \ \isacommand{apply}\isamarkupfalse%
\ {\isacharparenleft}subst\ Max{\isacharunderscore}le{\isacharunderscore}iff{\isacharparenright}\isanewline
\ \ \ \ \isacommand{apply}\isamarkupfalse%
\ {\isacharparenleft}metis\ {\isacharparenleft}mono{\isacharunderscore}tags{\isacharparenright}\ finite{\isacharunderscore}imageI\ finite{\isacharunderscore}less{\isacharunderscore}ub\ le{\isacharunderscore}square{\isacharparenright}\isanewline
\ \ \ \ \isacommand{apply}\isamarkupfalse%
\ simp\isanewline
\ \ \ \ \isacommand{apply}\isamarkupfalse%
\ {\isacharparenleft}metis\ le{\isadigit{0}}\ mult{\isacharunderscore}{\isadigit{0}}{\isacharunderscore}right{\isacharparenright}\isanewline
\ \ \ \ \isacommand{apply}\isamarkupfalse%
\ auto\isanewline
\ \ \ \ \isacommand{proof}\isamarkupfalse%
\ {\isacharminus}\isanewline
\ \ \ \ \ \ \isacommand{fix}\isamarkupfalse%
\ q\isanewline
\ \ \ \ \ \ \isacommand{assume}\isamarkupfalse%
\ {\isachardoublequoteopen}q\ {\isacharasterisk}\ q\ {\isasymle}\ n{\isachardoublequoteclose}\isanewline
\ \ \ \ \ \ \isacommand{show}\isamarkupfalse%
\ {\isachardoublequoteopen}Max\ {\isacharbraceleft}m{\isachardot}\ m\ {\isacharasterisk}\ m\ {\isasymle}\ n{\isacharbraceright}\ {\isacharasterisk}\ q\ {\isasymle}\ n{\isachardoublequoteclose}\isanewline
\ \ \ \ \ \ \isacommand{proof}\isamarkupfalse%
\ {\isacharparenleft}cases\ {\isachardoublequoteopen}q\ {\isachargreater}\ {\isadigit{0}}{\isachardoublequoteclose}{\isacharparenright}\isanewline
\ \ \ \ \ \ \ \ \isacommand{case}\isamarkupfalse%
\ False\ \isacommand{then}\isamarkupfalse%
\ \isacommand{show}\isamarkupfalse%
\ {\isacharquery}thesis\ \isacommand{by}\isamarkupfalse%
\ simp\isanewline
\ \ \ \ \ \ \isacommand{next}\isamarkupfalse%
\isanewline
\ \ \ \ \ \ \ \ \isacommand{case}\isamarkupfalse%
\ True\ \isacommand{then}\isamarkupfalse%
\ \isacommand{have}\isamarkupfalse%
\ {\isachardoublequoteopen}mono\ {\isacharparenleft}times\ q{\isacharparenright}{\isachardoublequoteclose}\ \isacommand{by}\isamarkupfalse%
\ {\isacharparenleft}rule\ mono{\isacharunderscore}times{\isacharunderscore}nat{\isacharparenright}\isanewline
\ \ \ \ \ \ \ \ \isacommand{then}\isamarkupfalse%
\ \isacommand{have}\isamarkupfalse%
\ {\isachardoublequoteopen}q\ {\isacharasterisk}\ Max\ {\isacharbraceleft}m{\isachardot}\ m\ {\isacharasterisk}\ m\ {\isasymle}\ n{\isacharbraceright}\ {\isacharequal}\ Max\ {\isacharparenleft}times\ q\ {\isacharbackquote}\ {\isacharbraceleft}m{\isachardot}\ m\ {\isacharasterisk}\ m\ {\isasymle}\ n{\isacharbraceright}{\isacharparenright}{\isachardoublequoteclose}\isanewline
\ \ \ \ \ \ \ \ \ \ \isacommand{using}\isamarkupfalse%
\ sqrt{\isacharunderscore}aux\ {\isacharbrackleft}of\ n{\isacharbrackright}\ \isacommand{by}\isamarkupfalse%
\ {\isacharparenleft}auto\ simp\ add{\isacharcolon}\ power{\isadigit{2}}{\isacharunderscore}eq{\isacharunderscore}square\ intro{\isacharcolon}\ mono{\isacharunderscore}Max{\isacharunderscore}commute{\isacharparenright}\isanewline
\ \ \ \ \ \ \ \ \isacommand{then}\isamarkupfalse%
\ \isacommand{have}\isamarkupfalse%
\ {\isachardoublequoteopen}Max\ {\isacharbraceleft}m{\isachardot}\ m\ {\isacharasterisk}\ m\ {\isasymle}\ n{\isacharbraceright}\ {\isacharasterisk}\ q\ {\isacharequal}\ Max\ {\isacharparenleft}times\ q\ {\isacharbackquote}\ {\isacharbraceleft}m{\isachardot}\ m\ {\isacharasterisk}\ m\ {\isasymle}\ n{\isacharbraceright}{\isacharparenright}{\isachardoublequoteclose}\ \isacommand{by}\isamarkupfalse%
\ {\isacharparenleft}simp\ add{\isacharcolon}\ ac{\isacharunderscore}simps{\isacharparenright}\isanewline
\ \ \ \ \ \ \ \ \isacommand{then}\isamarkupfalse%
\ \isacommand{show}\isamarkupfalse%
\ {\isacharquery}thesis\ \isacommand{apply}\isamarkupfalse%
\ simp\isanewline
\ \ \ \ \ \ \ \ \ \ \isacommand{apply}\isamarkupfalse%
\ {\isacharparenleft}subst\ Max{\isacharunderscore}le{\isacharunderscore}iff{\isacharparenright}\isanewline
\ \ \ \ \ \ \ \ \ \ \isacommand{apply}\isamarkupfalse%
\ auto\isanewline
\ \ \ \ \ \ \ \ \ \ \isacommand{apply}\isamarkupfalse%
\ {\isacharparenleft}metis\ {\isacharparenleft}mono{\isacharunderscore}tags{\isacharparenright}\ finite{\isacharunderscore}imageI\ finite{\isacharunderscore}less{\isacharunderscore}ub\ le{\isacharunderscore}square{\isacharparenright}\isanewline
\ \ \ \ \ \ \ \ \ \ \isacommand{apply}\isamarkupfalse%
\ {\isacharparenleft}metis\ {\isacharbackquoteopen}q\ {\isacharasterisk}\ q\ {\isasymle}\ n{\isacharbackquoteclose}{\isacharparenright}\isanewline
\ \ \ \ \ \ \ \ \ \ \isacommand{using}\isamarkupfalse%
\ {\isacharbackquoteopen}q\ {\isacharasterisk}\ q\ {\isasymle}\ n{\isacharbackquoteclose}\ \isacommand{by}\isamarkupfalse%
\ {\isacharparenleft}metis\ le{\isacharunderscore}cases\ mult{\isacharunderscore}le{\isacharunderscore}mono{\isadigit{1}}\ mult{\isacharunderscore}le{\isacharunderscore}mono{\isadigit{2}}\ order{\isacharunderscore}trans{\isacharparenright}\isanewline
\ \ \ \ \ \ \isacommand{qed}\isamarkupfalse%
\isanewline
\ \ \ \ \isacommand{qed}\isamarkupfalse%
\isanewline
\ \ \isacommand{with}\isamarkupfalse%
\ {\isacharasterisk}\ \isacommand{show}\isamarkupfalse%
\ {\isacharquery}thesis\ \isacommand{by}\isamarkupfalse%
\ {\isacharparenleft}simp\ add{\isacharcolon}\ sqrt{\isacharunderscore}def\ power{\isadigit{2}}{\isacharunderscore}eq{\isacharunderscore}square{\isacharparenright}\isanewline
\isacommand{qed}\isamarkupfalse%
%
\endisatagproof
{\isafoldproof}%
%
\isadelimproof
\isanewline
%
\endisadelimproof
\isanewline
\isacommand{lemma}\isamarkupfalse%
\ sqrt{\isacharunderscore}le{\isacharcolon}\isanewline
\ \ {\isachardoublequoteopen}sqrt\ n\ {\isasymle}\ n{\isachardoublequoteclose}\isanewline
%
\isadelimproof
\ \ %
\endisadelimproof
%
\isatagproof
\isacommand{using}\isamarkupfalse%
\ sqrt{\isacharunderscore}aux\ {\isacharbrackleft}of\ n{\isacharbrackright}\ \isacommand{by}\isamarkupfalse%
\ {\isacharparenleft}auto\ simp\ add{\isacharcolon}\ sqrt{\isacharunderscore}def\ intro{\isacharcolon}\ power{\isadigit{2}}{\isacharunderscore}nat{\isacharunderscore}le{\isacharunderscore}imp{\isacharunderscore}le{\isacharparenright}%
\endisatagproof
{\isafoldproof}%
%
\isadelimproof
\isanewline
%
\endisadelimproof
\isanewline
\isacommand{hide{\isacharunderscore}const}\isamarkupfalse%
\ {\isacharparenleft}\isakeyword{open}{\isacharparenright}\ log\ sqrt\isanewline
%
\isadelimtheory
\isanewline
%
\endisadelimtheory
%
\isatagtheory
\isacommand{end}\isamarkupfalse%
%
\endisatagtheory
{\isafoldtheory}%
%
\isadelimtheory
%
\endisadelimtheory
\end{isabellebody}%
%%% Local Variables:
%%% mode: latex
%%% TeX-master: "root"
%%% End:


%
\begin{isabellebody}%
\def\isabellecontext{Indicator{\isacharunderscore}Function}%
%
\isamarkupheader{Indicator Function%
}
\isamarkuptrue%
%
\isadelimtheory
%
\endisadelimtheory
%
\isatagtheory
\isacommand{theory}\isamarkupfalse%
\ Indicator{\isacharunderscore}Function\isanewline
\isakeyword{imports}\ Complex{\isacharunderscore}Main\isanewline
\isakeyword{begin}%
\endisatagtheory
{\isafoldtheory}%
%
\isadelimtheory
%
\endisadelimtheory
\isanewline
\isanewline
\isacommand{definition}\isamarkupfalse%
\ {\isachardoublequoteopen}indicator\ S\ x\ {\isacharequal}\ {\isacharparenleft}if\ x\ {\isasymin}\ S\ then\ {\isadigit{1}}\ else\ {\isadigit{0}}{\isacharparenright}{\isachardoublequoteclose}\isanewline
\isanewline
\isacommand{lemma}\isamarkupfalse%
\ indicator{\isacharunderscore}simps{\isacharbrackleft}simp{\isacharbrackright}{\isacharcolon}\isanewline
\ \ {\isachardoublequoteopen}x\ {\isasymin}\ S\ {\isasymLongrightarrow}\ indicator\ S\ x\ {\isacharequal}\ {\isadigit{1}}{\isachardoublequoteclose}\isanewline
\ \ {\isachardoublequoteopen}x\ {\isasymnotin}\ S\ {\isasymLongrightarrow}\ indicator\ S\ x\ {\isacharequal}\ {\isadigit{0}}{\isachardoublequoteclose}\isanewline
%
\isadelimproof
\ \ %
\endisadelimproof
%
\isatagproof
\isacommand{unfolding}\isamarkupfalse%
\ indicator{\isacharunderscore}def\ \isacommand{by}\isamarkupfalse%
\ auto%
\endisatagproof
{\isafoldproof}%
%
\isadelimproof
\isanewline
%
\endisadelimproof
\isanewline
\isacommand{lemma}\isamarkupfalse%
\ indicator{\isacharunderscore}pos{\isacharunderscore}le{\isacharbrackleft}intro{\isacharcomma}\ simp{\isacharbrackright}{\isacharcolon}\ {\isachardoublequoteopen}{\isacharparenleft}{\isadigit{0}}{\isacharcolon}{\isacharcolon}{\isacharprime}a{\isacharcolon}{\isacharcolon}linordered{\isacharunderscore}semidom{\isacharparenright}\ {\isasymle}\ indicator\ S\ x{\isachardoublequoteclose}\isanewline
\ \ \isakeyword{and}\ indicator{\isacharunderscore}le{\isacharunderscore}{\isadigit{1}}{\isacharbrackleft}intro{\isacharcomma}\ simp{\isacharbrackright}{\isacharcolon}\ {\isachardoublequoteopen}indicator\ S\ x\ {\isasymle}\ {\isacharparenleft}{\isadigit{1}}{\isacharcolon}{\isacharcolon}{\isacharprime}a{\isacharcolon}{\isacharcolon}linordered{\isacharunderscore}semidom{\isacharparenright}{\isachardoublequoteclose}\isanewline
%
\isadelimproof
\ \ %
\endisadelimproof
%
\isatagproof
\isacommand{unfolding}\isamarkupfalse%
\ indicator{\isacharunderscore}def\ \isacommand{by}\isamarkupfalse%
\ auto%
\endisatagproof
{\isafoldproof}%
%
\isadelimproof
\isanewline
%
\endisadelimproof
\isanewline
\isacommand{lemma}\isamarkupfalse%
\ indicator{\isacharunderscore}abs{\isacharunderscore}le{\isacharunderscore}{\isadigit{1}}{\isacharcolon}\ {\isachardoublequoteopen}{\isasymbar}indicator\ S\ x{\isasymbar}\ {\isasymle}\ {\isacharparenleft}{\isadigit{1}}{\isacharcolon}{\isacharcolon}{\isacharprime}a{\isacharcolon}{\isacharcolon}linordered{\isacharunderscore}idom{\isacharparenright}{\isachardoublequoteclose}\isanewline
%
\isadelimproof
\ \ %
\endisadelimproof
%
\isatagproof
\isacommand{unfolding}\isamarkupfalse%
\ indicator{\isacharunderscore}def\ \isacommand{by}\isamarkupfalse%
\ auto%
\endisatagproof
{\isafoldproof}%
%
\isadelimproof
\isanewline
%
\endisadelimproof
\isanewline
\isacommand{lemma}\isamarkupfalse%
\ indicator{\isacharunderscore}eq{\isacharunderscore}{\isadigit{0}}{\isacharunderscore}iff{\isacharcolon}\ {\isachardoublequoteopen}indicator\ A\ x\ {\isacharequal}\ {\isacharparenleft}{\isadigit{0}}{\isacharcolon}{\isacharcolon}{\isacharunderscore}{\isacharcolon}{\isacharcolon}zero{\isacharunderscore}neq{\isacharunderscore}one{\isacharparenright}\ {\isasymlongleftrightarrow}\ x\ {\isasymnotin}\ A{\isachardoublequoteclose}\isanewline
%
\isadelimproof
\ \ %
\endisadelimproof
%
\isatagproof
\isacommand{by}\isamarkupfalse%
\ {\isacharparenleft}auto\ simp{\isacharcolon}\ indicator{\isacharunderscore}def{\isacharparenright}%
\endisatagproof
{\isafoldproof}%
%
\isadelimproof
\isanewline
%
\endisadelimproof
\isanewline
\isacommand{lemma}\isamarkupfalse%
\ indicator{\isacharunderscore}eq{\isacharunderscore}{\isadigit{1}}{\isacharunderscore}iff{\isacharcolon}\ {\isachardoublequoteopen}indicator\ A\ x\ {\isacharequal}\ {\isacharparenleft}{\isadigit{1}}{\isacharcolon}{\isacharcolon}{\isacharunderscore}{\isacharcolon}{\isacharcolon}zero{\isacharunderscore}neq{\isacharunderscore}one{\isacharparenright}\ {\isasymlongleftrightarrow}\ x\ {\isasymin}\ A{\isachardoublequoteclose}\isanewline
%
\isadelimproof
\ \ %
\endisadelimproof
%
\isatagproof
\isacommand{by}\isamarkupfalse%
\ {\isacharparenleft}auto\ simp{\isacharcolon}\ indicator{\isacharunderscore}def{\isacharparenright}%
\endisatagproof
{\isafoldproof}%
%
\isadelimproof
\isanewline
%
\endisadelimproof
\isanewline
\isacommand{lemma}\isamarkupfalse%
\ split{\isacharunderscore}indicator{\isacharcolon}\ {\isachardoublequoteopen}P\ {\isacharparenleft}indicator\ S\ x{\isacharparenright}\ {\isasymlongleftrightarrow}\ {\isacharparenleft}{\isacharparenleft}x\ {\isasymin}\ S\ {\isasymlongrightarrow}\ P\ {\isadigit{1}}{\isacharparenright}\ {\isasymand}\ {\isacharparenleft}x\ {\isasymnotin}\ S\ {\isasymlongrightarrow}\ P\ {\isadigit{0}}{\isacharparenright}{\isacharparenright}{\isachardoublequoteclose}\isanewline
%
\isadelimproof
\ \ %
\endisadelimproof
%
\isatagproof
\isacommand{unfolding}\isamarkupfalse%
\ indicator{\isacharunderscore}def\ \isacommand{by}\isamarkupfalse%
\ auto%
\endisatagproof
{\isafoldproof}%
%
\isadelimproof
\isanewline
%
\endisadelimproof
\isanewline
\isacommand{lemma}\isamarkupfalse%
\ split{\isacharunderscore}indicator{\isacharunderscore}asm{\isacharcolon}\ {\isachardoublequoteopen}P\ {\isacharparenleft}indicator\ S\ x{\isacharparenright}\ {\isasymlongleftrightarrow}\ {\isacharparenleft}{\isasymnot}\ {\isacharparenleft}x\ {\isasymin}\ S\ {\isasymand}\ {\isasymnot}\ P\ {\isadigit{1}}\ {\isasymor}\ x\ {\isasymnotin}\ S\ {\isasymand}\ {\isasymnot}\ P\ {\isadigit{0}}{\isacharparenright}{\isacharparenright}{\isachardoublequoteclose}\isanewline
%
\isadelimproof
\ \ %
\endisadelimproof
%
\isatagproof
\isacommand{unfolding}\isamarkupfalse%
\ indicator{\isacharunderscore}def\ \isacommand{by}\isamarkupfalse%
\ auto%
\endisatagproof
{\isafoldproof}%
%
\isadelimproof
\isanewline
%
\endisadelimproof
\isanewline
\isacommand{lemma}\isamarkupfalse%
\ indicator{\isacharunderscore}inter{\isacharunderscore}arith{\isacharcolon}\ {\isachardoublequoteopen}indicator\ {\isacharparenleft}A\ {\isasyminter}\ B{\isacharparenright}\ x\ {\isacharequal}\ indicator\ A\ x\ {\isacharasterisk}\ {\isacharparenleft}indicator\ B\ x{\isacharcolon}{\isacharcolon}{\isacharprime}a{\isacharcolon}{\isacharcolon}semiring{\isacharunderscore}{\isadigit{1}}{\isacharparenright}{\isachardoublequoteclose}\isanewline
%
\isadelimproof
\ \ %
\endisadelimproof
%
\isatagproof
\isacommand{unfolding}\isamarkupfalse%
\ indicator{\isacharunderscore}def\ \isacommand{by}\isamarkupfalse%
\ {\isacharparenleft}auto\ simp{\isacharcolon}\ min{\isacharunderscore}def\ max{\isacharunderscore}def{\isacharparenright}%
\endisatagproof
{\isafoldproof}%
%
\isadelimproof
\isanewline
%
\endisadelimproof
\isanewline
\isacommand{lemma}\isamarkupfalse%
\ indicator{\isacharunderscore}union{\isacharunderscore}arith{\isacharcolon}\ {\isachardoublequoteopen}indicator\ {\isacharparenleft}A\ {\isasymunion}\ B{\isacharparenright}\ x\ {\isacharequal}\ indicator\ A\ x\ {\isacharplus}\ indicator\ B\ x\ {\isacharminus}\ indicator\ A\ x\ {\isacharasterisk}\ {\isacharparenleft}indicator\ B\ x{\isacharcolon}{\isacharcolon}{\isacharprime}a{\isacharcolon}{\isacharcolon}ring{\isacharunderscore}{\isadigit{1}}{\isacharparenright}{\isachardoublequoteclose}\isanewline
%
\isadelimproof
\ \ %
\endisadelimproof
%
\isatagproof
\isacommand{unfolding}\isamarkupfalse%
\ indicator{\isacharunderscore}def\ \isacommand{by}\isamarkupfalse%
\ {\isacharparenleft}auto\ simp{\isacharcolon}\ min{\isacharunderscore}def\ max{\isacharunderscore}def{\isacharparenright}%
\endisatagproof
{\isafoldproof}%
%
\isadelimproof
\isanewline
%
\endisadelimproof
\isanewline
\isacommand{lemma}\isamarkupfalse%
\ indicator{\isacharunderscore}inter{\isacharunderscore}min{\isacharcolon}\ {\isachardoublequoteopen}indicator\ {\isacharparenleft}A\ {\isasyminter}\ B{\isacharparenright}\ x\ {\isacharequal}\ min\ {\isacharparenleft}indicator\ A\ x{\isacharparenright}\ {\isacharparenleft}indicator\ B\ x{\isacharcolon}{\isacharcolon}{\isacharprime}a{\isacharcolon}{\isacharcolon}linordered{\isacharunderscore}semidom{\isacharparenright}{\isachardoublequoteclose}\isanewline
\ \ \isakeyword{and}\ indicator{\isacharunderscore}union{\isacharunderscore}max{\isacharcolon}\ {\isachardoublequoteopen}indicator\ {\isacharparenleft}A\ {\isasymunion}\ B{\isacharparenright}\ x\ {\isacharequal}\ max\ {\isacharparenleft}indicator\ A\ x{\isacharparenright}\ {\isacharparenleft}indicator\ B\ x{\isacharcolon}{\isacharcolon}{\isacharprime}a{\isacharcolon}{\isacharcolon}linordered{\isacharunderscore}semidom{\isacharparenright}{\isachardoublequoteclose}\isanewline
%
\isadelimproof
\ \ %
\endisadelimproof
%
\isatagproof
\isacommand{unfolding}\isamarkupfalse%
\ indicator{\isacharunderscore}def\ \isacommand{by}\isamarkupfalse%
\ {\isacharparenleft}auto\ simp{\isacharcolon}\ min{\isacharunderscore}def\ max{\isacharunderscore}def{\isacharparenright}%
\endisatagproof
{\isafoldproof}%
%
\isadelimproof
\isanewline
%
\endisadelimproof
\isanewline
\isacommand{lemma}\isamarkupfalse%
\ indicator{\isacharunderscore}disj{\isacharunderscore}union{\isacharcolon}\ {\isachardoublequoteopen}A\ {\isasyminter}\ B\ {\isacharequal}\ {\isacharbraceleft}{\isacharbraceright}\ {\isasymLongrightarrow}\ \ indicator\ {\isacharparenleft}A\ {\isasymunion}\ B{\isacharparenright}\ x\ {\isacharequal}\ {\isacharparenleft}indicator\ A\ x\ {\isacharplus}\ indicator\ B\ x{\isacharcolon}{\isacharcolon}{\isacharprime}a{\isacharcolon}{\isacharcolon}linordered{\isacharunderscore}semidom{\isacharparenright}{\isachardoublequoteclose}\isanewline
%
\isadelimproof
\ \ %
\endisadelimproof
%
\isatagproof
\isacommand{by}\isamarkupfalse%
\ {\isacharparenleft}auto\ split{\isacharcolon}\ split{\isacharunderscore}indicator{\isacharparenright}%
\endisatagproof
{\isafoldproof}%
%
\isadelimproof
\isanewline
%
\endisadelimproof
\isanewline
\isacommand{lemma}\isamarkupfalse%
\ indicator{\isacharunderscore}compl{\isacharcolon}\ {\isachardoublequoteopen}indicator\ {\isacharparenleft}{\isacharminus}\ A{\isacharparenright}\ x\ {\isacharequal}\ {\isadigit{1}}\ {\isacharminus}\ {\isacharparenleft}indicator\ A\ x{\isacharcolon}{\isacharcolon}{\isacharprime}a{\isacharcolon}{\isacharcolon}ring{\isacharunderscore}{\isadigit{1}}{\isacharparenright}{\isachardoublequoteclose}\isanewline
\ \ \isakeyword{and}\ indicator{\isacharunderscore}diff{\isacharcolon}\ {\isachardoublequoteopen}indicator\ {\isacharparenleft}A\ {\isacharminus}\ B{\isacharparenright}\ x\ {\isacharequal}\ indicator\ A\ x\ {\isacharasterisk}\ {\isacharparenleft}{\isadigit{1}}\ {\isacharminus}\ indicator\ B\ x{\isacharcolon}{\isacharcolon}{\isacharprime}a{\isacharcolon}{\isacharcolon}ring{\isacharunderscore}{\isadigit{1}}{\isacharparenright}{\isachardoublequoteclose}\isanewline
%
\isadelimproof
\ \ %
\endisadelimproof
%
\isatagproof
\isacommand{unfolding}\isamarkupfalse%
\ indicator{\isacharunderscore}def\ \isacommand{by}\isamarkupfalse%
\ {\isacharparenleft}auto\ simp{\isacharcolon}\ min{\isacharunderscore}def\ max{\isacharunderscore}def{\isacharparenright}%
\endisatagproof
{\isafoldproof}%
%
\isadelimproof
\isanewline
%
\endisadelimproof
\isanewline
\isacommand{lemma}\isamarkupfalse%
\ indicator{\isacharunderscore}times{\isacharcolon}\ {\isachardoublequoteopen}indicator\ {\isacharparenleft}A\ {\isasymtimes}\ B{\isacharparenright}\ x\ {\isacharequal}\ indicator\ A\ {\isacharparenleft}fst\ x{\isacharparenright}\ {\isacharasterisk}\ {\isacharparenleft}indicator\ B\ {\isacharparenleft}snd\ x{\isacharparenright}{\isacharcolon}{\isacharcolon}{\isacharprime}a{\isacharcolon}{\isacharcolon}semiring{\isacharunderscore}{\isadigit{1}}{\isacharparenright}{\isachardoublequoteclose}\isanewline
%
\isadelimproof
\ \ %
\endisadelimproof
%
\isatagproof
\isacommand{unfolding}\isamarkupfalse%
\ indicator{\isacharunderscore}def\ \isacommand{by}\isamarkupfalse%
\ {\isacharparenleft}cases\ x{\isacharparenright}\ auto%
\endisatagproof
{\isafoldproof}%
%
\isadelimproof
\isanewline
%
\endisadelimproof
\isanewline
\isacommand{lemma}\isamarkupfalse%
\ indicator{\isacharunderscore}sum{\isacharcolon}\ {\isachardoublequoteopen}indicator\ {\isacharparenleft}A\ {\isacharless}{\isacharplus}{\isachargreater}\ B{\isacharparenright}\ x\ {\isacharequal}\ {\isacharparenleft}case\ x\ of\ Inl\ x\ {\isasymRightarrow}\ indicator\ A\ x\ {\isacharbar}\ Inr\ x\ {\isasymRightarrow}\ indicator\ B\ x{\isacharparenright}{\isachardoublequoteclose}\isanewline
%
\isadelimproof
\ \ %
\endisadelimproof
%
\isatagproof
\isacommand{unfolding}\isamarkupfalse%
\ indicator{\isacharunderscore}def\ \isacommand{by}\isamarkupfalse%
\ {\isacharparenleft}cases\ x{\isacharparenright}\ auto%
\endisatagproof
{\isafoldproof}%
%
\isadelimproof
\isanewline
%
\endisadelimproof
\isanewline
\isacommand{lemma}\isamarkupfalse%
\isanewline
\ \ \isakeyword{fixes}\ f\ {\isacharcolon}{\isacharcolon}\ {\isachardoublequoteopen}{\isacharprime}a\ {\isasymRightarrow}\ {\isacharprime}b{\isacharcolon}{\isacharcolon}semiring{\isacharunderscore}{\isadigit{1}}{\isachardoublequoteclose}\ \isakeyword{assumes}\ {\isachardoublequoteopen}finite\ A{\isachardoublequoteclose}\isanewline
\ \ \isakeyword{shows}\ setsum{\isacharunderscore}mult{\isacharunderscore}indicator{\isacharbrackleft}simp{\isacharbrackright}{\isacharcolon}\ {\isachardoublequoteopen}{\isacharparenleft}{\isasymSum}x\ {\isasymin}\ A{\isachardot}\ f\ x\ {\isacharasterisk}\ indicator\ B\ x{\isacharparenright}\ {\isacharequal}\ {\isacharparenleft}{\isasymSum}x\ {\isasymin}\ A\ {\isasyminter}\ B{\isachardot}\ f\ x{\isacharparenright}{\isachardoublequoteclose}\isanewline
\ \ \isakeyword{and}\ setsum{\isacharunderscore}indicator{\isacharunderscore}mult{\isacharbrackleft}simp{\isacharbrackright}{\isacharcolon}\ {\isachardoublequoteopen}{\isacharparenleft}{\isasymSum}x\ {\isasymin}\ A{\isachardot}\ indicator\ B\ x\ {\isacharasterisk}\ f\ x{\isacharparenright}\ {\isacharequal}\ {\isacharparenleft}{\isasymSum}x\ {\isasymin}\ A\ {\isasyminter}\ B{\isachardot}\ f\ x{\isacharparenright}{\isachardoublequoteclose}\isanewline
%
\isadelimproof
\ \ %
\endisadelimproof
%
\isatagproof
\isacommand{unfolding}\isamarkupfalse%
\ indicator{\isacharunderscore}def\isanewline
\ \ \isacommand{using}\isamarkupfalse%
\ assms\ \isacommand{by}\isamarkupfalse%
\ {\isacharparenleft}auto\ intro{\isacharbang}{\isacharcolon}\ setsum{\isachardot}mono{\isacharunderscore}neutral{\isacharunderscore}cong{\isacharunderscore}right\ split{\isacharcolon}\ split{\isacharunderscore}if{\isacharunderscore}asm{\isacharparenright}%
\endisatagproof
{\isafoldproof}%
%
\isadelimproof
\isanewline
%
\endisadelimproof
\isanewline
\isacommand{lemma}\isamarkupfalse%
\ setsum{\isacharunderscore}indicator{\isacharunderscore}eq{\isacharunderscore}card{\isacharcolon}\isanewline
\ \ \isakeyword{assumes}\ {\isachardoublequoteopen}finite\ A{\isachardoublequoteclose}\isanewline
\ \ \isakeyword{shows}\ {\isachardoublequoteopen}{\isacharparenleft}SUM\ x\ {\isacharcolon}\ A{\isachardot}\ indicator\ B\ x{\isacharparenright}\ {\isacharequal}\ card\ {\isacharparenleft}A\ Int\ B{\isacharparenright}{\isachardoublequoteclose}\isanewline
%
\isadelimproof
\ \ %
\endisadelimproof
%
\isatagproof
\isacommand{using}\isamarkupfalse%
\ setsum{\isacharunderscore}mult{\isacharunderscore}indicator{\isacharbrackleft}OF\ assms{\isacharcomma}\ of\ {\isachardoublequoteopen}{\isacharpercent}x{\isachardot}\ {\isadigit{1}}{\isacharcolon}{\isacharcolon}nat{\isachardoublequoteclose}{\isacharbrackright}\isanewline
\ \ \isacommand{unfolding}\isamarkupfalse%
\ card{\isacharunderscore}eq{\isacharunderscore}setsum\ \isacommand{by}\isamarkupfalse%
\ simp%
\endisatagproof
{\isafoldproof}%
%
\isadelimproof
\isanewline
%
\endisadelimproof
\isanewline
\isacommand{lemma}\isamarkupfalse%
\ setsum{\isacharunderscore}indicator{\isacharunderscore}scaleR{\isacharbrackleft}simp{\isacharbrackright}{\isacharcolon}\isanewline
\ \ {\isachardoublequoteopen}finite\ A\ {\isasymLongrightarrow}\isanewline
\ \ \ \ {\isacharparenleft}{\isasymSum}x\ {\isasymin}\ A{\isachardot}\ indicator\ {\isacharparenleft}B\ x{\isacharparenright}\ {\isacharparenleft}g\ x{\isacharparenright}\ {\isacharasterisk}\isactrlsub R\ f\ x{\isacharparenright}\ {\isacharequal}\ {\isacharparenleft}{\isasymSum}x\ {\isasymin}\ {\isacharbraceleft}x{\isasymin}A{\isachardot}\ g\ x\ {\isasymin}\ B\ x{\isacharbraceright}{\isachardot}\ f\ x{\isacharcolon}{\isacharcolon}{\isacharprime}a{\isacharcolon}{\isacharcolon}real{\isacharunderscore}vector{\isacharparenright}{\isachardoublequoteclose}\isanewline
%
\isadelimproof
\ \ %
\endisadelimproof
%
\isatagproof
\isacommand{using}\isamarkupfalse%
\ assms\ \isacommand{by}\isamarkupfalse%
\ {\isacharparenleft}auto\ intro{\isacharbang}{\isacharcolon}\ setsum{\isachardot}mono{\isacharunderscore}neutral{\isacharunderscore}cong{\isacharunderscore}right\ split{\isacharcolon}\ split{\isacharunderscore}if{\isacharunderscore}asm\ simp{\isacharcolon}\ indicator{\isacharunderscore}def{\isacharparenright}%
\endisatagproof
{\isafoldproof}%
%
\isadelimproof
\isanewline
%
\endisadelimproof
\isanewline
\isacommand{lemma}\isamarkupfalse%
\ LIMSEQ{\isacharunderscore}indicator{\isacharunderscore}incseq{\isacharcolon}\isanewline
\ \ \isakeyword{assumes}\ {\isachardoublequoteopen}incseq\ A{\isachardoublequoteclose}\isanewline
\ \ \isakeyword{shows}\ {\isachardoublequoteopen}{\isacharparenleft}{\isasymlambda}i{\isachardot}\ indicator\ {\isacharparenleft}A\ i{\isacharparenright}\ x\ {\isacharcolon}{\isacharcolon}\ {\isacharprime}a\ {\isacharcolon}{\isacharcolon}\ {\isacharbraceleft}topological{\isacharunderscore}space{\isacharcomma}\ one{\isacharcomma}\ zero{\isacharbraceright}{\isacharparenright}\ {\isacharminus}{\isacharminus}{\isacharminus}{\isacharminus}{\isachargreater}\ indicator\ {\isacharparenleft}{\isasymUnion}i{\isachardot}\ A\ i{\isacharparenright}\ x{\isachardoublequoteclose}\isanewline
%
\isadelimproof
%
\endisadelimproof
%
\isatagproof
\isacommand{proof}\isamarkupfalse%
\ cases\isanewline
\ \ \isacommand{assume}\isamarkupfalse%
\ {\isachardoublequoteopen}{\isasymexists}i{\isachardot}\ x\ {\isasymin}\ A\ i{\isachardoublequoteclose}\isanewline
\ \ \isacommand{then}\isamarkupfalse%
\ \isacommand{obtain}\isamarkupfalse%
\ i\ \isakeyword{where}\ {\isachardoublequoteopen}x\ {\isasymin}\ A\ i{\isachardoublequoteclose}\isanewline
\ \ \ \ \isacommand{by}\isamarkupfalse%
\ auto\isanewline
\ \ \isacommand{then}\isamarkupfalse%
\ \isacommand{have}\isamarkupfalse%
\ \isanewline
\ \ \ \ {\isachardoublequoteopen}{\isasymAnd}n{\isachardot}\ {\isacharparenleft}indicator\ {\isacharparenleft}A\ {\isacharparenleft}n\ {\isacharplus}\ i{\isacharparenright}{\isacharparenright}\ x\ {\isacharcolon}{\isacharcolon}\ {\isacharprime}a{\isacharparenright}\ {\isacharequal}\ {\isadigit{1}}{\isachardoublequoteclose}\isanewline
\ \ \ \ {\isachardoublequoteopen}{\isacharparenleft}indicator\ {\isacharparenleft}{\isasymUnion}\ i{\isachardot}\ A\ i{\isacharparenright}\ x\ {\isacharcolon}{\isacharcolon}\ {\isacharprime}a{\isacharparenright}\ {\isacharequal}\ {\isadigit{1}}{\isachardoublequoteclose}\isanewline
\ \ \ \ \isacommand{using}\isamarkupfalse%
\ incseqD{\isacharbrackleft}OF\ {\isacharbackquoteopen}incseq\ A{\isacharbackquoteclose}{\isacharcomma}\ of\ i\ {\isachardoublequoteopen}n\ {\isacharplus}\ i{\isachardoublequoteclose}\ \isakeyword{for}\ n{\isacharbrackright}\ {\isacharbackquoteopen}x\ {\isasymin}\ A\ i{\isacharbackquoteclose}\ \isacommand{by}\isamarkupfalse%
\ {\isacharparenleft}auto\ simp{\isacharcolon}\ indicator{\isacharunderscore}def{\isacharparenright}\isanewline
\ \ \isacommand{then}\isamarkupfalse%
\ \isacommand{show}\isamarkupfalse%
\ {\isacharquery}thesis\isanewline
\ \ \ \ \isacommand{by}\isamarkupfalse%
\ {\isacharparenleft}rule{\isacharunderscore}tac\ LIMSEQ{\isacharunderscore}offset{\isacharbrackleft}of\ {\isacharunderscore}\ i{\isacharbrackright}{\isacharparenright}\ {\isacharparenleft}simp\ add{\isacharcolon}\ tendsto{\isacharunderscore}const{\isacharparenright}\isanewline
\isacommand{qed}\isamarkupfalse%
\ {\isacharparenleft}auto\ simp{\isacharcolon}\ indicator{\isacharunderscore}def\ tendsto{\isacharunderscore}const{\isacharparenright}%
\endisatagproof
{\isafoldproof}%
%
\isadelimproof
\isanewline
%
\endisadelimproof
\isanewline
\isacommand{lemma}\isamarkupfalse%
\ LIMSEQ{\isacharunderscore}indicator{\isacharunderscore}UN{\isacharcolon}\isanewline
\ \ {\isachardoublequoteopen}{\isacharparenleft}{\isasymlambda}k{\isachardot}\ indicator\ {\isacharparenleft}{\isasymUnion}\ i{\isacharless}k{\isachardot}\ A\ i{\isacharparenright}\ x\ {\isacharcolon}{\isacharcolon}\ {\isacharprime}a\ {\isacharcolon}{\isacharcolon}\ {\isacharbraceleft}topological{\isacharunderscore}space{\isacharcomma}\ one{\isacharcomma}\ zero{\isacharbraceright}{\isacharparenright}\ {\isacharminus}{\isacharminus}{\isacharminus}{\isacharminus}{\isachargreater}\ indicator\ {\isacharparenleft}{\isasymUnion}i{\isachardot}\ A\ i{\isacharparenright}\ x{\isachardoublequoteclose}\isanewline
%
\isadelimproof
%
\endisadelimproof
%
\isatagproof
\isacommand{proof}\isamarkupfalse%
\ {\isacharminus}\isanewline
\ \ \isacommand{have}\isamarkupfalse%
\ {\isachardoublequoteopen}{\isacharparenleft}{\isasymlambda}k{\isachardot}\ indicator\ {\isacharparenleft}{\isasymUnion}\ i{\isacharless}k{\isachardot}\ A\ i{\isacharparenright}\ x{\isacharcolon}{\isacharcolon}{\isacharprime}a{\isacharparenright}\ {\isacharminus}{\isacharminus}{\isacharminus}{\isacharminus}{\isachargreater}\ indicator\ {\isacharparenleft}{\isasymUnion}k{\isachardot}\ {\isasymUnion}\ i{\isacharless}k{\isachardot}\ A\ i{\isacharparenright}\ x{\isachardoublequoteclose}\isanewline
\ \ \ \ \isacommand{by}\isamarkupfalse%
\ {\isacharparenleft}intro\ LIMSEQ{\isacharunderscore}indicator{\isacharunderscore}incseq{\isacharparenright}\ {\isacharparenleft}auto\ simp{\isacharcolon}\ incseq{\isacharunderscore}def\ intro{\isacharcolon}\ less{\isacharunderscore}le{\isacharunderscore}trans{\isacharparenright}\isanewline
\ \ \isacommand{also}\isamarkupfalse%
\ \isacommand{have}\isamarkupfalse%
\ {\isachardoublequoteopen}{\isacharparenleft}{\isasymUnion}k{\isachardot}\ {\isasymUnion}\ i{\isacharless}k{\isachardot}\ A\ i{\isacharparenright}\ {\isacharequal}\ {\isacharparenleft}{\isasymUnion}i{\isachardot}\ A\ i{\isacharparenright}{\isachardoublequoteclose}\isanewline
\ \ \ \ \isacommand{by}\isamarkupfalse%
\ auto\isanewline
\ \ \isacommand{finally}\isamarkupfalse%
\ \isacommand{show}\isamarkupfalse%
\ {\isacharquery}thesis\ \isacommand{{\isachardot}}\isamarkupfalse%
\isanewline
\isacommand{qed}\isamarkupfalse%
%
\endisatagproof
{\isafoldproof}%
%
\isadelimproof
\isanewline
%
\endisadelimproof
\isanewline
\isacommand{lemma}\isamarkupfalse%
\ LIMSEQ{\isacharunderscore}indicator{\isacharunderscore}decseq{\isacharcolon}\isanewline
\ \ \isakeyword{assumes}\ {\isachardoublequoteopen}decseq\ A{\isachardoublequoteclose}\isanewline
\ \ \isakeyword{shows}\ {\isachardoublequoteopen}{\isacharparenleft}{\isasymlambda}i{\isachardot}\ indicator\ {\isacharparenleft}A\ i{\isacharparenright}\ x\ {\isacharcolon}{\isacharcolon}\ {\isacharprime}a\ {\isacharcolon}{\isacharcolon}\ {\isacharbraceleft}topological{\isacharunderscore}space{\isacharcomma}\ one{\isacharcomma}\ zero{\isacharbraceright}{\isacharparenright}\ {\isacharminus}{\isacharminus}{\isacharminus}{\isacharminus}{\isachargreater}\ indicator\ {\isacharparenleft}{\isasymInter}i{\isachardot}\ A\ i{\isacharparenright}\ x{\isachardoublequoteclose}\isanewline
%
\isadelimproof
%
\endisadelimproof
%
\isatagproof
\isacommand{proof}\isamarkupfalse%
\ cases\isanewline
\ \ \isacommand{assume}\isamarkupfalse%
\ {\isachardoublequoteopen}{\isasymexists}i{\isachardot}\ x\ {\isasymnotin}\ A\ i{\isachardoublequoteclose}\isanewline
\ \ \isacommand{then}\isamarkupfalse%
\ \isacommand{obtain}\isamarkupfalse%
\ i\ \isakeyword{where}\ {\isachardoublequoteopen}x\ {\isasymnotin}\ A\ i{\isachardoublequoteclose}\isanewline
\ \ \ \ \isacommand{by}\isamarkupfalse%
\ auto\isanewline
\ \ \isacommand{then}\isamarkupfalse%
\ \isacommand{have}\isamarkupfalse%
\ \isanewline
\ \ \ \ {\isachardoublequoteopen}{\isasymAnd}n{\isachardot}\ {\isacharparenleft}indicator\ {\isacharparenleft}A\ {\isacharparenleft}n\ {\isacharplus}\ i{\isacharparenright}{\isacharparenright}\ x\ {\isacharcolon}{\isacharcolon}\ {\isacharprime}a{\isacharparenright}\ {\isacharequal}\ {\isadigit{0}}{\isachardoublequoteclose}\isanewline
\ \ \ \ {\isachardoublequoteopen}{\isacharparenleft}indicator\ {\isacharparenleft}{\isasymInter}i{\isachardot}\ A\ i{\isacharparenright}\ x\ {\isacharcolon}{\isacharcolon}\ {\isacharprime}a{\isacharparenright}\ {\isacharequal}\ {\isadigit{0}}{\isachardoublequoteclose}\isanewline
\ \ \ \ \isacommand{using}\isamarkupfalse%
\ decseqD{\isacharbrackleft}OF\ {\isacharbackquoteopen}decseq\ A{\isacharbackquoteclose}{\isacharcomma}\ of\ i\ {\isachardoublequoteopen}n\ {\isacharplus}\ i{\isachardoublequoteclose}\ \isakeyword{for}\ n{\isacharbrackright}\ {\isacharbackquoteopen}x\ {\isasymnotin}\ A\ i{\isacharbackquoteclose}\ \isacommand{by}\isamarkupfalse%
\ {\isacharparenleft}auto\ simp{\isacharcolon}\ indicator{\isacharunderscore}def{\isacharparenright}\isanewline
\ \ \isacommand{then}\isamarkupfalse%
\ \isacommand{show}\isamarkupfalse%
\ {\isacharquery}thesis\isanewline
\ \ \ \ \isacommand{by}\isamarkupfalse%
\ {\isacharparenleft}rule{\isacharunderscore}tac\ LIMSEQ{\isacharunderscore}offset{\isacharbrackleft}of\ {\isacharunderscore}\ i{\isacharbrackright}{\isacharparenright}\ {\isacharparenleft}simp\ add{\isacharcolon}\ tendsto{\isacharunderscore}const{\isacharparenright}\isanewline
\isacommand{qed}\isamarkupfalse%
\ {\isacharparenleft}auto\ simp{\isacharcolon}\ indicator{\isacharunderscore}def\ tendsto{\isacharunderscore}const{\isacharparenright}%
\endisatagproof
{\isafoldproof}%
%
\isadelimproof
\isanewline
%
\endisadelimproof
\isanewline
\isacommand{lemma}\isamarkupfalse%
\ LIMSEQ{\isacharunderscore}indicator{\isacharunderscore}INT{\isacharcolon}\isanewline
\ \ {\isachardoublequoteopen}{\isacharparenleft}{\isasymlambda}k{\isachardot}\ indicator\ {\isacharparenleft}{\isasymInter}i{\isacharless}k{\isachardot}\ A\ i{\isacharparenright}\ x\ {\isacharcolon}{\isacharcolon}\ {\isacharprime}a\ {\isacharcolon}{\isacharcolon}\ {\isacharbraceleft}topological{\isacharunderscore}space{\isacharcomma}\ one{\isacharcomma}\ zero{\isacharbraceright}{\isacharparenright}\ {\isacharminus}{\isacharminus}{\isacharminus}{\isacharminus}{\isachargreater}\ indicator\ {\isacharparenleft}{\isasymInter}i{\isachardot}\ A\ i{\isacharparenright}\ x{\isachardoublequoteclose}\isanewline
%
\isadelimproof
%
\endisadelimproof
%
\isatagproof
\isacommand{proof}\isamarkupfalse%
\ {\isacharminus}\isanewline
\ \ \isacommand{have}\isamarkupfalse%
\ {\isachardoublequoteopen}{\isacharparenleft}{\isasymlambda}k{\isachardot}\ indicator\ {\isacharparenleft}{\isasymInter}i{\isacharless}k{\isachardot}\ A\ i{\isacharparenright}\ x{\isacharcolon}{\isacharcolon}{\isacharprime}a{\isacharparenright}\ {\isacharminus}{\isacharminus}{\isacharminus}{\isacharminus}{\isachargreater}\ indicator\ {\isacharparenleft}{\isasymInter}k{\isachardot}\ {\isasymInter}i{\isacharless}k{\isachardot}\ A\ i{\isacharparenright}\ x{\isachardoublequoteclose}\isanewline
\ \ \ \ \isacommand{by}\isamarkupfalse%
\ {\isacharparenleft}intro\ LIMSEQ{\isacharunderscore}indicator{\isacharunderscore}decseq{\isacharparenright}\ {\isacharparenleft}auto\ simp{\isacharcolon}\ decseq{\isacharunderscore}def\ intro{\isacharcolon}\ less{\isacharunderscore}le{\isacharunderscore}trans{\isacharparenright}\isanewline
\ \ \isacommand{also}\isamarkupfalse%
\ \isacommand{have}\isamarkupfalse%
\ {\isachardoublequoteopen}{\isacharparenleft}{\isasymInter}k{\isachardot}\ {\isasymInter}\ i{\isacharless}k{\isachardot}\ A\ i{\isacharparenright}\ {\isacharequal}\ {\isacharparenleft}{\isasymInter}i{\isachardot}\ A\ i{\isacharparenright}{\isachardoublequoteclose}\isanewline
\ \ \ \ \isacommand{by}\isamarkupfalse%
\ auto\isanewline
\ \ \isacommand{finally}\isamarkupfalse%
\ \isacommand{show}\isamarkupfalse%
\ {\isacharquery}thesis\ \isacommand{{\isachardot}}\isamarkupfalse%
\isanewline
\isacommand{qed}\isamarkupfalse%
%
\endisatagproof
{\isafoldproof}%
%
\isadelimproof
\isanewline
%
\endisadelimproof
\isanewline
\isacommand{lemma}\isamarkupfalse%
\ indicator{\isacharunderscore}add{\isacharcolon}\isanewline
\ \ {\isachardoublequoteopen}A\ {\isasyminter}\ B\ {\isacharequal}\ {\isacharbraceleft}{\isacharbraceright}\ {\isasymLongrightarrow}\ {\isacharparenleft}indicator\ A\ x{\isacharcolon}{\isacharcolon}{\isacharunderscore}{\isacharcolon}{\isacharcolon}monoid{\isacharunderscore}add{\isacharparenright}\ {\isacharplus}\ indicator\ B\ x\ {\isacharequal}\ indicator\ {\isacharparenleft}A\ {\isasymunion}\ B{\isacharparenright}\ x{\isachardoublequoteclose}\isanewline
%
\isadelimproof
\ \ %
\endisadelimproof
%
\isatagproof
\isacommand{unfolding}\isamarkupfalse%
\ indicator{\isacharunderscore}def\ \isacommand{by}\isamarkupfalse%
\ auto%
\endisatagproof
{\isafoldproof}%
%
\isadelimproof
\isanewline
%
\endisadelimproof
\isanewline
\isacommand{lemma}\isamarkupfalse%
\ of{\isacharunderscore}real{\isacharunderscore}indicator{\isacharcolon}\ {\isachardoublequoteopen}of{\isacharunderscore}real\ {\isacharparenleft}indicator\ A\ x{\isacharparenright}\ {\isacharequal}\ indicator\ A\ x{\isachardoublequoteclose}\isanewline
%
\isadelimproof
\ \ %
\endisadelimproof
%
\isatagproof
\isacommand{by}\isamarkupfalse%
\ {\isacharparenleft}simp\ split{\isacharcolon}\ split{\isacharunderscore}indicator{\isacharparenright}%
\endisatagproof
{\isafoldproof}%
%
\isadelimproof
\isanewline
%
\endisadelimproof
\isanewline
\isacommand{lemma}\isamarkupfalse%
\ real{\isacharunderscore}of{\isacharunderscore}nat{\isacharunderscore}indicator{\isacharcolon}\ {\isachardoublequoteopen}real\ {\isacharparenleft}indicator\ A\ x\ {\isacharcolon}{\isacharcolon}\ nat{\isacharparenright}\ {\isacharequal}\ indicator\ A\ x{\isachardoublequoteclose}\isanewline
%
\isadelimproof
\ \ %
\endisadelimproof
%
\isatagproof
\isacommand{by}\isamarkupfalse%
\ {\isacharparenleft}simp\ split{\isacharcolon}\ split{\isacharunderscore}indicator{\isacharparenright}%
\endisatagproof
{\isafoldproof}%
%
\isadelimproof
\isanewline
%
\endisadelimproof
\isanewline
\isacommand{lemma}\isamarkupfalse%
\ abs{\isacharunderscore}indicator{\isacharcolon}\ {\isachardoublequoteopen}{\isasymbar}indicator\ A\ x\ {\isacharcolon}{\isacharcolon}\ {\isacharprime}a{\isacharcolon}{\isacharcolon}linordered{\isacharunderscore}idom{\isasymbar}\ {\isacharequal}\ indicator\ A\ x{\isachardoublequoteclose}\isanewline
%
\isadelimproof
\ \ %
\endisadelimproof
%
\isatagproof
\isacommand{by}\isamarkupfalse%
\ {\isacharparenleft}simp\ split{\isacharcolon}\ split{\isacharunderscore}indicator{\isacharparenright}%
\endisatagproof
{\isafoldproof}%
%
\isadelimproof
\isanewline
%
\endisadelimproof
\isanewline
\isacommand{lemma}\isamarkupfalse%
\ mult{\isacharunderscore}indicator{\isacharunderscore}subset{\isacharcolon}\isanewline
\ \ {\isachardoublequoteopen}A\ {\isasymsubseteq}\ B\ {\isasymLongrightarrow}\ indicator\ A\ x\ {\isacharasterisk}\ indicator\ B\ x\ {\isacharequal}\ {\isacharparenleft}indicator\ A\ x\ {\isacharcolon}{\isacharcolon}\ {\isacharprime}a{\isacharcolon}{\isacharcolon}{\isacharbraceleft}comm{\isacharunderscore}semiring{\isacharunderscore}{\isadigit{1}}{\isacharbraceright}{\isacharparenright}{\isachardoublequoteclose}\isanewline
%
\isadelimproof
\ \ %
\endisadelimproof
%
\isatagproof
\isacommand{by}\isamarkupfalse%
\ {\isacharparenleft}auto\ split{\isacharcolon}\ split{\isacharunderscore}indicator\ simp{\isacharcolon}\ fun{\isacharunderscore}eq{\isacharunderscore}iff{\isacharparenright}%
\endisatagproof
{\isafoldproof}%
%
\isadelimproof
\isanewline
%
\endisadelimproof
\isanewline
\isacommand{lemma}\isamarkupfalse%
\ indicator{\isacharunderscore}sums{\isacharcolon}\ \isanewline
\ \ \isakeyword{assumes}\ {\isachardoublequoteopen}{\isasymAnd}i\ j{\isachardot}\ i\ {\isasymnoteq}\ j\ {\isasymLongrightarrow}\ A\ i\ {\isasyminter}\ A\ j\ {\isacharequal}\ {\isacharbraceleft}{\isacharbraceright}{\isachardoublequoteclose}\isanewline
\ \ \isakeyword{shows}\ {\isachardoublequoteopen}{\isacharparenleft}{\isasymlambda}i{\isachardot}\ indicator\ {\isacharparenleft}A\ i{\isacharparenright}\ x{\isacharcolon}{\isacharcolon}real{\isacharparenright}\ sums\ indicator\ {\isacharparenleft}{\isasymUnion}i{\isachardot}\ A\ i{\isacharparenright}\ x{\isachardoublequoteclose}\isanewline
%
\isadelimproof
%
\endisadelimproof
%
\isatagproof
\isacommand{proof}\isamarkupfalse%
\ cases\isanewline
\ \ \isacommand{assume}\isamarkupfalse%
\ {\isachardoublequoteopen}{\isasymexists}i{\isachardot}\ x\ {\isasymin}\ A\ i{\isachardoublequoteclose}\isanewline
\ \ \isacommand{then}\isamarkupfalse%
\ \isacommand{obtain}\isamarkupfalse%
\ i\ \isakeyword{where}\ i{\isacharcolon}\ {\isachardoublequoteopen}x\ {\isasymin}\ A\ i{\isachardoublequoteclose}\ \isacommand{{\isachardot}{\isachardot}}\isamarkupfalse%
\isanewline
\ \ \isacommand{with}\isamarkupfalse%
\ assms\ \isacommand{have}\isamarkupfalse%
\ {\isachardoublequoteopen}{\isacharparenleft}{\isasymlambda}i{\isachardot}\ indicator\ {\isacharparenleft}A\ i{\isacharparenright}\ x{\isacharcolon}{\isacharcolon}real{\isacharparenright}\ sums\ {\isacharparenleft}{\isasymSum}i{\isasymin}{\isacharbraceleft}i{\isacharbraceright}{\isachardot}\ indicator\ {\isacharparenleft}A\ i{\isacharparenright}\ x{\isacharparenright}{\isachardoublequoteclose}\isanewline
\ \ \ \ \isacommand{by}\isamarkupfalse%
\ {\isacharparenleft}intro\ sums{\isacharunderscore}finite{\isacharparenright}\ {\isacharparenleft}auto\ split{\isacharcolon}\ split{\isacharunderscore}indicator{\isacharparenright}\isanewline
\ \ \isacommand{also}\isamarkupfalse%
\ \isacommand{have}\isamarkupfalse%
\ {\isachardoublequoteopen}{\isacharparenleft}{\isasymSum}i{\isasymin}{\isacharbraceleft}i{\isacharbraceright}{\isachardot}\ indicator\ {\isacharparenleft}A\ i{\isacharparenright}\ x{\isacharparenright}\ {\isacharequal}\ indicator\ {\isacharparenleft}{\isasymUnion}i{\isachardot}\ A\ i{\isacharparenright}\ x{\isachardoublequoteclose}\isanewline
\ \ \ \ \isacommand{using}\isamarkupfalse%
\ i\ \isacommand{by}\isamarkupfalse%
\ {\isacharparenleft}auto\ split{\isacharcolon}\ split{\isacharunderscore}indicator{\isacharparenright}\isanewline
\ \ \isacommand{finally}\isamarkupfalse%
\ \isacommand{show}\isamarkupfalse%
\ {\isacharquery}thesis\ \isacommand{{\isachardot}}\isamarkupfalse%
\isanewline
\isacommand{qed}\isamarkupfalse%
\ simp%
\endisatagproof
{\isafoldproof}%
%
\isadelimproof
\isanewline
%
\endisadelimproof
%
\isadelimtheory
\isanewline
%
\endisadelimtheory
%
\isatagtheory
\isacommand{end}\isamarkupfalse%
%
\endisatagtheory
{\isafoldtheory}%
%
\isadelimtheory
%
\endisadelimtheory
\end{isabellebody}%
%%% Local Variables:
%%% mode: latex
%%% TeX-master: "root"
%%% End:


%
\begin{isabellebody}%
\def\isabellecontext{Argmax}%
%
\isamarkupheader{Locus where a function or a list (of linord type) attains its maximum value%
}
\isamarkuptrue%
%
\isadelimtheory
%
\endisadelimtheory
%
\isatagtheory
\isacommand{theory}\isamarkupfalse%
\ Argmax\isanewline
\isakeyword{imports}\ Main\isanewline
\isanewline
\isakeyword{begin}%
\endisatagtheory
{\isafoldtheory}%
%
\isadelimtheory
%
\endisadelimtheory
%
\begin{isamarkuptext}%
the subset of elements of a set where a function reaches its maximum%
\end{isamarkuptext}%
\isamarkuptrue%
\isacommand{fun}\isamarkupfalse%
\ argmax\ {\isacharcolon}{\isacharcolon}\ {\isachardoublequoteopen}{\isacharparenleft}{\isacharprime}a\ {\isasymRightarrow}\ {\isacharprime}b{\isasymColon}linorder{\isacharparenright}\ {\isasymRightarrow}\ {\isacharprime}a\ set\ {\isasymRightarrow}\ {\isacharprime}a\ set{\isachardoublequoteclose}\isanewline
\isakeyword{where}\ {\isachardoublequoteopen}argmax\ f\ A\ {\isacharequal}\ {\isacharbraceleft}\ x\ {\isasymin}\ A\ {\isachardot}\ f\ x\ {\isacharequal}\ Max\ {\isacharparenleft}f\ {\isacharbackquote}\ A{\isacharparenright}\ {\isacharbraceright}{\isachardoublequoteclose}\isanewline
\isanewline
\isacommand{lemma}\isamarkupfalse%
\ mm{\isadigit{7}}{\isadigit{9}}{\isacharcolon}\ {\isachardoublequoteopen}argmax\ f\ A\ {\isacharequal}\ A\ {\isasyminter}\ f\ {\isacharminus}{\isacharbackquote}\ {\isacharbraceleft}Max\ {\isacharparenleft}f\ {\isacharbackquote}\ A{\isacharparenright}{\isacharbraceright}{\isachardoublequoteclose}%
\isadelimproof
\ %
\endisadelimproof
%
\isatagproof
\isacommand{by}\isamarkupfalse%
\ force%
\endisatagproof
{\isafoldproof}%
%
\isadelimproof
%
\endisadelimproof
\isanewline
\isacommand{lemma}\isamarkupfalse%
\ mm{\isadigit{8}}{\isadigit{6}}b{\isacharcolon}\ \isakeyword{assumes}\ {\isachardoublequoteopen}y\ {\isasymin}\ f{\isacharbackquote}A{\isachardoublequoteclose}\ \isakeyword{shows}\ {\isachardoublequoteopen}A\ {\isasyminter}\ f\ {\isacharminus}{\isacharbackquote}\ {\isacharbraceleft}y{\isacharbraceright}\ {\isasymnoteq}\ {\isacharbraceleft}{\isacharbraceright}{\isachardoublequoteclose}%
\isadelimproof
\ %
\endisadelimproof
%
\isatagproof
\isacommand{using}\isamarkupfalse%
\ assms\ \isacommand{by}\isamarkupfalse%
\ blast%
\endisatagproof
{\isafoldproof}%
%
\isadelimproof
%
\endisadelimproof
%
\begin{isamarkuptext}%
The arg max of a function over a non-empty set is non-empty.%
\end{isamarkuptext}%
\isamarkuptrue%
\isacommand{corollary}\isamarkupfalse%
\ argmax{\isacharunderscore}non{\isacharunderscore}empty{\isacharunderscore}iff{\isacharcolon}\ \isakeyword{assumes}\ {\isachardoublequoteopen}finite\ X{\isachardoublequoteclose}\ {\isachardoublequoteopen}X\ {\isasymnoteq}\ {\isacharbraceleft}{\isacharbraceright}{\isachardoublequoteclose}\ \isakeyword{shows}\ {\isachardoublequoteopen}argmax\ f\ X\ {\isasymnoteq}{\isacharbraceleft}{\isacharbraceright}{\isachardoublequoteclose}\isanewline
%
\isadelimproof
%
\endisadelimproof
%
\isatagproof
\isacommand{using}\isamarkupfalse%
\ assms\ Max{\isacharunderscore}in\ finite{\isacharunderscore}imageI\ image{\isacharunderscore}is{\isacharunderscore}empty\ mm{\isadigit{7}}{\isadigit{9}}\ mm{\isadigit{8}}{\isadigit{6}}b\ \isacommand{by}\isamarkupfalse%
\ {\isacharparenleft}metis{\isacharparenleft}no{\isacharunderscore}types{\isacharparenright}{\isacharparenright}%
\endisatagproof
{\isafoldproof}%
%
\isadelimproof
%
\endisadelimproof
%
\begin{isamarkuptext}%
We want the elements of a list satisfying a given predicate;
but, rather than returning them directly, we return the (sorted) list of their indices. 
This is done, in different ways, by \isa{filterpositions} and \isa{filterpositions{\isadigit{2}}}.%
\end{isamarkuptext}%
\isamarkuptrue%
\isacommand{definition}\isamarkupfalse%
\ filterpositions\ \isanewline
\isanewline
{\isacharcolon}{\isacharcolon}\ {\isachardoublequoteopen}{\isacharparenleft}{\isacharprime}a\ {\isacharequal}{\isachargreater}\ bool{\isacharparenright}\ {\isacharequal}{\isachargreater}\ {\isacharprime}a\ list\ {\isacharequal}{\isachargreater}\ nat\ list{\isachardoublequoteclose}\isanewline
\isakeyword{where}\ {\isachardoublequoteopen}filterpositions\ P\ l\ {\isacharequal}\ map\ snd\ {\isacharparenleft}filter\ {\isacharparenleft}P\ o\ fst{\isacharparenright}\ {\isacharparenleft}zip\ l\ {\isacharparenleft}upt\ {\isadigit{0}}\ {\isacharparenleft}size\ l{\isacharparenright}{\isacharparenright}{\isacharparenright}{\isacharparenright}{\isachardoublequoteclose}\isanewline
\isanewline
\isanewline
\isanewline
\isacommand{definition}\isamarkupfalse%
\ filterpositions{\isadigit{2}}\ \isanewline
{\isacharcolon}{\isacharcolon}\ {\isachardoublequoteopen}{\isacharparenleft}{\isacharprime}a\ {\isacharequal}{\isachargreater}\ bool{\isacharparenright}\ {\isacharequal}{\isachargreater}\ {\isacharprime}a\ list\ {\isacharequal}{\isachargreater}\ nat\ list{\isachardoublequoteclose}\isanewline
\isakeyword{where}\ {\isachardoublequoteopen}filterpositions{\isadigit{2}}\ P\ l\ {\isacharequal}\ {\isacharbrackleft}n{\isachardot}\ n\ {\isasymleftarrow}\ {\isacharbrackleft}{\isadigit{0}}{\isachardot}{\isachardot}{\isacharless}size\ l{\isacharbrackright}{\isacharcomma}\ P\ {\isacharparenleft}l{\isacharbang}n{\isacharparenright}{\isacharbrackright}{\isachardoublequoteclose}\isanewline
\isanewline
\isacommand{definition}\isamarkupfalse%
\ maxpositions\ {\isacharcolon}{\isacharcolon}\ {\isachardoublequoteopen}{\isacharprime}a{\isacharcolon}{\isacharcolon}linorder\ list\ {\isacharequal}{\isachargreater}\ nat\ list{\isachardoublequoteclose}\ \isakeyword{where}\isanewline
{\isachardoublequoteopen}maxpositions\ l\ {\isacharequal}\ filterpositions{\isadigit{2}}\ {\isacharparenleft}{\isacharpercent}x\ {\isachardot}\ x\ {\isasymge}\ Max\ {\isacharparenleft}set\ l{\isacharparenright}{\isacharparenright}\ l{\isachardoublequoteclose}\isanewline
\isanewline
\isacommand{lemma}\isamarkupfalse%
\ ll{\isadigit{5}}{\isacharcolon}\ {\isachardoublequoteopen}maxpositions\ l\ {\isacharequal}\ {\isacharbrackleft}n{\isachardot}\ n{\isasymleftarrow}{\isacharbrackleft}{\isadigit{0}}{\isachardot}{\isachardot}{\isacharless}size\ l{\isacharbrackright}{\isacharcomma}\ l{\isacharbang}n\ {\isasymge}\ Max{\isacharparenleft}set\ l{\isacharparenright}{\isacharbrackright}{\isachardoublequoteclose}\ \isanewline
%
\isadelimproof
%
\endisadelimproof
%
\isatagproof
\isacommand{using}\isamarkupfalse%
\ assms\ \isacommand{unfolding}\isamarkupfalse%
\ maxpositions{\isacharunderscore}def\ filterpositions{\isadigit{2}}{\isacharunderscore}def\ \isacommand{by}\isamarkupfalse%
\ fastforce%
\endisatagproof
{\isafoldproof}%
%
\isadelimproof
\isanewline
%
\endisadelimproof
\isanewline
\isacommand{definition}\isamarkupfalse%
\ argmaxList\isanewline
{\isacharcolon}{\isacharcolon}\ {\isachardoublequoteopen}{\isacharparenleft}{\isacharprime}a\ {\isacharequal}{\isachargreater}\ {\isacharparenleft}{\isacharprime}b{\isacharcolon}{\isacharcolon}linorder{\isacharparenright}{\isacharparenright}\ {\isacharequal}{\isachargreater}\ {\isacharprime}a\ list\ {\isacharequal}{\isachargreater}\ {\isacharprime}a\ list{\isachardoublequoteclose}\isanewline
\isakeyword{where}\ {\isachardoublequoteopen}argmaxList\ f\ l\ {\isacharequal}\ map\ {\isacharparenleft}nth\ l{\isacharparenright}\ {\isacharparenleft}maxpositions\ {\isacharparenleft}map\ f\ l{\isacharparenright}{\isacharparenright}{\isachardoublequoteclose}\isanewline
\isanewline
\isacommand{lemma}\isamarkupfalse%
\ {\isachardoublequoteopen}{\isacharbrackleft}n\ {\isachardot}\ n\ {\isacharless}{\isacharminus}\ {\isacharbrackleft}{\isadigit{0}}{\isachardot}{\isachardot}{\isacharless}m{\isacharbrackright}{\isacharcomma}\ {\isacharparenleft}n\ {\isasymin}\ set\ {\isacharbrackleft}{\isadigit{0}}{\isachardot}{\isachardot}{\isacharless}m{\isacharbrackright}\ {\isacharampersand}\ P\ n{\isacharparenright}{\isacharbrackright}\ \isanewline
{\isacharequal}\ {\isacharbrackleft}n\ {\isachardot}\ n\ {\isacharless}{\isacharminus}\ {\isacharbrackleft}{\isadigit{0}}{\isachardot}{\isachardot}{\isacharless}m{\isacharbrackright}{\isacharcomma}\ n\ {\isasymin}\ set\ {\isacharbrackleft}{\isadigit{0}}{\isachardot}{\isachardot}{\isacharless}m{\isacharbrackright}{\isacharcomma}\ P\ n{\isacharbrackright}{\isachardoublequoteclose}%
\isadelimproof
\ %
\endisadelimproof
%
\isatagproof
\isacommand{by}\isamarkupfalse%
\ meson%
\endisatagproof
{\isafoldproof}%
%
\isadelimproof
%
\endisadelimproof
\isanewline
\isanewline
\isacommand{lemma}\isamarkupfalse%
\ ll{\isadigit{7}}b{\isacharcolon}\ {\isachardoublequoteopen}{\isacharbrackleft}n\ {\isachardot}\ n\ {\isacharless}{\isacharminus}\ l{\isacharcomma}\ P\ n{\isacharbrackright}\ {\isacharequal}\ {\isacharbrackleft}n\ {\isachardot}\ n\ {\isacharless}{\isacharminus}\ l{\isacharcomma}\ n\ {\isasymin}\ set\ l{\isacharcomma}\ P\ n{\isacharbrackright}{\isachardoublequoteclose}\ \isanewline
%
\isadelimproof
%
\endisadelimproof
%
\isatagproof
\isacommand{proof}\isamarkupfalse%
\ {\isacharminus}\ \isanewline
\ \isanewline
\ \ \isacommand{have}\isamarkupfalse%
\ {\isachardoublequoteopen}map\ {\isacharparenleft}{\isasymlambda}uu{\isachardot}\ if\ P\ uu\ then\ {\isacharbrackleft}uu{\isacharbrackright}\ else\ {\isacharbrackleft}{\isacharbrackright}{\isacharparenright}\ l\ {\isacharequal}\ \isanewline
\ \ \ \ map\ {\isacharparenleft}{\isasymlambda}uu{\isachardot}\ if\ uu\ {\isasymin}\ set\ l\ then\ if\ P\ uu\ then\ {\isacharbrackleft}uu{\isacharbrackright}\ else\ {\isacharbrackleft}{\isacharbrackright}\ else\ {\isacharbrackleft}{\isacharbrackright}{\isacharparenright}\ l{\isachardoublequoteclose}\ \isacommand{by}\isamarkupfalse%
\ simp\isanewline
\ \ \isacommand{thus}\isamarkupfalse%
\ {\isachardoublequoteopen}concat\ {\isacharparenleft}map\ {\isacharparenleft}{\isasymlambda}n{\isachardot}\ if\ P\ n\ then\ {\isacharbrackleft}n{\isacharbrackright}\ else\ {\isacharbrackleft}{\isacharbrackright}{\isacharparenright}\ l{\isacharparenright}\ {\isacharequal}\ \isanewline
\ \ \ \ concat\ {\isacharparenleft}map\ {\isacharparenleft}{\isasymlambda}n{\isachardot}\ if\ n\ {\isasymin}\ set\ l\ then\ if\ P\ n\ then\ {\isacharbrackleft}n{\isacharbrackright}\ else\ {\isacharbrackleft}{\isacharbrackright}\ else\ {\isacharbrackleft}{\isacharbrackright}{\isacharparenright}\ l{\isacharparenright}{\isachardoublequoteclose}\ \isacommand{by}\isamarkupfalse%
\ presburger\isanewline
\isacommand{qed}\isamarkupfalse%
%
\endisatagproof
{\isafoldproof}%
%
\isadelimproof
\isanewline
%
\endisadelimproof
\isanewline
\isacommand{lemma}\isamarkupfalse%
\ ll{\isadigit{7}}{\isacharcolon}\ {\isachardoublequoteopen}{\isacharbrackleft}n\ {\isachardot}\ n\ {\isacharless}{\isacharminus}\ {\isacharbrackleft}{\isadigit{0}}{\isachardot}{\isachardot}{\isacharless}m{\isacharbrackright}{\isacharcomma}\ P\ n{\isacharbrackright}\ {\isacharequal}\ {\isacharbrackleft}n\ {\isachardot}\ n\ {\isacharless}{\isacharminus}\ {\isacharbrackleft}{\isadigit{0}}{\isachardot}{\isachardot}{\isacharless}m{\isacharbrackright}{\isacharcomma}\ n\ {\isasymin}\ set\ {\isacharbrackleft}{\isadigit{0}}{\isachardot}{\isachardot}{\isacharless}m{\isacharbrackright}{\isacharcomma}\ P\ n{\isacharbrackright}{\isachardoublequoteclose}%
\isadelimproof
\ %
\endisadelimproof
%
\isatagproof
\isacommand{using}\isamarkupfalse%
\ ll{\isadigit{7}}b\ \isacommand{by}\isamarkupfalse%
\ fast%
\endisatagproof
{\isafoldproof}%
%
\isadelimproof
%
\endisadelimproof
\isanewline
\isanewline
\isanewline
\isacommand{lemma}\isamarkupfalse%
\ ll{\isadigit{1}}{\isadigit{0}}{\isacharcolon}\ \isakeyword{fixes}\ f\ m\ P\ \isakeyword{shows}\ {\isachardoublequoteopen}{\isacharparenleft}map\ f\ {\isacharbrackleft}n\ {\isachardot}\ n\ {\isacharless}{\isacharminus}\ {\isacharbrackleft}{\isadigit{0}}{\isachardot}{\isachardot}{\isacharless}m{\isacharbrackright}{\isacharcomma}\ P\ n{\isacharbrackright}{\isacharparenright}\ {\isacharequal}\ {\isacharbrackleft}\ f\ n\ {\isachardot}\ n\ {\isacharless}{\isacharminus}\ {\isacharbrackleft}{\isadigit{0}}{\isachardot}{\isachardot}{\isacharless}m{\isacharbrackright}{\isacharcomma}\ P\ n{\isacharbrackright}{\isachardoublequoteclose}\ \isanewline
%
\isadelimproof
%
\endisadelimproof
%
\isatagproof
\isacommand{by}\isamarkupfalse%
\ {\isacharparenleft}induct\ m{\isacharparenright}\ auto%
\endisatagproof
{\isafoldproof}%
%
\isadelimproof
\isanewline
%
\endisadelimproof
\isanewline
\isacommand{lemma}\isamarkupfalse%
\ map{\isacharunderscore}commutes{\isacharunderscore}a{\isacharcolon}\ {\isachardoublequoteopen}{\isacharbrackleft}f\ n\ {\isachardot}\ n\ {\isacharless}{\isacharminus}\ {\isacharbrackleft}{\isacharbrackright}{\isacharcomma}\ Q\ {\isacharparenleft}f\ n{\isacharparenright}{\isacharbrackright}\ {\isacharequal}\ {\isacharbrackleft}x\ {\isacharless}{\isacharminus}\ {\isacharparenleft}map\ f\ {\isacharbrackleft}{\isacharbrackright}{\isacharparenright}{\isachardot}\ Q\ x{\isacharbrackright}{\isachardoublequoteclose}%
\isadelimproof
\ %
\endisadelimproof
%
\isatagproof
\isacommand{by}\isamarkupfalse%
\ simp%
\endisatagproof
{\isafoldproof}%
%
\isadelimproof
%
\endisadelimproof
\isanewline
\isanewline
\isacommand{lemma}\isamarkupfalse%
\ map{\isacharunderscore}commutes{\isacharunderscore}b{\isacharcolon}\ {\isachardoublequoteopen}{\isasymforall}\ x\ xs{\isachardot}\ {\isacharparenleft}{\isacharbrackleft}f\ n\ {\isachardot}\ n\ {\isacharless}{\isacharminus}\ xs{\isacharcomma}\ Q\ {\isacharparenleft}f\ n{\isacharparenright}{\isacharbrackright}\ {\isacharequal}\ {\isacharbrackleft}x\ {\isacharless}{\isacharminus}\ {\isacharparenleft}map\ f\ xs{\isacharparenright}{\isachardot}\ Q\ x{\isacharbrackright}\isanewline
{\isasymlongrightarrow}\ {\isacharbrackleft}f\ n\ {\isachardot}\ n\ {\isacharless}{\isacharminus}\ {\isacharparenleft}x{\isacharhash}xs{\isacharparenright}{\isacharcomma}\ Q\ {\isacharparenleft}f\ n{\isacharparenright}{\isacharbrackright}\ {\isacharequal}\ {\isacharbrackleft}x\ {\isacharless}{\isacharminus}\ {\isacharparenleft}map\ f\ {\isacharparenleft}x{\isacharhash}xs{\isacharparenright}{\isacharparenright}{\isachardot}\ Q\ x{\isacharbrackright}{\isacharparenright}{\isachardoublequoteclose}%
\isadelimproof
\ %
\endisadelimproof
%
\isatagproof
\isacommand{using}\isamarkupfalse%
\ assms\ \isacommand{by}\isamarkupfalse%
\ simp%
\endisatagproof
{\isafoldproof}%
%
\isadelimproof
%
\endisadelimproof
\isanewline
\isanewline
\isacommand{lemma}\isamarkupfalse%
\ myStructInduct{\isacharcolon}\ \isakeyword{assumes}\ {\isachardoublequoteopen}P\ {\isacharbrackleft}{\isacharbrackright}{\isachardoublequoteclose}\ {\isachardoublequoteopen}{\isasymforall}x\ xs{\isachardot}\ P\ {\isacharparenleft}xs{\isacharparenright}\ {\isasymlongrightarrow}\ P\ {\isacharparenleft}x{\isacharhash}xs{\isacharparenright}{\isachardoublequoteclose}\ \isakeyword{shows}\ {\isachardoublequoteopen}P\ l{\isachardoublequoteclose}\ \isanewline
%
\isadelimproof
%
\endisadelimproof
%
\isatagproof
\isacommand{using}\isamarkupfalse%
\ assms\ list{\isacharunderscore}nonempty{\isacharunderscore}induct\ \isacommand{by}\isamarkupfalse%
\ {\isacharparenleft}metis{\isacharparenright}%
\endisatagproof
{\isafoldproof}%
%
\isadelimproof
\isanewline
%
\endisadelimproof
\isanewline
\isanewline
\isacommand{lemma}\isamarkupfalse%
\ map{\isacharunderscore}commutes{\isacharcolon}\ \isakeyword{fixes}\ f{\isacharcolon}{\isacharcolon}{\isachardoublequoteopen}{\isacharprime}a\ {\isacharequal}{\isachargreater}\ {\isacharprime}b{\isachardoublequoteclose}\ \isakeyword{fixes}\ Q{\isacharcolon}{\isacharcolon}{\isachardoublequoteopen}{\isacharprime}b\ {\isacharequal}{\isachargreater}\ bool{\isachardoublequoteclose}\ \isakeyword{fixes}\ xs{\isacharcolon}{\isacharcolon}{\isachardoublequoteopen}{\isacharprime}a\ list{\isachardoublequoteclose}\ \isanewline
\isakeyword{shows}\ {\isachardoublequoteopen}{\isacharbrackleft}f\ n\ {\isachardot}\ n\ {\isacharless}{\isacharminus}\ xs{\isacharcomma}\ Q\ {\isacharparenleft}f\ n{\isacharparenright}{\isacharbrackright}\ {\isacharequal}\ {\isacharbrackleft}x\ {\isacharless}{\isacharminus}\ {\isacharparenleft}map\ f\ xs{\isacharparenright}{\isachardot}\ Q\ x{\isacharbrackright}{\isachardoublequoteclose}\isanewline
%
\isadelimproof
%
\endisadelimproof
%
\isatagproof
\isacommand{using}\isamarkupfalse%
\ map{\isacharunderscore}commutes{\isacharunderscore}a\ map{\isacharunderscore}commutes{\isacharunderscore}b\ myStructInduct\ \isacommand{by}\isamarkupfalse%
\ fast%
\endisatagproof
{\isafoldproof}%
%
\isadelimproof
\isanewline
%
\endisadelimproof
\isanewline
\isacommand{lemma}\isamarkupfalse%
\ ll{\isadigit{9}}{\isacharcolon}\ \isakeyword{fixes}\ f\ l\ \isakeyword{shows}\ {\isachardoublequoteopen}maxpositions\ {\isacharparenleft}map\ f\ l{\isacharparenright}\ {\isacharequal}\isanewline
{\isacharbrackleft}n\ {\isachardot}\ n\ {\isacharless}{\isacharminus}\ {\isacharbrackleft}{\isadigit{0}}{\isachardot}{\isachardot}{\isacharless}size\ l{\isacharbrackright}{\isacharcomma}\ f\ {\isacharparenleft}l{\isacharbang}n{\isacharparenright}\ {\isasymge}\ Max\ {\isacharparenleft}f{\isacharbackquote}{\isacharparenleft}set\ l{\isacharparenright}{\isacharparenright}{\isacharbrackright}{\isachardoublequoteclose}\ {\isacharparenleft}\isakeyword{is}\ {\isachardoublequoteopen}maxpositions\ {\isacharparenleft}{\isacharquery}fl{\isacharparenright}\ {\isacharequal}\ {\isacharunderscore}{\isachardoublequoteclose}{\isacharparenright}\isanewline
%
\isadelimproof
%
\endisadelimproof
%
\isatagproof
\isacommand{proof}\isamarkupfalse%
\ {\isacharminus}\isanewline
\ \ \isacommand{have}\isamarkupfalse%
\ {\isachardoublequoteopen}maxpositions\ {\isacharquery}fl\ {\isacharequal}\ \isanewline
\ \ {\isacharbrackleft}n{\isachardot}\ n\ {\isacharless}{\isacharminus}\ {\isacharbrackleft}{\isadigit{0}}{\isachardot}{\isachardot}{\isacharless}size\ {\isacharquery}fl{\isacharbrackright}{\isacharcomma}\ n{\isasymin}\ set{\isacharbrackleft}{\isadigit{0}}{\isachardot}{\isachardot}{\isacharless}size\ {\isacharquery}fl{\isacharbrackright}{\isacharcomma}\ {\isacharquery}fl{\isacharbang}n\ {\isasymge}\ Max\ {\isacharparenleft}set\ {\isacharquery}fl{\isacharparenright}{\isacharbrackright}{\isachardoublequoteclose}\isanewline
\ \ \isacommand{using}\isamarkupfalse%
\ ll{\isadigit{7}}b\ \isacommand{unfolding}\isamarkupfalse%
\ filterpositions{\isadigit{2}}{\isacharunderscore}def\ maxpositions{\isacharunderscore}def\ \isacommand{{\isachardot}}\isamarkupfalse%
\isanewline
\ \ \isacommand{also}\isamarkupfalse%
\ \isacommand{have}\isamarkupfalse%
\ {\isachardoublequoteopen}{\isachardot}{\isachardot}{\isachardot}\ {\isacharequal}\ \isanewline
\ \ {\isacharbrackleft}n\ {\isachardot}\ n\ {\isacharless}{\isacharminus}\ {\isacharbrackleft}{\isadigit{0}}{\isachardot}{\isachardot}{\isacharless}size\ l{\isacharbrackright}{\isacharcomma}\ {\isacharparenleft}n{\isacharless}size\ l{\isacharparenright}{\isacharcomma}\ {\isacharparenleft}{\isacharquery}fl{\isacharbang}n\ \ {\isasymge}\ Max\ {\isacharparenleft}set\ {\isacharquery}fl{\isacharparenright}{\isacharparenright}{\isacharbrackright}{\isachardoublequoteclose}\ \isacommand{by}\isamarkupfalse%
\ simp\isanewline
\ \ \isacommand{also}\isamarkupfalse%
\ \isacommand{have}\isamarkupfalse%
\ {\isachardoublequoteopen}{\isachardot}{\isachardot}{\isachardot}\ {\isacharequal}\ \isanewline
\ \ {\isacharbrackleft}n\ {\isachardot}\ n\ {\isacharless}{\isacharminus}\ {\isacharbrackleft}{\isadigit{0}}{\isachardot}{\isachardot}{\isacharless}size\ l{\isacharbrackright}{\isacharcomma}\ {\isacharparenleft}n{\isacharless}size\ l{\isacharparenright}\ {\isasymand}\ {\isacharparenleft}f\ {\isacharparenleft}l{\isacharbang}n{\isacharparenright}\ \ {\isasymge}\ Max\ {\isacharparenleft}set\ {\isacharquery}fl{\isacharparenright}{\isacharparenright}{\isacharbrackright}{\isachardoublequoteclose}\ \isanewline
\ \ \isacommand{using}\isamarkupfalse%
\ nth{\isacharunderscore}map\ \isacommand{by}\isamarkupfalse%
\ {\isacharparenleft}metis\ {\isacharparenleft}poly{\isacharunderscore}guards{\isacharunderscore}query{\isacharcomma}\ hide{\isacharunderscore}lams{\isacharparenright}{\isacharparenright}\ \isacommand{also}\isamarkupfalse%
\ \isacommand{have}\isamarkupfalse%
\ {\isachardoublequoteopen}{\isachardot}{\isachardot}{\isachardot}\ {\isacharequal}\ \isanewline
\ \ {\isacharbrackleft}n\ {\isachardot}\ n\ {\isacharless}{\isacharminus}\ {\isacharbrackleft}{\isadigit{0}}{\isachardot}{\isachardot}{\isacharless}size\ l{\isacharbrackright}{\isacharcomma}\ {\isacharparenleft}n{\isasymin}\ set\ {\isacharbrackleft}{\isadigit{0}}{\isachardot}{\isachardot}{\isacharless}size\ l{\isacharbrackright}{\isacharparenright}{\isacharcomma}{\isacharparenleft}f\ {\isacharparenleft}l{\isacharbang}n{\isacharparenright}\ \ {\isasymge}\ Max\ {\isacharparenleft}set\ {\isacharquery}fl{\isacharparenright}{\isacharparenright}{\isacharbrackright}{\isachardoublequoteclose}\ \isanewline
\ \ \isacommand{using}\isamarkupfalse%
\ atLeastLessThan{\isacharunderscore}iff\ le{\isadigit{0}}\ set{\isacharunderscore}upt\ \isacommand{by}\isamarkupfalse%
\ {\isacharparenleft}metis{\isacharparenleft}no{\isacharunderscore}types{\isacharparenright}{\isacharparenright}\isanewline
\ \ \isacommand{also}\isamarkupfalse%
\ \isacommand{have}\isamarkupfalse%
\ {\isachardoublequoteopen}{\isachardot}{\isachardot}{\isachardot}\ {\isacharequal}\ \ \isanewline
\ \ {\isacharbrackleft}n\ {\isachardot}\ n\ {\isacharless}{\isacharminus}\ {\isacharbrackleft}{\isadigit{0}}{\isachardot}{\isachardot}{\isacharless}size\ l{\isacharbrackright}{\isacharcomma}\ f\ {\isacharparenleft}l{\isacharbang}n{\isacharparenright}\ {\isasymge}\ Max\ {\isacharparenleft}set\ {\isacharquery}fl{\isacharparenright}{\isacharbrackright}{\isachardoublequoteclose}\ \isacommand{using}\isamarkupfalse%
\ ll{\isadigit{7}}\ \isacommand{by}\isamarkupfalse%
\ presburger\ \isanewline
\ \ \isacommand{finally}\isamarkupfalse%
\ \isacommand{show}\isamarkupfalse%
\ {\isacharquery}thesis\ \isacommand{by}\isamarkupfalse%
\ auto\isanewline
\isacommand{qed}\isamarkupfalse%
%
\endisatagproof
{\isafoldproof}%
%
\isadelimproof
\isanewline
%
\endisadelimproof
\isanewline
\isacommand{lemma}\isamarkupfalse%
\ ll{\isadigit{1}}{\isadigit{1}}{\isacharcolon}\ \isakeyword{fixes}\ f\ l\ \isakeyword{shows}\ {\isachardoublequoteopen}argmaxList\ f\ l\ {\isacharequal}\ {\isacharbrackleft}\ l{\isacharbang}n\ {\isachardot}\ n\ {\isacharless}{\isacharminus}\ {\isacharbrackleft}{\isadigit{0}}{\isachardot}{\isachardot}{\isacharless}size\ l{\isacharbrackright}{\isacharcomma}\ f\ {\isacharparenleft}l{\isacharbang}n{\isacharparenright}\ {\isasymge}\ Max\ {\isacharparenleft}f{\isacharbackquote}{\isacharparenleft}set\ l{\isacharparenright}{\isacharparenright}{\isacharbrackright}{\isachardoublequoteclose}\isanewline
%
\isadelimproof
%
\endisadelimproof
%
\isatagproof
\isacommand{unfolding}\isamarkupfalse%
\ ll{\isadigit{9}}\ argmaxList{\isacharunderscore}def\ \isacommand{by}\isamarkupfalse%
\ {\isacharparenleft}metis\ ll{\isadigit{1}}{\isadigit{0}}{\isacharparenright}%
\endisatagproof
{\isafoldproof}%
%
\isadelimproof
\isanewline
%
\endisadelimproof
\isanewline
\isacommand{theorem}\isamarkupfalse%
\ argmaxadequacy{\isacharcolon}\ \isanewline
\isanewline
\isakeyword{fixes}\ f{\isacharcolon}{\isacharcolon}{\isachardoublequoteopen}{\isacharprime}a\ {\isacharequal}{\isachargreater}\ {\isacharparenleft}{\isacharprime}b{\isacharcolon}{\isacharcolon}linorder{\isacharparenright}{\isachardoublequoteclose}\ \isakeyword{fixes}\ l{\isacharcolon}{\isacharcolon}{\isachardoublequoteopen}{\isacharprime}a\ list{\isachardoublequoteclose}\ \isakeyword{shows}\ \isanewline
{\isachardoublequoteopen}argmaxList\ f\ l\ {\isacharequal}\ {\isacharbrackleft}\ x\ {\isacharless}{\isacharminus}\ l{\isachardot}\ f\ x\ {\isasymge}\ Max\ {\isacharparenleft}f{\isacharbackquote}{\isacharparenleft}set\ l{\isacharparenright}{\isacharparenright}{\isacharbrackright}{\isachardoublequoteclose}\ {\isacharparenleft}\isakeyword{is}\ {\isachardoublequoteopen}{\isacharquery}lh{\isacharequal}{\isacharunderscore}{\isachardoublequoteclose}{\isacharparenright}\isanewline
%
\isadelimproof
%
\endisadelimproof
%
\isatagproof
\isacommand{proof}\isamarkupfalse%
\ {\isacharminus}\isanewline
\ \ \isacommand{let}\isamarkupfalse%
\ {\isacharquery}P{\isacharequal}{\isachardoublequoteopen}{\isacharpercent}\ y{\isacharcolon}{\isacharcolon}{\isacharparenleft}{\isacharprime}b{\isacharcolon}{\isacharcolon}linorder{\isacharparenright}\ {\isachardot}\ y\ {\isasymge}\ Max\ {\isacharparenleft}f{\isacharbackquote}{\isacharparenleft}set\ l{\isacharparenright}{\isacharparenright}{\isachardoublequoteclose}\isanewline
\ \ \isacommand{let}\isamarkupfalse%
\ {\isacharquery}mh{\isacharequal}{\isachardoublequoteopen}{\isacharbrackleft}nth\ l\ n\ {\isachardot}\ n\ {\isacharless}{\isacharminus}\ {\isacharbrackleft}{\isadigit{0}}{\isachardot}{\isachardot}{\isacharless}size\ l{\isacharbrackright}{\isacharcomma}\ {\isacharquery}P\ {\isacharparenleft}f\ {\isacharparenleft}nth\ l\ n{\isacharparenright}{\isacharparenright}{\isacharbrackright}{\isachardoublequoteclose}\isanewline
\ \ \isacommand{let}\isamarkupfalse%
\ {\isacharquery}rh{\isacharequal}{\isachardoublequoteopen}{\isacharbrackleft}\ x\ {\isacharless}{\isacharminus}\ {\isacharparenleft}map\ {\isacharparenleft}nth\ l{\isacharparenright}\ {\isacharbrackleft}{\isadigit{0}}{\isachardot}{\isachardot}{\isacharless}size\ l{\isacharbrackright}{\isacharparenright}{\isachardot}\ {\isacharquery}P\ {\isacharparenleft}f\ x{\isacharparenright}{\isacharbrackright}{\isachardoublequoteclose}\isanewline
\ \ \isacommand{have}\isamarkupfalse%
\ {\isachardoublequoteopen}{\isacharquery}lh\ {\isacharequal}\ {\isacharquery}mh{\isachardoublequoteclose}\ \isacommand{using}\isamarkupfalse%
\ ll{\isadigit{1}}{\isadigit{1}}\ \isacommand{by}\isamarkupfalse%
\ fast\isanewline
\ \ \isacommand{also}\isamarkupfalse%
\ \isacommand{have}\isamarkupfalse%
\ {\isachardoublequoteopen}{\isachardot}{\isachardot}{\isachardot}\ {\isacharequal}\ {\isacharquery}rh{\isachardoublequoteclose}\ \isacommand{using}\isamarkupfalse%
\ map{\isacharunderscore}commutes\ \isacommand{by}\isamarkupfalse%
\ fast\isanewline
\ \ \isacommand{also}\isamarkupfalse%
\ \isacommand{have}\isamarkupfalse%
\ {\isachardoublequoteopen}{\isachardot}{\isachardot}{\isachardot}{\isacharequal}\ {\isacharbrackleft}x\ {\isacharless}{\isacharminus}\ l{\isachardot}\ {\isacharquery}P\ {\isacharparenleft}f\ x{\isacharparenright}{\isacharbrackright}{\isachardoublequoteclose}\ \isacommand{using}\isamarkupfalse%
\ map{\isacharunderscore}nth\ \isacommand{by}\isamarkupfalse%
\ metis\isanewline
\ \ \isacommand{finally}\isamarkupfalse%
\ \isacommand{show}\isamarkupfalse%
\ {\isacharquery}thesis\ \isacommand{by}\isamarkupfalse%
\ force\isanewline
\isacommand{qed}\isamarkupfalse%
%
\endisatagproof
{\isafoldproof}%
%
\isadelimproof
\isanewline
%
\endisadelimproof
%
\isadelimtheory
\isanewline
%
\endisadelimtheory
%
\isatagtheory
\isacommand{end}\isamarkupfalse%
%
\endisatagtheory
{\isafoldtheory}%
%
\isadelimtheory
%
\endisadelimtheory
\end{isabellebody}%
%%% Local Variables:
%%% mode: latex
%%% TeX-master: "root"
%%% End:


%
\begin{isabellebody}%
\def\isabellecontext{MiscTools}%
%
\isamarkupheader{Toolbox of various definitions and theorems about sets, relations and lists%
}
\isamarkuptrue%
%
\isadelimtheory
%
\endisadelimtheory
%
\isatagtheory
\isacommand{theory}\isamarkupfalse%
\ MiscTools\ \isanewline
\isanewline
\isakeyword{imports}\ \isanewline
RelationProperties\isanewline
{\isachardoublequoteopen}{\isachartilde}{\isachartilde}{\isacharslash}src{\isacharslash}HOL{\isacharslash}Library{\isacharslash}Discrete{\isachardoublequoteclose}\isanewline
Main\isanewline
RelationOperators\isanewline
{\isachardoublequoteopen}{\isachartilde}{\isachartilde}{\isacharslash}src{\isacharslash}HOL{\isacharslash}Library{\isacharslash}Code{\isacharunderscore}Target{\isacharunderscore}Nat{\isachardoublequoteclose}\isanewline
{\isachardoublequoteopen}{\isachartilde}{\isachartilde}{\isacharslash}src{\isacharslash}HOL{\isacharslash}Library{\isacharslash}Indicator{\isacharunderscore}Function{\isachardoublequoteclose}\isanewline
Argmax\isanewline
\isanewline
\isakeyword{begin}%
\endisatagtheory
{\isafoldtheory}%
%
\isadelimtheory
%
\endisadelimtheory
%
\isamarkupsection{Facts and notations about relations, sets and functions.%
}
\isamarkuptrue%
\isacommand{notation}\isamarkupfalse%
\ paste\ {\isacharparenleft}\isakeyword{infix}\ {\isachardoublequoteopen}{\isacharplus}{\isacharless}{\isachardoublequoteclose}\ {\isadigit{7}}{\isadigit{5}}{\isacharparenright}%
\begin{isamarkuptext}%
\isa{{\isacharplus}{\isacharless}} abbreviation permits to shorten the notation for altering a function f in a single point by giving a pair (a, b) so that the new function has value b with argument a.%
\end{isamarkuptext}%
\isamarkuptrue%
\isacommand{abbreviation}\isamarkupfalse%
\ singlepaste\isanewline
\ \ \isakeyword{where}\ {\isachardoublequoteopen}singlepaste\ f\ pair\ {\isacharequal}{\isacharequal}\ f\ {\isacharplus}{\isacharasterisk}\ {\isacharbraceleft}{\isacharparenleft}fst\ pair{\isacharcomma}\ snd\ pair{\isacharparenright}{\isacharbraceright}{\isachardoublequoteclose}\isanewline
\ \ \isacommand{notation}\isamarkupfalse%
\ singlepaste\ {\isacharparenleft}\isakeyword{infix}\ {\isachardoublequoteopen}{\isacharplus}{\isacharless}{\isachardoublequoteclose}\ {\isadigit{7}}{\isadigit{5}}{\isacharparenright}%
\begin{isamarkuptext}%
\isa{{\isacharminus}{\isacharminus}} abbreviation permits to shorten the notation for considering a function outside a single point.%
\end{isamarkuptext}%
\isamarkuptrue%
\isacommand{abbreviation}\isamarkupfalse%
\ singleoutside\ {\isacharparenleft}\isakeyword{infix}\ {\isachardoublequoteopen}{\isacharminus}{\isacharminus}{\isachardoublequoteclose}\ {\isadigit{7}}{\isadigit{5}}{\isacharparenright}\isanewline
\ \ \isakeyword{where}\ {\isachardoublequoteopen}f\ {\isacharminus}{\isacharminus}\ x\ {\isasymequiv}\ f\ outside\ {\isacharbraceleft}x{\isacharbraceright}{\isachardoublequoteclose}%
\begin{isamarkuptext}%
Turns a HOL function into a set-theoretical function%
\end{isamarkuptext}%
\isamarkuptrue%
\isacommand{definition}\isamarkupfalse%
\ \ \isanewline
\ \ {\isachardoublequoteopen}Graph\ f\ {\isacharequal}\ {\isacharbraceleft}{\isacharparenleft}x{\isacharcomma}\ f\ x{\isacharparenright}\ {\isacharbar}\ x\ {\isachardot}\ True{\isacharbraceright}{\isachardoublequoteclose}%
\begin{isamarkuptext}%
Inverts \isa{Graph} (which is equivalently done by \isa{op\ {\isacharcomma}{\isacharcomma}}).%
\end{isamarkuptext}%
\isamarkuptrue%
\isacommand{definition}\isamarkupfalse%
\isanewline
\ \ {\isachardoublequoteopen}toFunction\ R\ {\isacharequal}\ {\isacharparenleft}{\isasymlambda}\ x\ {\isachardot}\ {\isacharparenleft}R\ {\isacharcomma}{\isacharcomma}\ x{\isacharparenright}{\isacharparenright}{\isachardoublequoteclose}\isanewline
\isanewline
\isanewline
\isacommand{lemma}\isamarkupfalse%
\ \isanewline
\ \ {\isachardoublequoteopen}toFunction\ {\isacharequal}\ eval{\isacharunderscore}rel{\isachardoublequoteclose}\ \isanewline
%
\isadelimproof
\ \ %
\endisadelimproof
%
\isatagproof
\isacommand{using}\isamarkupfalse%
\ toFunction{\isacharunderscore}def\ eval{\isacharunderscore}rel{\isacharunderscore}def\ \isacommand{by}\isamarkupfalse%
\ blast%
\endisatagproof
{\isafoldproof}%
%
\isadelimproof
\isanewline
%
\endisadelimproof
\isanewline
\isacommand{lemma}\isamarkupfalse%
\ lm{\isadigit{0}}{\isadigit{0}}{\isadigit{1}}{\isacharcolon}\ \isanewline
\ \ {\isachardoublequoteopen}{\isacharparenleft}{\isacharparenleft}P\ {\isasymunion}\ Q{\isacharparenright}\ {\isacharbar}{\isacharbar}\ X{\isacharparenright}\ {\isacharequal}\ {\isacharparenleft}{\isacharparenleft}P\ {\isacharbar}{\isacharbar}\ X{\isacharparenright}\ {\isasymunion}\ {\isacharparenleft}Q{\isacharbar}{\isacharbar}X{\isacharparenright}{\isacharparenright}{\isachardoublequoteclose}\ \isanewline
%
\isadelimproof
\ \ %
\endisadelimproof
%
\isatagproof
\isacommand{unfolding}\isamarkupfalse%
\ restrict{\isacharunderscore}def\ \isacommand{using}\isamarkupfalse%
\ assms\ \isacommand{by}\isamarkupfalse%
\ blast%
\endisatagproof
{\isafoldproof}%
%
\isadelimproof
%
\endisadelimproof
%
\begin{isamarkuptext}%
update behaves like P +* Q (paste), but without enlarging P's Domain. update is the set theoretic equivalent of the lambda function update \isa{fun{\isacharunderscore}upd}%
\end{isamarkuptext}%
\isamarkuptrue%
\isacommand{definition}\isamarkupfalse%
\ update\ \isanewline
\ \ \isakeyword{where}\ {\isachardoublequoteopen}update\ P\ Q\ {\isacharequal}\ P\ {\isacharplus}{\isacharasterisk}\ {\isacharparenleft}Q\ {\isacharbar}{\isacharbar}\ {\isacharparenleft}Domain\ P{\isacharparenright}{\isacharparenright}{\isachardoublequoteclose}\isanewline
\ \ \isacommand{notation}\isamarkupfalse%
\ update\ {\isacharparenleft}\isakeyword{infix}\ {\isachardoublequoteopen}{\isacharplus}{\isacharcircum}{\isachardoublequoteclose}\ {\isadigit{7}}{\isadigit{5}}{\isacharparenright}\isanewline
\isanewline
\isanewline
\isacommand{definition}\isamarkupfalse%
\ runiqer\ \ {\isacharcolon}{\isacharcolon}\ {\isachardoublequoteopen}{\isacharparenleft}{\isacharprime}a\ {\isasymtimes}\ {\isacharprime}b{\isacharparenright}\ set\ {\isacharequal}{\isachargreater}\ {\isacharparenleft}{\isacharprime}a\ {\isasymtimes}\ {\isacharprime}b{\isacharparenright}\ set{\isachardoublequoteclose}\isanewline
\ \ \isakeyword{where}\ {\isachardoublequoteopen}runiqer\ R\ {\isacharequal}\ {\isacharbraceleft}\ {\isacharparenleft}x{\isacharcomma}\ THE\ y{\isachardot}\ y\ {\isasymin}\ R\ {\isacharbackquote}{\isacharbackquote}\ {\isacharbraceleft}x{\isacharbraceright}{\isacharparenright}{\isacharbar}\ x{\isachardot}\ x\ {\isasymin}\ Domain\ R\ {\isacharbraceright}{\isachardoublequoteclose}%
\begin{isamarkuptext}%
\isa{graph} is like \isa{Graph}, but with a built-in restriction to a given set \isa{X}.
This makes it computable for finite X, whereas \isa{Graph\ f\ {\isacharbar}{\isacharbar}\ X} is not computable. 
Duplicates the eponymous definition found in \isa{Function{\isacharunderscore}Order}, which is otherwise not needed.%
\end{isamarkuptext}%
\isamarkuptrue%
\isacommand{definition}\isamarkupfalse%
\ graph\ \isanewline
\ \ \isakeyword{where}\ {\isachardoublequoteopen}graph\ X\ f\ {\isacharequal}\ {\isacharbraceleft}{\isacharparenleft}x{\isacharcomma}\ f\ x{\isacharparenright}\ {\isacharbar}\ x{\isachardot}\ x\ {\isasymin}\ X{\isacharbraceright}{\isachardoublequoteclose}\ \isanewline
\isanewline
\isacommand{lemma}\isamarkupfalse%
\ lm{\isadigit{0}}{\isadigit{0}}{\isadigit{2}}{\isacharcolon}\ \isanewline
\ \ \isakeyword{assumes}\ {\isachardoublequoteopen}runiq\ R{\isachardoublequoteclose}\ \isanewline
\ \ \isakeyword{shows}\ {\isachardoublequoteopen}R\ {\isasymsupseteq}\ graph\ {\isacharparenleft}Domain\ R{\isacharparenright}\ {\isacharparenleft}toFunction\ R{\isacharparenright}{\isachardoublequoteclose}\ \isanewline
%
\isadelimproof
\ \ %
\endisadelimproof
%
\isatagproof
\isacommand{unfolding}\isamarkupfalse%
\ graph{\isacharunderscore}def\ toFunction{\isacharunderscore}def\isanewline
\ \ \isacommand{using}\isamarkupfalse%
\ assms\ graph{\isacharunderscore}def\ toFunction{\isacharunderscore}def\ eval{\isacharunderscore}runiq{\isacharunderscore}rel\ \isacommand{by}\isamarkupfalse%
\ fastforce%
\endisatagproof
{\isafoldproof}%
%
\isadelimproof
\isanewline
%
\endisadelimproof
\isanewline
\isacommand{lemma}\isamarkupfalse%
\ lm{\isadigit{0}}{\isadigit{0}}{\isadigit{3}}{\isacharcolon}\ \isanewline
\ \ \isakeyword{assumes}\ {\isachardoublequoteopen}runiq\ R{\isachardoublequoteclose}\ \isanewline
\ \ \isakeyword{shows}\ {\isachardoublequoteopen}R\ {\isasymsubseteq}\ graph\ {\isacharparenleft}Domain\ R{\isacharparenright}\ {\isacharparenleft}toFunction\ R{\isacharparenright}{\isachardoublequoteclose}\ \isanewline
%
\isadelimproof
\ \ %
\endisadelimproof
%
\isatagproof
\isacommand{unfolding}\isamarkupfalse%
\ graph{\isacharunderscore}def\ toFunction{\isacharunderscore}def\isanewline
\ \ \isacommand{using}\isamarkupfalse%
\ assms\ eval{\isacharunderscore}runiq{\isacharunderscore}rel\ runiq{\isacharunderscore}basic\ Domain{\isachardot}DomainI\ mem{\isacharunderscore}Collect{\isacharunderscore}eq\ subrelI\ \isacommand{by}\isamarkupfalse%
\ fastforce%
\endisatagproof
{\isafoldproof}%
%
\isadelimproof
\isanewline
%
\endisadelimproof
\isanewline
\isacommand{lemma}\isamarkupfalse%
\ lm{\isadigit{0}}{\isadigit{0}}{\isadigit{4}}{\isacharcolon}\ \isanewline
\ \ \isakeyword{assumes}\ {\isachardoublequoteopen}runiq\ R{\isachardoublequoteclose}\ \isanewline
\ \ \isakeyword{shows}\ {\isachardoublequoteopen}R\ {\isacharequal}\ graph\ {\isacharparenleft}Domain\ R{\isacharparenright}\ {\isacharparenleft}toFunction\ R{\isacharparenright}{\isachardoublequoteclose}\isanewline
%
\isadelimproof
\ \ %
\endisadelimproof
%
\isatagproof
\isacommand{using}\isamarkupfalse%
\ assms\ lm{\isadigit{0}}{\isadigit{0}}{\isadigit{2}}\ lm{\isadigit{0}}{\isadigit{0}}{\isadigit{3}}\ \isacommand{by}\isamarkupfalse%
\ fast%
\endisatagproof
{\isafoldproof}%
%
\isadelimproof
\isanewline
%
\endisadelimproof
\isanewline
\isacommand{lemma}\isamarkupfalse%
\ domainOfGraph{\isacharcolon}\ \isanewline
\ \ {\isachardoublequoteopen}runiq{\isacharparenleft}graph\ X\ f{\isacharparenright}\ {\isacharampersand}\ Domain{\isacharparenleft}graph\ X\ f{\isacharparenright}{\isacharequal}X{\isachardoublequoteclose}\ \isanewline
%
\isadelimproof
\ \ %
\endisadelimproof
%
\isatagproof
\isacommand{unfolding}\isamarkupfalse%
\ graph{\isacharunderscore}def\ \isanewline
\ \ \isacommand{using}\isamarkupfalse%
\ rightUniqueRestrictedGraph\ \isacommand{by}\isamarkupfalse%
\ fast%
\endisatagproof
{\isafoldproof}%
%
\isadelimproof
\isanewline
%
\endisadelimproof
\isanewline
\isanewline
\isacommand{abbreviation}\isamarkupfalse%
\ {\isachardoublequoteopen}eval{\isacharunderscore}rel{\isadigit{2}}\ {\isacharparenleft}R{\isacharcolon}{\isacharcolon}{\isacharparenleft}{\isacharprime}a\ {\isasymtimes}\ {\isacharparenleft}{\isacharprime}b\ set{\isacharparenright}{\isacharparenright}\ set{\isacharparenright}\ {\isacharparenleft}x{\isacharcolon}{\isacharcolon}{\isacharprime}a{\isacharparenright}\ {\isacharequal}{\isacharequal}\ {\isasymUnion}\ {\isacharparenleft}R{\isacharbackquote}{\isacharbackquote}{\isacharbraceleft}x{\isacharbraceright}{\isacharparenright}{\isachardoublequoteclose}\isanewline
\ \ \isacommand{notation}\isamarkupfalse%
\ eval{\isacharunderscore}rel{\isadigit{2}}\ {\isacharparenleft}\isakeyword{infix}\ {\isachardoublequoteopen}{\isacharcomma}{\isacharcomma}{\isacharcomma}{\isachardoublequoteclose}\ {\isadigit{7}}{\isadigit{5}}{\isacharparenright}\isanewline
\isanewline
\isacommand{lemma}\isamarkupfalse%
\ imageEquivalence{\isacharcolon}\ \isanewline
\ \ \isakeyword{assumes}\ {\isachardoublequoteopen}runiq\ {\isacharparenleft}f{\isacharcolon}{\isacharcolon}{\isacharparenleft}{\isacharparenleft}{\isacharprime}a\ {\isasymtimes}\ {\isacharparenleft}{\isacharprime}b\ set{\isacharparenright}{\isacharparenright}\ set{\isacharparenright}{\isacharparenright}{\isachardoublequoteclose}\ {\isachardoublequoteopen}x\ {\isasymin}\ Domain\ f{\isachardoublequoteclose}\ \isanewline
\ \ \isakeyword{shows}\ {\isachardoublequoteopen}f{\isacharcomma}{\isacharcomma}x\ {\isacharequal}\ f{\isacharcomma}{\isacharcomma}{\isacharcomma}x{\isachardoublequoteclose}\isanewline
%
\isadelimproof
\ \ %
\endisadelimproof
%
\isatagproof
\isacommand{using}\isamarkupfalse%
\ assms\ Image{\isacharunderscore}runiq{\isacharunderscore}eq{\isacharunderscore}eval\ cSup{\isacharunderscore}singleton\ \isacommand{by}\isamarkupfalse%
\ metis%
\endisatagproof
{\isafoldproof}%
%
\isadelimproof
\isanewline
%
\endisadelimproof
\isanewline
\isanewline
\isacommand{lemma}\isamarkupfalse%
\ lm{\isadigit{0}}{\isadigit{0}}{\isadigit{5}}{\isacharcolon}\ \isanewline
\ \ {\isachardoublequoteopen}Graph\ f{\isacharequal}graph\ UNIV\ f{\isachardoublequoteclose}\ \isanewline
%
\isadelimproof
\ \ %
\endisadelimproof
%
\isatagproof
\isacommand{unfolding}\isamarkupfalse%
\ Graph{\isacharunderscore}def\ graph{\isacharunderscore}def\ \isacommand{by}\isamarkupfalse%
\ simp%
\endisatagproof
{\isafoldproof}%
%
\isadelimproof
\isanewline
%
\endisadelimproof
\isanewline
\isacommand{lemma}\isamarkupfalse%
\ graphIntersection{\isacharcolon}\ \isanewline
\ \ {\isachardoublequoteopen}graph\ {\isacharparenleft}X\ {\isasyminter}\ Y{\isacharparenright}\ f\ {\isasymsubseteq}\ {\isacharparenleft}{\isacharparenleft}graph\ X\ f{\isacharparenright}\ {\isacharbar}{\isacharbar}\ Y{\isacharparenright}{\isachardoublequoteclose}\ \isanewline
%
\isadelimproof
\ \ %
\endisadelimproof
%
\isatagproof
\isacommand{unfolding}\isamarkupfalse%
\ graph{\isacharunderscore}def\ \isanewline
\ \ \isacommand{using}\isamarkupfalse%
\ Int{\isacharunderscore}iff\ mem{\isacharunderscore}Collect{\isacharunderscore}eq\ restrict{\isacharunderscore}ext\ subrelI\ \isacommand{by}\isamarkupfalse%
\ auto%
\endisatagproof
{\isafoldproof}%
%
\isadelimproof
\isanewline
%
\endisadelimproof
\isanewline
\isacommand{definition}\isamarkupfalse%
\ runiqs\ \isanewline
\ \ \isakeyword{where}\ {\isachardoublequoteopen}runiqs{\isacharequal}{\isacharbraceleft}f{\isachardot}\ runiq\ f{\isacharbraceright}{\isachardoublequoteclose}\isanewline
\isanewline
\isacommand{lemma}\isamarkupfalse%
\ outsideOutside{\isacharcolon}\ \isanewline
\ \ {\isachardoublequoteopen}{\isacharparenleft}{\isacharparenleft}P\ outside\ X{\isacharparenright}\ outside\ Y{\isacharparenright}\ {\isacharequal}\ P\ outside\ {\isacharparenleft}X{\isasymunion}Y{\isacharparenright}{\isachardoublequoteclose}\ \isanewline
%
\isadelimproof
\ \ %
\endisadelimproof
%
\isatagproof
\isacommand{unfolding}\isamarkupfalse%
\ Outside{\isacharunderscore}def\ \isacommand{by}\isamarkupfalse%
\ blast%
\endisatagproof
{\isafoldproof}%
%
\isadelimproof
\isanewline
%
\endisadelimproof
\isanewline
\isacommand{corollary}\isamarkupfalse%
\ lm{\isadigit{0}}{\isadigit{0}}{\isadigit{6}}{\isacharcolon}\ \isanewline
\ \ {\isachardoublequoteopen}{\isacharparenleft}{\isacharparenleft}P\ outside\ X{\isacharparenright}\ outside\ X{\isacharparenright}\ {\isacharequal}\ P\ outside\ X{\isachardoublequoteclose}\ \isanewline
%
\isadelimproof
\ \ %
\endisadelimproof
%
\isatagproof
\isacommand{using}\isamarkupfalse%
\ outsideOutside\ \isacommand{by}\isamarkupfalse%
\ force%
\endisatagproof
{\isafoldproof}%
%
\isadelimproof
\ \isanewline
%
\endisadelimproof
\isanewline
\isacommand{lemma}\isamarkupfalse%
\ lm{\isadigit{0}}{\isadigit{0}}{\isadigit{7}}{\isacharcolon}\ \isanewline
\ \ \isakeyword{assumes}\ {\isachardoublequoteopen}{\isacharparenleft}X\ {\isasyminter}\ Domain\ P{\isacharparenright}\ {\isasymsubseteq}\ Domain\ Q{\isachardoublequoteclose}\ \isanewline
\ \ \isakeyword{shows}\ {\isachardoublequoteopen}P\ {\isacharplus}{\isacharasterisk}\ Q\ {\isacharequal}\ {\isacharparenleft}P\ outside\ X{\isacharparenright}\ {\isacharplus}{\isacharasterisk}\ Q{\isachardoublequoteclose}\ \isanewline
%
\isadelimproof
\ \ %
\endisadelimproof
%
\isatagproof
\isacommand{unfolding}\isamarkupfalse%
\ paste{\isacharunderscore}def\ Outside{\isacharunderscore}def\ \isacommand{using}\isamarkupfalse%
\ assms\ \isacommand{by}\isamarkupfalse%
\ blast%
\endisatagproof
{\isafoldproof}%
%
\isadelimproof
\isanewline
%
\endisadelimproof
\isanewline
\isacommand{corollary}\isamarkupfalse%
\ lm{\isadigit{0}}{\isadigit{0}}{\isadigit{8}}{\isacharcolon}\ \isanewline
\ \ {\isachardoublequoteopen}P\ {\isacharplus}{\isacharasterisk}\ Q\ {\isacharequal}\ {\isacharparenleft}P\ outside\ {\isacharparenleft}Domain\ Q{\isacharparenright}{\isacharparenright}\ {\isacharplus}{\isacharasterisk}\ Q{\isachardoublequoteclose}\ \isanewline
%
\isadelimproof
\ \ %
\endisadelimproof
%
\isatagproof
\isacommand{using}\isamarkupfalse%
\ lm{\isadigit{0}}{\isadigit{0}}{\isadigit{7}}\ \isacommand{by}\isamarkupfalse%
\ fast%
\endisatagproof
{\isafoldproof}%
%
\isadelimproof
\isanewline
%
\endisadelimproof
\isanewline
\isacommand{corollary}\isamarkupfalse%
\ outsideUnion{\isacharcolon}\ \isanewline
\ \ {\isachardoublequoteopen}R\ {\isacharequal}\ {\isacharparenleft}R\ outside\ {\isacharbraceleft}x{\isacharbraceright}{\isacharparenright}\ {\isasymunion}\ {\isacharparenleft}{\isacharbraceleft}x{\isacharbraceright}\ {\isasymtimes}\ {\isacharparenleft}R\ {\isacharbackquote}{\isacharbackquote}\ {\isacharbraceleft}x{\isacharbraceright}{\isacharparenright}{\isacharparenright}{\isachardoublequoteclose}\ \isanewline
%
\isadelimproof
\ \ %
\endisadelimproof
%
\isatagproof
\isacommand{using}\isamarkupfalse%
\ restrict{\isacharunderscore}to{\isacharunderscore}singleton\ outside{\isacharunderscore}union{\isacharunderscore}restrict\ \isacommand{by}\isamarkupfalse%
\ metis%
\endisatagproof
{\isafoldproof}%
%
\isadelimproof
\isanewline
%
\endisadelimproof
\isanewline
\isacommand{lemma}\isamarkupfalse%
\ lm{\isadigit{0}}{\isadigit{0}}{\isadigit{9}}{\isacharcolon}\ \isanewline
\ \ {\isachardoublequoteopen}P\ {\isacharequal}\ P\ {\isasymunion}\ {\isacharbraceleft}x{\isacharbraceright}{\isasymtimes}P{\isacharbackquote}{\isacharbackquote}{\isacharbraceleft}x{\isacharbraceright}{\isachardoublequoteclose}\ \isanewline
%
\isadelimproof
\ \ %
\endisadelimproof
%
\isatagproof
\isacommand{using}\isamarkupfalse%
\ assms\ \isacommand{by}\isamarkupfalse%
\ {\isacharparenleft}metis\ outsideUnion\ sup{\isachardot}right{\isacharunderscore}idem{\isacharparenright}%
\endisatagproof
{\isafoldproof}%
%
\isadelimproof
\isanewline
%
\endisadelimproof
\isanewline
\isacommand{corollary}\isamarkupfalse%
\ lm{\isadigit{0}}{\isadigit{1}}{\isadigit{0}}{\isacharcolon}\ \isanewline
\ \ {\isachardoublequoteopen}R\ {\isacharequal}\ {\isacharparenleft}R\ outside\ {\isacharbraceleft}x{\isacharbraceright}{\isacharparenright}\ {\isacharplus}{\isacharasterisk}\ {\isacharparenleft}{\isacharbraceleft}x{\isacharbraceright}\ {\isasymtimes}\ {\isacharparenleft}R\ {\isacharbackquote}{\isacharbackquote}\ {\isacharbraceleft}x{\isacharbraceright}{\isacharparenright}{\isacharparenright}{\isachardoublequoteclose}\ \isanewline
%
\isadelimproof
\ \ %
\endisadelimproof
%
\isatagproof
\isacommand{by}\isamarkupfalse%
\ {\isacharparenleft}metis\ paste{\isacharunderscore}outside{\isacharunderscore}restrict\ restrict{\isacharunderscore}to{\isacharunderscore}singleton{\isacharparenright}%
\endisatagproof
{\isafoldproof}%
%
\isadelimproof
\isanewline
%
\endisadelimproof
\isanewline
\isacommand{lemma}\isamarkupfalse%
\ lm{\isadigit{0}}{\isadigit{1}}{\isadigit{1}}{\isacharcolon}\ \isanewline
\ \ {\isachardoublequoteopen}R\ {\isasymsubseteq}\ R\ {\isacharplus}{\isacharasterisk}\ {\isacharparenleft}{\isacharbraceleft}x{\isacharbraceright}\ {\isasymtimes}\ {\isacharparenleft}R{\isacharbackquote}{\isacharbackquote}{\isacharbraceleft}x{\isacharbraceright}{\isacharparenright}{\isacharparenright}{\isachardoublequoteclose}\ \isanewline
%
\isadelimproof
\ \ %
\endisadelimproof
%
\isatagproof
\isacommand{using}\isamarkupfalse%
\ lm{\isadigit{0}}{\isadigit{1}}{\isadigit{0}}\ lm{\isadigit{0}}{\isadigit{0}}{\isadigit{8}}\ paste{\isacharunderscore}def\ Outside{\isacharunderscore}def\ \isacommand{by}\isamarkupfalse%
\ fast%
\endisatagproof
{\isafoldproof}%
%
\isadelimproof
\isanewline
%
\endisadelimproof
\isanewline
\isacommand{lemma}\isamarkupfalse%
\ lm{\isadigit{0}}{\isadigit{1}}{\isadigit{2}}{\isacharcolon}\ \isanewline
\ \ {\isachardoublequoteopen}R\ {\isasymsupseteq}\ R{\isacharplus}{\isacharasterisk}{\isacharparenleft}{\isacharbraceleft}x{\isacharbraceright}\ {\isasymtimes}\ {\isacharparenleft}R{\isacharbackquote}{\isacharbackquote}{\isacharbraceleft}x{\isacharbraceright}{\isacharparenright}{\isacharparenright}{\isachardoublequoteclose}\ \isanewline
%
\isadelimproof
\ \ %
\endisadelimproof
%
\isatagproof
\isacommand{by}\isamarkupfalse%
\ {\isacharparenleft}metis\ Un{\isacharunderscore}least\ Un{\isacharunderscore}upper{\isadigit{1}}\ outside{\isacharunderscore}union{\isacharunderscore}restrict\ paste{\isacharunderscore}def\ \isanewline
\ \ \ \ \ \ \ \ \ \ \ \ restrict{\isacharunderscore}to{\isacharunderscore}singleton\ restriction{\isacharunderscore}is{\isacharunderscore}subrel{\isacharparenright}%
\endisatagproof
{\isafoldproof}%
%
\isadelimproof
\isanewline
%
\endisadelimproof
\isanewline
\isacommand{lemma}\isamarkupfalse%
\ lm{\isadigit{0}}{\isadigit{1}}{\isadigit{3}}{\isacharcolon}\ \isanewline
\ \ {\isachardoublequoteopen}R\ {\isacharequal}\ R\ {\isacharplus}{\isacharasterisk}\ {\isacharparenleft}{\isacharbraceleft}x{\isacharbraceright}\ {\isasymtimes}\ {\isacharparenleft}R{\isacharbackquote}{\isacharbackquote}{\isacharbraceleft}x{\isacharbraceright}{\isacharparenright}{\isacharparenright}{\isachardoublequoteclose}\ \isanewline
%
\isadelimproof
\ \ %
\endisadelimproof
%
\isatagproof
\isacommand{using}\isamarkupfalse%
\ lm{\isadigit{0}}{\isadigit{1}}{\isadigit{1}}\ lm{\isadigit{0}}{\isadigit{1}}{\isadigit{2}}\ \isacommand{by}\isamarkupfalse%
\ force%
\endisatagproof
{\isafoldproof}%
%
\isadelimproof
\isanewline
%
\endisadelimproof
\isanewline
\isacommand{lemma}\isamarkupfalse%
\ rightUniqueTrivialCartes{\isacharcolon}\ \isanewline
\ \ \isakeyword{assumes}\ {\isachardoublequoteopen}trivial\ Y{\isachardoublequoteclose}\ \isanewline
\ \ \isakeyword{shows}\ {\isachardoublequoteopen}runiq\ {\isacharparenleft}X\ {\isasymtimes}\ Y{\isacharparenright}{\isachardoublequoteclose}\ \isanewline
%
\isadelimproof
\ \ %
\endisadelimproof
%
\isatagproof
\isacommand{using}\isamarkupfalse%
\ assms\ runiq{\isacharunderscore}def\ Image{\isacharunderscore}subset\ lm{\isadigit{0}}{\isadigit{1}}{\isadigit{3}}\ trivial{\isacharunderscore}subset\ lm{\isadigit{0}}{\isadigit{1}}{\isadigit{1}}\ \isacommand{by}\isamarkupfalse%
\ {\isacharparenleft}metis{\isacharparenleft}no{\isacharunderscore}types{\isacharparenright}{\isacharparenright}%
\endisatagproof
{\isafoldproof}%
%
\isadelimproof
\isanewline
%
\endisadelimproof
\isanewline
\isanewline
\isacommand{lemma}\isamarkupfalse%
\ lm{\isadigit{0}}{\isadigit{1}}{\isadigit{4}}{\isacharcolon}\ \isanewline
\ \ {\isachardoublequoteopen}runiq\ {\isacharparenleft}{\isacharparenleft}X\ {\isasymtimes}\ {\isacharbraceleft}x{\isacharbraceright}{\isacharparenright}\ {\isacharplus}{\isacharasterisk}\ {\isacharparenleft}Y\ {\isasymtimes}\ {\isacharbraceleft}y{\isacharbraceright}{\isacharparenright}{\isacharparenright}{\isachardoublequoteclose}\ \isanewline
%
\isadelimproof
\ \ %
\endisadelimproof
%
\isatagproof
\isacommand{using}\isamarkupfalse%
\ assms\ rightUniqueTrivialCartes\ trivial{\isacharunderscore}singleton\ runiq{\isacharunderscore}paste{\isadigit{2}}\ \isacommand{by}\isamarkupfalse%
\ metis%
\endisatagproof
{\isafoldproof}%
%
\isadelimproof
\isanewline
%
\endisadelimproof
\isanewline
\isacommand{lemma}\isamarkupfalse%
\ lm{\isadigit{0}}{\isadigit{1}}{\isadigit{5}}{\isacharcolon}\ \isanewline
\ \ {\isachardoublequoteopen}{\isacharparenleft}P\ {\isacharbar}{\isacharbar}\ {\isacharparenleft}X\ {\isasyminter}\ Y{\isacharparenright}{\isacharparenright}\ {\isasymsubseteq}\ {\isacharparenleft}P{\isacharbar}{\isacharbar}X{\isacharparenright}\ \ \ \ {\isacharampersand}\ \ \ \ P\ outside\ {\isacharparenleft}X\ {\isasymunion}\ Y{\isacharparenright}\ {\isasymsubseteq}\ P\ outside\ X{\isachardoublequoteclose}\ \isanewline
%
\isadelimproof
\ \ %
\endisadelimproof
%
\isatagproof
\isacommand{using}\isamarkupfalse%
\ Outside{\isacharunderscore}def\ restrict{\isacharunderscore}def\ Sigma{\isacharunderscore}Un{\isacharunderscore}distrib{\isadigit{1}}\ Un{\isacharunderscore}upper{\isadigit{1}}\ inf{\isacharunderscore}mono\ Diff{\isacharunderscore}mono\ subset{\isacharunderscore}refl\ \isanewline
\ \ \isacommand{by}\isamarkupfalse%
\ {\isacharparenleft}metis\ {\isacharparenleft}lifting{\isacharparenright}\ Sigma{\isacharunderscore}mono\ inf{\isacharunderscore}le{\isadigit{1}}{\isacharparenright}%
\endisatagproof
{\isafoldproof}%
%
\isadelimproof
\isanewline
%
\endisadelimproof
\isanewline
\isacommand{lemma}\isamarkupfalse%
\ lm{\isadigit{0}}{\isadigit{1}}{\isadigit{6}}{\isacharcolon}\ \isanewline
\ \ {\isachardoublequoteopen}P\ {\isacharbar}{\isacharbar}\ X\ {\isasymsubseteq}\ {\isacharparenleft}P{\isacharbar}{\isacharbar}{\isacharparenleft}X\ {\isasymunion}\ Y{\isacharparenright}{\isacharparenright}\ \ \ \ \ \ \ {\isacharampersand}\ \ \ \ P\ outside\ X\ {\isasymsubseteq}\ P\ outside\ {\isacharparenleft}X\ {\isasyminter}\ Y{\isacharparenright}{\isachardoublequoteclose}\ \isanewline
%
\isadelimproof
\ \ %
\endisadelimproof
%
\isatagproof
\isacommand{using}\isamarkupfalse%
\ lm{\isadigit{0}}{\isadigit{1}}{\isadigit{5}}\ distrib{\isacharunderscore}sup{\isacharunderscore}le\ sup{\isacharunderscore}idem\ le{\isacharunderscore}inf{\isacharunderscore}iff\ subset{\isacharunderscore}antisym\ sup{\isacharunderscore}commute\isanewline
\ \ \isacommand{by}\isamarkupfalse%
\ {\isacharparenleft}metis\ sup{\isacharunderscore}ge{\isadigit{1}}{\isacharparenright}%
\endisatagproof
{\isafoldproof}%
%
\isadelimproof
\isanewline
%
\endisadelimproof
\isanewline
\isacommand{lemma}\isamarkupfalse%
\ lm{\isadigit{0}}{\isadigit{1}}{\isadigit{7}}{\isacharcolon}\ \isanewline
\ \ {\isachardoublequoteopen}P{\isacharbackquote}{\isacharbackquote}{\isacharparenleft}X\ {\isasyminter}\ Domain\ P{\isacharparenright}\ {\isacharequal}\ P{\isacharbackquote}{\isacharbackquote}X{\isachardoublequoteclose}\ \isanewline
%
\isadelimproof
\ \ %
\endisadelimproof
%
\isatagproof
\isacommand{by}\isamarkupfalse%
\ blast%
\endisatagproof
{\isafoldproof}%
%
\isadelimproof
\isanewline
%
\endisadelimproof
\isanewline
\isacommand{lemma}\isamarkupfalse%
\ cardinalityOneSubset{\isacharcolon}\ \isanewline
\ \ \isakeyword{assumes}\ {\isachardoublequoteopen}card\ X{\isacharequal}{\isadigit{1}}{\isachardoublequoteclose}\ \isakeyword{and}\ {\isachardoublequoteopen}X\ {\isasymsubseteq}\ Y{\isachardoublequoteclose}\ \isanewline
\ \ \isakeyword{shows}\ {\isachardoublequoteopen}Union\ X\ {\isasymin}\ Y{\isachardoublequoteclose}\ \isanewline
%
\isadelimproof
\ \ %
\endisadelimproof
%
\isatagproof
\isacommand{using}\isamarkupfalse%
\ assms\ cardinalityOneTheElemIdentity\ \isacommand{by}\isamarkupfalse%
\ {\isacharparenleft}metis\ cSup{\isacharunderscore}singleton\ insert{\isacharunderscore}subset{\isacharparenright}%
\endisatagproof
{\isafoldproof}%
%
\isadelimproof
\isanewline
%
\endisadelimproof
\isanewline
\isacommand{lemma}\isamarkupfalse%
\ cardinalityOneTheElem{\isacharcolon}\ \isanewline
\ \ \isakeyword{assumes}\ {\isachardoublequoteopen}card\ X{\isacharequal}{\isadigit{1}}{\isachardoublequoteclose}\ {\isachardoublequoteopen}X\ {\isasymsubseteq}\ Y{\isachardoublequoteclose}\ \isanewline
\ \ \isakeyword{shows}\ {\isachardoublequoteopen}the{\isacharunderscore}elem\ X\ {\isasymin}\ Y{\isachardoublequoteclose}\ \isanewline
%
\isadelimproof
\ \ %
\endisadelimproof
%
\isatagproof
\isacommand{using}\isamarkupfalse%
\ assms\ \isacommand{by}\isamarkupfalse%
\ {\isacharparenleft}metis\ {\isacharparenleft}full{\isacharunderscore}types{\isacharparenright}\ insert{\isacharunderscore}subset\ cardinalityOneTheElemIdentity{\isacharparenright}%
\endisatagproof
{\isafoldproof}%
%
\isadelimproof
\isanewline
%
\endisadelimproof
\isanewline
\isacommand{lemma}\isamarkupfalse%
\ lm{\isadigit{0}}{\isadigit{1}}{\isadigit{8}}{\isacharcolon}\ \isanewline
\ \ {\isachardoublequoteopen}{\isacharparenleft}R\ outside\ X{\isadigit{1}}{\isacharparenright}\ outside\ X{\isadigit{2}}\ {\isacharequal}\ {\isacharparenleft}R\ outside\ X{\isadigit{2}}{\isacharparenright}\ outside\ X{\isadigit{1}}{\isachardoublequoteclose}\ \isanewline
%
\isadelimproof
\ \ %
\endisadelimproof
%
\isatagproof
\isacommand{by}\isamarkupfalse%
\ {\isacharparenleft}metis\ outsideOutside\ sup{\isacharunderscore}commute{\isacharparenright}%
\endisatagproof
{\isafoldproof}%
%
\isadelimproof
%
\endisadelimproof
%
\isamarkupsection{ordered relations%
}
\isamarkuptrue%
\isacommand{lemma}\isamarkupfalse%
\ lm{\isadigit{0}}{\isadigit{1}}{\isadigit{9}}{\isacharcolon}\ \isanewline
\ \ \isakeyword{assumes}\ {\isachardoublequoteopen}card\ X{\isasymge}{\isadigit{1}}{\isachardoublequoteclose}\ {\isachardoublequoteopen}{\isasymforall}x{\isasymin}X{\isachardot}\ y\ {\isachargreater}\ x{\isachardoublequoteclose}\ \isanewline
\ \ \isakeyword{shows}\ {\isachardoublequoteopen}y\ {\isachargreater}\ Max\ X{\isachardoublequoteclose}\ \isanewline
%
\isadelimproof
\ \ %
\endisadelimproof
%
\isatagproof
\isacommand{using}\isamarkupfalse%
\ assms\ \isacommand{by}\isamarkupfalse%
\ {\isacharparenleft}metis\ {\isacharparenleft}poly{\isacharunderscore}guards{\isacharunderscore}query{\isacharparenright}\ Max{\isacharunderscore}in\ One{\isacharunderscore}nat{\isacharunderscore}def\ card{\isacharunderscore}eq{\isacharunderscore}{\isadigit{0}}{\isacharunderscore}iff\ lessI\ not{\isacharunderscore}le{\isacharparenright}%
\endisatagproof
{\isafoldproof}%
%
\isadelimproof
\isanewline
%
\endisadelimproof
\isanewline
\isanewline
\isacommand{lemma}\isamarkupfalse%
\ lm{\isadigit{0}}{\isadigit{2}}{\isadigit{0}}{\isacharcolon}\ \isanewline
\ \ \isakeyword{assumes}\ {\isachardoublequoteopen}finite\ X{\isachardoublequoteclose}\ {\isachardoublequoteopen}mx\ {\isasymin}\ X{\isachardoublequoteclose}\ {\isachardoublequoteopen}f\ x\ {\isacharless}\ f\ mx{\isachardoublequoteclose}\ \isanewline
\ \ \isakeyword{shows}{\isachardoublequoteopen}x\ {\isasymnotin}\ argmax\ f\ X{\isachardoublequoteclose}\ \isanewline
%
\isadelimproof
\ \ %
\endisadelimproof
%
\isatagproof
\isacommand{using}\isamarkupfalse%
\ assms\ not{\isacharunderscore}less\ \isacommand{by}\isamarkupfalse%
\ fastforce%
\endisatagproof
{\isafoldproof}%
%
\isadelimproof
\isanewline
%
\endisadelimproof
\isanewline
\isacommand{lemma}\isamarkupfalse%
\ lm{\isadigit{0}}{\isadigit{2}}{\isadigit{1}}{\isacharcolon}\ \isanewline
\ \ \isakeyword{assumes}\ {\isachardoublequoteopen}finite\ X{\isachardoublequoteclose}\ {\isachardoublequoteopen}mx\ {\isasymin}\ X{\isachardoublequoteclose}\ {\isachardoublequoteopen}{\isasymforall}x\ {\isasymin}\ X{\isacharminus}{\isacharbraceleft}mx{\isacharbraceright}{\isachardot}\ f\ x\ {\isacharless}\ f\ mx{\isachardoublequoteclose}\ \isanewline
\ \ \isakeyword{shows}\ {\isachardoublequoteopen}argmax\ f\ X\ {\isasymsubseteq}\ {\isacharbraceleft}mx{\isacharbraceright}{\isachardoublequoteclose}\isanewline
%
\isadelimproof
\ \ %
\endisadelimproof
%
\isatagproof
\isacommand{using}\isamarkupfalse%
\ assms\ mk{\isacharunderscore}disjoint{\isacharunderscore}insert\ \isacommand{by}\isamarkupfalse%
\ force%
\endisatagproof
{\isafoldproof}%
%
\isadelimproof
\isanewline
%
\endisadelimproof
\isanewline
\isacommand{lemma}\isamarkupfalse%
\ lm{\isadigit{0}}{\isadigit{2}}{\isadigit{2}}{\isacharcolon}\ \isanewline
\ \ \isakeyword{assumes}\ {\isachardoublequoteopen}finite\ X{\isachardoublequoteclose}\ {\isachardoublequoteopen}mx\ {\isasymin}\ X{\isachardoublequoteclose}\ {\isachardoublequoteopen}{\isasymforall}x\ {\isasymin}\ X{\isacharminus}{\isacharbraceleft}mx{\isacharbraceright}{\isachardot}\ f\ x\ {\isacharless}\ f\ mx{\isachardoublequoteclose}\ \isanewline
\ \ \isakeyword{shows}\ {\isachardoublequoteopen}argmax\ f\ X\ {\isacharequal}\ {\isacharbraceleft}mx{\isacharbraceright}{\isachardoublequoteclose}\ \isanewline
%
\isadelimproof
\ \ %
\endisadelimproof
%
\isatagproof
\isacommand{using}\isamarkupfalse%
\ assms\ lm{\isadigit{0}}{\isadigit{2}}{\isadigit{1}}\ \isacommand{by}\isamarkupfalse%
\ {\isacharparenleft}metis\ argmax{\isacharunderscore}non{\isacharunderscore}empty{\isacharunderscore}iff\ equals{\isadigit{0}}D\ subset{\isacharunderscore}singletonD{\isacharparenright}%
\endisatagproof
{\isafoldproof}%
%
\isadelimproof
\isanewline
%
\endisadelimproof
\isanewline
\isanewline
\isacommand{corollary}\isamarkupfalse%
\ argmaxProperty{\isacharcolon}\ \isanewline
\ \ {\isachardoublequoteopen}{\isacharparenleft}finite\ X\ {\isacharampersand}\ mx\ {\isasymin}\ X\ {\isacharampersand}\ {\isacharparenleft}{\isasymforall}aa\ {\isasymin}\ X{\isacharminus}{\isacharbraceleft}mx{\isacharbraceright}{\isachardot}\ f\ aa\ {\isacharless}\ f\ mx{\isacharparenright}{\isacharparenright}\ {\isasymlongrightarrow}\ argmax\ f\ X\ {\isacharequal}\ {\isacharbraceleft}mx{\isacharbraceright}{\isachardoublequoteclose}\isanewline
%
\isadelimproof
\ \ %
\endisadelimproof
%
\isatagproof
\isacommand{using}\isamarkupfalse%
\ assms\ lm{\isadigit{0}}{\isadigit{2}}{\isadigit{2}}\ \isacommand{by}\isamarkupfalse%
\ metis%
\endisatagproof
{\isafoldproof}%
%
\isadelimproof
\isanewline
%
\endisadelimproof
\isanewline
\isacommand{corollary}\isamarkupfalse%
\ lm{\isadigit{0}}{\isadigit{2}}{\isadigit{3}}{\isacharcolon}\ \isanewline
\ \ \isakeyword{assumes}\ {\isachardoublequoteopen}finite\ X{\isachardoublequoteclose}\ {\isachardoublequoteopen}mx\ {\isasymin}\ X{\isachardoublequoteclose}\ {\isachardoublequoteopen}{\isasymforall}x\ {\isasymin}\ X{\isachardot}\ x\ {\isasymnoteq}\ mx\ {\isasymlongrightarrow}\ f\ x\ {\isacharless}\ f\ mx{\isachardoublequoteclose}\ \isanewline
\ \ \isakeyword{shows}\ {\isachardoublequoteopen}argmax\ f\ X\ {\isacharequal}\ {\isacharbraceleft}mx{\isacharbraceright}{\isachardoublequoteclose}\isanewline
%
\isadelimproof
\ \ %
\endisadelimproof
%
\isatagproof
\isacommand{using}\isamarkupfalse%
\ assms\ lm{\isadigit{0}}{\isadigit{2}}{\isadigit{2}}\ \isacommand{by}\isamarkupfalse%
\ {\isacharparenleft}metis\ Diff{\isacharunderscore}iff\ insertI{\isadigit{1}}{\isacharparenright}%
\endisatagproof
{\isafoldproof}%
%
\isadelimproof
\isanewline
%
\endisadelimproof
\isanewline
\isacommand{lemma}\isamarkupfalse%
\ lm{\isadigit{0}}{\isadigit{2}}{\isadigit{4}}{\isacharcolon}\ \isanewline
\ \ \isakeyword{assumes}\ {\isachardoublequoteopen}f\ {\isasymcirc}\ g\ {\isacharequal}\ id{\isachardoublequoteclose}\ \isanewline
\ \ \isakeyword{shows}\ {\isachardoublequoteopen}inj{\isacharunderscore}on\ g\ UNIV{\isachardoublequoteclose}%
\isadelimproof
\ %
\endisadelimproof
%
\isatagproof
\isacommand{using}\isamarkupfalse%
\ assms\ \isanewline
\ \ \isacommand{by}\isamarkupfalse%
\ {\isacharparenleft}metis\ inj{\isacharunderscore}on{\isacharunderscore}id\ inj{\isacharunderscore}on{\isacharunderscore}imageI{\isadigit{2}}{\isacharparenright}%
\endisatagproof
{\isafoldproof}%
%
\isadelimproof
%
\endisadelimproof
\isanewline
\isanewline
\isanewline
\isacommand{lemma}\isamarkupfalse%
\ lm{\isadigit{0}}{\isadigit{2}}{\isadigit{5}}{\isacharcolon}\ \isanewline
\ \ \isakeyword{assumes}\ {\isachardoublequoteopen}inj{\isacharunderscore}on\ f\ X{\isachardoublequoteclose}\ \isanewline
\ \ \isakeyword{shows}\ {\isachardoublequoteopen}inj{\isacharunderscore}on\ {\isacharparenleft}image\ f{\isacharparenright}\ {\isacharparenleft}Pow\ X{\isacharparenright}{\isachardoublequoteclose}\isanewline
%
\isadelimproof
\ \ %
\endisadelimproof
%
\isatagproof
\isacommand{using}\isamarkupfalse%
\ assms\ inj{\isacharunderscore}on{\isacharunderscore}image{\isacharunderscore}eq{\isacharunderscore}iff\ inj{\isacharunderscore}onI\ PowD\ \isacommand{by}\isamarkupfalse%
\ {\isacharparenleft}metis\ {\isacharparenleft}mono{\isacharunderscore}tags{\isacharcomma}\ lifting{\isacharparenright}{\isacharparenright}%
\endisatagproof
{\isafoldproof}%
%
\isadelimproof
\isanewline
%
\endisadelimproof
\isanewline
\isacommand{lemma}\isamarkupfalse%
\ injectionPowerset{\isacharcolon}\ \isanewline
\ \ \isakeyword{assumes}\ {\isachardoublequoteopen}inj{\isacharunderscore}on\ f\ Y{\isachardoublequoteclose}\ {\isachardoublequoteopen}X\ {\isasymsubseteq}\ Y{\isachardoublequoteclose}\ \isanewline
\ \ \isakeyword{shows}\ {\isachardoublequoteopen}inj{\isacharunderscore}on\ {\isacharparenleft}image\ f{\isacharparenright}\ {\isacharparenleft}Pow\ X{\isacharparenright}{\isachardoublequoteclose}\isanewline
%
\isadelimproof
\ \ %
\endisadelimproof
%
\isatagproof
\isacommand{using}\isamarkupfalse%
\ assms\ lm{\isadigit{0}}{\isadigit{2}}{\isadigit{5}}\ \isacommand{by}\isamarkupfalse%
\ {\isacharparenleft}metis\ subset{\isacharunderscore}inj{\isacharunderscore}on{\isacharparenright}%
\endisatagproof
{\isafoldproof}%
%
\isadelimproof
\isanewline
%
\endisadelimproof
\isanewline
\isanewline
\isacommand{definition}\isamarkupfalse%
\ finestpart\ \isanewline
\ \ \isakeyword{where}\ {\isachardoublequoteopen}finestpart\ X\ {\isacharequal}\ {\isacharparenleft}{\isacharpercent}x{\isachardot}\ {\isacharbraceleft}x{\isacharbraceright}{\isacharparenright}\ {\isacharbackquote}\ X{\isachardoublequoteclose}\isanewline
\isanewline
\isacommand{lemma}\isamarkupfalse%
\ finestPart{\isacharcolon}\ \isanewline
\ \ {\isachardoublequoteopen}finestpart\ X\ {\isacharequal}\ {\isacharbraceleft}{\isacharbraceleft}x{\isacharbraceright}{\isacharbar}x\ {\isachardot}\ x{\isasymin}X{\isacharbraceright}{\isachardoublequoteclose}\ \isanewline
%
\isadelimproof
\ \ %
\endisadelimproof
%
\isatagproof
\isacommand{unfolding}\isamarkupfalse%
\ finestpart{\isacharunderscore}def\ \isacommand{by}\isamarkupfalse%
\ blast%
\endisatagproof
{\isafoldproof}%
%
\isadelimproof
\isanewline
%
\endisadelimproof
\isanewline
\isacommand{lemma}\isamarkupfalse%
\ finestPartUnion{\isacharcolon}\ \isanewline
\ \ {\isachardoublequoteopen}X{\isacharequal}{\isasymUnion}\ {\isacharparenleft}finestpart\ X{\isacharparenright}{\isachardoublequoteclose}\ \isanewline
%
\isadelimproof
\ \ %
\endisadelimproof
%
\isatagproof
\isacommand{using}\isamarkupfalse%
\ finestPart\ \isacommand{by}\isamarkupfalse%
\ auto%
\endisatagproof
{\isafoldproof}%
%
\isadelimproof
\isanewline
%
\endisadelimproof
\isanewline
\isacommand{lemma}\isamarkupfalse%
\ lm{\isadigit{0}}{\isadigit{2}}{\isadigit{6}}{\isacharcolon}\ \isanewline
\ \ {\isachardoublequoteopen}Union\ {\isasymcirc}\ finestpart\ {\isacharequal}\ id{\isachardoublequoteclose}\ \isanewline
%
\isadelimproof
\ \ %
\endisadelimproof
%
\isatagproof
\isacommand{using}\isamarkupfalse%
\ finestpart{\isacharunderscore}def\ finestPartUnion\ \isacommand{by}\isamarkupfalse%
\ fastforce%
\endisatagproof
{\isafoldproof}%
%
\isadelimproof
\isanewline
%
\endisadelimproof
\isanewline
\isacommand{lemma}\isamarkupfalse%
\ lm{\isadigit{0}}{\isadigit{2}}{\isadigit{7}}{\isacharcolon}\ \isanewline
\ \ {\isachardoublequoteopen}inj{\isacharunderscore}on\ Union\ {\isacharparenleft}finestpart\ {\isacharbackquote}\ UNIV{\isacharparenright}{\isachardoublequoteclose}\ \isanewline
%
\isadelimproof
\ \ %
\endisadelimproof
%
\isatagproof
\isacommand{using}\isamarkupfalse%
\ assms\ lm{\isadigit{0}}{\isadigit{2}}{\isadigit{6}}\ \isacommand{by}\isamarkupfalse%
\ {\isacharparenleft}metis\ inj{\isacharunderscore}on{\isacharunderscore}id\ inj{\isacharunderscore}on{\isacharunderscore}imageI{\isacharparenright}%
\endisatagproof
{\isafoldproof}%
%
\isadelimproof
\isanewline
%
\endisadelimproof
\isanewline
\isacommand{lemma}\isamarkupfalse%
\ nonEqualitySetOfSets{\isacharcolon}\ \isanewline
\ \ \isakeyword{assumes}\ {\isachardoublequoteopen}X\ {\isasymnoteq}\ Y{\isachardoublequoteclose}\ \isanewline
\ \ \isakeyword{shows}\ {\isachardoublequoteopen}{\isacharbraceleft}{\isacharbraceleft}x{\isacharbraceright}{\isacharbar}\ x{\isachardot}\ x\ {\isasymin}\ X{\isacharbraceright}\ {\isasymnoteq}\ {\isacharbraceleft}{\isacharbraceleft}x{\isacharbraceright}{\isacharbar}\ x{\isachardot}\ x\ {\isasymin}\ Y{\isacharbraceright}{\isachardoublequoteclose}\ \isanewline
%
\isadelimproof
\ \ %
\endisadelimproof
%
\isatagproof
\isacommand{using}\isamarkupfalse%
\ assms\ \isacommand{by}\isamarkupfalse%
\ auto%
\endisatagproof
{\isafoldproof}%
%
\isadelimproof
\isanewline
%
\endisadelimproof
\isanewline
\isacommand{corollary}\isamarkupfalse%
\ lm{\isadigit{0}}{\isadigit{2}}{\isadigit{8}}{\isacharcolon}\ \isanewline
\ \ {\isachardoublequoteopen}inj{\isacharunderscore}on\ finestpart\ UNIV{\isachardoublequoteclose}\ \isanewline
%
\isadelimproof
\ \ %
\endisadelimproof
%
\isatagproof
\isacommand{using}\isamarkupfalse%
\ nonEqualitySetOfSets\ finestPart\ \isacommand{by}\isamarkupfalse%
\ {\isacharparenleft}metis\ {\isacharparenleft}lifting{\isacharcomma}\ no{\isacharunderscore}types{\isacharparenright}\ injI{\isacharparenright}%
\endisatagproof
{\isafoldproof}%
%
\isadelimproof
\isanewline
%
\endisadelimproof
\isanewline
\isanewline
\isacommand{lemma}\isamarkupfalse%
\ unionFinestPart{\isacharcolon}\ \isanewline
\ \ {\isachardoublequoteopen}{\isacharbraceleft}Y\ {\isacharbar}\ Y{\isachardot}\ EX\ x{\isachardot}{\isacharparenleft}{\isacharparenleft}Y\ {\isasymin}\ finestpart\ x{\isacharparenright}\ {\isacharampersand}\ {\isacharparenleft}x\ {\isasymin}\ X{\isacharparenright}{\isacharparenright}{\isacharbraceright}\ {\isacharequal}\ {\isasymUnion}{\isacharparenleft}finestpart{\isacharbackquote}X{\isacharparenright}{\isachardoublequoteclose}\ \isanewline
%
\isadelimproof
\ \ %
\endisadelimproof
%
\isatagproof
\isacommand{by}\isamarkupfalse%
\ auto%
\endisatagproof
{\isafoldproof}%
%
\isadelimproof
\isanewline
%
\endisadelimproof
\isanewline
\isanewline
\isacommand{lemma}\isamarkupfalse%
\ rangeSetOfPairs{\isacharcolon}\ \isanewline
\ \ {\isachardoublequoteopen}Range\ {\isacharbraceleft}{\isacharparenleft}fst\ pair{\isacharcomma}\ Y{\isacharparenright}{\isacharbar}\ Y\ pair{\isachardot}\ Y\ {\isasymin}\ finestpart\ {\isacharparenleft}snd\ pair{\isacharparenright}\ {\isacharampersand}\ pair\ {\isasymin}\ X{\isacharbraceright}\ {\isacharequal}\ \isanewline
\ \ \ {\isacharbraceleft}Y{\isachardot}\ EX\ x{\isachardot}\ {\isacharparenleft}{\isacharparenleft}Y\ {\isasymin}\ finestpart\ x{\isacharparenright}\ {\isacharampersand}\ {\isacharparenleft}x\ {\isasymin}\ Range\ X{\isacharparenright}{\isacharparenright}{\isacharbraceright}{\isachardoublequoteclose}\ \isanewline
%
\isadelimproof
\ \ %
\endisadelimproof
%
\isatagproof
\isacommand{by}\isamarkupfalse%
\ auto%
\endisatagproof
{\isafoldproof}%
%
\isadelimproof
\isanewline
%
\endisadelimproof
\isanewline
\isanewline
\isacommand{lemma}\isamarkupfalse%
\ setOfPairsEquality{\isacharcolon}\ \isanewline
\ \ {\isachardoublequoteopen}{\isacharbraceleft}{\isacharparenleft}fst\ pair{\isacharcomma}\ {\isacharbraceleft}y{\isacharbraceright}{\isacharparenright}{\isacharbar}\ y\ pair{\isachardot}\ y\ {\isasymin}\ snd\ pair\ {\isacharampersand}\ pair\ {\isasymin}\ X{\isacharbraceright}\ {\isacharequal}\ \isanewline
\ \ \ {\isacharbraceleft}{\isacharparenleft}fst\ pair{\isacharcomma}\ Y{\isacharparenright}{\isacharbar}\ Y\ pair{\isachardot}\ Y\ {\isasymin}\ finestpart\ {\isacharparenleft}snd\ pair{\isacharparenright}\ {\isacharampersand}\ pair\ {\isasymin}\ X{\isacharbraceright}{\isachardoublequoteclose}\isanewline
%
\isadelimproof
\ \ %
\endisadelimproof
%
\isatagproof
\isacommand{using}\isamarkupfalse%
\ finestpart{\isacharunderscore}def\ \isacommand{by}\isamarkupfalse%
\ fastforce%
\endisatagproof
{\isafoldproof}%
%
\isadelimproof
\isanewline
%
\endisadelimproof
\isanewline
\isacommand{lemma}\isamarkupfalse%
\ setOfPairs{\isacharcolon}\ \isanewline
\ \ {\isachardoublequoteopen}{\isacharbraceleft}{\isacharparenleft}fst\ pair{\isacharcomma}\ {\isacharbraceleft}y{\isacharbraceright}{\isacharparenright}{\isacharbar}\ y{\isachardot}\ y\ {\isasymin}\ \ snd\ pair{\isacharbraceright}\ {\isacharequal}\ \isanewline
\ \ \ {\isacharbraceleft}fst\ pair{\isacharbraceright}\ {\isasymtimes}\ {\isacharbraceleft}{\isacharbraceleft}y{\isacharbraceright}{\isacharbar}\ y{\isachardot}\ y\ {\isasymin}\ snd\ pair{\isacharbraceright}{\isachardoublequoteclose}\ \isanewline
%
\isadelimproof
\ \ %
\endisadelimproof
%
\isatagproof
\isacommand{by}\isamarkupfalse%
\ fastforce%
\endisatagproof
{\isafoldproof}%
%
\isadelimproof
\isanewline
%
\endisadelimproof
\isanewline
\isacommand{lemma}\isamarkupfalse%
\ lm{\isadigit{0}}{\isadigit{2}}{\isadigit{9}}{\isacharcolon}\ \isanewline
\ \ {\isachardoublequoteopen}x\ {\isasymin}\ X\ {\isacharequal}\ {\isacharparenleft}{\isacharbraceleft}x{\isacharbraceright}\ {\isasymin}\ finestpart\ X{\isacharparenright}{\isachardoublequoteclose}\ \isanewline
%
\isadelimproof
\ \ %
\endisadelimproof
%
\isatagproof
\isacommand{using}\isamarkupfalse%
\ finestpart{\isacharunderscore}def\ \isacommand{by}\isamarkupfalse%
\ force%
\endisatagproof
{\isafoldproof}%
%
\isadelimproof
\isanewline
%
\endisadelimproof
\isanewline
\isacommand{lemma}\isamarkupfalse%
\ pairDifference{\isacharcolon}\ \isanewline
\ \ {\isachardoublequoteopen}{\isacharbraceleft}{\isacharparenleft}x{\isacharcomma}X{\isacharparenright}{\isacharbraceright}{\isacharminus}{\isacharbraceleft}{\isacharparenleft}x{\isacharcomma}Y{\isacharparenright}{\isacharbraceright}\ {\isacharequal}\ {\isacharbraceleft}x{\isacharbraceright}{\isasymtimes}{\isacharparenleft}{\isacharbraceleft}X{\isacharbraceright}{\isacharminus}{\isacharbraceleft}Y{\isacharbraceright}{\isacharparenright}{\isachardoublequoteclose}\ \isanewline
%
\isadelimproof
\ \ %
\endisadelimproof
%
\isatagproof
\isacommand{by}\isamarkupfalse%
\ blast%
\endisatagproof
{\isafoldproof}%
%
\isadelimproof
\isanewline
%
\endisadelimproof
\isanewline
\isacommand{lemma}\isamarkupfalse%
\ lm{\isadigit{0}}{\isadigit{3}}{\isadigit{0}}{\isacharcolon}\ \isanewline
\ \ \isakeyword{assumes}\ {\isachardoublequoteopen}{\isasymUnion}\ P\ {\isacharequal}\ X{\isachardoublequoteclose}\ \isanewline
\ \ \isakeyword{shows}\ {\isachardoublequoteopen}P\ {\isasymsubseteq}\ Pow\ X{\isachardoublequoteclose}\ \isanewline
%
\isadelimproof
\ \ %
\endisadelimproof
%
\isatagproof
\isacommand{using}\isamarkupfalse%
\ assms\ \isacommand{by}\isamarkupfalse%
\ blast%
\endisatagproof
{\isafoldproof}%
%
\isadelimproof
\isanewline
%
\endisadelimproof
\isanewline
\isacommand{lemma}\isamarkupfalse%
\ lm{\isadigit{0}}{\isadigit{3}}{\isadigit{1}}{\isacharcolon}\ \isanewline
\ \ {\isachardoublequoteopen}argmax\ f\ {\isacharbraceleft}x{\isacharbraceright}\ {\isacharequal}\ {\isacharbraceleft}x{\isacharbraceright}{\isachardoublequoteclose}\ \isanewline
%
\isadelimproof
\ \ %
\endisadelimproof
%
\isatagproof
\isacommand{using}\isamarkupfalse%
\ argmax{\isacharunderscore}def\ \isacommand{by}\isamarkupfalse%
\ auto%
\endisatagproof
{\isafoldproof}%
%
\isadelimproof
\isanewline
%
\endisadelimproof
\isanewline
\isacommand{lemma}\isamarkupfalse%
\ sortingSameSet{\isacharcolon}\ \isanewline
\ \ \isakeyword{assumes}\ {\isachardoublequoteopen}finite\ X{\isachardoublequoteclose}\ \isanewline
\ \ \isakeyword{shows}\ {\isachardoublequoteopen}set\ {\isacharparenleft}sorted{\isacharunderscore}list{\isacharunderscore}of{\isacharunderscore}set\ X{\isacharparenright}\ {\isacharequal}\ X{\isachardoublequoteclose}\ \isanewline
%
\isadelimproof
\ \ %
\endisadelimproof
%
\isatagproof
\isacommand{using}\isamarkupfalse%
\ assms\ \isacommand{by}\isamarkupfalse%
\ simp%
\endisatagproof
{\isafoldproof}%
%
\isadelimproof
\isanewline
%
\endisadelimproof
\isanewline
\isanewline
\isacommand{lemma}\isamarkupfalse%
\ lm{\isadigit{0}}{\isadigit{3}}{\isadigit{2}}{\isacharcolon}\ \isanewline
\ \ \isakeyword{assumes}\ {\isachardoublequoteopen}finite\ A{\isachardoublequoteclose}\ \isanewline
\ \ \isakeyword{shows}\ {\isachardoublequoteopen}setsum\ f\ A\ {\isacharequal}\ setsum\ f\ {\isacharparenleft}A\ {\isasyminter}\ B{\isacharparenright}\ {\isacharplus}\ setsum\ f\ {\isacharparenleft}A\ {\isacharminus}\ B{\isacharparenright}{\isachardoublequoteclose}\ \isanewline
%
\isadelimproof
\ \ %
\endisadelimproof
%
\isatagproof
\isacommand{using}\isamarkupfalse%
\ assms\ \isacommand{by}\isamarkupfalse%
\ {\isacharparenleft}metis\ DiffD{\isadigit{2}}\ Int{\isacharunderscore}iff\ Un{\isacharunderscore}Diff{\isacharunderscore}Int\ Un{\isacharunderscore}commute\ finite{\isacharunderscore}Un\ setsum{\isachardot}union{\isacharunderscore}inter{\isacharunderscore}neutral{\isacharparenright}%
\endisatagproof
{\isafoldproof}%
%
\isadelimproof
\isanewline
%
\endisadelimproof
\isanewline
\isacommand{corollary}\isamarkupfalse%
\ setsumOutside{\isacharcolon}\ \isanewline
\ \ \isakeyword{assumes}\ {\isachardoublequoteopen}finite\ g{\isachardoublequoteclose}\ \isanewline
\ \ \isakeyword{shows}\ {\isachardoublequoteopen}setsum\ f\ g\ {\isacharequal}\ setsum\ f\ {\isacharparenleft}g\ outside\ X{\isacharparenright}\ {\isacharplus}\ {\isacharparenleft}setsum\ f\ {\isacharparenleft}g{\isacharbar}{\isacharbar}X{\isacharparenright}{\isacharparenright}{\isachardoublequoteclose}\ \isanewline
%
\isadelimproof
\ \ %
\endisadelimproof
%
\isatagproof
\isacommand{unfolding}\isamarkupfalse%
\ Outside{\isacharunderscore}def\ restrict{\isacharunderscore}def\ \isacommand{using}\isamarkupfalse%
\ assms\ add{\isachardot}commute\ inf{\isacharunderscore}commute\ lm{\isadigit{0}}{\isadigit{3}}{\isadigit{2}}\ \isacommand{by}\isamarkupfalse%
\ {\isacharparenleft}metis{\isacharparenright}%
\endisatagproof
{\isafoldproof}%
%
\isadelimproof
\isanewline
%
\endisadelimproof
\isanewline
\isacommand{lemma}\isamarkupfalse%
\ lm{\isadigit{0}}{\isadigit{3}}{\isadigit{3}}{\isacharcolon}\ \isanewline
\ \ \isakeyword{assumes}\ {\isachardoublequoteopen}{\isacharparenleft}Domain\ P\ {\isasymsubseteq}\ Domain\ Q{\isacharparenright}{\isachardoublequoteclose}\ \isanewline
\ \ \isakeyword{shows}\ {\isachardoublequoteopen}{\isacharparenleft}P\ {\isacharplus}{\isacharasterisk}\ Q{\isacharparenright}\ {\isacharequal}\ Q{\isachardoublequoteclose}\isanewline
%
\isadelimproof
\ \ %
\endisadelimproof
%
\isatagproof
\isacommand{unfolding}\isamarkupfalse%
\ paste{\isacharunderscore}def\ Outside{\isacharunderscore}def\ \isacommand{using}\isamarkupfalse%
\ assms\ \isacommand{by}\isamarkupfalse%
\ fast%
\endisatagproof
{\isafoldproof}%
%
\isadelimproof
\isanewline
%
\endisadelimproof
\isanewline
\isacommand{lemma}\isamarkupfalse%
\ lm{\isadigit{0}}{\isadigit{3}}{\isadigit{4}}{\isacharcolon}\ \isanewline
\ \ \isakeyword{assumes}\ {\isachardoublequoteopen}{\isacharparenleft}P\ {\isacharplus}{\isacharasterisk}\ Q{\isacharequal}Q{\isacharparenright}{\isachardoublequoteclose}\ \isanewline
\ \ \isakeyword{shows}\ {\isachardoublequoteopen}{\isacharparenleft}Domain\ P\ {\isasymsubseteq}\ Domain\ Q{\isacharparenright}{\isachardoublequoteclose}\isanewline
%
\isadelimproof
\ \ %
\endisadelimproof
%
\isatagproof
\isacommand{using}\isamarkupfalse%
\ assms\ paste{\isacharunderscore}def\ Outside{\isacharunderscore}def\ \isacommand{by}\isamarkupfalse%
\ blast%
\endisatagproof
{\isafoldproof}%
%
\isadelimproof
\isanewline
%
\endisadelimproof
\isanewline
\isacommand{lemma}\isamarkupfalse%
\ lm{\isadigit{0}}{\isadigit{3}}{\isadigit{5}}{\isacharcolon}\ \isanewline
\ \ {\isachardoublequoteopen}{\isacharparenleft}Domain\ P\ {\isasymsubseteq}\ Domain\ Q{\isacharparenright}\ {\isacharequal}\ {\isacharparenleft}P\ {\isacharplus}{\isacharasterisk}\ Q{\isacharequal}Q{\isacharparenright}{\isachardoublequoteclose}\ \isanewline
%
\isadelimproof
\ \ %
\endisadelimproof
%
\isatagproof
\isacommand{using}\isamarkupfalse%
\ lm{\isadigit{0}}{\isadigit{3}}{\isadigit{3}}\ lm{\isadigit{0}}{\isadigit{3}}{\isadigit{4}}\ \isacommand{by}\isamarkupfalse%
\ metis%
\endisatagproof
{\isafoldproof}%
%
\isadelimproof
\isanewline
%
\endisadelimproof
\isanewline
\isacommand{lemma}\isamarkupfalse%
\ \isanewline
\ \ {\isachardoublequoteopen}{\isacharparenleft}P{\isacharbar}{\isacharbar}{\isacharparenleft}Domain\ Q{\isacharparenright}{\isacharparenright}\ {\isacharplus}{\isacharasterisk}\ Q\ {\isacharequal}\ Q{\isachardoublequoteclose}\ \isanewline
%
\isadelimproof
\ \ %
\endisadelimproof
%
\isatagproof
\isacommand{by}\isamarkupfalse%
\ {\isacharparenleft}metis\ Int{\isacharunderscore}lower{\isadigit{2}}\ restrictedDomain\ lm{\isadigit{0}}{\isadigit{3}}{\isadigit{5}}{\isacharparenright}%
\endisatagproof
{\isafoldproof}%
%
\isadelimproof
\isanewline
%
\endisadelimproof
\isanewline
\isacommand{lemma}\isamarkupfalse%
\ lm{\isadigit{0}}{\isadigit{3}}{\isadigit{6}}{\isacharcolon}\ \isanewline
\ \ {\isachardoublequoteopen}P{\isacharbar}{\isacharbar}X\ \ \ {\isacharequal}\ \ \ P\ outside\ {\isacharparenleft}Domain\ P\ {\isacharminus}\ X{\isacharparenright}{\isachardoublequoteclose}\ \isanewline
%
\isadelimproof
\ \ %
\endisadelimproof
%
\isatagproof
\isacommand{using}\isamarkupfalse%
\ Outside{\isacharunderscore}def\ restrict{\isacharunderscore}def\ \isacommand{by}\isamarkupfalse%
\ fastforce%
\endisatagproof
{\isafoldproof}%
%
\isadelimproof
\isanewline
%
\endisadelimproof
\isanewline
\isacommand{lemma}\isamarkupfalse%
\ lm{\isadigit{0}}{\isadigit{3}}{\isadigit{7}}{\isacharcolon}\ \isanewline
\ \ {\isachardoublequoteopen}{\isacharparenleft}P\ outside\ X{\isacharparenright}\ {\isasymsubseteq}\ \ \ \ P\ {\isacharbar}{\isacharbar}\ {\isacharparenleft}{\isacharparenleft}Domain\ P{\isacharparenright}{\isacharminus}X{\isacharparenright}{\isachardoublequoteclose}\ \isanewline
%
\isadelimproof
\ \ %
\endisadelimproof
%
\isatagproof
\isacommand{using}\isamarkupfalse%
\ lm{\isadigit{0}}{\isadigit{3}}{\isadigit{6}}\ lm{\isadigit{0}}{\isadigit{1}}{\isadigit{6}}\ \isacommand{by}\isamarkupfalse%
\ {\isacharparenleft}metis\ Int{\isacharunderscore}commute\ restrictedDomain\ outside{\isacharunderscore}reduces{\isacharunderscore}domain{\isacharparenright}%
\endisatagproof
{\isafoldproof}%
%
\isadelimproof
\isanewline
%
\endisadelimproof
\isanewline
\isacommand{lemma}\isamarkupfalse%
\ lm{\isadigit{0}}{\isadigit{3}}{\isadigit{8}}{\isacharcolon}\ \isanewline
\ \ {\isachardoublequoteopen}Domain\ {\isacharparenleft}P\ outside\ X{\isacharparenright}\ {\isasyminter}\ Domain\ {\isacharparenleft}Q\ {\isacharbar}{\isacharbar}\ X{\isacharparenright}\ {\isacharequal}\ {\isacharbraceleft}{\isacharbraceright}{\isachardoublequoteclose}\ \isanewline
%
\isadelimproof
\ \ %
\endisadelimproof
%
\isatagproof
\isacommand{using}\isamarkupfalse%
\ lm{\isadigit{0}}{\isadigit{3}}{\isadigit{6}}\isanewline
\ \ \isacommand{by}\isamarkupfalse%
\ {\isacharparenleft}metis\ Diff{\isacharunderscore}disjoint\ Domain{\isacharunderscore}empty{\isacharunderscore}iff\ Int{\isacharunderscore}Diff\ inf{\isacharunderscore}commute\ restrictedDomain\ \ \ \ \ \isanewline
\ \ \ \ \ \ \ \ \ \ \ \ outside{\isacharunderscore}reduces{\isacharunderscore}domain\ restrict{\isacharunderscore}empty{\isacharparenright}%
\endisatagproof
{\isafoldproof}%
%
\isadelimproof
\isanewline
%
\endisadelimproof
\isanewline
\isacommand{lemma}\isamarkupfalse%
\ lm{\isadigit{0}}{\isadigit{3}}{\isadigit{9}}{\isacharcolon}\ \isanewline
\ \ {\isachardoublequoteopen}{\isacharparenleft}P\ outside\ X{\isacharparenright}\ {\isasyminter}\ {\isacharparenleft}Q\ {\isacharbar}{\isacharbar}\ X{\isacharparenright}\ {\isacharequal}\ {\isacharbraceleft}{\isacharbraceright}{\isachardoublequoteclose}\ \isanewline
%
\isadelimproof
\ \ %
\endisadelimproof
%
\isatagproof
\isacommand{using}\isamarkupfalse%
\ lm{\isadigit{0}}{\isadigit{3}}{\isadigit{8}}\ \isacommand{by}\isamarkupfalse%
\ fast%
\endisatagproof
{\isafoldproof}%
%
\isadelimproof
\isanewline
%
\endisadelimproof
\isanewline
\isacommand{lemma}\isamarkupfalse%
\ lm{\isadigit{0}}{\isadigit{4}}{\isadigit{0}}{\isacharcolon}\ \isanewline
\ \ {\isachardoublequoteopen}{\isacharparenleft}P\ outside\ {\isacharparenleft}X\ {\isasymunion}\ Y{\isacharparenright}{\isacharparenright}\ {\isasyminter}\ {\isacharparenleft}Q\ {\isacharbar}{\isacharbar}\ X{\isacharparenright}\ {\isacharequal}\ {\isacharbraceleft}{\isacharbraceright}\ \ \ {\isacharampersand}\ \ \ {\isacharparenleft}P\ outside\ X{\isacharparenright}\ {\isasyminter}\ {\isacharparenleft}Q\ {\isacharbar}{\isacharbar}\ {\isacharparenleft}X\ {\isasyminter}\ Z{\isacharparenright}{\isacharparenright}\ {\isacharequal}\ {\isacharbraceleft}{\isacharbraceright}{\isachardoublequoteclose}\ \isanewline
%
\isadelimproof
\ \ %
\endisadelimproof
%
\isatagproof
\isacommand{using}\isamarkupfalse%
\ assms\ Outside{\isacharunderscore}def\ restrict{\isacharunderscore}def\ lm{\isadigit{0}}{\isadigit{3}}{\isadigit{9}}\ lm{\isadigit{0}}{\isadigit{1}}{\isadigit{5}}\ \isacommand{by}\isamarkupfalse%
\ fast%
\endisatagproof
{\isafoldproof}%
%
\isadelimproof
\isanewline
%
\endisadelimproof
\isanewline
\isacommand{lemma}\isamarkupfalse%
\ lm{\isadigit{0}}{\isadigit{4}}{\isadigit{1}}{\isacharcolon}\ \isanewline
\ \ {\isachardoublequoteopen}P\ outside\ X\ \ \ \ {\isacharequal}\ \ \ \ P\ {\isacharbar}{\isacharbar}\ {\isacharparenleft}{\isacharparenleft}Domain\ P{\isacharparenright}\ {\isacharminus}\ X{\isacharparenright}{\isachardoublequoteclose}\ \isanewline
%
\isadelimproof
\ \ %
\endisadelimproof
%
\isatagproof
\isacommand{using}\isamarkupfalse%
\ Outside{\isacharunderscore}def\ restrict{\isacharunderscore}def\ \ lm{\isadigit{0}}{\isadigit{3}}{\isadigit{7}}\ \isacommand{by}\isamarkupfalse%
\ fast%
\endisatagproof
{\isafoldproof}%
%
\isadelimproof
\isanewline
%
\endisadelimproof
\isanewline
\isacommand{lemma}\isamarkupfalse%
\ lm{\isadigit{0}}{\isadigit{4}}{\isadigit{2}}{\isacharcolon}\ \isanewline
\ \ {\isachardoublequoteopen}R{\isacharbackquote}{\isacharbackquote}{\isacharparenleft}X{\isacharminus}Y{\isacharparenright}\ {\isacharequal}\ {\isacharparenleft}R{\isacharbar}{\isacharbar}X{\isacharparenright}{\isacharbackquote}{\isacharbackquote}{\isacharparenleft}X{\isacharminus}Y{\isacharparenright}{\isachardoublequoteclose}\ \isanewline
%
\isadelimproof
\ \ %
\endisadelimproof
%
\isatagproof
\isacommand{using}\isamarkupfalse%
\ restrict{\isacharunderscore}def\ \isacommand{by}\isamarkupfalse%
\ blast%
\endisatagproof
{\isafoldproof}%
%
\isadelimproof
\isanewline
%
\endisadelimproof
\isanewline
\isanewline
\isacommand{lemma}\isamarkupfalse%
\ lm{\isadigit{0}}{\isadigit{4}}{\isadigit{3}}{\isacharcolon}\ \isanewline
\ \ \isakeyword{assumes}\ {\isachardoublequoteopen}{\isasymUnion}\ XX\ {\isasymsubseteq}\ X{\isachardoublequoteclose}\ {\isachardoublequoteopen}x\ {\isasymin}\ XX{\isachardoublequoteclose}\ {\isachardoublequoteopen}x\ {\isasymnoteq}\ {\isacharbraceleft}{\isacharbraceright}{\isachardoublequoteclose}\ \isanewline
\ \ \isakeyword{shows}\ {\isachardoublequoteopen}x\ {\isasyminter}\ X\ {\isasymnoteq}\ {\isacharbraceleft}{\isacharbraceright}{\isachardoublequoteclose}\ \isanewline
%
\isadelimproof
\ \ %
\endisadelimproof
%
\isatagproof
\isacommand{using}\isamarkupfalse%
\ assms\ \isacommand{by}\isamarkupfalse%
\ blast%
\endisatagproof
{\isafoldproof}%
%
\isadelimproof
\isanewline
%
\endisadelimproof
\isanewline
\isanewline
\isacommand{lemma}\isamarkupfalse%
\ lm{\isadigit{0}}{\isadigit{4}}{\isadigit{4}}{\isacharcolon}\ \isanewline
\ \ \isakeyword{assumes}\ {\isachardoublequoteopen}{\isasymforall}l\ {\isasymin}\ set\ L{\isadigit{1}}{\isachardot}\ set\ L{\isadigit{2}}\ {\isacharequal}\ f{\isadigit{2}}\ {\isacharparenleft}set\ l{\isacharparenright}\ N{\isachardoublequoteclose}\ \isanewline
\ \ \isakeyword{shows}\ {\isachardoublequoteopen}set\ {\isacharbrackleft}set\ L{\isadigit{2}}{\isachardot}\ l\ {\isacharless}{\isacharminus}\ L{\isadigit{1}}{\isacharbrackright}\ \ {\isacharequal}\ \ {\isacharbraceleft}f{\isadigit{2}}\ P\ N{\isacharbar}\ P{\isachardot}\ P\ {\isasymin}\ set\ {\isacharparenleft}map\ set\ L{\isadigit{1}}{\isacharparenright}{\isacharbraceright}{\isachardoublequoteclose}\ \isanewline
%
\isadelimproof
\ \ %
\endisadelimproof
%
\isatagproof
\isacommand{using}\isamarkupfalse%
\ assms\ \isacommand{by}\isamarkupfalse%
\ auto%
\endisatagproof
{\isafoldproof}%
%
\isadelimproof
\isanewline
%
\endisadelimproof
\isanewline
\isanewline
\isacommand{lemma}\isamarkupfalse%
\ setVsList{\isacharcolon}\ \isanewline
\ \ \isakeyword{assumes}\ {\isachardoublequoteopen}{\isasymforall}l\ {\isasymin}\ set\ {\isacharparenleft}g{\isadigit{1}}\ G{\isacharparenright}{\isachardot}\ set\ {\isacharparenleft}g{\isadigit{2}}\ l\ N{\isacharparenright}\ {\isacharequal}\ f{\isadigit{2}}\ {\isacharparenleft}set\ l{\isacharparenright}\ N{\isachardoublequoteclose}\ \isanewline
\ \ \isakeyword{shows}\ {\isachardoublequoteopen}set\ {\isacharbrackleft}set\ {\isacharparenleft}g{\isadigit{2}}\ l\ N{\isacharparenright}{\isachardot}\ l\ {\isacharless}{\isacharminus}\ {\isacharparenleft}g{\isadigit{1}}\ G{\isacharparenright}{\isacharbrackright}\ \ {\isacharequal}\ \ {\isacharbraceleft}f{\isadigit{2}}\ P\ N{\isacharbar}\ P{\isachardot}\ P\ {\isasymin}\ set\ {\isacharparenleft}map\ set\ {\isacharparenleft}g{\isadigit{1}}\ G{\isacharparenright}{\isacharparenright}{\isacharbraceright}{\isachardoublequoteclose}\ \isanewline
%
\isadelimproof
\ \ %
\endisadelimproof
%
\isatagproof
\isacommand{using}\isamarkupfalse%
\ assms\ \isacommand{by}\isamarkupfalse%
\ auto%
\endisatagproof
{\isafoldproof}%
%
\isadelimproof
\isanewline
%
\endisadelimproof
\isanewline
\isacommand{lemma}\isamarkupfalse%
\ lm{\isadigit{0}}{\isadigit{4}}{\isadigit{5}}{\isacharcolon}\ \isanewline
\ \ {\isachardoublequoteopen}{\isacharparenleft}{\isasymforall}l\ {\isasymin}\ set\ {\isacharparenleft}g{\isadigit{1}}\ G{\isacharparenright}{\isachardot}\ set\ {\isacharparenleft}g{\isadigit{2}}\ l\ N{\isacharparenright}\ {\isacharequal}\ f{\isadigit{2}}\ {\isacharparenleft}set\ l{\isacharparenright}\ N{\isacharparenright}\ {\isacharminus}{\isacharminus}{\isachargreater}\ \ \isanewline
\ \ \ \ \ {\isacharbraceleft}f{\isadigit{2}}\ P\ N{\isacharbar}\ P{\isachardot}\ P\ {\isasymin}\ set\ {\isacharparenleft}map\ set\ {\isacharparenleft}g{\isadigit{1}}\ G{\isacharparenright}{\isacharparenright}{\isacharbraceright}\ {\isacharequal}\ set\ {\isacharbrackleft}set\ {\isacharparenleft}g{\isadigit{2}}\ l\ N{\isacharparenright}{\isachardot}\ l\ {\isacharless}{\isacharminus}\ g{\isadigit{1}}\ G{\isacharbrackright}{\isachardoublequoteclose}\ \isanewline
%
\isadelimproof
\ \ %
\endisadelimproof
%
\isatagproof
\isacommand{by}\isamarkupfalse%
\ auto%
\endisatagproof
{\isafoldproof}%
%
\isadelimproof
\isanewline
%
\endisadelimproof
\isanewline
\isacommand{lemma}\isamarkupfalse%
\ lm{\isadigit{0}}{\isadigit{4}}{\isadigit{6}}{\isacharcolon}\ \isanewline
\ \ \isakeyword{assumes}\ {\isachardoublequoteopen}X\ {\isasyminter}\ Y\ \ {\isacharequal}\ \ {\isacharbraceleft}{\isacharbraceright}{\isachardoublequoteclose}\ \isanewline
\ \ \isakeyword{shows}\ {\isachardoublequoteopen}R{\isacharbackquote}{\isacharbackquote}X\ {\isacharequal}\ {\isacharparenleft}R\ outside\ Y{\isacharparenright}{\isacharbackquote}{\isacharbackquote}X{\isachardoublequoteclose}\isanewline
%
\isadelimproof
\ \ %
\endisadelimproof
%
\isatagproof
\isacommand{using}\isamarkupfalse%
\ assms\ Outside{\isacharunderscore}def\ Image{\isacharunderscore}def\ \isacommand{by}\isamarkupfalse%
\ blast%
\endisatagproof
{\isafoldproof}%
%
\isadelimproof
\isanewline
%
\endisadelimproof
\isanewline
\isacommand{lemma}\isamarkupfalse%
\ lm{\isadigit{0}}{\isadigit{4}}{\isadigit{7}}{\isacharcolon}\ \isanewline
\ \ \isakeyword{assumes}\ {\isachardoublequoteopen}{\isacharparenleft}Range\ P{\isacharparenright}\ {\isasyminter}\ {\isacharparenleft}Range\ Q{\isacharparenright}\ {\isacharequal}\ {\isacharbraceleft}{\isacharbraceright}{\isachardoublequoteclose}\ {\isachardoublequoteopen}runiq\ {\isacharparenleft}P{\isacharcircum}{\isacharminus}{\isadigit{1}}{\isacharparenright}{\isachardoublequoteclose}\ {\isachardoublequoteopen}runiq\ {\isacharparenleft}Q{\isacharcircum}{\isacharminus}{\isadigit{1}}{\isacharparenright}{\isachardoublequoteclose}\ \isanewline
\ \ \isakeyword{shows}\ {\isachardoublequoteopen}runiq\ {\isacharparenleft}{\isacharparenleft}P\ {\isasymunion}\ Q{\isacharparenright}{\isacharcircum}{\isacharminus}{\isadigit{1}}{\isacharparenright}{\isachardoublequoteclose}\isanewline
%
\isadelimproof
\ \ %
\endisadelimproof
%
\isatagproof
\isacommand{using}\isamarkupfalse%
\ assms\ \isacommand{by}\isamarkupfalse%
\ {\isacharparenleft}metis\ Domain{\isacharunderscore}converse\ converse{\isacharunderscore}Un\ disj{\isacharunderscore}Un{\isacharunderscore}runiq{\isacharparenright}%
\endisatagproof
{\isafoldproof}%
%
\isadelimproof
\isanewline
%
\endisadelimproof
\isanewline
\isacommand{lemma}\isamarkupfalse%
\ lm{\isadigit{0}}{\isadigit{4}}{\isadigit{8}}{\isacharcolon}\ \isanewline
\ \ \isakeyword{assumes}\ {\isachardoublequoteopen}{\isacharparenleft}Range\ P{\isacharparenright}\ {\isasyminter}\ {\isacharparenleft}Range\ Q{\isacharparenright}\ {\isacharequal}\ {\isacharbraceleft}{\isacharbraceright}{\isachardoublequoteclose}\ {\isachardoublequoteopen}runiq\ {\isacharparenleft}P{\isacharcircum}{\isacharminus}{\isadigit{1}}{\isacharparenright}{\isachardoublequoteclose}\ {\isachardoublequoteopen}runiq\ {\isacharparenleft}Q{\isacharcircum}{\isacharminus}{\isadigit{1}}{\isacharparenright}{\isachardoublequoteclose}\ \isanewline
\ \ \isakeyword{shows}\ {\isachardoublequoteopen}runiq\ {\isacharparenleft}{\isacharparenleft}P\ {\isacharplus}{\isacharasterisk}\ Q{\isacharparenright}{\isacharcircum}{\isacharminus}{\isadigit{1}}{\isacharparenright}{\isachardoublequoteclose}\isanewline
%
\isadelimproof
\ \ %
\endisadelimproof
%
\isatagproof
\isacommand{using}\isamarkupfalse%
\ lm{\isadigit{0}}{\isadigit{4}}{\isadigit{7}}\ assms\ subrel{\isacharunderscore}runiq\ \isacommand{by}\isamarkupfalse%
\ {\isacharparenleft}metis\ converse{\isacharunderscore}converse\ converse{\isacharunderscore}subset{\isacharunderscore}swap\ paste{\isacharunderscore}sub{\isacharunderscore}Un{\isacharparenright}%
\endisatagproof
{\isafoldproof}%
%
\isadelimproof
\isanewline
%
\endisadelimproof
\isanewline
\isacommand{lemma}\isamarkupfalse%
\ lm{\isadigit{0}}{\isadigit{4}}{\isadigit{9}}{\isacharcolon}\ \isanewline
\ \ \isakeyword{assumes}\ {\isachardoublequoteopen}runiq\ R{\isachardoublequoteclose}\ \isanewline
\ \ \isakeyword{shows}\ {\isachardoublequoteopen}card\ {\isacharparenleft}R\ {\isacharbackquote}{\isacharbackquote}\ {\isacharbraceleft}a{\isacharbraceright}{\isacharparenright}\ {\isacharequal}\ {\isadigit{1}}\ {\isasymlongleftrightarrow}\ a\ {\isasymin}\ Domain\ R{\isachardoublequoteclose}\isanewline
%
\isadelimproof
\ \ %
\endisadelimproof
%
\isatagproof
\isacommand{using}\isamarkupfalse%
\ assms\ card{\isacharunderscore}Suc{\isacharunderscore}eq\ One{\isacharunderscore}nat{\isacharunderscore}def\ \ \isanewline
\ \ \isacommand{by}\isamarkupfalse%
\ {\isacharparenleft}metis\ Image{\isacharunderscore}within{\isacharunderscore}domain{\isacharprime}\ Suc{\isacharunderscore}neq{\isacharunderscore}Zero\ assms\ rightUniqueSetCardinality{\isacharparenright}%
\endisatagproof
{\isafoldproof}%
%
\isadelimproof
\isanewline
%
\endisadelimproof
\isanewline
\isanewline
\isacommand{lemma}\isamarkupfalse%
\ lm{\isadigit{0}}{\isadigit{5}}{\isadigit{0}}{\isacharcolon}\ \isanewline
\ \ {\isachardoublequoteopen}inj{\isacharunderscore}on\ \ {\isacharparenleft}{\isacharpercent}a{\isachardot}\ {\isacharparenleft}{\isacharparenleft}fst\ a{\isacharcomma}\ fst\ {\isacharparenleft}snd\ a{\isacharparenright}{\isacharparenright}{\isacharcomma}\ snd\ {\isacharparenleft}snd\ a{\isacharparenright}{\isacharparenright}{\isacharparenright}\ UNIV{\isachardoublequoteclose}\ \isanewline
%
\isadelimproof
\ \ %
\endisadelimproof
%
\isatagproof
\isacommand{by}\isamarkupfalse%
\ {\isacharparenleft}metis\ {\isacharparenleft}lifting{\isacharcomma}\ mono{\isacharunderscore}tags{\isacharparenright}\ Pair{\isacharunderscore}fst{\isacharunderscore}snd{\isacharunderscore}eq\ Pair{\isacharunderscore}inject\ injI{\isacharparenright}%
\endisatagproof
{\isafoldproof}%
%
\isadelimproof
\isanewline
%
\endisadelimproof
\isanewline
\isacommand{lemma}\isamarkupfalse%
\ lm{\isadigit{0}}{\isadigit{5}}{\isadigit{1}}{\isacharcolon}\ \isanewline
\ \ \isakeyword{assumes}\ {\isachardoublequoteopen}finite\ X{\isachardoublequoteclose}\ {\isachardoublequoteopen}x\ {\isachargreater}\ Max\ X{\isachardoublequoteclose}\ \isanewline
\ \ \isakeyword{shows}\ {\isachardoublequoteopen}x\ {\isasymnotin}\ X{\isachardoublequoteclose}\ \isanewline
%
\isadelimproof
\ \ %
\endisadelimproof
%
\isatagproof
\isacommand{using}\isamarkupfalse%
\ assms\ Max{\isachardot}coboundedI\ \isacommand{by}\isamarkupfalse%
\ {\isacharparenleft}metis\ leD{\isacharparenright}%
\endisatagproof
{\isafoldproof}%
%
\isadelimproof
\isanewline
%
\endisadelimproof
\isanewline
\isacommand{lemma}\isamarkupfalse%
\ lm{\isadigit{0}}{\isadigit{5}}{\isadigit{2}}{\isacharcolon}\ \isanewline
\ \ \isakeyword{assumes}\ {\isachardoublequoteopen}finite\ A{\isachardoublequoteclose}\ {\isachardoublequoteopen}A\ {\isasymnoteq}\ {\isacharbraceleft}{\isacharbraceright}{\isachardoublequoteclose}\ \isanewline
\ \ \isakeyword{shows}\ {\isachardoublequoteopen}Max\ {\isacharparenleft}f{\isacharbackquote}A{\isacharparenright}\ {\isasymin}\ f{\isacharbackquote}A{\isachardoublequoteclose}\ \isanewline
%
\isadelimproof
\ \ %
\endisadelimproof
%
\isatagproof
\isacommand{using}\isamarkupfalse%
\ assms\ \isacommand{by}\isamarkupfalse%
\ {\isacharparenleft}metis\ Max{\isacharunderscore}in\ finite{\isacharunderscore}imageI\ image{\isacharunderscore}is{\isacharunderscore}empty{\isacharparenright}%
\endisatagproof
{\isafoldproof}%
%
\isadelimproof
\isanewline
%
\endisadelimproof
\isanewline
\isanewline
\isacommand{lemma}\isamarkupfalse%
\ lm{\isadigit{0}}{\isadigit{5}}{\isadigit{3}}{\isacharcolon}\ \isanewline
\ \ {\isachardoublequoteopen}argmax\ f\ A\ {\isasymsubseteq}\ f\ {\isacharminus}{\isacharbackquote}\ {\isacharbraceleft}Max\ {\isacharparenleft}f\ {\isacharbackquote}\ A{\isacharparenright}{\isacharbraceright}{\isachardoublequoteclose}\ \isanewline
%
\isadelimproof
\ \ %
\endisadelimproof
%
\isatagproof
\isacommand{by}\isamarkupfalse%
\ force%
\endisatagproof
{\isafoldproof}%
%
\isadelimproof
\isanewline
%
\endisadelimproof
\isanewline
\isacommand{lemma}\isamarkupfalse%
\ lm{\isadigit{0}}{\isadigit{5}}{\isadigit{4}}{\isacharcolon}\ \isanewline
\ \ {\isachardoublequoteopen}argmax\ f\ A\ {\isacharequal}\ A\ {\isasyminter}\ {\isacharbraceleft}\ x\ {\isachardot}\ f\ x\ {\isacharequal}\ Max\ {\isacharparenleft}f\ {\isacharbackquote}\ A{\isacharparenright}\ {\isacharbraceright}{\isachardoublequoteclose}\ \isanewline
%
\isadelimproof
\ \ %
\endisadelimproof
%
\isatagproof
\isacommand{by}\isamarkupfalse%
\ auto%
\endisatagproof
{\isafoldproof}%
%
\isadelimproof
\isanewline
%
\endisadelimproof
\isanewline
\isacommand{lemma}\isamarkupfalse%
\ lm{\isadigit{0}}{\isadigit{5}}{\isadigit{5}}{\isacharcolon}\ \isanewline
\ \ {\isachardoublequoteopen}{\isacharparenleft}x\ {\isasymin}\ argmax\ f\ X{\isacharparenright}\ {\isacharequal}\ {\isacharparenleft}x\ {\isasymin}\ X\ {\isacharampersand}\ f\ x\ {\isacharequal}\ Max\ {\isacharparenleft}f\ {\isacharbackquote}\ X{\isacharparenright}{\isacharparenright}{\isachardoublequoteclose}\ \isanewline
%
\isadelimproof
\ \ %
\endisadelimproof
%
\isatagproof
\isacommand{using}\isamarkupfalse%
\ argmax{\isachardot}simps\ mem{\isacharunderscore}Collect{\isacharunderscore}eq\ \isacommand{by}\isamarkupfalse%
\ {\isacharparenleft}metis\ {\isacharparenleft}mono{\isacharunderscore}tags{\isacharcomma}\ lifting{\isacharparenright}{\isacharparenright}%
\endisatagproof
{\isafoldproof}%
%
\isadelimproof
\isanewline
%
\endisadelimproof
\isanewline
\isacommand{lemma}\isamarkupfalse%
\ rangeEmpty{\isacharcolon}\ \isanewline
\ \ {\isachardoublequoteopen}Range\ {\isacharminus}{\isacharbackquote}\ {\isacharbraceleft}{\isacharbraceleft}{\isacharbraceright}{\isacharbraceright}\ {\isacharequal}\ {\isacharbraceleft}{\isacharbraceleft}{\isacharbraceright}{\isacharbraceright}{\isachardoublequoteclose}\ \isanewline
%
\isadelimproof
\ \ %
\endisadelimproof
%
\isatagproof
\isacommand{by}\isamarkupfalse%
\ auto%
\endisatagproof
{\isafoldproof}%
%
\isadelimproof
\isanewline
%
\endisadelimproof
\isanewline
\isacommand{lemma}\isamarkupfalse%
\ finitePairSecondRange{\isacharcolon}\ \isanewline
\ \ {\isachardoublequoteopen}{\isacharparenleft}{\isasymforall}\ pair\ {\isasymin}\ R{\isachardot}\ finite\ {\isacharparenleft}snd\ pair{\isacharparenright}{\isacharparenright}\ {\isacharequal}\ {\isacharparenleft}{\isasymforall}\ y\ {\isasymin}\ Range\ R{\isachardot}\ finite\ y{\isacharparenright}{\isachardoublequoteclose}\ \isanewline
%
\isadelimproof
\ \ %
\endisadelimproof
%
\isatagproof
\isacommand{by}\isamarkupfalse%
\ fastforce%
\endisatagproof
{\isafoldproof}%
%
\isadelimproof
\isanewline
%
\endisadelimproof
\isanewline
\isacommand{lemma}\isamarkupfalse%
\ lm{\isadigit{0}}{\isadigit{5}}{\isadigit{6}}{\isacharcolon}\ \isanewline
\ \ {\isachardoublequoteopen}fst\ {\isacharbackquote}\ P\ {\isacharequal}\ snd\ {\isacharbackquote}\ {\isacharparenleft}P{\isacharcircum}{\isacharminus}{\isadigit{1}}{\isacharparenright}{\isachardoublequoteclose}\ \isanewline
%
\isadelimproof
\ \ %
\endisadelimproof
%
\isatagproof
\isacommand{by}\isamarkupfalse%
\ force%
\endisatagproof
{\isafoldproof}%
%
\isadelimproof
\isanewline
%
\endisadelimproof
\isanewline
\isacommand{lemma}\isamarkupfalse%
\ lm{\isadigit{0}}{\isadigit{5}}{\isadigit{7}}{\isacharcolon}\ \isanewline
\ \ {\isachardoublequoteopen}fst\ pair\ {\isacharequal}\ snd\ {\isacharparenleft}flip\ pair{\isacharparenright}\ {\isacharampersand}\ snd\ pair\ {\isacharequal}\ fst\ {\isacharparenleft}flip\ pair{\isacharparenright}{\isachardoublequoteclose}\ \isanewline
%
\isadelimproof
\ \ %
\endisadelimproof
%
\isatagproof
\isacommand{unfolding}\isamarkupfalse%
\ flip{\isacharunderscore}def\ \isacommand{by}\isamarkupfalse%
\ simp%
\endisatagproof
{\isafoldproof}%
%
\isadelimproof
\isanewline
%
\endisadelimproof
\isanewline
\isacommand{lemma}\isamarkupfalse%
\ flip{\isacharunderscore}flip{\isadigit{2}}{\isacharcolon}\ \isanewline
\ \ {\isachardoublequoteopen}flip\ {\isasymcirc}\ flip\ \ \ {\isacharequal}\ \ \ id{\isachardoublequoteclose}\ \isanewline
%
\isadelimproof
\ \ %
\endisadelimproof
%
\isatagproof
\isacommand{using}\isamarkupfalse%
\ flip{\isacharunderscore}flip\ \isacommand{by}\isamarkupfalse%
\ fastforce%
\endisatagproof
{\isafoldproof}%
%
\isadelimproof
\isanewline
%
\endisadelimproof
\isanewline
\isacommand{lemma}\isamarkupfalse%
\ lm{\isadigit{0}}{\isadigit{5}}{\isadigit{8}}{\isacharcolon}\ \isanewline
\ \ {\isachardoublequoteopen}fst\ {\isacharequal}\ {\isacharparenleft}snd{\isasymcirc}flip{\isacharparenright}{\isachardoublequoteclose}\ \isanewline
%
\isadelimproof
\ \ %
\endisadelimproof
%
\isatagproof
\isacommand{using}\isamarkupfalse%
\ lm{\isadigit{0}}{\isadigit{5}}{\isadigit{7}}\ \isacommand{by}\isamarkupfalse%
\ fastforce%
\endisatagproof
{\isafoldproof}%
%
\isadelimproof
\isanewline
%
\endisadelimproof
\isanewline
\isacommand{lemma}\isamarkupfalse%
\ lm{\isadigit{0}}{\isadigit{5}}{\isadigit{9}}{\isacharcolon}\ \isanewline
\ \ {\isachardoublequoteopen}snd\ {\isacharequal}\ {\isacharparenleft}fst{\isasymcirc}flip{\isacharparenright}{\isachardoublequoteclose}\ \isanewline
%
\isadelimproof
\ \ %
\endisadelimproof
%
\isatagproof
\isacommand{using}\isamarkupfalse%
\ lm{\isadigit{0}}{\isadigit{5}}{\isadigit{7}}\ \isacommand{by}\isamarkupfalse%
\ fastforce%
\endisatagproof
{\isafoldproof}%
%
\isadelimproof
\isanewline
%
\endisadelimproof
\isanewline
\isacommand{lemma}\isamarkupfalse%
\ lm{\isadigit{0}}{\isadigit{6}}{\isadigit{0}}{\isacharcolon}\ \isanewline
\ \ {\isachardoublequoteopen}inj{\isacharunderscore}on\ fst\ P\ {\isacharequal}\ inj{\isacharunderscore}on\ {\isacharparenleft}snd{\isasymcirc}flip{\isacharparenright}\ P{\isachardoublequoteclose}\ \isanewline
%
\isadelimproof
\ \ %
\endisadelimproof
%
\isatagproof
\isacommand{using}\isamarkupfalse%
\ lm{\isadigit{0}}{\isadigit{5}}{\isadigit{8}}\ \isacommand{by}\isamarkupfalse%
\ metis%
\endisatagproof
{\isafoldproof}%
%
\isadelimproof
\isanewline
%
\endisadelimproof
\isanewline
\isacommand{lemma}\isamarkupfalse%
\ lm{\isadigit{0}}{\isadigit{6}}{\isadigit{2}}{\isacharcolon}\ \isanewline
\ \ {\isachardoublequoteopen}inj{\isacharunderscore}on\ fst\ P\ {\isacharequal}\ inj{\isacharunderscore}on\ snd\ {\isacharparenleft}P{\isacharcircum}{\isacharminus}{\isadigit{1}}{\isacharparenright}{\isachardoublequoteclose}\ \isanewline
%
\isadelimproof
\ \ %
\endisadelimproof
%
\isatagproof
\isacommand{using}\isamarkupfalse%
\ lm{\isadigit{0}}{\isadigit{6}}{\isadigit{0}}\ flip{\isacharunderscore}conv\ \isacommand{by}\isamarkupfalse%
\ {\isacharparenleft}metis\ converse{\isacharunderscore}converse\ inj{\isacharunderscore}on{\isacharunderscore}imageI\ lm{\isadigit{0}}{\isadigit{5}}{\isadigit{9}}{\isacharparenright}%
\endisatagproof
{\isafoldproof}%
%
\isadelimproof
\isanewline
%
\endisadelimproof
\isanewline
\isacommand{lemma}\isamarkupfalse%
\ setsumPairsInverse{\isacharcolon}\ \isanewline
\ \ \isakeyword{assumes}\ {\isachardoublequoteopen}runiq\ {\isacharparenleft}P{\isacharcircum}{\isacharminus}{\isadigit{1}}{\isacharparenright}{\isachardoublequoteclose}\ \isanewline
\ \ \isakeyword{shows}\ {\isachardoublequoteopen}setsum\ {\isacharparenleft}f\ {\isasymcirc}\ snd{\isacharparenright}\ P\ {\isacharequal}\ setsum\ f\ {\isacharparenleft}Range\ P{\isacharparenright}{\isachardoublequoteclose}\ \isanewline
%
\isadelimproof
\ \ %
\endisadelimproof
%
\isatagproof
\isacommand{using}\isamarkupfalse%
\ assms\ lm{\isadigit{0}}{\isadigit{6}}{\isadigit{2}}\ converse{\isacharunderscore}converse\ rightUniqueInjectiveOnFirst\ rightUniqueInjectiveOnFirst\isanewline
\ \ \ \ \ \ \ \ setsum{\isachardot}reindex\ snd{\isacharunderscore}eq{\isacharunderscore}Range\ \isanewline
\ \ \isacommand{by}\isamarkupfalse%
\ metis%
\endisatagproof
{\isafoldproof}%
%
\isadelimproof
\isanewline
%
\endisadelimproof
\isanewline
\isacommand{lemma}\isamarkupfalse%
\ notEmptyFinestpart{\isacharcolon}\ \isanewline
\ \ \isakeyword{assumes}\ {\isachardoublequoteopen}X\ {\isasymnoteq}\ {\isacharbraceleft}{\isacharbraceright}{\isachardoublequoteclose}\ \isanewline
\ \ \isakeyword{shows}\ {\isachardoublequoteopen}finestpart\ X\ {\isasymnoteq}\ {\isacharbraceleft}{\isacharbraceright}{\isachardoublequoteclose}\ \isanewline
%
\isadelimproof
\ \ %
\endisadelimproof
%
\isatagproof
\isacommand{using}\isamarkupfalse%
\ assms\ finestpart{\isacharunderscore}def\ \isacommand{by}\isamarkupfalse%
\ blast%
\endisatagproof
{\isafoldproof}%
%
\isadelimproof
\isanewline
%
\endisadelimproof
\isanewline
\isacommand{lemma}\isamarkupfalse%
\ lm{\isadigit{0}}{\isadigit{6}}{\isadigit{3}}{\isacharcolon}\ \isanewline
\ \ \isakeyword{assumes}\ {\isachardoublequoteopen}inj{\isacharunderscore}on\ g\ X{\isachardoublequoteclose}\ \isanewline
\ \ \isakeyword{shows}\ {\isachardoublequoteopen}setsum\ f\ {\isacharparenleft}g{\isacharbackquote}X{\isacharparenright}\ {\isacharequal}\ setsum\ {\isacharparenleft}f\ {\isasymcirc}\ g{\isacharparenright}\ X{\isachardoublequoteclose}\ \isanewline
%
\isadelimproof
\ \ %
\endisadelimproof
%
\isatagproof
\isacommand{using}\isamarkupfalse%
\ assms\ \isacommand{by}\isamarkupfalse%
\ {\isacharparenleft}metis\ setsum{\isachardot}reindex{\isacharparenright}%
\endisatagproof
{\isafoldproof}%
%
\isadelimproof
\isanewline
%
\endisadelimproof
\isanewline
\isacommand{lemma}\isamarkupfalse%
\ functionOnFirstEqualsSecond{\isacharcolon}\ \isanewline
\ \ \isakeyword{assumes}\ {\isachardoublequoteopen}runiq\ R{\isachardoublequoteclose}\ {\isachardoublequoteopen}z\ {\isasymin}\ R{\isachardoublequoteclose}\ \isanewline
\ \ \isakeyword{shows}\ {\isachardoublequoteopen}R{\isacharcomma}{\isacharcomma}{\isacharparenleft}fst\ z{\isacharparenright}\ {\isacharequal}\ snd\ z{\isachardoublequoteclose}\ \isanewline
%
\isadelimproof
\ \ %
\endisadelimproof
%
\isatagproof
\isacommand{using}\isamarkupfalse%
\ assms\ \isacommand{by}\isamarkupfalse%
\ {\isacharparenleft}metis\ rightUniquePair\ surjective{\isacharunderscore}pairing{\isacharparenright}%
\endisatagproof
{\isafoldproof}%
%
\isadelimproof
\isanewline
%
\endisadelimproof
\isanewline
\isacommand{lemma}\isamarkupfalse%
\ lm{\isadigit{0}}{\isadigit{6}}{\isadigit{4}}{\isacharcolon}\ \isanewline
\ \ \isakeyword{assumes}\ {\isachardoublequoteopen}runiq\ R{\isachardoublequoteclose}\ \isanewline
\ \ \isakeyword{shows}\ {\isachardoublequoteopen}setsum\ {\isacharparenleft}toFunction\ R{\isacharparenright}\ {\isacharparenleft}Domain\ R{\isacharparenright}\ {\isacharequal}\ setsum\ snd\ R{\isachardoublequoteclose}\ \isanewline
%
\isadelimproof
\ \ %
\endisadelimproof
%
\isatagproof
\isacommand{using}\isamarkupfalse%
\ assms\ toFunction{\isacharunderscore}def\ setsum{\isachardot}reindex{\isacharunderscore}cong\ functionOnFirstEqualsSecond\isanewline
\ \ \ \ \ \ \ \ rightUniqueInjectiveOnFirst\ \isanewline
\ \ \isacommand{by}\isamarkupfalse%
\ {\isacharparenleft}metis\ {\isacharparenleft}no{\isacharunderscore}types{\isacharparenright}\ fst{\isacharunderscore}eq{\isacharunderscore}Domain{\isacharparenright}%
\endisatagproof
{\isafoldproof}%
%
\isadelimproof
\isanewline
%
\endisadelimproof
\isanewline
\isacommand{corollary}\isamarkupfalse%
\ lm{\isadigit{0}}{\isadigit{6}}{\isadigit{5}}{\isacharcolon}\ \isanewline
\ \ \isakeyword{assumes}\ {\isachardoublequoteopen}runiq\ {\isacharparenleft}f{\isacharbar}{\isacharbar}X{\isacharparenright}{\isachardoublequoteclose}\ \isanewline
\ \ \isakeyword{shows}\ {\isachardoublequoteopen}setsum\ {\isacharparenleft}toFunction\ {\isacharparenleft}f{\isacharbar}{\isacharbar}X{\isacharparenright}{\isacharparenright}\ {\isacharparenleft}X\ {\isasyminter}\ Domain\ f{\isacharparenright}\ {\isacharequal}\ setsum\ snd\ {\isacharparenleft}f{\isacharbar}{\isacharbar}X{\isacharparenright}{\isachardoublequoteclose}\ \isanewline
%
\isadelimproof
\ \ %
\endisadelimproof
%
\isatagproof
\isacommand{using}\isamarkupfalse%
\ assms\ lm{\isadigit{0}}{\isadigit{6}}{\isadigit{4}}\ \isacommand{by}\isamarkupfalse%
\ {\isacharparenleft}metis\ Int{\isacharunderscore}commute\ restrictedDomain{\isacharparenright}%
\endisatagproof
{\isafoldproof}%
%
\isadelimproof
\isanewline
%
\endisadelimproof
\isanewline
\isacommand{lemma}\isamarkupfalse%
\ lm{\isadigit{0}}{\isadigit{6}}{\isadigit{6}}{\isacharcolon}\ \isanewline
\ \ {\isachardoublequoteopen}Range\ {\isacharparenleft}R\ outside\ X{\isacharparenright}\ {\isacharequal}\ R{\isacharbackquote}{\isacharbackquote}{\isacharparenleft}{\isacharparenleft}Domain\ R{\isacharparenright}\ {\isacharminus}\ X{\isacharparenright}{\isachardoublequoteclose}\ \isanewline
%
\isadelimproof
\ \ %
\endisadelimproof
%
\isatagproof
\isacommand{using}\isamarkupfalse%
\ assms\ \isanewline
\ \ \isacommand{by}\isamarkupfalse%
\ {\isacharparenleft}metis\ Diff{\isacharunderscore}idemp\ ImageE\ Range{\isachardot}intros\ Range{\isacharunderscore}outside{\isacharunderscore}sub{\isacharunderscore}Image{\isacharunderscore}Domain\ lm{\isadigit{0}}{\isadigit{4}}{\isadigit{1}}\isanewline
\ \ \ \ \ \ \ \ \ \ \ \ lm{\isadigit{0}}{\isadigit{4}}{\isadigit{2}}\ order{\isacharunderscore}class{\isachardot}order{\isachardot}antisym\ subsetI{\isacharparenright}%
\endisatagproof
{\isafoldproof}%
%
\isadelimproof
\isanewline
%
\endisadelimproof
\isanewline
\isacommand{lemma}\isamarkupfalse%
\ lm{\isadigit{0}}{\isadigit{6}}{\isadigit{7}}{\isacharcolon}\ \isanewline
\ \ {\isachardoublequoteopen}{\isacharparenleft}R{\isacharbar}{\isacharbar}X{\isacharparenright}\ {\isacharbackquote}{\isacharbackquote}\ X\ {\isacharequal}\ R{\isacharbackquote}{\isacharbackquote}X{\isachardoublequoteclose}\ \isanewline
%
\isadelimproof
\ \ %
\endisadelimproof
%
\isatagproof
\isacommand{using}\isamarkupfalse%
\ Int{\isacharunderscore}absorb\ doubleRestriction\ restrictedRange\ \isacommand{by}\isamarkupfalse%
\ metis%
\endisatagproof
{\isafoldproof}%
%
\isadelimproof
\isanewline
%
\endisadelimproof
\isanewline
\isacommand{lemma}\isamarkupfalse%
\ lm{\isadigit{0}}{\isadigit{6}}{\isadigit{8}}{\isacharcolon}\ \isanewline
\ \ \isakeyword{assumes}\ {\isachardoublequoteopen}x\ {\isasymin}\ Domain\ {\isacharparenleft}f{\isacharbar}{\isacharbar}X{\isacharparenright}{\isachardoublequoteclose}\ \isanewline
\ \ \isakeyword{shows}\ {\isachardoublequoteopen}{\isacharparenleft}f{\isacharbar}{\isacharbar}X{\isacharparenright}{\isacharbackquote}{\isacharbackquote}{\isacharbraceleft}x{\isacharbraceright}\ {\isacharequal}\ f{\isacharbackquote}{\isacharbackquote}{\isacharbraceleft}x{\isacharbraceright}{\isachardoublequoteclose}\ \isanewline
%
\isadelimproof
\ \ %
\endisadelimproof
%
\isatagproof
\isacommand{using}\isamarkupfalse%
\ assms\ doubleRestriction\ restrictedRange\ Int{\isacharunderscore}empty{\isacharunderscore}right\ Int{\isacharunderscore}iff\ \isanewline
\ \ \ \ \ \ \ \ Int{\isacharunderscore}insert{\isacharunderscore}right{\isacharunderscore}if{\isadigit{1}}\ restrictedDomain\ \isanewline
\ \ \isacommand{by}\isamarkupfalse%
\ metis%
\endisatagproof
{\isafoldproof}%
%
\isadelimproof
\isanewline
%
\endisadelimproof
\isanewline
\isacommand{lemma}\isamarkupfalse%
\ lm{\isadigit{0}}{\isadigit{6}}{\isadigit{9}}{\isacharcolon}\ \isanewline
\ \ \isakeyword{assumes}\ {\isachardoublequoteopen}x\ {\isasymin}\ X\ {\isasyminter}\ Domain\ f{\isachardoublequoteclose}\ {\isachardoublequoteopen}runiq\ {\isacharparenleft}f{\isacharbar}{\isacharbar}X{\isacharparenright}{\isachardoublequoteclose}\ \isanewline
\ \ \isakeyword{shows}\ {\isachardoublequoteopen}{\isacharparenleft}f{\isacharbar}{\isacharbar}X{\isacharparenright}{\isacharcomma}{\isacharcomma}x\ {\isacharequal}\ f{\isacharcomma}{\isacharcomma}x{\isachardoublequoteclose}\ \isanewline
%
\isadelimproof
\ \ %
\endisadelimproof
%
\isatagproof
\isacommand{using}\isamarkupfalse%
\ assms\ doubleRestriction\ restrictedRange\ Int{\isacharunderscore}empty{\isacharunderscore}right\ Int{\isacharunderscore}iff\ Int{\isacharunderscore}insert{\isacharunderscore}right{\isacharunderscore}if{\isadigit{1}}\isanewline
\ \ \ \ \ \ \ \ eval{\isacharunderscore}rel{\isachardot}simps\ \isanewline
\ \ \isacommand{by}\isamarkupfalse%
\ metis%
\endisatagproof
{\isafoldproof}%
%
\isadelimproof
\isanewline
%
\endisadelimproof
\isanewline
\isacommand{lemma}\isamarkupfalse%
\ lm{\isadigit{0}}{\isadigit{7}}{\isadigit{0}}{\isacharcolon}\ \isanewline
\ \ \isakeyword{assumes}\ {\isachardoublequoteopen}runiq\ {\isacharparenleft}f{\isacharbar}{\isacharbar}X{\isacharparenright}{\isachardoublequoteclose}\ \isanewline
\ \ \isakeyword{shows}\ {\isachardoublequoteopen}setsum\ {\isacharparenleft}toFunction\ {\isacharparenleft}f{\isacharbar}{\isacharbar}X{\isacharparenright}{\isacharparenright}\ {\isacharparenleft}X\ {\isasyminter}\ Domain\ f{\isacharparenright}\ {\isacharequal}\ setsum\ {\isacharparenleft}toFunction\ f{\isacharparenright}\ {\isacharparenleft}X\ {\isasyminter}\ Domain\ f{\isacharparenright}{\isachardoublequoteclose}\ \isanewline
%
\isadelimproof
\ \ %
\endisadelimproof
%
\isatagproof
\isacommand{using}\isamarkupfalse%
\ assms\ setsum{\isachardot}cong\ lm{\isadigit{0}}{\isadigit{6}}{\isadigit{9}}\ toFunction{\isacharunderscore}def\ \isacommand{by}\isamarkupfalse%
\ metis%
\endisatagproof
{\isafoldproof}%
%
\isadelimproof
\isanewline
%
\endisadelimproof
\isanewline
\isacommand{corollary}\isamarkupfalse%
\ setsumRestrictedToDomainInvariant{\isacharcolon}\ \isanewline
\ \ \isakeyword{assumes}\ {\isachardoublequoteopen}runiq\ {\isacharparenleft}f{\isacharbar}{\isacharbar}X{\isacharparenright}{\isachardoublequoteclose}\ \isanewline
\ \ \isakeyword{shows}\ {\isachardoublequoteopen}setsum\ {\isacharparenleft}toFunction\ f{\isacharparenright}\ {\isacharparenleft}X\ {\isasyminter}\ Domain\ f{\isacharparenright}\ {\isacharequal}\ setsum\ snd\ {\isacharparenleft}f{\isacharbar}{\isacharbar}X{\isacharparenright}{\isachardoublequoteclose}\ \isanewline
%
\isadelimproof
\ \ %
\endisadelimproof
%
\isatagproof
\isacommand{using}\isamarkupfalse%
\ assms\ lm{\isadigit{0}}{\isadigit{6}}{\isadigit{5}}\ lm{\isadigit{0}}{\isadigit{7}}{\isadigit{0}}\ \isacommand{by}\isamarkupfalse%
\ fastforce%
\endisatagproof
{\isafoldproof}%
%
\isadelimproof
\isanewline
%
\endisadelimproof
\isanewline
\isacommand{corollary}\isamarkupfalse%
\ setsumRestrictedOnFunction{\isacharcolon}\ \isanewline
\ \ \isakeyword{assumes}\ {\isachardoublequoteopen}runiq\ {\isacharparenleft}f{\isacharbar}{\isacharbar}X{\isacharparenright}{\isachardoublequoteclose}\ \isanewline
\ \ \isakeyword{shows}\ {\isachardoublequoteopen}setsum\ {\isacharparenleft}toFunction\ {\isacharparenleft}f{\isacharbar}{\isacharbar}X{\isacharparenright}{\isacharparenright}\ {\isacharparenleft}X\ {\isasyminter}\ Domain\ f{\isacharparenright}\ {\isacharequal}\ setsum\ snd\ {\isacharparenleft}f{\isacharbar}{\isacharbar}X{\isacharparenright}{\isachardoublequoteclose}\ \isanewline
%
\isadelimproof
\ \ %
\endisadelimproof
%
\isatagproof
\isacommand{using}\isamarkupfalse%
\ assms\ lm{\isadigit{0}}{\isadigit{6}}{\isadigit{4}}\ restrictedDomain\ Int{\isacharunderscore}commute\ \isacommand{by}\isamarkupfalse%
\ metis%
\endisatagproof
{\isafoldproof}%
%
\isadelimproof
\isanewline
%
\endisadelimproof
\isanewline
\isacommand{lemma}\isamarkupfalse%
\ cardFinestpart{\isacharcolon}\ \isanewline
\ \ {\isachardoublequoteopen}card\ {\isacharparenleft}finestpart\ X{\isacharparenright}\ {\isacharequal}\ card\ X{\isachardoublequoteclose}\ \isanewline
%
\isadelimproof
\ \ %
\endisadelimproof
%
\isatagproof
\isacommand{using}\isamarkupfalse%
\ finestpart{\isacharunderscore}def\ \isacommand{by}\isamarkupfalse%
\ {\isacharparenleft}metis\ {\isacharparenleft}lifting{\isacharparenright}\ card{\isacharunderscore}image\ inj{\isacharunderscore}on{\isacharunderscore}inverseI\ the{\isacharunderscore}elem{\isacharunderscore}eq{\isacharparenright}%
\endisatagproof
{\isafoldproof}%
%
\isadelimproof
\isanewline
%
\endisadelimproof
\isanewline
\isacommand{corollary}\isamarkupfalse%
\ lm{\isadigit{0}}{\isadigit{7}}{\isadigit{1}}{\isacharcolon}\ \isanewline
\ \ {\isachardoublequoteopen}finestpart\ {\isacharbraceleft}{\isacharbraceright}\ {\isacharequal}\ {\isacharbraceleft}{\isacharbraceright}\ \ \ \ {\isacharampersand}\ \ \ \ card\ {\isasymcirc}\ finestpart\ {\isacharequal}\ card{\isachardoublequoteclose}\ \isanewline
%
\isadelimproof
\ \ %
\endisadelimproof
%
\isatagproof
\isacommand{using}\isamarkupfalse%
\ cardFinestpart\ finestpart{\isacharunderscore}def\ \isacommand{by}\isamarkupfalse%
\ fastforce%
\endisatagproof
{\isafoldproof}%
%
\isadelimproof
\isanewline
%
\endisadelimproof
\isanewline
\isacommand{lemma}\isamarkupfalse%
\ finiteFinestpart{\isacharcolon}\ \isanewline
\ \ {\isachardoublequoteopen}finite\ {\isacharparenleft}finestpart\ X{\isacharparenright}\ {\isacharequal}\ finite\ X{\isachardoublequoteclose}\ \isanewline
%
\isadelimproof
\ \ %
\endisadelimproof
%
\isatagproof
\isacommand{using}\isamarkupfalse%
\ finestpart{\isacharunderscore}def\ lm{\isadigit{0}}{\isadigit{7}}{\isadigit{1}}\ \isanewline
\ \ \isacommand{by}\isamarkupfalse%
\ {\isacharparenleft}metis\ card{\isacharunderscore}eq{\isacharunderscore}{\isadigit{0}}{\isacharunderscore}iff\ empty{\isacharunderscore}is{\isacharunderscore}image\ finite{\isachardot}simps\ cardFinestpart{\isacharparenright}%
\endisatagproof
{\isafoldproof}%
%
\isadelimproof
\isanewline
%
\endisadelimproof
\isanewline
\isacommand{lemma}\isamarkupfalse%
\ lm{\isadigit{0}}{\isadigit{7}}{\isadigit{2}}{\isacharcolon}\ \isanewline
\ \ {\isachardoublequoteopen}finite\ {\isasymcirc}\ finestpart\ {\isacharequal}\ finite{\isachardoublequoteclose}\ \isanewline
%
\isadelimproof
\ \ %
\endisadelimproof
%
\isatagproof
\isacommand{using}\isamarkupfalse%
\ finiteFinestpart\ \isacommand{by}\isamarkupfalse%
\ fastforce%
\endisatagproof
{\isafoldproof}%
%
\isadelimproof
\isanewline
%
\endisadelimproof
\isanewline
\isacommand{lemma}\isamarkupfalse%
\ finestpartSubset{\isacharcolon}\ \isanewline
\ \ \isakeyword{assumes}\ {\isachardoublequoteopen}X\ {\isasymsubseteq}\ Y{\isachardoublequoteclose}\ \isanewline
\ \ \isakeyword{shows}\ {\isachardoublequoteopen}finestpart\ X\ {\isasymsubseteq}\ finestpart\ Y{\isachardoublequoteclose}\ \isanewline
%
\isadelimproof
\ \ %
\endisadelimproof
%
\isatagproof
\isacommand{using}\isamarkupfalse%
\ assms\ finestpart{\isacharunderscore}def\ \isacommand{by}\isamarkupfalse%
\ {\isacharparenleft}metis\ image{\isacharunderscore}mono{\isacharparenright}%
\endisatagproof
{\isafoldproof}%
%
\isadelimproof
\isanewline
%
\endisadelimproof
\isanewline
\isacommand{corollary}\isamarkupfalse%
\ lm{\isadigit{0}}{\isadigit{7}}{\isadigit{3}}{\isacharcolon}\ \isanewline
\ \ \isakeyword{assumes}\ {\isachardoublequoteopen}x\ {\isasymin}\ X{\isachardoublequoteclose}\ \isanewline
\ \ \isakeyword{shows}\ {\isachardoublequoteopen}finestpart\ x\ {\isasymsubseteq}\ finestpart\ {\isacharparenleft}{\isasymUnion}\ X{\isacharparenright}{\isachardoublequoteclose}\ \isanewline
%
\isadelimproof
\ \ %
\endisadelimproof
%
\isatagproof
\isacommand{using}\isamarkupfalse%
\ assms\ finestpartSubset\ \isacommand{by}\isamarkupfalse%
\ {\isacharparenleft}metis\ Union{\isacharunderscore}upper{\isacharparenright}%
\endisatagproof
{\isafoldproof}%
%
\isadelimproof
\isanewline
%
\endisadelimproof
\isanewline
\isacommand{lemma}\isamarkupfalse%
\ lm{\isadigit{0}}{\isadigit{7}}{\isadigit{4}}{\isacharcolon}\ \isanewline
\ \ {\isachardoublequoteopen}{\isasymUnion}\ {\isacharparenleft}finestpart\ {\isacharbackquote}\ XX{\isacharparenright}\ {\isasymsubseteq}\ finestpart\ {\isacharparenleft}{\isasymUnion}\ XX{\isacharparenright}{\isachardoublequoteclose}\ \isanewline
%
\isadelimproof
\ \ %
\endisadelimproof
%
\isatagproof
\isacommand{using}\isamarkupfalse%
\ finestpart{\isacharunderscore}def\ lm{\isadigit{0}}{\isadigit{7}}{\isadigit{3}}\ \isacommand{by}\isamarkupfalse%
\ force%
\endisatagproof
{\isafoldproof}%
%
\isadelimproof
\isanewline
%
\endisadelimproof
\isanewline
\isacommand{lemma}\isamarkupfalse%
\ lm{\isadigit{0}}{\isadigit{7}}{\isadigit{5}}{\isacharcolon}\ \isanewline
\ \ {\isachardoublequoteopen}{\isasymUnion}\ {\isacharparenleft}finestpart\ {\isacharbackquote}\ XX{\isacharparenright}\ {\isasymsupseteq}\ finestpart\ {\isacharparenleft}{\isasymUnion}\ XX{\isacharparenright}{\isachardoublequoteclose}\ \isanewline
\ \ {\isacharparenleft}\isakeyword{is}\ {\isachardoublequoteopen}{\isacharquery}L\ {\isasymsupseteq}\ {\isacharquery}R{\isachardoublequoteclose}{\isacharparenright}\ \isanewline
%
\isadelimproof
\ \ %
\endisadelimproof
%
\isatagproof
\isacommand{unfolding}\isamarkupfalse%
\ finestpart{\isacharunderscore}def\ \isacommand{using}\isamarkupfalse%
\ finestpart{\isacharunderscore}def\ \isacommand{by}\isamarkupfalse%
\ auto%
\endisatagproof
{\isafoldproof}%
%
\isadelimproof
\isanewline
%
\endisadelimproof
\isanewline
\isacommand{corollary}\isamarkupfalse%
\ commuteUnionFinestpart{\isacharcolon}\ \isanewline
\ \ {\isachardoublequoteopen}{\isasymUnion}\ {\isacharparenleft}finestpart\ {\isacharbackquote}\ XX{\isacharparenright}\ {\isacharequal}\ finestpart\ {\isacharparenleft}{\isasymUnion}\ XX{\isacharparenright}{\isachardoublequoteclose}\isanewline
%
\isadelimproof
\ \ %
\endisadelimproof
%
\isatagproof
\isacommand{using}\isamarkupfalse%
\ lm{\isadigit{0}}{\isadigit{7}}{\isadigit{4}}\ lm{\isadigit{0}}{\isadigit{7}}{\isadigit{5}}\ \isacommand{by}\isamarkupfalse%
\ fast%
\endisatagproof
{\isafoldproof}%
%
\isadelimproof
\isanewline
%
\endisadelimproof
\isanewline
\isacommand{lemma}\isamarkupfalse%
\ unionImage{\isacharcolon}\ \isanewline
\ \ \isakeyword{assumes}\ {\isachardoublequoteopen}runiq\ a{\isachardoublequoteclose}\ \isanewline
\ \ \isakeyword{shows}\ {\isachardoublequoteopen}{\isacharbraceleft}{\isacharparenleft}x{\isacharcomma}\ {\isacharbraceleft}y{\isacharbraceright}{\isacharparenright}{\isacharbar}\ x\ y{\isachardot}\ y\ {\isasymin}\ {\isasymUnion}\ {\isacharparenleft}a{\isacharbackquote}{\isacharbackquote}{\isacharbraceleft}x{\isacharbraceright}{\isacharparenright}\ {\isacharampersand}\ x\ {\isasymin}\ Domain\ a{\isacharbraceright}\ {\isacharequal}\ \isanewline
\ \ \ \ \ \ \ \ \ {\isacharbraceleft}{\isacharparenleft}x{\isacharcomma}\ {\isacharbraceleft}y{\isacharbraceright}{\isacharparenright}{\isacharbar}\ x\ y{\isachardot}\ y\ {\isasymin}\ a{\isacharcomma}{\isacharcomma}x\ {\isacharampersand}\ x\ {\isasymin}\ Domain\ a{\isacharbraceright}{\isachardoublequoteclose}\ \isanewline
%
\isadelimproof
\ \ %
\endisadelimproof
%
\isatagproof
\isacommand{using}\isamarkupfalse%
\ assms\ Image{\isacharunderscore}runiq{\isacharunderscore}eq{\isacharunderscore}eval\ \isanewline
\ \ \isacommand{by}\isamarkupfalse%
\ {\isacharparenleft}metis\ {\isacharparenleft}lifting{\isacharcomma}\ no{\isacharunderscore}types{\isacharparenright}\ cSup{\isacharunderscore}singleton{\isacharparenright}%
\endisatagproof
{\isafoldproof}%
%
\isadelimproof
\isanewline
%
\endisadelimproof
\isanewline
\isacommand{lemma}\isamarkupfalse%
\ lm{\isadigit{0}}{\isadigit{7}}{\isadigit{6}}{\isacharcolon}\ \isanewline
\ \ \isakeyword{assumes}\ {\isachardoublequoteopen}runiq\ P{\isachardoublequoteclose}\ \isanewline
\ \ \isakeyword{shows}\ {\isachardoublequoteopen}card\ {\isacharparenleft}Domain\ P{\isacharparenright}\ {\isacharequal}\ card\ P{\isachardoublequoteclose}\ \isanewline
%
\isadelimproof
\ \ %
\endisadelimproof
%
\isatagproof
\isacommand{using}\isamarkupfalse%
\ assms\ rightUniqueInjectiveOnFirst\ card{\isacharunderscore}image\ \isacommand{by}\isamarkupfalse%
\ {\isacharparenleft}metis\ Domain{\isacharunderscore}fst{\isacharparenright}%
\endisatagproof
{\isafoldproof}%
%
\isadelimproof
\isanewline
%
\endisadelimproof
\isanewline
\isacommand{lemma}\isamarkupfalse%
\ finiteDomainImpliesFinite{\isacharcolon}\ \isanewline
\ \ \isakeyword{assumes}\ {\isachardoublequoteopen}runiq\ f{\isachardoublequoteclose}\ \isanewline
\ \ \isakeyword{shows}\ {\isachardoublequoteopen}finite\ {\isacharparenleft}Domain\ f{\isacharparenright}\ {\isacharequal}\ finite\ f{\isachardoublequoteclose}\ \isanewline
%
\isadelimproof
\ \ %
\endisadelimproof
%
\isatagproof
\isacommand{using}\isamarkupfalse%
\ assms\ Domain{\isacharunderscore}empty{\isacharunderscore}iff\ card{\isacharunderscore}eq{\isacharunderscore}{\isadigit{0}}{\isacharunderscore}iff\ finite{\isachardot}emptyI\ lm{\isadigit{0}}{\isadigit{7}}{\isadigit{6}}\ \isacommand{by}\isamarkupfalse%
\ metis%
\endisatagproof
{\isafoldproof}%
%
\isadelimproof
\isanewline
%
\endisadelimproof
\isanewline
\isanewline
\isacommand{lemma}\isamarkupfalse%
\ sumCurry{\isacharcolon}\ \isanewline
\ \ {\isachardoublequoteopen}setsum\ {\isacharparenleft}{\isacharparenleft}curry\ f{\isacharparenright}\ x{\isacharparenright}\ Y\ {\isacharequal}\ setsum\ f\ {\isacharparenleft}{\isacharbraceleft}x{\isacharbraceright}\ {\isasymtimes}\ Y{\isacharparenright}{\isachardoublequoteclose}\isanewline
%
\isadelimproof
%
\endisadelimproof
%
\isatagproof
\isacommand{proof}\isamarkupfalse%
\ {\isacharminus}\isanewline
\ \ \isacommand{let}\isamarkupfalse%
\ {\isacharquery}f{\isacharequal}{\isachardoublequoteopen}{\isacharpercent}\ y{\isachardot}\ {\isacharparenleft}x{\isacharcomma}\ y{\isacharparenright}{\isachardoublequoteclose}\ \isacommand{let}\isamarkupfalse%
\ {\isacharquery}g{\isacharequal}{\isachardoublequoteopen}{\isacharparenleft}curry\ f{\isacharparenright}\ x{\isachardoublequoteclose}\ \isacommand{let}\isamarkupfalse%
\ {\isacharquery}h{\isacharequal}f\isanewline
\ \ \isacommand{have}\isamarkupfalse%
\ {\isachardoublequoteopen}inj{\isacharunderscore}on\ {\isacharquery}f\ Y{\isachardoublequoteclose}\ \isacommand{by}\isamarkupfalse%
\ {\isacharparenleft}metis{\isacharparenleft}no{\isacharunderscore}types{\isacharparenright}\ Pair{\isacharunderscore}inject\ inj{\isacharunderscore}onI{\isacharparenright}\ \isanewline
\ \ \isacommand{moreover}\isamarkupfalse%
\ \isacommand{have}\isamarkupfalse%
\ {\isachardoublequoteopen}{\isacharbraceleft}x{\isacharbraceright}\ {\isasymtimes}\ Y\ {\isacharequal}\ {\isacharquery}f\ {\isacharbackquote}\ Y{\isachardoublequoteclose}\ \isacommand{by}\isamarkupfalse%
\ fast\isanewline
\ \ \isacommand{moreover}\isamarkupfalse%
\ \isacommand{have}\isamarkupfalse%
\ {\isachardoublequoteopen}{\isasymforall}\ y{\isachardot}\ y\ {\isasymin}\ Y\ {\isasymlongrightarrow}\ {\isacharquery}g\ y\ {\isacharequal}\ {\isacharquery}h\ {\isacharparenleft}{\isacharquery}f\ y{\isacharparenright}{\isachardoublequoteclose}\ \isacommand{by}\isamarkupfalse%
\ simp\isanewline
\ \ \isacommand{ultimately}\isamarkupfalse%
\ \isacommand{show}\isamarkupfalse%
\ {\isacharquery}thesis\ \isacommand{using}\isamarkupfalse%
\ setsum{\isachardot}reindex{\isacharunderscore}cong\ \isacommand{by}\isamarkupfalse%
\ metis\isanewline
\isacommand{qed}\isamarkupfalse%
%
\endisatagproof
{\isafoldproof}%
%
\isadelimproof
\isanewline
%
\endisadelimproof
\isanewline
\isacommand{lemma}\isamarkupfalse%
\ lm{\isadigit{0}}{\isadigit{7}}{\isadigit{7}}{\isacharcolon}\ \isanewline
\ \ {\isachardoublequoteopen}setsum\ {\isacharparenleft}{\isacharpercent}y{\isachardot}\ f\ {\isacharparenleft}x{\isacharcomma}y{\isacharparenright}{\isacharparenright}\ Y\ {\isacharequal}\ setsum\ f\ {\isacharparenleft}{\isacharbraceleft}x{\isacharbraceright}{\isasymtimes}Y{\isacharparenright}{\isachardoublequoteclose}\ \isanewline
%
\isadelimproof
\ \ %
\endisadelimproof
%
\isatagproof
\isacommand{using}\isamarkupfalse%
\ sumCurry\ Sigma{\isacharunderscore}cong\ curry{\isacharunderscore}def\ setsum{\isachardot}cong\ \isacommand{by}\isamarkupfalse%
\ fastforce%
\endisatagproof
{\isafoldproof}%
%
\isadelimproof
\isanewline
%
\endisadelimproof
\isanewline
\isacommand{corollary}\isamarkupfalse%
\ lm{\isadigit{0}}{\isadigit{7}}{\isadigit{8}}{\isacharcolon}\ \isanewline
\ \ \isakeyword{assumes}\ {\isachardoublequoteopen}finite\ X{\isachardoublequoteclose}\ \isanewline
\ \ \isakeyword{shows}\ {\isachardoublequoteopen}setsum\ f\ X\ {\isacharequal}\ setsum\ f\ {\isacharparenleft}X{\isacharminus}Y{\isacharparenright}\ {\isacharplus}\ {\isacharparenleft}setsum\ f\ {\isacharparenleft}X\ {\isasyminter}\ Y{\isacharparenright}{\isacharparenright}{\isachardoublequoteclose}\ \isanewline
%
\isadelimproof
\ \ %
\endisadelimproof
%
\isatagproof
\isacommand{using}\isamarkupfalse%
\ assms\ Diff{\isacharunderscore}iff\ IntD{\isadigit{2}}\ Un{\isacharunderscore}Diff{\isacharunderscore}Int\ finite{\isacharunderscore}Un\ inf{\isacharunderscore}commute\ setsum{\isachardot}union{\isacharunderscore}inter{\isacharunderscore}neutral\ \isanewline
\ \ \isacommand{by}\isamarkupfalse%
\ metis%
\endisatagproof
{\isafoldproof}%
%
\isadelimproof
\isanewline
%
\endisadelimproof
\isanewline
\isacommand{lemma}\isamarkupfalse%
\ lm{\isadigit{0}}{\isadigit{7}}{\isadigit{9}}{\isacharcolon}\ \isanewline
\ \ {\isachardoublequoteopen}{\isacharparenleft}P\ {\isacharplus}{\isacharasterisk}\ Q{\isacharparenright}{\isacharbackquote}{\isacharbackquote}{\isacharparenleft}Domain\ Q{\isasyminter}X{\isacharparenright}\ \ {\isacharequal}\ \ Q{\isacharbackquote}{\isacharbackquote}{\isacharparenleft}Domain\ Q{\isasyminter}X{\isacharparenright}{\isachardoublequoteclose}\ \isanewline
%
\isadelimproof
\ \ %
\endisadelimproof
%
\isatagproof
\isacommand{unfolding}\isamarkupfalse%
\ paste{\isacharunderscore}def\ Outside{\isacharunderscore}def\ Image{\isacharunderscore}def\ Domain{\isacharunderscore}def\ \isacommand{by}\isamarkupfalse%
\ blast%
\endisatagproof
{\isafoldproof}%
%
\isadelimproof
\isanewline
%
\endisadelimproof
\isanewline
\isacommand{corollary}\isamarkupfalse%
\ lm{\isadigit{0}}{\isadigit{8}}{\isadigit{0}}{\isacharcolon}\ \isanewline
\ \ {\isachardoublequoteopen}{\isacharparenleft}P\ {\isacharplus}{\isacharasterisk}\ Q{\isacharparenright}{\isacharbackquote}{\isacharbackquote}{\isacharparenleft}X{\isasyminter}{\isacharparenleft}Domain\ Q{\isacharparenright}{\isacharparenright}\ \ {\isacharequal}\ \ Q{\isacharbackquote}{\isacharbackquote}X{\isachardoublequoteclose}\isanewline
%
\isadelimproof
\ \ %
\endisadelimproof
%
\isatagproof
\isacommand{using}\isamarkupfalse%
\ Int{\isacharunderscore}commute\ lm{\isadigit{0}}{\isadigit{7}}{\isadigit{9}}\ \isacommand{by}\isamarkupfalse%
\ {\isacharparenleft}metis\ lm{\isadigit{0}}{\isadigit{1}}{\isadigit{7}}{\isacharparenright}%
\endisatagproof
{\isafoldproof}%
%
\isadelimproof
\isanewline
%
\endisadelimproof
\isanewline
\isacommand{corollary}\isamarkupfalse%
\ lm{\isadigit{0}}{\isadigit{8}}{\isadigit{1}}{\isacharcolon}\ \isanewline
\ \ \isakeyword{assumes}\ {\isachardoublequoteopen}X\ {\isasyminter}\ {\isacharparenleft}Domain\ Q{\isacharparenright}\ {\isacharequal}\ {\isacharbraceleft}{\isacharbraceright}{\isachardoublequoteclose}\isanewline
\ \ \isakeyword{shows}\ {\isachardoublequoteopen}{\isacharparenleft}P\ {\isacharplus}{\isacharasterisk}\ Q{\isacharparenright}\ {\isacharbackquote}{\isacharbackquote}\ X\ {\isacharequal}\ {\isacharparenleft}P\ outside\ {\isacharparenleft}Domain\ Q{\isacharparenright}{\isacharparenright}{\isacharbackquote}{\isacharbackquote}\ X{\isachardoublequoteclose}\ \isanewline
%
\isadelimproof
\ \ %
\endisadelimproof
%
\isatagproof
\isacommand{using}\isamarkupfalse%
\ assms\ paste{\isacharunderscore}def\ \isacommand{by}\isamarkupfalse%
\ fast%
\endisatagproof
{\isafoldproof}%
%
\isadelimproof
\isanewline
%
\endisadelimproof
\isanewline
\isacommand{lemma}\isamarkupfalse%
\ lm{\isadigit{0}}{\isadigit{8}}{\isadigit{2}}{\isacharcolon}\ \isanewline
\ \ \isakeyword{assumes}\ {\isachardoublequoteopen}X{\isasyminter}Y\ {\isacharequal}\ {\isacharbraceleft}{\isacharbraceright}{\isachardoublequoteclose}\ \isanewline
\ \ \isakeyword{shows}\ {\isachardoublequoteopen}{\isacharparenleft}P\ outside\ Y{\isacharparenright}{\isacharbackquote}{\isacharbackquote}X{\isacharequal}P{\isacharbackquote}{\isacharbackquote}X{\isachardoublequoteclose}\ \isanewline
%
\isadelimproof
\ \ %
\endisadelimproof
%
\isatagproof
\isacommand{using}\isamarkupfalse%
\ assms\ Outside{\isacharunderscore}def\ \isacommand{by}\isamarkupfalse%
\ blast%
\endisatagproof
{\isafoldproof}%
%
\isadelimproof
\isanewline
%
\endisadelimproof
\isanewline
\isacommand{corollary}\isamarkupfalse%
\ lm{\isadigit{0}}{\isadigit{8}}{\isadigit{3}}{\isacharcolon}\ \isanewline
\ \ \isakeyword{assumes}\ {\isachardoublequoteopen}X{\isasyminter}\ {\isacharparenleft}Domain\ Q{\isacharparenright}\ {\isacharequal}\ {\isacharbraceleft}{\isacharbraceright}{\isachardoublequoteclose}\ \isanewline
\ \ \isakeyword{shows}\ {\isachardoublequoteopen}{\isacharparenleft}P\ {\isacharplus}{\isacharasterisk}\ Q{\isacharparenright}{\isacharbackquote}{\isacharbackquote}X{\isacharequal}P{\isacharbackquote}{\isacharbackquote}X{\isachardoublequoteclose}\ \isanewline
%
\isadelimproof
\ \ %
\endisadelimproof
%
\isatagproof
\isacommand{using}\isamarkupfalse%
\ assms\ lm{\isadigit{0}}{\isadigit{8}}{\isadigit{1}}\ lm{\isadigit{0}}{\isadigit{8}}{\isadigit{2}}\ \isacommand{by}\isamarkupfalse%
\ metis%
\endisatagproof
{\isafoldproof}%
%
\isadelimproof
\isanewline
%
\endisadelimproof
\isanewline
\isacommand{lemma}\isamarkupfalse%
\ lm{\isadigit{0}}{\isadigit{8}}{\isadigit{4}}{\isacharcolon}\ \isanewline
\ \ \isakeyword{assumes}\ {\isachardoublequoteopen}finite\ X{\isachardoublequoteclose}\ {\isachardoublequoteopen}finite\ Y{\isachardoublequoteclose}\ {\isachardoublequoteopen}card{\isacharparenleft}X{\isasyminter}Y{\isacharparenright}\ {\isacharequal}\ card\ X{\isachardoublequoteclose}\ \isanewline
\ \ \isakeyword{shows}\ {\isachardoublequoteopen}X\ {\isasymsubseteq}\ Y{\isachardoublequoteclose}\ \isanewline
%
\isadelimproof
\ \ %
\endisadelimproof
%
\isatagproof
\isacommand{using}\isamarkupfalse%
\ assms\ \isacommand{by}\isamarkupfalse%
\ {\isacharparenleft}metis\ Int{\isacharunderscore}lower{\isadigit{1}}\ Int{\isacharunderscore}lower{\isadigit{2}}\ card{\isacharunderscore}seteq\ order{\isacharunderscore}refl{\isacharparenright}%
\endisatagproof
{\isafoldproof}%
%
\isadelimproof
\isanewline
%
\endisadelimproof
\isanewline
\isacommand{lemma}\isamarkupfalse%
\ cardinalityIntersectionEquality{\isacharcolon}\ \isanewline
\ \ \isakeyword{assumes}\ {\isachardoublequoteopen}finite\ X{\isachardoublequoteclose}\ {\isachardoublequoteopen}finite\ Y{\isachardoublequoteclose}\ {\isachardoublequoteopen}card\ X\ {\isacharequal}\ card\ Y{\isachardoublequoteclose}\ \isanewline
\ \ \isakeyword{shows}\ {\isachardoublequoteopen}{\isacharparenleft}card\ {\isacharparenleft}X{\isasyminter}Y{\isacharparenright}\ {\isacharequal}\ card\ X{\isacharparenright}\ \ \ \ \ {\isacharequal}\ \ \ \ {\isacharparenleft}X\ {\isacharequal}\ Y{\isacharparenright}{\isachardoublequoteclose}\isanewline
%
\isadelimproof
\ \ %
\endisadelimproof
%
\isatagproof
\isacommand{using}\isamarkupfalse%
\ assms\ lm{\isadigit{0}}{\isadigit{8}}{\isadigit{4}}\ \isacommand{by}\isamarkupfalse%
\ {\isacharparenleft}metis\ card{\isacharunderscore}seteq\ le{\isacharunderscore}iff{\isacharunderscore}inf\ order{\isacharunderscore}refl{\isacharparenright}%
\endisatagproof
{\isafoldproof}%
%
\isadelimproof
\isanewline
%
\endisadelimproof
\isanewline
\isacommand{lemma}\isamarkupfalse%
\ lm{\isadigit{0}}{\isadigit{8}}{\isadigit{5}}{\isacharcolon}\ \ \isanewline
\ \ \isakeyword{assumes}\ {\isachardoublequoteopen}P\ xx{\isachardoublequoteclose}\ \isanewline
\ \ \isakeyword{shows}\ {\isachardoublequoteopen}{\isacharbraceleft}{\isacharparenleft}x{\isacharcomma}f\ x{\isacharparenright}{\isacharbar}\ x{\isachardot}\ P\ x{\isacharbraceright}{\isacharcomma}{\isacharcomma}xx\ \ \ {\isacharequal}\ \ \ f\ xx{\isachardoublequoteclose}\isanewline
%
\isadelimproof
%
\endisadelimproof
%
\isatagproof
\isacommand{proof}\isamarkupfalse%
\ {\isacharminus}\isanewline
\ \ \isacommand{let}\isamarkupfalse%
\ {\isacharquery}F{\isacharequal}{\isachardoublequoteopen}{\isacharbraceleft}{\isacharparenleft}x{\isacharcomma}f\ x{\isacharparenright}{\isacharbar}\ x{\isachardot}\ P\ x{\isacharbraceright}{\isachardoublequoteclose}\ \isacommand{let}\isamarkupfalse%
\ {\isacharquery}X{\isacharequal}{\isachardoublequoteopen}{\isacharquery}F{\isacharbackquote}{\isacharbackquote}{\isacharbraceleft}xx{\isacharbraceright}{\isachardoublequoteclose}\isanewline
\ \ \isacommand{have}\isamarkupfalse%
\ {\isachardoublequoteopen}{\isacharquery}X{\isacharequal}{\isacharbraceleft}f\ xx{\isacharbraceright}{\isachardoublequoteclose}\ \isacommand{using}\isamarkupfalse%
\ Image{\isacharunderscore}def\ assms\ \isacommand{by}\isamarkupfalse%
\ blast\ \isacommand{thus}\isamarkupfalse%
\ {\isacharquery}thesis\ \isacommand{by}\isamarkupfalse%
\ fastforce\ \isanewline
\isacommand{qed}\isamarkupfalse%
%
\endisatagproof
{\isafoldproof}%
%
\isadelimproof
\isanewline
%
\endisadelimproof
\isanewline
\isacommand{lemma}\isamarkupfalse%
\ graphEqImage{\isacharcolon}\ \isanewline
\ \ \isakeyword{assumes}\ {\isachardoublequoteopen}x\ {\isasymin}\ X{\isachardoublequoteclose}\ \isanewline
\ \ \isakeyword{shows}\ {\isachardoublequoteopen}graph\ X\ f{\isacharcomma}{\isacharcomma}x\ \ \ {\isacharequal}\ \ \ f\ x{\isachardoublequoteclose}\ \isanewline
%
\isadelimproof
\ \ %
\endisadelimproof
%
\isatagproof
\isacommand{unfolding}\isamarkupfalse%
\ graph{\isacharunderscore}def\ \isacommand{using}\isamarkupfalse%
\ assms\ lm{\isadigit{0}}{\isadigit{8}}{\isadigit{5}}\ \isacommand{by}\isamarkupfalse%
\ {\isacharparenleft}metis\ {\isacharparenleft}mono{\isacharunderscore}tags{\isacharparenright}\ Gr{\isacharunderscore}def{\isacharparenright}%
\endisatagproof
{\isafoldproof}%
%
\isadelimproof
\isanewline
%
\endisadelimproof
\isanewline
\isacommand{lemma}\isamarkupfalse%
\ lm{\isadigit{0}}{\isadigit{8}}{\isadigit{6}}{\isacharcolon}\ \isanewline
\ \ {\isachardoublequoteopen}Graph\ f{\isacharcomma}{\isacharcomma}x\ \ \ \ {\isacharequal}\ \ \ \ f\ x{\isachardoublequoteclose}\ \isanewline
%
\isadelimproof
\ \ %
\endisadelimproof
%
\isatagproof
\isacommand{using}\isamarkupfalse%
\ UNIV{\isacharunderscore}I\ graphEqImage\ lm{\isadigit{0}}{\isadigit{0}}{\isadigit{5}}\ \isacommand{by}\isamarkupfalse%
\ {\isacharparenleft}metis{\isacharparenleft}no{\isacharunderscore}types{\isacharparenright}{\isacharparenright}%
\endisatagproof
{\isafoldproof}%
%
\isadelimproof
\isanewline
%
\endisadelimproof
\isanewline
\isacommand{lemma}\isamarkupfalse%
\ lm{\isadigit{0}}{\isadigit{8}}{\isadigit{7}}{\isacharcolon}\ \isanewline
\ \ {\isachardoublequoteopen}toFunction\ {\isacharparenleft}Graph\ f{\isacharparenright}\ \ \ \ {\isacharequal}\ \ \ \ f{\isachardoublequoteclose}\ \ \ \ {\isacharparenleft}\isakeyword{is}\ {\isachardoublequoteopen}{\isacharquery}L{\isacharequal}{\isacharunderscore}{\isachardoublequoteclose}{\isacharparenright}\ \isanewline
%
\isadelimproof
%
\endisadelimproof
%
\isatagproof
\isacommand{proof}\isamarkupfalse%
\ {\isacharminus}\isanewline
\ \ \isacommand{{\isacharbraceleft}}\isamarkupfalse%
\isacommand{fix}\isamarkupfalse%
\ x\ \isacommand{have}\isamarkupfalse%
\ {\isachardoublequoteopen}{\isacharquery}L\ x{\isacharequal}f\ x{\isachardoublequoteclose}\ \isacommand{unfolding}\isamarkupfalse%
\ toFunction{\isacharunderscore}def\ lm{\isadigit{0}}{\isadigit{8}}{\isadigit{6}}\ \isacommand{by}\isamarkupfalse%
\ metis\isacommand{{\isacharbraceright}}\isamarkupfalse%
\ \isanewline
\ \ \isacommand{thus}\isamarkupfalse%
\ {\isacharquery}thesis\ \isacommand{by}\isamarkupfalse%
\ blast\ \isanewline
\isacommand{qed}\isamarkupfalse%
%
\endisatagproof
{\isafoldproof}%
%
\isadelimproof
\isanewline
%
\endisadelimproof
\isanewline
\isacommand{lemma}\isamarkupfalse%
\ lm{\isadigit{0}}{\isadigit{8}}{\isadigit{8}}{\isacharcolon}\ \isanewline
\ \ {\isachardoublequoteopen}R\ outside\ X\ {\isasymsubseteq}\ R{\isachardoublequoteclose}\ \isanewline
%
\isadelimproof
\ \ %
\endisadelimproof
%
\isatagproof
\isacommand{by}\isamarkupfalse%
\ {\isacharparenleft}metis\ outside{\isacharunderscore}union{\isacharunderscore}restrict\ subset{\isacharunderscore}Un{\isacharunderscore}eq\ sup{\isacharunderscore}left{\isacharunderscore}idem{\isacharparenright}%
\endisatagproof
{\isafoldproof}%
%
\isadelimproof
\isanewline
%
\endisadelimproof
\isanewline
\isacommand{lemma}\isamarkupfalse%
\ lm{\isadigit{0}}{\isadigit{8}}{\isadigit{9}}{\isacharcolon}\ \isanewline
\ \ {\isachardoublequoteopen}Range{\isacharparenleft}f\ outside\ X{\isacharparenright}\ {\isasymsupseteq}\ {\isacharparenleft}Range\ f{\isacharparenright}{\isacharminus}{\isacharparenleft}f{\isacharbackquote}{\isacharbackquote}X{\isacharparenright}{\isachardoublequoteclose}\ \isanewline
%
\isadelimproof
\ \ %
\endisadelimproof
%
\isatagproof
\isacommand{using}\isamarkupfalse%
\ assms\ Outside{\isacharunderscore}def\ \isacommand{by}\isamarkupfalse%
\ blast%
\endisatagproof
{\isafoldproof}%
%
\isadelimproof
\isanewline
%
\endisadelimproof
\isanewline
\isacommand{lemma}\isamarkupfalse%
\ lm{\isadigit{0}}{\isadigit{9}}{\isadigit{0}}{\isacharcolon}\ \isanewline
\ \ \isakeyword{assumes}\ {\isachardoublequoteopen}runiq\ P{\isachardoublequoteclose}\ \isanewline
\ \ \isakeyword{shows}\ {\isachardoublequoteopen}{\isacharparenleft}P{\isasyminverse}{\isacharbackquote}{\isacharbackquote}{\isacharparenleft}{\isacharparenleft}Range\ P{\isacharparenright}{\isacharminus}Y{\isacharparenright}{\isacharparenright}\ {\isasyminter}\ {\isacharparenleft}{\isacharparenleft}P{\isasyminverse}{\isacharparenright}{\isacharbackquote}{\isacharbackquote}Y{\isacharparenright}\ \ \ {\isacharequal}\ \ \ {\isacharbraceleft}{\isacharbraceright}{\isachardoublequoteclose}\isanewline
%
\isadelimproof
\ \ %
\endisadelimproof
%
\isatagproof
\isacommand{using}\isamarkupfalse%
\ assms\ rightUniqueFunctionAfterInverse\ \isacommand{by}\isamarkupfalse%
\ blast%
\endisatagproof
{\isafoldproof}%
%
\isadelimproof
\isanewline
%
\endisadelimproof
\isanewline
\isacommand{lemma}\isamarkupfalse%
\ lm{\isadigit{0}}{\isadigit{9}}{\isadigit{1}}{\isacharcolon}\ \isanewline
\ \ \isakeyword{assumes}\ {\isachardoublequoteopen}runiq\ {\isacharparenleft}P{\isasyminverse}{\isacharparenright}{\isachardoublequoteclose}\isanewline
\ \ \isakeyword{shows}\ {\isachardoublequoteopen}{\isacharparenleft}P{\isacharbackquote}{\isacharbackquote}{\isacharparenleft}{\isacharparenleft}Domain\ P{\isacharparenright}\ {\isacharminus}\ X{\isacharparenright}{\isacharparenright}\ {\isasyminter}\ {\isacharparenleft}P{\isacharbackquote}{\isacharbackquote}X{\isacharparenright}\ \ {\isacharequal}\ \ {\isacharbraceleft}{\isacharbraceright}{\isachardoublequoteclose}\isanewline
%
\isadelimproof
\ \ %
\endisadelimproof
%
\isatagproof
\isacommand{using}\isamarkupfalse%
\ assms\ rightUniqueFunctionAfterInverse\ \isacommand{by}\isamarkupfalse%
\ fast%
\endisatagproof
{\isafoldproof}%
%
\isadelimproof
\isanewline
%
\endisadelimproof
\isanewline
\isacommand{lemma}\isamarkupfalse%
\ lm{\isadigit{0}}{\isadigit{9}}{\isadigit{2}}{\isacharcolon}\ \isanewline
\ \ \isakeyword{assumes}\ {\isachardoublequoteopen}runiq\ f{\isachardoublequoteclose}\ {\isachardoublequoteopen}runiq\ {\isacharparenleft}f{\isacharcircum}{\isacharminus}{\isadigit{1}}{\isacharparenright}{\isachardoublequoteclose}\ \isanewline
\ \ \isakeyword{shows}\ {\isachardoublequoteopen}Range{\isacharparenleft}f\ outside\ X{\isacharparenright}\ {\isasymsubseteq}\ {\isacharparenleft}Range\ f{\isacharparenright}{\isacharminus}{\isacharparenleft}f{\isacharbackquote}{\isacharbackquote}X{\isacharparenright}{\isachardoublequoteclose}\ \isanewline
%
\isadelimproof
\ \ %
\endisadelimproof
%
\isatagproof
\isacommand{using}\isamarkupfalse%
\ assms\ Diff{\isacharunderscore}triv\ lm{\isadigit{0}}{\isadigit{9}}{\isadigit{1}}\ lm{\isadigit{0}}{\isadigit{6}}{\isadigit{6}}\ Diff{\isacharunderscore}iff\ ImageE\ Range{\isacharunderscore}iff\ subsetI\ \isacommand{by}\isamarkupfalse%
\ metis%
\endisatagproof
{\isafoldproof}%
%
\isadelimproof
\ \isanewline
%
\endisadelimproof
\isanewline
\isacommand{lemma}\isamarkupfalse%
\ rangeOutside{\isacharcolon}\ \isanewline
\ \ \isakeyword{assumes}\ {\isachardoublequoteopen}runiq\ f{\isachardoublequoteclose}\ {\isachardoublequoteopen}runiq\ {\isacharparenleft}f{\isacharcircum}{\isacharminus}{\isadigit{1}}{\isacharparenright}{\isachardoublequoteclose}\ \isanewline
\ \ \isakeyword{shows}\ {\isachardoublequoteopen}Range{\isacharparenleft}f\ outside\ X{\isacharparenright}\ {\isacharequal}\ {\isacharparenleft}Range\ f{\isacharparenright}{\isacharminus}{\isacharparenleft}f{\isacharbackquote}{\isacharbackquote}X{\isacharparenright}{\isachardoublequoteclose}\ \isanewline
%
\isadelimproof
\ \ %
\endisadelimproof
%
\isatagproof
\isacommand{using}\isamarkupfalse%
\ assms\ lm{\isadigit{0}}{\isadigit{8}}{\isadigit{9}}\ lm{\isadigit{0}}{\isadigit{9}}{\isadigit{2}}\ \isacommand{by}\isamarkupfalse%
\ {\isacharparenleft}metis\ order{\isacharunderscore}class{\isachardot}order{\isachardot}antisym{\isacharparenright}%
\endisatagproof
{\isafoldproof}%
%
\isadelimproof
\isanewline
%
\endisadelimproof
\isanewline
\isanewline
\isacommand{lemma}\isamarkupfalse%
\ unionIntersectionEmpty{\isacharcolon}\ \isanewline
\ \ {\isachardoublequoteopen}{\isacharparenleft}{\isasymforall}x{\isasymin}X{\isachardot}\ {\isasymforall}y{\isasymin}Y{\isachardot}\ x{\isasyminter}y\ {\isacharequal}\ {\isacharbraceleft}{\isacharbraceright}{\isacharparenright}\ {\isacharequal}\ {\isacharparenleft}{\isacharparenleft}{\isasymUnion}X{\isacharparenright}{\isasyminter}{\isacharparenleft}{\isasymUnion}\ Y{\isacharparenright}{\isacharequal}{\isacharbraceleft}{\isacharbraceright}{\isacharparenright}{\isachardoublequoteclose}\isanewline
%
\isadelimproof
\ \ %
\endisadelimproof
%
\isatagproof
\isacommand{by}\isamarkupfalse%
\ blast%
\endisatagproof
{\isafoldproof}%
%
\isadelimproof
\isanewline
%
\endisadelimproof
\isanewline
\isacommand{lemma}\isamarkupfalse%
\ setEqualityAsDifference{\isacharcolon}\ \isanewline
\ \ {\isachardoublequoteopen}{\isacharbraceleft}x{\isacharbraceright}{\isacharminus}{\isacharbraceleft}y{\isacharbraceright}\ {\isacharequal}\ {\isacharbraceleft}{\isacharbraceright}\ \ {\isacharequal}\ \ {\isacharparenleft}x\ {\isacharequal}\ y{\isacharparenright}{\isachardoublequoteclose}\ \isanewline
%
\isadelimproof
\ \ %
\endisadelimproof
%
\isatagproof
\isacommand{by}\isamarkupfalse%
\ auto%
\endisatagproof
{\isafoldproof}%
%
\isadelimproof
\isanewline
%
\endisadelimproof
\isanewline
\isacommand{lemma}\isamarkupfalse%
\ lm{\isadigit{0}}{\isadigit{9}}{\isadigit{3}}{\isacharcolon}\ \isanewline
\ \ \isakeyword{assumes}\ {\isachardoublequoteopen}R\ {\isasymnoteq}\ {\isacharbraceleft}{\isacharbraceright}{\isachardoublequoteclose}\ {\isachardoublequoteopen}Domain\ R\ {\isasyminter}\ X\ {\isasymnoteq}\ {\isacharbraceleft}{\isacharbraceright}{\isachardoublequoteclose}\ \isanewline
\ \ \isakeyword{shows}\ {\isachardoublequoteopen}R{\isacharbackquote}{\isacharbackquote}X\ {\isasymnoteq}\ {\isacharbraceleft}{\isacharbraceright}{\isachardoublequoteclose}\ \isanewline
%
\isadelimproof
\ \ %
\endisadelimproof
%
\isatagproof
\isacommand{using}\isamarkupfalse%
\ assms\ \isacommand{by}\isamarkupfalse%
\ blast%
\endisatagproof
{\isafoldproof}%
%
\isadelimproof
\isanewline
%
\endisadelimproof
\isanewline
\isacommand{lemma}\isamarkupfalse%
\ lm{\isadigit{0}}{\isadigit{9}}{\isadigit{4}}{\isacharcolon}\ \isanewline
\ \ {\isachardoublequoteopen}R{\isacharbackquote}{\isacharbackquote}{\isacharbraceleft}{\isacharbraceright}{\isacharequal}{\isacharbraceleft}{\isacharbraceright}{\isachardoublequoteclose}\ \isanewline
%
\isadelimproof
\ \ %
\endisadelimproof
%
\isatagproof
\isacommand{by}\isamarkupfalse%
\ {\isacharparenleft}metis\ Image{\isacharunderscore}empty{\isacharparenright}%
\endisatagproof
{\isafoldproof}%
%
\isadelimproof
\isanewline
%
\endisadelimproof
\isanewline
\isacommand{lemma}\isamarkupfalse%
\ lm{\isadigit{0}}{\isadigit{9}}{\isadigit{5}}{\isacharcolon}\ \isanewline
\ \ {\isachardoublequoteopen}R\ {\isasymsubseteq}\ {\isacharparenleft}Domain\ R{\isacharparenright}\ {\isasymtimes}\ {\isacharparenleft}Range\ R{\isacharparenright}{\isachardoublequoteclose}\ \isanewline
%
\isadelimproof
\ \ %
\endisadelimproof
%
\isatagproof
\isacommand{by}\isamarkupfalse%
\ auto%
\endisatagproof
{\isafoldproof}%
%
\isadelimproof
\isanewline
%
\endisadelimproof
\isanewline
\isacommand{lemma}\isamarkupfalse%
\ finiteRelationCharacterization{\isacharcolon}\ \isanewline
\ \ {\isachardoublequoteopen}{\isacharparenleft}finite\ {\isacharparenleft}Domain\ Q{\isacharparenright}\ {\isacharampersand}\ finite\ {\isacharparenleft}Range\ Q{\isacharparenright}{\isacharparenright}\ {\isacharequal}\ finite\ Q{\isachardoublequoteclose}\isanewline
%
\isadelimproof
\ \ %
\endisadelimproof
%
\isatagproof
\isacommand{using}\isamarkupfalse%
\ rev{\isacharunderscore}finite{\isacharunderscore}subset\ finite{\isacharunderscore}SigmaI\ lm{\isadigit{0}}{\isadigit{9}}{\isadigit{5}}\ finite{\isacharunderscore}Domain\ finite{\isacharunderscore}Range\ \isacommand{by}\isamarkupfalse%
\ metis%
\endisatagproof
{\isafoldproof}%
%
\isadelimproof
\isanewline
%
\endisadelimproof
\isanewline
\isacommand{lemma}\isamarkupfalse%
\ familyUnionFiniteEverySetFinite{\isacharcolon}\ \isanewline
\ \ \isakeyword{assumes}\ {\isachardoublequoteopen}finite\ {\isacharparenleft}{\isasymUnion}\ XX{\isacharparenright}{\isachardoublequoteclose}\ \isanewline
\ \ \isakeyword{shows}\ {\isachardoublequoteopen}{\isasymforall}X\ {\isasymin}\ XX{\isachardot}\ finite\ X{\isachardoublequoteclose}\ \isanewline
%
\isadelimproof
\ \ %
\endisadelimproof
%
\isatagproof
\isacommand{using}\isamarkupfalse%
\ assms\ \isacommand{by}\isamarkupfalse%
\ {\isacharparenleft}metis\ Union{\isacharunderscore}upper\ finite{\isacharunderscore}subset{\isacharparenright}%
\endisatagproof
{\isafoldproof}%
%
\isadelimproof
\isanewline
%
\endisadelimproof
\isanewline
\isacommand{lemma}\isamarkupfalse%
\ lm{\isadigit{0}}{\isadigit{9}}{\isadigit{6}}{\isacharcolon}\ \isanewline
\ \ \isakeyword{assumes}\ {\isachardoublequoteopen}runiq\ f{\isachardoublequoteclose}\ {\isachardoublequoteopen}X\ {\isasymsubseteq}\ {\isacharparenleft}f{\isacharcircum}{\isacharminus}{\isadigit{1}}{\isacharparenright}{\isacharbackquote}{\isacharbackquote}Y{\isachardoublequoteclose}\ \isanewline
\ \ \isakeyword{shows}\ {\isachardoublequoteopen}f{\isacharbackquote}{\isacharbackquote}X\ {\isasymsubseteq}\ Y{\isachardoublequoteclose}\ \isanewline
%
\isadelimproof
\ \ %
\endisadelimproof
%
\isatagproof
\isacommand{using}\isamarkupfalse%
\ assms\ rightUniqueFunctionAfterInverse\ \isacommand{by}\isamarkupfalse%
\ {\isacharparenleft}metis\ Image{\isacharunderscore}mono\ order{\isacharunderscore}refl\ subset{\isacharunderscore}trans{\isacharparenright}%
\endisatagproof
{\isafoldproof}%
%
\isadelimproof
\isanewline
%
\endisadelimproof
\isanewline
\isacommand{lemma}\isamarkupfalse%
\ lm{\isadigit{0}}{\isadigit{9}}{\isadigit{7}}{\isacharcolon}\ \isanewline
\ \ \isakeyword{assumes}\ {\isachardoublequoteopen}y\ {\isasymin}\ f{\isacharbackquote}{\isacharbackquote}{\isacharbraceleft}x{\isacharbraceright}{\isachardoublequoteclose}\ {\isachardoublequoteopen}runiq\ f{\isachardoublequoteclose}\ \isanewline
\ \ \isakeyword{shows}\ {\isachardoublequoteopen}f{\isacharcomma}{\isacharcomma}x\ {\isacharequal}\ y{\isachardoublequoteclose}\ \isanewline
%
\isadelimproof
\ \ %
\endisadelimproof
%
\isatagproof
\isacommand{using}\isamarkupfalse%
\ assms\ \isacommand{by}\isamarkupfalse%
\ {\isacharparenleft}metis\ Image{\isacharunderscore}singleton{\isacharunderscore}iff\ rightUniquePair{\isacharparenright}%
\endisatagproof
{\isafoldproof}%
%
\isadelimproof
%
\endisadelimproof
%
\isamarkupsection{Indicator function in set-theoretical form.%
}
\isamarkuptrue%
\isacommand{abbreviation}\isamarkupfalse%
\ \isanewline
\ \ {\isachardoublequoteopen}Outside{\isacharprime}\ X\ f\ {\isacharequal}{\isacharequal}\ f\ outside\ X{\isachardoublequoteclose}\isanewline
\isanewline
\isacommand{abbreviation}\isamarkupfalse%
\ \isanewline
\ \ {\isachardoublequoteopen}Chi\ X\ Y\ {\isacharequal}{\isacharequal}\ {\isacharparenleft}Y\ {\isasymtimes}\ {\isacharbraceleft}{\isadigit{0}}{\isacharcolon}{\isacharcolon}nat{\isacharbraceright}{\isacharparenright}\ {\isacharplus}{\isacharasterisk}\ {\isacharparenleft}X\ {\isasymtimes}\ {\isacharbraceleft}{\isadigit{1}}{\isacharbraceright}{\isacharparenright}{\isachardoublequoteclose}\isanewline
\ \ \isacommand{notation}\isamarkupfalse%
\ Chi\ {\isacharparenleft}\isakeyword{infix}\ {\isachardoublequoteopen}{\isacharless}{\isacharbar}{\isacharbar}{\isachardoublequoteclose}\ {\isadigit{8}}{\isadigit{0}}{\isacharparenright}\isanewline
\isanewline
\isacommand{abbreviation}\isamarkupfalse%
\ \isanewline
\ \ {\isachardoublequoteopen}chii\ X\ Y\ {\isacharequal}{\isacharequal}\ toFunction\ {\isacharparenleft}X\ {\isacharless}{\isacharbar}{\isacharbar}\ Y{\isacharparenright}{\isachardoublequoteclose}\isanewline
\ \ \isacommand{notation}\isamarkupfalse%
\ chii\ {\isacharparenleft}\isakeyword{infix}\ {\isachardoublequoteopen}{\isacharless}{\isacharbar}{\isachardoublequoteclose}\ {\isadigit{8}}{\isadigit{0}}{\isacharparenright}\isanewline
\isanewline
\isanewline
\isacommand{abbreviation}\isamarkupfalse%
\ \isanewline
\ \ {\isachardoublequoteopen}chi\ X\ {\isacharequal}{\isacharequal}\ indicator\ X{\isachardoublequoteclose}\isanewline
\isanewline
\isacommand{lemma}\isamarkupfalse%
\ lm{\isadigit{0}}{\isadigit{9}}{\isadigit{8}}{\isacharcolon}\ \isanewline
\ \ {\isachardoublequoteopen}runiq\ {\isacharparenleft}X\ {\isacharless}{\isacharbar}{\isacharbar}\ Y{\isacharparenright}{\isachardoublequoteclose}\ \isanewline
%
\isadelimproof
\ \ %
\endisadelimproof
%
\isatagproof
\isacommand{by}\isamarkupfalse%
\ {\isacharparenleft}rule\ lm{\isadigit{0}}{\isadigit{1}}{\isadigit{4}}{\isacharparenright}%
\endisatagproof
{\isafoldproof}%
%
\isadelimproof
\isanewline
%
\endisadelimproof
\isanewline
\isacommand{lemma}\isamarkupfalse%
\ lm{\isadigit{0}}{\isadigit{9}}{\isadigit{9}}{\isacharcolon}\ \isanewline
\ \ \isakeyword{assumes}\ {\isachardoublequoteopen}x\ {\isasymin}\ X{\isachardoublequoteclose}\ \isanewline
\ \ \isakeyword{shows}\ {\isachardoublequoteopen}{\isadigit{1}}\ {\isasymin}\ {\isacharparenleft}X\ {\isacharless}{\isacharbar}{\isacharbar}\ Y{\isacharparenright}\ {\isacharbackquote}{\isacharbackquote}\ {\isacharbraceleft}x{\isacharbraceright}{\isachardoublequoteclose}\ \isanewline
%
\isadelimproof
\ \ %
\endisadelimproof
%
\isatagproof
\isacommand{using}\isamarkupfalse%
\ assms\ toFunction{\isacharunderscore}def\ paste{\isacharunderscore}def\ Outside{\isacharunderscore}def\ runiq{\isacharunderscore}def\ lm{\isadigit{0}}{\isadigit{1}}{\isadigit{4}}\ \isacommand{by}\isamarkupfalse%
\ blast%
\endisatagproof
{\isafoldproof}%
%
\isadelimproof
\isanewline
%
\endisadelimproof
\isanewline
\isacommand{lemma}\isamarkupfalse%
\ lm{\isadigit{1}}{\isadigit{0}}{\isadigit{0}}{\isacharcolon}\ \isanewline
\ \ \isakeyword{assumes}\ {\isachardoublequoteopen}x\ {\isasymin}\ Y{\isacharminus}X{\isachardoublequoteclose}\ \isanewline
\ \ \isakeyword{shows}\ {\isachardoublequoteopen}{\isadigit{0}}\ {\isasymin}\ {\isacharparenleft}X\ {\isacharless}{\isacharbar}{\isacharbar}\ Y{\isacharparenright}\ {\isacharbackquote}{\isacharbackquote}\ {\isacharbraceleft}x{\isacharbraceright}{\isachardoublequoteclose}\ \isanewline
%
\isadelimproof
\ \ %
\endisadelimproof
%
\isatagproof
\isacommand{using}\isamarkupfalse%
\ assms\ toFunction{\isacharunderscore}def\ paste{\isacharunderscore}def\ Outside{\isacharunderscore}def\ runiq{\isacharunderscore}def\ lm{\isadigit{0}}{\isadigit{1}}{\isadigit{4}}\ \isacommand{by}\isamarkupfalse%
\ blast%
\endisatagproof
{\isafoldproof}%
%
\isadelimproof
\isanewline
%
\endisadelimproof
\isanewline
\isacommand{lemma}\isamarkupfalse%
\ lm{\isadigit{1}}{\isadigit{0}}{\isadigit{1}}{\isacharcolon}\ \isanewline
\ \ \isakeyword{assumes}\ {\isachardoublequoteopen}x\ {\isasymin}\ X\ {\isasymunion}\ Y{\isachardoublequoteclose}\ \isanewline
\ \ \isakeyword{shows}\ {\isachardoublequoteopen}{\isacharparenleft}X\ {\isacharless}{\isacharbar}{\isacharbar}\ Y{\isacharparenright}{\isacharcomma}{\isacharcomma}x\ {\isacharequal}\ chi\ X\ x{\isachardoublequoteclose}\ {\isacharparenleft}\isakeyword{is}\ {\isachardoublequoteopen}{\isacharquery}L{\isacharequal}{\isacharquery}R{\isachardoublequoteclose}{\isacharparenright}\ \isanewline
%
\isadelimproof
\ \ %
\endisadelimproof
%
\isatagproof
\isacommand{using}\isamarkupfalse%
\ assms\ lm{\isadigit{0}}{\isadigit{1}}{\isadigit{4}}\ lm{\isadigit{0}}{\isadigit{9}}{\isadigit{9}}\ lm{\isadigit{1}}{\isadigit{0}}{\isadigit{0}}\ lm{\isadigit{0}}{\isadigit{9}}{\isadigit{7}}\ \isanewline
\ \ \isacommand{by}\isamarkupfalse%
\ {\isacharparenleft}metis\ DiffI\ Un{\isacharunderscore}iff\ indicator{\isacharunderscore}simps{\isacharparenleft}{\isadigit{1}}{\isacharparenright}\ indicator{\isacharunderscore}simps{\isacharparenleft}{\isadigit{2}}{\isacharparenright}{\isacharparenright}%
\endisatagproof
{\isafoldproof}%
%
\isadelimproof
\isanewline
%
\endisadelimproof
\isanewline
\isacommand{lemma}\isamarkupfalse%
\ lm{\isadigit{1}}{\isadigit{0}}{\isadigit{2}}{\isacharcolon}\ \isanewline
\ \ \isakeyword{assumes}\ {\isachardoublequoteopen}x\ {\isasymin}\ X\ {\isasymunion}\ Y{\isachardoublequoteclose}\ \isanewline
\ \ \isakeyword{shows}\ {\isachardoublequoteopen}{\isacharparenleft}X\ {\isacharless}{\isacharbar}\ Y{\isacharparenright}\ x\ {\isacharequal}\ chi\ X\ x{\isachardoublequoteclose}\ \isanewline
%
\isadelimproof
\ \ %
\endisadelimproof
%
\isatagproof
\isacommand{using}\isamarkupfalse%
\ assms\ toFunction{\isacharunderscore}def\ lm{\isadigit{1}}{\isadigit{0}}{\isadigit{1}}\ \isacommand{by}\isamarkupfalse%
\ metis%
\endisatagproof
{\isafoldproof}%
%
\isadelimproof
\isanewline
%
\endisadelimproof
\isanewline
\isacommand{corollary}\isamarkupfalse%
\ lm{\isadigit{1}}{\isadigit{0}}{\isadigit{3}}{\isacharcolon}\ \isanewline
\ \ {\isachardoublequoteopen}setsum\ {\isacharparenleft}X\ {\isacharless}{\isacharbar}\ Y{\isacharparenright}\ {\isacharparenleft}X{\isasymunion}Y{\isacharparenright}\ {\isacharequal}\ setsum\ {\isacharparenleft}chi\ X{\isacharparenright}\ {\isacharparenleft}X{\isasymunion}Y{\isacharparenright}{\isachardoublequoteclose}\isanewline
%
\isadelimproof
\ \ %
\endisadelimproof
%
\isatagproof
\isacommand{using}\isamarkupfalse%
\ lm{\isadigit{1}}{\isadigit{0}}{\isadigit{2}}\ setsum{\isachardot}cong\ \isacommand{by}\isamarkupfalse%
\ metis%
\endisatagproof
{\isafoldproof}%
%
\isadelimproof
\isanewline
%
\endisadelimproof
\isanewline
\isacommand{corollary}\isamarkupfalse%
\ lm{\isadigit{1}}{\isadigit{0}}{\isadigit{4}}{\isacharcolon}\ \isanewline
\ \ \isakeyword{assumes}\ {\isachardoublequoteopen}{\isasymforall}x{\isasymin}X{\isachardot}\ f\ x\ {\isacharequal}\ g\ x{\isachardoublequoteclose}\ \isanewline
\ \ \isakeyword{shows}\ {\isachardoublequoteopen}setsum\ f\ X\ {\isacharequal}\ setsum\ g\ X{\isachardoublequoteclose}\ \isanewline
%
\isadelimproof
\ \ %
\endisadelimproof
%
\isatagproof
\isacommand{using}\isamarkupfalse%
\ assms\ \isacommand{by}\isamarkupfalse%
\ {\isacharparenleft}metis\ {\isacharparenleft}poly{\isacharunderscore}guards{\isacharunderscore}query{\isacharparenright}\ setsum{\isachardot}cong{\isacharparenright}%
\endisatagproof
{\isafoldproof}%
%
\isadelimproof
\isanewline
%
\endisadelimproof
\isanewline
\isacommand{corollary}\isamarkupfalse%
\ lm{\isadigit{1}}{\isadigit{0}}{\isadigit{5}}{\isacharcolon}\ \isanewline
\ \ \isakeyword{assumes}\ {\isachardoublequoteopen}{\isasymforall}x{\isasymin}X{\isachardot}\ f\ x\ {\isacharequal}\ g\ x{\isachardoublequoteclose}\ {\isachardoublequoteopen}Y{\isasymsubseteq}X{\isachardoublequoteclose}\ \isanewline
\ \ \isakeyword{shows}\ {\isachardoublequoteopen}setsum\ f\ Y\ {\isacharequal}\ setsum\ g\ Y{\isachardoublequoteclose}\ \isanewline
%
\isadelimproof
\ \ %
\endisadelimproof
%
\isatagproof
\isacommand{using}\isamarkupfalse%
\ assms\ lm{\isadigit{1}}{\isadigit{0}}{\isadigit{4}}\ \isacommand{by}\isamarkupfalse%
\ {\isacharparenleft}metis\ contra{\isacharunderscore}subsetD{\isacharparenright}%
\endisatagproof
{\isafoldproof}%
%
\isadelimproof
\isanewline
%
\endisadelimproof
\isanewline
\isacommand{corollary}\isamarkupfalse%
\ lm{\isadigit{1}}{\isadigit{0}}{\isadigit{6}}{\isacharcolon}\ \isanewline
\ \ \isakeyword{assumes}\ {\isachardoublequoteopen}Z\ {\isasymsubseteq}\ X\ {\isasymunion}\ Y{\isachardoublequoteclose}\ \isanewline
\ \ \isakeyword{shows}\ {\isachardoublequoteopen}setsum\ {\isacharparenleft}X\ {\isacharless}{\isacharbar}\ Y{\isacharparenright}\ Z\ {\isacharequal}\ setsum\ {\isacharparenleft}chi\ X{\isacharparenright}\ Z{\isachardoublequoteclose}\ \ \isanewline
%
\isadelimproof
%
\endisadelimproof
%
\isatagproof
\isacommand{proof}\isamarkupfalse%
\ {\isacharminus}\ \isanewline
\ \ \isacommand{have}\isamarkupfalse%
\ {\isachardoublequoteopen}{\isacharbang}x{\isacharcolon}Z{\isachardot}{\isacharparenleft}X{\isacharless}{\isacharbar}Y{\isacharparenright}\ x{\isacharequal}{\isacharparenleft}chi\ X{\isacharparenright}\ x{\isachardoublequoteclose}\ \isacommand{using}\isamarkupfalse%
\ assms\ lm{\isadigit{1}}{\isadigit{0}}{\isadigit{2}}\ in{\isacharunderscore}mono\ \isacommand{by}\isamarkupfalse%
\ metis\ \isanewline
\ \ \isacommand{thus}\isamarkupfalse%
\ {\isacharquery}thesis\ \isacommand{using}\isamarkupfalse%
\ lm{\isadigit{1}}{\isadigit{0}}{\isadigit{4}}\ \isacommand{by}\isamarkupfalse%
\ blast\ \isanewline
\isacommand{qed}\isamarkupfalse%
%
\endisatagproof
{\isafoldproof}%
%
\isadelimproof
\isanewline
%
\endisadelimproof
\isanewline
\isacommand{corollary}\isamarkupfalse%
\ lm{\isadigit{1}}{\isadigit{0}}{\isadigit{7}}{\isacharcolon}\ \isanewline
\ \ {\isachardoublequoteopen}setsum\ {\isacharparenleft}chi\ X{\isacharparenright}\ {\isacharparenleft}Z\ {\isacharminus}\ X{\isacharparenright}\ {\isacharequal}\ {\isadigit{0}}{\isachardoublequoteclose}\ \isanewline
%
\isadelimproof
\ \ %
\endisadelimproof
%
\isatagproof
\isacommand{by}\isamarkupfalse%
\ simp%
\endisatagproof
{\isafoldproof}%
%
\isadelimproof
\isanewline
%
\endisadelimproof
\isanewline
\isacommand{corollary}\isamarkupfalse%
\ lm{\isadigit{1}}{\isadigit{0}}{\isadigit{8}}{\isacharcolon}\ \isanewline
\ \ \isakeyword{assumes}\ {\isachardoublequoteopen}Z\ {\isasymsubseteq}\ X\ {\isasymunion}\ Y{\isachardoublequoteclose}\ \isanewline
\ \ \isakeyword{shows}\ {\isachardoublequoteopen}setsum\ {\isacharparenleft}X\ {\isacharless}{\isacharbar}\ Y{\isacharparenright}\ {\isacharparenleft}Z\ {\isacharminus}\ X{\isacharparenright}\ {\isacharequal}\ {\isadigit{0}}{\isachardoublequoteclose}\ \isanewline
%
\isadelimproof
\ \ %
\endisadelimproof
%
\isatagproof
\isacommand{using}\isamarkupfalse%
\ assms\ lm{\isadigit{1}}{\isadigit{0}}{\isadigit{7}}\ lm{\isadigit{1}}{\isadigit{0}}{\isadigit{6}}\ Diff{\isacharunderscore}iff\ in{\isacharunderscore}mono\ subsetI\ \isacommand{by}\isamarkupfalse%
\ metis%
\endisatagproof
{\isafoldproof}%
%
\isadelimproof
\isanewline
%
\endisadelimproof
\isanewline
\isacommand{corollary}\isamarkupfalse%
\ lm{\isadigit{1}}{\isadigit{0}}{\isadigit{9}}{\isacharcolon}\ \isanewline
\ \ \isakeyword{assumes}\ {\isachardoublequoteopen}finite\ Z{\isachardoublequoteclose}\ \isanewline
\ \ \isakeyword{shows}\ {\isachardoublequoteopen}setsum\ {\isacharparenleft}X\ {\isacharless}{\isacharbar}\ Y{\isacharparenright}\ Z\ \ \ \ {\isacharequal}\ \ \ \ setsum\ {\isacharparenleft}X\ {\isacharless}{\isacharbar}\ Y{\isacharparenright}\ {\isacharparenleft}Z\ {\isacharminus}\ X{\isacharparenright}\ \ \ {\isacharplus}\ \ {\isacharparenleft}setsum\ {\isacharparenleft}X\ {\isacharless}{\isacharbar}\ Y{\isacharparenright}\ {\isacharparenleft}Z\ {\isasyminter}\ X{\isacharparenright}{\isacharparenright}{\isachardoublequoteclose}\ \isanewline
%
\isadelimproof
\ \ %
\endisadelimproof
%
\isatagproof
\isacommand{using}\isamarkupfalse%
\ lm{\isadigit{0}}{\isadigit{7}}{\isadigit{8}}\ assms\ \isacommand{by}\isamarkupfalse%
\ blast%
\endisatagproof
{\isafoldproof}%
%
\isadelimproof
\isanewline
%
\endisadelimproof
\isanewline
\isacommand{corollary}\isamarkupfalse%
\ lm{\isadigit{1}}{\isadigit{1}}{\isadigit{0}}{\isacharcolon}\ \isanewline
\ \ \isakeyword{assumes}\ {\isachardoublequoteopen}Z\ {\isasymsubseteq}\ X\ {\isasymunion}\ Y{\isachardoublequoteclose}\ {\isachardoublequoteopen}finite\ Z{\isachardoublequoteclose}\ \isanewline
\ \ \isakeyword{shows}\ {\isachardoublequoteopen}setsum\ {\isacharparenleft}X\ {\isacharless}{\isacharbar}\ Y{\isacharparenright}\ Z\ {\isacharequal}\ setsum\ {\isacharparenleft}X\ {\isacharless}{\isacharbar}\ Y{\isacharparenright}\ {\isacharparenleft}Z\ {\isasyminter}\ X{\isacharparenright}{\isachardoublequoteclose}\ \isanewline
%
\isadelimproof
\ \ %
\endisadelimproof
%
\isatagproof
\isacommand{using}\isamarkupfalse%
\ assms\ lm{\isadigit{0}}{\isadigit{7}}{\isadigit{8}}\ lm{\isadigit{1}}{\isadigit{0}}{\isadigit{8}}\ comm{\isacharunderscore}monoid{\isacharunderscore}add{\isacharunderscore}class{\isachardot}add{\isachardot}left{\isacharunderscore}neutral\ \isacommand{by}\isamarkupfalse%
\ metis%
\endisatagproof
{\isafoldproof}%
%
\isadelimproof
\isanewline
%
\endisadelimproof
\isanewline
\isacommand{corollary}\isamarkupfalse%
\ lm{\isadigit{1}}{\isadigit{1}}{\isadigit{1}}{\isacharcolon}\ \isanewline
\ \ \isakeyword{assumes}\ {\isachardoublequoteopen}finite\ Z{\isachardoublequoteclose}\ \isanewline
\ \ \isakeyword{shows}\ {\isachardoublequoteopen}setsum\ {\isacharparenleft}chi\ X{\isacharparenright}\ Z\ {\isacharequal}\ card\ {\isacharparenleft}X\ {\isasyminter}\ Z{\isacharparenright}{\isachardoublequoteclose}\ \isanewline
%
\isadelimproof
\ \ %
\endisadelimproof
%
\isatagproof
\isacommand{using}\isamarkupfalse%
\ assms\ setsum{\isacharunderscore}indicator{\isacharunderscore}eq{\isacharunderscore}card\ \isacommand{by}\isamarkupfalse%
\ {\isacharparenleft}metis\ Int{\isacharunderscore}commute{\isacharparenright}%
\endisatagproof
{\isafoldproof}%
%
\isadelimproof
\isanewline
%
\endisadelimproof
\isanewline
\isacommand{corollary}\isamarkupfalse%
\ lm{\isadigit{1}}{\isadigit{1}}{\isadigit{2}}{\isacharcolon}\ \isanewline
\ \ \isakeyword{assumes}\ {\isachardoublequoteopen}Z\ {\isasymsubseteq}\ X\ {\isasymunion}\ Y{\isachardoublequoteclose}\ {\isachardoublequoteopen}finite\ Z{\isachardoublequoteclose}\ \isanewline
\ \ \isakeyword{shows}\ {\isachardoublequoteopen}setsum\ {\isacharparenleft}X\ {\isacharless}{\isacharbar}\ Y{\isacharparenright}\ Z\ {\isacharequal}\ card\ {\isacharparenleft}Z\ {\isasyminter}\ X{\isacharparenright}{\isachardoublequoteclose}\isanewline
%
\isadelimproof
\ \ %
\endisadelimproof
%
\isatagproof
\isacommand{using}\isamarkupfalse%
\ assms\ lm{\isadigit{1}}{\isadigit{1}}{\isadigit{1}}\ \isacommand{by}\isamarkupfalse%
\ {\isacharparenleft}metis\ lm{\isadigit{1}}{\isadigit{0}}{\isadigit{6}}\ setsum{\isacharunderscore}indicator{\isacharunderscore}eq{\isacharunderscore}card{\isacharparenright}%
\endisatagproof
{\isafoldproof}%
%
\isadelimproof
\isanewline
%
\endisadelimproof
\isanewline
\isacommand{corollary}\isamarkupfalse%
\ subsetCardinality{\isacharcolon}\ \isanewline
\ \ \isakeyword{assumes}\ {\isachardoublequoteopen}Z\ {\isasymsubseteq}\ X\ {\isasymunion}\ Y{\isachardoublequoteclose}\ {\isachardoublequoteopen}finite\ Z{\isachardoublequoteclose}\ \isanewline
\ \ \isakeyword{shows}\ {\isachardoublequoteopen}{\isacharparenleft}setsum\ {\isacharparenleft}X\ {\isacharless}{\isacharbar}\ Y{\isacharparenright}\ X{\isacharparenright}\ {\isacharminus}\ {\isacharparenleft}setsum\ {\isacharparenleft}X\ {\isacharless}{\isacharbar}\ Y{\isacharparenright}\ Z{\isacharparenright}\ {\isacharequal}\ card\ X\ {\isacharminus}\ card\ {\isacharparenleft}Z\ {\isasyminter}\ X{\isacharparenright}{\isachardoublequoteclose}\ \isanewline
%
\isadelimproof
\ \ %
\endisadelimproof
%
\isatagproof
\isacommand{using}\isamarkupfalse%
\ assms\ lm{\isadigit{1}}{\isadigit{1}}{\isadigit{2}}\ \isacommand{by}\isamarkupfalse%
\ {\isacharparenleft}metis\ Int{\isacharunderscore}absorb{\isadigit{2}}\ Un{\isacharunderscore}upper{\isadigit{1}}\ card{\isacharunderscore}infinite\ equalityE\ setsum{\isachardot}infinite{\isacharparenright}%
\endisatagproof
{\isafoldproof}%
%
\isadelimproof
\isanewline
%
\endisadelimproof
\isanewline
\isacommand{corollary}\isamarkupfalse%
\ differenceSetsumVsCardinality{\isacharcolon}\ \isanewline
\ \ \isakeyword{assumes}\ {\isachardoublequoteopen}Z\ {\isasymsubseteq}\ X\ {\isasymunion}\ Y{\isachardoublequoteclose}\ {\isachardoublequoteopen}finite\ Z{\isachardoublequoteclose}\ \isanewline
\ \ \isakeyword{shows}\ {\isachardoublequoteopen}int\ {\isacharparenleft}setsum\ {\isacharparenleft}X\ {\isacharless}{\isacharbar}\ Y{\isacharparenright}\ X{\isacharparenright}\ {\isacharminus}\ int\ {\isacharparenleft}setsum\ {\isacharparenleft}X\ {\isacharless}{\isacharbar}\ Y{\isacharparenright}\ Z{\isacharparenright}\ {\isacharequal}\ \ int\ {\isacharparenleft}card\ X{\isacharparenright}\ {\isacharminus}\ int\ {\isacharparenleft}card\ {\isacharparenleft}Z\ {\isasyminter}\ X{\isacharparenright}{\isacharparenright}{\isachardoublequoteclose}\ \isanewline
%
\isadelimproof
\ \ %
\endisadelimproof
%
\isatagproof
\isacommand{using}\isamarkupfalse%
\ assms\ lm{\isadigit{1}}{\isadigit{1}}{\isadigit{2}}\ \isacommand{by}\isamarkupfalse%
\ {\isacharparenleft}metis\ Int{\isacharunderscore}absorb{\isadigit{2}}\ Un{\isacharunderscore}upper{\isadigit{1}}\ card{\isacharunderscore}infinite\ equalityE\ setsum{\isachardot}infinite{\isacharparenright}%
\endisatagproof
{\isafoldproof}%
%
\isadelimproof
\isanewline
%
\endisadelimproof
\isanewline
\isanewline
\isacommand{lemma}\isamarkupfalse%
\ lm{\isadigit{1}}{\isadigit{1}}{\isadigit{3}}{\isacharcolon}\ \isanewline
\ \ {\isachardoublequoteopen}int\ {\isacharparenleft}n{\isacharcolon}{\isacharcolon}nat{\isacharparenright}\ {\isacharequal}\ real\ n{\isachardoublequoteclose}\ \isanewline
%
\isadelimproof
\ \ %
\endisadelimproof
%
\isatagproof
\isacommand{by}\isamarkupfalse%
\ simp%
\endisatagproof
{\isafoldproof}%
%
\isadelimproof
\isanewline
%
\endisadelimproof
\isanewline
\isanewline
\isacommand{corollary}\isamarkupfalse%
\ differenceSetsumVsCardinalityReal{\isacharcolon}\ \isanewline
\ \ \isakeyword{assumes}\ {\isachardoublequoteopen}Z{\isasymsubseteq}X{\isasymunion}Y{\isachardoublequoteclose}\ {\isachardoublequoteopen}finite\ Z{\isachardoublequoteclose}\ \isanewline
\ \ \isakeyword{shows}\ {\isachardoublequoteopen}real\ {\isacharparenleft}setsum\ {\isacharparenleft}X\ {\isacharless}{\isacharbar}\ Y{\isacharparenright}\ X{\isacharparenright}\ {\isacharminus}\ real\ {\isacharparenleft}setsum\ {\isacharparenleft}X\ {\isacharless}{\isacharbar}\ Y{\isacharparenright}\ Z{\isacharparenright}\ {\isacharequal}\ \isanewline
\ \ \ \ \ \ \ \ \ real\ {\isacharparenleft}card\ X{\isacharparenright}\ {\isacharminus}\ real\ {\isacharparenleft}card\ {\isacharparenleft}Z\ {\isasyminter}\ X{\isacharparenright}{\isacharparenright}{\isachardoublequoteclose}\ \isanewline
%
\isadelimproof
\ \ %
\endisadelimproof
%
\isatagproof
\isacommand{using}\isamarkupfalse%
\ assms\ lm{\isadigit{1}}{\isadigit{1}}{\isadigit{2}}\ \isacommand{by}\isamarkupfalse%
\ {\isacharparenleft}metis\ Int{\isacharunderscore}absorb{\isadigit{2}}\ Un{\isacharunderscore}upper{\isadigit{1}}\ card{\isacharunderscore}infinite\ equalityE\ setsum{\isachardot}infinite{\isacharparenright}%
\endisatagproof
{\isafoldproof}%
%
\isadelimproof
%
\endisadelimproof
%
\isamarkupsection{Lists%
}
\isamarkuptrue%
\isacommand{lemma}\isamarkupfalse%
\ lm{\isadigit{1}}{\isadigit{1}}{\isadigit{4}}{\isacharcolon}\ \isanewline
\ \ \isakeyword{assumes}\ {\isachardoublequoteopen}{\isasymexists}\ n\ {\isasymin}\ {\isacharbraceleft}{\isadigit{0}}{\isachardot}{\isachardot}{\isacharless}size\ l{\isacharbraceright}{\isachardot}\ P\ {\isacharparenleft}l{\isacharbang}n{\isacharparenright}{\isachardoublequoteclose}\ \isanewline
\ \ \isakeyword{shows}\ {\isachardoublequoteopen}{\isacharbrackleft}n{\isachardot}\ n\ {\isasymleftarrow}\ {\isacharbrackleft}{\isadigit{0}}{\isachardot}{\isachardot}{\isacharless}size\ l{\isacharbrackright}{\isacharcomma}\ P\ {\isacharparenleft}l{\isacharbang}n{\isacharparenright}{\isacharbrackright}\ {\isasymnoteq}\ {\isacharbrackleft}{\isacharbrackright}{\isachardoublequoteclose}\isanewline
%
\isadelimproof
\ \ %
\endisadelimproof
%
\isatagproof
\isacommand{using}\isamarkupfalse%
\ assms\ \isacommand{by}\isamarkupfalse%
\ auto%
\endisatagproof
{\isafoldproof}%
%
\isadelimproof
\isanewline
%
\endisadelimproof
\isanewline
\isanewline
\isacommand{lemma}\isamarkupfalse%
\ lm{\isadigit{1}}{\isadigit{1}}{\isadigit{5}}{\isacharcolon}\ \isanewline
\ \ \isakeyword{assumes}\ {\isachardoublequoteopen}ll\ {\isasymin}\ set\ {\isacharparenleft}l{\isacharcolon}{\isacharcolon}{\isacharprime}a\ list{\isacharparenright}{\isachardoublequoteclose}\ \isanewline
\ \ \isakeyword{shows}\ {\isachardoublequoteopen}{\isasymexists}\ n{\isasymin}\ {\isacharparenleft}nth\ l{\isacharparenright}\ {\isacharminus}{\isacharbackquote}\ {\isacharparenleft}set\ l{\isacharparenright}{\isachardot}\ ll{\isacharequal}l{\isacharbang}n{\isachardoublequoteclose}\isanewline
%
\isadelimproof
\ \ %
\endisadelimproof
%
\isatagproof
\isacommand{using}\isamarkupfalse%
\ assms{\isacharparenleft}{\isadigit{1}}{\isacharparenright}\ \isacommand{by}\isamarkupfalse%
\ {\isacharparenleft}metis\ in{\isacharunderscore}set{\isacharunderscore}conv{\isacharunderscore}nth\ vimageI{\isadigit{2}}{\isacharparenright}%
\endisatagproof
{\isafoldproof}%
%
\isadelimproof
\isanewline
%
\endisadelimproof
\isanewline
\isanewline
\isacommand{lemma}\isamarkupfalse%
\ lm{\isadigit{1}}{\isadigit{1}}{\isadigit{6}}{\isacharcolon}\ \isanewline
\ \ \isakeyword{assumes}\ {\isachardoublequoteopen}ll\ {\isasymin}\ set\ {\isacharparenleft}l{\isacharcolon}{\isacharcolon}{\isacharprime}a\ list{\isacharparenright}{\isachardoublequoteclose}\ \isanewline
\ \ \isakeyword{shows}\ {\isachardoublequoteopen}{\isasymexists}\ n{\isachardot}\ ll{\isacharequal}l{\isacharbang}n\ {\isacharampersand}\ n\ {\isacharless}\ size\ l\ {\isacharampersand}\ n\ {\isachargreater}{\isacharequal}\ {\isadigit{0}}{\isachardoublequoteclose}\isanewline
%
\isadelimproof
\ \ %
\endisadelimproof
%
\isatagproof
\isacommand{using}\isamarkupfalse%
\ assms\ in{\isacharunderscore}set{\isacharunderscore}conv{\isacharunderscore}nth\ \isacommand{by}\isamarkupfalse%
\ {\isacharparenleft}metis\ le{\isadigit{0}}{\isacharparenright}%
\endisatagproof
{\isafoldproof}%
%
\isadelimproof
\isanewline
%
\endisadelimproof
\isanewline
\isanewline
\isacommand{lemma}\isamarkupfalse%
\ lm{\isadigit{1}}{\isadigit{1}}{\isadigit{7}}{\isacharcolon}\ \isanewline
\ \ \isakeyword{assumes}\ {\isachardoublequoteopen}P\ {\isacharminus}{\isacharbackquote}\ {\isacharbraceleft}True{\isacharbraceright}\ {\isasyminter}\ set\ l\ {\isasymnoteq}\ {\isacharbraceleft}{\isacharbraceright}{\isachardoublequoteclose}\ \isanewline
\ \ \isakeyword{shows}\ {\isachardoublequoteopen}{\isasymexists}\ n\ {\isasymin}\ {\isacharbraceleft}{\isadigit{0}}{\isachardot}{\isachardot}{\isacharless}size\ l{\isacharbraceright}{\isachardot}\ P\ {\isacharparenleft}l{\isacharbang}n{\isacharparenright}{\isachardoublequoteclose}\ \isanewline
%
\isadelimproof
\ \ %
\endisadelimproof
%
\isatagproof
\isacommand{using}\isamarkupfalse%
\ assms\ lm{\isadigit{1}}{\isadigit{1}}{\isadigit{6}}\ \isacommand{by}\isamarkupfalse%
\ fastforce%
\endisatagproof
{\isafoldproof}%
%
\isadelimproof
\isanewline
%
\endisadelimproof
\isanewline
\isanewline
\isacommand{lemma}\isamarkupfalse%
\ nonEmptyListFiltered{\isacharcolon}\ \isanewline
\ \ \isakeyword{assumes}\ {\isachardoublequoteopen}P\ {\isacharminus}{\isacharbackquote}\ {\isacharbraceleft}True{\isacharbraceright}\ {\isasyminter}\ set\ l\ {\isasymnoteq}\ {\isacharbraceleft}{\isacharbraceright}{\isachardoublequoteclose}\ \isanewline
\ \ \isakeyword{shows}\ {\isachardoublequoteopen}{\isacharbrackleft}n{\isachardot}\ n\ {\isasymleftarrow}\ {\isacharbrackleft}{\isadigit{0}}{\isachardot}{\isachardot}{\isacharless}size\ l{\isacharbrackright}{\isacharcomma}\ P\ {\isacharparenleft}l{\isacharbang}n{\isacharparenright}{\isacharbrackright}\ {\isasymnoteq}\ {\isacharbrackleft}{\isacharbrackright}{\isachardoublequoteclose}\ \isanewline
%
\isadelimproof
\ \ %
\endisadelimproof
%
\isatagproof
\isacommand{using}\isamarkupfalse%
\ assms\ filterpositions{\isadigit{2}}{\isacharunderscore}def\ lm{\isadigit{1}}{\isadigit{1}}{\isadigit{7}}\ lm{\isadigit{1}}{\isadigit{1}}{\isadigit{4}}\ \isacommand{by}\isamarkupfalse%
\ metis%
\endisatagproof
{\isafoldproof}%
%
\isadelimproof
\isanewline
%
\endisadelimproof
\isanewline
\isanewline
\isacommand{lemma}\isamarkupfalse%
\ lm{\isadigit{1}}{\isadigit{1}}{\isadigit{8}}{\isacharcolon}\ \isanewline
\ \ {\isachardoublequoteopen}{\isacharparenleft}nth\ l{\isacharparenright}\ {\isacharbackquote}\ set\ {\isacharparenleft}{\isacharbrackleft}n{\isachardot}\ n\ {\isasymleftarrow}\ {\isacharbrackleft}{\isadigit{0}}{\isachardot}{\isachardot}{\isacharless}size\ l{\isacharbrackright}{\isacharcomma}\ {\isacharparenleft}{\isacharpercent}x{\isachardot}\ x{\isasymin}X{\isacharparenright}\ {\isacharparenleft}l{\isacharbang}n{\isacharparenright}{\isacharbrackright}{\isacharparenright}\ {\isasymsubseteq}\ X{\isasyminter}set\ l{\isachardoublequoteclose}\ \isanewline
%
\isadelimproof
\ \ %
\endisadelimproof
%
\isatagproof
\isacommand{by}\isamarkupfalse%
\ force%
\endisatagproof
{\isafoldproof}%
%
\isadelimproof
\isanewline
%
\endisadelimproof
\isanewline
\isanewline
\isacommand{corollary}\isamarkupfalse%
\ lm{\isadigit{1}}{\isadigit{1}}{\isadigit{9}}{\isacharcolon}\ \isanewline
\ \ {\isachardoublequoteopen}{\isacharparenleft}nth\ l{\isacharparenright}{\isacharbackquote}\ set\ {\isacharparenleft}filterpositions{\isadigit{2}}\ {\isacharparenleft}{\isacharpercent}x{\isachardot}{\isacharparenleft}x{\isasymin}X{\isacharparenright}{\isacharparenright}\ l{\isacharparenright}\ {\isasymsubseteq}\ X\ {\isasyminter}\ \ set\ l{\isachardoublequoteclose}\ \isanewline
%
\isadelimproof
\ \ %
\endisadelimproof
%
\isatagproof
\isacommand{unfolding}\isamarkupfalse%
\ filterpositions{\isadigit{2}}{\isacharunderscore}def\ \isacommand{using}\isamarkupfalse%
\ lm{\isadigit{1}}{\isadigit{1}}{\isadigit{8}}\ \isacommand{by}\isamarkupfalse%
\ fast%
\endisatagproof
{\isafoldproof}%
%
\isadelimproof
\isanewline
%
\endisadelimproof
\isanewline
\isacommand{lemma}\isamarkupfalse%
\ lm{\isadigit{1}}{\isadigit{2}}{\isadigit{0}}{\isacharcolon}\ \isanewline
\ \ {\isachardoublequoteopen}{\isacharparenleft}n{\isasymin}{\isacharbraceleft}{\isadigit{0}}{\isachardot}{\isachardot}{\isacharless}N{\isacharbraceright}{\isacharparenright}\ {\isacharequal}\ {\isacharparenleft}{\isacharparenleft}n{\isacharcolon}{\isacharcolon}nat{\isacharparenright}\ {\isacharless}\ N{\isacharparenright}{\isachardoublequoteclose}\ \isanewline
%
\isadelimproof
\ \ %
\endisadelimproof
%
\isatagproof
\isacommand{using}\isamarkupfalse%
\ atLeast{\isadigit{0}}LessThan\ lessThan{\isacharunderscore}iff\ \isacommand{by}\isamarkupfalse%
\ metis%
\endisatagproof
{\isafoldproof}%
%
\isadelimproof
\isanewline
%
\endisadelimproof
\isanewline
\isanewline
\isacommand{lemma}\isamarkupfalse%
\ lm{\isadigit{1}}{\isadigit{2}}{\isadigit{1}}{\isacharcolon}\ \isanewline
\ \ \isakeyword{assumes}\ {\isachardoublequoteopen}X\ {\isasymsubseteq}\ {\isacharbraceleft}{\isadigit{0}}{\isachardot}{\isachardot}{\isacharless}size\ list{\isacharbraceright}{\isachardoublequoteclose}\ \isanewline
\ \ \isakeyword{shows}\ {\isachardoublequoteopen}{\isacharparenleft}nth\ list{\isacharparenright}{\isacharbackquote}X\ {\isasymsubseteq}\ set\ list{\isachardoublequoteclose}\ \isanewline
%
\isadelimproof
\ \ %
\endisadelimproof
%
\isatagproof
\isacommand{using}\isamarkupfalse%
\ assms\ atLeastLessThan{\isacharunderscore}def\ atLeast{\isadigit{0}}LessThan\ lessThan{\isacharunderscore}iff\ \isacommand{by}\isamarkupfalse%
\ auto%
\endisatagproof
{\isafoldproof}%
%
\isadelimproof
\isanewline
%
\endisadelimproof
\isanewline
\isanewline
\isacommand{lemma}\isamarkupfalse%
\ lm{\isadigit{1}}{\isadigit{2}}{\isadigit{2}}{\isacharcolon}\ \isanewline
\ \ {\isachardoublequoteopen}set\ {\isacharparenleft}{\isacharbrackleft}n{\isachardot}\ n\ {\isasymleftarrow}\ {\isacharbrackleft}{\isadigit{0}}{\isachardot}{\isachardot}{\isacharless}size\ l{\isacharbrackright}{\isacharcomma}\ P\ {\isacharparenleft}l{\isacharbang}n{\isacharparenright}{\isacharbrackright}{\isacharparenright}\ {\isasymsubseteq}\ {\isacharbraceleft}{\isadigit{0}}{\isachardot}{\isachardot}{\isacharless}size\ l{\isacharbraceright}{\isachardoublequoteclose}\ \isanewline
%
\isadelimproof
\ \ %
\endisadelimproof
%
\isatagproof
\isacommand{by}\isamarkupfalse%
\ force%
\endisatagproof
{\isafoldproof}%
%
\isadelimproof
\isanewline
%
\endisadelimproof
\isanewline
\isanewline
\isacommand{lemma}\isamarkupfalse%
\ lm{\isadigit{1}}{\isadigit{2}}{\isadigit{3}}{\isacharcolon}\ \isanewline
\ \ {\isachardoublequoteopen}set\ {\isacharparenleft}filterpositions{\isadigit{2}}\ pre\ list{\isacharparenright}\ {\isasymsubseteq}\ {\isacharbraceleft}{\isadigit{0}}{\isachardot}{\isachardot}{\isacharless}size\ list{\isacharbraceright}{\isachardoublequoteclose}\ \isanewline
%
\isadelimproof
\ \ %
\endisadelimproof
%
\isatagproof
\isacommand{using}\isamarkupfalse%
\ filterpositions{\isadigit{2}}{\isacharunderscore}def\ lm{\isadigit{1}}{\isadigit{2}}{\isadigit{2}}\ \isacommand{by}\isamarkupfalse%
\ metis%
\endisatagproof
{\isafoldproof}%
%
\isadelimproof
%
\endisadelimproof
%
\isamarkupsubsection{Computing all the permutations of a list%
}
\isamarkuptrue%
\isacommand{abbreviation}\isamarkupfalse%
\ \isanewline
\ \ {\isachardoublequoteopen}rotateLeft\ {\isacharequal}{\isacharequal}\ rotate{\isachardoublequoteclose}\isanewline
\isacommand{abbreviation}\isamarkupfalse%
\ \isanewline
\ \ {\isachardoublequoteopen}rotateRight\ n\ l\ {\isacharequal}{\isacharequal}\ rotateLeft\ {\isacharparenleft}size\ l\ {\isacharminus}\ {\isacharparenleft}n\ mod\ {\isacharparenleft}size\ l{\isacharparenright}{\isacharparenright}{\isacharparenright}\ l{\isachardoublequoteclose}\isanewline
\isanewline
\isanewline
\isanewline
\isacommand{abbreviation}\isamarkupfalse%
\ \isanewline
\ \ {\isachardoublequoteopen}insertAt\ x\ l\ n\ {\isacharequal}{\isacharequal}\ rotateRight\ n\ {\isacharparenleft}x{\isacharhash}{\isacharparenleft}rotateLeft\ n\ l{\isacharparenright}{\isacharparenright}{\isachardoublequoteclose}\isanewline
\isanewline
\isanewline
\isacommand{fun}\isamarkupfalse%
\ perm{\isadigit{2}}\ \isakeyword{where}\isanewline
\ \ {\isachardoublequoteopen}perm{\isadigit{2}}\ {\isacharbrackleft}{\isacharbrackright}\ {\isacharequal}\ {\isacharparenleft}{\isacharpercent}n{\isachardot}\ {\isacharbrackleft}{\isacharbrackright}{\isacharparenright}{\isachardoublequoteclose}\ {\isacharbar}\ \isanewline
\ \ {\isachardoublequoteopen}perm{\isadigit{2}}\ {\isacharparenleft}x{\isacharhash}l{\isacharparenright}\ {\isacharequal}\ {\isacharparenleft}{\isacharpercent}n{\isachardot}\ insertAt\ x\ {\isacharparenleft}{\isacharparenleft}perm{\isadigit{2}}\ l{\isacharparenright}\ {\isacharparenleft}n\ div\ {\isacharparenleft}{\isadigit{1}}{\isacharplus}size\ l{\isacharparenright}{\isacharparenright}{\isacharparenright}\isanewline
\ \ \ \ \ \ \ \ \ \ \ \ \ \ \ \ \ \ \ \ \ \ {\isacharparenleft}n\ mod\ {\isacharparenleft}{\isadigit{1}}{\isacharplus}size\ l{\isacharparenright}{\isacharparenright}{\isacharparenright}{\isachardoublequoteclose}\isanewline
\isanewline
\isacommand{abbreviation}\isamarkupfalse%
\ \isanewline
\ \ {\isachardoublequoteopen}takeAll\ P\ list\ {\isacharequal}{\isacharequal}\ map\ {\isacharparenleft}nth\ list{\isacharparenright}\ {\isacharparenleft}filterpositions{\isadigit{2}}\ P\ list{\isacharparenright}{\isachardoublequoteclose}\isanewline
\isanewline
\isacommand{lemma}\isamarkupfalse%
\ permutationNotEmpty{\isacharcolon}\ \isanewline
\ \ \isakeyword{assumes}\ {\isachardoublequoteopen}l\ {\isasymnoteq}\ {\isacharbrackleft}{\isacharbrackright}{\isachardoublequoteclose}\ \isanewline
\ \ \isakeyword{shows}\ {\isachardoublequoteopen}perm{\isadigit{2}}\ l\ n\ {\isasymnoteq}\ {\isacharbrackleft}{\isacharbrackright}{\isachardoublequoteclose}\ \isanewline
%
\isadelimproof
\ \ %
\endisadelimproof
%
\isatagproof
\isacommand{using}\isamarkupfalse%
\ assms\ perm{\isadigit{2}}{\isacharunderscore}def\ perm{\isadigit{2}}{\isachardot}simps{\isacharparenleft}{\isadigit{2}}{\isacharparenright}\ rotate{\isacharunderscore}is{\isacharunderscore}Nil{\isacharunderscore}conv\ \isacommand{by}\isamarkupfalse%
\ {\isacharparenleft}metis\ neq{\isacharunderscore}Nil{\isacharunderscore}conv{\isacharparenright}%
\endisatagproof
{\isafoldproof}%
%
\isadelimproof
\isanewline
%
\endisadelimproof
\isanewline
\isacommand{lemma}\isamarkupfalse%
\ lm{\isadigit{1}}{\isadigit{2}}{\isadigit{4}}{\isacharcolon}\ \isanewline
\ \ {\isachardoublequoteopen}set\ {\isacharparenleft}takeAll\ P\ list{\isacharparenright}\ {\isacharequal}\ {\isacharparenleft}{\isacharparenleft}nth\ list{\isacharparenright}\ {\isacharbackquote}\ set\ {\isacharparenleft}filterpositions{\isadigit{2}}\ P\ list{\isacharparenright}{\isacharparenright}{\isachardoublequoteclose}\ \isanewline
%
\isadelimproof
\ \ %
\endisadelimproof
%
\isatagproof
\isacommand{by}\isamarkupfalse%
\ simp%
\endisatagproof
{\isafoldproof}%
%
\isadelimproof
\isanewline
%
\endisadelimproof
\isanewline
\isacommand{corollary}\isamarkupfalse%
\ listIntersectionWithSet{\isacharcolon}\ \isanewline
\ \ {\isachardoublequoteopen}set\ {\isacharparenleft}takeAll\ {\isacharparenleft}{\isacharpercent}x{\isachardot}{\isacharparenleft}x{\isasymin}X{\isacharparenright}{\isacharparenright}\ l{\isacharparenright}\ {\isasymsubseteq}\ \ {\isacharparenleft}X\ {\isasyminter}\ set\ l{\isacharparenright}{\isachardoublequoteclose}\ \isanewline
%
\isadelimproof
\ \ %
\endisadelimproof
%
\isatagproof
\isacommand{using}\isamarkupfalse%
\ lm{\isadigit{1}}{\isadigit{1}}{\isadigit{9}}\ lm{\isadigit{1}}{\isadigit{2}}{\isadigit{4}}\ \isacommand{by}\isamarkupfalse%
\ metis%
\endisatagproof
{\isafoldproof}%
%
\isadelimproof
\isanewline
%
\endisadelimproof
\isanewline
\isacommand{corollary}\isamarkupfalse%
\ lm{\isadigit{1}}{\isadigit{2}}{\isadigit{5}}{\isacharcolon}\ \isanewline
\ \ {\isachardoublequoteopen}set\ {\isacharparenleft}takeAll\ P\ list{\isacharparenright}\ {\isasymsubseteq}\ set\ list{\isachardoublequoteclose}\ \isanewline
%
\isadelimproof
\ \ %
\endisadelimproof
%
\isatagproof
\isacommand{using}\isamarkupfalse%
\ lm{\isadigit{1}}{\isadigit{2}}{\isadigit{3}}\ lm{\isadigit{1}}{\isadigit{2}}{\isadigit{4}}\ lm{\isadigit{1}}{\isadigit{2}}{\isadigit{1}}\ \isacommand{by}\isamarkupfalse%
\ metis%
\endisatagproof
{\isafoldproof}%
%
\isadelimproof
\isanewline
%
\endisadelimproof
\isanewline
\isacommand{lemma}\isamarkupfalse%
\ lm{\isadigit{1}}{\isadigit{2}}{\isadigit{6}}{\isacharcolon}\ \isanewline
\ \ {\isachardoublequoteopen}set\ {\isacharparenleft}insertAt\ x\ l\ n{\isacharparenright}\ {\isacharequal}\ {\isacharbraceleft}x{\isacharbraceright}\ {\isasymunion}\ set\ l{\isachardoublequoteclose}\ \isanewline
%
\isadelimproof
\ \ %
\endisadelimproof
%
\isatagproof
\isacommand{by}\isamarkupfalse%
\ simp%
\endisatagproof
{\isafoldproof}%
%
\isadelimproof
\isanewline
%
\endisadelimproof
\isanewline
\isacommand{lemma}\isamarkupfalse%
\ lm{\isadigit{1}}{\isadigit{2}}{\isadigit{7}}{\isacharcolon}\ \isanewline
\ \ {\isachardoublequoteopen}{\isasymforall}n{\isachardot}\ set\ {\isacharparenleft}perm{\isadigit{2}}\ {\isacharbrackleft}{\isacharbrackright}\ n{\isacharparenright}\ {\isacharequal}\ set\ {\isacharbrackleft}{\isacharbrackright}{\isachardoublequoteclose}\ \isanewline
%
\isadelimproof
\ \ %
\endisadelimproof
%
\isatagproof
\isacommand{by}\isamarkupfalse%
\ simp%
\endisatagproof
{\isafoldproof}%
%
\isadelimproof
\isanewline
%
\endisadelimproof
\isanewline
\isacommand{lemma}\isamarkupfalse%
\ lm{\isadigit{1}}{\isadigit{2}}{\isadigit{8}}{\isacharcolon}\ \isanewline
\ \ \isakeyword{assumes}\ {\isachardoublequoteopen}{\isasymforall}n{\isachardot}\ {\isacharparenleft}set\ {\isacharparenleft}perm{\isadigit{2}}\ l\ n{\isacharparenright}\ {\isacharequal}\ set\ l{\isacharparenright}{\isachardoublequoteclose}\ \isanewline
\ \ \isakeyword{shows}\ {\isachardoublequoteopen}set\ {\isacharparenleft}perm{\isadigit{2}}\ {\isacharparenleft}x{\isacharhash}l{\isacharparenright}\ n{\isacharparenright}\ {\isacharequal}\ {\isacharbraceleft}x{\isacharbraceright}\ {\isasymunion}\ set\ l{\isachardoublequoteclose}\ \isanewline
%
\isadelimproof
\ \ %
\endisadelimproof
%
\isatagproof
\isacommand{using}\isamarkupfalse%
\ assms\ perm{\isadigit{2}}{\isacharunderscore}def\ lm{\isadigit{1}}{\isadigit{2}}{\isadigit{6}}\ \isacommand{by}\isamarkupfalse%
\ force%
\endisatagproof
{\isafoldproof}%
%
\isadelimproof
\isanewline
%
\endisadelimproof
\isanewline
\ \isanewline
\isacommand{corollary}\isamarkupfalse%
\ permutationInvariance{\isacharcolon}\ \isanewline
\ \ \ {\isachardoublequoteopen}{\isasymforall}n{\isachardot}\ set\ {\isacharparenleft}perm{\isadigit{2}}\ {\isacharparenleft}l{\isacharcolon}{\isacharcolon}{\isacharprime}a\ list{\isacharparenright}\ n{\isacharparenright}\ {\isacharequal}\ set\ l{\isachardoublequoteclose}\ \isanewline
%
\isadelimproof
%
\endisadelimproof
%
\isatagproof
\isacommand{proof}\isamarkupfalse%
\ {\isacharparenleft}induct\ l{\isacharparenright}\isanewline
\ \ \ \isacommand{let}\isamarkupfalse%
\ {\isacharquery}P\ {\isacharequal}\ {\isachardoublequoteopen}{\isacharpercent}l{\isacharcolon}{\isacharcolon}{\isacharparenleft}{\isacharprime}a\ list{\isacharparenright}{\isachardot}\ {\isacharparenleft}{\isasymforall}n{\isachardot}\ set\ {\isacharparenleft}perm{\isadigit{2}}\ l\ n{\isacharparenright}\ \ {\isacharequal}\ \ set\ l{\isacharparenright}{\isachardoublequoteclose}\isanewline
\ \ \ \isacommand{show}\isamarkupfalse%
\ {\isachardoublequoteopen}{\isacharquery}P\ {\isacharbrackleft}{\isacharbrackright}{\isachardoublequoteclose}\ \isacommand{using}\isamarkupfalse%
\ lm{\isadigit{1}}{\isadigit{2}}{\isadigit{7}}\ \isacommand{by}\isamarkupfalse%
\ force\ \isanewline
\ \ \ \isacommand{fix}\isamarkupfalse%
\ x\ \isacommand{fix}\isamarkupfalse%
\ l\ \isanewline
\ \ \ \isacommand{assume}\isamarkupfalse%
\ {\isachardoublequoteopen}{\isacharquery}P\ l{\isachardoublequoteclose}\ \isacommand{then}\isamarkupfalse%
\ \isanewline
\ \ \ \isacommand{show}\isamarkupfalse%
\ {\isachardoublequoteopen}{\isacharquery}P\ {\isacharparenleft}x{\isacharhash}l{\isacharparenright}{\isachardoublequoteclose}\ \isacommand{by}\isamarkupfalse%
\ force\isanewline
\isacommand{qed}\isamarkupfalse%
%
\endisatagproof
{\isafoldproof}%
%
\isadelimproof
\isanewline
%
\endisadelimproof
\isanewline
\isanewline
\isacommand{corollary}\isamarkupfalse%
\ takeAllPermutation{\isacharcolon}\ \isanewline
\ \ {\isachardoublequoteopen}set\ {\isacharparenleft}perm{\isadigit{2}}\ {\isacharparenleft}takeAll\ {\isacharparenleft}{\isacharpercent}x{\isachardot}{\isacharparenleft}x{\isasymin}X{\isacharparenright}{\isacharparenright}\ l{\isacharparenright}\ n{\isacharparenright}\ \ {\isasymsubseteq}\ \ X\ {\isasyminter}\ set\ l{\isachardoublequoteclose}\ \isanewline
%
\isadelimproof
\ \ %
\endisadelimproof
%
\isatagproof
\isacommand{using}\isamarkupfalse%
\ listIntersectionWithSet\ permutationInvariance\ \isacommand{by}\isamarkupfalse%
\ metis%
\endisatagproof
{\isafoldproof}%
%
\isadelimproof
%
\endisadelimproof
%
\isamarkupsection{A more computable version of \isa{toFunction}.%
}
\isamarkuptrue%
\isacommand{abbreviation}\isamarkupfalse%
\ {\isachardoublequoteopen}toFunctionWithFallback\ R\ fallback\ {\isacharequal}{\isacharequal}\ {\isacharparenleft}{\isacharpercent}\ x{\isachardot}\ if\ {\isacharparenleft}R{\isacharbackquote}{\isacharbackquote}{\isacharbraceleft}x{\isacharbraceright}\ {\isacharequal}\ {\isacharbraceleft}R{\isacharcomma}{\isacharcomma}x{\isacharbraceright}{\isacharparenright}\ then\ {\isacharparenleft}R{\isacharcomma}{\isacharcomma}x{\isacharparenright}\ else\ fallback{\isacharparenright}{\isachardoublequoteclose}\isanewline
\isacommand{notation}\isamarkupfalse%
\ \isanewline
\ \ toFunctionWithFallback\ {\isacharparenleft}\isakeyword{infix}\ {\isachardoublequoteopen}Else{\isachardoublequoteclose}\ {\isadigit{7}}{\isadigit{5}}{\isacharparenright}\isanewline
\isanewline
\isacommand{abbreviation}\isamarkupfalse%
\ \isanewline
\ \ {\isachardoublequoteopen}setsum{\isacharprime}\ R\ X\ {\isacharequal}{\isacharequal}\ setsum\ {\isacharparenleft}R\ Else\ {\isadigit{0}}{\isacharparenright}\ X{\isachardoublequoteclose}\isanewline
\isanewline
\isacommand{lemma}\isamarkupfalse%
\ lm{\isadigit{1}}{\isadigit{2}}{\isadigit{9}}{\isacharcolon}\ \isanewline
\ \ \isakeyword{assumes}\ {\isachardoublequoteopen}runiq\ f{\isachardoublequoteclose}\ {\isachardoublequoteopen}x\ {\isasymin}\ Domain\ f{\isachardoublequoteclose}\ \isanewline
\ \ \isakeyword{shows}\ {\isachardoublequoteopen}{\isacharparenleft}f\ Else\ {\isadigit{0}}{\isacharparenright}\ x\ {\isacharequal}\ {\isacharparenleft}toFunction\ f{\isacharparenright}\ x{\isachardoublequoteclose}\ \isanewline
%
\isadelimproof
\ \ %
\endisadelimproof
%
\isatagproof
\isacommand{using}\isamarkupfalse%
\ assms\ \isacommand{by}\isamarkupfalse%
\ {\isacharparenleft}metis\ Image{\isacharunderscore}runiq{\isacharunderscore}eq{\isacharunderscore}eval\ toFunction{\isacharunderscore}def{\isacharparenright}%
\endisatagproof
{\isafoldproof}%
%
\isadelimproof
\isanewline
%
\endisadelimproof
\isanewline
\isacommand{lemma}\isamarkupfalse%
\ lm{\isadigit{1}}{\isadigit{3}}{\isadigit{0}}{\isacharcolon}\ \isanewline
\ \ \isakeyword{assumes}\ {\isachardoublequoteopen}runiq\ f{\isachardoublequoteclose}\ \isanewline
\ \ \isakeyword{shows}\ {\isachardoublequoteopen}setsum\ {\isacharparenleft}f\ Else\ {\isadigit{0}}{\isacharparenright}\ {\isacharparenleft}X{\isasyminter}{\isacharparenleft}Domain\ f{\isacharparenright}{\isacharparenright}\ \ {\isacharequal}\ \ setsum\ {\isacharparenleft}toFunction\ f{\isacharparenright}\ {\isacharparenleft}X{\isasyminter}{\isacharparenleft}Domain\ f{\isacharparenright}{\isacharparenright}{\isachardoublequoteclose}\ \isanewline
%
\isadelimproof
\ \ %
\endisadelimproof
%
\isatagproof
\isacommand{using}\isamarkupfalse%
\ assms\ setsum{\isachardot}cong\ lm{\isadigit{1}}{\isadigit{2}}{\isadigit{9}}\ \isacommand{by}\isamarkupfalse%
\ fastforce%
\endisatagproof
{\isafoldproof}%
%
\isadelimproof
\isanewline
%
\endisadelimproof
\isanewline
\isacommand{lemma}\isamarkupfalse%
\ lm{\isadigit{1}}{\isadigit{3}}{\isadigit{1}}{\isacharcolon}\ \isanewline
\ \ \isakeyword{assumes}\ {\isachardoublequoteopen}Y\ {\isasymsubseteq}\ f{\isacharminus}{\isacharbackquote}{\isacharbraceleft}{\isadigit{0}}{\isacharbraceright}{\isachardoublequoteclose}\ \isanewline
\ \ \isakeyword{shows}\ {\isachardoublequoteopen}setsum\ f\ Y\ \ {\isacharequal}\ \ {\isadigit{0}}{\isachardoublequoteclose}\ \isanewline
%
\isadelimproof
\ \ %
\endisadelimproof
%
\isatagproof
\isacommand{using}\isamarkupfalse%
\ assms\ \isacommand{by}\isamarkupfalse%
\ {\isacharparenleft}metis\ set{\isacharunderscore}rev{\isacharunderscore}mp\ setsum{\isachardot}neutral\ vimage{\isacharunderscore}singleton{\isacharunderscore}eq{\isacharparenright}%
\endisatagproof
{\isafoldproof}%
%
\isadelimproof
\isanewline
%
\endisadelimproof
\isanewline
\isacommand{lemma}\isamarkupfalse%
\ lm{\isadigit{1}}{\isadigit{3}}{\isadigit{2}}{\isacharcolon}\ \isanewline
\ \ \isakeyword{assumes}\ {\isachardoublequoteopen}Y\ {\isasymsubseteq}\ f{\isacharminus}{\isacharbackquote}{\isacharbraceleft}{\isadigit{0}}{\isacharbraceright}{\isachardoublequoteclose}\ {\isachardoublequoteopen}finite\ X{\isachardoublequoteclose}\ \isanewline
\ \ \isakeyword{shows}\ {\isachardoublequoteopen}setsum\ f\ X\ {\isacharequal}\ setsum\ f\ {\isacharparenleft}X{\isacharminus}Y{\isacharparenright}{\isachardoublequoteclose}\ \isanewline
%
\isadelimproof
\ \ %
\endisadelimproof
%
\isatagproof
\isacommand{using}\isamarkupfalse%
\ assms\ Int{\isacharunderscore}lower{\isadigit{2}}\ comm{\isacharunderscore}monoid{\isacharunderscore}add{\isacharunderscore}class{\isachardot}add{\isachardot}right{\isacharunderscore}neutral\ inf{\isachardot}boundedE\ inf{\isachardot}orderE\ lm{\isadigit{0}}{\isadigit{7}}{\isadigit{8}}\ lm{\isadigit{1}}{\isadigit{3}}{\isadigit{1}}\isanewline
\ \ \isacommand{by}\isamarkupfalse%
\ {\isacharparenleft}metis{\isacharparenleft}no{\isacharunderscore}types{\isacharparenright}{\isacharparenright}%
\endisatagproof
{\isafoldproof}%
%
\isadelimproof
\isanewline
%
\endisadelimproof
\isanewline
\isanewline
\isacommand{lemma}\isamarkupfalse%
\ lm{\isadigit{1}}{\isadigit{3}}{\isadigit{3}}{\isacharcolon}\ \isanewline
\ \ {\isachardoublequoteopen}{\isacharminus}{\isacharparenleft}Domain\ f{\isacharparenright}\ {\isasymsubseteq}\ {\isacharparenleft}f\ Else\ {\isadigit{0}}{\isacharparenright}{\isacharminus}{\isacharbackquote}{\isacharbraceleft}{\isadigit{0}}{\isacharbraceright}{\isachardoublequoteclose}\ \isanewline
%
\isadelimproof
\ \ %
\endisadelimproof
%
\isatagproof
\isacommand{by}\isamarkupfalse%
\ fastforce%
\endisatagproof
{\isafoldproof}%
%
\isadelimproof
\isanewline
%
\endisadelimproof
\isanewline
\isacommand{corollary}\isamarkupfalse%
\ lm{\isadigit{1}}{\isadigit{3}}{\isadigit{4}}{\isacharcolon}\ \isanewline
\ \ \isakeyword{assumes}\ {\isachardoublequoteopen}finite\ X{\isachardoublequoteclose}\ \isanewline
\ \ \isakeyword{shows}\ {\isachardoublequoteopen}setsum\ {\isacharparenleft}f\ Else\ {\isadigit{0}}{\isacharparenright}\ X\ \ \ \ {\isacharequal}\ \ \ setsum\ {\isacharparenleft}f\ Else\ {\isadigit{0}}{\isacharparenright}\ {\isacharparenleft}X{\isasyminter}Domain\ f{\isacharparenright}{\isachardoublequoteclose}\ \isanewline
%
\isadelimproof
%
\endisadelimproof
%
\isatagproof
\isacommand{proof}\isamarkupfalse%
\ {\isacharminus}\ \isanewline
\ \ \isacommand{have}\isamarkupfalse%
\ {\isachardoublequoteopen}X{\isasyminter}Domain\ f{\isacharequal}X{\isacharminus}{\isacharparenleft}{\isacharminus}Domain\ f{\isacharparenright}{\isachardoublequoteclose}\ \isacommand{by}\isamarkupfalse%
\ simp\ \isanewline
\ \ \isacommand{thus}\isamarkupfalse%
\ {\isacharquery}thesis\ \isacommand{using}\isamarkupfalse%
\ assms\ lm{\isadigit{1}}{\isadigit{3}}{\isadigit{3}}\ lm{\isadigit{1}}{\isadigit{3}}{\isadigit{2}}\ \isacommand{by}\isamarkupfalse%
\ fastforce\isanewline
\isacommand{qed}\isamarkupfalse%
%
\endisatagproof
{\isafoldproof}%
%
\isadelimproof
\isanewline
%
\endisadelimproof
\isanewline
\isacommand{corollary}\isamarkupfalse%
\ lm{\isadigit{1}}{\isadigit{3}}{\isadigit{5}}{\isacharcolon}\ \isanewline
\ \ \isakeyword{assumes}\ {\isachardoublequoteopen}finite\ X{\isachardoublequoteclose}\ \isanewline
\ \ \isakeyword{shows}\ {\isachardoublequoteopen}setsum\ {\isacharparenleft}f\ Else\ {\isadigit{0}}{\isacharparenright}\ {\isacharparenleft}X{\isasyminter}Domain\ f{\isacharparenright}\ \ \ {\isacharequal}\ \ \ setsum\ {\isacharparenleft}f\ Else\ {\isadigit{0}}{\isacharparenright}\ X{\isachardoublequoteclose}\isanewline
\ \ {\isacharparenleft}\isakeyword{is}\ {\isachardoublequoteopen}{\isacharquery}L{\isacharequal}{\isacharquery}R{\isachardoublequoteclose}{\isacharparenright}\ \isanewline
%
\isadelimproof
%
\endisadelimproof
%
\isatagproof
\isacommand{proof}\isamarkupfalse%
\ {\isacharminus}\ \isanewline
\ \ \isacommand{have}\isamarkupfalse%
\ {\isachardoublequoteopen}{\isacharquery}R{\isacharequal}{\isacharquery}L{\isachardoublequoteclose}\ \isacommand{using}\isamarkupfalse%
\ assms\ \isacommand{by}\isamarkupfalse%
\ {\isacharparenleft}rule\ lm{\isadigit{1}}{\isadigit{3}}{\isadigit{4}}{\isacharparenright}\ \isanewline
\ \ \isacommand{thus}\isamarkupfalse%
\ {\isacharquery}thesis\ \isacommand{by}\isamarkupfalse%
\ simp\ \isanewline
\isacommand{qed}\isamarkupfalse%
%
\endisatagproof
{\isafoldproof}%
%
\isadelimproof
\isanewline
%
\endisadelimproof
\isanewline
\isacommand{corollary}\isamarkupfalse%
\ lm{\isadigit{1}}{\isadigit{3}}{\isadigit{6}}{\isacharcolon}\ \isanewline
\ \ \isakeyword{assumes}\ {\isachardoublequoteopen}finite\ X{\isachardoublequoteclose}\ {\isachardoublequoteopen}runiq\ f{\isachardoublequoteclose}\ \isanewline
\ \ \isakeyword{shows}\ {\isachardoublequoteopen}setsum\ {\isacharparenleft}f\ Else\ {\isadigit{0}}{\isacharparenright}\ X\ {\isacharequal}\ setsum\ {\isacharparenleft}toFunction\ f{\isacharparenright}\ {\isacharparenleft}X{\isasyminter}Domain\ f{\isacharparenright}{\isachardoublequoteclose}\ \isanewline
\ \ {\isacharparenleft}\isakeyword{is}\ {\isachardoublequoteopen}{\isacharquery}L{\isacharequal}{\isacharquery}R{\isachardoublequoteclose}{\isacharparenright}\ \isanewline
%
\isadelimproof
%
\endisadelimproof
%
\isatagproof
\isacommand{proof}\isamarkupfalse%
\ {\isacharminus}\isanewline
\ \ \isacommand{have}\isamarkupfalse%
\ {\isachardoublequoteopen}{\isacharquery}R\ {\isacharequal}\ setsum\ {\isacharparenleft}f\ Else\ {\isadigit{0}}{\isacharparenright}\ {\isacharparenleft}X{\isasyminter}Domain\ f{\isacharparenright}\ {\isachardoublequoteclose}\ \isacommand{using}\isamarkupfalse%
\ assms{\isacharparenleft}{\isadigit{2}}{\isacharparenright}\ lm{\isadigit{1}}{\isadigit{3}}{\isadigit{0}}\ \isacommand{by}\isamarkupfalse%
\ fastforce\isanewline
\ \ \isacommand{moreover}\isamarkupfalse%
\ \isacommand{have}\isamarkupfalse%
\ {\isachardoublequoteopen}{\isachardot}{\isachardot}{\isachardot}\ {\isacharequal}\ {\isacharquery}L{\isachardoublequoteclose}\ \isacommand{using}\isamarkupfalse%
\ assms{\isacharparenleft}{\isadigit{1}}{\isacharparenright}\ \isacommand{by}\isamarkupfalse%
\ {\isacharparenleft}rule\ lm{\isadigit{1}}{\isadigit{3}}{\isadigit{5}}{\isacharparenright}\ \isanewline
\ \ \isacommand{ultimately}\isamarkupfalse%
\ \isacommand{show}\isamarkupfalse%
\ {\isacharquery}thesis\ \isacommand{by}\isamarkupfalse%
\ presburger\isanewline
\isacommand{qed}\isamarkupfalse%
%
\endisatagproof
{\isafoldproof}%
%
\isadelimproof
\isanewline
%
\endisadelimproof
\isanewline
\isacommand{lemma}\isamarkupfalse%
\ lm{\isadigit{1}}{\isadigit{3}}{\isadigit{7}}{\isacharcolon}\ \isanewline
\ \ {\isachardoublequoteopen}setsum\ {\isacharparenleft}f\ Else\ {\isadigit{0}}{\isacharparenright}\ X\ {\isacharequal}\ setsum{\isacharprime}\ f\ X{\isachardoublequoteclose}\ \isanewline
%
\isadelimproof
\ \ %
\endisadelimproof
%
\isatagproof
\isacommand{by}\isamarkupfalse%
\ fast%
\endisatagproof
{\isafoldproof}%
%
\isadelimproof
\isanewline
%
\endisadelimproof
\isanewline
\isacommand{corollary}\isamarkupfalse%
\ lm{\isadigit{1}}{\isadigit{3}}{\isadigit{8}}{\isacharcolon}\ \isanewline
\ \ \isakeyword{assumes}\ {\isachardoublequoteopen}finite\ X{\isachardoublequoteclose}\ {\isachardoublequoteopen}runiq\ f{\isachardoublequoteclose}\ \isanewline
\ \ \isakeyword{shows}\ {\isachardoublequoteopen}setsum\ {\isacharparenleft}toFunction\ f{\isacharparenright}\ {\isacharparenleft}X{\isasyminter}Domain\ f{\isacharparenright}\ \ \ {\isacharequal}\ \ \ setsum{\isacharprime}\ f\ X{\isachardoublequoteclose}\isanewline
%
\isadelimproof
\ \ %
\endisadelimproof
%
\isatagproof
\isacommand{using}\isamarkupfalse%
\ assms\ lm{\isadigit{1}}{\isadigit{3}}{\isadigit{7}}\ lm{\isadigit{1}}{\isadigit{3}}{\isadigit{6}}\ \isacommand{by}\isamarkupfalse%
\ fastforce%
\endisatagproof
{\isafoldproof}%
%
\isadelimproof
\isanewline
%
\endisadelimproof
\isanewline
\isacommand{lemma}\isamarkupfalse%
\ lm{\isadigit{1}}{\isadigit{3}}{\isadigit{9}}{\isacharcolon}\ \isanewline
\ \ {\isachardoublequoteopen}argmax\ {\isacharparenleft}setsum{\isacharprime}\ b{\isacharparenright}\ {\isacharequal}\ {\isacharparenleft}argmax\ {\isasymcirc}\ setsum{\isacharprime}{\isacharparenright}\ b{\isachardoublequoteclose}\ \isanewline
%
\isadelimproof
\ \ %
\endisadelimproof
%
\isatagproof
\isacommand{by}\isamarkupfalse%
\ simp%
\endisatagproof
{\isafoldproof}%
%
\isadelimproof
%
\endisadelimproof
%
\isamarkupsection{cardinalities of sets.%
}
\isamarkuptrue%
\isacommand{lemma}\isamarkupfalse%
\ lm{\isadigit{1}}{\isadigit{4}}{\isadigit{0}}{\isacharcolon}\ \isanewline
\ \ \isakeyword{assumes}\ {\isachardoublequoteopen}runiq\ R{\isachardoublequoteclose}\ {\isachardoublequoteopen}runiq\ {\isacharparenleft}R{\isacharcircum}{\isacharminus}{\isadigit{1}}{\isacharparenright}{\isachardoublequoteclose}\ \isanewline
\ \ \isakeyword{shows}\ {\isachardoublequoteopen}{\isacharparenleft}R{\isacharbackquote}{\isacharbackquote}A{\isacharparenright}\ {\isasyminter}\ {\isacharparenleft}R{\isacharbackquote}{\isacharbackquote}B{\isacharparenright}\ {\isacharequal}\ R{\isacharbackquote}{\isacharbackquote}{\isacharparenleft}A{\isasyminter}B{\isacharparenright}{\isachardoublequoteclose}\ \isanewline
%
\isadelimproof
\ \ %
\endisadelimproof
%
\isatagproof
\isacommand{using}\isamarkupfalse%
\ assms\ rightUniqueInjectiveOnFirst\ converse{\isacharunderscore}Image\ \isacommand{by}\isamarkupfalse%
\ force%
\endisatagproof
{\isafoldproof}%
%
\isadelimproof
\isanewline
%
\endisadelimproof
\isanewline
\isacommand{lemma}\isamarkupfalse%
\ intersectionEmptyRelationIntersectionEmpty{\isacharcolon}\ \isanewline
\ \ \isakeyword{assumes}\ {\isachardoublequoteopen}runiq\ {\isacharparenleft}R{\isacharcircum}{\isacharminus}{\isadigit{1}}{\isacharparenright}{\isachardoublequoteclose}\ {\isachardoublequoteopen}runiq\ R{\isachardoublequoteclose}\ {\isachardoublequoteopen}X{\isadigit{1}}\ {\isasyminter}\ X{\isadigit{2}}\ {\isacharequal}\ {\isacharbraceleft}{\isacharbraceright}{\isachardoublequoteclose}\ \isanewline
\ \ \isakeyword{shows}\ {\isachardoublequoteopen}{\isacharparenleft}R{\isacharbackquote}{\isacharbackquote}X{\isadigit{1}}{\isacharparenright}\ {\isasyminter}\ {\isacharparenleft}R{\isacharbackquote}{\isacharbackquote}X{\isadigit{2}}{\isacharparenright}\ {\isacharequal}\ {\isacharbraceleft}{\isacharbraceright}{\isachardoublequoteclose}\isanewline
%
\isadelimproof
\ \ %
\endisadelimproof
%
\isatagproof
\isacommand{using}\isamarkupfalse%
\ assms\ \isacommand{by}\isamarkupfalse%
\ {\isacharparenleft}metis\ disj{\isacharunderscore}Domain{\isacharunderscore}imp{\isacharunderscore}disj{\isacharunderscore}Image\ inf{\isacharunderscore}assoc\ inf{\isacharunderscore}bot{\isacharunderscore}right{\isacharparenright}%
\endisatagproof
{\isafoldproof}%
%
\isadelimproof
\isanewline
%
\endisadelimproof
\isanewline
\isacommand{lemma}\isamarkupfalse%
\ lm{\isadigit{1}}{\isadigit{4}}{\isadigit{1}}{\isacharcolon}\ \isanewline
\ \ \isakeyword{assumes}\ {\isachardoublequoteopen}runiq\ f{\isachardoublequoteclose}\ {\isachardoublequoteopen}trivial\ Y{\isachardoublequoteclose}\ \isanewline
\ \ \isakeyword{shows}\ {\isachardoublequoteopen}trivial\ {\isacharparenleft}f\ {\isacharbackquote}{\isacharbackquote}\ {\isacharparenleft}f{\isacharcircum}{\isacharminus}{\isadigit{1}}\ {\isacharbackquote}{\isacharbackquote}\ Y{\isacharparenright}{\isacharparenright}{\isachardoublequoteclose}\isanewline
%
\isadelimproof
\ \ %
\endisadelimproof
%
\isatagproof
\isacommand{using}\isamarkupfalse%
\ assms\ \isacommand{by}\isamarkupfalse%
\ {\isacharparenleft}metis\ rightUniqueFunctionAfterInverse\ trivial{\isacharunderscore}subset{\isacharparenright}%
\endisatagproof
{\isafoldproof}%
%
\isadelimproof
\isanewline
%
\endisadelimproof
\isanewline
\isacommand{lemma}\isamarkupfalse%
\ lm{\isadigit{1}}{\isadigit{4}}{\isadigit{2}}{\isacharcolon}\ \isanewline
\ \ \isakeyword{assumes}\ {\isachardoublequoteopen}trivial\ X{\isachardoublequoteclose}\ \isanewline
\ \ \isakeyword{shows}\ {\isachardoublequoteopen}card\ {\isacharparenleft}Pow\ X{\isacharparenright}{\isasymin}{\isacharbraceleft}{\isadigit{1}}{\isacharcomma}{\isadigit{2}}{\isacharbraceright}{\isachardoublequoteclose}\ \isanewline
%
\isadelimproof
\ \ %
\endisadelimproof
%
\isatagproof
\isacommand{using}\isamarkupfalse%
\ trivial{\isacharunderscore}empty{\isacharunderscore}or{\isacharunderscore}singleton\ card{\isacharunderscore}Pow\ Pow{\isacharunderscore}empty\ assms\ trivial{\isacharunderscore}implies{\isacharunderscore}finite\isanewline
\ \ \ \ \ \ \ \ cardinalityOneTheElemIdentity\ power{\isacharunderscore}one{\isacharunderscore}right\ the{\isacharunderscore}elem{\isacharunderscore}eq\ \isanewline
\ \ \isacommand{by}\isamarkupfalse%
\ {\isacharparenleft}metis\ insert{\isacharunderscore}iff{\isacharparenright}%
\endisatagproof
{\isafoldproof}%
%
\isadelimproof
\isanewline
%
\endisadelimproof
\isanewline
\isacommand{lemma}\isamarkupfalse%
\ lm{\isadigit{1}}{\isadigit{4}}{\isadigit{3}}{\isacharcolon}\ \isanewline
\ \ \isakeyword{assumes}\ {\isachardoublequoteopen}card\ {\isacharparenleft}Pow\ A{\isacharparenright}\ {\isacharequal}\ {\isadigit{1}}{\isachardoublequoteclose}\ \isanewline
\ \ \isakeyword{shows}\ {\isachardoublequoteopen}A\ {\isacharequal}\ {\isacharbraceleft}{\isacharbraceright}{\isachardoublequoteclose}\ \isanewline
%
\isadelimproof
\ \ %
\endisadelimproof
%
\isatagproof
\isacommand{using}\isamarkupfalse%
\ assms\ \isacommand{by}\isamarkupfalse%
\ {\isacharparenleft}metis\ Pow{\isacharunderscore}bottom\ Pow{\isacharunderscore}top\ cardinalityOneTheElemIdentity\ singletonD{\isacharparenright}%
\endisatagproof
{\isafoldproof}%
%
\isadelimproof
\isanewline
%
\endisadelimproof
\isanewline
\ \isanewline
\isacommand{lemma}\isamarkupfalse%
\ lm{\isadigit{1}}{\isadigit{4}}{\isadigit{4}}{\isacharcolon}\ \isanewline
\ \ {\isachardoublequoteopen}{\isacharparenleft}{\isasymnot}\ {\isacharparenleft}finite\ A{\isacharparenright}{\isacharparenright}\ {\isacharequal}\ {\isacharparenleft}card\ {\isacharparenleft}Pow\ A{\isacharparenright}\ {\isacharequal}\ {\isadigit{0}}{\isacharparenright}{\isachardoublequoteclose}\ \isanewline
%
\isadelimproof
\ \ %
\endisadelimproof
%
\isatagproof
\isacommand{by}\isamarkupfalse%
\ auto%
\endisatagproof
{\isafoldproof}%
%
\isadelimproof
\isanewline
%
\endisadelimproof
\isanewline
\isacommand{corollary}\isamarkupfalse%
\ lm{\isadigit{1}}{\isadigit{4}}{\isadigit{5}}{\isacharcolon}\ \isanewline
\ \ {\isachardoublequoteopen}{\isacharparenleft}finite\ A{\isacharparenright}\ {\isacharequal}\ {\isacharparenleft}card\ {\isacharparenleft}Pow\ A{\isacharparenright}\ {\isasymnoteq}\ {\isadigit{0}}{\isacharparenright}{\isachardoublequoteclose}\ \isanewline
%
\isadelimproof
\ \ %
\endisadelimproof
%
\isatagproof
\isacommand{using}\isamarkupfalse%
\ lm{\isadigit{1}}{\isadigit{4}}{\isadigit{4}}\ \isacommand{by}\isamarkupfalse%
\ metis%
\endisatagproof
{\isafoldproof}%
%
\isadelimproof
\isanewline
%
\endisadelimproof
\isanewline
\isacommand{lemma}\isamarkupfalse%
\ lm{\isadigit{1}}{\isadigit{4}}{\isadigit{6}}{\isacharcolon}\ \isanewline
\ \ \isakeyword{assumes}\ {\isachardoublequoteopen}card\ {\isacharparenleft}Pow\ A{\isacharparenright}\ {\isasymnoteq}\ {\isadigit{0}}{\isachardoublequoteclose}\ \isanewline
\ \ \isakeyword{shows}\ {\isachardoublequoteopen}card\ A{\isacharequal}Discrete{\isachardot}log\ {\isacharparenleft}card\ {\isacharparenleft}Pow\ A{\isacharparenright}{\isacharparenright}{\isachardoublequoteclose}\ \isanewline
%
\isadelimproof
\ \ %
\endisadelimproof
%
\isatagproof
\isacommand{using}\isamarkupfalse%
\ assms\ log{\isacharunderscore}exp\ card{\isacharunderscore}Pow\ \isacommand{by}\isamarkupfalse%
\ {\isacharparenleft}metis\ card{\isacharunderscore}infinite\ finite{\isacharunderscore}Pow{\isacharunderscore}iff{\isacharparenright}%
\endisatagproof
{\isafoldproof}%
%
\isadelimproof
\isanewline
%
\endisadelimproof
\isanewline
\isacommand{lemma}\isamarkupfalse%
\ lm{\isadigit{1}}{\isadigit{4}}{\isadigit{7}}{\isacharcolon}\ \isanewline
\ \ \isakeyword{assumes}\ {\isachardoublequoteopen}card\ {\isacharparenleft}Pow\ A{\isacharparenright}\ {\isacharequal}\ {\isadigit{2}}{\isachardoublequoteclose}\ \isanewline
\ \ \isakeyword{shows}\ {\isachardoublequoteopen}card\ A\ {\isacharequal}\ {\isadigit{1}}{\isachardoublequoteclose}\ \isanewline
%
\isadelimproof
\ \ %
\endisadelimproof
%
\isatagproof
\isacommand{using}\isamarkupfalse%
\ assms\ lm{\isadigit{1}}{\isadigit{4}}{\isadigit{6}}\ \isanewline
\ \ \isacommand{by}\isamarkupfalse%
\ {\isacharparenleft}metis{\isacharparenleft}no{\isacharunderscore}types{\isacharparenright}\ comm{\isacharunderscore}semiring{\isacharunderscore}{\isadigit{1}}{\isacharunderscore}class{\isachardot}normalizing{\isacharunderscore}semiring{\isacharunderscore}rules{\isacharparenleft}{\isadigit{3}}{\isadigit{3}}{\isacharparenright}\ \isanewline
\ \ \ \ \ \ \ \ \ \ \ \ \ \ \ \ \ \ \ \ \ \ log{\isacharunderscore}exp\ zero{\isacharunderscore}neq{\isacharunderscore}numeral{\isacharparenright}%
\endisatagproof
{\isafoldproof}%
%
\isadelimproof
\isanewline
%
\endisadelimproof
\isanewline
\isacommand{lemma}\isamarkupfalse%
\ lm{\isadigit{1}}{\isadigit{4}}{\isadigit{8}}{\isacharcolon}\ \isanewline
\ \ \isakeyword{assumes}\ {\isachardoublequoteopen}card\ {\isacharparenleft}Pow\ X{\isacharparenright}\ {\isacharequal}\ {\isadigit{1}}\ {\isasymor}\ card\ {\isacharparenleft}Pow\ X{\isacharparenright}\ {\isacharequal}\ {\isadigit{2}}{\isachardoublequoteclose}\ \isanewline
\ \ \isakeyword{shows}\ {\isachardoublequoteopen}trivial\ X{\isachardoublequoteclose}\ \isanewline
%
\isadelimproof
\ \ %
\endisadelimproof
%
\isatagproof
\isacommand{using}\isamarkupfalse%
\ assms\ trivial{\isacharunderscore}empty{\isacharunderscore}or{\isacharunderscore}singleton\ lm{\isadigit{1}}{\isadigit{4}}{\isadigit{3}}\ lm{\isadigit{1}}{\isadigit{4}}{\isadigit{7}}\ cardinalityOneTheElemIdentity\ \isacommand{by}\isamarkupfalse%
\ metis%
\endisatagproof
{\isafoldproof}%
%
\isadelimproof
\isanewline
%
\endisadelimproof
\isanewline
\isacommand{lemma}\isamarkupfalse%
\ lm{\isadigit{1}}{\isadigit{4}}{\isadigit{9}}{\isacharcolon}\ \isanewline
\ \ {\isachardoublequoteopen}trivial\ A\ {\isacharequal}\ {\isacharparenleft}card\ {\isacharparenleft}Pow\ A{\isacharparenright}\ {\isasymin}\ {\isacharbraceleft}{\isadigit{1}}{\isacharcomma}{\isadigit{2}}{\isacharbraceright}{\isacharparenright}{\isachardoublequoteclose}\ \isanewline
%
\isadelimproof
\ \ %
\endisadelimproof
%
\isatagproof
\isacommand{using}\isamarkupfalse%
\ lm{\isadigit{1}}{\isadigit{4}}{\isadigit{8}}\ lm{\isadigit{1}}{\isadigit{4}}{\isadigit{2}}\ \isacommand{by}\isamarkupfalse%
\ blast%
\endisatagproof
{\isafoldproof}%
%
\isadelimproof
\isanewline
%
\endisadelimproof
\isanewline
\isacommand{lemma}\isamarkupfalse%
\ lm{\isadigit{1}}{\isadigit{5}}{\isadigit{0}}{\isacharcolon}\ \isanewline
\ \ \isakeyword{assumes}\ {\isachardoublequoteopen}R\ {\isasymsubseteq}\ f{\isachardoublequoteclose}\ {\isachardoublequoteopen}runiq\ f{\isachardoublequoteclose}\ {\isachardoublequoteopen}Domain\ f\ {\isacharequal}\ Domain\ R{\isachardoublequoteclose}\ \isanewline
\ \ \isakeyword{shows}\ {\isachardoublequoteopen}runiq\ R{\isachardoublequoteclose}\isanewline
%
\isadelimproof
\ \ %
\endisadelimproof
%
\isatagproof
\isacommand{using}\isamarkupfalse%
\ assms\ \isacommand{by}\isamarkupfalse%
\ {\isacharparenleft}metis\ subrel{\isacharunderscore}runiq{\isacharparenright}%
\endisatagproof
{\isafoldproof}%
%
\isadelimproof
\isanewline
%
\endisadelimproof
\isanewline
\isacommand{lemma}\isamarkupfalse%
\ lm{\isadigit{1}}{\isadigit{5}}{\isadigit{1}}{\isacharcolon}\ \isanewline
\ \ \isakeyword{assumes}\ {\isachardoublequoteopen}f\ {\isasymsubseteq}\ g{\isachardoublequoteclose}\ {\isachardoublequoteopen}runiq\ g{\isachardoublequoteclose}\ {\isachardoublequoteopen}Domain\ f\ {\isacharequal}\ Domain\ g{\isachardoublequoteclose}\ \isanewline
\ \ \isakeyword{shows}\ {\isachardoublequoteopen}g\ {\isasymsubseteq}\ f{\isachardoublequoteclose}\isanewline
%
\isadelimproof
\ \ %
\endisadelimproof
%
\isatagproof
\isacommand{using}\isamarkupfalse%
\ assms\ Domain{\isacharunderscore}iff\ contra{\isacharunderscore}subsetD\ runiq{\isacharunderscore}wrt{\isacharunderscore}ex{\isadigit{1}}\ subrelI\isanewline
\ \ \isacommand{by}\isamarkupfalse%
\ {\isacharparenleft}metis\ {\isacharparenleft}full{\isacharunderscore}types{\isacharcomma}hide{\isacharunderscore}lams{\isacharparenright}{\isacharparenright}%
\endisatagproof
{\isafoldproof}%
%
\isadelimproof
\isanewline
%
\endisadelimproof
\isanewline
\isacommand{lemma}\isamarkupfalse%
\ lm{\isadigit{1}}{\isadigit{5}}{\isadigit{2}}{\isacharcolon}\ \isanewline
\ \ \isakeyword{assumes}\ {\isachardoublequoteopen}R\ {\isasymsubseteq}\ f{\isachardoublequoteclose}\ {\isachardoublequoteopen}runiq\ f{\isachardoublequoteclose}\ {\isachardoublequoteopen}Domain\ f\ {\isasymsubseteq}\ Domain\ R{\isachardoublequoteclose}\ \isanewline
\ \ \isakeyword{shows}\ {\isachardoublequoteopen}f\ {\isacharequal}\ R{\isachardoublequoteclose}\ \isanewline
%
\isadelimproof
\ \ %
\endisadelimproof
%
\isatagproof
\isacommand{using}\isamarkupfalse%
\ assms\ lm{\isadigit{1}}{\isadigit{5}}{\isadigit{1}}\ \isacommand{by}\isamarkupfalse%
\ {\isacharparenleft}metis\ Domain{\isacharunderscore}mono\ dual{\isacharunderscore}order{\isachardot}antisym{\isacharparenright}%
\endisatagproof
{\isafoldproof}%
%
\isadelimproof
\isanewline
%
\endisadelimproof
\isanewline
\isacommand{lemma}\isamarkupfalse%
\ lm{\isadigit{1}}{\isadigit{5}}{\isadigit{3}}{\isacharcolon}\ \isanewline
\ \ {\isachardoublequoteopen}graph\ X\ f\ {\isacharequal}\ {\isacharparenleft}Graph\ f{\isacharparenright}\ {\isacharbar}{\isacharbar}\ X{\isachardoublequoteclose}\ \isanewline
%
\isadelimproof
\ \ %
\endisadelimproof
%
\isatagproof
\isacommand{using}\isamarkupfalse%
\ inf{\isacharunderscore}top{\isachardot}left{\isacharunderscore}neutral\ lm{\isadigit{0}}{\isadigit{0}}{\isadigit{5}}\ domainOfGraph\ restrictedDomain\ lm{\isadigit{1}}{\isadigit{5}}{\isadigit{2}}\ graphIntersection\isanewline
\ \ \ \ \ \ \ \ restriction{\isacharunderscore}is{\isacharunderscore}subrel\ subrel{\isacharunderscore}runiq\ subset{\isacharunderscore}iff\ \isanewline
\ \ \isacommand{by}\isamarkupfalse%
\ {\isacharparenleft}metis\ {\isacharparenleft}erased{\isacharcomma}\ lifting{\isacharparenright}{\isacharparenright}%
\endisatagproof
{\isafoldproof}%
%
\isadelimproof
\isanewline
%
\endisadelimproof
\isanewline
\isacommand{lemma}\isamarkupfalse%
\ lm{\isadigit{1}}{\isadigit{5}}{\isadigit{4}}{\isacharcolon}\ \isanewline
\ \ {\isachardoublequoteopen}graph\ {\isacharparenleft}X\ {\isasyminter}\ Y{\isacharparenright}\ f\ {\isacharequal}\ {\isacharparenleft}graph\ X\ f{\isacharparenright}\ {\isacharbar}{\isacharbar}\ Y{\isachardoublequoteclose}\ \isanewline
%
\isadelimproof
\ \ %
\endisadelimproof
%
\isatagproof
\isacommand{using}\isamarkupfalse%
\ doubleRestriction\ lm{\isadigit{1}}{\isadigit{5}}{\isadigit{3}}\ \isacommand{by}\isamarkupfalse%
\ metis%
\endisatagproof
{\isafoldproof}%
%
\isadelimproof
\isanewline
%
\endisadelimproof
\isanewline
\isacommand{lemma}\isamarkupfalse%
\ restrictionVsIntersection{\isacharcolon}\isanewline
\ \ {\isachardoublequoteopen}{\isacharbraceleft}{\isacharparenleft}x{\isacharcomma}\ f\ x{\isacharparenright}{\isacharbar}\ x{\isachardot}\ x\ {\isasymin}\ X{\isadigit{2}}{\isacharbraceright}\ {\isacharbar}{\isacharbar}\ X{\isadigit{1}}\ {\isacharequal}\ {\isacharbraceleft}{\isacharparenleft}x{\isacharcomma}\ f\ x{\isacharparenright}{\isacharbar}\ x{\isachardot}\ x\ {\isasymin}\ X{\isadigit{2}}\ {\isasyminter}\ X{\isadigit{1}}{\isacharbraceright}{\isachardoublequoteclose}\ \isanewline
%
\isadelimproof
\ \ %
\endisadelimproof
%
\isatagproof
\isacommand{using}\isamarkupfalse%
\ graph{\isacharunderscore}def\ lm{\isadigit{1}}{\isadigit{5}}{\isadigit{4}}\ \isacommand{by}\isamarkupfalse%
\ metis%
\endisatagproof
{\isafoldproof}%
%
\isadelimproof
\isanewline
%
\endisadelimproof
\isanewline
\isacommand{lemma}\isamarkupfalse%
\ lm{\isadigit{1}}{\isadigit{5}}{\isadigit{5}}{\isacharcolon}\ \isanewline
\ \ \isakeyword{assumes}\ {\isachardoublequoteopen}runiq\ f{\isachardoublequoteclose}\ {\isachardoublequoteopen}X\ {\isasymsubseteq}\ Domain\ f{\isachardoublequoteclose}\ \isanewline
\ \ \isakeyword{shows}\ {\isachardoublequoteopen}graph\ X\ {\isacharparenleft}toFunction\ f{\isacharparenright}\ {\isacharequal}\ {\isacharparenleft}f{\isacharbar}{\isacharbar}X{\isacharparenright}{\isachardoublequoteclose}\ \isanewline
%
\isadelimproof
%
\endisadelimproof
%
\isatagproof
\isacommand{proof}\isamarkupfalse%
\ {\isacharminus}\isanewline
\ \ \isacommand{have}\isamarkupfalse%
\ {\isachardoublequoteopen}{\isasymAnd}v\ w{\isachardot}\ {\isacharparenleft}v{\isasymColon}{\isacharprime}a\ set{\isacharparenright}\ {\isasymsubseteq}\ w\ {\isasymlongrightarrow}\ w\ {\isasyminter}\ v\ {\isacharequal}\ v{\isachardoublequoteclose}\ \isacommand{by}\isamarkupfalse%
\ {\isacharparenleft}simp\ add{\isacharcolon}\ Int{\isacharunderscore}commute\ inf{\isachardot}absorb{\isadigit{1}}{\isacharparenright}\isanewline
\ \ \isacommand{thus}\isamarkupfalse%
\ {\isachardoublequoteopen}graph\ X\ {\isacharparenleft}toFunction\ f{\isacharparenright}\ {\isacharequal}\ f\ {\isacharbar}{\isacharbar}\ X{\isachardoublequoteclose}\ \isacommand{by}\isamarkupfalse%
\ {\isacharparenleft}metis\ assms{\isacharparenleft}{\isadigit{1}}{\isacharparenright}\ assms{\isacharparenleft}{\isadigit{2}}{\isacharparenright}\ doubleRestriction\ lm{\isadigit{0}}{\isadigit{0}}{\isadigit{4}}\ lm{\isadigit{1}}{\isadigit{5}}{\isadigit{3}}{\isacharparenright}\isanewline
\isacommand{qed}\isamarkupfalse%
%
\endisatagproof
{\isafoldproof}%
%
\isadelimproof
\isanewline
%
\endisadelimproof
\isanewline
\isacommand{lemma}\isamarkupfalse%
\ lm{\isadigit{1}}{\isadigit{5}}{\isadigit{6}}{\isacharcolon}\ \isanewline
\ \ {\isachardoublequoteopen}{\isacharparenleft}Graph\ f{\isacharparenright}\ {\isacharbackquote}{\isacharbackquote}\ X\ {\isacharequal}\ f\ {\isacharbackquote}\ X{\isachardoublequoteclose}\ \isanewline
%
\isadelimproof
\ \ %
\endisadelimproof
%
\isatagproof
\isacommand{unfolding}\isamarkupfalse%
\ Graph{\isacharunderscore}def\ image{\isacharunderscore}def\ \isacommand{by}\isamarkupfalse%
\ auto%
\endisatagproof
{\isafoldproof}%
%
\isadelimproof
\isanewline
%
\endisadelimproof
\isanewline
\isacommand{lemma}\isamarkupfalse%
\ lm{\isadigit{1}}{\isadigit{5}}{\isadigit{7}}{\isacharcolon}\ \isanewline
\ \ \isakeyword{assumes}\ {\isachardoublequoteopen}X\ {\isasymsubseteq}\ Domain\ f{\isachardoublequoteclose}\ {\isachardoublequoteopen}runiq\ f{\isachardoublequoteclose}\ \isanewline
\ \ \isakeyword{shows}\ {\isachardoublequoteopen}f{\isacharbackquote}{\isacharbackquote}X\ {\isacharequal}\ {\isacharparenleft}eval{\isacharunderscore}rel\ f{\isacharparenright}{\isacharbackquote}X{\isachardoublequoteclose}\isanewline
%
\isadelimproof
\ \ %
\endisadelimproof
%
\isatagproof
\isacommand{using}\isamarkupfalse%
\ assms\ lm{\isadigit{1}}{\isadigit{5}}{\isadigit{6}}\ \isacommand{by}\isamarkupfalse%
\ {\isacharparenleft}metis\ restrictedRange\ lm{\isadigit{1}}{\isadigit{5}}{\isadigit{3}}\ lm{\isadigit{1}}{\isadigit{5}}{\isadigit{5}}\ toFunction{\isacharunderscore}def{\isacharparenright}%
\endisatagproof
{\isafoldproof}%
%
\isadelimproof
\isanewline
%
\endisadelimproof
\isanewline
\isacommand{lemma}\isamarkupfalse%
\ cardOneImageCardOne{\isacharcolon}\ \isanewline
\ \ \isakeyword{assumes}\ {\isachardoublequoteopen}card\ A\ {\isacharequal}\ {\isadigit{1}}{\isachardoublequoteclose}\ \isanewline
\ \ \isakeyword{shows}\ {\isachardoublequoteopen}card\ {\isacharparenleft}f{\isacharbackquote}A{\isacharparenright}\ {\isacharequal}\ {\isadigit{1}}{\isachardoublequoteclose}\ \isanewline
%
\isadelimproof
\ \ %
\endisadelimproof
%
\isatagproof
\isacommand{using}\isamarkupfalse%
\ assms\ card{\isacharunderscore}image\ card{\isacharunderscore}image{\isacharunderscore}le\ \isanewline
\isacommand{proof}\isamarkupfalse%
\ {\isacharminus}\isanewline
\ \ \isacommand{have}\isamarkupfalse%
\ {\isachardoublequoteopen}finite\ {\isacharparenleft}f{\isacharbackquote}A{\isacharparenright}{\isachardoublequoteclose}\ \isacommand{using}\isamarkupfalse%
\ assms\ One{\isacharunderscore}nat{\isacharunderscore}def\ Suc{\isacharunderscore}not{\isacharunderscore}Zero\ card{\isacharunderscore}infinite\ finite{\isacharunderscore}imageI\ \isanewline
\ \ \ \ \ \ \ \isacommand{by}\isamarkupfalse%
\ {\isacharparenleft}metis{\isacharparenleft}no{\isacharunderscore}types{\isacharparenright}{\isacharparenright}\ \isanewline
\ \ \isacommand{moreover}\isamarkupfalse%
\ \isacommand{have}\isamarkupfalse%
\ {\isachardoublequoteopen}f{\isacharbackquote}A\ {\isasymnoteq}\ {\isacharbraceleft}{\isacharbraceright}{\isachardoublequoteclose}\ \isacommand{using}\isamarkupfalse%
\ assms\ \isacommand{by}\isamarkupfalse%
\ fastforce\isanewline
\ \ \isacommand{moreover}\isamarkupfalse%
\ \isacommand{have}\isamarkupfalse%
\ {\isachardoublequoteopen}card\ {\isacharparenleft}f{\isacharbackquote}A{\isacharparenright}\ {\isasymle}\ {\isadigit{1}}{\isachardoublequoteclose}\ \isacommand{using}\isamarkupfalse%
\ assms\ card{\isacharunderscore}image{\isacharunderscore}le\ One{\isacharunderscore}nat{\isacharunderscore}def\ Suc{\isacharunderscore}not{\isacharunderscore}Zero\ card{\isacharunderscore}infinite\isanewline
\ \ \ \ \ \ \ \isacommand{by}\isamarkupfalse%
\ {\isacharparenleft}metis{\isacharparenright}\isanewline
\ \ \isacommand{ultimately}\isamarkupfalse%
\ \isacommand{show}\isamarkupfalse%
\ {\isacharquery}thesis\ \isacommand{by}\isamarkupfalse%
\ {\isacharparenleft}metis\ assms\ image{\isacharunderscore}empty\ image{\isacharunderscore}insert\ \isanewline
\ \ \ \ \ \ \ \ \ \ \ \ \ \ \ \ \ \ \ \ \ \ \ \ \ \ \ \ \ \ \ \ \ \ \ \ cardinalityOneTheElemIdentity\ the{\isacharunderscore}elem{\isacharunderscore}eq{\isacharparenright}\isanewline
\isacommand{qed}\isamarkupfalse%
%
\endisatagproof
{\isafoldproof}%
%
\isadelimproof
\isanewline
%
\endisadelimproof
\isanewline
\isacommand{lemma}\isamarkupfalse%
\ cardOneTheElem{\isacharcolon}\ \isanewline
\ \ \isakeyword{assumes}\ {\isachardoublequoteopen}card\ A\ {\isacharequal}\ {\isadigit{1}}{\isachardoublequoteclose}\ \isanewline
\ \ \isakeyword{shows}\ {\isachardoublequoteopen}the{\isacharunderscore}elem\ {\isacharparenleft}f{\isacharbackquote}A{\isacharparenright}\ {\isacharequal}\ f\ {\isacharparenleft}the{\isacharunderscore}elem\ A{\isacharparenright}{\isachardoublequoteclose}\ \isanewline
%
\isadelimproof
\ \ %
\endisadelimproof
%
\isatagproof
\isacommand{using}\isamarkupfalse%
\ assms\ image{\isacharunderscore}empty\ image{\isacharunderscore}insert\ the{\isacharunderscore}elem{\isacharunderscore}eq\ \isacommand{by}\isamarkupfalse%
\ {\isacharparenleft}metis\ cardinalityOneTheElemIdentity{\isacharparenright}%
\endisatagproof
{\isafoldproof}%
%
\isadelimproof
\isanewline
%
\endisadelimproof
\isanewline
\isanewline
\isacommand{abbreviation}\isamarkupfalse%
\ \isanewline
\ \ {\isachardoublequoteopen}swap\ f\ {\isacharequal}{\isacharequal}\ curry\ {\isacharparenleft}{\isacharparenleft}split\ f{\isacharparenright}\ {\isasymcirc}\ flip{\isacharparenright}{\isachardoublequoteclose}\ \isanewline
\isanewline
\isanewline
\isacommand{lemma}\isamarkupfalse%
\ lm{\isadigit{1}}{\isadigit{5}}{\isadigit{8}}{\isacharcolon}\ \isanewline
\ \ {\isachardoublequoteopen}finite\ X\ \ \ {\isacharequal}\ \ {\isacharparenleft}X\ {\isasymin}\ range\ set{\isacharparenright}{\isachardoublequoteclose}\ \isanewline
%
\isadelimproof
\ \ %
\endisadelimproof
%
\isatagproof
\isacommand{by}\isamarkupfalse%
\ {\isacharparenleft}metis\ List{\isachardot}finite{\isacharunderscore}set\ finite{\isacharunderscore}list\ image{\isacharunderscore}iff\ rangeI{\isacharparenright}%
\endisatagproof
{\isafoldproof}%
%
\isadelimproof
\isanewline
%
\endisadelimproof
\isanewline
\isanewline
\isacommand{lemma}\isamarkupfalse%
\ lm{\isadigit{1}}{\isadigit{5}}{\isadigit{9}}{\isacharcolon}\ \isanewline
\ \ {\isachardoublequoteopen}finite\ {\isacharequal}\ {\isacharparenleft}{\isacharpercent}X{\isachardot}\ X{\isasymin}range\ set{\isacharparenright}{\isachardoublequoteclose}\ \isanewline
%
\isadelimproof
\ \ %
\endisadelimproof
%
\isatagproof
\isacommand{using}\isamarkupfalse%
\ lm{\isadigit{1}}{\isadigit{5}}{\isadigit{8}}\ \isacommand{by}\isamarkupfalse%
\ metis%
\endisatagproof
{\isafoldproof}%
%
\isadelimproof
\isanewline
%
\endisadelimproof
\isanewline
\isacommand{lemma}\isamarkupfalse%
\ lm{\isadigit{1}}{\isadigit{6}}{\isadigit{0}}{\isacharcolon}\ \isanewline
\ \ {\isachardoublequoteopen}swap\ f\ {\isacharequal}\ {\isacharparenleft}{\isacharpercent}x{\isachardot}\ {\isacharpercent}y{\isachardot}\ f\ y\ x{\isacharparenright}{\isachardoublequoteclose}\ \isanewline
%
\isadelimproof
\ \ %
\endisadelimproof
%
\isatagproof
\isacommand{by}\isamarkupfalse%
\ {\isacharparenleft}metis\ comp{\isacharunderscore}eq{\isacharunderscore}dest{\isacharunderscore}lhs\ curry{\isacharunderscore}def\ flip{\isacharunderscore}def\ fst{\isacharunderscore}conv\ old{\isachardot}prod{\isachardot}case\ snd{\isacharunderscore}conv{\isacharparenright}%
\endisatagproof
{\isafoldproof}%
%
\isadelimproof
%
\endisadelimproof
%
\isamarkupsection{some easy properties on real numbers%
}
\isamarkuptrue%
\isacommand{lemma}\isamarkupfalse%
\ lm{\isadigit{1}}{\isadigit{6}}{\isadigit{1}}{\isacharcolon}\ \isanewline
\ \ \isakeyword{fixes}\ a{\isacharcolon}{\isacharcolon}real\ \isanewline
\ \ \isakeyword{fixes}\ b\ c\ \isanewline
\ \ \isakeyword{shows}\ {\isachardoublequoteopen}a{\isacharasterisk}b\ {\isacharminus}\ a{\isacharasterisk}c{\isacharequal}a{\isacharasterisk}{\isacharparenleft}b{\isacharminus}c{\isacharparenright}{\isachardoublequoteclose}\isanewline
%
\isadelimproof
\ \ %
\endisadelimproof
%
\isatagproof
\isacommand{using}\isamarkupfalse%
\ assms\ \isacommand{by}\isamarkupfalse%
\ {\isacharparenleft}metis\ real{\isacharunderscore}scaleR{\isacharunderscore}def\ real{\isacharunderscore}vector{\isachardot}scale{\isacharunderscore}right{\isacharunderscore}diff{\isacharunderscore}distrib{\isacharparenright}%
\endisatagproof
{\isafoldproof}%
%
\isadelimproof
\isanewline
%
\endisadelimproof
\isanewline
\isacommand{lemma}\isamarkupfalse%
\ lm{\isadigit{1}}{\isadigit{6}}{\isadigit{2}}{\isacharcolon}\ \isanewline
\ \ \isakeyword{fixes}\ a{\isacharcolon}{\isacharcolon}real\ \isanewline
\ \ \isakeyword{fixes}\ b\ c\ \isanewline
\ \ \isakeyword{shows}\ {\isachardoublequoteopen}a{\isacharasterisk}b\ {\isacharminus}\ c{\isacharasterisk}b{\isacharequal}{\isacharparenleft}a{\isacharminus}c{\isacharparenright}{\isacharasterisk}b{\isachardoublequoteclose}\isanewline
%
\isadelimproof
\ \ %
\endisadelimproof
%
\isatagproof
\isacommand{using}\isamarkupfalse%
\ assms\ lm{\isadigit{1}}{\isadigit{6}}{\isadigit{1}}\ \isacommand{by}\isamarkupfalse%
\ {\isacharparenleft}metis\ comm{\isacharunderscore}semiring{\isacharunderscore}{\isadigit{1}}{\isacharunderscore}class{\isachardot}normalizing{\isacharunderscore}semiring{\isacharunderscore}rules{\isacharparenleft}{\isadigit{7}}{\isacharparenright}{\isacharparenright}%
\endisatagproof
{\isafoldproof}%
%
\isadelimproof
\isanewline
%
\endisadelimproof
%
\isadelimtheory
\isanewline
%
\endisadelimtheory
%
\isatagtheory
\isacommand{end}\isamarkupfalse%
%
\endisatagtheory
{\isafoldtheory}%
%
\isadelimtheory
%
\endisadelimtheory
\end{isabellebody}%
%%% Local Variables:
%%% mode: latex
%%% TeX-master: "root"
%%% End:


%
\begin{isabellebody}%
\def\isabellecontext{StrictCombinatorialAuction}%
%
\isamarkupheader{Definitions about those Combinatorial Auctions which are strict (i.e., which assign all the available goods)%
}
\isamarkuptrue%
%
\isadelimtheory
%
\endisadelimtheory
%
\isatagtheory
\isacommand{theory}\isamarkupfalse%
\ StrictCombinatorialAuction\isanewline
\isakeyword{imports}\ Complex{\isacharunderscore}Main\isanewline
\ \ Partitions\isanewline
\ \ MiscTools\isanewline
\isanewline
\isakeyword{begin}%
\endisatagtheory
{\isafoldtheory}%
%
\isadelimtheory
%
\endisadelimtheory
%
\isamarkupsection{Types%
}
\isamarkuptrue%
\isacommand{type{\isacharunderscore}synonym}\isamarkupfalse%
\ index\ {\isacharequal}\ {\isachardoublequoteopen}nat{\isachardoublequoteclose}\isanewline
\isacommand{type{\isacharunderscore}synonym}\isamarkupfalse%
\ participant\ {\isacharequal}\ index\isanewline
\isacommand{type{\isacharunderscore}synonym}\isamarkupfalse%
\ good\ {\isacharequal}\ nat\isanewline
\isacommand{type{\isacharunderscore}synonym}\isamarkupfalse%
\ goods\ {\isacharequal}\ {\isachardoublequoteopen}nat\ set{\isachardoublequoteclose}\ \isanewline
\isacommand{type{\isacharunderscore}synonym}\isamarkupfalse%
\ price\ {\isacharequal}\ real\isanewline
\isanewline
\isanewline
\isanewline
\isacommand{type{\isacharunderscore}synonym}\isamarkupfalse%
\ bids{\isadigit{3}}\ {\isacharequal}\ {\isachardoublequoteopen}{\isacharparenleft}{\isacharparenleft}participant\ {\isasymtimes}\ goods{\isacharparenright}\ {\isasymtimes}\ price{\isacharparenright}\ set{\isachardoublequoteclose}\isanewline
\isacommand{type{\isacharunderscore}synonym}\isamarkupfalse%
\ bids\ {\isacharequal}\ {\isachardoublequoteopen}participant\ {\isasymRightarrow}\ goods\ {\isasymRightarrow}\ price{\isachardoublequoteclose}\isanewline
\isacommand{type{\isacharunderscore}synonym}\isamarkupfalse%
\ allocation{\isacharunderscore}rel\ {\isacharequal}\ {\isachardoublequoteopen}{\isacharparenleft}goods\ {\isasymtimes}\ participant{\isacharparenright}\ set{\isachardoublequoteclose}\isanewline
\isacommand{type{\isacharunderscore}synonym}\isamarkupfalse%
\ allocation\ {\isacharequal}\ {\isachardoublequoteopen}{\isacharparenleft}participant\ {\isasymtimes}\ goods{\isacharparenright}\ set{\isachardoublequoteclose}\isanewline
\isanewline
\isanewline
\isacommand{type{\isacharunderscore}synonym}\isamarkupfalse%
\ payments\ {\isacharequal}\ {\isachardoublequoteopen}participant\ {\isasymRightarrow}\ price{\isachardoublequoteclose}\isanewline
\isacommand{type{\isacharunderscore}synonym}\isamarkupfalse%
\ altbids\ {\isacharequal}\ {\isachardoublequoteopen}{\isacharparenleft}participant\ {\isasymtimes}\ goods{\isacharparenright}\ {\isasymRightarrow}\ price{\isachardoublequoteclose}\isanewline
\isacommand{type{\isacharunderscore}synonym}\isamarkupfalse%
\ bidvector{\isacharequal}altbids\isanewline
\isacommand{abbreviation}\isamarkupfalse%
\ {\isachardoublequoteopen}altbids\ {\isacharparenleft}b{\isacharcolon}{\isacharcolon}bids{\isacharparenright}\ {\isacharequal}{\isacharequal}\ split\ b{\isachardoublequoteclose}\isanewline
\isanewline
\isacommand{abbreviation}\isamarkupfalse%
\ {\isachardoublequoteopen}proceeds\ {\isacharparenleft}b{\isacharcolon}{\isacharcolon}altbids{\isacharparenright}\ {\isacharparenleft}allo{\isacharcolon}{\isacharcolon}allocation{\isacharparenright}\ {\isacharequal}{\isacharequal}\ setsum\ b\ allo{\isachardoublequoteclose}\isanewline
\isanewline
\isacommand{abbreviation}\isamarkupfalse%
\ participants\ \isakeyword{where}\ {\isachardoublequoteopen}participants\ {\isacharparenleft}a{\isacharcolon}{\isacharcolon}allocation{\isacharparenright}\ {\isacharequal}{\isacharequal}\ Domain\ a{\isachardoublequoteclose}\isanewline
\isacommand{abbreviation}\isamarkupfalse%
\ goods{\isacharcolon}{\isacharcolon}{\isachardoublequoteopen}allocation\ {\isacharequal}{\isachargreater}\ goods{\isachardoublequoteclose}\ \isakeyword{where}\ {\isachardoublequoteopen}goods\ {\isacharparenleft}allo{\isacharcolon}{\isacharcolon}allocation{\isacharparenright}\ {\isacharequal}{\isacharequal}\ {\isasymUnion}\ {\isacharparenleft}Range\ allo{\isacharparenright}{\isachardoublequoteclose}%
\isamarkupsection{Allocations%
}
\isamarkuptrue%
\isacommand{fun}\isamarkupfalse%
\ possible{\isacharunderscore}allocations{\isacharunderscore}rel\ \isanewline
\isakeyword{where}\ {\isachardoublequoteopen}possible{\isacharunderscore}allocations{\isacharunderscore}rel\ G\ N\ {\isacharequal}\ Union\ {\isacharbraceleft}\ injections\ Y\ N\ {\isacharbar}\ Y\ {\isachardot}\ Y\ {\isasymin}\ all{\isacharunderscore}partitions\ G\ {\isacharbraceright}{\isachardoublequoteclose}\ \isanewline
\isanewline
\isacommand{abbreviation}\isamarkupfalse%
\ {\isachardoublequoteopen}is{\isacharunderscore}partition{\isacharunderscore}of{\isacharprime}\ P\ A\ {\isacharequal}{\isacharequal}\ {\isacharparenleft}{\isasymUnion}\ P\ {\isacharequal}\ A\ {\isasymand}\ is{\isacharunderscore}partition\ P{\isacharparenright}{\isachardoublequoteclose}\isanewline
\isanewline
\isacommand{abbreviation}\isamarkupfalse%
\ {\isachardoublequoteopen}all{\isacharunderscore}partitions{\isacharprime}\ A\ {\isacharequal}{\isacharequal}\ {\isacharbraceleft}P\ {\isachardot}\ is{\isacharunderscore}partition{\isacharunderscore}of{\isacharprime}\ P\ A{\isacharbraceright}{\isachardoublequoteclose}\isanewline
\isanewline
\isacommand{abbreviation}\isamarkupfalse%
\ {\isachardoublequoteopen}injections{\isacharprime}\ X\ Y\ {\isacharequal}{\isacharequal}\ {\isacharbraceleft}R\ {\isachardot}\ Domain\ R\ {\isacharequal}\ X\ {\isasymand}\ Range\ R\ {\isasymsubseteq}\ Y\ {\isasymand}\ runiq\ R\ {\isasymand}\ runiq\ {\isacharparenleft}R{\isasyminverse}{\isacharparenright}{\isacharbraceright}{\isachardoublequoteclose}\isanewline
\isanewline
\isacommand{abbreviation}\isamarkupfalse%
\ {\isachardoublequoteopen}possible{\isacharunderscore}allocations{\isacharunderscore}rel{\isacharprime}\ G\ N\ {\isacharequal}{\isacharequal}\ Union{\isacharbraceleft}injections{\isacharprime}\ Y\ N\ {\isacharbar}\ Y\ {\isachardot}\ Y\ {\isasymin}\ all{\isacharunderscore}partitions{\isacharprime}\ G{\isacharbraceright}{\isachardoublequoteclose}\isanewline
\isanewline
\isacommand{abbreviation}\isamarkupfalse%
\ possibleAllocationsRel\ \isakeyword{where}\ \isanewline
{\isachardoublequoteopen}possibleAllocationsRel\ N\ G\ {\isacharequal}{\isacharequal}\ converse\ {\isacharbackquote}\ {\isacharparenleft}possible{\isacharunderscore}allocations{\isacharunderscore}rel\ G\ N{\isacharparenright}{\isachardoublequoteclose}\isanewline
\isanewline
\isacommand{notepad}\isamarkupfalse%
\isanewline
\isakeyword{begin}\isanewline
%
\isadelimproof
\ \ %
\endisadelimproof
%
\isatagproof
\isacommand{fix}\isamarkupfalse%
\ Rs{\isacharcolon}{\isacharcolon}{\isachardoublequoteopen}{\isacharparenleft}{\isacharprime}a\ {\isasymtimes}\ {\isacharprime}b{\isacharparenright}\ set\ set{\isachardoublequoteclose}\isanewline
\ \ \isacommand{fix}\isamarkupfalse%
\ Sss{\isacharcolon}{\isacharcolon}{\isachardoublequoteopen}{\isacharprime}a\ set\ set{\isachardoublequoteclose}\isanewline
\ \ \isacommand{fix}\isamarkupfalse%
\ P{\isacharcolon}{\isacharcolon}{\isachardoublequoteopen}{\isacharprime}a\ set\ {\isasymRightarrow}\ {\isacharparenleft}{\isacharprime}a\ {\isasymtimes}\ {\isacharprime}b{\isacharparenright}\ set\ set{\isachardoublequoteclose}\isanewline
\ \ \isanewline
\ \ \isacommand{have}\isamarkupfalse%
\ {\isachardoublequoteopen}{\isacharbraceleft}\ R\ {\isachardot}\ {\isasymexists}\ Y\ {\isasymin}\ Sss\ {\isachardot}\ R\ {\isasymin}\ P\ Y\ {\isacharbraceright}\ {\isacharequal}\ {\isasymUnion}\ {\isacharbraceleft}\ P\ Y\ {\isacharbar}\ Y\ {\isachardot}\ Y\ {\isasymin}\ Sss\ {\isacharbraceright}{\isachardoublequoteclose}\ \isanewline
\ \ \isacommand{using}\isamarkupfalse%
\ Collect{\isacharunderscore}cong\ Union{\isacharunderscore}eq\ mem{\isacharunderscore}Collect{\isacharunderscore}eq\ \isacommand{by}\isamarkupfalse%
\ blast%
\endisatagproof
{\isafoldproof}%
%
\isadelimproof
\isanewline
%
\endisadelimproof
\isacommand{end}\isamarkupfalse%
%
\begin{isamarkuptext}%
algorithmic version of \isa{possible{\isacharunderscore}allocations{\isacharunderscore}rel}%
\end{isamarkuptext}%
\isamarkuptrue%
\isacommand{fun}\isamarkupfalse%
\ possible{\isacharunderscore}allocations{\isacharunderscore}alg\ {\isacharcolon}{\isacharcolon}\ {\isachardoublequoteopen}goods\ {\isasymRightarrow}\ participant\ set\ {\isasymRightarrow}\ allocation{\isacharunderscore}rel\ list{\isachardoublequoteclose}\isanewline
\isakeyword{where}\ {\isachardoublequoteopen}possible{\isacharunderscore}allocations{\isacharunderscore}alg\ G\ N\ {\isacharequal}\ \isanewline
concat\ {\isacharbrackleft}\ injections{\isacharunderscore}alg\ Y\ N\ {\isachardot}\ Y\ {\isasymleftarrow}\ all{\isacharunderscore}partitions{\isacharunderscore}alg\ G\ {\isacharbrackright}{\isachardoublequoteclose}\isanewline
\isacommand{abbreviation}\isamarkupfalse%
\ {\isachardoublequoteopen}possibleAllocationsAlg\ N\ G\ {\isacharequal}{\isacharequal}\ \isanewline
{\isacharparenleft}map\ converse\ {\isacharparenleft}possible{\isacharunderscore}allocations{\isacharunderscore}alg\ G\ N{\isacharparenright}{\isacharparenright}{\isachardoublequoteclose}\isanewline
\isacommand{abbreviation}\isamarkupfalse%
\ {\isachardoublequoteopen}possibleAllocationsAlg{\isadigit{2}}\ N\ G\ {\isacharequal}{\isacharequal}\ \isanewline
converse\ {\isacharbackquote}\ {\isacharparenleft}{\isasymUnion}\ set\ {\isacharbrackleft}set\ {\isacharparenleft}injections{\isacharunderscore}alg\ l\ N{\isacharparenright}\ {\isachardot}\ l\ {\isasymleftarrow}\ all{\isacharunderscore}partitions{\isacharunderscore}list\ G{\isacharbrackright}{\isacharparenright}{\isachardoublequoteclose}\isanewline
\isacommand{abbreviation}\isamarkupfalse%
\ {\isachardoublequoteopen}possibleAllocationsAlg{\isadigit{3}}\ N\ G\ {\isacharequal}{\isacharequal}\ \isanewline
map\ converse\ {\isacharparenleft}concat\ {\isacharbrackleft}{\isacharparenleft}injections{\isacharunderscore}alg\ l\ N{\isacharparenright}\ {\isachardot}\ l\ {\isasymleftarrow}\ all{\isacharunderscore}partitions{\isacharunderscore}list\ G{\isacharbrackright}{\isacharparenright}{\isachardoublequoteclose}\isanewline
\isacommand{lemma}\isamarkupfalse%
\ lm{\isadigit{0}}{\isadigit{1}}{\isacharcolon}\ {\isachardoublequoteopen}set\ {\isacharparenleft}possibleAllocationsAlg{\isadigit{3}}\ N\ G{\isacharparenright}\ {\isacharequal}\ possibleAllocationsAlg{\isadigit{2}}\ N\ G{\isachardoublequoteclose}\isanewline
%
\isadelimproof
%
\endisadelimproof
%
\isatagproof
\isacommand{using}\isamarkupfalse%
\ assms\ \isacommand{by}\isamarkupfalse%
\ auto%
\endisatagproof
{\isafoldproof}%
%
\isadelimproof
%
\endisadelimproof
%
\isamarkupsection{VCG mechanism%
}
\isamarkuptrue%
\isacommand{abbreviation}\isamarkupfalse%
\ {\isachardoublequoteopen}winningAllocationsRel\ N\ G\ b\ {\isacharequal}{\isacharequal}\ \isanewline
argmax\ {\isacharparenleft}setsum\ b{\isacharparenright}\ {\isacharparenleft}possibleAllocationsRel\ N\ G{\isacharparenright}{\isachardoublequoteclose}\isanewline
\isanewline
\isacommand{abbreviation}\isamarkupfalse%
\ {\isachardoublequoteopen}winningAllocationRel\ N\ G\ t\ b\ {\isacharequal}{\isacharequal}\ t\ {\isacharparenleft}winningAllocationsRel\ N\ G\ b{\isacharparenright}{\isachardoublequoteclose}\isanewline
\isanewline
\isacommand{abbreviation}\isamarkupfalse%
\ {\isachardoublequoteopen}winningAllocationsAlg\ N\ G\ b\ {\isacharequal}{\isacharequal}\ argmaxList\ {\isacharparenleft}proceeds\ b{\isacharparenright}\ {\isacharparenleft}possibleAllocationsAlg{\isadigit{3}}\ N\ G{\isacharparenright}{\isachardoublequoteclose}\isanewline
\isanewline
\isacommand{definition}\isamarkupfalse%
\ {\isachardoublequoteopen}winningAllocationAlg\ N\ G\ t\ b\ {\isacharequal}{\isacharequal}\ t\ {\isacharparenleft}winningAllocationsAlg\ N\ G\ b{\isacharparenright}{\isachardoublequoteclose}%
\begin{isamarkuptext}%
payments%
\end{isamarkuptext}%
\isamarkuptrue%
%
\begin{isamarkuptext}%
the maximum sum of bids of all bidders except bidder \isa{n}'s bid, computed over all possible allocations of all goods,
  i.e. the value reportedly generated by value maximization problem when solved without n's bids%
\end{isamarkuptext}%
\isamarkuptrue%
\isacommand{abbreviation}\isamarkupfalse%
\ {\isachardoublequoteopen}alpha\ N\ G\ b\ n\ {\isacharequal}{\isacharequal}\ Max\ {\isacharparenleft}{\isacharparenleft}setsum\ b{\isacharparenright}{\isacharbackquote}{\isacharparenleft}possibleAllocationsRel\ {\isacharparenleft}N{\isacharminus}{\isacharbraceleft}n{\isacharbraceright}{\isacharparenright}\ G{\isacharparenright}{\isacharparenright}{\isachardoublequoteclose}\isanewline
\isanewline
\isacommand{abbreviation}\isamarkupfalse%
\ {\isachardoublequoteopen}remainingValueRel\ N\ G\ t\ b\ n\ {\isacharequal}{\isacharequal}\ setsum\ b\ {\isacharparenleft}winningAllocationRel\ N\ G\ t\ b\ {\isacharminus}{\isacharminus}\ n{\isacharparenright}{\isachardoublequoteclose}\isanewline
\isanewline
\isacommand{abbreviation}\isamarkupfalse%
\ {\isachardoublequoteopen}paymentsRel\ N\ G\ t\ {\isacharequal}{\isacharequal}\ alpha\ N\ G\ {\isacharminus}\ remainingValueRel\ N\ G\ t{\isachardoublequoteclose}\isanewline
\isanewline
\isacommand{abbreviation}\isamarkupfalse%
\ {\isachardoublequoteopen}remainingValueAlg\ N\ G\ t\ b\ n\ {\isacharequal}{\isacharequal}\ proceeds\ b\ {\isacharparenleft}winningAllocationAlg\ N\ G\ t\ b\ {\isacharminus}{\isacharminus}\ n{\isacharparenright}{\isachardoublequoteclose}\isanewline
\isanewline
\isacommand{abbreviation}\isamarkupfalse%
\ {\isachardoublequoteopen}alphaAlg\ N\ G\ b\ n\ {\isacharequal}{\isacharequal}\ Max\ {\isacharparenleft}{\isacharparenleft}proceeds\ b{\isacharparenright}{\isacharbackquote}{\isacharparenleft}set\ {\isacharparenleft}possibleAllocationsAlg{\isadigit{3}}\ {\isacharparenleft}N{\isacharminus}{\isacharbraceleft}n{\isacharbraceright}{\isacharparenright}\ {\isacharparenleft}G{\isacharcolon}{\isacharcolon}{\isacharunderscore}\ list{\isacharparenright}{\isacharparenright}{\isacharparenright}{\isacharparenright}{\isachardoublequoteclose}\isanewline
\isacommand{definition}\isamarkupfalse%
\ {\isachardoublequoteopen}paymentsAlg\ N\ G\ t\ {\isacharequal}{\isacharequal}\ alphaAlg\ N\ G\ {\isacharminus}\ remainingValueAlg\ N\ G\ t{\isachardoublequoteclose}%
\isamarkupsection{Uniform tie breaking: definitions%
}
\isamarkuptrue%
%
\begin{isamarkuptext}%
To each allocation we associate the bid in which each participant bids for a set of goods 
the cardinality of the intersection of that set with the set she gets in the given allocation.
By construction, the revenue of an auction run using this bid is maximal on the given allocation,
and this maximal is unique.
We can then use the bid constructed this way \isa{tiebids{\isacharprime}} to break ties by running an auction 
having the same form as a normal auction (that is why we use the adjective ``uniform''), 
only with this special bid vector.%
\end{isamarkuptext}%
\isamarkuptrue%
\isacommand{abbreviation}\isamarkupfalse%
\ {\isachardoublequoteopen}omega\ pair\ {\isacharequal}{\isacharequal}\ {\isacharbraceleft}fst\ pair{\isacharbraceright}\ {\isasymtimes}\ {\isacharparenleft}finestpart\ {\isacharparenleft}snd\ pair{\isacharparenright}{\isacharparenright}{\isachardoublequoteclose}\isanewline
\isacommand{abbreviation}\isamarkupfalse%
\ {\isachardoublequoteopen}pseudoAllocation\ allocation\ {\isacharequal}{\isacharequal}\ {\isasymUnion}\ {\isacharparenleft}omega\ {\isacharbackquote}\ allocation{\isacharparenright}{\isachardoublequoteclose}\isanewline
\isanewline
\isacommand{abbreviation}\isamarkupfalse%
\ {\isachardoublequoteopen}bidMaximizedBy\ allocation\ N\ G\ {\isacharequal}{\isacharequal}\ \isanewline
{\isacharparenleft}{\isacharasterisk}\ {\isacharparenleft}N\ {\isasymtimes}\ finestpart\ G{\isacharparenright}\ {\isasymtimes}\ {\isacharbraceleft}{\isadigit{0}}{\isacharcolon}{\isacharcolon}price{\isacharbraceright}\ {\isacharplus}{\isacharasterisk}\ {\isacharparenleft}{\isacharparenleft}pseudoAllocation\ allocation{\isacharparenright}\ {\isasymtimes}\ {\isacharbraceleft}{\isadigit{1}}{\isacharbraceright}{\isacharparenright}\ {\isacharasterisk}{\isacharparenright}\isanewline
pseudoAllocation\ allocation\ {\isacharless}{\isacharbar}{\isacharbar}\ {\isacharparenleft}{\isacharparenleft}N\ {\isasymtimes}\ {\isacharparenleft}finestpart\ G{\isacharparenright}{\isacharparenright}{\isacharparenright}{\isachardoublequoteclose}\isanewline
\isacommand{abbreviation}\isamarkupfalse%
\ {\isachardoublequoteopen}maxbid{\isacharprime}\ a\ N\ G\ {\isacharequal}{\isacharequal}\ toFunction\ {\isacharparenleft}bidMaximizedBy\ a\ N\ G{\isacharparenright}{\isachardoublequoteclose}\isanewline
\isacommand{abbreviation}\isamarkupfalse%
\ {\isachardoublequoteopen}partialCompletionOf\ bids\ pair\ {\isacharequal}{\isacharequal}\ {\isacharparenleft}pair{\isacharcomma}\ setsum\ {\isacharparenleft}{\isacharpercent}g{\isachardot}\ bids\ {\isacharparenleft}fst\ pair{\isacharcomma}\ g{\isacharparenright}{\isacharparenright}\ {\isacharparenleft}finestpart\ {\isacharparenleft}snd\ pair{\isacharparenright}{\isacharparenright}{\isacharparenright}{\isachardoublequoteclose}\isanewline
\isacommand{abbreviation}\isamarkupfalse%
\ {\isachardoublequoteopen}test\ bids\ pair\ {\isacharequal}{\isacharequal}\ setsum\ {\isacharparenleft}{\isacharpercent}g{\isachardot}\ bids\ {\isacharparenleft}fst\ pair{\isacharcomma}\ g{\isacharparenright}{\isacharparenright}\ {\isacharparenleft}finestpart\ {\isacharparenleft}snd\ pair{\isacharparenright}{\isacharparenright}{\isachardoublequoteclose}\isanewline
\isacommand{abbreviation}\isamarkupfalse%
\ {\isachardoublequoteopen}LinearCompletion\ bids\ N\ G\ {\isacharequal}{\isacharequal}\ {\isacharparenleft}partialCompletionOf\ bids{\isacharparenright}\ {\isacharbackquote}\ {\isacharparenleft}N\ {\isasymtimes}\ {\isacharparenleft}Pow\ G\ {\isacharminus}\ {\isacharbraceleft}{\isacharbraceleft}{\isacharbraceright}{\isacharbraceright}{\isacharparenright}{\isacharparenright}{\isachardoublequoteclose}\isanewline
\isacommand{abbreviation}\isamarkupfalse%
\ {\isachardoublequoteopen}linearCompletion{\isacharprime}\ bids\ N\ G\ {\isacharequal}{\isacharequal}\ toFunction\ {\isacharparenleft}LinearCompletion\ bids\ N\ G{\isacharparenright}{\isachardoublequoteclose}\isanewline
\isacommand{abbreviation}\isamarkupfalse%
\ {\isachardoublequoteopen}tiebids{\isacharprime}\ a\ N\ G\ {\isacharequal}{\isacharequal}\ linearCompletion{\isacharprime}\ {\isacharparenleft}maxbid{\isacharprime}\ a\ N\ G{\isacharparenright}\ N\ G{\isachardoublequoteclose}\isanewline
\isacommand{abbreviation}\isamarkupfalse%
\ {\isachardoublequoteopen}Tiebids\ a\ N\ G\ {\isacharequal}{\isacharequal}\ LinearCompletion\ {\isacharparenleft}real{\isasymcirc}maxbid{\isacharprime}\ a\ N\ G{\isacharparenright}\ N\ G{\isachardoublequoteclose}\isanewline
\isacommand{abbreviation}\isamarkupfalse%
\ {\isachardoublequoteopen}chosenAllocation{\isacharprime}\ N\ G\ bids\ random\ {\isacharequal}{\isacharequal}\ \isanewline
hd{\isacharparenleft}perm{\isadigit{2}}\ {\isacharparenleft}takeAll\ {\isacharparenleft}{\isacharpercent}x{\isachardot}\ x{\isasymin}{\isacharparenleft}winningAllocationsRel\ N\ {\isacharparenleft}set\ G{\isacharparenright}\ bids{\isacharparenright}{\isacharparenright}\ {\isacharparenleft}possibleAllocationsAlg{\isadigit{3}}\ N\ G{\isacharparenright}{\isacharparenright}\ random{\isacharparenright}{\isachardoublequoteclose}\isanewline
\isacommand{abbreviation}\isamarkupfalse%
\ {\isachardoublequoteopen}resolvingBid{\isacharprime}\ N\ G\ bids\ random\ {\isacharequal}{\isacharequal}\ tiebids{\isacharprime}\ {\isacharparenleft}chosenAllocation{\isacharprime}\ N\ G\ bids\ random{\isacharparenright}\ N\ {\isacharparenleft}set\ G{\isacharparenright}{\isachardoublequoteclose}\isanewline
%
\isadelimtheory
\isanewline
%
\endisadelimtheory
%
\isatagtheory
\isacommand{end}\isamarkupfalse%
%
\endisatagtheory
{\isafoldtheory}%
%
\isadelimtheory
%
\endisadelimtheory
\end{isabellebody}%
%%% Local Variables:
%%% mode: latex
%%% TeX-master: "root"
%%% End:


%
\begin{isabellebody}%
\def\isabellecontext{Universes}%
%
\isamarkupheader{Sets of injections, partitions, allocations expressed as suitable subsets of the corresponding universes%
}
\isamarkuptrue%
%
\isadelimtheory
%
\endisadelimtheory
%
\isatagtheory
\isacommand{theory}\isamarkupfalse%
\ Universes\isanewline
\isanewline
\isakeyword{imports}\ \isanewline
{\isachardoublequoteopen}{\isachartilde}{\isachartilde}{\isacharslash}src{\isacharslash}HOL{\isacharslash}Library{\isacharslash}Code{\isacharunderscore}Target{\isacharunderscore}Nat{\isachardoublequoteclose}\isanewline
StrictCombinatorialAuction\ \isanewline
MiscTools\isanewline
{\isachardoublequoteopen}{\isachartilde}{\isachartilde}{\isacharslash}src{\isacharslash}HOL{\isacharslash}Library{\isacharslash}Indicator{\isacharunderscore}Function{\isachardoublequoteclose}\isanewline
\isanewline
\isakeyword{begin}%
\endisatagtheory
{\isafoldtheory}%
%
\isadelimtheory
%
\endisadelimtheory
%
\isamarkupsection{Preliminary lemmas%
}
\isamarkuptrue%
\isacommand{lemma}\isamarkupfalse%
\ lm{\isadigit{6}}{\isadigit{3}}{\isacharcolon}\ \isakeyword{assumes}\ {\isachardoublequoteopen}Y\ {\isasymin}\ set\ {\isacharparenleft}all{\isacharunderscore}partitions{\isacharunderscore}alg\ X{\isacharparenright}{\isachardoublequoteclose}\ \isakeyword{shows}\ {\isachardoublequoteopen}distinct\ Y{\isachardoublequoteclose}\isanewline
%
\isadelimproof
%
\endisadelimproof
%
\isatagproof
\isacommand{using}\isamarkupfalse%
\ assms\ distinct{\isacharunderscore}sorted{\isacharunderscore}list{\isacharunderscore}of{\isacharunderscore}set\ all{\isacharunderscore}partitions{\isacharunderscore}alg{\isacharunderscore}def\ all{\isacharunderscore}partitions{\isacharunderscore}paper{\isacharunderscore}equiv{\isacharunderscore}alg{\isacharprime}\isanewline
\isacommand{by}\isamarkupfalse%
\ metis%
\endisatagproof
{\isafoldproof}%
%
\isadelimproof
\isanewline
%
\endisadelimproof
\isanewline
\isacommand{lemma}\isamarkupfalse%
\ lm{\isadigit{6}}{\isadigit{5}}{\isacharcolon}\ \isakeyword{assumes}\ {\isachardoublequoteopen}finite\ G{\isachardoublequoteclose}\ \isakeyword{shows}\ {\isachardoublequoteopen}all{\isacharunderscore}partitions\ G\ {\isacharequal}\ set\ {\isacharbackquote}\ {\isacharparenleft}set\ {\isacharparenleft}all{\isacharunderscore}partitions{\isacharunderscore}alg\ G{\isacharparenright}{\isacharparenright}{\isachardoublequoteclose}\isanewline
%
\isadelimproof
%
\endisadelimproof
%
\isatagproof
\isacommand{using}\isamarkupfalse%
\ assms\ lm{\isadigit{6}}{\isadigit{4}}\ all{\isacharunderscore}partitions{\isacharunderscore}alg{\isacharunderscore}def\ all{\isacharunderscore}partitions{\isacharunderscore}paper{\isacharunderscore}equiv{\isacharunderscore}alg\isanewline
distinct{\isacharunderscore}sorted{\isacharunderscore}list{\isacharunderscore}of{\isacharunderscore}set\ image{\isacharunderscore}set\ \isacommand{by}\isamarkupfalse%
\ metis%
\endisatagproof
{\isafoldproof}%
%
\isadelimproof
\isanewline
%
\endisadelimproof
\isanewline
\isacommand{lemma}\isamarkupfalse%
\ \isakeyword{assumes}\ {\isachardoublequoteopen}Y\ {\isasymin}\ set\ {\isacharparenleft}all{\isacharunderscore}partitions{\isacharunderscore}alg\ G{\isacharparenright}{\isachardoublequoteclose}\ {\isachardoublequoteopen}card\ N\ {\isachargreater}\ {\isadigit{0}}{\isachardoublequoteclose}\ {\isachardoublequoteopen}finite\ N{\isachardoublequoteclose}\ {\isachardoublequoteopen}finite\ G{\isachardoublequoteclose}\ \isanewline
\isakeyword{shows}\ {\isachardoublequoteopen}injections\ {\isacharparenleft}set\ Y{\isacharparenright}\ N\ {\isacharequal}\ set\ {\isacharparenleft}injections{\isacharunderscore}alg\ Y\ N{\isacharparenright}{\isachardoublequoteclose}\isanewline
%
\isadelimproof
%
\endisadelimproof
%
\isatagproof
\isacommand{using}\isamarkupfalse%
\ assms\ injections{\isacharunderscore}equiv\ lm{\isadigit{6}}{\isadigit{3}}\ \isacommand{by}\isamarkupfalse%
\ metis%
\endisatagproof
{\isafoldproof}%
%
\isadelimproof
\isanewline
%
\endisadelimproof
\isanewline
\isacommand{lemma}\isamarkupfalse%
\ lm{\isadigit{6}}{\isadigit{7}}{\isacharcolon}\ \isakeyword{assumes}\ {\isachardoublequoteopen}l\ {\isasymin}\ set\ {\isacharparenleft}all{\isacharunderscore}partitions{\isacharunderscore}list\ G{\isacharparenright}{\isachardoublequoteclose}\ {\isachardoublequoteopen}distinct\ G{\isachardoublequoteclose}\ \isakeyword{shows}\ {\isachardoublequoteopen}distinct\ l{\isachardoublequoteclose}\ \isanewline
%
\isadelimproof
%
\endisadelimproof
%
\isatagproof
\isacommand{using}\isamarkupfalse%
\ assms\ all{\isacharunderscore}partitions{\isacharunderscore}list{\isacharunderscore}def\ \isacommand{by}\isamarkupfalse%
\ {\isacharparenleft}metis\ all{\isacharunderscore}partitions{\isacharunderscore}paper{\isacharunderscore}equiv{\isacharunderscore}alg{\isacharprime}{\isacharparenright}%
\endisatagproof
{\isafoldproof}%
%
\isadelimproof
\isanewline
%
\endisadelimproof
\isacommand{lemma}\isamarkupfalse%
\ lm{\isadigit{6}}{\isadigit{8}}{\isacharcolon}\ \isakeyword{assumes}\ {\isachardoublequoteopen}card\ N\ {\isachargreater}\ {\isadigit{0}}{\isachardoublequoteclose}\ {\isachardoublequoteopen}distinct\ G{\isachardoublequoteclose}\ \isakeyword{shows}\ \isanewline
{\isachardoublequoteopen}{\isasymforall}l\ {\isasymin}\ set\ {\isacharparenleft}all{\isacharunderscore}partitions{\isacharunderscore}list\ G{\isacharparenright}{\isachardot}\ set\ {\isacharparenleft}injections{\isacharunderscore}alg\ l\ N{\isacharparenright}\ {\isacharequal}\ injections\ {\isacharparenleft}set\ l{\isacharparenright}\ N{\isachardoublequoteclose}\isanewline
%
\isadelimproof
%
\endisadelimproof
%
\isatagproof
\isacommand{using}\isamarkupfalse%
\ lm{\isadigit{6}}{\isadigit{7}}\ injections{\isacharunderscore}equiv\ assms\ \isacommand{by}\isamarkupfalse%
\ blast%
\endisatagproof
{\isafoldproof}%
%
\isadelimproof
\isanewline
%
\endisadelimproof
\isanewline
\isacommand{lemma}\isamarkupfalse%
\ lm{\isadigit{6}}{\isadigit{9}}{\isacharcolon}\ \isakeyword{assumes}\ {\isachardoublequoteopen}card\ N\ {\isachargreater}\ {\isadigit{0}}{\isachardoublequoteclose}\ {\isachardoublequoteopen}distinct\ G{\isachardoublequoteclose}\isanewline
\isakeyword{shows}\ {\isachardoublequoteopen}{\isacharbraceleft}injections\ P\ N{\isacharbar}\ P{\isachardot}\ P\ {\isasymin}\ all{\isacharunderscore}partitions\ {\isacharparenleft}set\ G{\isacharparenright}{\isacharbraceright}\ {\isacharequal}\isanewline
set\ {\isacharbrackleft}set\ {\isacharparenleft}injections{\isacharunderscore}alg\ l\ N{\isacharparenright}\ {\isachardot}\ l\ {\isasymleftarrow}\ all{\isacharunderscore}partitions{\isacharunderscore}list\ G{\isacharbrackright}{\isachardoublequoteclose}%
\isadelimproof
\ %
\endisadelimproof
%
\isatagproof
\isacommand{using}\isamarkupfalse%
\ assms\ lm{\isadigit{6}}{\isadigit{6}}\ lm{\isadigit{6}}{\isadigit{8}}\ lm{\isadigit{6}}{\isadigit{6}}b\ \isanewline
\isacommand{proof}\isamarkupfalse%
\ {\isacharminus}\isanewline
\ \ \isacommand{let}\isamarkupfalse%
\ {\isacharquery}g{\isadigit{1}}{\isacharequal}all{\isacharunderscore}partitions{\isacharunderscore}list\ \isacommand{let}\isamarkupfalse%
\ {\isacharquery}f{\isadigit{2}}{\isacharequal}injections\ \isacommand{let}\isamarkupfalse%
\ {\isacharquery}g{\isadigit{2}}{\isacharequal}injections{\isacharunderscore}alg\isanewline
\ \ \isacommand{have}\isamarkupfalse%
\ {\isachardoublequoteopen}{\isasymforall}l\ {\isasymin}\ set\ {\isacharparenleft}{\isacharquery}g{\isadigit{1}}\ G{\isacharparenright}{\isachardot}\ set\ {\isacharparenleft}{\isacharquery}g{\isadigit{2}}\ l\ N{\isacharparenright}\ {\isacharequal}\ {\isacharquery}f{\isadigit{2}}\ {\isacharparenleft}set\ l{\isacharparenright}\ N{\isachardoublequoteclose}\ \isacommand{using}\isamarkupfalse%
\ assms\ lm{\isadigit{6}}{\isadigit{8}}\ \isacommand{by}\isamarkupfalse%
\ blast\isanewline
\ \ \isacommand{then}\isamarkupfalse%
\ \isacommand{have}\isamarkupfalse%
\ {\isachardoublequoteopen}set\ {\isacharbrackleft}set\ {\isacharparenleft}{\isacharquery}g{\isadigit{2}}\ l\ N{\isacharparenright}{\isachardot}\ l\ {\isacharless}{\isacharminus}\ {\isacharquery}g{\isadigit{1}}\ G{\isacharbrackright}\ {\isacharequal}\ {\isacharbraceleft}{\isacharquery}f{\isadigit{2}}\ P\ N{\isacharbar}\ P{\isachardot}\ P\ {\isasymin}\ set\ {\isacharparenleft}map\ set\ {\isacharparenleft}{\isacharquery}g{\isadigit{1}}\ G{\isacharparenright}{\isacharparenright}{\isacharbraceright}{\isachardoublequoteclose}\ \isacommand{apply}\isamarkupfalse%
\ {\isacharparenleft}rule\ lm{\isadigit{6}}{\isadigit{6}}{\isacharparenright}\ \isacommand{done}\isamarkupfalse%
\isanewline
\ \ \isacommand{moreover}\isamarkupfalse%
\ \isacommand{have}\isamarkupfalse%
\ {\isachardoublequoteopen}{\isachardot}{\isachardot}{\isachardot}\ {\isacharequal}\ {\isacharbraceleft}{\isacharquery}f{\isadigit{2}}\ P\ N{\isacharbar}\ P{\isachardot}\ P\ {\isasymin}\ all{\isacharunderscore}partitions\ {\isacharparenleft}set\ G{\isacharparenright}{\isacharbraceright}{\isachardoublequoteclose}\ \isacommand{using}\isamarkupfalse%
\ all{\isacharunderscore}partitions{\isacharunderscore}paper{\isacharunderscore}equiv{\isacharunderscore}alg\isanewline
\ \ assms\ \isacommand{by}\isamarkupfalse%
\ blast\isanewline
\ \ \isacommand{ultimately}\isamarkupfalse%
\ \isacommand{show}\isamarkupfalse%
\ {\isacharquery}thesis\ \isacommand{by}\isamarkupfalse%
\ presburger\isanewline
\isacommand{qed}\isamarkupfalse%
%
\endisatagproof
{\isafoldproof}%
%
\isadelimproof
%
\endisadelimproof
\isanewline
\isanewline
\isacommand{lemma}\isamarkupfalse%
\ lm{\isadigit{7}}{\isadigit{0}}{\isacharcolon}\ \isakeyword{assumes}\ {\isachardoublequoteopen}card\ N\ {\isachargreater}\ {\isadigit{0}}{\isachardoublequoteclose}\ {\isachardoublequoteopen}distinct\ G{\isachardoublequoteclose}\ \isakeyword{shows}\ \isanewline
{\isachardoublequoteopen}Union\ {\isacharbraceleft}injections\ P\ N{\isacharbar}\ P{\isachardot}\ P\ {\isasymin}\ all{\isacharunderscore}partitions\ {\isacharparenleft}set\ G{\isacharparenright}{\isacharbraceright}\ {\isacharequal}\isanewline
Union\ {\isacharparenleft}set\ {\isacharbrackleft}set\ {\isacharparenleft}injections{\isacharunderscore}alg\ l\ N{\isacharparenright}\ {\isachardot}\ l\ {\isasymleftarrow}\ all{\isacharunderscore}partitions{\isacharunderscore}list\ G{\isacharbrackright}{\isacharparenright}{\isachardoublequoteclose}\ {\isacharparenleft}\isakeyword{is}\ {\isachardoublequoteopen}Union\ {\isacharquery}L\ {\isacharequal}\ Union\ {\isacharquery}R{\isachardoublequoteclose}{\isacharparenright}\isanewline
%
\isadelimproof
%
\endisadelimproof
%
\isatagproof
\isacommand{proof}\isamarkupfalse%
\ {\isacharminus}\ \isacommand{have}\isamarkupfalse%
\ {\isachardoublequoteopen}{\isacharquery}L\ {\isacharequal}\ {\isacharquery}R{\isachardoublequoteclose}\ \isacommand{using}\isamarkupfalse%
\ assms\ \isacommand{by}\isamarkupfalse%
\ {\isacharparenleft}rule\ lm{\isadigit{6}}{\isadigit{9}}{\isacharparenright}\ \isacommand{thus}\isamarkupfalse%
\ {\isacharquery}thesis\ \isacommand{by}\isamarkupfalse%
\ presburger\ \isacommand{qed}\isamarkupfalse%
%
\endisatagproof
{\isafoldproof}%
%
\isadelimproof
\isanewline
%
\endisadelimproof
\isanewline
\isacommand{corollary}\isamarkupfalse%
\ lm{\isadigit{7}}{\isadigit{0}}b{\isacharcolon}\ \isakeyword{assumes}\ {\isachardoublequoteopen}card\ N\ {\isachargreater}\ {\isadigit{0}}{\isachardoublequoteclose}\ {\isachardoublequoteopen}distinct\ G{\isachardoublequoteclose}\ \isakeyword{shows}\ \isanewline
{\isachardoublequoteopen}possibleAllocationsRel\ N\ {\isacharparenleft}set\ G{\isacharparenright}\ {\isacharequal}\ possibleAllocationsAlg{\isadigit{2}}\ N\ G{\isachardoublequoteclose}\ {\isacharparenleft}\isakeyword{is}\ {\isachardoublequoteopen}{\isacharquery}L\ {\isacharequal}\ {\isacharquery}R{\isachardoublequoteclose}{\isacharparenright}%
\isadelimproof
\ %
\endisadelimproof
%
\isatagproof
\isacommand{using}\isamarkupfalse%
\ assms\ lm{\isadigit{7}}{\isadigit{0}}\ \isanewline
possible{\isacharunderscore}allocations{\isacharunderscore}rel{\isacharunderscore}def\ \isanewline
\isacommand{proof}\isamarkupfalse%
\ {\isacharminus}\isanewline
\ \ \isacommand{let}\isamarkupfalse%
\ {\isacharquery}LL{\isacharequal}{\isachardoublequoteopen}{\isasymUnion}\ {\isacharbraceleft}injections\ P\ N{\isacharbar}\ P{\isachardot}\ P\ {\isasymin}\ all{\isacharunderscore}partitions\ {\isacharparenleft}set\ G{\isacharparenright}{\isacharbraceright}{\isachardoublequoteclose}\isanewline
\ \ \isacommand{let}\isamarkupfalse%
\ {\isacharquery}RR{\isacharequal}{\isachardoublequoteopen}{\isasymUnion}\ {\isacharparenleft}set\ {\isacharbrackleft}set\ {\isacharparenleft}injections{\isacharunderscore}alg\ l\ N{\isacharparenright}\ {\isachardot}\ l\ {\isasymleftarrow}\ all{\isacharunderscore}partitions{\isacharunderscore}list\ G{\isacharbrackright}{\isacharparenright}{\isachardoublequoteclose}\isanewline
\ \ \isacommand{have}\isamarkupfalse%
\ {\isachardoublequoteopen}{\isacharquery}LL\ {\isacharequal}\ {\isacharquery}RR{\isachardoublequoteclose}\ \isacommand{using}\isamarkupfalse%
\ assms\ \isacommand{apply}\isamarkupfalse%
\ {\isacharparenleft}rule\ lm{\isadigit{7}}{\isadigit{0}}{\isacharparenright}\ \isacommand{done}\isamarkupfalse%
\isanewline
\ \ \isacommand{then}\isamarkupfalse%
\ \isacommand{have}\isamarkupfalse%
\ {\isachardoublequoteopen}converse\ {\isacharbackquote}\ {\isacharquery}LL\ {\isacharequal}\ converse\ {\isacharbackquote}\ {\isacharquery}RR{\isachardoublequoteclose}\ \isacommand{by}\isamarkupfalse%
\ presburger\isanewline
\ \ \isacommand{thus}\isamarkupfalse%
\ {\isacharquery}thesis\ \isacommand{using}\isamarkupfalse%
\ possible{\isacharunderscore}allocations{\isacharunderscore}rel{\isacharunderscore}def\ \isacommand{by}\isamarkupfalse%
\ force\isanewline
\isacommand{qed}\isamarkupfalse%
%
\endisatagproof
{\isafoldproof}%
%
\isadelimproof
%
\endisadelimproof
%
\isamarkupsection{Definitions of various subsets of \isa{UNIV}.%
}
\isamarkuptrue%
\isacommand{abbreviation}\isamarkupfalse%
\ {\isachardoublequoteopen}isChoice\ R\ {\isacharequal}{\isacharequal}\ {\isasymforall}x{\isachardot}\ R{\isacharbackquote}{\isacharbackquote}{\isacharbraceleft}x{\isacharbraceright}\ {\isasymsubseteq}\ x{\isachardoublequoteclose}\isanewline
\isacommand{abbreviation}\isamarkupfalse%
\ {\isachardoublequoteopen}dualOutside\ R\ Y\ {\isacharequal}{\isacharequal}\ R\ {\isacharminus}\ {\isacharparenleft}Domain\ R\ {\isasymtimes}\ Y{\isacharparenright}{\isachardoublequoteclose}\isanewline
\isacommand{notation}\isamarkupfalse%
\ dualOutside\ {\isacharparenleft}\isakeyword{infix}\ {\isachardoublequoteopen}{\isacharbar}{\isacharminus}{\isachardoublequoteclose}\ {\isadigit{7}}{\isadigit{5}}{\isacharparenright}\isanewline
\isacommand{notation}\isamarkupfalse%
\ Outside\ {\isacharparenleft}\isakeyword{infix}\ {\isachardoublequoteopen}{\isacharminus}{\isacharbar}{\isachardoublequoteclose}\ {\isadigit{7}}{\isadigit{5}}{\isacharparenright}\isanewline
\isanewline
\isacommand{abbreviation}\isamarkupfalse%
\ {\isachardoublequoteopen}partitionsUniverse\ {\isacharequal}{\isacharequal}\ {\isacharbraceleft}X{\isachardot}\ is{\isacharunderscore}partition\ X{\isacharbraceright}{\isachardoublequoteclose}\isanewline
\isacommand{lemma}\isamarkupfalse%
\ {\isachardoublequoteopen}partitionsUniverse\ {\isasymsubseteq}\ Pow\ UNIV{\isachardoublequoteclose}%
\isadelimproof
\ %
\endisadelimproof
%
\isatagproof
\isacommand{by}\isamarkupfalse%
\ simp%
\endisatagproof
{\isafoldproof}%
%
\isadelimproof
%
\endisadelimproof
\ \ \ \ \ \ \ \isanewline
\isacommand{abbreviation}\isamarkupfalse%
\ {\isachardoublequoteopen}partitionValuedUniverse\ {\isacharequal}{\isacharequal}\ {\isasymUnion}\ P\ {\isasymin}\ partitionsUniverse{\isachardot}\ Pow\ {\isacharparenleft}UNIV\ {\isasymtimes}\ P{\isacharparenright}{\isachardoublequoteclose}\isanewline
\isacommand{lemma}\isamarkupfalse%
\ {\isachardoublequoteopen}partitionValuedUniverse\ {\isasymsubseteq}\ Pow\ {\isacharparenleft}UNIV\ {\isasymtimes}\ {\isacharparenleft}Pow\ UNIV{\isacharparenright}{\isacharparenright}{\isachardoublequoteclose}%
\isadelimproof
\ %
\endisadelimproof
%
\isatagproof
\isacommand{by}\isamarkupfalse%
\ simp%
\endisatagproof
{\isafoldproof}%
%
\isadelimproof
%
\endisadelimproof
\isanewline
\isacommand{abbreviation}\isamarkupfalse%
\ {\isachardoublequoteopen}injectionsUniverse\ {\isacharequal}{\isacharequal}\ {\isacharbraceleft}R{\isachardot}\ {\isacharparenleft}runiq\ R{\isacharparenright}\ {\isacharampersand}\ {\isacharparenleft}runiq\ {\isacharparenleft}R{\isacharcircum}{\isacharminus}{\isadigit{1}}{\isacharparenright}{\isacharparenright}{\isacharbraceright}{\isachardoublequoteclose}\isanewline
\isacommand{abbreviation}\isamarkupfalse%
\ {\isachardoublequoteopen}allocationsUniverse\ {\isacharequal}{\isacharequal}\ injectionsUniverse\ {\isasyminter}\ partitionValuedUniverse{\isachardoublequoteclose}\isanewline
\isacommand{abbreviation}\isamarkupfalse%
\ {\isachardoublequoteopen}totalRels\ X\ Y\ {\isacharequal}{\isacharequal}\ {\isacharbraceleft}R{\isachardot}\ Domain\ R\ {\isacharequal}\ X\ {\isacharampersand}\ Range\ R\ {\isasymsubseteq}\ Y{\isacharbraceright}{\isachardoublequoteclose}\isanewline
\isacommand{abbreviation}\isamarkupfalse%
\ {\isachardoublequoteopen}strictCovers\ G\ {\isacharequal}{\isacharequal}\ Union\ {\isacharminus}{\isacharbackquote}\ {\isacharbraceleft}G{\isacharbraceright}{\isachardoublequoteclose}%
\isamarkupsection{Results about the sets defined in the previous section%
}
\isamarkuptrue%
\isacommand{lemma}\isamarkupfalse%
\ lm{\isadigit{0}}{\isadigit{1}}a{\isacharcolon}\ {\isachardoublequoteopen}partitionsUniverse\ {\isasymsubseteq}\ \ {\isacharbraceleft}P{\isacharminus}{\isacharbraceleft}{\isacharbraceleft}{\isacharbraceright}{\isacharbraceright}{\isacharbar}\ P{\isachardot}\ {\isasymInter}P\ {\isasymin}\ {\isacharbraceleft}{\isasymUnion}P{\isacharcomma}{\isacharbraceleft}{\isacharbraceright}{\isacharbraceright}{\isacharbraceright}{\isachardoublequoteclose}%
\isadelimproof
\ %
\endisadelimproof
%
\isatagproof
\isacommand{unfolding}\isamarkupfalse%
\ is{\isacharunderscore}partition{\isacharunderscore}def\ \isacommand{by}\isamarkupfalse%
\ auto%
\endisatagproof
{\isafoldproof}%
%
\isadelimproof
%
\endisadelimproof
\isanewline
\isacommand{lemma}\isamarkupfalse%
\ lm{\isadigit{0}}{\isadigit{4}}{\isacharcolon}\ \isakeyword{assumes}\ {\isachardoublequoteopen}{\isacharbang}x{\isadigit{1}}\ {\isacharcolon}\ X{\isachardot}\ {\isacharparenleft}x{\isadigit{1}}\ {\isasymnoteq}\ {\isacharbraceleft}{\isacharbraceright}\ {\isacharampersand}\ {\isacharparenleft}{\isacharbang}\ x{\isadigit{2}}\ {\isacharcolon}\ X{\isacharminus}{\isacharbraceleft}x{\isadigit{1}}{\isacharbraceright}{\isachardot}\ x{\isadigit{1}}\ {\isasyminter}\ x{\isadigit{2}}{\isacharequal}{\isacharbraceleft}{\isacharbraceright}{\isacharparenright}{\isacharparenright}{\isachardoublequoteclose}\ \isakeyword{shows}\ {\isachardoublequoteopen}is{\isacharunderscore}partition\ X{\isachardoublequoteclose}\ \isanewline
%
\isadelimproof
%
\endisadelimproof
%
\isatagproof
\isacommand{unfolding}\isamarkupfalse%
\ is{\isacharunderscore}partition{\isacharunderscore}def\ \isacommand{using}\isamarkupfalse%
\ assms\ \isacommand{by}\isamarkupfalse%
\ fast%
\endisatagproof
{\isafoldproof}%
%
\isadelimproof
\isanewline
%
\endisadelimproof
\isacommand{lemma}\isamarkupfalse%
\ lm{\isadigit{7}}{\isadigit{2}}{\isacharcolon}\ \isakeyword{assumes}\ {\isachardoublequoteopen}{\isasymforall}x\ {\isasymin}\ X{\isachardot}\ t\ x\ {\isasymin}\ x{\isachardoublequoteclose}\ \isakeyword{shows}\ {\isachardoublequoteopen}isChoice\ {\isacharparenleft}graph\ X\ t{\isacharparenright}{\isachardoublequoteclose}%
\isadelimproof
\ %
\endisadelimproof
%
\isatagproof
\isacommand{using}\isamarkupfalse%
\ assms\isanewline
\isacommand{by}\isamarkupfalse%
\ {\isacharparenleft}metis\ Image{\isacharunderscore}within{\isacharunderscore}domain{\isacharprime}\ empty{\isacharunderscore}subsetI\ insert{\isacharunderscore}subset\ ll{\isadigit{3}}{\isadigit{3}}\ ll{\isadigit{3}}{\isadigit{7}}\ runiq{\isacharunderscore}wrt{\isacharunderscore}eval{\isacharunderscore}rel\ subset{\isacharunderscore}trans{\isacharparenright}%
\endisatagproof
{\isafoldproof}%
%
\isadelimproof
%
\endisadelimproof
\isanewline
\isanewline
\isacommand{lemma}\isamarkupfalse%
\ {\isachardoublequoteopen}R\ {\isacharbar}{\isacharminus}\ Y\ {\isacharequal}\ {\isacharparenleft}R{\isacharcircum}{\isacharminus}{\isadigit{1}}\ {\isacharminus}{\isacharbar}\ Y{\isacharparenright}{\isacharcircum}{\isacharminus}{\isadigit{1}}{\isachardoublequoteclose}%
\isadelimproof
\ %
\endisadelimproof
%
\isatagproof
\isacommand{using}\isamarkupfalse%
\ Outside{\isacharunderscore}def\ \isacommand{by}\isamarkupfalse%
\ auto%
\endisatagproof
{\isafoldproof}%
%
\isadelimproof
%
\endisadelimproof
\isanewline
\isacommand{lemma}\isamarkupfalse%
\ lm{\isadigit{2}}{\isadigit{4}}{\isacharcolon}\ {\isachardoublequoteopen}injections{\isacharprime}\ X\ Y\ {\isacharequal}\ injections\ X\ Y{\isachardoublequoteclose}%
\isadelimproof
\ %
\endisadelimproof
%
\isatagproof
\isacommand{using}\isamarkupfalse%
\ injections{\isacharunderscore}def\ \isacommand{by}\isamarkupfalse%
\ metis%
\endisatagproof
{\isafoldproof}%
%
\isadelimproof
%
\endisadelimproof
\isanewline
\isacommand{lemma}\isamarkupfalse%
\ lm{\isadigit{2}}{\isadigit{5}}{\isacharcolon}\ {\isachardoublequoteopen}injections{\isacharprime}\ X\ Y\ {\isasymsubseteq}\ injectionsUniverse{\isachardoublequoteclose}%
\isadelimproof
\ %
\endisadelimproof
%
\isatagproof
\isacommand{by}\isamarkupfalse%
\ fast%
\endisatagproof
{\isafoldproof}%
%
\isadelimproof
%
\endisadelimproof
\isanewline
\isacommand{lemma}\isamarkupfalse%
\ lm{\isadigit{2}}{\isadigit{5}}b{\isacharcolon}\ {\isachardoublequoteopen}injections\ X\ Y\ {\isasymsubseteq}\ injectionsUniverse{\isachardoublequoteclose}%
\isadelimproof
\ %
\endisadelimproof
%
\isatagproof
\isacommand{using}\isamarkupfalse%
\ injections{\isacharunderscore}def\ \isacommand{by}\isamarkupfalse%
\ blast%
\endisatagproof
{\isafoldproof}%
%
\isadelimproof
%
\endisadelimproof
\isanewline
\isacommand{lemma}\isamarkupfalse%
\ lm{\isadigit{2}}{\isadigit{6}}{\isacharcolon}\ {\isachardoublequoteopen}injections{\isacharprime}\ X\ Y\ {\isacharequal}\ totalRels\ X\ Y\ {\isasyminter}\ injectionsUniverse{\isachardoublequoteclose}%
\isadelimproof
\ %
\endisadelimproof
%
\isatagproof
\isacommand{by}\isamarkupfalse%
\ fastforce%
\endisatagproof
{\isafoldproof}%
%
\isadelimproof
%
\endisadelimproof
\isanewline
\isanewline
\isacommand{lemma}\isamarkupfalse%
\ lm{\isadigit{4}}{\isadigit{7}}{\isacharcolon}\ \isakeyword{assumes}\ {\isachardoublequoteopen}a\ {\isasymin}\ possibleAllocationsRel\ N\ G{\isachardoublequoteclose}\ \isakeyword{shows}\ \isanewline
{\isachardoublequoteopen}a{\isacharcircum}{\isacharminus}{\isadigit{1}}\ {\isasymin}\ injections\ {\isacharparenleft}Range\ a{\isacharparenright}\ N\ {\isacharampersand}\ Range\ a\ partitions\ G\ {\isacharampersand}\ Domain\ a\ {\isasymsubseteq}\ N{\isachardoublequoteclose}\ \isanewline
%
\isadelimproof
%
\endisadelimproof
%
\isatagproof
\isacommand{unfolding}\isamarkupfalse%
\ injections{\isacharunderscore}def\ \isacommand{using}\isamarkupfalse%
\ assms\ all{\isacharunderscore}partitions{\isacharunderscore}def\ injections{\isacharunderscore}def\ \isacommand{by}\isamarkupfalse%
\ fastforce%
\endisatagproof
{\isafoldproof}%
%
\isadelimproof
\isanewline
%
\endisadelimproof
\isanewline
\isacommand{lemma}\isamarkupfalse%
\ lll{\isadigit{8}}{\isadigit{0}}{\isacharcolon}\ \isakeyword{assumes}\ {\isachardoublequoteopen}is{\isacharunderscore}partition\ XX{\isachardoublequoteclose}\ {\isachardoublequoteopen}YY\ {\isasymsubseteq}\ XX{\isachardoublequoteclose}\ \isakeyword{shows}\ {\isachardoublequoteopen}{\isacharparenleft}XX\ {\isacharminus}\ YY{\isacharparenright}\ partitions\ {\isacharparenleft}{\isasymUnion}\ XX\ {\isacharminus}\ {\isasymUnion}\ YY{\isacharparenright}{\isachardoublequoteclose}\isanewline
%
\isadelimproof
%
\endisadelimproof
%
\isatagproof
\isacommand{using}\isamarkupfalse%
\ is{\isacharunderscore}partition{\isacharunderscore}of{\isacharunderscore}def\ is{\isacharunderscore}partition{\isacharunderscore}def\ assms\isanewline
\isacommand{proof}\isamarkupfalse%
\ {\isacharminus}\isanewline
\ \ \isacommand{let}\isamarkupfalse%
\ {\isacharquery}xx{\isacharequal}{\isachardoublequoteopen}XX\ {\isacharminus}\ YY{\isachardoublequoteclose}\ \isacommand{let}\isamarkupfalse%
\ {\isacharquery}X{\isacharequal}{\isachardoublequoteopen}{\isasymUnion}\ XX{\isachardoublequoteclose}\ \isacommand{let}\isamarkupfalse%
\ {\isacharquery}Y{\isacharequal}{\isachardoublequoteopen}{\isasymUnion}\ YY{\isachardoublequoteclose}\isanewline
\ \ \isacommand{let}\isamarkupfalse%
\ {\isacharquery}x{\isacharequal}{\isachardoublequoteopen}{\isacharquery}X\ {\isacharminus}\ {\isacharquery}Y{\isachardoublequoteclose}\isanewline
\ \ \isacommand{have}\isamarkupfalse%
\ {\isachardoublequoteopen}{\isasymforall}\ y\ {\isasymin}\ YY{\isachardot}\ {\isasymforall}\ x{\isasymin}{\isacharquery}xx{\isachardot}\ y\ {\isasyminter}\ x{\isacharequal}{\isacharbraceleft}{\isacharbraceright}{\isachardoublequoteclose}\ \isacommand{using}\isamarkupfalse%
\ assms\ is{\isacharunderscore}partition{\isacharunderscore}def\ \isacommand{by}\isamarkupfalse%
\ {\isacharparenleft}metis\ Diff{\isacharunderscore}iff\ set{\isacharunderscore}rev{\isacharunderscore}mp{\isacharparenright}\isanewline
\ \ \isacommand{then}\isamarkupfalse%
\ \isacommand{have}\isamarkupfalse%
\ {\isachardoublequoteopen}{\isasymUnion}\ {\isacharquery}xx\ {\isasymsubseteq}\ {\isacharquery}x{\isachardoublequoteclose}\ \isacommand{using}\isamarkupfalse%
\ assms\ \isacommand{by}\isamarkupfalse%
\ blast\isanewline
\ \ \isacommand{then}\isamarkupfalse%
\ \isacommand{have}\isamarkupfalse%
\ {\isachardoublequoteopen}{\isasymUnion}\ {\isacharquery}xx\ {\isacharequal}\ {\isacharquery}x{\isachardoublequoteclose}\ \isacommand{by}\isamarkupfalse%
\ blast\isanewline
\ \ \isacommand{moreover}\isamarkupfalse%
\ \isacommand{have}\isamarkupfalse%
\ {\isachardoublequoteopen}is{\isacharunderscore}partition\ {\isacharquery}xx{\isachardoublequoteclose}\ \isacommand{using}\isamarkupfalse%
\ subset{\isacharunderscore}is{\isacharunderscore}partition\ \isacommand{by}\isamarkupfalse%
\ {\isacharparenleft}metis\ Diff{\isacharunderscore}subset\ assms{\isacharparenleft}{\isadigit{1}}{\isacharparenright}{\isacharparenright}\isanewline
\ \ \isacommand{ultimately}\isamarkupfalse%
\isanewline
\ \ \isacommand{show}\isamarkupfalse%
\ {\isacharquery}thesis\ \isacommand{using}\isamarkupfalse%
\ is{\isacharunderscore}partition{\isacharunderscore}of{\isacharunderscore}def\ \isacommand{by}\isamarkupfalse%
\ blast\isanewline
\isacommand{qed}\isamarkupfalse%
%
\endisatagproof
{\isafoldproof}%
%
\isadelimproof
\isanewline
%
\endisadelimproof
\isanewline
\isacommand{lemma}\isamarkupfalse%
\ lll{\isadigit{8}}{\isadigit{1}}a{\isacharcolon}\ \isakeyword{assumes}\ {\isachardoublequoteopen}a\ {\isasymin}\ possible{\isacharunderscore}allocations{\isacharunderscore}rel\ G\ N{\isachardoublequoteclose}\ \isakeyword{shows}\isanewline
{\isachardoublequoteopen}runiq\ a\ {\isacharampersand}\ runiq\ {\isacharparenleft}a{\isasyminverse}{\isacharparenright}\ {\isacharampersand}\ {\isacharparenleft}Domain\ a{\isacharparenright}\ partitions\ G\ {\isacharampersand}\ Range\ a\ {\isasymsubseteq}\ N{\isachardoublequoteclose}\ \isanewline
%
\isadelimproof
%
\endisadelimproof
%
\isatagproof
\isacommand{proof}\isamarkupfalse%
\ {\isacharminus}\isanewline
\ \ \isacommand{obtain}\isamarkupfalse%
\ Y\ \isakeyword{where}\isanewline
\ \ {\isadigit{0}}{\isacharcolon}\ {\isachardoublequoteopen}a\ {\isasymin}\ injections\ Y\ N\ {\isacharampersand}\ Y\ {\isasymin}\ all{\isacharunderscore}partitions\ G{\isachardoublequoteclose}\ \isacommand{using}\isamarkupfalse%
\ assms\ possible{\isacharunderscore}allocations{\isacharunderscore}rel{\isacharunderscore}def\ \isacommand{by}\isamarkupfalse%
\ auto\isanewline
\ \ \isacommand{show}\isamarkupfalse%
\ {\isacharquery}thesis\ \isacommand{using}\isamarkupfalse%
\ {\isadigit{0}}\ injections{\isacharunderscore}def\ all{\isacharunderscore}partitions{\isacharunderscore}def\ mem{\isacharunderscore}Collect{\isacharunderscore}eq\ \isacommand{by}\isamarkupfalse%
\ fastforce\isanewline
\isacommand{qed}\isamarkupfalse%
%
\endisatagproof
{\isafoldproof}%
%
\isadelimproof
\isanewline
%
\endisadelimproof
\isanewline
\isacommand{lemma}\isamarkupfalse%
\ lll{\isadigit{8}}{\isadigit{1}}b{\isacharcolon}\ \isakeyword{assumes}\ {\isachardoublequoteopen}runiq\ a{\isachardoublequoteclose}\ {\isachardoublequoteopen}runiq\ {\isacharparenleft}a{\isasyminverse}{\isacharparenright}{\isachardoublequoteclose}\ {\isachardoublequoteopen}{\isacharparenleft}Domain\ a{\isacharparenright}\ partitions\ G{\isachardoublequoteclose}\ {\isachardoublequoteopen}Range\ a\ {\isasymsubseteq}\ N{\isachardoublequoteclose}\isanewline
\isakeyword{shows}\ {\isachardoublequoteopen}a\ {\isasymin}\ possible{\isacharunderscore}allocations{\isacharunderscore}rel\ G\ N{\isachardoublequoteclose}\isanewline
%
\isadelimproof
%
\endisadelimproof
%
\isatagproof
\isacommand{proof}\isamarkupfalse%
\ {\isacharminus}\isanewline
\ \ \isacommand{have}\isamarkupfalse%
\ {\isachardoublequoteopen}a\ {\isasymin}\ injections\ {\isacharparenleft}Domain\ a{\isacharparenright}\ N{\isachardoublequoteclose}\ \isacommand{unfolding}\isamarkupfalse%
\ injections{\isacharunderscore}def\ \isacommand{using}\isamarkupfalse%
\ assms{\isacharparenleft}{\isadigit{1}}{\isacharparenright}\ assms{\isacharparenleft}{\isadigit{2}}{\isacharparenright}\ \ assms{\isacharparenleft}{\isadigit{4}}{\isacharparenright}\ \isacommand{by}\isamarkupfalse%
\ blast\isanewline
\ \ \isacommand{moreover}\isamarkupfalse%
\ \isacommand{have}\isamarkupfalse%
\ {\isachardoublequoteopen}Domain\ a\ {\isasymin}\ all{\isacharunderscore}partitions\ G{\isachardoublequoteclose}\ \isacommand{using}\isamarkupfalse%
\ assms{\isacharparenleft}{\isadigit{3}}{\isacharparenright}\ all{\isacharunderscore}partitions{\isacharunderscore}def\ \isacommand{by}\isamarkupfalse%
\ fast\isanewline
\ \ \isacommand{ultimately}\isamarkupfalse%
\ \isacommand{show}\isamarkupfalse%
\ {\isacharquery}thesis\ \isacommand{using}\isamarkupfalse%
\ assms{\isacharparenleft}{\isadigit{1}}{\isacharparenright}\ possible{\isacharunderscore}allocations{\isacharunderscore}rel{\isacharunderscore}def\ \isacommand{by}\isamarkupfalse%
\ auto\isanewline
\isacommand{qed}\isamarkupfalse%
%
\endisatagproof
{\isafoldproof}%
%
\isadelimproof
\isanewline
%
\endisadelimproof
\isanewline
\isacommand{lemma}\isamarkupfalse%
\ lll{\isadigit{8}}{\isadigit{1}}{\isacharcolon}\ {\isachardoublequoteopen}a\ {\isasymin}\ possible{\isacharunderscore}allocations{\isacharunderscore}rel\ G\ N\ {\isasymlongleftrightarrow}\isanewline
runiq\ a\ {\isacharampersand}\ runiq\ {\isacharparenleft}a{\isasyminverse}{\isacharparenright}\ {\isacharampersand}\ {\isacharparenleft}Domain\ a{\isacharparenright}\ partitions\ G\ {\isacharampersand}\ Range\ a\ {\isasymsubseteq}\ N{\isachardoublequoteclose}\isanewline
%
\isadelimproof
%
\endisadelimproof
%
\isatagproof
\isacommand{using}\isamarkupfalse%
\ lll{\isadigit{8}}{\isadigit{1}}a\ lll{\isadigit{8}}{\isadigit{1}}b\ \isacommand{by}\isamarkupfalse%
\ blast%
\endisatagproof
{\isafoldproof}%
%
\isadelimproof
\isanewline
%
\endisadelimproof
\isanewline
\isacommand{corollary}\isamarkupfalse%
\ \isakeyword{assumes}\ {\isachardoublequoteopen}runiq\ {\isacharparenleft}P{\isacharcircum}{\isacharminus}{\isadigit{1}}{\isacharparenright}{\isachardoublequoteclose}\ \isakeyword{shows}\ {\isachardoublequoteopen}Range\ {\isacharparenleft}P\ outside\ X{\isacharparenright}\ {\isasyminter}\ Range\ {\isacharparenleft}P\ {\isacharbar}{\isacharbar}\ X{\isacharparenright}{\isacharequal}{\isacharbraceleft}{\isacharbraceright}{\isachardoublequoteclose}\isanewline
%
\isadelimproof
%
\endisadelimproof
%
\isatagproof
\isacommand{using}\isamarkupfalse%
\ assms\ lll{\isadigit{7}}{\isadigit{8}}\ \isacommand{by}\isamarkupfalse%
\ {\isacharparenleft}metis\ lll{\isadigit{0}}{\isadigit{1}}\ lll{\isadigit{8}}{\isadigit{5}}{\isacharparenright}%
\endisatagproof
{\isafoldproof}%
%
\isadelimproof
\isanewline
%
\endisadelimproof
\isanewline
\isacommand{lemma}\isamarkupfalse%
\ lm{\isadigit{1}}{\isadigit{0}}{\isacharcolon}\ {\isachardoublequoteopen}possible{\isacharunderscore}allocations{\isacharunderscore}rel{\isacharprime}\ G\ N\ {\isasymsubseteq}\ injectionsUniverse{\isachardoublequoteclose}\isanewline
%
\isadelimproof
%
\endisadelimproof
%
\isatagproof
\isacommand{using}\isamarkupfalse%
\ assms\ \isacommand{by}\isamarkupfalse%
\ force%
\endisatagproof
{\isafoldproof}%
%
\isadelimproof
\isanewline
%
\endisadelimproof
\isanewline
\isacommand{lemma}\isamarkupfalse%
\ lm{\isadigit{0}}{\isadigit{9}}{\isacharcolon}\ {\isachardoublequoteopen}possible{\isacharunderscore}allocations{\isacharunderscore}rel\ G\ N\ {\isasymsubseteq}\ {\isacharbraceleft}a{\isachardot}\ Range\ a\ {\isasymsubseteq}\ N\ {\isacharampersand}\ Domain\ a\ {\isasymin}\ all{\isacharunderscore}partitions\ G{\isacharbraceright}{\isachardoublequoteclose}\isanewline
%
\isadelimproof
%
\endisadelimproof
%
\isatagproof
\isacommand{using}\isamarkupfalse%
\ assms\ possible{\isacharunderscore}allocations{\isacharunderscore}rel{\isacharunderscore}def\ injections{\isacharunderscore}def\ \isacommand{by}\isamarkupfalse%
\ fastforce%
\endisatagproof
{\isafoldproof}%
%
\isadelimproof
\isanewline
%
\endisadelimproof
\isanewline
\isacommand{lemma}\isamarkupfalse%
\ lm{\isadigit{1}}{\isadigit{1}}{\isacharcolon}\ {\isachardoublequoteopen}injections\ X\ Y\ {\isacharequal}\ injections{\isacharprime}\ X\ Y{\isachardoublequoteclose}%
\isadelimproof
\ %
\endisadelimproof
%
\isatagproof
\isacommand{using}\isamarkupfalse%
\ injections{\isacharunderscore}def\isanewline
\isacommand{by}\isamarkupfalse%
\ metis%
\endisatagproof
{\isafoldproof}%
%
\isadelimproof
%
\endisadelimproof
\isanewline
\isanewline
\isacommand{lemma}\isamarkupfalse%
\ lm{\isadigit{1}}{\isadigit{2}}{\isacharcolon}\ {\isachardoublequoteopen}all{\isacharunderscore}partitions\ X\ {\isacharequal}\ all{\isacharunderscore}partitions{\isacharprime}\ X{\isachardoublequoteclose}%
\isadelimproof
\ %
\endisadelimproof
%
\isatagproof
\isacommand{using}\isamarkupfalse%
\ all{\isacharunderscore}partitions{\isacharunderscore}def\ is{\isacharunderscore}partition{\isacharunderscore}of{\isacharunderscore}def\ \isanewline
\isacommand{by}\isamarkupfalse%
\ auto%
\endisatagproof
{\isafoldproof}%
%
\isadelimproof
%
\endisadelimproof
\isanewline
\isanewline
\isacommand{lemma}\isamarkupfalse%
\ lm{\isadigit{1}}{\isadigit{3}}{\isacharcolon}\ {\isachardoublequoteopen}possible{\isacharunderscore}allocations{\isacharunderscore}rel{\isacharprime}\ A\ B\ {\isacharequal}\ possible{\isacharunderscore}allocations{\isacharunderscore}rel\ A\ B{\isachardoublequoteclose}\ {\isacharparenleft}\isakeyword{is}\ {\isachardoublequoteopen}{\isacharquery}A{\isacharequal}{\isacharquery}B{\isachardoublequoteclose}{\isacharparenright}\isanewline
%
\isadelimproof
%
\endisadelimproof
%
\isatagproof
\isacommand{proof}\isamarkupfalse%
\ {\isacharminus}\isanewline
\ \ \isacommand{have}\isamarkupfalse%
\ {\isachardoublequoteopen}{\isacharquery}B{\isacharequal}{\isasymUnion}\ {\isacharbraceleft}\ injections\ Y\ B\ {\isacharbar}\ Y\ {\isachardot}\ Y\ {\isasymin}\ all{\isacharunderscore}partitions\ A\ {\isacharbraceright}{\isachardoublequoteclose}\isanewline
\ \ \isacommand{using}\isamarkupfalse%
\ possible{\isacharunderscore}allocations{\isacharunderscore}rel{\isacharunderscore}def\ \isacommand{by}\isamarkupfalse%
\ auto\ \isanewline
\ \ \isacommand{moreover}\isamarkupfalse%
\ \isacommand{have}\isamarkupfalse%
\ {\isachardoublequoteopen}{\isachardot}{\isachardot}{\isachardot}\ {\isacharequal}\ {\isacharquery}A{\isachardoublequoteclose}\ \isacommand{using}\isamarkupfalse%
\ injections{\isacharunderscore}def\ lm{\isadigit{1}}{\isadigit{2}}\ \isacommand{by}\isamarkupfalse%
\ metis\isanewline
\ \ \isacommand{ultimately}\isamarkupfalse%
\ \isacommand{show}\isamarkupfalse%
\ {\isacharquery}thesis\ \isacommand{by}\isamarkupfalse%
\ presburger\isanewline
\isacommand{qed}\isamarkupfalse%
%
\endisatagproof
{\isafoldproof}%
%
\isadelimproof
\isanewline
%
\endisadelimproof
\isanewline
\isacommand{lemma}\isamarkupfalse%
\ lm{\isadigit{1}}{\isadigit{4}}{\isacharcolon}\ {\isachardoublequoteopen}possible{\isacharunderscore}allocations{\isacharunderscore}rel\ G\ N\ {\isasymsubseteq}\ injectionsUniverse\ {\isasyminter}\ {\isacharbraceleft}a{\isachardot}\ Range\ a\ {\isasymsubseteq}\ N\ {\isacharampersand}\ Domain\ a\ {\isasymin}\ all{\isacharunderscore}partitions\ G{\isacharbraceright}{\isachardoublequoteclose}\isanewline
%
\isadelimproof
%
\endisadelimproof
%
\isatagproof
\isacommand{using}\isamarkupfalse%
\ assms\ lm{\isadigit{0}}{\isadigit{9}}\ lm{\isadigit{1}}{\isadigit{0}}\ possible{\isacharunderscore}allocations{\isacharunderscore}rel{\isacharunderscore}def\ injections{\isacharunderscore}def\ \isacommand{by}\isamarkupfalse%
\ fastforce%
\endisatagproof
{\isafoldproof}%
%
\isadelimproof
\isanewline
%
\endisadelimproof
\isanewline
\isacommand{lemma}\isamarkupfalse%
\ lm{\isadigit{1}}{\isadigit{5}}{\isacharcolon}\ {\isachardoublequoteopen}possible{\isacharunderscore}allocations{\isacharunderscore}rel\ G\ N\ {\isasymsupseteq}\ injectionsUniverse\ {\isasyminter}\ {\isacharbraceleft}a{\isachardot}\ Domain\ a\ {\isasymin}\ all{\isacharunderscore}partitions\ G\ {\isacharampersand}\ Range\ a\ {\isasymsubseteq}\ N{\isacharbraceright}{\isachardoublequoteclose}\isanewline
%
\isadelimproof
%
\endisadelimproof
%
\isatagproof
\isacommand{using}\isamarkupfalse%
\ possible{\isacharunderscore}allocations{\isacharunderscore}rel{\isacharunderscore}def\ injections{\isacharunderscore}def\ \isacommand{by}\isamarkupfalse%
\ auto%
\endisatagproof
{\isafoldproof}%
%
\isadelimproof
\isanewline
%
\endisadelimproof
\isanewline
\isacommand{lemma}\isamarkupfalse%
\ lm{\isadigit{1}}{\isadigit{6}}{\isacharcolon}\ {\isachardoublequoteopen}converse\ {\isacharbackquote}\ injectionsUniverse\ {\isacharequal}\ injectionsUniverse{\isachardoublequoteclose}%
\isadelimproof
\ %
\endisadelimproof
%
\isatagproof
\isacommand{by}\isamarkupfalse%
\ auto%
\endisatagproof
{\isafoldproof}%
%
\isadelimproof
%
\endisadelimproof
\isanewline
\isanewline
\isacommand{lemma}\isamarkupfalse%
\ lm{\isadigit{1}}{\isadigit{7}}{\isacharcolon}\ {\isachardoublequoteopen}possible{\isacharunderscore}allocations{\isacharunderscore}rel\ G\ N\ {\isacharequal}\ injectionsUniverse\ {\isasyminter}\ {\isacharbraceleft}a{\isachardot}\ Domain\ a\ {\isasymin}\ all{\isacharunderscore}partitions\ G\ {\isacharampersand}\ Range\ a\ {\isasymsubseteq}\ N{\isacharbraceright}{\isachardoublequoteclose}\isanewline
%
\isadelimproof
%
\endisadelimproof
%
\isatagproof
\isacommand{using}\isamarkupfalse%
\ lm{\isadigit{1}}{\isadigit{4}}\ lm{\isadigit{1}}{\isadigit{5}}\ \isacommand{by}\isamarkupfalse%
\ blast%
\endisatagproof
{\isafoldproof}%
%
\isadelimproof
\isanewline
%
\endisadelimproof
\isanewline
\isacommand{lemma}\isamarkupfalse%
\ lm{\isadigit{1}}{\isadigit{8}}{\isacharcolon}\ {\isachardoublequoteopen}converse{\isacharbackquote}{\isacharparenleft}A\ {\isasyminter}\ B{\isacharparenright}{\isacharequal}converse{\isacharbackquote}A\ {\isasyminter}\ {\isacharparenleft}converse{\isacharbackquote}B{\isacharparenright}{\isachardoublequoteclose}%
\isadelimproof
\ %
\endisadelimproof
%
\isatagproof
\isacommand{by}\isamarkupfalse%
\ force%
\endisatagproof
{\isafoldproof}%
%
\isadelimproof
%
\endisadelimproof
\isanewline
\isanewline
\isacommand{lemma}\isamarkupfalse%
\ lm{\isadigit{1}}{\isadigit{9}}{\isacharcolon}\ {\isachardoublequoteopen}possibleAllocationsRel\ N\ G\ {\isacharequal}\ injectionsUniverse\ {\isasyminter}\ {\isacharbraceleft}a{\isachardot}\ Domain\ a\ {\isasymsubseteq}\ N\ {\isacharampersand}\ Range\ a\ {\isasymin}\ all{\isacharunderscore}partitions\ G{\isacharbraceright}{\isachardoublequoteclose}\isanewline
%
\isadelimproof
%
\endisadelimproof
%
\isatagproof
\isacommand{proof}\isamarkupfalse%
\ {\isacharminus}\isanewline
\ \ \isacommand{let}\isamarkupfalse%
\ {\isacharquery}A{\isacharequal}{\isachardoublequoteopen}possible{\isacharunderscore}allocations{\isacharunderscore}rel\ G\ N{\isachardoublequoteclose}\ \isacommand{let}\isamarkupfalse%
\ {\isacharquery}c{\isacharequal}converse\ \isacommand{let}\isamarkupfalse%
\ {\isacharquery}I{\isacharequal}injectionsUniverse\ \isanewline
\ \ \isacommand{let}\isamarkupfalse%
\ {\isacharquery}P{\isacharequal}{\isachardoublequoteopen}all{\isacharunderscore}partitions\ G{\isachardoublequoteclose}\ \isacommand{let}\isamarkupfalse%
\ {\isacharquery}d{\isacharequal}Domain\ \isacommand{let}\isamarkupfalse%
\ {\isacharquery}r{\isacharequal}Range\isanewline
\ \ \isacommand{have}\isamarkupfalse%
\ {\isachardoublequoteopen}{\isacharquery}c{\isacharbackquote}{\isacharquery}A\ {\isacharequal}\ {\isacharparenleft}{\isacharquery}c{\isacharbackquote}{\isacharquery}I{\isacharparenright}\ {\isasyminter}\ {\isacharquery}c{\isacharbackquote}\ {\isacharparenleft}{\isacharbraceleft}a{\isachardot}\ {\isacharquery}r\ a\ {\isasymsubseteq}\ N\ {\isacharampersand}\ {\isacharquery}d\ a\ {\isasymin}\ {\isacharquery}P{\isacharbraceright}{\isacharparenright}{\isachardoublequoteclose}\ \isacommand{using}\isamarkupfalse%
\ lm{\isadigit{1}}{\isadigit{7}}\ \isacommand{by}\isamarkupfalse%
\ fastforce\isanewline
\ \ \isacommand{moreover}\isamarkupfalse%
\ \isacommand{have}\isamarkupfalse%
\ {\isachardoublequoteopen}{\isachardot}{\isachardot}{\isachardot}\ {\isacharequal}\ {\isacharparenleft}{\isacharquery}c{\isacharbackquote}{\isacharquery}I{\isacharparenright}\ {\isasyminter}\ {\isacharbraceleft}aa{\isachardot}\ {\isacharquery}d\ aa\ {\isasymsubseteq}\ N\ {\isacharampersand}\ {\isacharquery}r\ aa\ {\isasymin}\ {\isacharquery}P{\isacharbraceright}{\isachardoublequoteclose}\ \isacommand{by}\isamarkupfalse%
\ fastforce\isanewline
\ \ \isacommand{moreover}\isamarkupfalse%
\ \isacommand{have}\isamarkupfalse%
\ {\isachardoublequoteopen}{\isachardot}{\isachardot}{\isachardot}\ {\isacharequal}\ {\isacharquery}I\ {\isasyminter}\ {\isacharbraceleft}aa{\isachardot}\ {\isacharquery}d\ aa\ {\isasymsubseteq}\ N\ {\isacharampersand}\ {\isacharquery}r\ aa\ {\isasymin}\ {\isacharquery}P{\isacharbraceright}{\isachardoublequoteclose}\ \isacommand{using}\isamarkupfalse%
\ lm{\isadigit{1}}{\isadigit{6}}\ \isacommand{by}\isamarkupfalse%
\ metis\isanewline
\ \ \isacommand{ultimately}\isamarkupfalse%
\ \isacommand{show}\isamarkupfalse%
\ {\isacharquery}thesis\ \isacommand{by}\isamarkupfalse%
\ presburger\isanewline
\isacommand{qed}\isamarkupfalse%
%
\endisatagproof
{\isafoldproof}%
%
\isadelimproof
\isanewline
%
\endisadelimproof
\isanewline
\isacommand{corollary}\isamarkupfalse%
\ lm{\isadigit{1}}{\isadigit{9}}c{\isacharcolon}\ {\isachardoublequoteopen}a\ {\isasymin}\ possibleAllocationsRel\ N\ G\ {\isacharequal}\ \isanewline
{\isacharparenleft}a\ {\isasymin}\ injectionsUniverse\ {\isacharampersand}\ Domain\ a\ {\isasymsubseteq}\ N\ {\isacharampersand}\ Range\ a\ {\isasymin}\ all{\isacharunderscore}partitions\ G{\isacharparenright}{\isachardoublequoteclose}\ \isanewline
%
\isadelimproof
%
\endisadelimproof
%
\isatagproof
\isacommand{using}\isamarkupfalse%
\ lm{\isadigit{1}}{\isadigit{9}}\ Int{\isacharunderscore}Collect\ Int{\isacharunderscore}iff\ \isacommand{by}\isamarkupfalse%
\ {\isacharparenleft}metis\ {\isacharparenleft}lifting{\isacharparenright}{\isacharparenright}%
\endisatagproof
{\isafoldproof}%
%
\isadelimproof
\isanewline
%
\endisadelimproof
\isanewline
\isacommand{corollary}\isamarkupfalse%
\ lm{\isadigit{1}}{\isadigit{9}}d{\isacharcolon}\ \isakeyword{assumes}\ {\isachardoublequoteopen}a\ {\isasymin}\ possibleAllocationsRel\ N{\isadigit{1}}\ G{\isachardoublequoteclose}\ \isakeyword{shows}\ \isanewline
{\isachardoublequoteopen}a\ {\isasymin}\ possibleAllocationsRel\ {\isacharparenleft}N{\isadigit{1}}\ {\isasymunion}\ N{\isadigit{2}}{\isacharparenright}\ G{\isachardoublequoteclose}\isanewline
%
\isadelimproof
%
\endisadelimproof
%
\isatagproof
\isacommand{proof}\isamarkupfalse%
\ {\isacharminus}\ \isanewline
\isacommand{have}\isamarkupfalse%
\ {\isachardoublequoteopen}Domain\ a\ {\isasymsubseteq}\ N{\isadigit{1}}\ {\isasymunion}\ N{\isadigit{2}}{\isachardoublequoteclose}\ \isacommand{using}\isamarkupfalse%
\ assms{\isacharparenleft}{\isadigit{1}}{\isacharparenright}\ lm{\isadigit{1}}{\isadigit{9}}c\ \isacommand{by}\isamarkupfalse%
\ {\isacharparenleft}metis\ le{\isacharunderscore}supI{\isadigit{1}}{\isacharparenright}\ \isanewline
\isacommand{moreover}\isamarkupfalse%
\ \isacommand{have}\isamarkupfalse%
\ {\isachardoublequoteopen}a\ {\isasymin}\ injectionsUniverse\ {\isacharampersand}\ Range\ a\ {\isasymin}\ all{\isacharunderscore}partitions\ G{\isachardoublequoteclose}\ \isanewline
\isacommand{using}\isamarkupfalse%
\ assms\ lm{\isadigit{1}}{\isadigit{9}}c\ \isacommand{by}\isamarkupfalse%
\ blast\ \isacommand{ultimately}\isamarkupfalse%
\ \isacommand{show}\isamarkupfalse%
\ {\isacharquery}thesis\ \isacommand{using}\isamarkupfalse%
\ lm{\isadigit{1}}{\isadigit{9}}c\ \isacommand{by}\isamarkupfalse%
\ blast\ \isanewline
\isacommand{qed}\isamarkupfalse%
%
\endisatagproof
{\isafoldproof}%
%
\isadelimproof
\isanewline
%
\endisadelimproof
\isanewline
\isacommand{corollary}\isamarkupfalse%
\ lm{\isadigit{1}}{\isadigit{9}}b{\isacharcolon}\ {\isachardoublequoteopen}possibleAllocationsRel\ N{\isadigit{1}}\ G\ {\isasymsubseteq}\ possibleAllocationsRel\ {\isacharparenleft}N{\isadigit{1}}\ {\isasymunion}\ N{\isadigit{2}}{\isacharparenright}\ G{\isachardoublequoteclose}\isanewline
%
\isadelimproof
%
\endisadelimproof
%
\isatagproof
\isacommand{using}\isamarkupfalse%
\ lm{\isadigit{1}}{\isadigit{9}}d\ \isacommand{by}\isamarkupfalse%
\ {\isacharparenleft}metis\ subsetI{\isacharparenright}%
\endisatagproof
{\isafoldproof}%
%
\isadelimproof
\isanewline
%
\endisadelimproof
\isanewline
\isacommand{lemma}\isamarkupfalse%
\ \isakeyword{assumes}\ {\isachardoublequoteopen}x{\isasymnoteq}{\isacharbraceleft}{\isacharbraceright}{\isachardoublequoteclose}\ \isakeyword{shows}\ {\isachardoublequoteopen}is{\isacharunderscore}partition\ {\isacharbraceleft}x{\isacharbraceright}{\isachardoublequoteclose}%
\isadelimproof
\ %
\endisadelimproof
%
\isatagproof
\isacommand{unfolding}\isamarkupfalse%
\ is{\isacharunderscore}partition{\isacharunderscore}def\ \isacommand{using}\isamarkupfalse%
\ assms\ is{\isacharunderscore}partition{\isacharunderscore}def\ \isacommand{by}\isamarkupfalse%
\ force%
\endisatagproof
{\isafoldproof}%
%
\isadelimproof
%
\endisadelimproof
\isanewline
\isanewline
\isacommand{lemma}\isamarkupfalse%
\ lm{\isadigit{2}}{\isadigit{0}}d{\isacharcolon}\ \isakeyword{assumes}\ {\isachardoublequoteopen}{\isasymUnion}\ P{\isadigit{1}}\ {\isasyminter}\ {\isacharparenleft}{\isasymUnion}\ P{\isadigit{2}}{\isacharparenright}\ {\isacharequal}\ {\isacharbraceleft}{\isacharbraceright}{\isachardoublequoteclose}\ {\isachardoublequoteopen}is{\isacharunderscore}partition\ P{\isadigit{1}}{\isachardoublequoteclose}\ {\isachardoublequoteopen}is{\isacharunderscore}partition\ P{\isadigit{2}}{\isachardoublequoteclose}\ {\isachardoublequoteopen}X\ {\isasymin}\ P{\isadigit{1}}\ {\isasymunion}\ P{\isadigit{2}}{\isachardoublequoteclose}\ {\isachardoublequoteopen}Y\ {\isasymin}\ P{\isadigit{1}}\ {\isasymunion}\ P{\isadigit{2}}{\isachardoublequoteclose}\isanewline
{\isachardoublequoteopen}X\ {\isasyminter}\ Y\ {\isasymnoteq}\ {\isacharbraceleft}{\isacharbraceright}{\isachardoublequoteclose}\ \isakeyword{shows}\ {\isachardoublequoteopen}{\isacharparenleft}X\ {\isacharequal}\ Y{\isacharparenright}{\isachardoublequoteclose}%
\isadelimproof
\ %
\endisadelimproof
%
\isatagproof
\isacommand{unfolding}\isamarkupfalse%
\ is{\isacharunderscore}partition{\isacharunderscore}def\ \isacommand{using}\isamarkupfalse%
\ assms\ is{\isacharunderscore}partition{\isacharunderscore}def\ \isacommand{by}\isamarkupfalse%
\ fast%
\endisatagproof
{\isafoldproof}%
%
\isadelimproof
%
\endisadelimproof
\isanewline
\isanewline
\isacommand{lemma}\isamarkupfalse%
\ lm{\isadigit{2}}{\isadigit{0}}e{\isacharcolon}\ \isakeyword{assumes}\ {\isachardoublequoteopen}{\isasymUnion}\ P{\isadigit{1}}\ {\isasyminter}\ {\isacharparenleft}{\isasymUnion}\ P{\isadigit{2}}{\isacharparenright}\ {\isacharequal}\ {\isacharbraceleft}{\isacharbraceright}{\isachardoublequoteclose}\ {\isachardoublequoteopen}is{\isacharunderscore}partition\ P{\isadigit{1}}{\isachardoublequoteclose}\ {\isachardoublequoteopen}is{\isacharunderscore}partition\ P{\isadigit{2}}{\isachardoublequoteclose}\ {\isachardoublequoteopen}X\ {\isasymin}\ P{\isadigit{1}}\ {\isasymunion}\ P{\isadigit{2}}{\isachardoublequoteclose}\ {\isachardoublequoteopen}Y\ {\isasymin}\ P{\isadigit{1}}\ {\isasymunion}\ P{\isadigit{2}}{\isachardoublequoteclose}\isanewline
{\isachardoublequoteopen}{\isacharparenleft}X\ {\isacharequal}\ Y{\isacharparenright}{\isachardoublequoteclose}\ \isakeyword{shows}\ {\isachardoublequoteopen}X\ {\isasyminter}\ Y\ {\isasymnoteq}\ {\isacharbraceleft}{\isacharbraceright}{\isachardoublequoteclose}%
\isadelimproof
\ %
\endisadelimproof
%
\isatagproof
\isacommand{unfolding}\isamarkupfalse%
\ is{\isacharunderscore}partition{\isacharunderscore}def\ \isacommand{using}\isamarkupfalse%
\ assms\ is{\isacharunderscore}partition{\isacharunderscore}def\ \isacommand{by}\isamarkupfalse%
\ fast%
\endisatagproof
{\isafoldproof}%
%
\isadelimproof
%
\endisadelimproof
\isanewline
\isanewline
\isacommand{lemma}\isamarkupfalse%
\ lm{\isadigit{2}}{\isadigit{0}}{\isacharcolon}\ \isakeyword{assumes}\ {\isachardoublequoteopen}{\isasymUnion}\ P{\isadigit{1}}\ {\isasyminter}\ {\isacharparenleft}{\isasymUnion}\ P{\isadigit{2}}{\isacharparenright}\ {\isacharequal}\ {\isacharbraceleft}{\isacharbraceright}{\isachardoublequoteclose}\ {\isachardoublequoteopen}is{\isacharunderscore}partition\ P{\isadigit{1}}{\isachardoublequoteclose}\ {\isachardoublequoteopen}is{\isacharunderscore}partition\ P{\isadigit{2}}{\isachardoublequoteclose}\isanewline
\isakeyword{shows}\ {\isachardoublequoteopen}is{\isacharunderscore}partition\ {\isacharparenleft}P{\isadigit{1}}\ {\isasymunion}\ P{\isadigit{2}}{\isacharparenright}{\isachardoublequoteclose}%
\isadelimproof
\ %
\endisadelimproof
%
\isatagproof
\isacommand{unfolding}\isamarkupfalse%
\ is{\isacharunderscore}partition{\isacharunderscore}def\ \isacommand{using}\isamarkupfalse%
\ assms\ lm{\isadigit{2}}{\isadigit{0}}d\ lm{\isadigit{2}}{\isadigit{0}}e\ \isacommand{by}\isamarkupfalse%
\ metis%
\endisatagproof
{\isafoldproof}%
%
\isadelimproof
%
\endisadelimproof
\isanewline
\isanewline
\isacommand{lemma}\isamarkupfalse%
\ lm{\isadigit{2}}{\isadigit{1}}{\isacharcolon}\ {\isachardoublequoteopen}Range\ Q\ {\isasymunion}\ {\isacharparenleft}Range\ {\isacharparenleft}P\ outside\ {\isacharparenleft}Domain\ Q{\isacharparenright}{\isacharparenright}{\isacharparenright}\ {\isacharequal}\ Range\ {\isacharparenleft}P\ {\isacharplus}{\isacharasterisk}\ Q{\isacharparenright}{\isachardoublequoteclose}\isanewline
%
\isadelimproof
%
\endisadelimproof
%
\isatagproof
\isacommand{unfolding}\isamarkupfalse%
\ paste{\isacharunderscore}def\ Range{\isacharunderscore}Un{\isacharunderscore}eq\ Un{\isacharunderscore}commute\ \isacommand{by}\isamarkupfalse%
\ {\isacharparenleft}metis{\isacharparenleft}no{\isacharunderscore}types{\isacharparenright}{\isacharparenright}%
\endisatagproof
{\isafoldproof}%
%
\isadelimproof
\isanewline
%
\endisadelimproof
\isanewline
\isacommand{lemma}\isamarkupfalse%
\ lll{\isadigit{7}}{\isadigit{7}}c{\isacharcolon}\ \isakeyword{assumes}\ {\isachardoublequoteopen}a{\isadigit{1}}\ {\isasymin}\ injectionsUniverse{\isachardoublequoteclose}\ {\isachardoublequoteopen}a{\isadigit{2}}\ {\isasymin}\ injectionsUniverse{\isachardoublequoteclose}\ {\isachardoublequoteopen}Range\ a{\isadigit{1}}\ {\isasyminter}\ {\isacharparenleft}Range\ a{\isadigit{2}}{\isacharparenright}{\isacharequal}{\isacharbraceleft}{\isacharbraceright}{\isachardoublequoteclose}\isanewline
{\isachardoublequoteopen}Domain\ a{\isadigit{1}}\ {\isasyminter}\ {\isacharparenleft}Domain\ a{\isadigit{2}}{\isacharparenright}\ {\isacharequal}\ {\isacharbraceleft}{\isacharbraceright}{\isachardoublequoteclose}\ \isakeyword{shows}\ {\isachardoublequoteopen}a{\isadigit{1}}\ {\isasymunion}\ a{\isadigit{2}}\ {\isasymin}\ injectionsUniverse{\isachardoublequoteclose}\ \isanewline
%
\isadelimproof
%
\endisadelimproof
%
\isatagproof
\isacommand{using}\isamarkupfalse%
\ assms\ disj{\isacharunderscore}Un{\isacharunderscore}runiq\ \isacommand{by}\isamarkupfalse%
\ {\isacharparenleft}metis\ {\isacharparenleft}no{\isacharunderscore}types{\isacharparenright}\ Domain{\isacharunderscore}converse\ converse{\isacharunderscore}Un\ mem{\isacharunderscore}Collect{\isacharunderscore}eq{\isacharparenright}%
\endisatagproof
{\isafoldproof}%
%
\isadelimproof
\isanewline
%
\endisadelimproof
\isanewline
\isacommand{lemma}\isamarkupfalse%
\ lm{\isadigit{2}}{\isadigit{2}}{\isacharcolon}\ \isakeyword{assumes}\ {\isachardoublequoteopen}R\ {\isasymin}\ partitionValuedUniverse{\isachardoublequoteclose}\ \isakeyword{shows}\ {\isachardoublequoteopen}is{\isacharunderscore}partition\ {\isacharparenleft}Range\ R{\isacharparenright}{\isachardoublequoteclose}\isanewline
%
\isadelimproof
%
\endisadelimproof
%
\isatagproof
\isacommand{using}\isamarkupfalse%
\ assms\ \isanewline
\isacommand{proof}\isamarkupfalse%
\ {\isacharminus}\isanewline
\ \ \isacommand{obtain}\isamarkupfalse%
\ P\ \isakeyword{where}\isanewline
\ \ {\isadigit{0}}{\isacharcolon}\ {\isachardoublequoteopen}P\ {\isasymin}\ partitionsUniverse\ {\isacharampersand}\ R\ {\isasymsubseteq}\ UNIV\ {\isasymtimes}\ P{\isachardoublequoteclose}\ \isacommand{using}\isamarkupfalse%
\ assms\ \isacommand{by}\isamarkupfalse%
\ blast\isanewline
\ \ \isacommand{have}\isamarkupfalse%
\ {\isachardoublequoteopen}Range\ R\ {\isasymsubseteq}\ P{\isachardoublequoteclose}\ \isacommand{using}\isamarkupfalse%
\ {\isadigit{0}}\ \isacommand{by}\isamarkupfalse%
\ fast\isanewline
\ \ \isacommand{then}\isamarkupfalse%
\ \isacommand{show}\isamarkupfalse%
\ {\isacharquery}thesis\ \isacommand{using}\isamarkupfalse%
\ {\isadigit{0}}\ mem{\isacharunderscore}Collect{\isacharunderscore}eq\ subset{\isacharunderscore}is{\isacharunderscore}partition\ \isacommand{by}\isamarkupfalse%
\ {\isacharparenleft}metis{\isacharparenright}\isanewline
\isacommand{qed}\isamarkupfalse%
%
\endisatagproof
{\isafoldproof}%
%
\isadelimproof
\isanewline
%
\endisadelimproof
\isanewline
\isacommand{lemma}\isamarkupfalse%
\ lm{\isadigit{2}}{\isadigit{3}}{\isacharcolon}\ \isakeyword{assumes}\ {\isachardoublequoteopen}a{\isadigit{1}}\ {\isasymin}\ allocationsUniverse{\isachardoublequoteclose}\ {\isachardoublequoteopen}a{\isadigit{2}}\ {\isasymin}\ allocationsUniverse{\isachardoublequoteclose}\ {\isachardoublequoteopen}{\isasymUnion}\ {\isacharparenleft}Range\ a{\isadigit{1}}{\isacharparenright}\ {\isasyminter}\ {\isacharparenleft}{\isasymUnion}\ {\isacharparenleft}Range\ a{\isadigit{2}}{\isacharparenright}{\isacharparenright}{\isacharequal}{\isacharbraceleft}{\isacharbraceright}{\isachardoublequoteclose}\isanewline
{\isachardoublequoteopen}Domain\ a{\isadigit{1}}\ {\isasyminter}\ {\isacharparenleft}Domain\ a{\isadigit{2}}{\isacharparenright}\ {\isacharequal}\ {\isacharbraceleft}{\isacharbraceright}{\isachardoublequoteclose}\ \isakeyword{shows}\ {\isachardoublequoteopen}a{\isadigit{1}}\ {\isasymunion}\ a{\isadigit{2}}\ {\isasymin}\ allocationsUniverse{\isachardoublequoteclose}\isanewline
%
\isadelimproof
%
\endisadelimproof
%
\isatagproof
\isacommand{proof}\isamarkupfalse%
\ {\isacharminus}\isanewline
\ \ \isacommand{let}\isamarkupfalse%
\ {\isacharquery}a{\isacharequal}{\isachardoublequoteopen}a{\isadigit{1}}\ {\isasymunion}\ a{\isadigit{2}}{\isachardoublequoteclose}\ \isacommand{let}\isamarkupfalse%
\ {\isacharquery}b{\isadigit{1}}{\isacharequal}{\isachardoublequoteopen}a{\isadigit{1}}{\isacharcircum}{\isacharminus}{\isadigit{1}}{\isachardoublequoteclose}\ \isacommand{let}\isamarkupfalse%
\ {\isacharquery}b{\isadigit{2}}{\isacharequal}{\isachardoublequoteopen}a{\isadigit{2}}{\isacharcircum}{\isacharminus}{\isadigit{1}}{\isachardoublequoteclose}\ \isacommand{let}\isamarkupfalse%
\ {\isacharquery}r{\isacharequal}Range\ \isacommand{let}\isamarkupfalse%
\ {\isacharquery}d{\isacharequal}Domain\isanewline
\ \ \isacommand{let}\isamarkupfalse%
\ {\isacharquery}I{\isacharequal}injectionsUniverse\ \isacommand{let}\isamarkupfalse%
\ {\isacharquery}P{\isacharequal}partitionsUniverse\ \isacommand{let}\isamarkupfalse%
\ {\isacharquery}PV{\isacharequal}partitionValuedUniverse\ \isacommand{let}\isamarkupfalse%
\ {\isacharquery}u{\isacharequal}runiq\isanewline
\ \ \isacommand{let}\isamarkupfalse%
\ {\isacharquery}b{\isacharequal}{\isachardoublequoteopen}{\isacharquery}a{\isacharcircum}{\isacharminus}{\isadigit{1}}{\isachardoublequoteclose}\ \isacommand{let}\isamarkupfalse%
\ {\isacharquery}p{\isacharequal}is{\isacharunderscore}partition\isanewline
\ \ \isacommand{have}\isamarkupfalse%
\ {\isachardoublequoteopen}{\isacharquery}p\ {\isacharparenleft}{\isacharquery}r\ a{\isadigit{1}}{\isacharparenright}\ {\isacharampersand}\ {\isacharquery}p\ {\isacharparenleft}{\isacharquery}r\ a{\isadigit{2}}{\isacharparenright}{\isachardoublequoteclose}\ \isacommand{using}\isamarkupfalse%
\ assms\ lm{\isadigit{2}}{\isadigit{2}}\ \isacommand{by}\isamarkupfalse%
\ blast\ \isacommand{then}\isamarkupfalse%
\isanewline
\ \ \isacommand{moreover}\isamarkupfalse%
\ \isacommand{have}\isamarkupfalse%
\ {\isachardoublequoteopen}{\isacharquery}p\ {\isacharparenleft}{\isacharquery}r\ a{\isadigit{1}}\ {\isasymunion}\ {\isacharquery}r\ a{\isadigit{2}}{\isacharparenright}{\isachardoublequoteclose}\ \isacommand{using}\isamarkupfalse%
\ assms\ \isacommand{by}\isamarkupfalse%
\ {\isacharparenleft}metis\ lm{\isadigit{2}}{\isadigit{0}}{\isacharparenright}\isanewline
\ \ \isacommand{then}\isamarkupfalse%
\ \isacommand{moreover}\isamarkupfalse%
\ \isacommand{have}\isamarkupfalse%
\ {\isachardoublequoteopen}{\isacharparenleft}{\isacharquery}r\ a{\isadigit{1}}\ {\isasymunion}\ {\isacharquery}r\ a{\isadigit{2}}{\isacharparenright}\ {\isasymin}\ {\isacharquery}P{\isachardoublequoteclose}\ \isacommand{by}\isamarkupfalse%
\ simp\isanewline
\ \ \isacommand{moreover}\isamarkupfalse%
\ \isacommand{have}\isamarkupfalse%
\ {\isachardoublequoteopen}{\isacharquery}r\ {\isacharquery}a\ {\isacharequal}\ {\isacharparenleft}{\isacharquery}r\ a{\isadigit{1}}\ {\isasymunion}\ {\isacharquery}r\ a{\isadigit{2}}{\isacharparenright}{\isachardoublequoteclose}\ \isacommand{using}\isamarkupfalse%
\ assms\ \isacommand{by}\isamarkupfalse%
\ fast\isanewline
\ \ \isacommand{ultimately}\isamarkupfalse%
\ \isacommand{moreover}\isamarkupfalse%
\ \isacommand{have}\isamarkupfalse%
\ {\isachardoublequoteopen}{\isacharquery}p\ {\isacharparenleft}{\isacharquery}r\ {\isacharquery}a{\isacharparenright}{\isachardoublequoteclose}\ \isacommand{using}\isamarkupfalse%
\ lm{\isadigit{2}}{\isadigit{0}}\ assms\ \isacommand{by}\isamarkupfalse%
\ fastforce\isanewline
\ \ \isacommand{then}\isamarkupfalse%
\ \isacommand{moreover}\isamarkupfalse%
\ \isacommand{have}\isamarkupfalse%
\ {\isachardoublequoteopen}{\isacharquery}a\ {\isasymin}\ {\isacharquery}PV{\isachardoublequoteclose}\ \isacommand{using}\isamarkupfalse%
\ assms\ \isacommand{by}\isamarkupfalse%
\ fast\isanewline
\ \ \isacommand{moreover}\isamarkupfalse%
\ \isacommand{have}\isamarkupfalse%
\ {\isachardoublequoteopen}{\isacharquery}r\ a{\isadigit{1}}\ {\isasyminter}\ {\isacharparenleft}{\isacharquery}r\ a{\isadigit{2}}{\isacharparenright}\ {\isasymsubseteq}\ Pow\ {\isacharparenleft}{\isasymUnion}\ {\isacharparenleft}{\isacharquery}r\ a{\isadigit{1}}{\isacharparenright}\ {\isasyminter}\ {\isacharparenleft}{\isasymUnion}\ {\isacharparenleft}{\isacharquery}r\ a{\isadigit{2}}{\isacharparenright}{\isacharparenright}{\isacharparenright}{\isachardoublequoteclose}\ \isacommand{by}\isamarkupfalse%
\ auto\isanewline
\ \ \isacommand{ultimately}\isamarkupfalse%
\ \isacommand{moreover}\isamarkupfalse%
\ \isacommand{have}\isamarkupfalse%
\ {\isachardoublequoteopen}{\isacharbraceleft}{\isacharbraceright}\ {\isasymnotin}\ {\isacharparenleft}{\isacharquery}r\ a{\isadigit{1}}{\isacharparenright}\ {\isacharampersand}\ {\isacharbraceleft}{\isacharbraceright}\ {\isasymnotin}\ {\isacharparenleft}{\isacharquery}r\ a{\isadigit{2}}{\isacharparenright}{\isachardoublequoteclose}\ \isacommand{using}\isamarkupfalse%
\ is{\isacharunderscore}partition{\isacharunderscore}def\ \isacommand{by}\isamarkupfalse%
\ {\isacharparenleft}metis\ Int{\isacharunderscore}empty{\isacharunderscore}left{\isacharparenright}\isanewline
\ \ \isacommand{ultimately}\isamarkupfalse%
\ \isacommand{moreover}\isamarkupfalse%
\ \isacommand{have}\isamarkupfalse%
\ {\isachardoublequoteopen}{\isacharquery}r\ a{\isadigit{1}}\ {\isasyminter}\ {\isacharparenleft}{\isacharquery}r\ a{\isadigit{2}}{\isacharparenright}\ {\isacharequal}\ {\isacharbraceleft}{\isacharbraceright}{\isachardoublequoteclose}\ \isacommand{using}\isamarkupfalse%
\ assms\ lm{\isadigit{2}}{\isadigit{2}}\ is{\isacharunderscore}partition{\isacharunderscore}def\ \isacommand{by}\isamarkupfalse%
\ auto\isanewline
\ \ \isacommand{ultimately}\isamarkupfalse%
\ \isacommand{moreover}\isamarkupfalse%
\ \isacommand{have}\isamarkupfalse%
\ {\isachardoublequoteopen}{\isacharquery}a\ {\isasymin}\ {\isacharquery}I{\isachardoublequoteclose}\ \isacommand{using}\isamarkupfalse%
\ lll{\isadigit{7}}{\isadigit{7}}c\ assms\ \isacommand{by}\isamarkupfalse%
\ fastforce\isanewline
\ \ \isacommand{ultimately}\isamarkupfalse%
\ \isacommand{show}\isamarkupfalse%
\ {\isacharquery}thesis\ \isacommand{by}\isamarkupfalse%
\ blast\isanewline
\isacommand{qed}\isamarkupfalse%
%
\endisatagproof
{\isafoldproof}%
%
\isadelimproof
\isanewline
%
\endisadelimproof
\isanewline
\isacommand{lemma}\isamarkupfalse%
\ lm{\isadigit{2}}{\isadigit{7}}{\isacharcolon}\ \isakeyword{assumes}\ {\isachardoublequoteopen}a\ {\isasymin}\ injectionsUniverse{\isachardoublequoteclose}\ \isakeyword{shows}\ {\isachardoublequoteopen}a\ {\isacharminus}\ b\ {\isasymin}\ injectionsUniverse{\isachardoublequoteclose}%
\isadelimproof
\ %
\endisadelimproof
%
\isatagproof
\isacommand{using}\isamarkupfalse%
\ assms\ \isanewline
\isacommand{by}\isamarkupfalse%
\ {\isacharparenleft}metis\ {\isacharparenleft}lifting{\isacharparenright}\ Diff{\isacharunderscore}subset\ converse{\isacharunderscore}mono\ mem{\isacharunderscore}Collect{\isacharunderscore}eq\ subrel{\isacharunderscore}runiq{\isacharparenright}%
\endisatagproof
{\isafoldproof}%
%
\isadelimproof
%
\endisadelimproof
\isanewline
\isanewline
\isacommand{lemma}\isamarkupfalse%
\ lm{\isadigit{3}}{\isadigit{0}}b{\isacharcolon}\ {\isachardoublequoteopen}{\isacharbraceleft}a{\isachardot}\ Domain\ a\ {\isasymsubseteq}\ N\ {\isacharampersand}\ Range\ a\ {\isasymin}\ all{\isacharunderscore}partitions\ G{\isacharbraceright}\ {\isacharequal}\isanewline
{\isacharparenleft}Range\ {\isacharminus}{\isacharbackquote}\ {\isacharparenleft}all{\isacharunderscore}partitions\ G{\isacharparenright}{\isacharparenright}\ {\isasyminter}\ {\isacharparenleft}Domain\ {\isacharminus}{\isacharbackquote}{\isacharparenleft}Pow\ N{\isacharparenright}{\isacharparenright}{\isachardoublequoteclose}\ \isanewline
%
\isadelimproof
%
\endisadelimproof
%
\isatagproof
\isacommand{by}\isamarkupfalse%
\ fastforce%
\endisatagproof
{\isafoldproof}%
%
\isadelimproof
\isanewline
%
\endisadelimproof
\isanewline
\isacommand{lemma}\isamarkupfalse%
\ lm{\isadigit{3}}{\isadigit{0}}{\isacharcolon}\ {\isachardoublequoteopen}possibleAllocationsRel\ N\ G\ {\isacharequal}\ injectionsUniverse\ {\isasyminter}\ {\isacharparenleft}{\isacharparenleft}Range\ {\isacharminus}{\isacharbackquote}\ {\isacharparenleft}all{\isacharunderscore}partitions\ G{\isacharparenright}{\isacharparenright}\isanewline
{\isasyminter}\ {\isacharparenleft}Domain\ {\isacharminus}{\isacharbackquote}{\isacharparenleft}Pow\ N{\isacharparenright}{\isacharparenright}{\isacharparenright}{\isachardoublequoteclose}\isanewline
%
\isadelimproof
%
\endisadelimproof
%
\isatagproof
\isacommand{using}\isamarkupfalse%
\ lm{\isadigit{1}}{\isadigit{9}}\ lm{\isadigit{3}}{\isadigit{0}}b\ \isacommand{by}\isamarkupfalse%
\ metis%
\endisatagproof
{\isafoldproof}%
%
\isadelimproof
\ \isanewline
%
\endisadelimproof
\isanewline
\isacommand{lemma}\isamarkupfalse%
\ lm{\isadigit{2}}{\isadigit{8}}a{\isacharcolon}\ \isakeyword{assumes}\ {\isachardoublequoteopen}a\ {\isasymin}\ possibleAllocationsRel\ N\ G{\isachardoublequoteclose}\ \isakeyword{shows}\ {\isachardoublequoteopen}{\isacharparenleft}a{\isacharcircum}{\isacharminus}{\isadigit{1}}\ {\isasymin}\ injections\ {\isacharparenleft}Range\ a{\isacharparenright}\ N\ {\isacharampersand}\ Range\ a\ {\isasymin}\ all{\isacharunderscore}partitions\ G{\isacharparenright}{\isachardoublequoteclose}\isanewline
%
\isadelimproof
%
\endisadelimproof
%
\isatagproof
\isacommand{using}\isamarkupfalse%
\ assms\ \isanewline
\isacommand{by}\isamarkupfalse%
\ {\isacharparenleft}metis\ {\isacharparenleft}mono{\isacharunderscore}tags{\isacharcomma}\ hide{\isacharunderscore}lams{\isacharparenright}\ lm{\isadigit{1}}{\isadigit{9}}c\ lm{\isadigit{4}}{\isadigit{7}}{\isacharparenright}%
\endisatagproof
{\isafoldproof}%
%
\isadelimproof
\isanewline
%
\endisadelimproof
\isanewline
\isacommand{lemma}\isamarkupfalse%
\ lm{\isadigit{2}}{\isadigit{8}}c{\isacharcolon}\ \isakeyword{assumes}\ {\isachardoublequoteopen}a{\isacharcircum}{\isacharminus}{\isadigit{1}}\ {\isasymin}\ injections\ {\isacharparenleft}Range\ a{\isacharparenright}\ N{\isachardoublequoteclose}\ {\isachardoublequoteopen}Range\ a\ {\isasymin}\ all{\isacharunderscore}partitions\ G{\isachardoublequoteclose}\ \isanewline
\isakeyword{shows}\ {\isachardoublequoteopen}a\ {\isasymin}\ possibleAllocationsRel\ N\ G{\isachardoublequoteclose}%
\isadelimproof
\ %
\endisadelimproof
%
\isatagproof
\isacommand{using}\isamarkupfalse%
\ assms\ image{\isacharunderscore}iff\ \isacommand{by}\isamarkupfalse%
\ fastforce%
\endisatagproof
{\isafoldproof}%
%
\isadelimproof
%
\endisadelimproof
\isanewline
\isanewline
\isacommand{lemma}\isamarkupfalse%
\ lm{\isadigit{2}}{\isadigit{8}}{\isacharcolon}\ {\isachardoublequoteopen}a\ {\isasymin}\ possibleAllocationsRel\ N\ G\ {\isacharequal}\ {\isacharparenleft}a{\isacharcircum}{\isacharminus}{\isadigit{1}}\ {\isasymin}\ injections\ {\isacharparenleft}Range\ a{\isacharparenright}\ N\ {\isacharampersand}\ Range\ a\ {\isasymin}\ all{\isacharunderscore}partitions\ G{\isacharparenright}{\isachardoublequoteclose}\isanewline
%
\isadelimproof
%
\endisadelimproof
%
\isatagproof
\isacommand{using}\isamarkupfalse%
\ lm{\isadigit{2}}{\isadigit{8}}a\ lm{\isadigit{2}}{\isadigit{8}}c\ \isacommand{by}\isamarkupfalse%
\ metis%
\endisatagproof
{\isafoldproof}%
%
\isadelimproof
\isanewline
%
\endisadelimproof
\isanewline
\isacommand{lemma}\isamarkupfalse%
\ lm{\isadigit{2}}{\isadigit{8}}d{\isacharcolon}\ \isakeyword{assumes}\ {\isachardoublequoteopen}a\ {\isasymin}\ possibleAllocationsRel\ N\ G{\isachardoublequoteclose}\ \isakeyword{shows}\ {\isachardoublequoteopen}{\isacharparenleft}a\ {\isasymin}\ injections\ {\isacharparenleft}Domain\ a{\isacharparenright}\ {\isacharparenleft}Range\ a{\isacharparenright}\ \isanewline
{\isacharampersand}\ Range\ a\ {\isasymin}\ all{\isacharunderscore}partitions\ G\ {\isacharampersand}\ Domain\ a\ {\isasymsubseteq}\ N{\isacharparenright}{\isachardoublequoteclose}%
\isadelimproof
\ %
\endisadelimproof
%
\isatagproof
\isacommand{using}\isamarkupfalse%
\ assms\ lm{\isadigit{2}}{\isadigit{8}}a\ \isanewline
\isacommand{by}\isamarkupfalse%
\ {\isacharparenleft}metis\ {\isacharparenleft}erased{\isacharcomma}\ lifting{\isacharparenright}\ Domain{\isacharunderscore}converse\ converse{\isacharunderscore}converse\ injectionsI\ injections{\isacharunderscore}def\ mem{\isacharunderscore}Collect{\isacharunderscore}eq\ order{\isacharunderscore}refl{\isacharparenright}%
\endisatagproof
{\isafoldproof}%
%
\isadelimproof
%
\endisadelimproof
\isanewline
\isanewline
\isacommand{lemma}\isamarkupfalse%
\ lm{\isadigit{2}}{\isadigit{8}}e{\isacharcolon}\ \isakeyword{assumes}\ {\isachardoublequoteopen}a\ {\isasymin}\ injections\ {\isacharparenleft}Domain\ a{\isacharparenright}\ {\isacharparenleft}Range\ a{\isacharparenright}{\isachardoublequoteclose}\ \isanewline
{\isachardoublequoteopen}Range\ a\ {\isasymin}\ all{\isacharunderscore}partitions\ G{\isachardoublequoteclose}\ {\isachardoublequoteopen}Domain\ a\ {\isasymsubseteq}\ N{\isachardoublequoteclose}\ \isakeyword{shows}\ {\isachardoublequoteopen}a\ {\isasymin}\ possibleAllocationsRel\ N\ G{\isachardoublequoteclose}\ \isanewline
%
\isadelimproof
%
\endisadelimproof
%
\isatagproof
\isacommand{using}\isamarkupfalse%
\ assms\ mem{\isacharunderscore}Collect{\isacharunderscore}eq\ lm{\isadigit{1}}{\isadigit{9}}c\ injections{\isacharunderscore}def\ \isacommand{by}\isamarkupfalse%
\ {\isacharparenleft}metis\ {\isacharparenleft}erased{\isacharcomma}\ lifting{\isacharparenright}{\isacharparenright}%
\endisatagproof
{\isafoldproof}%
%
\isadelimproof
\isanewline
%
\endisadelimproof
\isanewline
\isacommand{lemma}\isamarkupfalse%
\ lm{\isadigit{2}}{\isadigit{8}}b{\isacharcolon}\ {\isachardoublequoteopen}a\ {\isasymin}\ possibleAllocationsRel\ N\ G\ {\isacharequal}\ {\isacharparenleft}a\ {\isasymin}\ injections\ {\isacharparenleft}Domain\ a{\isacharparenright}\ {\isacharparenleft}Range\ a{\isacharparenright}\ \isanewline
{\isacharampersand}\ Range\ a\ {\isasymin}\ all{\isacharunderscore}partitions\ G\ {\isacharampersand}\ Domain\ a\ {\isasymsubseteq}\ N{\isacharparenright}{\isachardoublequoteclose}%
\isadelimproof
\ %
\endisadelimproof
%
\isatagproof
\isacommand{using}\isamarkupfalse%
\ lm{\isadigit{2}}{\isadigit{8}}d\ lm{\isadigit{2}}{\isadigit{8}}e\ \isacommand{by}\isamarkupfalse%
\ metis%
\endisatagproof
{\isafoldproof}%
%
\isadelimproof
%
\endisadelimproof
\isanewline
\isanewline
\isacommand{lemma}\isamarkupfalse%
\ lm{\isadigit{2}}{\isadigit{9}}{\isacharcolon}\ {\isachardoublequoteopen}possibleAllocationsRel\ N\ G\ {\isasymsupseteq}\ injectionsUniverse\ {\isasyminter}\ {\isacharparenleft}Range\ {\isacharminus}{\isacharbackquote}\ {\isacharparenleft}all{\isacharunderscore}partitions\ G{\isacharparenright}{\isacharparenright}\isanewline
{\isasyminter}\ {\isacharparenleft}Domain\ {\isacharminus}{\isacharbackquote}{\isacharparenleft}Pow\ N{\isacharparenright}{\isacharparenright}{\isachardoublequoteclose}%
\isadelimproof
\ %
\endisadelimproof
%
\isatagproof
\isacommand{using}\isamarkupfalse%
\ subsetI\ Int{\isacharunderscore}assoc\ lm{\isadigit{3}}{\isadigit{0}}\isanewline
\isacommand{by}\isamarkupfalse%
\ metis%
\endisatagproof
{\isafoldproof}%
%
\isadelimproof
%
\endisadelimproof
\ \isanewline
\isanewline
\isacommand{corollary}\isamarkupfalse%
\ lm{\isadigit{3}}{\isadigit{1}}{\isacharcolon}\ {\isachardoublequoteopen}possibleAllocationsRel\ N\ G\ {\isacharequal}\ injectionsUniverse\ {\isasyminter}\ {\isacharparenleft}Range\ {\isacharminus}{\isacharbackquote}\ {\isacharparenleft}all{\isacharunderscore}partitions\ G{\isacharparenright}{\isacharparenright}\isanewline
{\isasyminter}\ {\isacharparenleft}Domain\ {\isacharminus}{\isacharbackquote}{\isacharparenleft}Pow\ N{\isacharparenright}{\isacharparenright}{\isachardoublequoteclose}%
\isadelimproof
\ %
\endisadelimproof
%
\isatagproof
\isacommand{using}\isamarkupfalse%
\ lm{\isadigit{3}}{\isadigit{0}}\ Int{\isacharunderscore}assoc\ \isacommand{by}\isamarkupfalse%
\ {\isacharparenleft}metis{\isacharparenright}%
\endisatagproof
{\isafoldproof}%
%
\isadelimproof
%
\endisadelimproof
\isanewline
\isanewline
\isacommand{lemma}\isamarkupfalse%
\ lm{\isadigit{3}}{\isadigit{2}}{\isacharcolon}\ \isakeyword{assumes}\ {\isachardoublequoteopen}a\ {\isasymin}\ partitionValuedUniverse{\isachardoublequoteclose}\ \isakeyword{shows}\ {\isachardoublequoteopen}a\ {\isacharminus}\ b\ {\isasymin}\ partitionValuedUniverse{\isachardoublequoteclose}\ \isanewline
%
\isadelimproof
%
\endisadelimproof
%
\isatagproof
\isacommand{using}\isamarkupfalse%
\ assms\ subset{\isacharunderscore}is{\isacharunderscore}partition\ \isacommand{by}\isamarkupfalse%
\ fast%
\endisatagproof
{\isafoldproof}%
%
\isadelimproof
\isanewline
%
\endisadelimproof
\isanewline
\isacommand{lemma}\isamarkupfalse%
\ lm{\isadigit{3}}{\isadigit{5}}{\isacharcolon}\ \isakeyword{assumes}\ {\isachardoublequoteopen}a\ {\isasymin}\ allocationsUniverse{\isachardoublequoteclose}\ \isakeyword{shows}\ {\isachardoublequoteopen}a\ {\isacharminus}\ b\ {\isasymin}\ allocationsUniverse{\isachardoublequoteclose}%
\isadelimproof
\ %
\endisadelimproof
%
\isatagproof
\isacommand{using}\isamarkupfalse%
\ assms\ \isanewline
lm{\isadigit{2}}{\isadigit{7}}\ lm{\isadigit{3}}{\isadigit{2}}\ \isacommand{by}\isamarkupfalse%
\ auto%
\endisatagproof
{\isafoldproof}%
%
\isadelimproof
%
\endisadelimproof
\isanewline
\isanewline
\isacommand{lemma}\isamarkupfalse%
\ lm{\isadigit{3}}{\isadigit{3}}{\isacharcolon}\ \isakeyword{assumes}\ {\isachardoublequoteopen}a\ {\isasymin}\ injectionsUniverse{\isachardoublequoteclose}\ \isakeyword{shows}\ {\isachardoublequoteopen}a\ {\isasymin}\ injections\ {\isacharparenleft}Domain\ a{\isacharparenright}\ {\isacharparenleft}Range\ a{\isacharparenright}{\isachardoublequoteclose}\isanewline
%
\isadelimproof
%
\endisadelimproof
%
\isatagproof
\isacommand{using}\isamarkupfalse%
\ assms\ \isacommand{by}\isamarkupfalse%
\ {\isacharparenleft}metis\ {\isacharparenleft}lifting{\isacharparenright}\ injectionsI\ mem{\isacharunderscore}Collect{\isacharunderscore}eq\ order{\isacharunderscore}refl{\isacharparenright}%
\endisatagproof
{\isafoldproof}%
%
\isadelimproof
\isanewline
%
\endisadelimproof
\isanewline
\isacommand{lemma}\isamarkupfalse%
\ lm{\isadigit{3}}{\isadigit{4}}{\isacharcolon}\ \isakeyword{assumes}\ {\isachardoublequoteopen}a\ {\isasymin}\ allocationsUniverse{\isachardoublequoteclose}\ \isakeyword{shows}\ {\isachardoublequoteopen}a\ {\isasymin}\ possibleAllocationsRel\ {\isacharparenleft}Domain\ a{\isacharparenright}\ {\isacharparenleft}{\isasymUnion}\ {\isacharparenleft}Range\ a{\isacharparenright}{\isacharparenright}{\isachardoublequoteclose}\isanewline
%
\isadelimproof
%
\endisadelimproof
%
\isatagproof
\isacommand{proof}\isamarkupfalse%
\ {\isacharminus}\isanewline
\isacommand{let}\isamarkupfalse%
\ {\isacharquery}r{\isacharequal}Range\ \isacommand{let}\isamarkupfalse%
\ {\isacharquery}p{\isacharequal}is{\isacharunderscore}partition\ \isacommand{let}\isamarkupfalse%
\ {\isacharquery}P{\isacharequal}all{\isacharunderscore}partitions\ \isacommand{have}\isamarkupfalse%
\ {\isachardoublequoteopen}{\isacharquery}p\ {\isacharparenleft}{\isacharquery}r\ a{\isacharparenright}{\isachardoublequoteclose}\ \isacommand{using}\isamarkupfalse%
\ \isanewline
assms\ lm{\isadigit{2}}{\isadigit{2}}\ Int{\isacharunderscore}iff\ \isacommand{by}\isamarkupfalse%
\ blast\ \isacommand{then}\isamarkupfalse%
\ \isacommand{have}\isamarkupfalse%
\ {\isachardoublequoteopen}{\isacharquery}r\ a\ {\isasymin}\ {\isacharquery}P\ {\isacharparenleft}{\isasymUnion}\ {\isacharparenleft}{\isacharquery}r\ a{\isacharparenright}{\isacharparenright}{\isachardoublequoteclose}\ \isacommand{unfolding}\isamarkupfalse%
\ all{\isacharunderscore}partitions{\isacharunderscore}def\ \isanewline
\isacommand{using}\isamarkupfalse%
\ is{\isacharunderscore}partition{\isacharunderscore}of{\isacharunderscore}def\ \ mem{\isacharunderscore}Collect{\isacharunderscore}eq\ \isacommand{by}\isamarkupfalse%
\ {\isacharparenleft}metis{\isacharparenright}\ \isacommand{then}\isamarkupfalse%
\ \isacommand{show}\isamarkupfalse%
\ {\isacharquery}thesis\ \isacommand{using}\isamarkupfalse%
\ \isanewline
assms\ IntI\ Int{\isacharunderscore}lower{\isadigit{1}}\ equalityE\ lm{\isadigit{1}}{\isadigit{9}}\ mem{\isacharunderscore}Collect{\isacharunderscore}eq\ set{\isacharunderscore}rev{\isacharunderscore}mp\ \isacommand{by}\isamarkupfalse%
\ {\isacharparenleft}metis\ {\isacharparenleft}lifting{\isacharcomma}\ no{\isacharunderscore}types{\isacharparenright}{\isacharparenright}\isanewline
\isacommand{qed}\isamarkupfalse%
%
\endisatagproof
{\isafoldproof}%
%
\isadelimproof
\isanewline
%
\endisadelimproof
\isanewline
\isacommand{lemma}\isamarkupfalse%
\ lm{\isadigit{3}}{\isadigit{6}}{\isacharcolon}\ {\isachardoublequoteopen}{\isacharbraceleft}X{\isacharbraceright}\ {\isasymin}\ partitionsUniverse\ {\isacharequal}\ {\isacharparenleft}X\ {\isasymnoteq}\ {\isacharbraceleft}{\isacharbraceright}{\isacharparenright}{\isachardoublequoteclose}%
\isadelimproof
\ %
\endisadelimproof
%
\isatagproof
\isacommand{using}\isamarkupfalse%
\ is{\isacharunderscore}partition{\isacharunderscore}def\ \isacommand{by}\isamarkupfalse%
\ fastforce%
\endisatagproof
{\isafoldproof}%
%
\isadelimproof
%
\endisadelimproof
\isanewline
\isanewline
\isacommand{lemma}\isamarkupfalse%
\ lm{\isadigit{3}}{\isadigit{6}}b{\isacharcolon}\ {\isachardoublequoteopen}{\isacharbraceleft}{\isacharparenleft}x{\isacharcomma}\ X{\isacharparenright}{\isacharbraceright}\ {\isacharminus}\ {\isacharbraceleft}{\isacharparenleft}x{\isacharcomma}\ {\isacharbraceleft}{\isacharbraceright}{\isacharparenright}{\isacharbraceright}\ {\isasymin}\ partitionValuedUniverse{\isachardoublequoteclose}%
\isadelimproof
\ %
\endisadelimproof
%
\isatagproof
\isacommand{using}\isamarkupfalse%
\ lm{\isadigit{3}}{\isadigit{6}}\ \isacommand{by}\isamarkupfalse%
\ auto%
\endisatagproof
{\isafoldproof}%
%
\isadelimproof
%
\endisadelimproof
\isanewline
\isanewline
\isacommand{lemma}\isamarkupfalse%
\ {\isachardoublequoteopen}runiq\ {\isacharbraceleft}{\isacharparenleft}x{\isacharcomma}X{\isacharparenright}{\isacharbraceright}{\isachardoublequoteclose}\ \isanewline
%
\isadelimproof
%
\endisadelimproof
%
\isatagproof
\isacommand{by}\isamarkupfalse%
\ {\isacharparenleft}metis\ runiq{\isacharunderscore}singleton{\isacharunderscore}rel{\isacharparenright}%
\endisatagproof
{\isafoldproof}%
%
\isadelimproof
\isanewline
%
\endisadelimproof
\isanewline
\isacommand{lemma}\isamarkupfalse%
\ lm{\isadigit{3}}{\isadigit{7}}{\isacharcolon}\ {\isachardoublequoteopen}{\isacharbraceleft}{\isacharparenleft}x{\isacharcomma}\ X{\isacharparenright}{\isacharbraceright}\ {\isasymin}\ injectionsUniverse{\isachardoublequoteclose}%
\isadelimproof
\ %
\endisadelimproof
%
\isatagproof
\isacommand{unfolding}\isamarkupfalse%
\ runiq{\isacharunderscore}basic\ \isacommand{using}\isamarkupfalse%
\ runiq{\isacharunderscore}singleton{\isacharunderscore}rel\ \isacommand{by}\isamarkupfalse%
\ blast%
\endisatagproof
{\isafoldproof}%
%
\isadelimproof
%
\endisadelimproof
\isanewline
\isanewline
\isacommand{lemma}\isamarkupfalse%
\ lm{\isadigit{3}}{\isadigit{8}}{\isacharcolon}\ {\isachardoublequoteopen}{\isacharbraceleft}{\isacharparenleft}x{\isacharcomma}X{\isacharparenright}{\isacharbraceright}\ {\isacharminus}\ {\isacharbraceleft}{\isacharparenleft}x{\isacharcomma}{\isacharbraceleft}{\isacharbraceright}{\isacharparenright}{\isacharbraceright}\ {\isasymin}\ allocationsUniverse{\isachardoublequoteclose}%
\isadelimproof
\ %
\endisadelimproof
%
\isatagproof
\isacommand{using}\isamarkupfalse%
\ lm{\isadigit{3}}{\isadigit{6}}b\ lm{\isadigit{3}}{\isadigit{7}}\ lm{\isadigit{2}}{\isadigit{7}}\ Int{\isacharunderscore}iff\ \isacommand{by}\isamarkupfalse%
\ {\isacharparenleft}metis\ {\isacharparenleft}no{\isacharunderscore}types{\isacharparenright}{\isacharparenright}%
\endisatagproof
{\isafoldproof}%
%
\isadelimproof
%
\endisadelimproof
\isanewline
\isanewline
\isacommand{lemma}\isamarkupfalse%
\ \isakeyword{assumes}\ {\isachardoublequoteopen}is{\isacharunderscore}partition\ Y{\isachardoublequoteclose}\ {\isachardoublequoteopen}X\ {\isasymsubseteq}\ Y{\isachardoublequoteclose}\ \isakeyword{shows}\ {\isachardoublequoteopen}is{\isacharunderscore}partition\ X{\isachardoublequoteclose}%
\isadelimproof
\ %
\endisadelimproof
%
\isatagproof
\isacommand{using}\isamarkupfalse%
\ assms\ subset{\isacharunderscore}is{\isacharunderscore}partition\isanewline
\isacommand{by}\isamarkupfalse%
\ {\isacharparenleft}metis{\isacharparenleft}no{\isacharunderscore}types{\isacharparenright}{\isacharparenright}%
\endisatagproof
{\isafoldproof}%
%
\isadelimproof
%
\endisadelimproof
\isanewline
\isanewline
\isacommand{lemma}\isamarkupfalse%
\ lm{\isadigit{4}}{\isadigit{1}}{\isacharcolon}\ \isakeyword{assumes}\ {\isachardoublequoteopen}is{\isacharunderscore}partition\ PP{\isachardoublequoteclose}\ {\isachardoublequoteopen}is{\isacharunderscore}partition\ {\isacharparenleft}Union\ PP{\isacharparenright}{\isachardoublequoteclose}\ \isakeyword{shows}\ {\isachardoublequoteopen}is{\isacharunderscore}partition\ {\isacharparenleft}Union\ {\isacharbackquote}\ PP{\isacharparenright}{\isachardoublequoteclose}\isanewline
%
\isadelimproof
%
\endisadelimproof
%
\isatagproof
\isacommand{proof}\isamarkupfalse%
\ {\isacharminus}\isanewline
\isacommand{let}\isamarkupfalse%
\ {\isacharquery}p{\isacharequal}is{\isacharunderscore}partition\ \isacommand{let}\isamarkupfalse%
\ {\isacharquery}U{\isacharequal}Union\ \isacommand{let}\isamarkupfalse%
\ {\isacharquery}P{\isadigit{2}}{\isacharequal}{\isachardoublequoteopen}{\isacharquery}U\ PP{\isachardoublequoteclose}\ \isacommand{let}\isamarkupfalse%
\ {\isacharquery}P{\isadigit{1}}{\isacharequal}{\isachardoublequoteopen}{\isacharquery}U\ {\isacharbackquote}\ PP{\isachardoublequoteclose}\ \isacommand{have}\isamarkupfalse%
\ \isanewline
{\isadigit{0}}{\isacharcolon}\ {\isachardoublequoteopen}{\isasymforall}\ X{\isasymin}{\isacharquery}P{\isadigit{1}}{\isachardot}\ {\isasymforall}\ Y\ {\isasymin}\ {\isacharquery}P{\isadigit{1}}{\isachardot}\ {\isacharparenleft}X\ {\isasyminter}\ Y\ {\isacharequal}\ {\isacharbraceleft}{\isacharbraceright}\ {\isasymlongrightarrow}\ X\ {\isasymnoteq}\ Y{\isacharparenright}{\isachardoublequoteclose}\ \isacommand{using}\isamarkupfalse%
\ assms\ is{\isacharunderscore}partition{\isacharunderscore}def\ Int{\isacharunderscore}absorb\ \isanewline
Int{\isacharunderscore}empty{\isacharunderscore}left\ UnionI\ Union{\isacharunderscore}disjoint\ ex{\isacharunderscore}in{\isacharunderscore}conv\ imageE\ \isacommand{by}\isamarkupfalse%
\ {\isacharparenleft}metis\ {\isacharparenleft}hide{\isacharunderscore}lams{\isacharcomma}\ no{\isacharunderscore}types{\isacharparenright}{\isacharparenright}\isanewline
\isacommand{{\isacharbraceleft}}\isamarkupfalse%
\isanewline
\ \ \isacommand{fix}\isamarkupfalse%
\ X\ Y\ \isacommand{assume}\isamarkupfalse%
\ \isanewline
\ \ {\isadigit{2}}{\isacharcolon}\ {\isachardoublequoteopen}X\ {\isasymin}\ {\isacharquery}P{\isadigit{1}}\ {\isacharampersand}\ Y{\isasymin}{\isacharquery}P{\isadigit{1}}\ {\isacharampersand}\ X\ {\isasymnoteq}\ Y{\isachardoublequoteclose}\isanewline
\ \ \isacommand{then}\isamarkupfalse%
\ \isacommand{obtain}\isamarkupfalse%
\ XX\ YY\ \isakeyword{where}\ \isanewline
\ \ {\isadigit{1}}{\isacharcolon}\ {\isachardoublequoteopen}X\ {\isacharequal}\ {\isacharquery}U\ XX\ {\isacharampersand}\ Y{\isacharequal}{\isacharquery}U\ YY\ {\isacharampersand}\ XX{\isasymin}PP\ {\isacharampersand}\ YY{\isasymin}PP{\isachardoublequoteclose}\ \isacommand{by}\isamarkupfalse%
\ blast\isanewline
\ \ \isacommand{then}\isamarkupfalse%
\ \isacommand{have}\isamarkupfalse%
\ {\isachardoublequoteopen}XX\ {\isasymsubseteq}\ Union\ PP\ {\isacharampersand}\ YY\ {\isasymsubseteq}\ Union\ PP\ {\isacharampersand}\ XX\ {\isasyminter}\ YY\ {\isacharequal}\ {\isacharbraceleft}{\isacharbraceright}{\isachardoublequoteclose}\ \isanewline
\ \ \isacommand{using}\isamarkupfalse%
\ {\isadigit{2}}\ {\isadigit{1}}\ is{\isacharunderscore}partition{\isacharunderscore}def\ assms{\isacharparenleft}{\isadigit{1}}{\isacharparenright}\ Sup{\isacharunderscore}upper\ \isacommand{by}\isamarkupfalse%
\ metis\isanewline
\ \ \isacommand{then}\isamarkupfalse%
\ \isacommand{moreover}\isamarkupfalse%
\ \isacommand{have}\isamarkupfalse%
\ {\isachardoublequoteopen}{\isasymforall}\ x{\isasymin}XX{\isachardot}\ {\isasymforall}\ y{\isasymin}YY{\isachardot}\ x\ {\isasyminter}\ y\ {\isacharequal}\ {\isacharbraceleft}{\isacharbraceright}{\isachardoublequoteclose}\ \isacommand{using}\isamarkupfalse%
\ {\isadigit{1}}\ assms{\isacharparenleft}{\isadigit{2}}{\isacharparenright}\ is{\isacharunderscore}partition{\isacharunderscore}def\ \isanewline
\isacommand{by}\isamarkupfalse%
\ {\isacharparenleft}metis\ IntI\ empty{\isacharunderscore}iff\ subsetCE{\isacharparenright}\isanewline
\ \ \isacommand{ultimately}\isamarkupfalse%
\ \isacommand{have}\isamarkupfalse%
\ {\isachardoublequoteopen}X\ {\isasyminter}\ Y{\isacharequal}{\isacharbraceleft}{\isacharbraceright}{\isachardoublequoteclose}\ \isacommand{using}\isamarkupfalse%
\ assms\ {\isadigit{0}}\ {\isadigit{1}}\ {\isadigit{2}}\ is{\isacharunderscore}partition{\isacharunderscore}def\ \isacommand{by}\isamarkupfalse%
\ auto\isanewline
\isacommand{{\isacharbraceright}}\isamarkupfalse%
\isanewline
\isacommand{then}\isamarkupfalse%
\ \isacommand{show}\isamarkupfalse%
\ {\isacharquery}thesis\ \isacommand{using}\isamarkupfalse%
\ {\isadigit{0}}\ is{\isacharunderscore}partition{\isacharunderscore}def\ \isacommand{by}\isamarkupfalse%
\ metis\isanewline
\isacommand{qed}\isamarkupfalse%
%
\endisatagproof
{\isafoldproof}%
%
\isadelimproof
\isanewline
%
\endisadelimproof
\isanewline
\isacommand{lemma}\isamarkupfalse%
\ lm{\isadigit{4}}{\isadigit{3}}{\isacharcolon}\ \isakeyword{assumes}\ {\isachardoublequoteopen}a\ {\isasymin}\ allocationsUniverse{\isachardoublequoteclose}\ \isakeyword{shows}\ \isanewline
{\isachardoublequoteopen}{\isacharparenleft}a\ {\isacharminus}\ {\isacharparenleft}{\isacharparenleft}X{\isasymunion}{\isacharbraceleft}i{\isacharbraceright}{\isacharparenright}{\isasymtimes}{\isacharparenleft}Range\ a{\isacharparenright}{\isacharparenright}{\isacharparenright}\ {\isasymunion}\ {\isacharparenleft}{\isacharbraceleft}{\isacharparenleft}i{\isacharcomma}\ {\isasymUnion}\ {\isacharparenleft}a{\isacharbackquote}{\isacharbackquote}{\isacharparenleft}X\ {\isasymunion}\ {\isacharbraceleft}i{\isacharbraceright}{\isacharparenright}{\isacharparenright}{\isacharparenright}{\isacharbraceright}\ {\isacharminus}\ {\isacharbraceleft}{\isacharparenleft}i{\isacharcomma}{\isacharbraceleft}{\isacharbraceright}{\isacharparenright}{\isacharbraceright}{\isacharparenright}\ {\isasymin}\ allocationsUniverse\ {\isacharampersand}\ \isanewline
{\isasymUnion}\ {\isacharparenleft}Range\ {\isacharparenleft}{\isacharparenleft}a\ {\isacharminus}\ {\isacharparenleft}{\isacharparenleft}X{\isasymunion}{\isacharbraceleft}i{\isacharbraceright}{\isacharparenright}{\isasymtimes}{\isacharparenleft}Range\ a{\isacharparenright}{\isacharparenright}{\isacharparenright}\ {\isasymunion}\ {\isacharparenleft}{\isacharbraceleft}{\isacharparenleft}i{\isacharcomma}\ {\isasymUnion}\ {\isacharparenleft}a{\isacharbackquote}{\isacharbackquote}{\isacharparenleft}X\ {\isasymunion}\ {\isacharbraceleft}i{\isacharbraceright}{\isacharparenright}{\isacharparenright}{\isacharparenright}{\isacharbraceright}\ {\isacharminus}\ {\isacharbraceleft}{\isacharparenleft}i{\isacharcomma}{\isacharbraceleft}{\isacharbraceright}{\isacharparenright}{\isacharbraceright}{\isacharparenright}{\isacharparenright}{\isacharparenright}\ {\isacharequal}\ {\isasymUnion}{\isacharparenleft}Range\ a{\isacharparenright}{\isachardoublequoteclose}\isanewline
%
\isadelimproof
%
\endisadelimproof
%
\isatagproof
\isacommand{proof}\isamarkupfalse%
\ {\isacharminus}\isanewline
\ \ \isacommand{let}\isamarkupfalse%
\ {\isacharquery}d{\isacharequal}Domain\ \isacommand{let}\isamarkupfalse%
\ {\isacharquery}r{\isacharequal}Range\ \isacommand{let}\isamarkupfalse%
\ {\isacharquery}U{\isacharequal}Union\ \isacommand{let}\isamarkupfalse%
\ {\isacharquery}p{\isacharequal}is{\isacharunderscore}partition\ \isacommand{let}\isamarkupfalse%
\ {\isacharquery}P{\isacharequal}partitionsUniverse\ \isacommand{let}\isamarkupfalse%
\ {\isacharquery}u{\isacharequal}runiq\ \isanewline
\ \ \isacommand{let}\isamarkupfalse%
\ {\isacharquery}Xi{\isacharequal}{\isachardoublequoteopen}X\ {\isasymunion}\ {\isacharbraceleft}i{\isacharbraceright}{\isachardoublequoteclose}\ \isacommand{let}\isamarkupfalse%
\ {\isacharquery}b{\isacharequal}{\isachardoublequoteopen}{\isacharquery}Xi\ {\isasymtimes}\ {\isacharparenleft}{\isacharquery}r\ a{\isacharparenright}{\isachardoublequoteclose}\ \isacommand{let}\isamarkupfalse%
\ {\isacharquery}a{\isadigit{1}}{\isacharequal}{\isachardoublequoteopen}a\ {\isacharminus}\ {\isacharquery}b{\isachardoublequoteclose}\ \isacommand{let}\isamarkupfalse%
\ {\isacharquery}Yi{\isacharequal}{\isachardoublequoteopen}a{\isacharbackquote}{\isacharbackquote}{\isacharquery}Xi{\isachardoublequoteclose}\ \isacommand{let}\isamarkupfalse%
\ {\isacharquery}Y{\isacharequal}{\isachardoublequoteopen}{\isacharquery}U\ {\isacharquery}Yi{\isachardoublequoteclose}\ \isanewline
\ \ \isacommand{let}\isamarkupfalse%
\ {\isacharquery}A{\isadigit{2}}{\isacharequal}{\isachardoublequoteopen}{\isacharbraceleft}{\isacharparenleft}i{\isacharcomma}\ {\isacharquery}Y{\isacharparenright}{\isacharbraceright}{\isachardoublequoteclose}\ \isacommand{let}\isamarkupfalse%
\ {\isacharquery}a{\isadigit{3}}{\isacharequal}{\isachardoublequoteopen}{\isacharbraceleft}{\isacharparenleft}i{\isacharcomma}{\isacharbraceleft}{\isacharbraceright}{\isacharparenright}{\isacharbraceright}{\isachardoublequoteclose}\ \isacommand{let}\isamarkupfalse%
\ {\isacharquery}a{\isadigit{2}}{\isacharequal}{\isachardoublequoteopen}{\isacharquery}A{\isadigit{2}}\ {\isacharminus}\ {\isacharquery}a{\isadigit{3}}{\isachardoublequoteclose}\ \isacommand{let}\isamarkupfalse%
\ {\isacharquery}aa{\isadigit{1}}{\isacharequal}{\isachardoublequoteopen}a\ outside\ {\isacharquery}Xi{\isachardoublequoteclose}\ \isanewline
\ \ \isacommand{let}\isamarkupfalse%
\ {\isacharquery}c{\isacharequal}{\isachardoublequoteopen}{\isacharquery}a{\isadigit{1}}\ {\isasymunion}\ {\isacharquery}a{\isadigit{2}}{\isachardoublequoteclose}\ \isacommand{let}\isamarkupfalse%
\ {\isacharquery}t{\isadigit{1}}{\isacharequal}{\isachardoublequoteopen}{\isacharquery}c\ {\isasymin}\ allocationsUniverse{\isachardoublequoteclose}\ \isacommand{have}\isamarkupfalse%
\ \isanewline
\ \ {\isadigit{7}}{\isacharcolon}\ {\isachardoublequoteopen}{\isacharquery}U{\isacharparenleft}{\isacharquery}r{\isacharparenleft}{\isacharquery}a{\isadigit{1}}{\isasymunion}{\isacharquery}a{\isadigit{2}}{\isacharparenright}{\isacharparenright}{\isacharequal}{\isacharquery}U{\isacharparenleft}{\isacharquery}r\ {\isacharquery}a{\isadigit{1}}{\isacharparenright}\ {\isasymunion}\ {\isacharparenleft}{\isacharquery}U{\isacharparenleft}{\isacharquery}r\ {\isacharquery}a{\isadigit{2}}{\isacharparenright}{\isacharparenright}{\isachardoublequoteclose}\ \isacommand{by}\isamarkupfalse%
\ {\isacharparenleft}metis\ Range{\isacharunderscore}Un{\isacharunderscore}eq\ Union{\isacharunderscore}Un{\isacharunderscore}distrib{\isacharparenright}\ \isacommand{have}\isamarkupfalse%
\ \isanewline
\ \ {\isadigit{5}}{\isacharcolon}\ {\isachardoublequoteopen}{\isacharquery}U{\isacharparenleft}{\isacharquery}r\ a{\isacharparenright}\ {\isasymsubseteq}\ {\isacharquery}U{\isacharparenleft}{\isacharquery}r\ {\isacharquery}a{\isadigit{1}}{\isacharparenright}\ {\isasymunion}\ {\isacharquery}U{\isacharparenleft}a{\isacharbackquote}{\isacharbackquote}{\isacharquery}Xi{\isacharparenright}\ {\isacharampersand}\ {\isacharquery}U{\isacharparenleft}{\isacharquery}r\ {\isacharquery}a{\isadigit{1}}{\isacharparenright}\ {\isasymunion}\ {\isacharquery}U{\isacharparenleft}{\isacharquery}r\ {\isacharquery}a{\isadigit{2}}{\isacharparenright}\ {\isasymsubseteq}\ {\isacharquery}U{\isacharparenleft}{\isacharquery}r\ a{\isacharparenright}{\isachardoublequoteclose}\ \isacommand{by}\isamarkupfalse%
\ blast\ \isacommand{have}\isamarkupfalse%
\isanewline
\ \ {\isadigit{1}}{\isacharcolon}\ {\isachardoublequoteopen}{\isacharquery}u\ a\ {\isacharampersand}\ {\isacharquery}u\ {\isacharparenleft}a{\isacharcircum}{\isacharminus}{\isadigit{1}}{\isacharparenright}\ {\isacharampersand}\ {\isacharquery}p\ {\isacharparenleft}{\isacharquery}r\ a{\isacharparenright}\ {\isacharampersand}\ {\isacharquery}r\ {\isacharquery}a{\isadigit{1}}\ {\isasymsubseteq}\ {\isacharquery}r\ a\ {\isacharampersand}\ {\isacharquery}Yi\ {\isasymsubseteq}\ {\isacharquery}r\ a{\isachardoublequoteclose}\ \isanewline
\ \ \isacommand{using}\isamarkupfalse%
\ assms\ Int{\isacharunderscore}iff\ lm{\isadigit{2}}{\isadigit{2}}\ mem{\isacharunderscore}Collect{\isacharunderscore}eq\ \isacommand{by}\isamarkupfalse%
\ auto\ \isacommand{then}\isamarkupfalse%
\ \isacommand{have}\isamarkupfalse%
\ \isanewline
\ \ {\isadigit{2}}{\isacharcolon}\ {\isachardoublequoteopen}{\isacharquery}p\ {\isacharparenleft}{\isacharquery}r\ {\isacharquery}a{\isadigit{1}}{\isacharparenright}\ {\isacharampersand}\ {\isacharquery}p\ {\isacharquery}Yi{\isachardoublequoteclose}\ \isacommand{using}\isamarkupfalse%
\ subset{\isacharunderscore}is{\isacharunderscore}partition\ \isacommand{by}\isamarkupfalse%
\ metis\ \isacommand{have}\isamarkupfalse%
\ \isanewline
\ \ {\isachardoublequoteopen}{\isacharquery}a{\isadigit{1}}\ {\isasymin}\ allocationsUniverse\ {\isacharampersand}\ {\isacharquery}a{\isadigit{2}}\ {\isasymin}\ allocationsUniverse{\isachardoublequoteclose}\ \isacommand{using}\isamarkupfalse%
\ lm{\isadigit{3}}{\isadigit{8}}\ assms{\isacharparenleft}{\isadigit{1}}{\isacharparenright}\ lm{\isadigit{3}}{\isadigit{5}}\ \isacommand{by}\isamarkupfalse%
\ fastforce\ \isacommand{then}\isamarkupfalse%
\ \isacommand{have}\isamarkupfalse%
\ \isanewline
\ \ {\isachardoublequoteopen}{\isacharparenleft}{\isacharquery}a{\isadigit{1}}\ {\isacharequal}\ {\isacharbraceleft}{\isacharbraceright}\ {\isasymor}\ {\isacharquery}a{\isadigit{2}}\ {\isacharequal}\ {\isacharbraceleft}{\isacharbraceright}{\isacharparenright}{\isasymlongrightarrow}\ {\isacharquery}t{\isadigit{1}}{\isachardoublequoteclose}\ \isacommand{using}\isamarkupfalse%
\ Un{\isacharunderscore}empty{\isacharunderscore}left\ \isacommand{by}\isamarkupfalse%
\ {\isacharparenleft}metis\ {\isacharparenleft}lifting{\isacharcomma}\ no{\isacharunderscore}types{\isacharparenright}\ Un{\isacharunderscore}absorb{\isadigit{2}}\ empty{\isacharunderscore}subsetI{\isacharparenright}\ \isacommand{moreover}\isamarkupfalse%
\ \isacommand{have}\isamarkupfalse%
\ \isanewline
\ \ {\isachardoublequoteopen}{\isacharparenleft}{\isacharquery}a{\isadigit{1}}\ {\isacharequal}\ {\isacharbraceleft}{\isacharbraceright}\ {\isasymor}\ {\isacharquery}a{\isadigit{2}}\ {\isacharequal}\ {\isacharbraceleft}{\isacharbraceright}{\isacharparenright}{\isasymlongrightarrow}\ {\isacharquery}U\ {\isacharparenleft}{\isacharquery}r\ a{\isacharparenright}\ {\isacharequal}\ {\isacharquery}U\ {\isacharparenleft}{\isacharquery}r\ {\isacharquery}a{\isadigit{1}}{\isacharparenright}\ {\isasymunion}\ {\isacharquery}U\ {\isacharparenleft}{\isacharquery}r\ {\isacharquery}a{\isadigit{2}}{\isacharparenright}{\isachardoublequoteclose}\ \isacommand{by}\isamarkupfalse%
\ fast\ \isacommand{ultimately}\isamarkupfalse%
\ \isacommand{have}\isamarkupfalse%
\ \isanewline
\ \ {\isadigit{3}}{\isacharcolon}\ {\isachardoublequoteopen}{\isacharparenleft}{\isacharquery}a{\isadigit{1}}\ {\isacharequal}\ {\isacharbraceleft}{\isacharbraceright}\ {\isasymor}\ {\isacharquery}a{\isadigit{2}}\ {\isacharequal}\ {\isacharbraceleft}{\isacharbraceright}{\isacharparenright}{\isasymlongrightarrow}\ {\isacharquery}thesis{\isachardoublequoteclose}\ \isacommand{using}\isamarkupfalse%
\ {\isadigit{7}}\ \isacommand{by}\isamarkupfalse%
\ presburger\isanewline
\ \ \isacommand{{\isacharbraceleft}}\isamarkupfalse%
\ \isanewline
\ \ \ \ \isacommand{assume}\isamarkupfalse%
\isanewline
\ \ \ \ {\isadigit{0}}{\isacharcolon}\ {\isachardoublequoteopen}{\isacharquery}a{\isadigit{1}}{\isasymnoteq}{\isacharbraceleft}{\isacharbraceright}\ {\isacharampersand}\ {\isacharquery}a{\isadigit{2}}{\isasymnoteq}{\isacharbraceleft}{\isacharbraceright}{\isachardoublequoteclose}\ \isacommand{then}\isamarkupfalse%
\ \isacommand{have}\isamarkupfalse%
\ {\isachardoublequoteopen}{\isacharquery}r\ {\isacharquery}a{\isadigit{2}}{\isasymsupseteq}{\isacharbraceleft}{\isacharquery}Y{\isacharbraceright}{\isachardoublequoteclose}\ \isacommand{using}\isamarkupfalse%
\ Diff{\isacharunderscore}cancel\ Range{\isacharunderscore}insert\ empty{\isacharunderscore}subsetI\ \isanewline
\ \ \ \ insert{\isacharunderscore}Diff{\isacharunderscore}single\ insert{\isacharunderscore}iff\ insert{\isacharunderscore}subset\ \isacommand{by}\isamarkupfalse%
\ {\isacharparenleft}metis\ {\isacharparenleft}hide{\isacharunderscore}lams{\isacharcomma}\ no{\isacharunderscore}types{\isacharparenright}{\isacharparenright}\ \isacommand{then}\isamarkupfalse%
\ \isacommand{have}\isamarkupfalse%
\ \isanewline
\ \ \ \ {\isadigit{6}}{\isacharcolon}\ {\isachardoublequoteopen}{\isacharquery}U\ {\isacharparenleft}{\isacharquery}r\ a{\isacharparenright}\ {\isacharequal}\ {\isacharquery}U\ {\isacharparenleft}{\isacharquery}r\ {\isacharquery}a{\isadigit{1}}{\isacharparenright}\ {\isasymunion}\ {\isacharquery}U\ {\isacharparenleft}{\isacharquery}r\ {\isacharquery}a{\isadigit{2}}{\isacharparenright}{\isachardoublequoteclose}\ \isacommand{using}\isamarkupfalse%
\ {\isadigit{5}}\ \isacommand{by}\isamarkupfalse%
\ blast\isanewline
\ \ \ \ \isacommand{have}\isamarkupfalse%
\ {\isachardoublequoteopen}{\isacharquery}r\ {\isacharquery}a{\isadigit{1}}\ {\isasymnoteq}\ {\isacharbraceleft}{\isacharbraceright}\ {\isacharampersand}\ {\isacharquery}r\ {\isacharquery}a{\isadigit{2}}\ {\isasymnoteq}\ {\isacharbraceleft}{\isacharbraceright}{\isachardoublequoteclose}\ \isacommand{using}\isamarkupfalse%
\ {\isadigit{0}}\ \isacommand{by}\isamarkupfalse%
\ auto\isanewline
\ \ \ \ \isacommand{moreover}\isamarkupfalse%
\ \isacommand{have}\isamarkupfalse%
\ {\isachardoublequoteopen}{\isacharquery}r\ {\isacharquery}a{\isadigit{1}}\ {\isasymsubseteq}\ a{\isacharbackquote}{\isacharbackquote}{\isacharparenleft}{\isacharquery}d\ {\isacharquery}a{\isadigit{1}}{\isacharparenright}{\isachardoublequoteclose}\ \isacommand{using}\isamarkupfalse%
\ assms\ \isacommand{by}\isamarkupfalse%
\ blast\isanewline
\ \ \ \ \isacommand{moreover}\isamarkupfalse%
\ \isacommand{have}\isamarkupfalse%
\ {\isachardoublequoteopen}{\isacharquery}Yi\ {\isasyminter}\ {\isacharparenleft}a{\isacharbackquote}{\isacharbackquote}{\isacharparenleft}{\isacharquery}d\ a\ {\isacharminus}\ {\isacharquery}Xi{\isacharparenright}{\isacharparenright}\ {\isacharequal}\ {\isacharbraceleft}{\isacharbraceright}{\isachardoublequoteclose}\ \isacommand{using}\isamarkupfalse%
\ assms\ {\isadigit{0}}\ {\isadigit{1}}\ lm{\isadigit{4}}{\isadigit{0}}\isanewline
\ \ \ \ \isacommand{by}\isamarkupfalse%
\ {\isacharparenleft}metis\ Diff{\isacharunderscore}disjoint{\isacharparenright}\isanewline
\ \ \ \ \isacommand{ultimately}\isamarkupfalse%
\ \isacommand{moreover}\isamarkupfalse%
\ \isacommand{have}\isamarkupfalse%
\ {\isachardoublequoteopen}{\isacharquery}r\ {\isacharquery}a{\isadigit{1}}\ {\isasyminter}\ {\isacharquery}Yi\ {\isacharequal}\ {\isacharbraceleft}{\isacharbraceright}\ {\isacharampersand}\ {\isacharquery}Yi\ {\isasymnoteq}\ {\isacharbraceleft}{\isacharbraceright}{\isachardoublequoteclose}\ \isacommand{by}\isamarkupfalse%
\ blast\isanewline
\ \ \ \ \isacommand{ultimately}\isamarkupfalse%
\ \isacommand{moreover}\isamarkupfalse%
\ \isacommand{have}\isamarkupfalse%
\ {\isachardoublequoteopen}{\isacharquery}p\ {\isacharbraceleft}{\isacharquery}r\ {\isacharquery}a{\isadigit{1}}{\isacharcomma}\ {\isacharquery}Yi{\isacharbraceright}{\isachardoublequoteclose}\ \isacommand{unfolding}\isamarkupfalse%
\ is{\isacharunderscore}partition{\isacharunderscore}def\ \isacommand{using}\isamarkupfalse%
\ \ \isanewline
IntI\ Int{\isacharunderscore}commute\ empty{\isacharunderscore}iff\ insert{\isacharunderscore}iff\ subsetI\ subset{\isacharunderscore}empty\ \isacommand{by}\isamarkupfalse%
\ metis\isanewline
\ \ \ \ \isacommand{moreover}\isamarkupfalse%
\ \isacommand{have}\isamarkupfalse%
\ {\isachardoublequoteopen}{\isacharquery}U\ {\isacharbraceleft}{\isacharquery}r\ {\isacharquery}a{\isadigit{1}}{\isacharcomma}\ {\isacharquery}Yi{\isacharbraceright}\ {\isasymsubseteq}\ {\isacharquery}r\ a{\isachardoublequoteclose}\ \isacommand{by}\isamarkupfalse%
\ auto\isanewline
\ \ \ \ \isacommand{then}\isamarkupfalse%
\ \isacommand{moreover}\isamarkupfalse%
\ \isacommand{have}\isamarkupfalse%
\ {\isachardoublequoteopen}{\isacharquery}p\ {\isacharparenleft}{\isacharquery}U\ {\isacharbraceleft}{\isacharquery}r\ {\isacharquery}a{\isadigit{1}}{\isacharcomma}\ {\isacharquery}Yi{\isacharbraceright}{\isacharparenright}{\isachardoublequoteclose}\ \isacommand{by}\isamarkupfalse%
\ {\isacharparenleft}metis\ {\isachardoublequoteopen}{\isadigit{1}}{\isachardoublequoteclose}\ Outside{\isacharunderscore}def\ subset{\isacharunderscore}is{\isacharunderscore}partition{\isacharparenright}\isanewline
\ \ \ \ \isacommand{ultimately}\isamarkupfalse%
\ \isacommand{moreover}\isamarkupfalse%
\ \isacommand{have}\isamarkupfalse%
\ {\isachardoublequoteopen}{\isacharquery}p\ {\isacharparenleft}{\isacharquery}U{\isacharbackquote}{\isacharbraceleft}{\isacharparenleft}{\isacharquery}r\ {\isacharquery}a{\isadigit{1}}{\isacharparenright}{\isacharcomma}\ {\isacharquery}Yi{\isacharbraceright}{\isacharparenright}{\isachardoublequoteclose}\ \isacommand{using}\isamarkupfalse%
\ lm{\isadigit{4}}{\isadigit{1}}\ \isacommand{by}\isamarkupfalse%
\ fast\isanewline
\ \ \ \ \isacommand{moreover}\isamarkupfalse%
\ \isacommand{have}\isamarkupfalse%
\ {\isachardoublequoteopen}{\isachardot}{\isachardot}{\isachardot}\ {\isacharequal}\ {\isacharbraceleft}{\isacharquery}U\ {\isacharparenleft}{\isacharquery}r\ {\isacharquery}a{\isadigit{1}}{\isacharparenright}{\isacharcomma}\ {\isacharquery}Y{\isacharbraceright}{\isachardoublequoteclose}\ \isacommand{by}\isamarkupfalse%
\ force\isanewline
\ \ \ \ \isacommand{ultimately}\isamarkupfalse%
\ \isacommand{moreover}\isamarkupfalse%
\ \isacommand{have}\isamarkupfalse%
\ {\isachardoublequoteopen}{\isasymforall}\ x\ {\isasymin}\ {\isacharquery}r\ {\isacharquery}a{\isadigit{1}}{\isachardot}\ {\isasymforall}\ y{\isasymin}{\isacharquery}Yi{\isachardot}\ x\ {\isasymnoteq}\ y{\isachardoublequoteclose}\ \isanewline
\ \ \ \ \isacommand{using}\isamarkupfalse%
\ IntI\ empty{\isacharunderscore}iff\ \isacommand{by}\isamarkupfalse%
\ metis\isanewline
\ \ \ \ \isacommand{ultimately}\isamarkupfalse%
\ \isacommand{moreover}\isamarkupfalse%
\ \isacommand{have}\isamarkupfalse%
\ {\isachardoublequoteopen}{\isasymforall}\ x\ {\isasymin}\ {\isacharquery}r\ {\isacharquery}a{\isadigit{1}}{\isachardot}\ {\isasymforall}\ y{\isasymin}{\isacharquery}Yi{\isachardot}\ x\ {\isasyminter}\ y\ {\isacharequal}\ {\isacharbraceleft}{\isacharbraceright}{\isachardoublequoteclose}\ \isacommand{using}\isamarkupfalse%
\ {\isadigit{0}}\ {\isadigit{1}}\ {\isadigit{2}}\ is{\isacharunderscore}partition{\isacharunderscore}def\isanewline
\ \ \ \ \isacommand{by}\isamarkupfalse%
\ {\isacharparenleft}metis\ set{\isacharunderscore}rev{\isacharunderscore}mp{\isacharparenright}\isanewline
\ \ \ \ \isacommand{ultimately}\isamarkupfalse%
\ \isacommand{have}\isamarkupfalse%
\ {\isachardoublequoteopen}{\isacharquery}U\ {\isacharparenleft}{\isacharquery}r\ {\isacharquery}a{\isadigit{1}}{\isacharparenright}\ {\isasyminter}\ {\isacharquery}Y\ {\isacharequal}\ {\isacharbraceleft}{\isacharbraceright}{\isachardoublequoteclose}\ \isacommand{using}\isamarkupfalse%
\ lm{\isadigit{4}}{\isadigit{2}}\ \isanewline
\isacommand{proof}\isamarkupfalse%
\ {\isacharminus}\isanewline
\ \ \isacommand{have}\isamarkupfalse%
\ {\isachardoublequoteopen}{\isasymforall}v{\isadigit{0}}{\isachardot}\ v{\isadigit{0}}\ {\isasymin}\ Range\ {\isacharparenleft}a\ {\isacharminus}\ {\isacharparenleft}X\ {\isasymunion}\ {\isacharbraceleft}i{\isacharbraceright}{\isacharparenright}\ {\isasymtimes}\ Range\ a{\isacharparenright}\ {\isasymlongrightarrow}\ {\isacharparenleft}{\isasymforall}v{\isadigit{1}}{\isachardot}\ v{\isadigit{1}}\ {\isasymin}\ a\ {\isacharbackquote}{\isacharbackquote}\ {\isacharparenleft}X\ {\isasymunion}\ {\isacharbraceleft}i{\isacharbraceright}{\isacharparenright}\ {\isasymlongrightarrow}\ v{\isadigit{0}}\ {\isasyminter}\ v{\isadigit{1}}\ {\isacharequal}\ {\isacharbraceleft}{\isacharbraceright}{\isacharparenright}{\isachardoublequoteclose}\ \isanewline
\isacommand{by}\isamarkupfalse%
\ {\isacharparenleft}metis\ {\isacharparenleft}no{\isacharunderscore}types{\isacharparenright}\ {\isacharbackquoteopen}{\isasymforall}x{\isasymin}Range\ {\isacharparenleft}a\ {\isacharminus}\ {\isacharparenleft}X\ {\isasymunion}\ {\isacharbraceleft}i{\isacharbraceright}{\isacharparenright}\ {\isasymtimes}\ Range\ a{\isacharparenright}{\isachardot}\ {\isasymforall}y{\isasymin}a\ {\isacharbackquote}{\isacharbackquote}\ {\isacharparenleft}X\ {\isasymunion}\ {\isacharbraceleft}i{\isacharbraceright}{\isacharparenright}{\isachardot}\ x\ {\isasyminter}\ y\ {\isacharequal}\ {\isacharbraceleft}{\isacharbraceright}{\isacharbackquoteclose}{\isacharparenright}\ \isanewline
\ \ \isacommand{thus}\isamarkupfalse%
\ {\isachardoublequoteopen}{\isasymUnion}Range\ {\isacharparenleft}a\ {\isacharminus}\ {\isacharparenleft}X\ {\isasymunion}\ {\isacharbraceleft}i{\isacharbraceright}{\isacharparenright}\ {\isasymtimes}\ Range\ a{\isacharparenright}\ {\isasyminter}\ {\isasymUnion}{\isacharparenleft}a\ {\isacharbackquote}{\isacharbackquote}\ {\isacharparenleft}X\ {\isasymunion}\ {\isacharbraceleft}i{\isacharbraceright}{\isacharparenright}{\isacharparenright}\ {\isacharequal}\ {\isacharbraceleft}{\isacharbraceright}{\isachardoublequoteclose}\ \isacommand{by}\isamarkupfalse%
\ blast\isanewline
\isacommand{qed}\isamarkupfalse%
\ \isacommand{then}\isamarkupfalse%
\ \isacommand{have}\isamarkupfalse%
\ \isanewline
\ \ \ \ {\isachardoublequoteopen}{\isacharquery}U\ {\isacharparenleft}{\isacharquery}r\ {\isacharquery}a{\isadigit{1}}{\isacharparenright}\ {\isasyminter}\ {\isacharparenleft}{\isacharquery}U\ {\isacharparenleft}{\isacharquery}r\ {\isacharquery}a{\isadigit{2}}{\isacharparenright}{\isacharparenright}\ {\isacharequal}\ {\isacharbraceleft}{\isacharbraceright}{\isachardoublequoteclose}\ \isacommand{by}\isamarkupfalse%
\ blast\isanewline
\ \ \ \ \isacommand{moreover}\isamarkupfalse%
\ \isacommand{have}\isamarkupfalse%
\ {\isachardoublequoteopen}{\isacharquery}d\ {\isacharquery}a{\isadigit{1}}\ {\isasyminter}\ {\isacharparenleft}{\isacharquery}d\ {\isacharquery}a{\isadigit{2}}{\isacharparenright}\ {\isacharequal}\ {\isacharbraceleft}{\isacharbraceright}{\isachardoublequoteclose}\ \isacommand{by}\isamarkupfalse%
\ blast\isanewline
\ \ \ \ \isacommand{moreover}\isamarkupfalse%
\ \isacommand{have}\isamarkupfalse%
\ {\isachardoublequoteopen}{\isacharquery}a{\isadigit{1}}\ {\isasymin}\ allocationsUniverse{\isachardoublequoteclose}\ \isacommand{using}\isamarkupfalse%
\ assms{\isacharparenleft}{\isadigit{1}}{\isacharparenright}\ lm{\isadigit{3}}{\isadigit{5}}\ \isacommand{by}\isamarkupfalse%
\ blast\isanewline
\ \ \ \ \isacommand{moreover}\isamarkupfalse%
\ \isacommand{have}\isamarkupfalse%
\ {\isachardoublequoteopen}{\isacharquery}a{\isadigit{2}}\ {\isasymin}\ allocationsUniverse{\isachardoublequoteclose}\ \isacommand{using}\isamarkupfalse%
\ lm{\isadigit{3}}{\isadigit{8}}\ \isacommand{by}\isamarkupfalse%
\ fastforce\isanewline
\ \ \ \ \isacommand{ultimately}\isamarkupfalse%
\ \isacommand{have}\isamarkupfalse%
\ {\isachardoublequoteopen}{\isacharquery}a{\isadigit{1}}\ {\isasymin}\ allocationsUniverse\ {\isacharampersand}\ \isanewline
\ \ \ \ {\isacharquery}a{\isadigit{2}}\ {\isasymin}\ allocationsUniverse\ {\isacharampersand}\isanewline
\ \ \ \ {\isasymUnion}Range\ {\isacharquery}a{\isadigit{1}}\ {\isasyminter}\ {\isasymUnion}Range\ {\isacharquery}a{\isadigit{2}}\ {\isacharequal}\ {\isacharbraceleft}{\isacharbraceright}\ {\isacharampersand}\ Domain\ {\isacharquery}a{\isadigit{1}}\ {\isasyminter}\ Domain\ {\isacharquery}a{\isadigit{2}}\ {\isacharequal}\ {\isacharbraceleft}{\isacharbraceright}{\isachardoublequoteclose}\ \isanewline
\isacommand{by}\isamarkupfalse%
\ blast\ \isacommand{then}\isamarkupfalse%
\ \isacommand{have}\isamarkupfalse%
\ \isanewline
{\isacharquery}t{\isadigit{1}}\ \isacommand{using}\isamarkupfalse%
\ lm{\isadigit{2}}{\isadigit{3}}\ \isacommand{by}\isamarkupfalse%
\ auto\ \ \ \ \ \ \ \isanewline
\ \ \ \ \isacommand{then}\isamarkupfalse%
\ \isacommand{have}\isamarkupfalse%
\ {\isacharquery}thesis\ \isacommand{using}\isamarkupfalse%
\ {\isadigit{6}}\ {\isadigit{7}}\ \isacommand{by}\isamarkupfalse%
\ presburger\isanewline
\ \ \isacommand{{\isacharbraceright}}\isamarkupfalse%
\isanewline
\ \ \isacommand{then}\isamarkupfalse%
\ \isacommand{show}\isamarkupfalse%
\ {\isacharquery}thesis\ \isacommand{using}\isamarkupfalse%
\ {\isadigit{3}}\ \isacommand{by}\isamarkupfalse%
\ linarith\isanewline
\isacommand{qed}\isamarkupfalse%
%
\endisatagproof
{\isafoldproof}%
%
\isadelimproof
\isanewline
%
\endisadelimproof
\isanewline
\isacommand{lemma}\isamarkupfalse%
\ lm{\isadigit{4}}{\isadigit{5}}{\isacharcolon}\ \isakeyword{assumes}\ {\isachardoublequoteopen}Domain\ a\ {\isasyminter}\ X\ {\isasymnoteq}\ {\isacharbraceleft}{\isacharbraceright}{\isachardoublequoteclose}\ {\isachardoublequoteopen}a\ {\isasymin}\ allocationsUniverse{\isachardoublequoteclose}\ \isakeyword{shows}\isanewline
{\isachardoublequoteopen}{\isasymUnion}{\isacharparenleft}a{\isacharbackquote}{\isacharbackquote}X{\isacharparenright}\ {\isasymnoteq}\ {\isacharbraceleft}{\isacharbraceright}{\isachardoublequoteclose}\ \isanewline
%
\isadelimproof
%
\endisadelimproof
%
\isatagproof
\isacommand{proof}\isamarkupfalse%
\ {\isacharminus}\isanewline
\ \ \isacommand{let}\isamarkupfalse%
\ {\isacharquery}p{\isacharequal}is{\isacharunderscore}partition\ \isacommand{let}\isamarkupfalse%
\ {\isacharquery}r{\isacharequal}Range\isanewline
\ \ \isacommand{have}\isamarkupfalse%
\ {\isachardoublequoteopen}{\isacharquery}p\ {\isacharparenleft}{\isacharquery}r\ a{\isacharparenright}{\isachardoublequoteclose}\ \isacommand{using}\isamarkupfalse%
\ assms\ Int{\isacharunderscore}iff\ lm{\isadigit{2}}{\isadigit{2}}\ \isacommand{by}\isamarkupfalse%
\ auto\isanewline
\ \ \isacommand{moreover}\isamarkupfalse%
\ \isacommand{have}\isamarkupfalse%
\ {\isachardoublequoteopen}a{\isacharbackquote}{\isacharbackquote}X\ {\isasymsubseteq}\ {\isacharquery}r\ a{\isachardoublequoteclose}\ \isacommand{by}\isamarkupfalse%
\ fast\isanewline
\ \ \isacommand{ultimately}\isamarkupfalse%
\ \isacommand{have}\isamarkupfalse%
\ {\isachardoublequoteopen}{\isacharquery}p\ {\isacharparenleft}a{\isacharbackquote}{\isacharbackquote}X{\isacharparenright}{\isachardoublequoteclose}\ \isacommand{using}\isamarkupfalse%
\ assms\ subset{\isacharunderscore}is{\isacharunderscore}partition\ \isacommand{by}\isamarkupfalse%
\ blast\isanewline
\ \ \isacommand{moreover}\isamarkupfalse%
\ \isacommand{have}\isamarkupfalse%
\ {\isachardoublequoteopen}a{\isacharbackquote}{\isacharbackquote}X\ {\isasymnoteq}\ {\isacharbraceleft}{\isacharbraceright}{\isachardoublequoteclose}\ \isacommand{using}\isamarkupfalse%
\ assms\ \isacommand{by}\isamarkupfalse%
\ fast\isanewline
\ \ \isacommand{ultimately}\isamarkupfalse%
\ \isacommand{show}\isamarkupfalse%
\ {\isacharquery}thesis\ \isacommand{by}\isamarkupfalse%
\ {\isacharparenleft}metis\ Union{\isacharunderscore}member\ all{\isacharunderscore}not{\isacharunderscore}in{\isacharunderscore}conv\ no{\isacharunderscore}empty{\isacharunderscore}eq{\isacharunderscore}class{\isacharparenright}\isanewline
\isacommand{qed}\isamarkupfalse%
%
\endisatagproof
{\isafoldproof}%
%
\isadelimproof
%
\endisadelimproof
\isanewline
\isanewline
\isacommand{corollary}\isamarkupfalse%
\ lm{\isadigit{4}}{\isadigit{5}}b{\isacharcolon}\ \isakeyword{assumes}\ {\isachardoublequoteopen}Domain\ a\ {\isasyminter}\ X\ {\isasymnoteq}\ {\isacharbraceleft}{\isacharbraceright}{\isachardoublequoteclose}\ {\isachardoublequoteopen}a\ {\isasymin}\ allocationsUniverse{\isachardoublequoteclose}\ \isakeyword{shows}\ \isanewline
{\isachardoublequoteopen}{\isacharbraceleft}{\isasymUnion}{\isacharparenleft}a{\isacharbackquote}{\isacharbackquote}{\isacharparenleft}X{\isasymunion}{\isacharbraceleft}i{\isacharbraceright}{\isacharparenright}{\isacharparenright}{\isacharbraceright}{\isacharminus}{\isacharbraceleft}{\isacharbraceleft}{\isacharbraceright}{\isacharbraceright}\ {\isacharequal}\ {\isacharbraceleft}{\isasymUnion}{\isacharparenleft}a{\isacharbackquote}{\isacharbackquote}{\isacharparenleft}X{\isasymunion}{\isacharbraceleft}i{\isacharbraceright}{\isacharparenright}{\isacharparenright}{\isacharbraceright}{\isachardoublequoteclose}%
\isadelimproof
\ %
\endisadelimproof
%
\isatagproof
\isacommand{using}\isamarkupfalse%
\ assms\ lm{\isadigit{4}}{\isadigit{5}}\ \isacommand{by}\isamarkupfalse%
\ fast%
\endisatagproof
{\isafoldproof}%
%
\isadelimproof
%
\endisadelimproof
\isanewline
\isanewline
\isacommand{corollary}\isamarkupfalse%
\ lm{\isadigit{4}}{\isadigit{3}}b{\isacharcolon}\ \isakeyword{assumes}\ {\isachardoublequoteopen}a\ {\isasymin}\ allocationsUniverse{\isachardoublequoteclose}\ \isakeyword{shows}\ \isanewline
{\isachardoublequoteopen}{\isacharparenleft}a\ outside\ {\isacharparenleft}X{\isasymunion}{\isacharbraceleft}i{\isacharbraceright}{\isacharparenright}{\isacharparenright}\ {\isasymunion}\ {\isacharparenleft}{\isacharbraceleft}i{\isacharbraceright}{\isasymtimes}{\isacharparenleft}{\isacharbraceleft}{\isasymUnion}{\isacharparenleft}a{\isacharbackquote}{\isacharbackquote}{\isacharparenleft}X{\isasymunion}{\isacharbraceleft}i{\isacharbraceright}{\isacharparenright}{\isacharparenright}{\isacharbraceright}{\isacharminus}{\isacharbraceleft}{\isacharbraceleft}{\isacharbraceright}{\isacharbraceright}{\isacharparenright}{\isacharparenright}\ {\isasymin}\ allocationsUniverse\ {\isacharampersand}\ \isanewline
{\isasymUnion}{\isacharparenleft}Range{\isacharparenleft}{\isacharparenleft}a\ outside\ {\isacharparenleft}X{\isasymunion}{\isacharbraceleft}i{\isacharbraceright}{\isacharparenright}{\isacharparenright}\ {\isasymunion}\ {\isacharparenleft}{\isacharbraceleft}i{\isacharbraceright}{\isasymtimes}{\isacharparenleft}{\isacharbraceleft}{\isasymUnion}{\isacharparenleft}a{\isacharbackquote}{\isacharbackquote}{\isacharparenleft}X{\isasymunion}{\isacharbraceleft}i{\isacharbraceright}{\isacharparenright}{\isacharparenright}{\isacharbraceright}{\isacharminus}{\isacharbraceleft}{\isacharbraceleft}{\isacharbraceright}{\isacharbraceright}{\isacharparenright}{\isacharparenright}{\isacharparenright}{\isacharparenright}\ {\isacharequal}\ {\isasymUnion}{\isacharparenleft}Range\ a{\isacharparenright}{\isachardoublequoteclose}\isanewline
%
\isadelimproof
%
\endisadelimproof
%
\isatagproof
\isacommand{proof}\isamarkupfalse%
\ {\isacharminus}\isanewline
\isacommand{have}\isamarkupfalse%
\ {\isachardoublequoteopen}a\ {\isacharminus}\ {\isacharparenleft}{\isacharparenleft}X{\isasymunion}{\isacharbraceleft}i{\isacharbraceright}{\isacharparenright}{\isasymtimes}{\isacharparenleft}Range\ a{\isacharparenright}{\isacharparenright}\ {\isacharequal}\ a\ outside\ {\isacharparenleft}X\ {\isasymunion}\ {\isacharbraceleft}i{\isacharbraceright}{\isacharparenright}{\isachardoublequoteclose}\ \isacommand{using}\isamarkupfalse%
\ Outside{\isacharunderscore}def\ \isacommand{by}\isamarkupfalse%
\ metis\isanewline
\isacommand{moreover}\isamarkupfalse%
\ \isacommand{have}\isamarkupfalse%
\ {\isachardoublequoteopen}{\isacharparenleft}a\ {\isacharminus}\ {\isacharparenleft}{\isacharparenleft}X{\isasymunion}{\isacharbraceleft}i{\isacharbraceright}{\isacharparenright}{\isasymtimes}{\isacharparenleft}Range\ a{\isacharparenright}{\isacharparenright}{\isacharparenright}\ {\isasymunion}\ {\isacharparenleft}{\isacharbraceleft}{\isacharparenleft}i{\isacharcomma}\ {\isasymUnion}\ {\isacharparenleft}a{\isacharbackquote}{\isacharbackquote}{\isacharparenleft}X\ {\isasymunion}\ {\isacharbraceleft}i{\isacharbraceright}{\isacharparenright}{\isacharparenright}{\isacharparenright}{\isacharbraceright}\ {\isacharminus}\ {\isacharbraceleft}{\isacharparenleft}i{\isacharcomma}{\isacharbraceleft}{\isacharbraceright}{\isacharparenright}{\isacharbraceright}{\isacharparenright}\ {\isasymin}\ allocationsUniverse{\isachardoublequoteclose}\isanewline
\isacommand{using}\isamarkupfalse%
\ assms\ lm{\isadigit{4}}{\isadigit{3}}\ \isacommand{by}\isamarkupfalse%
\ fastforce\isanewline
\isacommand{moreover}\isamarkupfalse%
\ \isacommand{have}\isamarkupfalse%
\ {\isachardoublequoteopen}{\isasymUnion}\ {\isacharparenleft}Range\ {\isacharparenleft}{\isacharparenleft}a\ {\isacharminus}\ {\isacharparenleft}{\isacharparenleft}X{\isasymunion}{\isacharbraceleft}i{\isacharbraceright}{\isacharparenright}{\isasymtimes}{\isacharparenleft}Range\ a{\isacharparenright}{\isacharparenright}{\isacharparenright}\ {\isasymunion}\ {\isacharparenleft}{\isacharbraceleft}{\isacharparenleft}i{\isacharcomma}\ {\isasymUnion}\ {\isacharparenleft}a{\isacharbackquote}{\isacharbackquote}{\isacharparenleft}X\ {\isasymunion}\ {\isacharbraceleft}i{\isacharbraceright}{\isacharparenright}{\isacharparenright}{\isacharparenright}{\isacharbraceright}\ {\isacharminus}\ {\isacharbraceleft}{\isacharparenleft}i{\isacharcomma}{\isacharbraceleft}{\isacharbraceright}{\isacharparenright}{\isacharbraceright}{\isacharparenright}{\isacharparenright}{\isacharparenright}\ {\isacharequal}\ {\isasymUnion}{\isacharparenleft}Range\ a{\isacharparenright}{\isachardoublequoteclose}\isanewline
\isacommand{using}\isamarkupfalse%
\ assms\ lm{\isadigit{4}}{\isadigit{3}}\ \isacommand{by}\isamarkupfalse%
\ {\isacharparenleft}metis\ {\isacharparenleft}no{\isacharunderscore}types{\isacharparenright}{\isacharparenright}\isanewline
\isacommand{ultimately}\isamarkupfalse%
\ \isacommand{have}\isamarkupfalse%
\isanewline
{\isachardoublequoteopen}{\isacharparenleft}a\ outside\ {\isacharparenleft}X{\isasymunion}{\isacharbraceleft}i{\isacharbraceright}{\isacharparenright}{\isacharparenright}\ {\isasymunion}\ {\isacharparenleft}{\isacharbraceleft}{\isacharparenleft}i{\isacharcomma}\ {\isasymUnion}\ {\isacharparenleft}a{\isacharbackquote}{\isacharbackquote}{\isacharparenleft}X\ {\isasymunion}\ {\isacharbraceleft}i{\isacharbraceright}{\isacharparenright}{\isacharparenright}{\isacharparenright}{\isacharbraceright}\ {\isacharminus}\ {\isacharbraceleft}{\isacharparenleft}i{\isacharcomma}{\isacharbraceleft}{\isacharbraceright}{\isacharparenright}{\isacharbraceright}{\isacharparenright}\ {\isasymin}\ allocationsUniverse\ {\isacharampersand}\ \isanewline
{\isasymUnion}\ {\isacharparenleft}Range\ {\isacharparenleft}{\isacharparenleft}a\ outside\ {\isacharparenleft}X{\isasymunion}{\isacharbraceleft}i{\isacharbraceright}{\isacharparenright}{\isacharparenright}\ {\isasymunion}\ {\isacharparenleft}{\isacharbraceleft}{\isacharparenleft}i{\isacharcomma}\ {\isasymUnion}\ {\isacharparenleft}a{\isacharbackquote}{\isacharbackquote}{\isacharparenleft}X\ {\isasymunion}\ {\isacharbraceleft}i{\isacharbraceright}{\isacharparenright}{\isacharparenright}{\isacharparenright}{\isacharbraceright}\ {\isacharminus}\ {\isacharbraceleft}{\isacharparenleft}i{\isacharcomma}{\isacharbraceleft}{\isacharbraceright}{\isacharparenright}{\isacharbraceright}{\isacharparenright}{\isacharparenright}{\isacharparenright}\ {\isacharequal}\ {\isasymUnion}{\isacharparenleft}Range\ a{\isacharparenright}{\isachardoublequoteclose}\ \isacommand{by}\isamarkupfalse%
\isanewline
presburger\isanewline
\isacommand{moreover}\isamarkupfalse%
\ \isacommand{have}\isamarkupfalse%
\ {\isachardoublequoteopen}{\isacharbraceleft}{\isacharparenleft}i{\isacharcomma}\ {\isasymUnion}\ {\isacharparenleft}a{\isacharbackquote}{\isacharbackquote}{\isacharparenleft}X\ {\isasymunion}\ {\isacharbraceleft}i{\isacharbraceright}{\isacharparenright}{\isacharparenright}{\isacharparenright}{\isacharbraceright}\ {\isacharminus}\ {\isacharbraceleft}{\isacharparenleft}i{\isacharcomma}{\isacharbraceleft}{\isacharbraceright}{\isacharparenright}{\isacharbraceright}\ {\isacharequal}\ {\isacharbraceleft}i{\isacharbraceright}\ {\isasymtimes}\ {\isacharparenleft}{\isacharbraceleft}{\isasymUnion}\ {\isacharparenleft}a{\isacharbackquote}{\isacharbackquote}{\isacharparenleft}X{\isasymunion}{\isacharbraceleft}i{\isacharbraceright}{\isacharparenright}{\isacharparenright}{\isacharbraceright}\ {\isacharminus}\ {\isacharbraceleft}{\isacharbraceleft}{\isacharbraceright}{\isacharbraceright}{\isacharparenright}{\isachardoublequoteclose}\ \isanewline
\isacommand{by}\isamarkupfalse%
\ fast\isanewline
\isacommand{ultimately}\isamarkupfalse%
\ \isacommand{show}\isamarkupfalse%
\ {\isacharquery}thesis\ \isacommand{by}\isamarkupfalse%
\ auto\isanewline
\isacommand{qed}\isamarkupfalse%
%
\endisatagproof
{\isafoldproof}%
%
\isadelimproof
\isanewline
%
\endisadelimproof
\isanewline
\isacommand{corollary}\isamarkupfalse%
\ lm{\isadigit{4}}{\isadigit{3}}c{\isacharcolon}\ \isakeyword{assumes}\ {\isachardoublequoteopen}a\ {\isasymin}\ allocationsUniverse{\isachardoublequoteclose}\ {\isachardoublequoteopen}Domain\ a\ {\isasyminter}\ X\ {\isasymnoteq}\ {\isacharbraceleft}{\isacharbraceright}{\isachardoublequoteclose}\ \isakeyword{shows}\ \isanewline
{\isachardoublequoteopen}{\isacharparenleft}a\ outside\ {\isacharparenleft}X{\isasymunion}{\isacharbraceleft}i{\isacharbraceright}{\isacharparenright}{\isacharparenright}\ {\isasymunion}\ {\isacharparenleft}{\isacharbraceleft}i{\isacharbraceright}{\isasymtimes}{\isacharbraceleft}{\isasymUnion}{\isacharparenleft}a{\isacharbackquote}{\isacharbackquote}{\isacharparenleft}X{\isasymunion}{\isacharbraceleft}i{\isacharbraceright}{\isacharparenright}{\isacharparenright}{\isacharbraceright}{\isacharparenright}\ {\isasymin}\ allocationsUniverse\ {\isacharampersand}\ \isanewline
{\isasymUnion}{\isacharparenleft}Range{\isacharparenleft}{\isacharparenleft}a\ outside\ {\isacharparenleft}X{\isasymunion}{\isacharbraceleft}i{\isacharbraceright}{\isacharparenright}{\isacharparenright}\ {\isasymunion}\ {\isacharparenleft}{\isacharbraceleft}i{\isacharbraceright}{\isasymtimes}{\isacharbraceleft}{\isasymUnion}{\isacharparenleft}a{\isacharbackquote}{\isacharbackquote}{\isacharparenleft}X{\isasymunion}{\isacharbraceleft}i{\isacharbraceright}{\isacharparenright}{\isacharparenright}{\isacharbraceright}{\isacharparenright}{\isacharparenright}{\isacharparenright}\ {\isacharequal}\ {\isasymUnion}{\isacharparenleft}Range\ a{\isacharparenright}{\isachardoublequoteclose}\isanewline
%
\isadelimproof
%
\endisadelimproof
%
\isatagproof
\isacommand{using}\isamarkupfalse%
\ assms\ lm{\isadigit{4}}{\isadigit{3}}b\ lm{\isadigit{4}}{\isadigit{5}}b\isanewline
\isacommand{proof}\isamarkupfalse%
\ {\isacharminus}\isanewline
\isacommand{let}\isamarkupfalse%
\ {\isacharquery}t{\isadigit{1}}{\isacharequal}{\isachardoublequoteopen}{\isacharparenleft}a\ outside\ {\isacharparenleft}X{\isasymunion}{\isacharbraceleft}i{\isacharbraceright}{\isacharparenright}{\isacharparenright}\ {\isasymunion}\ {\isacharparenleft}{\isacharbraceleft}i{\isacharbraceright}{\isasymtimes}{\isacharparenleft}{\isacharbraceleft}{\isasymUnion}{\isacharparenleft}a{\isacharbackquote}{\isacharbackquote}{\isacharparenleft}X{\isasymunion}{\isacharbraceleft}i{\isacharbraceright}{\isacharparenright}{\isacharparenright}{\isacharbraceright}{\isacharminus}{\isacharbraceleft}{\isacharbraceleft}{\isacharbraceright}{\isacharbraceright}{\isacharparenright}{\isacharparenright}\ {\isasymin}\ allocationsUniverse{\isachardoublequoteclose}\isanewline
\isacommand{let}\isamarkupfalse%
\ {\isacharquery}t{\isadigit{2}}{\isacharequal}{\isachardoublequoteopen}{\isasymUnion}{\isacharparenleft}Range{\isacharparenleft}{\isacharparenleft}a\ outside\ {\isacharparenleft}X{\isasymunion}{\isacharbraceleft}i{\isacharbraceright}{\isacharparenright}{\isacharparenright}\ {\isasymunion}\ {\isacharparenleft}{\isacharbraceleft}i{\isacharbraceright}{\isasymtimes}{\isacharparenleft}{\isacharbraceleft}{\isasymUnion}{\isacharparenleft}a{\isacharbackquote}{\isacharbackquote}{\isacharparenleft}X{\isasymunion}{\isacharbraceleft}i{\isacharbraceright}{\isacharparenright}{\isacharparenright}{\isacharbraceright}{\isacharminus}{\isacharbraceleft}{\isacharbraceleft}{\isacharbraceright}{\isacharbraceright}{\isacharparenright}{\isacharparenright}{\isacharparenright}{\isacharparenright}\ {\isacharequal}\ {\isasymUnion}{\isacharparenleft}Range\ a{\isacharparenright}{\isachardoublequoteclose}\isanewline
\isacommand{have}\isamarkupfalse%
\ \isanewline
{\isadigit{0}}{\isacharcolon}\ {\isachardoublequoteopen}a\ {\isasymin}\ allocationsUniverse{\isachardoublequoteclose}\ \isacommand{using}\isamarkupfalse%
\ assms{\isacharparenleft}{\isadigit{1}}{\isacharparenright}\ \isacommand{by}\isamarkupfalse%
\ fast\ \isanewline
\isacommand{then}\isamarkupfalse%
\ \isacommand{have}\isamarkupfalse%
\ {\isachardoublequoteopen}{\isacharquery}t{\isadigit{1}}\ {\isacharampersand}\ {\isacharquery}t{\isadigit{2}}{\isachardoublequoteclose}\ \isacommand{using}\isamarkupfalse%
\ lm{\isadigit{4}}{\isadigit{3}}b\ \isanewline
\isacommand{proof}\isamarkupfalse%
\ {\isacharminus}\ \isanewline
\isanewline
\ \ \isacommand{have}\isamarkupfalse%
\ {\isachardoublequoteopen}a\ {\isasymin}\ allocationsUniverse\ {\isasymlongrightarrow}\ a\ {\isacharminus}{\isacharbar}\ {\isacharparenleft}X\ {\isasymunion}\ {\isacharbraceleft}i{\isacharbraceright}{\isacharparenright}\ {\isasymunion}\ {\isacharbraceleft}i{\isacharbraceright}\ {\isasymtimes}\ {\isacharparenleft}{\isacharbraceleft}{\isasymUnion}{\isacharparenleft}a\ {\isacharbackquote}{\isacharbackquote}\ {\isacharparenleft}X\ {\isasymunion}\ {\isacharbraceleft}i{\isacharbraceright}{\isacharparenright}{\isacharparenright}{\isacharbraceright}\ {\isacharminus}\ {\isacharbraceleft}{\isacharbraceleft}{\isacharbraceright}{\isacharbraceright}{\isacharparenright}\ {\isasymin}\ allocationsUniverse{\isachardoublequoteclose}\isanewline
\ \ \ \ \isacommand{using}\isamarkupfalse%
\ lm{\isadigit{4}}{\isadigit{3}}b\ \isacommand{by}\isamarkupfalse%
\ fastforce\isanewline
\ \ \isacommand{hence}\isamarkupfalse%
\ {\isachardoublequoteopen}a\ {\isacharminus}{\isacharbar}\ {\isacharparenleft}X\ {\isasymunion}\ {\isacharbraceleft}i{\isacharbraceright}{\isacharparenright}\ {\isasymunion}\ {\isacharbraceleft}i{\isacharbraceright}\ {\isasymtimes}\ {\isacharparenleft}{\isacharbraceleft}{\isasymUnion}{\isacharparenleft}a\ {\isacharbackquote}{\isacharbackquote}\ {\isacharparenleft}X\ {\isasymunion}\ {\isacharbraceleft}i{\isacharbraceright}{\isacharparenright}{\isacharparenright}{\isacharbraceright}\ {\isacharminus}\ {\isacharbraceleft}{\isacharbraceleft}{\isacharbraceright}{\isacharbraceright}{\isacharparenright}\ {\isasymin}\ allocationsUniverse{\isachardoublequoteclose}\isanewline
\ \ \ \ \isacommand{by}\isamarkupfalse%
\ {\isacharparenleft}metis\ {\isachardoublequoteopen}{\isadigit{0}}{\isachardoublequoteclose}{\isacharparenright}\isanewline
\ \ \isacommand{thus}\isamarkupfalse%
\ {\isachardoublequoteopen}a\ {\isacharminus}{\isacharbar}\ {\isacharparenleft}X\ {\isasymunion}\ {\isacharbraceleft}i{\isacharbraceright}{\isacharparenright}\ {\isasymunion}\ {\isacharbraceleft}i{\isacharbraceright}\ {\isasymtimes}\ {\isacharparenleft}{\isacharbraceleft}{\isasymUnion}{\isacharparenleft}a\ {\isacharbackquote}{\isacharbackquote}\ {\isacharparenleft}X\ {\isasymunion}\ {\isacharbraceleft}i{\isacharbraceright}{\isacharparenright}{\isacharparenright}{\isacharbraceright}\ {\isacharminus}\ {\isacharbraceleft}{\isacharbraceleft}{\isacharbraceright}{\isacharbraceright}{\isacharparenright}\ {\isasymin}\ allocationsUniverse\ {\isasymand}\ {\isasymUnion}Range\ {\isacharparenleft}a\ {\isacharminus}{\isacharbar}\ {\isacharparenleft}X\ {\isasymunion}\ {\isacharbraceleft}i{\isacharbraceright}{\isacharparenright}\ {\isasymunion}\ {\isacharbraceleft}i{\isacharbraceright}\ {\isasymtimes}\ {\isacharparenleft}{\isacharbraceleft}{\isasymUnion}{\isacharparenleft}a\ {\isacharbackquote}{\isacharbackquote}\ {\isacharparenleft}X\ {\isasymunion}\ {\isacharbraceleft}i{\isacharbraceright}{\isacharparenright}{\isacharparenright}{\isacharbraceright}\ {\isacharminus}\ {\isacharbraceleft}{\isacharbraceleft}{\isacharbraceright}{\isacharbraceright}{\isacharparenright}{\isacharparenright}\ {\isacharequal}\ {\isasymUnion}Range\ a{\isachardoublequoteclose}\isanewline
\ \ \ \ \isacommand{using}\isamarkupfalse%
\ {\isachardoublequoteopen}{\isadigit{0}}{\isachardoublequoteclose}\ \isacommand{by}\isamarkupfalse%
\ {\isacharparenleft}metis\ {\isacharparenleft}no{\isacharunderscore}types{\isacharparenright}\ lm{\isadigit{4}}{\isadigit{3}}b{\isacharparenright}\isanewline
\isacommand{qed}\isamarkupfalse%
\isanewline
\isacommand{moreover}\isamarkupfalse%
\ \isacommand{have}\isamarkupfalse%
\ \isanewline
{\isadigit{1}}{\isacharcolon}\ {\isachardoublequoteopen}{\isacharbraceleft}{\isasymUnion}{\isacharparenleft}a{\isacharbackquote}{\isacharbackquote}{\isacharparenleft}X{\isasymunion}{\isacharbraceleft}i{\isacharbraceright}{\isacharparenright}{\isacharparenright}{\isacharbraceright}{\isacharminus}{\isacharbraceleft}{\isacharbraceleft}{\isacharbraceright}{\isacharbraceright}\ {\isacharequal}\ {\isacharbraceleft}{\isasymUnion}{\isacharparenleft}a{\isacharbackquote}{\isacharbackquote}{\isacharparenleft}X{\isasymunion}{\isacharbraceleft}i{\isacharbraceright}{\isacharparenright}{\isacharparenright}{\isacharbraceright}{\isachardoublequoteclose}\ \isacommand{using}\isamarkupfalse%
\ lm{\isadigit{4}}{\isadigit{5}}\ assms\ \isacommand{by}\isamarkupfalse%
\ fast\isanewline
\isacommand{ultimately}\isamarkupfalse%
\ \isacommand{show}\isamarkupfalse%
\ {\isacharquery}thesis\ \isacommand{by}\isamarkupfalse%
\ auto\isanewline
\isacommand{qed}\isamarkupfalse%
%
\endisatagproof
{\isafoldproof}%
%
\isadelimproof
\isanewline
%
\endisadelimproof
\isanewline
\isacommand{abbreviation}\isamarkupfalse%
\ {\isachardoublequoteopen}condition{\isadigit{1}}\ b\ i\ {\isacharequal}{\isacharequal}\ {\isacharparenleft}{\isasymforall}\ t\ t{\isacharprime}{\isachardot}\ {\isacharparenleft}trivial\ t\ {\isacharampersand}\ trivial\ t{\isacharprime}\ {\isacharampersand}\ Union\ t\ {\isasymsubseteq}\ Union\ t{\isacharprime}{\isacharparenright}\ {\isasymlongrightarrow}\isanewline
setsum\ b\ {\isacharparenleft}{\isacharbraceleft}i{\isacharbraceright}{\isasymtimes}t{\isacharparenright}\ {\isasymle}\ setsum\ b\ {\isacharparenleft}{\isacharbraceleft}i{\isacharbraceright}{\isasymtimes}t{\isacharprime}{\isacharparenright}{\isacharparenright}{\isachardoublequoteclose}\ \isanewline
\isanewline
\isanewline
\isacommand{abbreviation}\isamarkupfalse%
\ {\isachardoublequoteopen}condition{\isadigit{1}}b\ b\ i\ {\isacharequal}{\isacharequal}\ {\isasymforall}\ X\ Y{\isachardot}\ setsum\ b\ {\isacharparenleft}{\isacharbraceleft}i{\isacharbraceright}{\isasymtimes}{\isacharbraceleft}X{\isacharbraceright}{\isacharparenright}\ {\isasymle}\ setsum\ b\ {\isacharparenleft}{\isacharbraceleft}i{\isacharbraceright}{\isasymtimes}{\isacharbraceleft}X\ {\isasymunion}\ Y{\isacharbraceright}{\isacharparenright}{\isachardoublequoteclose}\isanewline
\isanewline
\isacommand{lemma}\isamarkupfalse%
\ lm{\isadigit{4}}{\isadigit{6}}{\isacharcolon}\ \isakeyword{assumes}\ {\isachardoublequoteopen}condition{\isadigit{1}}\ b\ i{\isachardoublequoteclose}\ {\isachardoublequoteopen}runiq\ a{\isachardoublequoteclose}\ \isakeyword{shows}\ \isanewline
{\isachardoublequoteopen}setsum\ b\ {\isacharparenleft}{\isacharbraceleft}i{\isacharbraceright}{\isasymtimes}{\isacharparenleft}{\isacharparenleft}a\ outside\ X{\isacharparenright}{\isacharbackquote}{\isacharbackquote}{\isacharbraceleft}i{\isacharbraceright}{\isacharparenright}{\isacharparenright}\ {\isasymle}\ setsum\ b\ {\isacharparenleft}{\isacharbraceleft}i{\isacharbraceright}{\isasymtimes}{\isacharbraceleft}{\isasymUnion}{\isacharparenleft}a{\isacharbackquote}{\isacharbackquote}{\isacharparenleft}X{\isasymunion}{\isacharbraceleft}i{\isacharbraceright}{\isacharparenright}{\isacharparenright}{\isacharbraceright}{\isacharparenright}{\isachardoublequoteclose}\isanewline
%
\isadelimproof
%
\endisadelimproof
%
\isatagproof
\isacommand{proof}\isamarkupfalse%
\ {\isacharminus}\isanewline
\ \ \isacommand{let}\isamarkupfalse%
\ {\isacharquery}u{\isacharequal}runiq\ \isacommand{let}\isamarkupfalse%
\ {\isacharquery}I{\isacharequal}{\isachardoublequoteopen}{\isacharbraceleft}i{\isacharbraceright}{\isachardoublequoteclose}\ \isacommand{let}\isamarkupfalse%
\ {\isacharquery}R{\isacharequal}{\isachardoublequoteopen}a\ outside\ X{\isachardoublequoteclose}\ \isacommand{let}\isamarkupfalse%
\ {\isacharquery}U{\isacharequal}Union\ \isacommand{let}\isamarkupfalse%
\ {\isacharquery}Xi{\isacharequal}{\isachardoublequoteopen}X\ {\isasymunion}{\isacharquery}I{\isachardoublequoteclose}\isanewline
\ \ \isacommand{let}\isamarkupfalse%
\ {\isacharquery}t{\isadigit{1}}{\isacharequal}{\isachardoublequoteopen}{\isacharquery}R{\isacharbackquote}{\isacharbackquote}{\isacharquery}I{\isachardoublequoteclose}\ \isacommand{let}\isamarkupfalse%
\ {\isacharquery}t{\isadigit{2}}{\isacharequal}{\isachardoublequoteopen}{\isacharbraceleft}{\isacharquery}U\ {\isacharparenleft}a{\isacharbackquote}{\isacharbackquote}{\isacharquery}Xi{\isacharparenright}{\isacharbraceright}{\isachardoublequoteclose}\isanewline
\ \ \isacommand{have}\isamarkupfalse%
\ {\isachardoublequoteopen}{\isacharquery}U\ {\isacharparenleft}{\isacharquery}R{\isacharbackquote}{\isacharbackquote}{\isacharquery}I{\isacharparenright}\ {\isasymsubseteq}\ {\isacharquery}U\ {\isacharparenleft}{\isacharquery}R{\isacharbackquote}{\isacharbackquote}{\isacharparenleft}X{\isasymunion}{\isacharquery}I{\isacharparenright}{\isacharparenright}{\isachardoublequoteclose}\ \isacommand{by}\isamarkupfalse%
\ blast\isanewline
\ \ \isacommand{moreover}\isamarkupfalse%
\ \isacommand{have}\isamarkupfalse%
\ {\isachardoublequoteopen}{\isachardot}{\isachardot}{\isachardot}\ {\isasymsubseteq}\ {\isacharquery}U\ {\isacharparenleft}a{\isacharbackquote}{\isacharbackquote}{\isacharparenleft}X{\isasymunion}{\isacharquery}I{\isacharparenright}{\isacharparenright}{\isachardoublequoteclose}\ \isacommand{using}\isamarkupfalse%
\ Outside{\isacharunderscore}def\ \isacommand{by}\isamarkupfalse%
\ blast\isanewline
\ \ \isacommand{ultimately}\isamarkupfalse%
\ \isacommand{have}\isamarkupfalse%
\ {\isachardoublequoteopen}{\isacharquery}U\ {\isacharparenleft}{\isacharquery}R{\isacharbackquote}{\isacharbackquote}{\isacharquery}I{\isacharparenright}\ {\isasymsubseteq}\ {\isacharquery}U\ {\isacharparenleft}a{\isacharbackquote}{\isacharbackquote}{\isacharparenleft}X{\isasymunion}{\isacharquery}I{\isacharparenright}{\isacharparenright}{\isachardoublequoteclose}\ \isacommand{by}\isamarkupfalse%
\ auto\isanewline
\ \ \isacommand{then}\isamarkupfalse%
\ \isacommand{have}\isamarkupfalse%
\ \isanewline
\ \ {\isadigit{0}}{\isacharcolon}\ {\isachardoublequoteopen}{\isacharquery}U\ {\isacharquery}t{\isadigit{1}}\ {\isasymsubseteq}\ {\isacharquery}U\ {\isacharquery}t{\isadigit{2}}{\isachardoublequoteclose}\ \isacommand{by}\isamarkupfalse%
\ auto\isanewline
\ \ \isacommand{have}\isamarkupfalse%
\ {\isachardoublequoteopen}{\isacharquery}u\ a{\isachardoublequoteclose}\ \isacommand{using}\isamarkupfalse%
\ assms\ \isacommand{by}\isamarkupfalse%
\ fast\ \isanewline
\ \ \isacommand{moreover}\isamarkupfalse%
\ \isacommand{have}\isamarkupfalse%
\ {\isachardoublequoteopen}{\isacharquery}R\ {\isasymsubseteq}\ a{\isachardoublequoteclose}\ \isacommand{using}\isamarkupfalse%
\ Outside{\isacharunderscore}def\ \isacommand{by}\isamarkupfalse%
\ blast\ \isacommand{ultimately}\isamarkupfalse%
\isanewline
\ \ \isacommand{have}\isamarkupfalse%
\ {\isachardoublequoteopen}{\isacharquery}u\ {\isacharquery}R{\isachardoublequoteclose}\ \isacommand{using}\isamarkupfalse%
\ subrel{\isacharunderscore}runiq\ \isacommand{by}\isamarkupfalse%
\ metis\isanewline
\ \ \isacommand{then}\isamarkupfalse%
\ \isacommand{have}\isamarkupfalse%
\ {\isachardoublequoteopen}trivial\ {\isacharquery}t{\isadigit{1}}{\isachardoublequoteclose}\ \isacommand{by}\isamarkupfalse%
\ {\isacharparenleft}metis\ runiq{\isacharunderscore}alt{\isacharparenright}\isanewline
\ \ \isacommand{moreover}\isamarkupfalse%
\ \isacommand{have}\isamarkupfalse%
\ {\isachardoublequoteopen}trivial\ {\isacharquery}t{\isadigit{2}}{\isachardoublequoteclose}\ \isacommand{by}\isamarkupfalse%
\ {\isacharparenleft}metis\ trivial{\isacharunderscore}singleton{\isacharparenright}\isanewline
\ \ \isacommand{ultimately}\isamarkupfalse%
\ \isacommand{show}\isamarkupfalse%
\ {\isacharquery}thesis\ \isacommand{using}\isamarkupfalse%
\ assms\ {\isadigit{0}}\ \isacommand{by}\isamarkupfalse%
\ blast\isanewline
\isacommand{qed}\isamarkupfalse%
%
\endisatagproof
{\isafoldproof}%
%
\isadelimproof
\isanewline
%
\endisadelimproof
\isanewline
\isacommand{lemma}\isamarkupfalse%
\ lm{\isadigit{4}}{\isadigit{8}}{\isacharcolon}\ {\isachardoublequoteopen}possibleAllocationsRel\ N\ G\ {\isasymsubseteq}\ injectionsUniverse{\isachardoublequoteclose}%
\isadelimproof
\ %
\endisadelimproof
%
\isatagproof
\isacommand{using}\isamarkupfalse%
\ \ lm{\isadigit{1}}{\isadigit{9}}\ \isacommand{by}\isamarkupfalse%
\ fast%
\endisatagproof
{\isafoldproof}%
%
\isadelimproof
%
\endisadelimproof
\isanewline
\isanewline
\isacommand{lemma}\isamarkupfalse%
\ lm{\isadigit{4}}{\isadigit{9}}{\isacharcolon}\ {\isachardoublequoteopen}possibleAllocationsRel\ N\ G\ {\isasymsubseteq}\ partitionValuedUniverse{\isachardoublequoteclose}\isanewline
%
\isadelimproof
%
\endisadelimproof
%
\isatagproof
\isacommand{using}\isamarkupfalse%
\ assms\ lm{\isadigit{4}}{\isadigit{7}}\ is{\isacharunderscore}partition{\isacharunderscore}of{\isacharunderscore}def\ is{\isacharunderscore}partition{\isacharunderscore}def\ \isacommand{by}\isamarkupfalse%
\ blast%
\endisatagproof
{\isafoldproof}%
%
\isadelimproof
\isanewline
%
\endisadelimproof
\isanewline
\isacommand{corollary}\isamarkupfalse%
\ lm{\isadigit{5}}{\isadigit{0}}{\isacharcolon}\ {\isachardoublequoteopen}possibleAllocationsRel\ N\ G\ {\isasymsubseteq}\ allocationsUniverse{\isachardoublequoteclose}%
\isadelimproof
\ %
\endisadelimproof
%
\isatagproof
\isacommand{using}\isamarkupfalse%
\ lm{\isadigit{4}}{\isadigit{8}}\ lm{\isadigit{4}}{\isadigit{9}}\ \isanewline
\isacommand{by}\isamarkupfalse%
\ {\isacharparenleft}metis\ {\isacharparenleft}lifting{\isacharcomma}\ mono{\isacharunderscore}tags{\isacharparenright}\ inf{\isachardot}bounded{\isacharunderscore}iff{\isacharparenright}%
\endisatagproof
{\isafoldproof}%
%
\isadelimproof
%
\endisadelimproof
\isanewline
\isanewline
\isacommand{lemma}\isamarkupfalse%
\ mm{\isadigit{4}}{\isadigit{5}}{\isacharcolon}\ \isakeyword{assumes}\ {\isachardoublequoteopen}XX\ {\isasymin}\ partitionValuedUniverse{\isachardoublequoteclose}\ \isakeyword{shows}\ {\isachardoublequoteopen}{\isacharbraceleft}{\isacharbraceright}\ {\isasymnotin}\ Range\ XX{\isachardoublequoteclose}%
\isadelimproof
\ %
\endisadelimproof
%
\isatagproof
\isacommand{using}\isamarkupfalse%
\ assms\ \isanewline
mem{\isacharunderscore}Collect{\isacharunderscore}eq\ no{\isacharunderscore}empty{\isacharunderscore}eq{\isacharunderscore}class\ \isacommand{by}\isamarkupfalse%
\ auto%
\endisatagproof
{\isafoldproof}%
%
\isadelimproof
%
\endisadelimproof
\isanewline
\isacommand{corollary}\isamarkupfalse%
\ mm{\isadigit{4}}{\isadigit{5}}b{\isacharcolon}\ \isakeyword{assumes}\ {\isachardoublequoteopen}a\ {\isasymin}\ possibleAllocationsRel\ N\ G{\isachardoublequoteclose}\ \isakeyword{shows}\ {\isachardoublequoteopen}{\isacharbraceleft}{\isacharbraceright}\ {\isasymnotin}\ Range\ a{\isachardoublequoteclose}%
\isadelimproof
\ %
\endisadelimproof
%
\isatagproof
\isacommand{using}\isamarkupfalse%
\ assms\ mm{\isadigit{4}}{\isadigit{5}}\ \isanewline
lm{\isadigit{5}}{\isadigit{0}}\ \isacommand{by}\isamarkupfalse%
\ blast%
\endisatagproof
{\isafoldproof}%
%
\isadelimproof
%
\endisadelimproof
\isanewline
\isacommand{lemma}\isamarkupfalse%
\ mm{\isadigit{6}}{\isadigit{3}}{\isacharcolon}\ \isakeyword{assumes}\ {\isachardoublequoteopen}a\ {\isasymin}\ possibleAllocationsRel\ N\ G{\isachardoublequoteclose}\ \isakeyword{shows}\ {\isachardoublequoteopen}Range\ a\ {\isasymsubseteq}\ Pow\ G{\isachardoublequoteclose}\isanewline
%
\isadelimproof
%
\endisadelimproof
%
\isatagproof
\isacommand{using}\isamarkupfalse%
\ assms\ lm{\isadigit{4}}{\isadigit{7}}\ is{\isacharunderscore}partition{\isacharunderscore}of{\isacharunderscore}def\ \isacommand{by}\isamarkupfalse%
\ {\isacharparenleft}metis\ subset{\isacharunderscore}Pow{\isacharunderscore}Union{\isacharparenright}%
\endisatagproof
{\isafoldproof}%
%
\isadelimproof
\isanewline
%
\endisadelimproof
\isacommand{corollary}\isamarkupfalse%
\ mm{\isadigit{6}}{\isadigit{3}}b{\isacharcolon}\ \isakeyword{assumes}\ {\isachardoublequoteopen}a\ {\isasymin}\ possibleAllocationsRel\ N\ G{\isachardoublequoteclose}\ \isakeyword{shows}\ {\isachardoublequoteopen}Domain\ a\ {\isasymsubseteq}\ N\ {\isacharampersand}\ Range\ a\ {\isasymsubseteq}\ Pow\ G\ {\isacharminus}\ {\isacharbraceleft}{\isacharbraceleft}{\isacharbraceright}{\isacharbraceright}{\isachardoublequoteclose}%
\isadelimproof
\ %
\endisadelimproof
%
\isatagproof
\isacommand{using}\isamarkupfalse%
\isanewline
assms\ mm{\isadigit{6}}{\isadigit{3}}\ insert{\isacharunderscore}Diff{\isacharunderscore}single\ mm{\isadigit{4}}{\isadigit{5}}b\ subset{\isacharunderscore}insert\ lm{\isadigit{4}}{\isadigit{7}}\ \isacommand{by}\isamarkupfalse%
\ metis%
\endisatagproof
{\isafoldproof}%
%
\isadelimproof
%
\endisadelimproof
\isanewline
\isacommand{corollary}\isamarkupfalse%
\ mm{\isadigit{6}}{\isadigit{3}}c{\isacharcolon}\ \isakeyword{assumes}\ {\isachardoublequoteopen}a\ {\isasymin}\ possibleAllocationsRel\ N\ G{\isachardoublequoteclose}\ \isakeyword{shows}\ {\isachardoublequoteopen}a\ {\isasymsubseteq}\ N\ {\isasymtimes}\ {\isacharparenleft}Pow\ G\ {\isacharminus}\ {\isacharbraceleft}{\isacharbraceleft}{\isacharbraceright}{\isacharbraceright}{\isacharparenright}{\isachardoublequoteclose}\ \isanewline
%
\isadelimproof
%
\endisadelimproof
%
\isatagproof
\isacommand{using}\isamarkupfalse%
\ assms\ mm{\isadigit{6}}{\isadigit{3}}b\ \isacommand{by}\isamarkupfalse%
\ blast%
\endisatagproof
{\isafoldproof}%
%
\isadelimproof
\isanewline
%
\endisadelimproof
\isacommand{corollary}\isamarkupfalse%
\ mm{\isadigit{6}}{\isadigit{3}}e{\isacharcolon}\ {\isachardoublequoteopen}possibleAllocationsRel\ N\ G\ {\isasymsubseteq}\ Pow\ {\isacharparenleft}N{\isasymtimes}{\isacharparenleft}Pow\ G{\isacharminus}{\isacharbraceleft}{\isacharbraceleft}{\isacharbraceright}{\isacharbraceright}{\isacharparenright}{\isacharparenright}{\isachardoublequoteclose}%
\isadelimproof
\ %
\endisadelimproof
%
\isatagproof
\isacommand{using}\isamarkupfalse%
\ mm{\isadigit{6}}{\isadigit{3}}c\ \isacommand{by}\isamarkupfalse%
\ blast%
\endisatagproof
{\isafoldproof}%
%
\isadelimproof
%
\endisadelimproof
\isanewline
\isanewline
\isacommand{lemma}\isamarkupfalse%
\ lm{\isadigit{5}}{\isadigit{1}}{\isacharcolon}\ \isakeyword{assumes}\ \ \isanewline
{\isachardoublequoteopen}a\ {\isasymin}\ possibleAllocationsRel\ N\ G{\isachardoublequoteclose}\ \isanewline
{\isachardoublequoteopen}i{\isasymin}N{\isacharminus}X{\isachardoublequoteclose}\ \isanewline
{\isachardoublequoteopen}Domain\ a\ {\isasyminter}\ X\ {\isasymnoteq}\ {\isacharbraceleft}{\isacharbraceright}{\isachardoublequoteclose}\ \isanewline
\isakeyword{shows}\ \isanewline
{\isachardoublequoteopen}a\ outside\ {\isacharparenleft}X\ {\isasymunion}\ {\isacharbraceleft}i{\isacharbraceright}{\isacharparenright}\ {\isasymunion}\ {\isacharparenleft}{\isacharbraceleft}i{\isacharbraceright}\ {\isasymtimes}\ {\isacharbraceleft}{\isasymUnion}{\isacharparenleft}a{\isacharbackquote}{\isacharbackquote}{\isacharparenleft}X{\isasymunion}{\isacharbraceleft}i{\isacharbraceright}{\isacharparenright}{\isacharparenright}{\isacharbraceright}{\isacharparenright}\ {\isasymin}\ possibleAllocationsRel\ {\isacharparenleft}N{\isacharminus}X{\isacharparenright}\ {\isacharparenleft}{\isasymUnion}\ {\isacharparenleft}Range\ a{\isacharparenright}{\isacharparenright}{\isachardoublequoteclose}\isanewline
%
\isadelimproof
%
\endisadelimproof
%
\isatagproof
\isacommand{proof}\isamarkupfalse%
\ {\isacharminus}\isanewline
\ \ \isacommand{let}\isamarkupfalse%
\ {\isacharquery}R{\isacharequal}{\isachardoublequoteopen}a\ outside\ X{\isachardoublequoteclose}\ \isacommand{let}\isamarkupfalse%
\ {\isacharquery}I{\isacharequal}{\isachardoublequoteopen}{\isacharbraceleft}i{\isacharbraceright}{\isachardoublequoteclose}\ \isacommand{let}\isamarkupfalse%
\ {\isacharquery}U{\isacharequal}Union\ \isacommand{let}\isamarkupfalse%
\ {\isacharquery}u{\isacharequal}runiq\ \isacommand{let}\isamarkupfalse%
\ {\isacharquery}r{\isacharequal}Range\ \isacommand{let}\isamarkupfalse%
\ {\isacharquery}d{\isacharequal}Domain\isanewline
\ \ \isacommand{let}\isamarkupfalse%
\ {\isacharquery}aa{\isacharequal}{\isachardoublequoteopen}a\ outside\ {\isacharparenleft}X\ {\isasymunion}\ {\isacharbraceleft}i{\isacharbraceright}{\isacharparenright}\ {\isasymunion}\ {\isacharparenleft}{\isacharbraceleft}i{\isacharbraceright}\ {\isasymtimes}\ {\isacharbraceleft}{\isacharquery}U{\isacharparenleft}a{\isacharbackquote}{\isacharbackquote}{\isacharparenleft}X{\isasymunion}{\isacharbraceleft}i{\isacharbraceright}{\isacharparenright}{\isacharparenright}{\isacharbraceright}{\isacharparenright}{\isachardoublequoteclose}\ \isacommand{have}\isamarkupfalse%
\ \isanewline
\ \ {\isadigit{1}}{\isacharcolon}\ {\isachardoublequoteopen}a\ {\isasymin}\ allocationsUniverse{\isachardoublequoteclose}\ \isacommand{using}\isamarkupfalse%
\ assms{\isacharparenleft}{\isadigit{1}}{\isacharparenright}\ lm{\isadigit{5}}{\isadigit{0}}\ \ set{\isacharunderscore}rev{\isacharunderscore}mp\ \isacommand{by}\isamarkupfalse%
\ blast\isanewline
\ \ \isacommand{have}\isamarkupfalse%
\ {\isachardoublequoteopen}i\ {\isasymnotin}\ X{\isachardoublequoteclose}\ \isacommand{using}\isamarkupfalse%
\ assms\ \isacommand{by}\isamarkupfalse%
\ fast\ \isacommand{then}\isamarkupfalse%
\ \isacommand{have}\isamarkupfalse%
\ \isanewline
\ \ {\isadigit{2}}{\isacharcolon}\ {\isachardoublequoteopen}{\isacharquery}d\ a\ {\isacharminus}\ X\ {\isasymunion}\ {\isacharbraceleft}i{\isacharbraceright}\ {\isacharequal}\ {\isacharquery}d\ a\ {\isasymunion}\ {\isacharbraceleft}i{\isacharbraceright}\ {\isacharminus}\ X{\isachardoublequoteclose}\ \isacommand{by}\isamarkupfalse%
\ fast\isanewline
\ \ \isacommand{have}\isamarkupfalse%
\ {\isachardoublequoteopen}a\ {\isasymin}\ allocationsUniverse{\isachardoublequoteclose}\ \isacommand{using}\isamarkupfalse%
\ {\isadigit{1}}\ \isacommand{by}\isamarkupfalse%
\ fast\ \isacommand{moreover}\isamarkupfalse%
\ \isacommand{have}\isamarkupfalse%
\ {\isachardoublequoteopen}{\isacharquery}d\ a\ {\isasyminter}\ X\ {\isasymnoteq}\ {\isacharbraceleft}{\isacharbraceright}{\isachardoublequoteclose}\ \isacommand{using}\isamarkupfalse%
\ assms\ \isacommand{by}\isamarkupfalse%
\ fast\ \isanewline
\ \ \isacommand{ultimately}\isamarkupfalse%
\ \isacommand{have}\isamarkupfalse%
\ {\isachardoublequoteopen}{\isacharquery}aa\ {\isasymin}\ allocationsUniverse\ {\isacharampersand}\ {\isacharquery}U\ {\isacharparenleft}{\isacharquery}r\ {\isacharquery}aa{\isacharparenright}\ {\isacharequal}\ {\isacharquery}U\ {\isacharparenleft}{\isacharquery}r\ a{\isacharparenright}{\isachardoublequoteclose}\ \isacommand{apply}\isamarkupfalse%
\ {\isacharparenleft}rule\ lm{\isadigit{4}}{\isadigit{3}}c{\isacharparenright}\ \isacommand{done}\isamarkupfalse%
\isanewline
\ \ \isacommand{then}\isamarkupfalse%
\ \isacommand{have}\isamarkupfalse%
\ {\isachardoublequoteopen}{\isacharquery}aa\ {\isasymin}\ possibleAllocationsRel\ {\isacharparenleft}{\isacharquery}d\ {\isacharquery}aa{\isacharparenright}\ {\isacharparenleft}{\isacharquery}U\ {\isacharparenleft}{\isacharquery}r\ a{\isacharparenright}{\isacharparenright}{\isachardoublequoteclose}\isanewline
\isacommand{using}\isamarkupfalse%
\ lm{\isadigit{3}}{\isadigit{4}}\ \isacommand{by}\isamarkupfalse%
\ {\isacharparenleft}metis\ {\isacharparenleft}lifting{\isacharcomma}\ mono{\isacharunderscore}tags{\isacharparenright}{\isacharparenright}\isanewline
\isacommand{then}\isamarkupfalse%
\ \isacommand{have}\isamarkupfalse%
\ {\isachardoublequoteopen}{\isacharquery}aa\ {\isasymin}\ possibleAllocationsRel\ {\isacharparenleft}{\isacharquery}d\ {\isacharquery}aa\ {\isasymunion}\ {\isacharparenleft}{\isacharquery}d\ a\ {\isacharminus}\ X\ {\isasymunion}\ {\isacharbraceleft}i{\isacharbraceright}{\isacharparenright}{\isacharparenright}\ \ {\isacharparenleft}{\isacharquery}U\ {\isacharparenleft}{\isacharquery}r\ a{\isacharparenright}{\isacharparenright}{\isachardoublequoteclose}\isanewline
\isacommand{by}\isamarkupfalse%
\ {\isacharparenleft}metis\ lm{\isadigit{1}}{\isadigit{9}}d{\isacharparenright}\isanewline
\ \ \isacommand{moreover}\isamarkupfalse%
\ \isacommand{have}\isamarkupfalse%
\ {\isachardoublequoteopen}{\isacharquery}d\ a\ {\isacharminus}\ X\ {\isasymunion}\ {\isacharbraceleft}i{\isacharbraceright}\ {\isacharequal}\ {\isacharquery}d\ {\isacharquery}aa\ {\isasymunion}\ {\isacharparenleft}{\isacharquery}d\ a\ {\isacharminus}\ X\ {\isasymunion}\ {\isacharbraceleft}i{\isacharbraceright}{\isacharparenright}{\isachardoublequoteclose}\ \isacommand{using}\isamarkupfalse%
\ Outside{\isacharunderscore}def\ \isacommand{by}\isamarkupfalse%
\ auto\isanewline
\ \ \isacommand{ultimately}\isamarkupfalse%
\ \isacommand{have}\isamarkupfalse%
\ {\isachardoublequoteopen}{\isacharquery}aa\ {\isasymin}\ possibleAllocationsRel\ {\isacharparenleft}{\isacharquery}d\ a\ {\isacharminus}\ X\ {\isasymunion}\ {\isacharbraceleft}i{\isacharbraceright}{\isacharparenright}\ {\isacharparenleft}{\isacharquery}U\ {\isacharparenleft}{\isacharquery}r\ a{\isacharparenright}{\isacharparenright}{\isachardoublequoteclose}\ \isacommand{by}\isamarkupfalse%
\ simp\isanewline
\ \ \isacommand{then}\isamarkupfalse%
\ \isacommand{have}\isamarkupfalse%
\ {\isachardoublequoteopen}{\isacharquery}aa\ {\isasymin}\ possibleAllocationsRel\ {\isacharparenleft}{\isacharquery}d\ a\ {\isasymunion}\ {\isacharbraceleft}i{\isacharbraceright}\ {\isacharminus}\ X{\isacharparenright}\ {\isacharparenleft}{\isacharquery}U\ {\isacharparenleft}{\isacharquery}r\ a{\isacharparenright}{\isacharparenright}{\isachardoublequoteclose}\ \isacommand{using}\isamarkupfalse%
\ {\isadigit{2}}\ \isacommand{by}\isamarkupfalse%
\ simp\isanewline
\ \ \isacommand{moreover}\isamarkupfalse%
\ \isacommand{have}\isamarkupfalse%
\ {\isachardoublequoteopen}{\isacharquery}d\ a\ {\isasymsubseteq}\ N{\isachardoublequoteclose}\ \isacommand{using}\isamarkupfalse%
\ assms{\isacharparenleft}{\isadigit{1}}{\isacharparenright}\ lm{\isadigit{1}}{\isadigit{9}}c\ \isacommand{by}\isamarkupfalse%
\ metis\isanewline
\ \ \isacommand{then}\isamarkupfalse%
\ \isacommand{moreover}\isamarkupfalse%
\ \isacommand{have}\isamarkupfalse%
\ {\isachardoublequoteopen}{\isacharparenleft}{\isacharquery}d\ a\ {\isasymunion}\ {\isacharbraceleft}i{\isacharbraceright}\ {\isacharminus}\ X{\isacharparenright}\ {\isasymunion}\ {\isacharparenleft}N\ {\isacharminus}\ X{\isacharparenright}\ {\isacharequal}\ N\ {\isacharminus}\ X{\isachardoublequoteclose}\ \isacommand{using}\isamarkupfalse%
\ assms\ \isacommand{by}\isamarkupfalse%
\ fast\isanewline
\ \ \isacommand{ultimately}\isamarkupfalse%
\ \isacommand{have}\isamarkupfalse%
\ {\isachardoublequoteopen}{\isacharquery}aa\ {\isasymin}\ possibleAllocationsRel\ {\isacharparenleft}N\ {\isacharminus}\ X{\isacharparenright}\ {\isacharparenleft}{\isacharquery}U\ {\isacharparenleft}{\isacharquery}r\ a{\isacharparenright}{\isacharparenright}{\isachardoublequoteclose}\ \isacommand{using}\isamarkupfalse%
\ lm{\isadigit{1}}{\isadigit{9}}b\ \isanewline
\ \ \isacommand{by}\isamarkupfalse%
\ {\isacharparenleft}metis\ {\isacharparenleft}lifting{\isacharcomma}\ no{\isacharunderscore}types{\isacharparenright}\ in{\isacharunderscore}mono{\isacharparenright}\isanewline
\ \ \isacommand{then}\isamarkupfalse%
\ \isacommand{show}\isamarkupfalse%
\ {\isacharquery}thesis\ \isacommand{by}\isamarkupfalse%
\ fast\isanewline
\isacommand{qed}\isamarkupfalse%
%
\endisatagproof
{\isafoldproof}%
%
\isadelimproof
\isanewline
%
\endisadelimproof
\isanewline
\isacommand{lemma}\isamarkupfalse%
\ lm{\isadigit{5}}{\isadigit{2}}{\isacharcolon}\ \isakeyword{assumes}\ \isanewline
{\isachardoublequoteopen}condition{\isadigit{1}}\ {\isacharparenleft}b{\isacharcolon}{\isacharcolon}{\isacharunderscore}\ {\isacharequal}{\isachargreater}\ real{\isacharparenright}\ i{\isachardoublequoteclose}\ \isanewline
{\isachardoublequoteopen}a\ {\isasymin}\ allocationsUniverse{\isachardoublequoteclose}\ \isanewline
{\isachardoublequoteopen}Domain\ a\ {\isasyminter}\ X\ {\isasymnoteq}\ {\isacharbraceleft}{\isacharbraceright}{\isachardoublequoteclose}\ \isanewline
{\isachardoublequoteopen}finite\ a{\isachardoublequoteclose}\ \isakeyword{shows}\ \isanewline
{\isachardoublequoteopen}setsum\ b\ {\isacharparenleft}a\ outside\ X{\isacharparenright}\ {\isasymle}\ setsum\ b\ {\isacharparenleft}a\ outside\ {\isacharparenleft}X\ {\isasymunion}\ {\isacharbraceleft}i{\isacharbraceright}{\isacharparenright}\ {\isasymunion}\ {\isacharparenleft}{\isacharbraceleft}i{\isacharbraceright}\ {\isasymtimes}\ {\isacharbraceleft}{\isasymUnion}{\isacharparenleft}a{\isacharbackquote}{\isacharbackquote}{\isacharparenleft}X{\isasymunion}{\isacharbraceleft}i{\isacharbraceright}{\isacharparenright}{\isacharparenright}{\isacharbraceright}{\isacharparenright}{\isacharparenright}{\isachardoublequoteclose}\isanewline
%
\isadelimproof
%
\endisadelimproof
%
\isatagproof
\isacommand{proof}\isamarkupfalse%
\ {\isacharminus}\isanewline
\ \ \isacommand{let}\isamarkupfalse%
\ {\isacharquery}R{\isacharequal}{\isachardoublequoteopen}a\ outside\ X{\isachardoublequoteclose}\ \isacommand{let}\isamarkupfalse%
\ {\isacharquery}I{\isacharequal}{\isachardoublequoteopen}{\isacharbraceleft}i{\isacharbraceright}{\isachardoublequoteclose}\ \isacommand{let}\isamarkupfalse%
\ {\isacharquery}U{\isacharequal}Union\ \isacommand{let}\isamarkupfalse%
\ {\isacharquery}u{\isacharequal}runiq\ \isacommand{let}\isamarkupfalse%
\ {\isacharquery}r{\isacharequal}Range\ \isacommand{let}\isamarkupfalse%
\ {\isacharquery}d{\isacharequal}Domain\isanewline
\ \ \isacommand{let}\isamarkupfalse%
\ {\isacharquery}aa{\isacharequal}{\isachardoublequoteopen}a\ outside\ {\isacharparenleft}X\ {\isasymunion}\ {\isacharbraceleft}i{\isacharbraceright}{\isacharparenright}\ {\isasymunion}\ {\isacharparenleft}{\isacharbraceleft}i{\isacharbraceright}\ {\isasymtimes}\ {\isacharbraceleft}{\isacharquery}U{\isacharparenleft}a{\isacharbackquote}{\isacharbackquote}{\isacharparenleft}X{\isasymunion}{\isacharbraceleft}i{\isacharbraceright}{\isacharparenright}{\isacharparenright}{\isacharbraceright}{\isacharparenright}{\isachardoublequoteclose}\isanewline
\ \ \isacommand{have}\isamarkupfalse%
\ {\isachardoublequoteopen}a\ {\isasymin}\ injectionsUniverse{\isachardoublequoteclose}\ \isacommand{using}\isamarkupfalse%
\ assms\ \isacommand{by}\isamarkupfalse%
\ fast\ \isacommand{then}\isamarkupfalse%
\ \isacommand{have}\isamarkupfalse%
\ \isanewline
\ \ {\isadigit{0}}{\isacharcolon}\ {\isachardoublequoteopen}{\isacharquery}u\ a{\isachardoublequoteclose}\ \isacommand{by}\isamarkupfalse%
\ simp\isanewline
\ \ \isacommand{moreover}\isamarkupfalse%
\ \isacommand{have}\isamarkupfalse%
\ {\isachardoublequoteopen}{\isacharquery}R\ {\isasymsubseteq}\ a\ {\isacharampersand}\ {\isacharquery}R{\isacharminus}{\isacharminus}i\ {\isasymsubseteq}\ a{\isachardoublequoteclose}\ \isacommand{using}\isamarkupfalse%
\ Outside{\isacharunderscore}def\ \isacommand{by}\isamarkupfalse%
\ blast\isanewline
\ \ \isacommand{ultimately}\isamarkupfalse%
\ \isacommand{have}\isamarkupfalse%
\ {\isachardoublequoteopen}finite\ {\isacharparenleft}{\isacharquery}R\ {\isacharminus}{\isacharminus}\ i{\isacharparenright}\ {\isacharampersand}\ {\isacharquery}u\ {\isacharparenleft}{\isacharquery}R{\isacharminus}{\isacharminus}i{\isacharparenright}\ {\isacharampersand}\ {\isacharquery}u\ {\isacharquery}R{\isachardoublequoteclose}\ \isacommand{using}\isamarkupfalse%
\ finite{\isacharunderscore}subset\ subrel{\isacharunderscore}runiq\isanewline
\ \ \isacommand{by}\isamarkupfalse%
\ {\isacharparenleft}metis\ assms{\isacharparenleft}{\isadigit{4}}{\isacharparenright}{\isacharparenright}\isanewline
\ \ \isacommand{then}\isamarkupfalse%
\ \isacommand{moreover}\isamarkupfalse%
\ \isacommand{have}\isamarkupfalse%
\ {\isachardoublequoteopen}trivial\ {\isacharparenleft}{\isacharbraceleft}i{\isacharbraceright}{\isasymtimes}{\isacharparenleft}{\isacharquery}R{\isacharbackquote}{\isacharbackquote}{\isacharbraceleft}i{\isacharbraceright}{\isacharparenright}{\isacharparenright}{\isachardoublequoteclose}\ \isacommand{using}\isamarkupfalse%
\ runiq{\isacharunderscore}def\ \isanewline
\ \ \isacommand{by}\isamarkupfalse%
\ {\isacharparenleft}metis\ ll{\isadigit{4}}{\isadigit{0}}\ trivial{\isacharunderscore}singleton{\isacharparenright}\isanewline
\ \ \isacommand{moreover}\isamarkupfalse%
\ \isacommand{have}\isamarkupfalse%
\ {\isachardoublequoteopen}{\isasymforall}X{\isachardot}\ {\isacharparenleft}{\isacharquery}R\ {\isacharminus}{\isacharminus}\ i{\isacharparenright}\ {\isasyminter}\ {\isacharparenleft}{\isacharbraceleft}i{\isacharbraceright}{\isasymtimes}X{\isacharparenright}{\isacharequal}{\isacharbraceleft}{\isacharbraceright}{\isachardoublequoteclose}\ \isacommand{using}\isamarkupfalse%
\ outside{\isacharunderscore}reduces{\isacharunderscore}domain\ \isacommand{by}\isamarkupfalse%
\ force\isanewline
\ \ \isacommand{ultimately}\isamarkupfalse%
\ \isacommand{have}\isamarkupfalse%
\ \isanewline
\ \ {\isadigit{1}}{\isacharcolon}\ {\isachardoublequoteopen}finite\ {\isacharparenleft}{\isacharquery}R{\isacharminus}{\isacharminus}i{\isacharparenright}\ {\isacharampersand}\ finite\ {\isacharparenleft}{\isacharbraceleft}i{\isacharbraceright}{\isasymtimes}{\isacharparenleft}{\isacharquery}R{\isacharbackquote}{\isacharbackquote}{\isacharbraceleft}i{\isacharbraceright}{\isacharparenright}{\isacharparenright}\ {\isacharampersand}\ {\isacharparenleft}{\isacharquery}R\ {\isacharminus}{\isacharminus}\ i{\isacharparenright}\ {\isasyminter}\ {\isacharparenleft}{\isacharbraceleft}i{\isacharbraceright}{\isasymtimes}{\isacharparenleft}{\isacharquery}R{\isacharbackquote}{\isacharbackquote}{\isacharbraceleft}i{\isacharbraceright}{\isacharparenright}{\isacharparenright}{\isacharequal}{\isacharbraceleft}{\isacharbraceright}\ {\isacharampersand}\ \isanewline
\ \ finite\ {\isacharparenleft}{\isacharbraceleft}i{\isacharbraceright}\ {\isasymtimes}\ {\isacharbraceleft}{\isacharquery}U{\isacharparenleft}a{\isacharbackquote}{\isacharbackquote}{\isacharparenleft}X{\isasymunion}{\isacharbraceleft}i{\isacharbraceright}{\isacharparenright}{\isacharparenright}{\isacharbraceright}{\isacharparenright}\ {\isacharampersand}\ {\isacharparenleft}{\isacharquery}R\ {\isacharminus}{\isacharminus}\ i{\isacharparenright}\ {\isasyminter}\ {\isacharparenleft}{\isacharbraceleft}i{\isacharbraceright}\ {\isasymtimes}\ {\isacharbraceleft}{\isacharquery}U{\isacharparenleft}a{\isacharbackquote}{\isacharbackquote}{\isacharparenleft}X{\isasymunion}{\isacharbraceleft}i{\isacharbraceright}{\isacharparenright}{\isacharparenright}{\isacharbraceright}{\isacharparenright}{\isacharequal}{\isacharbraceleft}{\isacharbraceright}{\isachardoublequoteclose}\ \isanewline
\ \ \isacommand{using}\isamarkupfalse%
\ Outside{\isacharunderscore}def\ lm{\isadigit{5}}{\isadigit{4}}\ \isacommand{by}\isamarkupfalse%
\ fast\ \isanewline
\ \ \isacommand{have}\isamarkupfalse%
\ {\isachardoublequoteopen}{\isacharquery}R\ {\isacharequal}\ {\isacharparenleft}{\isacharquery}R\ {\isacharminus}{\isacharminus}\ i{\isacharparenright}\ {\isasymunion}\ {\isacharparenleft}{\isacharbraceleft}i{\isacharbraceright}{\isasymtimes}{\isacharquery}R{\isacharbackquote}{\isacharbackquote}{\isacharbraceleft}i{\isacharbraceright}{\isacharparenright}{\isachardoublequoteclose}\ \isacommand{by}\isamarkupfalse%
\ {\isacharparenleft}metis\ l{\isadigit{3}}{\isadigit{9}}{\isacharparenright}\isanewline
\ \ \isacommand{then}\isamarkupfalse%
\ \isacommand{have}\isamarkupfalse%
\ {\isachardoublequoteopen}setsum\ b\ {\isacharquery}R\ {\isacharequal}\ setsum\ b\ {\isacharparenleft}{\isacharquery}R\ {\isacharminus}{\isacharminus}\ i{\isacharparenright}\ {\isacharplus}\ setsum\ b\ {\isacharparenleft}{\isacharbraceleft}i{\isacharbraceright}{\isasymtimes}{\isacharparenleft}{\isacharquery}R{\isacharbackquote}{\isacharbackquote}{\isacharbraceleft}i{\isacharbraceright}{\isacharparenright}{\isacharparenright}{\isachardoublequoteclose}\ \isanewline
\ \ \isacommand{using}\isamarkupfalse%
\ {\isadigit{1}}\ setsum{\isachardot}union{\isacharunderscore}disjoint\ \isacommand{by}\isamarkupfalse%
\ {\isacharparenleft}metis\ {\isacharparenleft}lifting{\isacharparenright}\ setsum{\isachardot}union{\isacharunderscore}disjoint{\isacharparenright}\isanewline
\ \ \isacommand{moreover}\isamarkupfalse%
\ \isacommand{have}\isamarkupfalse%
\ {\isachardoublequoteopen}setsum\ b\ {\isacharparenleft}{\isacharbraceleft}i{\isacharbraceright}{\isasymtimes}{\isacharparenleft}{\isacharquery}R{\isacharbackquote}{\isacharbackquote}{\isacharbraceleft}i{\isacharbraceright}{\isacharparenright}{\isacharparenright}\ {\isasymle}\ setsum\ b\ {\isacharparenleft}{\isacharbraceleft}i{\isacharbraceright}{\isasymtimes}{\isacharbraceleft}{\isacharquery}U{\isacharparenleft}a{\isacharbackquote}{\isacharbackquote}{\isacharparenleft}X{\isasymunion}{\isacharbraceleft}i{\isacharbraceright}{\isacharparenright}{\isacharparenright}{\isacharbraceright}{\isacharparenright}{\isachardoublequoteclose}\ \isacommand{using}\isamarkupfalse%
\ lm{\isadigit{4}}{\isadigit{6}}\ \isanewline
\ \ assms{\isacharparenleft}{\isadigit{1}}{\isacharparenright}\ {\isadigit{0}}\ \isacommand{by}\isamarkupfalse%
\ metis\isanewline
\ \ \isacommand{ultimately}\isamarkupfalse%
\ \isacommand{have}\isamarkupfalse%
\ {\isachardoublequoteopen}setsum\ b\ {\isacharquery}R\ {\isasymle}\ setsum\ b\ {\isacharparenleft}{\isacharquery}R\ {\isacharminus}{\isacharminus}\ i{\isacharparenright}\ {\isacharplus}\ setsum\ b\ {\isacharparenleft}{\isacharbraceleft}i{\isacharbraceright}{\isasymtimes}{\isacharbraceleft}{\isacharquery}U{\isacharparenleft}a{\isacharbackquote}{\isacharbackquote}{\isacharparenleft}X{\isasymunion}{\isacharbraceleft}i{\isacharbraceright}{\isacharparenright}{\isacharparenright}{\isacharbraceright}{\isacharparenright}{\isachardoublequoteclose}\ \isacommand{by}\isamarkupfalse%
\ linarith\isanewline
\ \ \isacommand{moreover}\isamarkupfalse%
\ \isacommand{have}\isamarkupfalse%
\ {\isachardoublequoteopen}{\isachardot}{\isachardot}{\isachardot}\ {\isacharequal}\ setsum\ b\ {\isacharparenleft}{\isacharquery}R\ {\isacharminus}{\isacharminus}\ i\ {\isasymunion}\ {\isacharparenleft}{\isacharbraceleft}i{\isacharbraceright}\ {\isasymtimes}\ {\isacharbraceleft}{\isacharquery}U{\isacharparenleft}a{\isacharbackquote}{\isacharbackquote}{\isacharparenleft}X{\isasymunion}{\isacharbraceleft}i{\isacharbraceright}{\isacharparenright}{\isacharparenright}{\isacharbraceright}{\isacharparenright}{\isacharparenright}{\isachardoublequoteclose}\ \isanewline
\ \ \isacommand{using}\isamarkupfalse%
\ {\isadigit{1}}\ setsum{\isachardot}union{\isacharunderscore}disjoint\ \isacommand{by}\isamarkupfalse%
\ auto\isanewline
\ \ \isacommand{moreover}\isamarkupfalse%
\ \isacommand{have}\isamarkupfalse%
\ {\isachardoublequoteopen}{\isachardot}{\isachardot}{\isachardot}\ {\isacharequal}\ setsum\ b\ {\isacharquery}aa{\isachardoublequoteclose}\ \isacommand{by}\isamarkupfalse%
\ {\isacharparenleft}metis\ ll{\isadigit{5}}{\isadigit{2}}{\isacharparenright}\isanewline
\ \ \isacommand{ultimately}\isamarkupfalse%
\ \isacommand{show}\isamarkupfalse%
\ {\isacharquery}thesis\ \isacommand{by}\isamarkupfalse%
\ linarith\isanewline
\isacommand{qed}\isamarkupfalse%
%
\endisatagproof
{\isafoldproof}%
%
\isadelimproof
\isanewline
%
\endisadelimproof
\isanewline
\isacommand{lemma}\isamarkupfalse%
\ lm{\isadigit{5}}{\isadigit{5}}{\isacharcolon}\ \isakeyword{assumes}\ {\isachardoublequoteopen}finite\ X{\isachardoublequoteclose}\ {\isachardoublequoteopen}XX\ {\isasymin}\ all{\isacharunderscore}partitions\ X{\isachardoublequoteclose}\ \isakeyword{shows}\ {\isachardoublequoteopen}finite\ XX{\isachardoublequoteclose}%
\isadelimproof
\ %
\endisadelimproof
%
\isatagproof
\isacommand{using}\isamarkupfalse%
\ \isanewline
all{\isacharunderscore}partitions{\isacharunderscore}def\ is{\isacharunderscore}partition{\isacharunderscore}of{\isacharunderscore}def\ \isanewline
\isacommand{by}\isamarkupfalse%
\ {\isacharparenleft}metis\ assms{\isacharparenleft}{\isadigit{1}}{\isacharparenright}\ assms{\isacharparenleft}{\isadigit{2}}{\isacharparenright}\ finite{\isacharunderscore}UnionD\ mem{\isacharunderscore}Collect{\isacharunderscore}eq{\isacharparenright}%
\endisatagproof
{\isafoldproof}%
%
\isadelimproof
%
\endisadelimproof
\isanewline
\isanewline
\isacommand{lemma}\isamarkupfalse%
\ lm{\isadigit{5}}{\isadigit{8}}{\isacharcolon}\ \isakeyword{assumes}\ {\isachardoublequoteopen}finite\ N{\isachardoublequoteclose}\ {\isachardoublequoteopen}finite\ G{\isachardoublequoteclose}\ {\isachardoublequoteopen}a\ {\isasymin}\ possibleAllocationsRel\ N\ G{\isachardoublequoteclose}\isanewline
\isakeyword{shows}\ {\isachardoublequoteopen}finite\ a{\isachardoublequoteclose}%
\isadelimproof
\ %
\endisadelimproof
%
\isatagproof
\isacommand{using}\isamarkupfalse%
\ assms\ lm{\isadigit{5}}{\isadigit{7}}\ rev{\isacharunderscore}finite{\isacharunderscore}subset\ \isacommand{by}\isamarkupfalse%
\ {\isacharparenleft}metis\ lm{\isadigit{2}}{\isadigit{8}}b\ lm{\isadigit{5}}{\isadigit{5}}{\isacharparenright}%
\endisatagproof
{\isafoldproof}%
%
\isadelimproof
%
\endisadelimproof
\isanewline
\isanewline
\isacommand{lemma}\isamarkupfalse%
\ lm{\isadigit{5}}{\isadigit{9}}{\isacharcolon}\ \isakeyword{assumes}\ {\isachardoublequoteopen}finite\ N{\isachardoublequoteclose}\ {\isachardoublequoteopen}finite\ G{\isachardoublequoteclose}\ \isakeyword{shows}\ {\isachardoublequoteopen}finite\ {\isacharparenleft}possibleAllocationsRel\ N\ G{\isacharparenright}{\isachardoublequoteclose}\isanewline
%
\isadelimproof
%
\endisadelimproof
%
\isatagproof
\isacommand{proof}\isamarkupfalse%
\ {\isacharminus}\isanewline
\isacommand{have}\isamarkupfalse%
\ {\isachardoublequoteopen}finite\ {\isacharparenleft}Pow{\isacharparenleft}N{\isasymtimes}{\isacharparenleft}Pow\ G{\isacharminus}{\isacharbraceleft}{\isacharbraceleft}{\isacharbraceright}{\isacharbraceright}{\isacharparenright}{\isacharparenright}{\isacharparenright}{\isachardoublequoteclose}\ \isacommand{using}\isamarkupfalse%
\ assms\ finite{\isacharunderscore}Pow{\isacharunderscore}iff\ \isacommand{by}\isamarkupfalse%
\ blast\isanewline
\isacommand{then}\isamarkupfalse%
\ \isacommand{show}\isamarkupfalse%
\ {\isacharquery}thesis\ \isacommand{using}\isamarkupfalse%
\ mm{\isadigit{6}}{\isadigit{3}}e\ rev{\isacharunderscore}finite{\isacharunderscore}subset\ \isacommand{by}\isamarkupfalse%
\ {\isacharparenleft}metis{\isacharparenleft}no{\isacharunderscore}types{\isacharparenright}{\isacharparenright}\isanewline
\isacommand{qed}\isamarkupfalse%
%
\endisatagproof
{\isafoldproof}%
%
\isadelimproof
\isanewline
%
\endisadelimproof
\isanewline
\isacommand{corollary}\isamarkupfalse%
\ lm{\isadigit{5}}{\isadigit{3}}{\isacharcolon}\ \isakeyword{assumes}\ {\isachardoublequoteopen}condition{\isadigit{1}}\ {\isacharparenleft}b{\isacharcolon}{\isacharcolon}{\isacharunderscore}\ {\isacharequal}{\isachargreater}\ real{\isacharparenright}\ i{\isachardoublequoteclose}\ {\isachardoublequoteopen}a\ {\isasymin}\ possibleAllocationsRel\ N\ G{\isachardoublequoteclose}\ {\isachardoublequoteopen}i{\isasymin}N{\isacharminus}X{\isachardoublequoteclose}\ \isanewline
{\isachardoublequoteopen}Domain\ a\ {\isasyminter}\ X\ {\isasymnoteq}\ {\isacharbraceleft}{\isacharbraceright}{\isachardoublequoteclose}\ {\isachardoublequoteopen}finite\ N{\isachardoublequoteclose}\ {\isachardoublequoteopen}finite\ G{\isachardoublequoteclose}\ \isakeyword{shows}\isanewline
{\isachardoublequoteopen}Max\ {\isacharparenleft}{\isacharparenleft}setsum\ b{\isacharparenright}{\isacharbackquote}{\isacharparenleft}possibleAllocationsRel\ {\isacharparenleft}N{\isacharminus}X{\isacharparenright}\ G{\isacharparenright}{\isacharparenright}\ {\isasymge}\ setsum\ b\ {\isacharparenleft}a\ outside\ X{\isacharparenright}{\isachardoublequoteclose}\isanewline
%
\isadelimproof
%
\endisadelimproof
%
\isatagproof
\isacommand{proof}\isamarkupfalse%
\ {\isacharminus}\isanewline
\isacommand{let}\isamarkupfalse%
\ {\isacharquery}aa{\isacharequal}{\isachardoublequoteopen}a\ outside\ {\isacharparenleft}X\ {\isasymunion}\ {\isacharbraceleft}i{\isacharbraceright}{\isacharparenright}\ {\isasymunion}\ {\isacharparenleft}{\isacharbraceleft}i{\isacharbraceright}\ {\isasymtimes}\ {\isacharbraceleft}{\isasymUnion}{\isacharparenleft}a{\isacharbackquote}{\isacharbackquote}{\isacharparenleft}X{\isasymunion}{\isacharbraceleft}i{\isacharbraceright}{\isacharparenright}{\isacharparenright}{\isacharbraceright}{\isacharparenright}{\isachardoublequoteclose}\isanewline
\isacommand{have}\isamarkupfalse%
\ {\isachardoublequoteopen}condition{\isadigit{1}}\ {\isacharparenleft}b{\isacharcolon}{\isacharcolon}{\isacharunderscore}\ {\isacharequal}{\isachargreater}\ real{\isacharparenright}\ i{\isachardoublequoteclose}\ \isacommand{using}\isamarkupfalse%
\ assms{\isacharparenleft}{\isadigit{1}}{\isacharparenright}\ \isacommand{by}\isamarkupfalse%
\ fast\isanewline
\isacommand{moreover}\isamarkupfalse%
\ \isacommand{have}\isamarkupfalse%
\ {\isachardoublequoteopen}a\ {\isasymin}\ allocationsUniverse{\isachardoublequoteclose}\ \isacommand{using}\isamarkupfalse%
\ assms{\isacharparenleft}{\isadigit{2}}{\isacharparenright}\ lm{\isadigit{5}}{\isadigit{0}}\ \isacommand{by}\isamarkupfalse%
\ blast\isanewline
\isacommand{moreover}\isamarkupfalse%
\ \isacommand{have}\isamarkupfalse%
\ {\isachardoublequoteopen}Domain\ a\ {\isasyminter}\ X\ {\isasymnoteq}\ {\isacharbraceleft}{\isacharbraceright}{\isachardoublequoteclose}\ \isacommand{using}\isamarkupfalse%
\ assms{\isacharparenleft}{\isadigit{4}}{\isacharparenright}\ \isacommand{by}\isamarkupfalse%
\ auto\isanewline
\isacommand{moreover}\isamarkupfalse%
\ \isacommand{have}\isamarkupfalse%
\ {\isachardoublequoteopen}finite\ a{\isachardoublequoteclose}\ \isacommand{using}\isamarkupfalse%
\ assms\ lm{\isadigit{5}}{\isadigit{8}}\ \isacommand{by}\isamarkupfalse%
\ metis\ \isacommand{ultimately}\isamarkupfalse%
\ \isacommand{have}\isamarkupfalse%
\ \isanewline
{\isadigit{0}}{\isacharcolon}\ {\isachardoublequoteopen}setsum\ b\ {\isacharparenleft}a\ outside\ X{\isacharparenright}\ {\isasymle}\ setsum\ b\ {\isacharquery}aa{\isachardoublequoteclose}\ \isacommand{by}\isamarkupfalse%
\ {\isacharparenleft}rule\ lm{\isadigit{5}}{\isadigit{2}}{\isacharparenright}\isanewline
\isacommand{have}\isamarkupfalse%
\ {\isachardoublequoteopen}{\isacharquery}aa\ {\isasymin}\ possibleAllocationsRel\ {\isacharparenleft}N{\isacharminus}X{\isacharparenright}\ {\isacharparenleft}{\isasymUnion}\ {\isacharparenleft}Range\ a{\isacharparenright}{\isacharparenright}{\isachardoublequoteclose}\ \isacommand{using}\isamarkupfalse%
\ assms\ lm{\isadigit{5}}{\isadigit{1}}\ \isacommand{by}\isamarkupfalse%
\ metis\isanewline
\isacommand{moreover}\isamarkupfalse%
\ \isacommand{have}\isamarkupfalse%
\ {\isachardoublequoteopen}{\isasymUnion}\ {\isacharparenleft}Range\ a{\isacharparenright}\ {\isacharequal}\ G{\isachardoublequoteclose}\ \isacommand{using}\isamarkupfalse%
\ assms\ lm{\isadigit{4}}{\isadigit{7}}\ is{\isacharunderscore}partition{\isacharunderscore}of{\isacharunderscore}def\ \isacommand{by}\isamarkupfalse%
\ metis\isanewline
\isacommand{ultimately}\isamarkupfalse%
\ \isacommand{have}\isamarkupfalse%
\ {\isachardoublequoteopen}setsum\ b\ {\isacharquery}aa\ {\isasymin}\ {\isacharparenleft}setsum\ b{\isacharparenright}{\isacharbackquote}{\isacharparenleft}possibleAllocationsRel\ {\isacharparenleft}N{\isacharminus}X{\isacharparenright}\ G{\isacharparenright}{\isachardoublequoteclose}\ \isacommand{by}\isamarkupfalse%
\ {\isacharparenleft}metis\ imageI{\isacharparenright}\isanewline
\isacommand{moreover}\isamarkupfalse%
\ \isacommand{have}\isamarkupfalse%
\ {\isachardoublequoteopen}finite\ {\isacharparenleft}{\isacharparenleft}setsum\ b{\isacharparenright}{\isacharbackquote}{\isacharparenleft}possibleAllocationsRel\ {\isacharparenleft}N{\isacharminus}X{\isacharparenright}\ G{\isacharparenright}{\isacharparenright}{\isachardoublequoteclose}\ \isacommand{using}\isamarkupfalse%
\ assms\ lm{\isadigit{5}}{\isadigit{9}}\ assms{\isacharparenleft}{\isadigit{5}}{\isacharcomma}{\isadigit{6}}{\isacharparenright}\isanewline
\isacommand{by}\isamarkupfalse%
\ {\isacharparenleft}metis\ finite{\isacharunderscore}Diff\ finite{\isacharunderscore}imageI{\isacharparenright}\isanewline
\isacommand{ultimately}\isamarkupfalse%
\ \isacommand{have}\isamarkupfalse%
\ {\isachardoublequoteopen}setsum\ b\ {\isacharquery}aa\ {\isasymle}\ Max\ {\isacharparenleft}{\isacharparenleft}setsum\ b{\isacharparenright}{\isacharbackquote}{\isacharparenleft}possibleAllocationsRel\ {\isacharparenleft}N{\isacharminus}X{\isacharparenright}\ G{\isacharparenright}{\isacharparenright}{\isachardoublequoteclose}\ \isacommand{by}\isamarkupfalse%
\ auto\isanewline
\isacommand{then}\isamarkupfalse%
\ \isacommand{show}\isamarkupfalse%
\ {\isacharquery}thesis\ \isacommand{using}\isamarkupfalse%
\ {\isadigit{0}}\ \isacommand{by}\isamarkupfalse%
\ linarith\isanewline
\isacommand{qed}\isamarkupfalse%
%
\endisatagproof
{\isafoldproof}%
%
\isadelimproof
\isanewline
%
\endisadelimproof
\isanewline
\isacommand{lemma}\isamarkupfalse%
\ \isakeyword{assumes}\ {\isachardoublequoteopen}f\ {\isasymin}\ partitionValuedUniverse{\isachardoublequoteclose}\ \isakeyword{shows}\ {\isachardoublequoteopen}{\isacharbraceleft}{\isacharbraceright}\ {\isasymnotin}\ Range\ f{\isachardoublequoteclose}%
\isadelimproof
\ %
\endisadelimproof
%
\isatagproof
\isacommand{using}\isamarkupfalse%
\ assms\ \isacommand{by}\isamarkupfalse%
\ {\isacharparenleft}metis\ lm{\isadigit{2}}{\isadigit{2}}\ no{\isacharunderscore}empty{\isacharunderscore}eq{\isacharunderscore}class{\isacharparenright}%
\endisatagproof
{\isafoldproof}%
%
\isadelimproof
%
\endisadelimproof
\isanewline
\isanewline
\isacommand{lemma}\isamarkupfalse%
\ mm{\isadigit{3}}{\isadigit{3}}{\isacharcolon}\ \isakeyword{assumes}\ {\isachardoublequoteopen}finite\ XX{\isachardoublequoteclose}\ {\isachardoublequoteopen}{\isasymforall}X\ {\isasymin}\ XX{\isachardot}\ finite\ X{\isachardoublequoteclose}\ {\isachardoublequoteopen}is{\isacharunderscore}partition\ XX{\isachardoublequoteclose}\ \isakeyword{shows}\ \isanewline
{\isachardoublequoteopen}card\ {\isacharparenleft}{\isasymUnion}\ XX{\isacharparenright}\ {\isacharequal}\ setsum\ card\ XX{\isachardoublequoteclose}%
\isadelimproof
\ %
\endisadelimproof
%
\isatagproof
\isacommand{using}\isamarkupfalse%
\ assms\ is{\isacharunderscore}partition{\isacharunderscore}def\ card{\isacharunderscore}Union{\isacharunderscore}disjoint\ \isacommand{by}\isamarkupfalse%
\ fast%
\endisatagproof
{\isafoldproof}%
%
\isadelimproof
%
\endisadelimproof
\isanewline
\isanewline
\isacommand{corollary}\isamarkupfalse%
\ mm{\isadigit{3}}{\isadigit{3}}b{\isacharcolon}\ \isakeyword{assumes}\ {\isachardoublequoteopen}XX\ partitions\ X{\isachardoublequoteclose}\ {\isachardoublequoteopen}finite\ X{\isachardoublequoteclose}\ {\isachardoublequoteopen}finite\ XX{\isachardoublequoteclose}\ \isakeyword{shows}\ \isanewline
{\isachardoublequoteopen}card\ {\isacharparenleft}{\isasymUnion}\ XX{\isacharparenright}\ {\isacharequal}\ setsum\ card\ XX{\isachardoublequoteclose}%
\isadelimproof
\ %
\endisadelimproof
%
\isatagproof
\isacommand{using}\isamarkupfalse%
\ assms\ mm{\isadigit{3}}{\isadigit{3}}\ \isacommand{by}\isamarkupfalse%
\ {\isacharparenleft}metis\ is{\isacharunderscore}partition{\isacharunderscore}of{\isacharunderscore}def\ lll{\isadigit{4}}{\isadigit{1}}{\isacharparenright}%
\endisatagproof
{\isafoldproof}%
%
\isadelimproof
%
\endisadelimproof
\isanewline
\isanewline
\isacommand{lemma}\isamarkupfalse%
\ setsum{\isacharunderscore}Union{\isacharunderscore}disjoint{\isacharunderscore}{\isadigit{4}}{\isacharcolon}\ \isakeyword{assumes}\ {\isachardoublequoteopen}{\isasymforall}A{\isasymin}C{\isachardot}\ finite\ A{\isachardoublequoteclose}\ {\isachardoublequoteopen}{\isasymforall}A{\isasymin}C{\isachardot}\ {\isasymforall}B{\isasymin}C{\isachardot}\ A\ {\isasymnoteq}\ B\ {\isasymlongrightarrow}\ A\ Int\ B\ {\isacharequal}\ {\isacharbraceleft}{\isacharbraceright}{\isachardoublequoteclose}\ \isanewline
\isakeyword{shows}\ {\isachardoublequoteopen}setsum\ f\ {\isacharparenleft}Union\ C{\isacharparenright}\ {\isacharequal}\ setsum\ {\isacharparenleft}setsum\ f{\isacharparenright}\ C{\isachardoublequoteclose}%
\isadelimproof
\ %
\endisadelimproof
%
\isatagproof
\isacommand{using}\isamarkupfalse%
\ assms\ setsum{\isachardot}Union{\isacharunderscore}disjoint\ \isacommand{by}\isamarkupfalse%
\ fastforce%
\endisatagproof
{\isafoldproof}%
%
\isadelimproof
%
\endisadelimproof
\isanewline
\isanewline
\isanewline
\isacommand{corollary}\isamarkupfalse%
\ setsum{\isacharunderscore}Union{\isacharunderscore}disjoint{\isacharunderscore}{\isadigit{2}}{\isacharcolon}\ \isakeyword{assumes}\ {\isachardoublequoteopen}{\isasymforall}x{\isasymin}X{\isachardot}\ finite\ x{\isachardoublequoteclose}\ {\isachardoublequoteopen}is{\isacharunderscore}partition\ X{\isachardoublequoteclose}\ \isakeyword{shows}\ \isanewline
{\isachardoublequoteopen}setsum\ f\ {\isacharparenleft}{\isasymUnion}\ X{\isacharparenright}\ {\isacharequal}\ setsum\ {\isacharparenleft}setsum\ f{\isacharparenright}\ X{\isachardoublequoteclose}%
\isadelimproof
\ %
\endisadelimproof
%
\isatagproof
\isacommand{using}\isamarkupfalse%
\ assms\ setsum{\isacharunderscore}Union{\isacharunderscore}disjoint{\isacharunderscore}{\isadigit{4}}\ is{\isacharunderscore}partition{\isacharunderscore}def\ \isacommand{by}\isamarkupfalse%
\ fast%
\endisatagproof
{\isafoldproof}%
%
\isadelimproof
%
\endisadelimproof
\isanewline
\isanewline
\isacommand{corollary}\isamarkupfalse%
\ setsum{\isacharunderscore}Union{\isacharunderscore}disjoint{\isacharunderscore}{\isadigit{3}}{\isacharcolon}\ \isakeyword{assumes}\ {\isachardoublequoteopen}{\isasymforall}x{\isasymin}X{\isachardot}\ finite\ x{\isachardoublequoteclose}\ {\isachardoublequoteopen}X\ partitions\ XX{\isachardoublequoteclose}\ \isakeyword{shows}\isanewline
{\isachardoublequoteopen}setsum\ f\ XX\ {\isacharequal}\ setsum\ {\isacharparenleft}setsum\ f{\isacharparenright}\ X{\isachardoublequoteclose}%
\isadelimproof
\ %
\endisadelimproof
%
\isatagproof
\isacommand{using}\isamarkupfalse%
\ assms\ \isacommand{by}\isamarkupfalse%
\ {\isacharparenleft}metis\ is{\isacharunderscore}partition{\isacharunderscore}of{\isacharunderscore}def\ setsum{\isacharunderscore}Union{\isacharunderscore}disjoint{\isacharunderscore}{\isadigit{2}}{\isacharparenright}%
\endisatagproof
{\isafoldproof}%
%
\isadelimproof
%
\endisadelimproof
\isanewline
\isanewline
\isacommand{corollary}\isamarkupfalse%
\ setsum{\isacharunderscore}associativity{\isacharcolon}\ \isakeyword{assumes}\ {\isachardoublequoteopen}finite\ x{\isachardoublequoteclose}\ {\isachardoublequoteopen}X\ partitions\ x{\isachardoublequoteclose}\ \isakeyword{shows}\isanewline
{\isachardoublequoteopen}setsum\ f\ x\ {\isacharequal}\ setsum\ {\isacharparenleft}setsum\ f{\isacharparenright}\ X{\isachardoublequoteclose}%
\isadelimproof
\ %
\endisadelimproof
%
\isatagproof
\isacommand{using}\isamarkupfalse%
\ assms\ setsum{\isacharunderscore}Union{\isacharunderscore}disjoint{\isacharunderscore}{\isadigit{3}}\ \isacommand{by}\isamarkupfalse%
\ {\isacharparenleft}metis\ is{\isacharunderscore}partition{\isacharunderscore}of{\isacharunderscore}def\ lll{\isadigit{4}}{\isadigit{1}}{\isacharparenright}%
\endisatagproof
{\isafoldproof}%
%
\isadelimproof
%
\endisadelimproof
\isanewline
\isanewline
\isacommand{lemma}\isamarkupfalse%
\ lm{\isadigit{1}}{\isadigit{9}}e{\isacharcolon}\ \isakeyword{assumes}\ {\isachardoublequoteopen}a{\isasymin}allocationsUniverse{\isachardoublequoteclose}\ {\isachardoublequoteopen}Domain\ a{\isasymsubseteq}N{\isachardoublequoteclose}\ {\isachardoublequoteopen}{\isasymUnion}Range\ a{\isacharequal}G{\isachardoublequoteclose}\ \isakeyword{shows}\ \isanewline
{\isachardoublequoteopen}a\ {\isasymin}possibleAllocationsRel\ N\ G{\isachardoublequoteclose}%
\isadelimproof
\ %
\endisadelimproof
%
\isatagproof
\isacommand{using}\isamarkupfalse%
\ assms\ lm{\isadigit{1}}{\isadigit{9}}c\ lm{\isadigit{3}}{\isadigit{4}}\ \isacommand{by}\isamarkupfalse%
\ {\isacharparenleft}metis\ {\isacharparenleft}mono{\isacharunderscore}tags{\isacharcomma}\ lifting{\isacharparenright}{\isacharparenright}%
\endisatagproof
{\isafoldproof}%
%
\isadelimproof
%
\endisadelimproof
\isanewline
\isanewline
\isacommand{corollary}\isamarkupfalse%
\ nn{\isadigit{2}}{\isadigit{4}}a{\isacharcolon}\ {\isachardoublequoteopen}{\isacharparenleft}allocationsUniverse{\isasyminter}{\isacharbraceleft}a{\isachardot}\ Domain\ a{\isasymsubseteq}N\ {\isacharampersand}\ {\isasymUnion}Range\ a{\isacharequal}G{\isacharbraceright}{\isacharparenright}{\isasymsubseteq}possibleAllocationsRel\ N\ G{\isachardoublequoteclose}\isanewline
%
\isadelimproof
%
\endisadelimproof
%
\isatagproof
\isacommand{using}\isamarkupfalse%
\ lm{\isadigit{1}}{\isadigit{9}}e\ \isacommand{by}\isamarkupfalse%
\ fastforce%
\endisatagproof
{\isafoldproof}%
%
\isadelimproof
\isanewline
%
\endisadelimproof
\isacommand{corollary}\isamarkupfalse%
\ nn{\isadigit{2}}{\isadigit{4}}f{\isacharcolon}\ {\isachardoublequoteopen}possibleAllocationsRel\ N\ G\ {\isasymsubseteq}\ {\isacharbraceleft}a{\isachardot}\ Domain\ a{\isasymsubseteq}N{\isacharbraceright}{\isachardoublequoteclose}%
\isadelimproof
\ %
\endisadelimproof
%
\isatagproof
\isacommand{using}\isamarkupfalse%
\ lm{\isadigit{4}}{\isadigit{7}}\ \isacommand{by}\isamarkupfalse%
\ blast%
\endisatagproof
{\isafoldproof}%
%
\isadelimproof
%
\endisadelimproof
\isanewline
\isacommand{corollary}\isamarkupfalse%
\ nn{\isadigit{2}}{\isadigit{4}}g{\isacharcolon}\ {\isachardoublequoteopen}possibleAllocationsRel\ N\ G\ {\isasymsubseteq}\ {\isacharbraceleft}a{\isachardot}\ {\isasymUnion}Range\ a{\isacharequal}G{\isacharbraceright}{\isachardoublequoteclose}%
\isadelimproof
\ %
\endisadelimproof
%
\isatagproof
\isacommand{using}\isamarkupfalse%
\ is{\isacharunderscore}partition{\isacharunderscore}of{\isacharunderscore}def\ lm{\isadigit{4}}{\isadigit{7}}\ mem{\isacharunderscore}Collect{\isacharunderscore}eq\ subsetI\isanewline
\isacommand{by}\isamarkupfalse%
\ {\isacharparenleft}metis{\isacharparenleft}mono{\isacharunderscore}tags{\isacharparenright}{\isacharparenright}%
\endisatagproof
{\isafoldproof}%
%
\isadelimproof
%
\endisadelimproof
\isanewline
\isacommand{corollary}\isamarkupfalse%
\ \isakeyword{assumes}\ {\isachardoublequoteopen}a\ {\isasymin}\ possibleAllocationsRel\ N\ G{\isachardoublequoteclose}\ \isakeyword{shows}\ {\isachardoublequoteopen}{\isasymUnion}\ Range\ a\ {\isacharequal}\ G{\isachardoublequoteclose}%
\isadelimproof
\ %
\endisadelimproof
%
\isatagproof
\isacommand{using}\isamarkupfalse%
\ assms\ \isanewline
\isacommand{by}\isamarkupfalse%
\ {\isacharparenleft}metis\ is{\isacharunderscore}partition{\isacharunderscore}of{\isacharunderscore}def\ lm{\isadigit{4}}{\isadigit{7}}{\isacharparenright}%
\endisatagproof
{\isafoldproof}%
%
\isadelimproof
%
\endisadelimproof
\isanewline
\isacommand{corollary}\isamarkupfalse%
\ nn{\isadigit{2}}{\isadigit{4}}e{\isacharcolon}\ {\isachardoublequoteopen}\isanewline
possibleAllocationsRel\ N\ G\ {\isasymsubseteq}\ \ allocationsUniverse\ {\isacharampersand}\isanewline
possibleAllocationsRel\ N\ G\ {\isasymsubseteq}\ {\isacharbraceleft}a{\isachardot}\ Domain\ a{\isasymsubseteq}N\ {\isacharampersand}\ {\isasymUnion}Range\ a{\isacharequal}G{\isacharbraceright}{\isachardoublequoteclose}%
\isadelimproof
\ %
\endisadelimproof
%
\isatagproof
\isacommand{using}\isamarkupfalse%
\ nn{\isadigit{2}}{\isadigit{4}}f\ nn{\isadigit{2}}{\isadigit{4}}g\isanewline
conj{\isacharunderscore}subset{\isacharunderscore}def\ lm{\isadigit{5}}{\isadigit{0}}\ \isacommand{by}\isamarkupfalse%
\ {\isacharparenleft}metis\ {\isacharparenleft}no{\isacharunderscore}types{\isacharparenright}{\isacharparenright}%
\endisatagproof
{\isafoldproof}%
%
\isadelimproof
%
\endisadelimproof
\isanewline
\isanewline
\isacommand{corollary}\isamarkupfalse%
\ nn{\isadigit{2}}{\isadigit{4}}b{\isacharcolon}\ {\isachardoublequoteopen}possibleAllocationsRel\ N\ G\ {\isasymsubseteq}\ allocationsUniverse{\isasyminter}{\isacharbraceleft}a{\isachardot}\ Domain\ a{\isasymsubseteq}N\ {\isacharampersand}\ {\isasymUnion}Range\ a{\isacharequal}G{\isacharbraceright}{\isachardoublequoteclose}\isanewline
{\isacharparenleft}\isakeyword{is}\ {\isachardoublequoteopen}{\isacharquery}L\ {\isasymsubseteq}\ {\isacharquery}R{\isadigit{1}}\ {\isasyminter}\ {\isacharquery}R{\isadigit{2}}{\isachardoublequoteclose}{\isacharparenright}\isanewline
%
\isadelimproof
%
\endisadelimproof
%
\isatagproof
\isacommand{proof}\isamarkupfalse%
\ {\isacharminus}\ \isacommand{have}\isamarkupfalse%
\ {\isachardoublequoteopen}{\isacharquery}L\ {\isasymsubseteq}\ {\isacharquery}R{\isadigit{1}}\ {\isacharampersand}\ {\isacharquery}L\ {\isasymsubseteq}\ {\isacharquery}R{\isadigit{2}}{\isachardoublequoteclose}\ \isacommand{by}\isamarkupfalse%
\ {\isacharparenleft}rule\ nn{\isadigit{2}}{\isadigit{4}}e{\isacharparenright}\ \isacommand{thus}\isamarkupfalse%
\ {\isacharquery}thesis\ \isacommand{by}\isamarkupfalse%
\ auto\ \isacommand{qed}\isamarkupfalse%
%
\endisatagproof
{\isafoldproof}%
%
\isadelimproof
\isanewline
%
\endisadelimproof
\isanewline
\isacommand{corollary}\isamarkupfalse%
\ nn{\isadigit{2}}{\isadigit{4}}{\isacharcolon}\ {\isachardoublequoteopen}possibleAllocationsRel\ N\ G\ {\isacharequal}\ {\isacharparenleft}allocationsUniverse{\isasyminter}{\isacharbraceleft}a{\isachardot}\ Domain\ a{\isasymsubseteq}N\ {\isacharampersand}\ {\isasymUnion}Range\ a{\isacharequal}G{\isacharbraceright}{\isacharparenright}{\isachardoublequoteclose}\ \isanewline
{\isacharparenleft}\isakeyword{is}\ {\isachardoublequoteopen}{\isacharquery}L\ {\isacharequal}\ {\isacharquery}R{\isachardoublequoteclose}{\isacharparenright}\ \isanewline
%
\isadelimproof
%
\endisadelimproof
%
\isatagproof
\isacommand{proof}\isamarkupfalse%
\ {\isacharminus}\isanewline
\ \ \isacommand{have}\isamarkupfalse%
\ {\isachardoublequoteopen}{\isacharquery}L\ {\isasymsubseteq}\ {\isacharquery}R{\isachardoublequoteclose}\ \isacommand{using}\isamarkupfalse%
\ nn{\isadigit{2}}{\isadigit{4}}b\ \isacommand{by}\isamarkupfalse%
\ metis\ \isacommand{moreover}\isamarkupfalse%
\ \isacommand{have}\isamarkupfalse%
\ {\isachardoublequoteopen}{\isacharquery}R\ {\isasymsubseteq}\ {\isacharquery}L{\isachardoublequoteclose}\ \isacommand{using}\isamarkupfalse%
\ nn{\isadigit{2}}{\isadigit{4}}a\ \isacommand{by}\isamarkupfalse%
\ fast\isanewline
\ \ \isacommand{ultimately}\isamarkupfalse%
\ \isacommand{show}\isamarkupfalse%
\ {\isacharquery}thesis\ \isacommand{by}\isamarkupfalse%
\ force\isanewline
\isacommand{qed}\isamarkupfalse%
%
\endisatagproof
{\isafoldproof}%
%
\isadelimproof
\isanewline
%
\endisadelimproof
\isanewline
\isacommand{corollary}\isamarkupfalse%
\ nn{\isadigit{2}}{\isadigit{4}}c{\isacharcolon}\ {\isachardoublequoteopen}b\ {\isasymin}\ possibleAllocationsRel\ N\ G{\isacharequal}{\isacharparenleft}b{\isasymin}allocationsUniverse{\isacharampersand}\ Domain\ b{\isasymsubseteq}N\ {\isacharampersand}\ {\isasymUnion}Range\ b\ {\isacharequal}\ G{\isacharparenright}{\isachardoublequoteclose}\ \isanewline
%
\isadelimproof
%
\endisadelimproof
%
\isatagproof
\isacommand{using}\isamarkupfalse%
\ nn{\isadigit{2}}{\isadigit{4}}\ Int{\isacharunderscore}Collect\ \isacommand{by}\isamarkupfalse%
\ {\isacharparenleft}metis\ {\isacharparenleft}mono{\isacharunderscore}tags{\isacharcomma}\ lifting{\isacharparenright}{\isacharparenright}%
\endisatagproof
{\isafoldproof}%
%
\isadelimproof
\isanewline
%
\endisadelimproof
\isanewline
\isacommand{corollary}\isamarkupfalse%
\ lm{\isadigit{3}}{\isadigit{5}}d{\isacharcolon}\ \isakeyword{assumes}\ {\isachardoublequoteopen}a\ {\isasymin}\ allocationsUniverse{\isachardoublequoteclose}\ \isakeyword{shows}\ {\isachardoublequoteopen}a\ outside\ X\ {\isasymin}\ allocationsUniverse{\isachardoublequoteclose}%
\isadelimproof
\ %
\endisadelimproof
%
\isatagproof
\isacommand{using}\isamarkupfalse%
\ assms\ Outside{\isacharunderscore}def\isanewline
\isacommand{by}\isamarkupfalse%
\ {\isacharparenleft}metis\ {\isacharparenleft}lifting{\isacharcomma}\ mono{\isacharunderscore}tags{\isacharparenright}\ lm{\isadigit{3}}{\isadigit{5}}{\isacharparenright}%
\endisatagproof
{\isafoldproof}%
%
\isadelimproof
%
\endisadelimproof
\isanewline
%
\isadelimtheory
\isanewline
%
\endisadelimtheory
%
\isatagtheory
\isacommand{end}\isamarkupfalse%
%
\endisatagtheory
{\isafoldtheory}%
%
\isadelimtheory
%
\endisadelimtheory
\end{isabellebody}%
%%% Local Variables:
%%% mode: latex
%%% TeX-master: "root"
%%% End:


%
\begin{isabellebody}%
\def\isabellecontext{UniformTieBreaking}%
%
\isamarkupheader{Termination theorem for uniform tie-breaking%
}
\isamarkuptrue%
%
\isadelimtheory
%
\endisadelimtheory
%
\isatagtheory
\isacommand{theory}\isamarkupfalse%
\ UniformTieBreaking\isanewline
\isanewline
\isakeyword{imports}\ \isanewline
StrictCombinatorialAuction\isanewline
Universes\isanewline
{\isachardoublequoteopen}{\isachartilde}{\isachartilde}{\isacharslash}src{\isacharslash}HOL{\isacharslash}Library{\isacharslash}Code{\isacharunderscore}Target{\isacharunderscore}Nat{\isachardoublequoteclose}\isanewline
\isanewline
\isakeyword{begin}%
\endisatagtheory
{\isafoldtheory}%
%
\isadelimtheory
%
\endisadelimtheory
%
\isamarkupsection{Termination theorem for the uniform tie-breaking scheme \isa{{\isasymlambda}N\ G\ bids\ random{\isachardot}\ linearCompletion{\isacharprime}\ {\isacharparenleft}pseudoAllocation\ {\isacharparenleft}hd\ {\isacharparenleft}perm{\isadigit{2}}\ {\isacharparenleft}takeAll\ {\isacharparenleft}{\isasymlambda}x{\isachardot}\ winningAllocationRel\ N\ {\isacharparenleft}set\ G{\isacharparenright}\ {\isacharparenleft}op\ {\isasymin}\ x{\isacharparenright}\ bids{\isacharparenright}\ {\isacharparenleft}possibleAllocationsAlg{\isadigit{3}}\ N\ G{\isacharparenright}{\isacharparenright}\ random{\isacharparenright}{\isacharparenright}\ {\isacharless}{\isacharbar}\ {\isacharparenleft}N\ {\isasymtimes}\ finestpart\ {\isacharparenleft}set\ G{\isacharparenright}{\isacharparenright}{\isacharparenright}\ N\ {\isacharparenleft}set\ G{\isacharparenright}}%
}
\isamarkuptrue%
\isacommand{corollary}\isamarkupfalse%
\ lm{\isadigit{0}}{\isadigit{3}}{\isacharcolon}\ {\isachardoublequoteopen}winningAllocationsRel\ N\ G\ b\ {\isasymsubseteq}\ possibleAllocationsRel\ N\ G{\isachardoublequoteclose}\ \isanewline
%
\isadelimproof
%
\endisadelimproof
%
\isatagproof
\isacommand{using}\isamarkupfalse%
\ lm{\isadigit{0}}{\isadigit{2}}\ mem{\isacharunderscore}Collect{\isacharunderscore}eq\ subsetI\ \isacommand{by}\isamarkupfalse%
\ auto%
\endisatagproof
{\isafoldproof}%
%
\isadelimproof
\isanewline
%
\endisadelimproof
\isanewline
\isacommand{lemma}\isamarkupfalse%
\ lm{\isadigit{3}}{\isadigit{5}}b{\isacharcolon}\ \isakeyword{assumes}\ {\isachardoublequoteopen}a\ {\isasymin}\ allocationsUniverse{\isachardoublequoteclose}\ {\isachardoublequoteopen}c\ {\isasymsubseteq}\ a{\isachardoublequoteclose}\ \isakeyword{shows}\ {\isachardoublequoteopen}c\ {\isasymin}\ allocationsUniverse{\isachardoublequoteclose}\ \ \isanewline
%
\isadelimproof
%
\endisadelimproof
%
\isatagproof
\isacommand{proof}\isamarkupfalse%
\ {\isacharminus}\ \isacommand{have}\isamarkupfalse%
\ {\isachardoublequoteopen}c{\isacharequal}a{\isacharminus}{\isacharparenleft}a{\isacharminus}c{\isacharparenright}{\isachardoublequoteclose}\ \isacommand{using}\isamarkupfalse%
\ assms{\isacharparenleft}{\isadigit{2}}{\isacharparenright}\ \isacommand{by}\isamarkupfalse%
\ blast\ \isacommand{thus}\isamarkupfalse%
\ {\isacharquery}thesis\ \isacommand{using}\isamarkupfalse%
\ assms{\isacharparenleft}{\isadigit{1}}{\isacharparenright}\ lm{\isadigit{3}}{\isadigit{5}}\ \isacommand{by}\isamarkupfalse%
\ {\isacharparenleft}metis\ {\isacharparenleft}no{\isacharunderscore}types{\isacharparenright}{\isacharparenright}\ \isacommand{qed}\isamarkupfalse%
%
\endisatagproof
{\isafoldproof}%
%
\isadelimproof
\isanewline
%
\endisadelimproof
\isacommand{lemma}\isamarkupfalse%
\ lm{\isadigit{3}}{\isadigit{5}}c{\isacharcolon}\ \isakeyword{assumes}\ {\isachardoublequoteopen}a\ {\isasymin}\ allocationsUniverse{\isachardoublequoteclose}\ \isakeyword{shows}\ {\isachardoublequoteopen}a\ outside\ X\ {\isasymin}\ allocationsUniverse{\isachardoublequoteclose}\isanewline
%
\isadelimproof
%
\endisadelimproof
%
\isatagproof
\isacommand{using}\isamarkupfalse%
\ assms\ lm{\isadigit{3}}{\isadigit{5}}\ Outside{\isacharunderscore}def\ \isacommand{by}\isamarkupfalse%
\ {\isacharparenleft}metis\ {\isacharparenleft}no{\isacharunderscore}types{\isacharparenright}{\isacharparenright}%
\endisatagproof
{\isafoldproof}%
%
\isadelimproof
\isanewline
%
\endisadelimproof
\isanewline
\isacommand{corollary}\isamarkupfalse%
\ lm{\isadigit{3}}{\isadigit{8}}d{\isacharcolon}\ {\isachardoublequoteopen}{\isacharbraceleft}x{\isacharbraceright}{\isasymtimes}{\isacharparenleft}{\isacharbraceleft}X{\isacharbraceright}{\isacharminus}{\isacharbraceleft}{\isacharbraceleft}{\isacharbraceright}{\isacharbraceright}{\isacharparenright}\ {\isasymin}\ allocationsUniverse{\isachardoublequoteclose}%
\isadelimproof
\ %
\endisadelimproof
%
\isatagproof
\isacommand{using}\isamarkupfalse%
\ lm{\isadigit{3}}{\isadigit{8}}\ nn{\isadigit{4}}{\isadigit{3}}\ \isacommand{by}\isamarkupfalse%
\ metis%
\endisatagproof
{\isafoldproof}%
%
\isadelimproof
%
\endisadelimproof
\isanewline
\isacommand{corollary}\isamarkupfalse%
\ lm{\isadigit{3}}{\isadigit{8}}b{\isacharcolon}\ {\isachardoublequoteopen}{\isacharbraceleft}{\isacharparenleft}x{\isacharcomma}{\isacharbraceleft}y{\isacharbraceright}{\isacharparenright}{\isacharbraceright}\ {\isasymin}\ allocationsUniverse{\isachardoublequoteclose}%
\isadelimproof
\ %
\endisadelimproof
%
\isatagproof
\isacommand{using}\isamarkupfalse%
\ lm{\isadigit{3}}{\isadigit{8}}\ lm{\isadigit{4}}{\isadigit{4}}\ insert{\isacharunderscore}not{\isacharunderscore}empty\ \isanewline
\isacommand{proof}\isamarkupfalse%
\ {\isacharminus}\isanewline
\ \ \isacommand{have}\isamarkupfalse%
\ {\isachardoublequoteopen}{\isacharparenleft}x{\isacharcomma}\ {\isacharbraceleft}y{\isacharbraceright}{\isacharparenright}\ {\isasymnoteq}\ {\isacharparenleft}x{\isacharcomma}\ {\isacharbraceleft}{\isacharbraceright}{\isacharparenright}{\isachardoublequoteclose}\ \isacommand{by}\isamarkupfalse%
\ blast\isanewline
\ \ \isacommand{thus}\isamarkupfalse%
\ {\isachardoublequoteopen}{\isacharbraceleft}{\isacharparenleft}x{\isacharcomma}\ {\isacharbraceleft}y{\isacharbraceright}{\isacharparenright}{\isacharbraceright}\ {\isasymin}\ allocationsUniverse{\isachardoublequoteclose}\ \isacommand{by}\isamarkupfalse%
\ {\isacharparenleft}metis\ {\isacharparenleft}no{\isacharunderscore}types{\isacharparenright}\ insert{\isacharunderscore}Diff{\isacharunderscore}if\ insert{\isacharunderscore}iff\ lm{\isadigit{3}}{\isadigit{8}}\ lm{\isadigit{4}}{\isadigit{4}}{\isacharparenright}\isanewline
\isacommand{qed}\isamarkupfalse%
%
\endisatagproof
{\isafoldproof}%
%
\isadelimproof
%
\endisadelimproof
\isanewline
\isacommand{corollary}\isamarkupfalse%
\ lm{\isadigit{3}}{\isadigit{8}}c{\isacharcolon}\ {\isachardoublequoteopen}allocationsUniverse\ {\isasymnoteq}\ {\isacharbraceleft}{\isacharbraceright}{\isachardoublequoteclose}%
\isadelimproof
\ %
\endisadelimproof
%
\isatagproof
\isacommand{using}\isamarkupfalse%
\ lm{\isadigit{3}}{\isadigit{8}}b\ \isacommand{by}\isamarkupfalse%
\ fast%
\endisatagproof
{\isafoldproof}%
%
\isadelimproof
%
\endisadelimproof
\isanewline
\isacommand{corollary}\isamarkupfalse%
\ nn{\isadigit{3}}{\isadigit{9}}{\isacharcolon}\ {\isachardoublequoteopen}{\isacharbraceleft}{\isacharbraceright}\ {\isasymin}\ allocationsUniverse{\isachardoublequoteclose}%
\isadelimproof
\ %
\endisadelimproof
%
\isatagproof
\isacommand{using}\isamarkupfalse%
\ lm{\isadigit{3}}{\isadigit{5}}b\ lm{\isadigit{3}}{\isadigit{8}}b\ \isacommand{by}\isamarkupfalse%
\ {\isacharparenleft}metis\ {\isacharparenleft}lifting{\isacharcomma}\ mono{\isacharunderscore}tags{\isacharparenright}\ empty{\isacharunderscore}subsetI{\isacharparenright}%
\endisatagproof
{\isafoldproof}%
%
\isadelimproof
%
\endisadelimproof
\isanewline
\isacommand{lemma}\isamarkupfalse%
\ mm{\isadigit{8}}{\isadigit{7}}{\isacharcolon}\ \isakeyword{assumes}\ {\isachardoublequoteopen}G\ {\isasymnoteq}\ {\isacharbraceleft}{\isacharbraceright}{\isachardoublequoteclose}\ \isakeyword{shows}\ {\isachardoublequoteopen}{\isacharbraceleft}G{\isacharbraceright}\ {\isasymin}\ all{\isacharunderscore}partitions\ G{\isachardoublequoteclose}%
\isadelimproof
\ %
\endisadelimproof
%
\isatagproof
\isacommand{using}\isamarkupfalse%
\ all{\isacharunderscore}partitions{\isacharunderscore}def\ is{\isacharunderscore}partition{\isacharunderscore}of{\isacharunderscore}def\ \isanewline
is{\isacharunderscore}partition{\isacharunderscore}def\ assms\ \isacommand{by}\isamarkupfalse%
\ force%
\endisatagproof
{\isafoldproof}%
%
\isadelimproof
%
\endisadelimproof
\isanewline
\isacommand{lemma}\isamarkupfalse%
\ mm{\isadigit{8}}{\isadigit{8}}{\isacharcolon}\ \isakeyword{assumes}\ {\isachardoublequoteopen}n\ {\isasymin}\ N{\isachardoublequoteclose}\ \isakeyword{shows}\ {\isachardoublequoteopen}{\isacharbraceleft}{\isacharparenleft}G{\isacharcomma}n{\isacharparenright}{\isacharbraceright}\ {\isasymin}\ totalRels\ {\isacharbraceleft}G{\isacharbraceright}\ N{\isachardoublequoteclose}%
\isadelimproof
\ %
\endisadelimproof
%
\isatagproof
\isacommand{using}\isamarkupfalse%
\ assms\ \isacommand{by}\isamarkupfalse%
\ force%
\endisatagproof
{\isafoldproof}%
%
\isadelimproof
%
\endisadelimproof
\isanewline
\isacommand{lemma}\isamarkupfalse%
\ mm{\isadigit{8}}{\isadigit{9}}{\isacharcolon}\ \isakeyword{assumes}\ {\isachardoublequoteopen}n{\isasymin}N{\isachardoublequoteclose}\ \isakeyword{shows}\ {\isachardoublequoteopen}{\isacharbraceleft}{\isacharparenleft}G{\isacharcomma}n{\isacharparenright}{\isacharbraceright}\ {\isasymin}\ injections\ {\isacharbraceleft}G{\isacharbraceright}\ N{\isachardoublequoteclose}\ \isanewline
%
\isadelimproof
%
\endisadelimproof
%
\isatagproof
\isacommand{using}\isamarkupfalse%
\ assms\ possible{\isacharunderscore}allocations{\isacharunderscore}rel{\isacharunderscore}def\ injections{\isacharunderscore}def\ mm{\isadigit{8}}{\isadigit{7}}\ all{\isacharunderscore}partitions{\isacharunderscore}def\ \isanewline
is{\isacharunderscore}partition{\isacharunderscore}def\ is{\isacharunderscore}partition{\isacharunderscore}of{\isacharunderscore}def\ lm{\isadigit{2}}{\isadigit{6}}\ mm{\isadigit{8}}{\isadigit{8}}\ lm{\isadigit{3}}{\isadigit{7}}\ lm{\isadigit{2}}{\isadigit{4}}\ \isacommand{by}\isamarkupfalse%
\ fastforce%
\endisatagproof
{\isafoldproof}%
%
\isadelimproof
\isanewline
%
\endisadelimproof
\isacommand{corollary}\isamarkupfalse%
\ mm{\isadigit{9}}{\isadigit{0}}{\isacharcolon}\ \isakeyword{assumes}\ {\isachardoublequoteopen}\ G{\isasymnoteq}{\isacharbraceleft}{\isacharbraceright}{\isachardoublequoteclose}\ {\isachardoublequoteopen}n{\isasymin}N{\isachardoublequoteclose}\ \isakeyword{shows}\ {\isachardoublequoteopen}{\isacharbraceleft}{\isacharparenleft}G{\isacharcomma}n{\isacharparenright}{\isacharbraceright}\ {\isasymin}\ possible{\isacharunderscore}allocations{\isacharunderscore}rel\ G\ N{\isachardoublequoteclose}\isanewline
%
\isadelimproof
%
\endisadelimproof
%
\isatagproof
\isacommand{proof}\isamarkupfalse%
\ {\isacharminus}\isanewline
\ \ \isacommand{have}\isamarkupfalse%
\ {\isachardoublequoteopen}{\isacharbraceleft}{\isacharparenleft}G{\isacharcomma}n{\isacharparenright}{\isacharbraceright}\ {\isasymin}\ injections\ {\isacharbraceleft}G{\isacharbraceright}\ N{\isachardoublequoteclose}\ \isacommand{using}\isamarkupfalse%
\ assms\ mm{\isadigit{8}}{\isadigit{9}}\ \isacommand{by}\isamarkupfalse%
\ fast\isanewline
\ \ \isacommand{moreover}\isamarkupfalse%
\ \isacommand{have}\isamarkupfalse%
\ {\isachardoublequoteopen}{\isacharbraceleft}G{\isacharbraceright}\ {\isasymin}\ all{\isacharunderscore}partitions\ G{\isachardoublequoteclose}\ \isacommand{using}\isamarkupfalse%
\ assms\ mm{\isadigit{8}}{\isadigit{7}}\ \isacommand{by}\isamarkupfalse%
\ metis\isanewline
\ \ \isacommand{ultimately}\isamarkupfalse%
\ \isacommand{show}\isamarkupfalse%
\ {\isacharquery}thesis\ \isacommand{using}\isamarkupfalse%
\ possible{\isacharunderscore}allocations{\isacharunderscore}rel{\isacharunderscore}def\ \isacommand{by}\isamarkupfalse%
\ auto\isanewline
\isacommand{qed}\isamarkupfalse%
%
\endisatagproof
{\isafoldproof}%
%
\isadelimproof
\isanewline
%
\endisadelimproof
\isacommand{corollary}\isamarkupfalse%
\ mm{\isadigit{9}}{\isadigit{0}}b{\isacharcolon}\ \isakeyword{assumes}\ {\isachardoublequoteopen}N\ {\isasymnoteq}\ {\isacharbraceleft}{\isacharbraceright}{\isachardoublequoteclose}\ {\isachardoublequoteopen}G{\isasymnoteq}{\isacharbraceleft}{\isacharbraceright}{\isachardoublequoteclose}\ \isakeyword{shows}\ {\isachardoublequoteopen}possibleAllocationsRel\ N\ G\ {\isasymnoteq}\ {\isacharbraceleft}{\isacharbraceright}{\isachardoublequoteclose}\isanewline
%
\isadelimproof
%
\endisadelimproof
%
\isatagproof
\isacommand{using}\isamarkupfalse%
\ assms\ mm{\isadigit{9}}{\isadigit{0}}\ \isacommand{by}\isamarkupfalse%
\ {\isacharparenleft}metis\ {\isacharparenleft}hide{\isacharunderscore}lams{\isacharcomma}\ no{\isacharunderscore}types{\isacharparenright}\ equals{\isadigit{0}}I\ image{\isacharunderscore}insert\ insert{\isacharunderscore}absorb\ insert{\isacharunderscore}not{\isacharunderscore}empty{\isacharparenright}%
\endisatagproof
{\isafoldproof}%
%
\isadelimproof
\isanewline
%
\endisadelimproof
\isacommand{corollary}\isamarkupfalse%
\ mm{\isadigit{9}}{\isadigit{1}}{\isacharcolon}\ \isakeyword{assumes}\ {\isachardoublequoteopen}N\ {\isasymnoteq}\ {\isacharbraceleft}{\isacharbraceright}{\isachardoublequoteclose}\ {\isachardoublequoteopen}finite\ N{\isachardoublequoteclose}\ {\isachardoublequoteopen}G\ {\isasymnoteq}\ {\isacharbraceleft}{\isacharbraceright}{\isachardoublequoteclose}\ {\isachardoublequoteopen}finite\ G{\isachardoublequoteclose}\ \isakeyword{shows}\ \isanewline
{\isachardoublequoteopen}winningAllocationsRel\ N\ G\ bids\ {\isasymnoteq}\ {\isacharbraceleft}{\isacharbraceright}\ {\isacharampersand}\ finite\ {\isacharparenleft}winningAllocationsRel\ N\ G\ bids{\isacharparenright}{\isachardoublequoteclose}\ \isanewline
%
\isadelimproof
%
\endisadelimproof
%
\isatagproof
\isacommand{using}\isamarkupfalse%
\ assms\ mm{\isadigit{9}}{\isadigit{0}}b\ lm{\isadigit{5}}{\isadigit{9}}\ argmax{\isacharunderscore}non{\isacharunderscore}empty{\isacharunderscore}iff\ \isacommand{by}\isamarkupfalse%
\ {\isacharparenleft}metis\ lm{\isadigit{0}}{\isadigit{3}}\ rev{\isacharunderscore}finite{\isacharunderscore}subset{\isacharparenright}%
\endisatagproof
{\isafoldproof}%
%
\isadelimproof
\isanewline
%
\endisadelimproof
\isanewline
\isacommand{lemma}\isamarkupfalse%
\ mm{\isadigit{5}}{\isadigit{2}}{\isacharcolon}\ {\isachardoublequoteopen}possibleAllocationsRel\ N\ {\isacharbraceleft}{\isacharbraceright}\ {\isasymsubseteq}\ {\isacharbraceleft}{\isacharbraceleft}{\isacharbraceright}{\isacharbraceright}{\isachardoublequoteclose}%
\isadelimproof
\ %
\endisadelimproof
%
\isatagproof
\isacommand{using}\isamarkupfalse%
\ emptyset{\isacharunderscore}part{\isacharunderscore}emptyset{\isadigit{3}}\ mm{\isadigit{5}}{\isadigit{1}}\ \isanewline
lm{\isadigit{2}}{\isadigit{8}}b\ mem{\isacharunderscore}Collect{\isacharunderscore}eq\ subsetI\ vimage{\isacharunderscore}def\ \isacommand{by}\isamarkupfalse%
\ metis%
\endisatagproof
{\isafoldproof}%
%
\isadelimproof
%
\endisadelimproof
\isanewline
\isanewline
\isacommand{lemma}\isamarkupfalse%
\ mm{\isadigit{4}}{\isadigit{2}}{\isacharcolon}\ \isakeyword{assumes}\ {\isachardoublequoteopen}a\ {\isasymin}\ possibleAllocationsRel\ N\ G{\isachardoublequoteclose}\ {\isachardoublequoteopen}finite\ G{\isachardoublequoteclose}\ \isakeyword{shows}\ {\isachardoublequoteopen}finite\ {\isacharparenleft}Range\ a{\isacharparenright}{\isachardoublequoteclose}\ \isanewline
%
\isadelimproof
%
\endisadelimproof
%
\isatagproof
\isacommand{using}\isamarkupfalse%
\ assms\ lm{\isadigit{5}}{\isadigit{5}}\ \isacommand{by}\isamarkupfalse%
\ {\isacharparenleft}metis\ lm{\isadigit{2}}{\isadigit{8}}{\isacharparenright}%
\endisatagproof
{\isafoldproof}%
%
\isadelimproof
\isanewline
%
\endisadelimproof
\isanewline
\isacommand{corollary}\isamarkupfalse%
\ mm{\isadigit{4}}{\isadigit{4}}{\isacharcolon}\ \isakeyword{assumes}\ {\isachardoublequoteopen}a\ {\isasymin}\ possibleAllocationsRel\ N\ G{\isachardoublequoteclose}\ {\isachardoublequoteopen}finite\ G{\isachardoublequoteclose}\ \isakeyword{shows}\ {\isachardoublequoteopen}finite\ a{\isachardoublequoteclose}\isanewline
%
\isadelimproof
%
\endisadelimproof
%
\isatagproof
\isacommand{using}\isamarkupfalse%
\ assms\ mm{\isadigit{4}}{\isadigit{2}}\ mm{\isadigit{4}}{\isadigit{3}}\ finite{\isacharunderscore}converse\ \isanewline
\isacommand{by}\isamarkupfalse%
\ {\isacharparenleft}metis\ {\isacharparenleft}erased{\isacharcomma}\ hide{\isacharunderscore}lams{\isacharparenright}\ Range{\isacharunderscore}converse\ imageE\ lll{\isadigit{8}}{\isadigit{1}}{\isacharparenright}%
\endisatagproof
{\isafoldproof}%
%
\isadelimproof
\isanewline
%
\endisadelimproof
\isanewline
\isacommand{lemma}\isamarkupfalse%
\ \isakeyword{assumes}\ {\isachardoublequoteopen}a\ {\isasymin}\ possibleAllocationsRel\ N\ G{\isachardoublequoteclose}\ \isakeyword{shows}\ {\isachardoublequoteopen}{\isasymUnion}\ Range\ a\ {\isacharequal}\ G{\isachardoublequoteclose}%
\isadelimproof
\ %
\endisadelimproof
%
\isatagproof
\isacommand{using}\isamarkupfalse%
\ assms\ \isanewline
\isacommand{by}\isamarkupfalse%
\ {\isacharparenleft}metis\ is{\isacharunderscore}partition{\isacharunderscore}of{\isacharunderscore}def\ lm{\isadigit{4}}{\isadigit{7}}{\isacharparenright}%
\endisatagproof
{\isafoldproof}%
%
\isadelimproof
%
\endisadelimproof
\isanewline
\isanewline
\isacommand{lemma}\isamarkupfalse%
\ mm{\isadigit{4}}{\isadigit{1}}{\isacharcolon}\ \isakeyword{assumes}\ {\isachardoublequoteopen}a\ {\isasymin}\ possibleAllocationsRel\ N\ G{\isachardoublequoteclose}\ {\isachardoublequoteopen}finite\ G{\isachardoublequoteclose}\ \isakeyword{shows}\isanewline
{\isachardoublequoteopen}{\isasymforall}\ y\ {\isasymin}\ Range\ a{\isachardot}\ finite\ y{\isachardoublequoteclose}%
\isadelimproof
\ %
\endisadelimproof
%
\isatagproof
\isacommand{using}\isamarkupfalse%
\ assms\ is{\isacharunderscore}partition{\isacharunderscore}of{\isacharunderscore}def\ lm{\isadigit{4}}{\isadigit{7}}\ \isacommand{by}\isamarkupfalse%
\ {\isacharparenleft}metis\ Union{\isacharunderscore}upper\ rev{\isacharunderscore}finite{\isacharunderscore}subset{\isacharparenright}%
\endisatagproof
{\isafoldproof}%
%
\isadelimproof
%
\endisadelimproof
\isanewline
\isanewline
\isacommand{corollary}\isamarkupfalse%
\ mm{\isadigit{3}}{\isadigit{3}}c{\isacharcolon}\ \isakeyword{assumes}\ {\isachardoublequoteopen}a\ {\isasymin}\ possibleAllocationsRel\ N\ G{\isachardoublequoteclose}\ {\isachardoublequoteopen}finite\ G{\isachardoublequoteclose}\ \isakeyword{shows}\ \isanewline
{\isachardoublequoteopen}card\ G\ {\isacharequal}\ setsum\ card\ {\isacharparenleft}Range\ a{\isacharparenright}{\isachardoublequoteclose}%
\isadelimproof
\ %
\endisadelimproof
%
\isatagproof
\isacommand{using}\isamarkupfalse%
\ assms\ mm{\isadigit{3}}{\isadigit{3}}b\ mm{\isadigit{4}}{\isadigit{2}}\ lm{\isadigit{4}}{\isadigit{7}}\ \isacommand{by}\isamarkupfalse%
\ {\isacharparenleft}metis\ is{\isacharunderscore}partition{\isacharunderscore}of{\isacharunderscore}def{\isacharparenright}%
\endisatagproof
{\isafoldproof}%
%
\isadelimproof
%
\endisadelimproof
\isanewline
\isanewline
\isanewline
\isanewline
\isanewline
\isanewline
\isanewline
\isanewline
\isanewline
\isanewline
\isanewline
\isanewline
\isanewline
\isanewline
\isanewline
\isanewline
\isanewline
\isanewline
\isanewline
\isanewline
\isanewline
\isanewline
\isanewline
\isanewline
\isanewline
\isacommand{lemma}\isamarkupfalse%
\ mm{\isadigit{6}}{\isadigit{6}}{\isacharcolon}\ {\isachardoublequoteopen}LinearCompletion\ bids\ N\ G\ {\isacharequal}\ \isanewline
{\isacharbraceleft}{\isacharparenleft}pair{\isacharcomma}setsum\ {\isacharparenleft}{\isacharpercent}g{\isachardot}\ bids\ {\isacharparenleft}fst\ pair{\isacharcomma}\ g{\isacharparenright}{\isacharparenright}\ {\isacharparenleft}finestpart\ {\isacharparenleft}snd\ pair{\isacharparenright}{\isacharparenright}{\isacharparenright}{\isacharbar}pair{\isachardot}\ pair\ {\isasymin}\ N\ {\isasymtimes}\ {\isacharparenleft}Pow\ G{\isacharminus}{\isacharbraceleft}{\isacharbraceleft}{\isacharbraceright}{\isacharbraceright}{\isacharparenright}{\isacharbraceright}{\isachardoublequoteclose}%
\isadelimproof
\ %
\endisadelimproof
%
\isatagproof
\isacommand{by}\isamarkupfalse%
\ blast%
\endisatagproof
{\isafoldproof}%
%
\isadelimproof
%
\endisadelimproof
\isanewline
\isacommand{corollary}\isamarkupfalse%
\ mm{\isadigit{6}}{\isadigit{5}}b{\isacharcolon}\ \isanewline
{\isachardoublequoteopen}{\isacharbraceleft}{\isacharparenleft}pair{\isacharcomma}setsum\ {\isacharparenleft}{\isacharpercent}g{\isachardot}\ bids\ {\isacharparenleft}fst\ pair{\isacharcomma}\ g{\isacharparenright}{\isacharparenright}\ {\isacharparenleft}finestpart\ {\isacharparenleft}snd\ pair{\isacharparenright}{\isacharparenright}{\isacharparenright}{\isacharbar}pair{\isachardot}\ pair\ {\isasymin}\ N\ {\isasymtimes}\ {\isacharparenleft}Pow\ G{\isacharminus}{\isacharbraceleft}{\isacharbraceleft}{\isacharbraceright}{\isacharbraceright}{\isacharparenright}{\isacharbraceright}\ {\isacharbar}{\isacharbar}\ a\ {\isacharequal}\ \isanewline
{\isacharbraceleft}{\isacharparenleft}pair{\isacharcomma}setsum\ {\isacharparenleft}{\isacharpercent}g{\isachardot}\ bids\ {\isacharparenleft}fst\ pair{\isacharcomma}\ g{\isacharparenright}{\isacharparenright}\ {\isacharparenleft}finestpart\ {\isacharparenleft}snd\ pair{\isacharparenright}{\isacharparenright}{\isacharparenright}{\isacharbar}pair{\isachardot}\ pair\ {\isasymin}\ {\isacharparenleft}N\ {\isasymtimes}\ {\isacharparenleft}Pow\ G\ {\isacharminus}\ {\isacharbraceleft}{\isacharbraceleft}{\isacharbraceright}{\isacharbraceright}{\isacharparenright}{\isacharparenright}\ {\isasyminter}\ a{\isacharbraceright}{\isachardoublequoteclose}\isanewline
%
\isadelimproof
%
\endisadelimproof
%
\isatagproof
\isacommand{by}\isamarkupfalse%
\ {\isacharparenleft}metis\ mm{\isadigit{6}}{\isadigit{5}}{\isacharparenright}%
\endisatagproof
{\isafoldproof}%
%
\isadelimproof
\isanewline
%
\endisadelimproof
\isacommand{corollary}\isamarkupfalse%
\ mm{\isadigit{6}}{\isadigit{6}}b{\isacharcolon}\ {\isachardoublequoteopen}{\isacharparenleft}LinearCompletion\ bids\ N\ G{\isacharparenright}\ {\isacharbar}{\isacharbar}\ a\ {\isacharequal}\ \isanewline
{\isacharbraceleft}{\isacharparenleft}pair{\isacharcomma}setsum\ {\isacharparenleft}{\isacharpercent}g{\isachardot}\ bids\ {\isacharparenleft}fst\ pair{\isacharcomma}\ g{\isacharparenright}{\isacharparenright}\ {\isacharparenleft}finestpart\ {\isacharparenleft}snd\ pair{\isacharparenright}{\isacharparenright}{\isacharparenright}{\isacharbar}pair{\isachardot}\ pair\ {\isasymin}\ {\isacharparenleft}N\ {\isasymtimes}\ {\isacharparenleft}Pow\ G\ {\isacharminus}\ {\isacharbraceleft}{\isacharbraceleft}{\isacharbraceright}{\isacharbraceright}{\isacharparenright}{\isacharparenright}\ {\isasyminter}\ a{\isacharbraceright}{\isachardoublequoteclose}\isanewline
{\isacharparenleft}\isakeyword{is}\ {\isachardoublequoteopen}{\isacharquery}L{\isacharequal}{\isacharquery}R{\isachardoublequoteclose}{\isacharparenright}%
\isadelimproof
\ %
\endisadelimproof
%
\isatagproof
\isacommand{using}\isamarkupfalse%
\ mm{\isadigit{6}}{\isadigit{5}}b\ mm{\isadigit{6}}{\isadigit{6}}\ \isanewline
\isacommand{proof}\isamarkupfalse%
\ {\isacharminus}\isanewline
\isacommand{let}\isamarkupfalse%
\ {\isacharquery}l{\isacharequal}LinearCompletion\isanewline
\isacommand{let}\isamarkupfalse%
\ {\isacharquery}M{\isacharequal}{\isachardoublequoteopen}{\isacharbraceleft}{\isacharparenleft}pair{\isacharcomma}setsum\ {\isacharparenleft}{\isacharpercent}g{\isachardot}\ bids\ {\isacharparenleft}fst\ pair{\isacharcomma}\ g{\isacharparenright}{\isacharparenright}\ {\isacharparenleft}finestpart\ {\isacharparenleft}snd\ pair{\isacharparenright}{\isacharparenright}{\isacharparenright}{\isacharbar}pair{\isachardot}\ pair\ {\isasymin}\ N\ {\isasymtimes}\ {\isacharparenleft}Pow\ G{\isacharminus}{\isacharbraceleft}{\isacharbraceleft}{\isacharbraceright}{\isacharbraceright}{\isacharparenright}{\isacharbraceright}{\isachardoublequoteclose}\isanewline
\isacommand{have}\isamarkupfalse%
\ {\isachardoublequoteopen}{\isacharquery}l\ bids\ N\ G\ {\isacharequal}\ {\isacharquery}M{\isachardoublequoteclose}\ \isacommand{by}\isamarkupfalse%
\ {\isacharparenleft}rule\ mm{\isadigit{6}}{\isadigit{6}}{\isacharparenright}\isanewline
\isacommand{then}\isamarkupfalse%
\ \isacommand{have}\isamarkupfalse%
\ {\isachardoublequoteopen}{\isacharquery}L\ {\isacharequal}\ {\isacharparenleft}{\isacharquery}M\ {\isacharbar}{\isacharbar}\ a{\isacharparenright}{\isachardoublequoteclose}\ \isacommand{by}\isamarkupfalse%
\ presburger\isanewline
\isacommand{moreover}\isamarkupfalse%
\ \isacommand{have}\isamarkupfalse%
\ {\isachardoublequoteopen}{\isachardot}{\isachardot}{\isachardot}\ {\isacharequal}\ {\isacharquery}R{\isachardoublequoteclose}\ \isacommand{by}\isamarkupfalse%
\ {\isacharparenleft}rule\ mm{\isadigit{6}}{\isadigit{5}}b{\isacharparenright}\isanewline
\isacommand{ultimately}\isamarkupfalse%
\ \isacommand{show}\isamarkupfalse%
\ {\isacharquery}thesis\ \isacommand{by}\isamarkupfalse%
\ presburger\isanewline
\isacommand{qed}\isamarkupfalse%
%
\endisatagproof
{\isafoldproof}%
%
\isadelimproof
%
\endisadelimproof
\isanewline
\isacommand{lemma}\isamarkupfalse%
\ mm{\isadigit{6}}{\isadigit{6}}c{\isacharcolon}\ {\isachardoublequoteopen}{\isacharparenleft}partialCompletionOf\ bids{\isacharparenright}\ {\isacharbackquote}\ {\isacharparenleft}{\isacharparenleft}N\ {\isasymtimes}\ {\isacharparenleft}Pow\ G\ {\isacharminus}\ {\isacharbraceleft}{\isacharbraceleft}{\isacharbraceright}{\isacharbraceright}{\isacharparenright}{\isacharparenright}\ {\isasyminter}\ a{\isacharparenright}\ {\isacharequal}\ \isanewline
{\isacharbraceleft}{\isacharparenleft}pair{\isacharcomma}setsum\ {\isacharparenleft}{\isacharpercent}g{\isachardot}\ bids\ {\isacharparenleft}fst\ pair{\isacharcomma}\ g{\isacharparenright}{\isacharparenright}\ {\isacharparenleft}finestpart\ {\isacharparenleft}snd\ pair{\isacharparenright}{\isacharparenright}{\isacharparenright}{\isacharbar}pair{\isachardot}\ pair\ {\isasymin}\ {\isacharparenleft}N\ {\isasymtimes}\ {\isacharparenleft}Pow\ G{\isacharminus}{\isacharbraceleft}{\isacharbraceleft}{\isacharbraceright}{\isacharbraceright}{\isacharparenright}{\isacharparenright}\ {\isasyminter}\ a{\isacharbraceright}{\isachardoublequoteclose}\isanewline
%
\isadelimproof
%
\endisadelimproof
%
\isatagproof
\isacommand{by}\isamarkupfalse%
\ blast%
\endisatagproof
{\isafoldproof}%
%
\isadelimproof
\isanewline
%
\endisadelimproof
\isacommand{corollary}\isamarkupfalse%
\ mm{\isadigit{6}}{\isadigit{6}}d{\isacharcolon}\ {\isachardoublequoteopen}{\isacharparenleft}LinearCompletion\ bids\ N\ G{\isacharparenright}\ {\isacharbar}{\isacharbar}\ a\ {\isacharequal}\ {\isacharparenleft}partialCompletionOf\ bids{\isacharparenright}\ {\isacharbackquote}\ {\isacharparenleft}{\isacharparenleft}N\ {\isasymtimes}\ {\isacharparenleft}Pow\ G\ {\isacharminus}\ {\isacharbraceleft}{\isacharbraceleft}{\isacharbraceright}{\isacharbraceright}{\isacharparenright}{\isacharparenright}\ {\isasyminter}\ a{\isacharparenright}{\isachardoublequoteclose}\isanewline
{\isacharparenleft}\isakeyword{is}\ {\isachardoublequoteopen}{\isacharquery}L{\isacharequal}{\isacharquery}R{\isachardoublequoteclose}{\isacharparenright}\isanewline
%
\isadelimproof
%
\endisadelimproof
%
\isatagproof
\isacommand{using}\isamarkupfalse%
\ mm{\isadigit{6}}{\isadigit{6}}c\ mm{\isadigit{6}}{\isadigit{6}}b\ \isanewline
\isacommand{proof}\isamarkupfalse%
\ {\isacharminus}\isanewline
\isacommand{let}\isamarkupfalse%
\ {\isacharquery}l{\isacharequal}LinearCompletion\ \isacommand{let}\isamarkupfalse%
\ {\isacharquery}p{\isacharequal}partialCompletionOf\ \isacommand{let}\isamarkupfalse%
\ {\isacharquery}M{\isacharequal}{\isachardoublequoteopen}{\isacharbraceleft}\isanewline
{\isacharparenleft}pair{\isacharcomma}setsum\ {\isacharparenleft}{\isacharpercent}g{\isachardot}\ bids\ {\isacharparenleft}fst\ pair{\isacharcomma}\ g{\isacharparenright}{\isacharparenright}\ {\isacharparenleft}finestpart\ {\isacharparenleft}snd\ pair{\isacharparenright}{\isacharparenright}{\isacharparenright}{\isacharbar}pair{\isachardot}\ pair\ {\isasymin}\ {\isacharparenleft}N\ {\isasymtimes}\ {\isacharparenleft}Pow\ G\ {\isacharminus}\ {\isacharbraceleft}{\isacharbraceleft}{\isacharbraceright}{\isacharbraceright}{\isacharparenright}{\isacharparenright}\ {\isasyminter}\ a{\isacharbraceright}{\isachardoublequoteclose}\isanewline
\isacommand{have}\isamarkupfalse%
\ {\isachardoublequoteopen}{\isacharquery}L\ {\isacharequal}\ {\isacharquery}M{\isachardoublequoteclose}\ \isacommand{by}\isamarkupfalse%
\ {\isacharparenleft}rule\ mm{\isadigit{6}}{\isadigit{6}}b{\isacharparenright}\isanewline
\isacommand{moreover}\isamarkupfalse%
\ \isacommand{have}\isamarkupfalse%
\ {\isachardoublequoteopen}{\isachardot}{\isachardot}{\isachardot}\ {\isacharequal}\ {\isacharquery}R{\isachardoublequoteclose}\ \isacommand{using}\isamarkupfalse%
\ mm{\isadigit{6}}{\isadigit{6}}c\ \isacommand{by}\isamarkupfalse%
\ blast\isanewline
\isacommand{ultimately}\isamarkupfalse%
\ \isacommand{show}\isamarkupfalse%
\ {\isacharquery}thesis\ \isacommand{by}\isamarkupfalse%
\ presburger\isanewline
\isacommand{qed}\isamarkupfalse%
%
\endisatagproof
{\isafoldproof}%
%
\isadelimproof
\isanewline
%
\endisadelimproof
\isacommand{lemma}\isamarkupfalse%
\ mm{\isadigit{5}}{\isadigit{7}}{\isacharcolon}\ {\isachardoublequoteopen}inj{\isacharunderscore}on\ {\isacharparenleft}partialCompletionOf\ bids{\isacharparenright}\ UNIV{\isachardoublequoteclose}%
\isadelimproof
\ %
\endisadelimproof
%
\isatagproof
\isacommand{using}\isamarkupfalse%
\ assms\ \isacommand{by}\isamarkupfalse%
\ {\isacharparenleft}metis\ {\isacharparenleft}lifting{\isacharparenright}\ fst{\isacharunderscore}conv\ inj{\isacharunderscore}on{\isacharunderscore}inverseI{\isacharparenright}%
\endisatagproof
{\isafoldproof}%
%
\isadelimproof
%
\endisadelimproof
\isanewline
\isacommand{corollary}\isamarkupfalse%
\ mm{\isadigit{5}}{\isadigit{7}}b{\isacharcolon}\ {\isachardoublequoteopen}inj{\isacharunderscore}on\ {\isacharparenleft}partialCompletionOf\ bids{\isacharparenright}\ X{\isachardoublequoteclose}%
\isadelimproof
\ %
\endisadelimproof
%
\isatagproof
\isacommand{using}\isamarkupfalse%
\ fst{\isacharunderscore}conv\ inj{\isacharunderscore}on{\isacharunderscore}inverseI\ \isacommand{by}\isamarkupfalse%
\ {\isacharparenleft}metis\ {\isacharparenleft}lifting{\isacharparenright}{\isacharparenright}%
\endisatagproof
{\isafoldproof}%
%
\isadelimproof
%
\endisadelimproof
\isanewline
\isacommand{lemma}\isamarkupfalse%
\ mm{\isadigit{5}}{\isadigit{8}}{\isacharcolon}\ {\isachardoublequoteopen}setsum\ snd\ {\isacharparenleft}LinearCompletion\ bids\ N\ G{\isacharparenright}\ {\isacharequal}\ \isanewline
setsum\ {\isacharparenleft}snd\ {\isasymcirc}\ {\isacharparenleft}partialCompletionOf\ bids{\isacharparenright}{\isacharparenright}\ {\isacharparenleft}N\ {\isasymtimes}\ {\isacharparenleft}Pow\ G\ {\isacharminus}\ {\isacharbraceleft}{\isacharbraceleft}{\isacharbraceright}{\isacharbraceright}{\isacharparenright}{\isacharparenright}{\isachardoublequoteclose}%
\isadelimproof
\ %
\endisadelimproof
%
\isatagproof
\isacommand{using}\isamarkupfalse%
\ assms\ \isanewline
mm{\isadigit{5}}{\isadigit{7}}b\ setsum{\isachardot}reindex\ \isacommand{by}\isamarkupfalse%
\ blast%
\endisatagproof
{\isafoldproof}%
%
\isadelimproof
%
\endisadelimproof
\ \isanewline
\isacommand{corollary}\isamarkupfalse%
\ mm{\isadigit{2}}{\isadigit{5}}{\isacharcolon}\ {\isachardoublequoteopen}snd\ {\isacharparenleft}partialCompletionOf\ bids\ pair{\isacharparenright}{\isacharequal}setsum\ bids\ {\isacharparenleft}omega\ pair{\isacharparenright}{\isachardoublequoteclose}%
\isadelimproof
\ %
\endisadelimproof
%
\isatagproof
\isacommand{using}\isamarkupfalse%
\ mm{\isadigit{2}}{\isadigit{4}}\ \isacommand{by}\isamarkupfalse%
\ force%
\endisatagproof
{\isafoldproof}%
%
\isadelimproof
%
\endisadelimproof
\isanewline
\isacommand{corollary}\isamarkupfalse%
\ mm{\isadigit{2}}{\isadigit{5}}b{\isacharcolon}\ {\isachardoublequoteopen}snd\ {\isasymcirc}\ partialCompletionOf\ bids\ {\isacharequal}\ {\isacharparenleft}setsum\ bids{\isacharparenright}\ {\isasymcirc}\ omega{\isachardoublequoteclose}%
\isadelimproof
\ %
\endisadelimproof
%
\isatagproof
\isacommand{using}\isamarkupfalse%
\ mm{\isadigit{2}}{\isadigit{5}}\ \isacommand{by}\isamarkupfalse%
\ fastforce%
\endisatagproof
{\isafoldproof}%
%
\isadelimproof
%
\endisadelimproof
\isanewline
\isanewline
\isanewline
\isanewline
\isacommand{lemma}\isamarkupfalse%
\ mm{\isadigit{2}}{\isadigit{7}}{\isacharcolon}\ \isakeyword{assumes}\ {\isachardoublequoteopen}finite\ \ {\isacharparenleft}finestpart\ {\isacharparenleft}snd\ pair{\isacharparenright}{\isacharparenright}{\isachardoublequoteclose}\ \isakeyword{shows}\ \isanewline
{\isachardoublequoteopen}card\ {\isacharparenleft}omega\ pair{\isacharparenright}\ {\isacharequal}\ card\ {\isacharparenleft}finestpart\ {\isacharparenleft}snd\ pair{\isacharparenright}{\isacharparenright}{\isachardoublequoteclose}%
\isadelimproof
\ %
\endisadelimproof
%
\isatagproof
\isacommand{using}\isamarkupfalse%
\ assms\ \isacommand{by}\isamarkupfalse%
\ force%
\endisatagproof
{\isafoldproof}%
%
\isadelimproof
%
\endisadelimproof
\isanewline
\isanewline
\isacommand{corollary}\isamarkupfalse%
\ \isakeyword{assumes}\ {\isachardoublequoteopen}finite\ {\isacharparenleft}snd\ pair{\isacharparenright}{\isachardoublequoteclose}\ \isakeyword{shows}\ {\isachardoublequoteopen}card\ {\isacharparenleft}omega\ pair{\isacharparenright}\ {\isacharequal}\ card\ {\isacharparenleft}snd\ pair{\isacharparenright}{\isachardoublequoteclose}\ \isanewline
%
\isadelimproof
%
\endisadelimproof
%
\isatagproof
\isacommand{using}\isamarkupfalse%
\ assms\ mm{\isadigit{2}}{\isadigit{6}}\ card{\isacharunderscore}cartesian{\isacharunderscore}product{\isacharunderscore}singleton\ \isacommand{by}\isamarkupfalse%
\ metis%
\endisatagproof
{\isafoldproof}%
%
\isadelimproof
\isanewline
%
\endisadelimproof
\isanewline
\isacommand{lemma}\isamarkupfalse%
\ mm{\isadigit{3}}{\isadigit{0}}{\isacharcolon}\ \isakeyword{assumes}\ {\isachardoublequoteopen}{\isacharbraceleft}{\isacharbraceright}\ {\isasymnotin}\ Range\ f{\isachardoublequoteclose}\ {\isachardoublequoteopen}runiq\ f{\isachardoublequoteclose}\ \isakeyword{shows}\ {\isachardoublequoteopen}is{\isacharunderscore}partition\ {\isacharparenleft}omega\ {\isacharbackquote}\ f{\isacharparenright}{\isachardoublequoteclose}\isanewline
%
\isadelimproof
\isanewline
%
\endisadelimproof
%
\isatagproof
\isacommand{proof}\isamarkupfalse%
\ {\isacharminus}\isanewline
\isacommand{let}\isamarkupfalse%
\ {\isacharquery}X{\isacharequal}{\isachardoublequoteopen}omega\ {\isacharbackquote}\ f{\isachardoublequoteclose}\ \isacommand{let}\isamarkupfalse%
\ {\isacharquery}p{\isacharequal}finestpart\isanewline
\ \ \isacommand{{\isacharbraceleft}}\isamarkupfalse%
\ \isacommand{fix}\isamarkupfalse%
\ y{\isadigit{1}}\ y{\isadigit{2}}\ \isacommand{assume}\isamarkupfalse%
\ {\isachardoublequoteopen}y{\isadigit{1}}\ {\isasymin}\ {\isacharquery}X\ {\isacharampersand}\ y{\isadigit{2}}\ {\isasymin}\ {\isacharquery}X{\isachardoublequoteclose}\isanewline
\ \ \ \ \isacommand{then}\isamarkupfalse%
\ \isacommand{obtain}\isamarkupfalse%
\ pair{\isadigit{1}}\ pair{\isadigit{2}}\ \isakeyword{where}\ \isanewline
\ \ \ \ {\isadigit{0}}{\isacharcolon}\ {\isachardoublequoteopen}y{\isadigit{1}}\ {\isacharequal}\ omega\ pair{\isadigit{1}}\ {\isacharampersand}\ y{\isadigit{2}}\ {\isacharequal}\ omega\ pair{\isadigit{2}}\ {\isacharampersand}\ pair{\isadigit{1}}\ {\isasymin}\ f\ {\isacharampersand}\ pair{\isadigit{2}}\ {\isasymin}\ f{\isachardoublequoteclose}\ \isacommand{by}\isamarkupfalse%
\ blast\isanewline
\ \ \ \ \isacommand{then}\isamarkupfalse%
\ \isacommand{moreover}\isamarkupfalse%
\ \isacommand{have}\isamarkupfalse%
\ {\isachardoublequoteopen}snd\ pair{\isadigit{1}}\ {\isasymnoteq}\ {\isacharbraceleft}{\isacharbraceright}\ {\isacharampersand}\ snd\ pair{\isadigit{1}}\ {\isasymnoteq}\ {\isacharbraceleft}{\isacharbraceright}{\isachardoublequoteclose}\ \isacommand{using}\isamarkupfalse%
\ assms\isanewline
\isacommand{by}\isamarkupfalse%
\ {\isacharparenleft}metis\ rev{\isacharunderscore}image{\isacharunderscore}eqI\ snd{\isacharunderscore}eq{\isacharunderscore}Range{\isacharparenright}\isanewline
\ \ \ \ \isacommand{ultimately}\isamarkupfalse%
\ \isacommand{moreover}\isamarkupfalse%
\ \isacommand{have}\isamarkupfalse%
\ {\isachardoublequoteopen}fst\ pair{\isadigit{1}}\ {\isacharequal}\ fst\ pair{\isadigit{2}}\ {\isasymlongleftrightarrow}\ pair{\isadigit{1}}\ {\isacharequal}\ pair{\isadigit{2}}{\isachardoublequoteclose}\ \isacommand{using}\isamarkupfalse%
\ assms\isanewline
\ \ \ \ runiq{\isacharunderscore}basic\ surjective{\isacharunderscore}pairing\ \isacommand{by}\isamarkupfalse%
\ metis\isanewline
\ \ \ \ \isacommand{ultimately}\isamarkupfalse%
\ \isacommand{moreover}\isamarkupfalse%
\ \isacommand{have}\isamarkupfalse%
\ {\isachardoublequoteopen}y{\isadigit{1}}\ {\isasyminter}\ y{\isadigit{2}}\ {\isasymnoteq}\ {\isacharbraceleft}{\isacharbraceright}\ {\isasymlongrightarrow}\ y{\isadigit{1}}\ {\isacharequal}\ y{\isadigit{2}}{\isachardoublequoteclose}\ \isacommand{using}\isamarkupfalse%
\ assms\ {\isadigit{0}}\ \isacommand{by}\isamarkupfalse%
\ fast\isanewline
\ \ \ \ \isacommand{ultimately}\isamarkupfalse%
\ \isacommand{have}\isamarkupfalse%
\ {\isachardoublequoteopen}y{\isadigit{1}}\ {\isacharequal}\ y{\isadigit{2}}\ {\isasymlongleftrightarrow}\ y{\isadigit{1}}\ {\isasyminter}\ y{\isadigit{2}}\ {\isasymnoteq}\ {\isacharbraceleft}{\isacharbraceright}{\isachardoublequoteclose}\ \isacommand{using}\isamarkupfalse%
\ assms\ mm{\isadigit{2}}{\isadigit{9}}\ \isanewline
\ \ \ \ \isacommand{by}\isamarkupfalse%
\ {\isacharparenleft}metis\ Int{\isacharunderscore}absorb\ Times{\isacharunderscore}empty\ insert{\isacharunderscore}not{\isacharunderscore}empty{\isacharparenright}\isanewline
\ \ \ \ \isacommand{{\isacharbraceright}}\isamarkupfalse%
\isanewline
\ \ \isacommand{thus}\isamarkupfalse%
\ {\isacharquery}thesis\ \isacommand{using}\isamarkupfalse%
\ is{\isacharunderscore}partition{\isacharunderscore}def\ \isacommand{by}\isamarkupfalse%
\ {\isacharparenleft}metis\ {\isacharparenleft}lifting{\isacharcomma}\ no{\isacharunderscore}types{\isacharparenright}\ inf{\isacharunderscore}commute\ inf{\isacharunderscore}sup{\isacharunderscore}aci{\isacharparenleft}{\isadigit{1}}{\isacharparenright}{\isacharparenright}\isanewline
\isacommand{qed}\isamarkupfalse%
%
\endisatagproof
{\isafoldproof}%
%
\isadelimproof
\isanewline
%
\endisadelimproof
\isanewline
\isacommand{lemma}\isamarkupfalse%
\ mm{\isadigit{3}}{\isadigit{2}}{\isacharcolon}\ \isakeyword{assumes}\ {\isachardoublequoteopen}{\isacharbraceleft}{\isacharbraceright}\ {\isasymnotin}\ Range\ X{\isachardoublequoteclose}\ \isakeyword{shows}\ {\isachardoublequoteopen}inj{\isacharunderscore}on\ omega\ X{\isachardoublequoteclose}\isanewline
%
\isadelimproof
%
\endisadelimproof
%
\isatagproof
\isacommand{proof}\isamarkupfalse%
\ {\isacharminus}\isanewline
\isacommand{let}\isamarkupfalse%
\ {\isacharquery}p{\isacharequal}finestpart\isanewline
\isacommand{{\isacharbraceleft}}\isamarkupfalse%
\isanewline
\ \ \isacommand{fix}\isamarkupfalse%
\ pair{\isadigit{1}}\ pair{\isadigit{2}}\ \isacommand{assume}\isamarkupfalse%
\ {\isachardoublequoteopen}pair{\isadigit{1}}\ {\isasymin}\ X\ {\isacharampersand}\ pair{\isadigit{2}}\ {\isasymin}\ X{\isachardoublequoteclose}\ \isacommand{then}\isamarkupfalse%
\ \isacommand{have}\isamarkupfalse%
\ \isanewline
\ \ {\isadigit{0}}{\isacharcolon}\ {\isachardoublequoteopen}snd\ pair{\isadigit{1}}\ {\isasymnoteq}\ {\isacharbraceleft}{\isacharbraceright}\ {\isacharampersand}\ snd\ pair{\isadigit{2}}\ {\isasymnoteq}\ {\isacharbraceleft}{\isacharbraceright}{\isachardoublequoteclose}\ \isacommand{using}\isamarkupfalse%
\ assms\ \isacommand{by}\isamarkupfalse%
\ {\isacharparenleft}metis\ Range{\isachardot}intros\ surjective{\isacharunderscore}pairing{\isacharparenright}\isanewline
\ \ \isacommand{assume}\isamarkupfalse%
\ {\isachardoublequoteopen}omega\ pair{\isadigit{1}}\ {\isacharequal}\ omega\ pair{\isadigit{2}}{\isachardoublequoteclose}\ \isacommand{then}\isamarkupfalse%
\ \isacommand{moreover}\isamarkupfalse%
\ \isacommand{have}\isamarkupfalse%
\ {\isachardoublequoteopen}{\isacharquery}p\ {\isacharparenleft}snd\ pair{\isadigit{1}}{\isacharparenright}\ {\isacharequal}\ {\isacharquery}p\ {\isacharparenleft}snd\ pair{\isadigit{2}}{\isacharparenright}{\isachardoublequoteclose}\ \isacommand{by}\isamarkupfalse%
\ blast\isanewline
\ \ \isacommand{then}\isamarkupfalse%
\ \isacommand{moreover}\isamarkupfalse%
\ \isacommand{have}\isamarkupfalse%
\ {\isachardoublequoteopen}snd\ pair{\isadigit{1}}\ {\isacharequal}\ snd\ pair{\isadigit{2}}{\isachardoublequoteclose}\ \isacommand{by}\isamarkupfalse%
\ {\isacharparenleft}metis\ ll{\isadigit{6}}{\isadigit{4}}\ mm{\isadigit{3}}{\isadigit{1}}{\isacharparenright}\isanewline
\ \ \isacommand{ultimately}\isamarkupfalse%
\ \isacommand{moreover}\isamarkupfalse%
\ \isacommand{have}\isamarkupfalse%
\ {\isachardoublequoteopen}{\isacharbraceleft}fst\ pair{\isadigit{1}}{\isacharbraceright}\ {\isacharequal}\ {\isacharbraceleft}fst\ pair{\isadigit{2}}{\isacharbraceright}{\isachardoublequoteclose}\ \isacommand{using}\isamarkupfalse%
\ {\isadigit{0}}\ mm{\isadigit{2}}{\isadigit{9}}\ \isacommand{by}\isamarkupfalse%
\ {\isacharparenleft}metis\ fst{\isacharunderscore}image{\isacharunderscore}times{\isacharparenright}\isanewline
\ \ \isacommand{ultimately}\isamarkupfalse%
\ \isacommand{have}\isamarkupfalse%
\ {\isachardoublequoteopen}pair{\isadigit{1}}\ {\isacharequal}\ pair{\isadigit{2}}{\isachardoublequoteclose}\ \isacommand{by}\isamarkupfalse%
\ {\isacharparenleft}metis\ prod{\isacharunderscore}eqI\ singleton{\isacharunderscore}inject{\isacharparenright}\isanewline
\isacommand{{\isacharbraceright}}\isamarkupfalse%
\isanewline
\isacommand{thus}\isamarkupfalse%
\ {\isacharquery}thesis\ \isacommand{by}\isamarkupfalse%
\ {\isacharparenleft}metis\ {\isacharparenleft}lifting{\isacharcomma}\ no{\isacharunderscore}types{\isacharparenright}\ inj{\isacharunderscore}onI{\isacharparenright}\isanewline
\isacommand{qed}\isamarkupfalse%
%
\endisatagproof
{\isafoldproof}%
%
\isadelimproof
\isanewline
%
\endisadelimproof
\isanewline
\isacommand{lemma}\isamarkupfalse%
\ mm{\isadigit{3}}{\isadigit{6}}{\isacharcolon}\ \isakeyword{assumes}\ {\isachardoublequoteopen}{\isacharbraceleft}{\isacharbraceright}\ {\isasymnotin}\ Range\ a{\isachardoublequoteclose}\ \isanewline
{\isachardoublequoteopen}finite\ {\isacharparenleft}omega\ {\isacharbackquote}\ a{\isacharparenright}{\isachardoublequoteclose}\ {\isachardoublequoteopen}{\isasymforall}X\ {\isasymin}\ omega\ {\isacharbackquote}\ a{\isachardot}\ finite\ X{\isachardoublequoteclose}\ {\isachardoublequoteopen}is{\isacharunderscore}partition\ {\isacharparenleft}omega\ {\isacharbackquote}\ a{\isacharparenright}{\isachardoublequoteclose}\isanewline
\isakeyword{shows}\ {\isachardoublequoteopen}card\ {\isacharparenleft}pseudoAllocation\ a{\isacharparenright}\ {\isacharequal}\ setsum\ {\isacharparenleft}card\ {\isasymcirc}\ omega{\isacharparenright}\ a{\isachardoublequoteclose}\ {\isacharparenleft}\isakeyword{is}\ {\isachardoublequoteopen}{\isacharquery}L\ {\isacharequal}\ {\isacharquery}R{\isachardoublequoteclose}{\isacharparenright}\isanewline
%
\isadelimproof
%
\endisadelimproof
%
\isatagproof
\isacommand{using}\isamarkupfalse%
\ assms\ mm{\isadigit{3}}{\isadigit{3}}\ UniformTieBreaking{\isachardot}mm{\isadigit{3}}{\isadigit{2}}\ setsum{\isachardot}reindex\ \isanewline
\isacommand{proof}\isamarkupfalse%
\ {\isacharminus}\isanewline
\isacommand{have}\isamarkupfalse%
\ {\isachardoublequoteopen}{\isacharquery}L\ {\isacharequal}\ setsum\ card\ {\isacharparenleft}omega\ {\isacharbackquote}\ a{\isacharparenright}{\isachardoublequoteclose}\ \isacommand{using}\isamarkupfalse%
\ assms{\isacharparenleft}{\isadigit{2}}{\isacharcomma}{\isadigit{3}}{\isacharcomma}{\isadigit{4}}{\isacharparenright}\ \isacommand{by}\isamarkupfalse%
\ {\isacharparenleft}rule\ mm{\isadigit{3}}{\isadigit{3}}{\isacharparenright}\isanewline
\isacommand{moreover}\isamarkupfalse%
\ \isacommand{have}\isamarkupfalse%
\ {\isachardoublequoteopen}{\isachardot}{\isachardot}{\isachardot}\ {\isacharequal}\ {\isacharquery}R{\isachardoublequoteclose}\ \isacommand{using}\isamarkupfalse%
\ assms{\isacharparenleft}{\isadigit{1}}{\isacharparenright}\ mm{\isadigit{3}}{\isadigit{2}}\ setsum{\isachardot}reindex\ \isacommand{by}\isamarkupfalse%
\ blast\isanewline
\isacommand{ultimately}\isamarkupfalse%
\ \isacommand{show}\isamarkupfalse%
\ {\isacharquery}thesis\ \isacommand{by}\isamarkupfalse%
\ presburger\isanewline
\isacommand{qed}\isamarkupfalse%
%
\endisatagproof
{\isafoldproof}%
%
\isadelimproof
\isanewline
%
\endisadelimproof
\isanewline
\isacommand{lemma}\isamarkupfalse%
\ mm{\isadigit{3}}{\isadigit{5}}{\isacharcolon}\ {\isachardoublequoteopen}card\ {\isacharparenleft}omega\ pair{\isacharparenright}{\isacharequal}\ card\ {\isacharparenleft}snd\ pair{\isacharparenright}{\isachardoublequoteclose}\ \isanewline
%
\isadelimproof
%
\endisadelimproof
%
\isatagproof
\isacommand{using}\isamarkupfalse%
\ mm{\isadigit{2}}{\isadigit{6}}\ card{\isacharunderscore}cartesian{\isacharunderscore}product{\isacharunderscore}singleton\ \isacommand{by}\isamarkupfalse%
\ metis%
\endisatagproof
{\isafoldproof}%
%
\isadelimproof
\isanewline
%
\endisadelimproof
\isanewline
\isacommand{corollary}\isamarkupfalse%
\ mm{\isadigit{3}}{\isadigit{5}}b{\isacharcolon}\ {\isachardoublequoteopen}card\ {\isasymcirc}\ omega\ {\isacharequal}\ card\ {\isasymcirc}\ snd{\isachardoublequoteclose}%
\isadelimproof
\ %
\endisadelimproof
%
\isatagproof
\isacommand{using}\isamarkupfalse%
\ mm{\isadigit{3}}{\isadigit{5}}\ \isacommand{by}\isamarkupfalse%
\ fastforce%
\endisatagproof
{\isafoldproof}%
%
\isadelimproof
%
\endisadelimproof
\isanewline
\isanewline
\isacommand{corollary}\isamarkupfalse%
\ mm{\isadigit{3}}{\isadigit{7}}{\isacharcolon}\ \isakeyword{assumes}\ {\isachardoublequoteopen}{\isacharbraceleft}{\isacharbraceright}\ {\isasymnotin}\ Range\ a{\isachardoublequoteclose}\ {\isachardoublequoteopen}{\isasymforall}\ pair\ {\isasymin}\ a{\isachardot}\ finite\ {\isacharparenleft}snd\ pair{\isacharparenright}{\isachardoublequoteclose}\ {\isachardoublequoteopen}finite\ a{\isachardoublequoteclose}\ {\isachardoublequoteopen}runiq\ a{\isachardoublequoteclose}\ \isanewline
\isakeyword{shows}\ {\isachardoublequoteopen}card\ {\isacharparenleft}pseudoAllocation\ a{\isacharparenright}\ {\isacharequal}\ setsum\ {\isacharparenleft}card\ {\isasymcirc}\ snd{\isacharparenright}\ a{\isachardoublequoteclose}\isanewline
%
\isadelimproof
%
\endisadelimproof
%
\isatagproof
\isacommand{proof}\isamarkupfalse%
\ {\isacharminus}\isanewline
\isacommand{let}\isamarkupfalse%
\ {\isacharquery}P{\isacharequal}pseudoAllocation\ \isacommand{let}\isamarkupfalse%
\ {\isacharquery}c{\isacharequal}card\isanewline
\isacommand{have}\isamarkupfalse%
\ {\isachardoublequoteopen}{\isasymforall}\ pair\ {\isasymin}\ a{\isachardot}\ finite\ {\isacharparenleft}omega\ pair{\isacharparenright}{\isachardoublequoteclose}\ \isacommand{using}\isamarkupfalse%
\ mm{\isadigit{4}}{\isadigit{0}}\ assms\ \isacommand{by}\isamarkupfalse%
\ blast\ \isacommand{moreover}\isamarkupfalse%
\isanewline
\isacommand{have}\isamarkupfalse%
\ {\isachardoublequoteopen}is{\isacharunderscore}partition\ {\isacharparenleft}omega\ {\isacharbackquote}\ a{\isacharparenright}{\isachardoublequoteclose}\ \isacommand{using}\isamarkupfalse%
\ assms\ mm{\isadigit{3}}{\isadigit{0}}\ \isacommand{by}\isamarkupfalse%
\ force\ \isacommand{ultimately}\isamarkupfalse%
\isanewline
\isacommand{have}\isamarkupfalse%
\ {\isachardoublequoteopen}{\isacharquery}c\ {\isacharparenleft}{\isacharquery}P\ a{\isacharparenright}\ {\isacharequal}\ setsum\ {\isacharparenleft}{\isacharquery}c\ {\isasymcirc}\ omega{\isacharparenright}\ a{\isachardoublequoteclose}\ \isacommand{using}\isamarkupfalse%
\ assms\ mm{\isadigit{3}}{\isadigit{6}}\ \isacommand{by}\isamarkupfalse%
\ force\isanewline
\isacommand{moreover}\isamarkupfalse%
\ \isacommand{have}\isamarkupfalse%
\ {\isachardoublequoteopen}{\isachardot}{\isachardot}{\isachardot}\ {\isacharequal}\ setsum\ {\isacharparenleft}{\isacharquery}c\ {\isasymcirc}\ snd{\isacharparenright}\ a{\isachardoublequoteclose}\ \isacommand{using}\isamarkupfalse%
\ mm{\isadigit{3}}{\isadigit{5}}b\ \isacommand{by}\isamarkupfalse%
\ metis\isanewline
\isacommand{ultimately}\isamarkupfalse%
\ \isacommand{show}\isamarkupfalse%
\ {\isacharquery}thesis\ \isacommand{by}\isamarkupfalse%
\ presburger\isanewline
\isacommand{qed}\isamarkupfalse%
%
\endisatagproof
{\isafoldproof}%
%
\isadelimproof
\isanewline
%
\endisadelimproof
\isanewline
\isacommand{corollary}\isamarkupfalse%
\ mm{\isadigit{4}}{\isadigit{6}}{\isacharcolon}\ \isakeyword{assumes}\ \isanewline
{\isachardoublequoteopen}runiq\ {\isacharparenleft}a{\isacharcircum}{\isacharminus}{\isadigit{1}}{\isacharparenright}{\isachardoublequoteclose}\ {\isachardoublequoteopen}runiq\ a{\isachardoublequoteclose}\ {\isachardoublequoteopen}finite\ a{\isachardoublequoteclose}\ {\isachardoublequoteopen}{\isacharbraceleft}{\isacharbraceright}\ {\isasymnotin}\ Range\ a{\isachardoublequoteclose}\ {\isachardoublequoteopen}{\isasymforall}\ pair\ {\isasymin}\ a{\isachardot}\ finite\ {\isacharparenleft}snd\ pair{\isacharparenright}{\isachardoublequoteclose}\ \isakeyword{shows}\ \isanewline
{\isachardoublequoteopen}card\ {\isacharparenleft}pseudoAllocation\ a{\isacharparenright}\ {\isacharequal}\ setsum\ card\ {\isacharparenleft}Range\ a{\isacharparenright}{\isachardoublequoteclose}%
\isadelimproof
\ %
\endisadelimproof
%
\isatagproof
\isacommand{using}\isamarkupfalse%
\ assms\ mm{\isadigit{3}}{\isadigit{9}}\ mm{\isadigit{3}}{\isadigit{7}}\ \isacommand{by}\isamarkupfalse%
\ force%
\endisatagproof
{\isafoldproof}%
%
\isadelimproof
%
\endisadelimproof
\isanewline
\isanewline
\isacommand{corollary}\isamarkupfalse%
\ mm{\isadigit{4}}{\isadigit{8}}{\isacharcolon}\ \isakeyword{assumes}\ {\isachardoublequoteopen}a\ {\isasymin}\ possibleAllocationsRel\ N\ G{\isachardoublequoteclose}\ {\isachardoublequoteopen}finite\ G{\isachardoublequoteclose}\ \isakeyword{shows}\ \isanewline
{\isachardoublequoteopen}card\ {\isacharparenleft}pseudoAllocation\ a{\isacharparenright}\ {\isacharequal}\ card\ G{\isachardoublequoteclose}\isanewline
%
\isadelimproof
%
\endisadelimproof
%
\isatagproof
\isacommand{proof}\isamarkupfalse%
\ {\isacharminus}\isanewline
\ \ \isacommand{have}\isamarkupfalse%
\ {\isachardoublequoteopen}{\isacharbraceleft}{\isacharbraceright}\ {\isasymnotin}\ Range\ a{\isachardoublequoteclose}\ \isacommand{using}\isamarkupfalse%
\ assms\ mm{\isadigit{4}}{\isadigit{5}}b\ \isacommand{by}\isamarkupfalse%
\ blast\isanewline
\ \ \isacommand{moreover}\isamarkupfalse%
\ \isacommand{have}\isamarkupfalse%
\ {\isachardoublequoteopen}{\isasymforall}\ pair\ {\isasymin}\ a{\isachardot}\ finite\ {\isacharparenleft}snd\ pair{\isacharparenright}{\isachardoublequoteclose}\ \isacommand{using}\isamarkupfalse%
\ assms\ mm{\isadigit{4}}{\isadigit{1}}\ mm{\isadigit{4}}{\isadigit{7}}\ \isacommand{by}\isamarkupfalse%
\ metis\isanewline
\ \ \isacommand{moreover}\isamarkupfalse%
\ \isacommand{have}\isamarkupfalse%
\ {\isachardoublequoteopen}finite\ a{\isachardoublequoteclose}\ \isacommand{using}\isamarkupfalse%
\ assms\ mm{\isadigit{4}}{\isadigit{4}}\ \isacommand{by}\isamarkupfalse%
\ blast\isanewline
\ \ \isacommand{moreover}\isamarkupfalse%
\ \isacommand{have}\isamarkupfalse%
\ {\isachardoublequoteopen}runiq\ a{\isachardoublequoteclose}\ \isacommand{using}\isamarkupfalse%
\ assms\ \isacommand{by}\isamarkupfalse%
\ {\isacharparenleft}metis\ {\isacharparenleft}lifting{\isacharparenright}\ Int{\isacharunderscore}lower{\isadigit{1}}\ in{\isacharunderscore}mono\ lm{\isadigit{1}}{\isadigit{9}}\ mem{\isacharunderscore}Collect{\isacharunderscore}eq{\isacharparenright}\isanewline
\ \ \isacommand{moreover}\isamarkupfalse%
\ \isacommand{have}\isamarkupfalse%
\ {\isachardoublequoteopen}runiq\ {\isacharparenleft}a{\isacharcircum}{\isacharminus}{\isadigit{1}}{\isacharparenright}{\isachardoublequoteclose}\ \isacommand{using}\isamarkupfalse%
\ assms\ \isacommand{by}\isamarkupfalse%
\ {\isacharparenleft}metis\ {\isacharparenleft}mono{\isacharunderscore}tags{\isacharparenright}\ injections{\isacharunderscore}def\ lm{\isadigit{2}}{\isadigit{8}}b\ mem{\isacharunderscore}Collect{\isacharunderscore}eq{\isacharparenright}\isanewline
\ \ \isacommand{ultimately}\isamarkupfalse%
\ \isacommand{have}\isamarkupfalse%
\ {\isachardoublequoteopen}card\ {\isacharparenleft}pseudoAllocation\ a{\isacharparenright}\ {\isacharequal}\ setsum\ card\ {\isacharparenleft}Range\ a{\isacharparenright}{\isachardoublequoteclose}\ \isacommand{using}\isamarkupfalse%
\ mm{\isadigit{4}}{\isadigit{6}}\ \isacommand{by}\isamarkupfalse%
\ fast\isanewline
\ \ \isacommand{moreover}\isamarkupfalse%
\ \isacommand{have}\isamarkupfalse%
\ {\isachardoublequoteopen}{\isachardot}{\isachardot}{\isachardot}\ {\isacharequal}\ card\ G{\isachardoublequoteclose}\ \isacommand{using}\isamarkupfalse%
\ assms\ mm{\isadigit{3}}{\isadigit{3}}c\ \isacommand{by}\isamarkupfalse%
\ metis\isanewline
\ \ \isacommand{ultimately}\isamarkupfalse%
\ \isacommand{show}\isamarkupfalse%
\ {\isacharquery}thesis\ \isacommand{by}\isamarkupfalse%
\ presburger\isanewline
\isacommand{qed}\isamarkupfalse%
%
\endisatagproof
{\isafoldproof}%
%
\isadelimproof
\isanewline
%
\endisadelimproof
\isanewline
\isacommand{corollary}\isamarkupfalse%
\ mm{\isadigit{4}}{\isadigit{9}}{\isacharcolon}\ \isakeyword{assumes}\ \isanewline
{\isachardoublequoteopen}pseudoAllocation\ aa\ {\isasymsubseteq}\ pseudoAllocation\ a\ {\isasymunion}\ {\isacharparenleft}N\ {\isasymtimes}\ {\isacharparenleft}finestpart\ G{\isacharparenright}{\isacharparenright}{\isachardoublequoteclose}\ {\isachardoublequoteopen}finite\ {\isacharparenleft}pseudoAllocation\ aa{\isacharparenright}{\isachardoublequoteclose}\isanewline
\isakeyword{shows}\ {\isachardoublequoteopen}setsum\ {\isacharparenleft}toFunction\ {\isacharparenleft}bidMaximizedBy\ a\ N\ G{\isacharparenright}{\isacharparenright}\ {\isacharparenleft}pseudoAllocation\ a{\isacharparenright}\ {\isacharminus}\ \isanewline
{\isacharparenleft}setsum\ {\isacharparenleft}toFunction\ {\isacharparenleft}bidMaximizedBy\ a\ N\ G{\isacharparenright}{\isacharparenright}\ {\isacharparenleft}pseudoAllocation\ aa{\isacharparenright}{\isacharparenright}\ {\isacharequal}\ \isanewline
card\ {\isacharparenleft}pseudoAllocation\ a{\isacharparenright}\ {\isacharminus}\ card\ {\isacharparenleft}pseudoAllocation\ aa\ {\isasyminter}\ {\isacharparenleft}pseudoAllocation\ a{\isacharparenright}{\isacharparenright}{\isachardoublequoteclose}%
\isadelimproof
\ %
\endisadelimproof
%
\isatagproof
\isacommand{using}\isamarkupfalse%
\ mm{\isadigit{2}}{\isadigit{8}}\ assms\isanewline
\isacommand{by}\isamarkupfalse%
\ blast%
\endisatagproof
{\isafoldproof}%
%
\isadelimproof
%
\endisadelimproof
\isanewline
\isanewline
\isacommand{corollary}\isamarkupfalse%
\ mm{\isadigit{4}}{\isadigit{9}}c{\isacharcolon}\ \isakeyword{assumes}\ \isanewline
{\isachardoublequoteopen}pseudoAllocation\ aa\ {\isasymsubseteq}\ pseudoAllocation\ a\ {\isasymunion}\ {\isacharparenleft}N\ {\isasymtimes}\ {\isacharparenleft}finestpart\ G{\isacharparenright}{\isacharparenright}{\isachardoublequoteclose}\ {\isachardoublequoteopen}finite\ {\isacharparenleft}pseudoAllocation\ aa{\isacharparenright}{\isachardoublequoteclose}\isanewline
\isakeyword{shows}\ {\isachardoublequoteopen}int\ {\isacharparenleft}setsum\ {\isacharparenleft}maxbid{\isacharprime}\ a\ N\ G{\isacharparenright}\ {\isacharparenleft}pseudoAllocation\ a{\isacharparenright}{\isacharparenright}\ {\isacharminus}\ \isanewline
int\ {\isacharparenleft}setsum\ {\isacharparenleft}maxbid{\isacharprime}\ a\ N\ G{\isacharparenright}\ {\isacharparenleft}pseudoAllocation\ aa{\isacharparenright}{\isacharparenright}\ {\isacharequal}\ \isanewline
int\ {\isacharparenleft}card\ {\isacharparenleft}pseudoAllocation\ a{\isacharparenright}{\isacharparenright}\ {\isacharminus}\ int\ {\isacharparenleft}card\ {\isacharparenleft}pseudoAllocation\ aa\ {\isasyminter}\ {\isacharparenleft}pseudoAllocation\ a{\isacharparenright}{\isacharparenright}{\isacharparenright}{\isachardoublequoteclose}%
\isadelimproof
\ %
\endisadelimproof
%
\isatagproof
\isacommand{using}\isamarkupfalse%
\ mm{\isadigit{2}}{\isadigit{8}}b\ assms\isanewline
\isacommand{by}\isamarkupfalse%
\ blast%
\endisatagproof
{\isafoldproof}%
%
\isadelimproof
%
\endisadelimproof
\isanewline
\isanewline
\isacommand{lemma}\isamarkupfalse%
\ mm{\isadigit{5}}{\isadigit{0}}{\isacharcolon}\ {\isachardoublequoteopen}pseudoAllocation\ {\isacharbraceleft}{\isacharbraceright}\ {\isacharequal}\ {\isacharbraceleft}{\isacharbraceright}{\isachardoublequoteclose}%
\isadelimproof
\ %
\endisadelimproof
%
\isatagproof
\isacommand{by}\isamarkupfalse%
\ simp%
\endisatagproof
{\isafoldproof}%
%
\isadelimproof
%
\endisadelimproof
\isanewline
\isanewline
\isacommand{corollary}\isamarkupfalse%
\ mm{\isadigit{5}}{\isadigit{3}}b{\isacharcolon}\ \isakeyword{assumes}\ {\isachardoublequoteopen}a\ {\isasymin}\ possibleAllocationsRel\ N\ {\isacharbraceleft}{\isacharbraceright}{\isachardoublequoteclose}\ \isakeyword{shows}\ {\isachardoublequoteopen}{\isacharparenleft}pseudoAllocation\ a{\isacharparenright}{\isacharequal}{\isacharbraceleft}{\isacharbraceright}{\isachardoublequoteclose}\isanewline
%
\isadelimproof
%
\endisadelimproof
%
\isatagproof
\isacommand{using}\isamarkupfalse%
\ assms\ mm{\isadigit{5}}{\isadigit{2}}\ \isacommand{by}\isamarkupfalse%
\ blast%
\endisatagproof
{\isafoldproof}%
%
\isadelimproof
\isanewline
%
\endisadelimproof
\isanewline
\isacommand{corollary}\isamarkupfalse%
\ mm{\isadigit{5}}{\isadigit{3}}{\isacharcolon}\ \isakeyword{assumes}\ {\isachardoublequoteopen}a\ {\isasymin}\ possibleAllocationsRel\ N\ G{\isachardoublequoteclose}\ {\isachardoublequoteopen}finite\ G{\isachardoublequoteclose}\ {\isachardoublequoteopen}G\ {\isasymnoteq}\ {\isacharbraceleft}{\isacharbraceright}{\isachardoublequoteclose}\isanewline
\isakeyword{shows}\ {\isachardoublequoteopen}finite\ {\isacharparenleft}pseudoAllocation\ a{\isacharparenright}{\isachardoublequoteclose}\ \isanewline
%
\isadelimproof
%
\endisadelimproof
%
\isatagproof
\isacommand{proof}\isamarkupfalse%
\ {\isacharminus}\isanewline
\ \ \isacommand{have}\isamarkupfalse%
\ {\isachardoublequoteopen}card\ {\isacharparenleft}pseudoAllocation\ a{\isacharparenright}\ {\isacharequal}\ card\ G{\isachardoublequoteclose}\ \isacommand{using}\isamarkupfalse%
\ assms{\isacharparenleft}{\isadigit{1}}{\isacharcomma}{\isadigit{2}}{\isacharparenright}\ mm{\isadigit{4}}{\isadigit{8}}\ \isacommand{by}\isamarkupfalse%
\ blast\isanewline
\ \ \isacommand{thus}\isamarkupfalse%
\ {\isachardoublequoteopen}finite\ {\isacharparenleft}pseudoAllocation\ a{\isacharparenright}{\isachardoublequoteclose}\ \isacommand{using}\isamarkupfalse%
\ assms{\isacharparenleft}{\isadigit{2}}{\isacharcomma}{\isadigit{3}}{\isacharparenright}\ \isacommand{by}\isamarkupfalse%
\ fastforce\isanewline
\isacommand{qed}\isamarkupfalse%
%
\endisatagproof
{\isafoldproof}%
%
\isadelimproof
\isanewline
%
\endisadelimproof
\isanewline
\isacommand{corollary}\isamarkupfalse%
\ mm{\isadigit{5}}{\isadigit{4}}{\isacharcolon}\ \isakeyword{assumes}\ {\isachardoublequoteopen}a\ {\isasymin}\ possibleAllocationsRel\ N\ G{\isachardoublequoteclose}\ {\isachardoublequoteopen}finite\ G{\isachardoublequoteclose}\ \isakeyword{shows}\ \isanewline
{\isachardoublequoteopen}finite\ {\isacharparenleft}pseudoAllocation\ a{\isacharparenright}{\isachardoublequoteclose}%
\isadelimproof
\ %
\endisadelimproof
%
\isatagproof
\isacommand{using}\isamarkupfalse%
\ assms\ finite{\isachardot}emptyI\ mm{\isadigit{5}}{\isadigit{3}}\ mm{\isadigit{5}}{\isadigit{3}}b\ \isacommand{by}\isamarkupfalse%
\ {\isacharparenleft}metis\ {\isacharparenleft}no{\isacharunderscore}types{\isacharparenright}{\isacharparenright}%
\endisatagproof
{\isafoldproof}%
%
\isadelimproof
%
\endisadelimproof
\isanewline
\isanewline
\isacommand{lemma}\isamarkupfalse%
\ mm{\isadigit{5}}{\isadigit{6}}{\isacharcolon}\ \isakeyword{assumes}\ {\isachardoublequoteopen}a\ {\isasymin}\ possibleAllocationsRel\ N\ G{\isachardoublequoteclose}\ {\isachardoublequoteopen}aa\ {\isasymin}\ possibleAllocationsRel\ N\ G{\isachardoublequoteclose}\ {\isachardoublequoteopen}finite\ G{\isachardoublequoteclose}\ \isakeyword{shows}\ \isanewline
{\isachardoublequoteopen}{\isacharparenleft}card\ {\isacharparenleft}pseudoAllocation\ aa\ {\isasyminter}\ {\isacharparenleft}pseudoAllocation\ a{\isacharparenright}{\isacharparenright}\ {\isacharequal}\ card\ {\isacharparenleft}pseudoAllocation\ a{\isacharparenright}{\isacharparenright}\ {\isacharequal}\ \isanewline
{\isacharparenleft}pseudoAllocation\ a\ {\isacharequal}\ pseudoAllocation\ aa{\isacharparenright}{\isachardoublequoteclose}%
\isadelimproof
\ %
\endisadelimproof
%
\isatagproof
\isacommand{using}\isamarkupfalse%
\ assms\ mm{\isadigit{4}}{\isadigit{8}}\ mm{\isadigit{2}}{\isadigit{3}}b\ \isanewline
\isacommand{proof}\isamarkupfalse%
\ {\isacharminus}\isanewline
\isacommand{let}\isamarkupfalse%
\ {\isacharquery}P{\isacharequal}pseudoAllocation\ \isacommand{let}\isamarkupfalse%
\ {\isacharquery}c{\isacharequal}card\ \isacommand{let}\isamarkupfalse%
\ {\isacharquery}A{\isacharequal}{\isachardoublequoteopen}{\isacharquery}P\ a{\isachardoublequoteclose}\ \isacommand{let}\isamarkupfalse%
\ {\isacharquery}AA{\isacharequal}{\isachardoublequoteopen}{\isacharquery}P\ aa{\isachardoublequoteclose}\isanewline
\isacommand{have}\isamarkupfalse%
\ {\isachardoublequoteopen}{\isacharquery}c\ {\isacharquery}A{\isacharequal}{\isacharquery}c\ G\ {\isacharampersand}\ {\isacharquery}c\ {\isacharquery}AA{\isacharequal}{\isacharquery}c\ G{\isachardoublequoteclose}\ \isacommand{using}\isamarkupfalse%
\ assms\ mm{\isadigit{4}}{\isadigit{8}}\ \isacommand{by}\isamarkupfalse%
\ {\isacharparenleft}metis\ {\isacharparenleft}lifting{\isacharcomma}\ mono{\isacharunderscore}tags{\isacharparenright}{\isacharparenright}\isanewline
\isacommand{moreover}\isamarkupfalse%
\ \isacommand{have}\isamarkupfalse%
\ {\isachardoublequoteopen}finite\ {\isacharquery}A\ {\isacharampersand}\ finite\ {\isacharquery}AA{\isachardoublequoteclose}\ \isacommand{using}\isamarkupfalse%
\ assms\ mm{\isadigit{5}}{\isadigit{4}}\ \isacommand{by}\isamarkupfalse%
\ blast\isanewline
\isacommand{ultimately}\isamarkupfalse%
\ \isacommand{show}\isamarkupfalse%
\ {\isacharquery}thesis\ \isacommand{using}\isamarkupfalse%
\ assms\ mm{\isadigit{2}}{\isadigit{3}}b\ \isacommand{by}\isamarkupfalse%
\ {\isacharparenleft}metis{\isacharparenleft}no{\isacharunderscore}types{\isacharcomma}lifting{\isacharparenright}{\isacharparenright}\isanewline
\isacommand{qed}\isamarkupfalse%
%
\endisatagproof
{\isafoldproof}%
%
\isadelimproof
%
\endisadelimproof
\isanewline
\isanewline
\isacommand{lemma}\isamarkupfalse%
\ mm{\isadigit{5}}{\isadigit{5}}{\isacharcolon}\ {\isachardoublequoteopen}omega\ pair\ {\isacharequal}\ {\isacharbraceleft}fst\ pair{\isacharbraceright}\ {\isasymtimes}\ {\isacharbraceleft}{\isacharbraceleft}y{\isacharbraceright}{\isacharbar}\ y{\isachardot}\ y\ {\isasymin}\ snd\ pair{\isacharbraceright}{\isachardoublequoteclose}%
\isadelimproof
\ %
\endisadelimproof
%
\isatagproof
\isacommand{using}\isamarkupfalse%
\ finestpart{\isacharunderscore}def\ ll{\isadigit{6}}{\isadigit{4}}\ \isacommand{by}\isamarkupfalse%
\ auto%
\endisatagproof
{\isafoldproof}%
%
\isadelimproof
%
\endisadelimproof
\isanewline
\isanewline
\isacommand{lemma}\isamarkupfalse%
\ mm{\isadigit{5}}{\isadigit{5}}c{\isacharcolon}\ {\isachardoublequoteopen}omega\ pair\ {\isacharequal}\ {\isacharbraceleft}{\isacharparenleft}fst\ pair{\isacharcomma}\ {\isacharbraceleft}y{\isacharbraceright}{\isacharparenright}{\isacharbar}\ y{\isachardot}\ y\ {\isasymin}\ \ snd\ pair{\isacharbraceright}{\isachardoublequoteclose}%
\isadelimproof
\ %
\endisadelimproof
%
\isatagproof
\isacommand{using}\isamarkupfalse%
\ mm{\isadigit{5}}{\isadigit{5}}\ mm{\isadigit{5}}{\isadigit{5}}b\ \isacommand{by}\isamarkupfalse%
\ metis%
\endisatagproof
{\isafoldproof}%
%
\isadelimproof
%
\endisadelimproof
\isanewline
\isanewline
\isacommand{lemma}\isamarkupfalse%
\ mm{\isadigit{5}}{\isadigit{5}}d{\isacharcolon}\ {\isachardoublequoteopen}pseudoAllocation\ a\ {\isacharequal}\ {\isasymUnion}\ {\isacharbraceleft}{\isacharbraceleft}{\isacharparenleft}fst\ pair{\isacharcomma}\ {\isacharbraceleft}y{\isacharbraceright}{\isacharparenright}{\isacharbar}\ y{\isachardot}\ y\ {\isasymin}\ snd\ pair{\isacharbraceright}{\isacharbar}\ pair{\isachardot}\ pair\ {\isasymin}\ a{\isacharbraceright}{\isachardoublequoteclose}\isanewline
%
\isadelimproof
%
\endisadelimproof
%
\isatagproof
\isacommand{using}\isamarkupfalse%
\ mm{\isadigit{5}}{\isadigit{5}}c\ \isacommand{by}\isamarkupfalse%
\ blast%
\endisatagproof
{\isafoldproof}%
%
\isadelimproof
\isanewline
%
\endisadelimproof
\isacommand{lemma}\isamarkupfalse%
\ mm{\isadigit{5}}{\isadigit{5}}e{\isacharcolon}\ {\isachardoublequoteopen}{\isasymUnion}\ {\isacharbraceleft}{\isacharbraceleft}{\isacharparenleft}fst\ pair{\isacharcomma}\ {\isacharbraceleft}y{\isacharbraceright}{\isacharparenright}{\isacharbar}\ y{\isachardot}\ y\ {\isasymin}\ snd\ pair{\isacharbraceright}{\isacharbar}\ pair{\isachardot}\ pair\ {\isasymin}\ a{\isacharbraceright}{\isacharequal}\isanewline
{\isacharbraceleft}{\isacharparenleft}fst\ pair{\isacharcomma}\ {\isacharbraceleft}y{\isacharbraceright}{\isacharparenright}{\isacharbar}\ y\ pair{\isachardot}\ y\ {\isasymin}\ snd\ pair\ {\isacharampersand}\ pair\ {\isasymin}\ a{\isacharbraceright}{\isachardoublequoteclose}%
\isadelimproof
\ %
\endisadelimproof
%
\isatagproof
\isacommand{by}\isamarkupfalse%
\ blast%
\endisatagproof
{\isafoldproof}%
%
\isadelimproof
%
\endisadelimproof
\isanewline
\isanewline
\isacommand{corollary}\isamarkupfalse%
\ mm{\isadigit{5}}{\isadigit{5}}k{\isacharcolon}\ {\isachardoublequoteopen}pseudoAllocation\ a\ {\isacharequal}\ {\isacharbraceleft}{\isacharparenleft}fst\ pair{\isacharcomma}\ Y{\isacharparenright}{\isacharbar}\ Y\ pair{\isachardot}\ Y\ {\isasymin}\ finestpart\ {\isacharparenleft}snd\ pair{\isacharparenright}\ {\isacharampersand}\ pair\ {\isasymin}\ a{\isacharbraceright}{\isachardoublequoteclose}\ \isanewline
%
\isadelimproof
%
\endisadelimproof
%
\isatagproof
\isacommand{using}\isamarkupfalse%
\ mm{\isadigit{5}}{\isadigit{5}}j\ \isacommand{by}\isamarkupfalse%
\ blast%
\endisatagproof
{\isafoldproof}%
%
\isadelimproof
\isanewline
%
\endisadelimproof
\isanewline
\isacommand{lemma}\isamarkupfalse%
\ mm{\isadigit{5}}{\isadigit{5}}u{\isacharcolon}\ \isakeyword{assumes}\ {\isachardoublequoteopen}runiq\ a{\isachardoublequoteclose}\ \isakeyword{shows}\ \isanewline
{\isachardoublequoteopen}{\isacharbraceleft}{\isacharparenleft}fst\ pair{\isacharcomma}\ Y{\isacharparenright}{\isacharbar}\ Y\ pair{\isachardot}\ Y\ {\isasymin}\ finestpart\ {\isacharparenleft}snd\ pair{\isacharparenright}\ {\isacharampersand}\ pair\ {\isasymin}\ a{\isacharbraceright}\ {\isacharequal}\ {\isacharbraceleft}{\isacharparenleft}x{\isacharcomma}\ Y{\isacharparenright}{\isacharbar}\ Y\ x{\isachardot}\ Y\ {\isasymin}\ finestpart\ {\isacharparenleft}a{\isacharcomma}{\isacharcomma}x{\isacharparenright}\ {\isacharampersand}\ x\ {\isasymin}\ Domain\ a{\isacharbraceright}{\isachardoublequoteclose}\isanewline
{\isacharparenleft}\isakeyword{is}\ {\isachardoublequoteopen}{\isacharquery}L{\isacharequal}{\isacharquery}R{\isachardoublequoteclose}{\isacharparenright}%
\isadelimproof
\ %
\endisadelimproof
%
\isatagproof
\isacommand{using}\isamarkupfalse%
\ assms\ Domain{\isachardot}DomainI\ fst{\isacharunderscore}conv\ mm{\isadigit{6}}{\isadigit{0}}\ runiq{\isacharunderscore}wrt{\isacharunderscore}ex{\isadigit{1}}\ surjective{\isacharunderscore}pairing\isanewline
\isacommand{by}\isamarkupfalse%
\ {\isacharparenleft}metis{\isacharparenleft}hide{\isacharunderscore}lams{\isacharcomma}no{\isacharunderscore}types{\isacharparenright}{\isacharparenright}%
\endisatagproof
{\isafoldproof}%
%
\isadelimproof
%
\endisadelimproof
\isanewline
\isanewline
\isacommand{corollary}\isamarkupfalse%
\ mm{\isadigit{5}}{\isadigit{5}}v{\isacharcolon}\ \isakeyword{assumes}\ {\isachardoublequoteopen}runiq\ a{\isachardoublequoteclose}\ \isakeyword{shows}\ {\isachardoublequoteopen}pseudoAllocation\ a\ {\isacharequal}\ {\isacharbraceleft}{\isacharparenleft}x{\isacharcomma}\ Y{\isacharparenright}{\isacharbar}\ Y\ x{\isachardot}\ Y\ {\isasymin}\ finestpart\ {\isacharparenleft}a{\isacharcomma}{\isacharcomma}x{\isacharparenright}\ {\isacharampersand}\ x\ {\isasymin}\ Domain\ a{\isacharbraceright}{\isachardoublequoteclose}\isanewline
%
\isadelimproof
%
\endisadelimproof
%
\isatagproof
\isacommand{using}\isamarkupfalse%
\ assms\ mm{\isadigit{5}}{\isadigit{5}}u\ mm{\isadigit{5}}{\isadigit{5}}k\ \isacommand{by}\isamarkupfalse%
\ fastforce%
\endisatagproof
{\isafoldproof}%
%
\isadelimproof
\isanewline
%
\endisadelimproof
\isanewline
\isacommand{corollary}\isamarkupfalse%
\ mm{\isadigit{5}}{\isadigit{5}}t{\isacharcolon}\ {\isachardoublequoteopen}Range\ {\isacharparenleft}pseudoAllocation\ a{\isacharparenright}\ {\isacharequal}\ {\isasymUnion}\ {\isacharparenleft}finestpart\ {\isacharbackquote}\ {\isacharparenleft}Range\ a{\isacharparenright}{\isacharparenright}{\isachardoublequoteclose}\ \isanewline
%
\isadelimproof
%
\endisadelimproof
%
\isatagproof
\isacommand{using}\isamarkupfalse%
\ mm{\isadigit{5}}{\isadigit{5}}k\ mm{\isadigit{5}}{\isadigit{5}}l\ mm{\isadigit{5}}{\isadigit{5}}m\ \isacommand{by}\isamarkupfalse%
\ fastforce%
\endisatagproof
{\isafoldproof}%
%
\isadelimproof
\isanewline
%
\endisadelimproof
\isanewline
\isacommand{corollary}\isamarkupfalse%
\ mm{\isadigit{5}}{\isadigit{5}}s{\isacharcolon}\ {\isachardoublequoteopen}Range\ {\isacharparenleft}pseudoAllocation\ a{\isacharparenright}\ {\isacharequal}\ finestpart\ {\isacharparenleft}{\isasymUnion}\ Range\ a{\isacharparenright}{\isachardoublequoteclose}%
\isadelimproof
\ %
\endisadelimproof
%
\isatagproof
\isacommand{using}\isamarkupfalse%
\ mm{\isadigit{5}}{\isadigit{5}}r\ mm{\isadigit{5}}{\isadigit{5}}t\ \isacommand{by}\isamarkupfalse%
\ metis%
\endisatagproof
{\isafoldproof}%
%
\isadelimproof
%
\endisadelimproof
\ \isanewline
\isanewline
\isanewline
\isacommand{lemma}\isamarkupfalse%
\ mm{\isadigit{5}}{\isadigit{5}}f{\isacharcolon}\ {\isachardoublequoteopen}pseudoAllocation\ a\ {\isacharequal}\ {\isacharbraceleft}{\isacharparenleft}fst\ pair{\isacharcomma}\ {\isacharbraceleft}y{\isacharbraceright}{\isacharparenright}{\isacharbar}\ y\ pair{\isachardot}\ y\ {\isasymin}\ snd\ pair\ {\isacharampersand}\ pair\ {\isasymin}\ a{\isacharbraceright}{\isachardoublequoteclose}%
\isadelimproof
\ %
\endisadelimproof
%
\isatagproof
\isacommand{using}\isamarkupfalse%
\ mm{\isadigit{5}}{\isadigit{5}}d\ \isanewline
mm{\isadigit{5}}{\isadigit{5}}e\ \isacommand{by}\isamarkupfalse%
\ {\isacharparenleft}metis\ {\isacharparenleft}full{\isacharunderscore}types{\isacharparenright}{\isacharparenright}%
\endisatagproof
{\isafoldproof}%
%
\isadelimproof
%
\endisadelimproof
\isanewline
\isacommand{lemma}\isamarkupfalse%
\ mm{\isadigit{5}}{\isadigit{5}}g{\isacharcolon}\ {\isachardoublequoteopen}{\isacharbraceleft}{\isacharparenleft}fst\ pair{\isacharcomma}\ {\isacharbraceleft}y{\isacharbraceright}{\isacharparenright}{\isacharbar}\ y\ pair{\isachardot}\ y\ {\isasymin}\ snd\ pair\ {\isacharampersand}\ pair\ {\isasymin}\ a{\isacharbraceright}{\isacharequal}\isanewline
{\isacharbraceleft}{\isacharparenleft}x{\isacharcomma}\ {\isacharbraceleft}y{\isacharbraceright}{\isacharparenright}{\isacharbar}\ x\ y{\isachardot}\ y\ {\isasymin}\ {\isasymUnion}\ {\isacharparenleft}a{\isacharbackquote}{\isacharbackquote}{\isacharbraceleft}x{\isacharbraceright}{\isacharparenright}\ {\isacharampersand}\ x\ {\isasymin}\ Domain\ a{\isacharbraceright}{\isachardoublequoteclose}%
\isadelimproof
\ %
\endisadelimproof
%
\isatagproof
\isacommand{by}\isamarkupfalse%
\ auto%
\endisatagproof
{\isafoldproof}%
%
\isadelimproof
%
\endisadelimproof
\isanewline
\isanewline
\isacommand{lemma}\isamarkupfalse%
\ mm{\isadigit{5}}{\isadigit{5}}i{\isacharcolon}\ {\isachardoublequoteopen}pseudoAllocation\ a\ {\isacharequal}\ {\isacharbraceleft}{\isacharparenleft}x{\isacharcomma}\ {\isacharbraceleft}y{\isacharbraceright}{\isacharparenright}{\isacharbar}\ x\ y{\isachardot}\ y\ {\isasymin}\ {\isasymUnion}\ {\isacharparenleft}a{\isacharbackquote}{\isacharbackquote}{\isacharbraceleft}x{\isacharbraceright}{\isacharparenright}\ {\isacharampersand}\ x\ {\isasymin}\ Domain\ a{\isacharbraceright}{\isachardoublequoteclose}\ {\isacharparenleft}\isakeyword{is}\ {\isachardoublequoteopen}{\isacharquery}L{\isacharequal}{\isacharquery}R{\isachardoublequoteclose}{\isacharparenright}\isanewline
%
\isadelimproof
%
\endisadelimproof
%
\isatagproof
\isacommand{proof}\isamarkupfalse%
\ {\isacharminus}\isanewline
\isacommand{have}\isamarkupfalse%
\ {\isachardoublequoteopen}{\isacharquery}L{\isacharequal}{\isacharbraceleft}{\isacharparenleft}fst\ pair{\isacharcomma}\ {\isacharbraceleft}y{\isacharbraceright}{\isacharparenright}{\isacharbar}\ y\ pair{\isachardot}\ y\ {\isasymin}\ snd\ pair\ {\isacharampersand}\ pair\ {\isasymin}\ a{\isacharbraceright}{\isachardoublequoteclose}\ \isacommand{by}\isamarkupfalse%
\ {\isacharparenleft}rule\ mm{\isadigit{5}}{\isadigit{5}}f{\isacharparenright}\isanewline
\isacommand{moreover}\isamarkupfalse%
\ \isacommand{have}\isamarkupfalse%
\ {\isachardoublequoteopen}{\isachardot}{\isachardot}{\isachardot}\ {\isacharequal}\ {\isacharquery}R{\isachardoublequoteclose}\ \isacommand{by}\isamarkupfalse%
\ {\isacharparenleft}rule\ mm{\isadigit{5}}{\isadigit{5}}g{\isacharparenright}\ \isacommand{ultimately}\isamarkupfalse%
\ \isacommand{show}\isamarkupfalse%
\ {\isacharquery}thesis\ \isacommand{by}\isamarkupfalse%
\ presburger\isanewline
\isacommand{qed}\isamarkupfalse%
%
\endisatagproof
{\isafoldproof}%
%
\isadelimproof
\isanewline
%
\endisadelimproof
\isanewline
\isacommand{lemma}\isamarkupfalse%
\ mm{\isadigit{6}}{\isadigit{2}}{\isacharcolon}\ {\isachardoublequoteopen}runiq\ {\isacharparenleft}LinearCompletion\ bids\ N\ G{\isacharparenright}{\isachardoublequoteclose}%
\isadelimproof
\ %
\endisadelimproof
%
\isatagproof
\isacommand{using}\isamarkupfalse%
\ assms\ \isacommand{by}\isamarkupfalse%
\ {\isacharparenleft}metis\ graph{\isacharunderscore}def\ image{\isacharunderscore}Collect{\isacharunderscore}mem\ ll{\isadigit{3}}{\isadigit{7}}{\isacharparenright}%
\endisatagproof
{\isafoldproof}%
%
\isadelimproof
%
\endisadelimproof
\isanewline
\isacommand{corollary}\isamarkupfalse%
\ mm{\isadigit{6}}{\isadigit{2}}b{\isacharcolon}\ {\isachardoublequoteopen}runiq\ {\isacharparenleft}LinearCompletion\ bids\ N\ G\ {\isacharbar}{\isacharbar}\ a{\isacharparenright}{\isachardoublequoteclose}\isanewline
%
\isadelimproof
%
\endisadelimproof
%
\isatagproof
\isacommand{unfolding}\isamarkupfalse%
\ restrict{\isacharunderscore}def\ \isacommand{using}\isamarkupfalse%
\ mm{\isadigit{6}}{\isadigit{2}}\ subrel{\isacharunderscore}runiq\ Int{\isacharunderscore}commute\ \isacommand{by}\isamarkupfalse%
\ blast%
\endisatagproof
{\isafoldproof}%
%
\isadelimproof
\isanewline
%
\endisadelimproof
\isacommand{lemma}\isamarkupfalse%
\ mm{\isadigit{6}}{\isadigit{4}}{\isacharcolon}\ {\isachardoublequoteopen}N\ {\isasymtimes}\ {\isacharparenleft}Pow\ G\ {\isacharminus}\ {\isacharbraceleft}{\isacharbraceleft}{\isacharbraceright}{\isacharbraceright}{\isacharparenright}\ {\isacharequal}\ Domain\ {\isacharparenleft}LinearCompletion\ bids\ N\ G{\isacharparenright}{\isachardoublequoteclose}%
\isadelimproof
\ %
\endisadelimproof
%
\isatagproof
\isacommand{by}\isamarkupfalse%
\ blast%
\endisatagproof
{\isafoldproof}%
%
\isadelimproof
%
\endisadelimproof
\isanewline
\isanewline
\isacommand{corollary}\isamarkupfalse%
\ mm{\isadigit{6}}{\isadigit{3}}d{\isacharcolon}\ \isakeyword{assumes}\ {\isachardoublequoteopen}a\ {\isasymin}\ possibleAllocationsRel\ N\ G{\isachardoublequoteclose}\ \isakeyword{shows}\ {\isachardoublequoteopen}a\ {\isasymsubseteq}\ Domain\ {\isacharparenleft}LinearCompletion\ bids\ N\ G{\isacharparenright}{\isachardoublequoteclose}\isanewline
%
\isadelimproof
%
\endisadelimproof
%
\isatagproof
\isacommand{proof}\isamarkupfalse%
\ {\isacharminus}\isanewline
\isacommand{let}\isamarkupfalse%
\ {\isacharquery}p{\isacharequal}possibleAllocationsRel\ \isacommand{let}\isamarkupfalse%
\ {\isacharquery}L{\isacharequal}LinearCompletion\isanewline
\isacommand{have}\isamarkupfalse%
\ {\isachardoublequoteopen}a\ {\isasymsubseteq}\ N\ {\isasymtimes}\ {\isacharparenleft}Pow\ G\ {\isacharminus}\ {\isacharbraceleft}{\isacharbraceleft}{\isacharbraceright}{\isacharbraceright}{\isacharparenright}{\isachardoublequoteclose}\ \isacommand{using}\isamarkupfalse%
\ assms\ mm{\isadigit{6}}{\isadigit{3}}c\ \isacommand{by}\isamarkupfalse%
\ metis\isanewline
\isacommand{moreover}\isamarkupfalse%
\ \isacommand{have}\isamarkupfalse%
\ {\isachardoublequoteopen}N\ {\isasymtimes}\ {\isacharparenleft}Pow\ G\ {\isacharminus}\ {\isacharbraceleft}{\isacharbraceleft}{\isacharbraceright}{\isacharbraceright}{\isacharparenright}{\isacharequal}Domain\ {\isacharparenleft}{\isacharquery}L\ bids\ N\ G{\isacharparenright}{\isachardoublequoteclose}\ \isacommand{using}\isamarkupfalse%
\ mm{\isadigit{6}}{\isadigit{4}}\ \isacommand{by}\isamarkupfalse%
\ blast\isanewline
\isacommand{ultimately}\isamarkupfalse%
\ \isacommand{show}\isamarkupfalse%
\ {\isacharquery}thesis\ \isacommand{by}\isamarkupfalse%
\ blast\isanewline
\isacommand{qed}\isamarkupfalse%
%
\endisatagproof
{\isafoldproof}%
%
\isadelimproof
\isanewline
%
\endisadelimproof
\isanewline
\isacommand{corollary}\isamarkupfalse%
\ mm{\isadigit{5}}{\isadigit{9}}d{\isacharcolon}\ {\isachardoublequoteopen}setsum\ {\isacharparenleft}linearCompletion{\isacharprime}\ bids\ N\ G{\isacharparenright}\ {\isacharparenleft}a\ {\isasyminter}\ {\isacharparenleft}Domain\ {\isacharparenleft}LinearCompletion\ bids\ N\ G{\isacharparenright}{\isacharparenright}{\isacharparenright}\ {\isacharequal}\ \isanewline
setsum\ snd\ {\isacharparenleft}{\isacharparenleft}LinearCompletion\ bids\ N\ G{\isacharparenright}\ {\isacharbar}{\isacharbar}\ a{\isacharparenright}{\isachardoublequoteclose}%
\isadelimproof
\ %
\endisadelimproof
%
\isatagproof
\isacommand{using}\isamarkupfalse%
\ assms\ mm{\isadigit{5}}{\isadigit{9}}c\ mm{\isadigit{6}}{\isadigit{2}}b\ \isacommand{by}\isamarkupfalse%
\ fast%
\endisatagproof
{\isafoldproof}%
%
\isadelimproof
%
\endisadelimproof
\isanewline
\isanewline
\isacommand{corollary}\isamarkupfalse%
\ mm{\isadigit{5}}{\isadigit{9}}e{\isacharcolon}\ \isakeyword{assumes}\ {\isachardoublequoteopen}a\ {\isasymin}\ possibleAllocationsRel\ N\ G{\isachardoublequoteclose}\ \isakeyword{shows}\ \isanewline
{\isachardoublequoteopen}setsum\ {\isacharparenleft}linearCompletion{\isacharprime}\ bids\ N\ G{\isacharparenright}\ a\ {\isacharequal}\ setsum\ snd\ {\isacharparenleft}{\isacharparenleft}LinearCompletion\ bids\ N\ G{\isacharparenright}\ {\isacharbar}{\isacharbar}\ a{\isacharparenright}{\isachardoublequoteclose}\ \isanewline
%
\isadelimproof
%
\endisadelimproof
%
\isatagproof
\isacommand{proof}\isamarkupfalse%
\ {\isacharminus}\isanewline
\isacommand{let}\isamarkupfalse%
\ {\isacharquery}l{\isacharequal}linearCompletion{\isacharprime}\ \isacommand{let}\isamarkupfalse%
\ {\isacharquery}L{\isacharequal}LinearCompletion\isanewline
\isacommand{have}\isamarkupfalse%
\ {\isachardoublequoteopen}a\ {\isasymsubseteq}\ Domain\ {\isacharparenleft}{\isacharquery}L\ bids\ N\ G{\isacharparenright}{\isachardoublequoteclose}\ \isacommand{using}\isamarkupfalse%
\ assms\ \isacommand{by}\isamarkupfalse%
\ {\isacharparenleft}rule\ mm{\isadigit{6}}{\isadigit{3}}d{\isacharparenright}\ \isacommand{then}\isamarkupfalse%
\isanewline
\isacommand{have}\isamarkupfalse%
\ {\isachardoublequoteopen}a\ {\isacharequal}\ a\ {\isasyminter}\ Domain\ {\isacharparenleft}{\isacharquery}L\ bids\ N\ G{\isacharparenright}{\isachardoublequoteclose}\ \isacommand{by}\isamarkupfalse%
\ blast\ \isacommand{then}\isamarkupfalse%
\isanewline
\isacommand{have}\isamarkupfalse%
\ {\isachardoublequoteopen}setsum\ {\isacharparenleft}{\isacharquery}l\ bids\ N\ G{\isacharparenright}\ a\ {\isacharequal}\ setsum\ {\isacharparenleft}{\isacharquery}l\ bids\ N\ G{\isacharparenright}\ {\isacharparenleft}a\ {\isasyminter}\ Domain\ {\isacharparenleft}{\isacharquery}L\ bids\ N\ G{\isacharparenright}{\isacharparenright}{\isachardoublequoteclose}\ \isacommand{by}\isamarkupfalse%
\ presburger\isanewline
\isacommand{thus}\isamarkupfalse%
\ {\isacharquery}thesis\ \isacommand{using}\isamarkupfalse%
\ mm{\isadigit{5}}{\isadigit{9}}d\ \isacommand{by}\isamarkupfalse%
\ auto\isanewline
\isacommand{qed}\isamarkupfalse%
%
\endisatagproof
{\isafoldproof}%
%
\isadelimproof
\isanewline
%
\endisadelimproof
\isanewline
\isacommand{corollary}\isamarkupfalse%
\ mm{\isadigit{5}}{\isadigit{9}}f{\isacharcolon}\ \isakeyword{assumes}\ {\isachardoublequoteopen}a\ {\isasymin}\ possibleAllocationsRel\ N\ G{\isachardoublequoteclose}\ \isakeyword{shows}\ \isanewline
{\isachardoublequoteopen}setsum\ {\isacharparenleft}linearCompletion{\isacharprime}\ bids\ N\ G{\isacharparenright}\ a\ {\isacharequal}\ setsum\ snd\ {\isacharparenleft}{\isacharparenleft}partialCompletionOf\ bids{\isacharparenright}\ {\isacharbackquote}\ {\isacharparenleft}{\isacharparenleft}N\ {\isasymtimes}\ {\isacharparenleft}Pow\ G\ {\isacharminus}\ {\isacharbraceleft}{\isacharbraceleft}{\isacharbraceright}{\isacharbraceright}{\isacharparenright}{\isacharparenright}\ {\isasyminter}\ a{\isacharparenright}{\isacharparenright}{\isachardoublequoteclose}\isanewline
{\isacharparenleft}\isakeyword{is}\ {\isachardoublequoteopen}{\isacharquery}X{\isacharequal}{\isacharquery}R{\isachardoublequoteclose}{\isacharparenright}\isanewline
%
\isadelimproof
%
\endisadelimproof
%
\isatagproof
\isacommand{proof}\isamarkupfalse%
\ {\isacharminus}\isanewline
\isacommand{let}\isamarkupfalse%
\ {\isacharquery}p{\isacharequal}partialCompletionOf\ \isacommand{let}\isamarkupfalse%
\ {\isacharquery}L{\isacharequal}LinearCompletion\ \isacommand{let}\isamarkupfalse%
\ {\isacharquery}l{\isacharequal}linearCompletion{\isacharprime}\isanewline
\isacommand{let}\isamarkupfalse%
\ {\isacharquery}A{\isacharequal}{\isachardoublequoteopen}N\ {\isasymtimes}\ {\isacharparenleft}Pow\ G\ {\isacharminus}\ {\isacharbraceleft}{\isacharbraceleft}{\isacharbraceright}{\isacharbraceright}{\isacharparenright}{\isachardoublequoteclose}\ \isacommand{let}\isamarkupfalse%
\ {\isacharquery}inner{\isadigit{2}}{\isacharequal}{\isachardoublequoteopen}{\isacharparenleft}{\isacharquery}p\ bids{\isacharparenright}{\isacharbackquote}{\isacharparenleft}{\isacharquery}A\ {\isasyminter}\ a{\isacharparenright}{\isachardoublequoteclose}\ \isacommand{let}\isamarkupfalse%
\ {\isacharquery}inner{\isadigit{1}}{\isacharequal}{\isachardoublequoteopen}{\isacharparenleft}{\isacharquery}L\ bids\ N\ G{\isacharparenright}{\isacharbar}{\isacharbar}a{\isachardoublequoteclose}\isanewline
\isacommand{have}\isamarkupfalse%
\ {\isachardoublequoteopen}{\isacharquery}R\ {\isacharequal}\ setsum\ snd\ {\isacharquery}inner{\isadigit{1}}{\isachardoublequoteclose}\ \isacommand{using}\isamarkupfalse%
\ assms\ mm{\isadigit{6}}{\isadigit{6}}d\ \isacommand{by}\isamarkupfalse%
\ {\isacharparenleft}metis\ {\isacharparenleft}no{\isacharunderscore}types{\isacharparenright}{\isacharparenright}\isanewline
\isacommand{moreover}\isamarkupfalse%
\ \isacommand{have}\isamarkupfalse%
\ {\isachardoublequoteopen}setsum\ {\isacharparenleft}{\isacharquery}l\ bids\ N\ G{\isacharparenright}\ a\ {\isacharequal}\ setsum\ snd\ {\isacharquery}inner{\isadigit{1}}{\isachardoublequoteclose}\ \isacommand{using}\isamarkupfalse%
\ assms\ \isacommand{by}\isamarkupfalse%
\ {\isacharparenleft}rule\ mm{\isadigit{5}}{\isadigit{9}}e{\isacharparenright}\isanewline
\isacommand{ultimately}\isamarkupfalse%
\ \isacommand{show}\isamarkupfalse%
\ {\isacharquery}thesis\ \isacommand{by}\isamarkupfalse%
\ presburger\isanewline
\isacommand{qed}\isamarkupfalse%
%
\endisatagproof
{\isafoldproof}%
%
\isadelimproof
\isanewline
%
\endisadelimproof
\isacommand{corollary}\isamarkupfalse%
\ mm{\isadigit{5}}{\isadigit{9}}g{\isacharcolon}\ \isakeyword{assumes}\ {\isachardoublequoteopen}a\ {\isasymin}\ possibleAllocationsRel\ N\ G{\isachardoublequoteclose}\ \isakeyword{shows}\ \isanewline
{\isachardoublequoteopen}setsum\ {\isacharparenleft}linearCompletion{\isacharprime}\ bids\ N\ G{\isacharparenright}\ a\ {\isacharequal}\ setsum\ snd\ {\isacharparenleft}{\isacharparenleft}partialCompletionOf\ bids{\isacharparenright}\ {\isacharbackquote}\ a{\isacharparenright}{\isachardoublequoteclose}\ {\isacharparenleft}\isakeyword{is}\ {\isachardoublequoteopen}{\isacharquery}L{\isacharequal}{\isacharquery}R{\isachardoublequoteclose}{\isacharparenright}\isanewline
%
\isadelimproof
%
\endisadelimproof
%
\isatagproof
\isacommand{using}\isamarkupfalse%
\ assms\ mm{\isadigit{5}}{\isadigit{9}}f\ mm{\isadigit{6}}{\isadigit{3}}c\ \isanewline
\isacommand{proof}\isamarkupfalse%
\ {\isacharminus}\isanewline
\isacommand{let}\isamarkupfalse%
\ {\isacharquery}p{\isacharequal}partialCompletionOf\ \isacommand{let}\isamarkupfalse%
\ {\isacharquery}l{\isacharequal}linearCompletion{\isacharprime}\isanewline
\isacommand{have}\isamarkupfalse%
\ {\isachardoublequoteopen}{\isacharquery}L\ {\isacharequal}\ setsum\ snd\ {\isacharparenleft}{\isacharparenleft}{\isacharquery}p\ bids{\isacharparenright}{\isacharbackquote}{\isacharparenleft}{\isacharparenleft}N\ {\isasymtimes}\ {\isacharparenleft}Pow\ G\ {\isacharminus}\ {\isacharbraceleft}{\isacharbraceleft}{\isacharbraceright}{\isacharbraceright}{\isacharparenright}{\isacharparenright}{\isasyminter}\ a{\isacharparenright}{\isacharparenright}{\isachardoublequoteclose}\ \isacommand{using}\isamarkupfalse%
\ assms\ \isacommand{by}\isamarkupfalse%
\ {\isacharparenleft}rule\ mm{\isadigit{5}}{\isadigit{9}}f{\isacharparenright}\isanewline
\isacommand{moreover}\isamarkupfalse%
\ \isacommand{have}\isamarkupfalse%
\ {\isachardoublequoteopen}{\isachardot}{\isachardot}{\isachardot}\ {\isacharequal}\ {\isacharquery}R{\isachardoublequoteclose}\ \isacommand{using}\isamarkupfalse%
\ assms\ mm{\isadigit{6}}{\isadigit{3}}c\ Int{\isacharunderscore}absorb{\isadigit{1}}\ \isacommand{by}\isamarkupfalse%
\ {\isacharparenleft}metis\ {\isacharparenleft}no{\isacharunderscore}types{\isacharparenright}{\isacharparenright}\isanewline
\isacommand{ultimately}\isamarkupfalse%
\ \isacommand{show}\isamarkupfalse%
\ {\isacharquery}thesis\ \isacommand{by}\isamarkupfalse%
\ presburger\isanewline
\isacommand{qed}\isamarkupfalse%
%
\endisatagproof
{\isafoldproof}%
%
\isadelimproof
\isanewline
%
\endisadelimproof
\isacommand{corollary}\isamarkupfalse%
\ mm{\isadigit{5}}{\isadigit{7}}c{\isacharcolon}\ {\isachardoublequoteopen}setsum\ snd\ {\isacharparenleft}{\isacharparenleft}partialCompletionOf\ bids{\isacharparenright}\ {\isacharbackquote}\ a{\isacharparenright}\ {\isacharequal}\ setsum\ {\isacharparenleft}snd\ {\isasymcirc}\ {\isacharparenleft}partialCompletionOf\ bids{\isacharparenright}{\isacharparenright}\ a{\isachardoublequoteclose}\isanewline
%
\isadelimproof
%
\endisadelimproof
%
\isatagproof
\isacommand{using}\isamarkupfalse%
\ assms\ setsum{\isachardot}reindex\ mm{\isadigit{5}}{\isadigit{7}}b\ \isacommand{by}\isamarkupfalse%
\ blast%
\endisatagproof
{\isafoldproof}%
%
\isadelimproof
\isanewline
%
\endisadelimproof
\isacommand{corollary}\isamarkupfalse%
\ mm{\isadigit{5}}{\isadigit{9}}h{\isacharcolon}\ \isakeyword{assumes}\ {\isachardoublequoteopen}a\ {\isasymin}\ possibleAllocationsRel\ N\ G{\isachardoublequoteclose}\ \isakeyword{shows}\ \isanewline
{\isachardoublequoteopen}setsum\ {\isacharparenleft}linearCompletion{\isacharprime}\ bids\ N\ G{\isacharparenright}\ a\ {\isacharequal}\ setsum\ {\isacharparenleft}snd\ {\isasymcirc}\ {\isacharparenleft}partialCompletionOf\ bids{\isacharparenright}{\isacharparenright}\ a{\isachardoublequoteclose}\ {\isacharparenleft}\isakeyword{is}\ {\isachardoublequoteopen}{\isacharquery}L{\isacharequal}{\isacharquery}R{\isachardoublequoteclose}{\isacharparenright}\isanewline
%
\isadelimproof
%
\endisadelimproof
%
\isatagproof
\isacommand{using}\isamarkupfalse%
\ assms\ mm{\isadigit{5}}{\isadigit{9}}g\ mm{\isadigit{5}}{\isadigit{7}}c\ \isanewline
\isacommand{proof}\isamarkupfalse%
\ {\isacharminus}\isanewline
\isacommand{let}\isamarkupfalse%
\ {\isacharquery}p{\isacharequal}partialCompletionOf\ \isacommand{let}\isamarkupfalse%
\ {\isacharquery}l{\isacharequal}linearCompletion{\isacharprime}\isanewline
\isacommand{have}\isamarkupfalse%
\ {\isachardoublequoteopen}{\isacharquery}L\ {\isacharequal}\ setsum\ snd\ {\isacharparenleft}{\isacharparenleft}{\isacharquery}p\ bids{\isacharparenright}{\isacharbackquote}\ a{\isacharparenright}{\isachardoublequoteclose}\ \isacommand{using}\isamarkupfalse%
\ assms\ \isacommand{by}\isamarkupfalse%
\ {\isacharparenleft}rule\ mm{\isadigit{5}}{\isadigit{9}}g{\isacharparenright}\isanewline
\isacommand{moreover}\isamarkupfalse%
\ \isacommand{have}\isamarkupfalse%
\ {\isachardoublequoteopen}{\isachardot}{\isachardot}{\isachardot}\ {\isacharequal}\ {\isacharquery}R{\isachardoublequoteclose}\ \isacommand{using}\isamarkupfalse%
\ assms\ mm{\isadigit{5}}{\isadigit{7}}c\ \isacommand{by}\isamarkupfalse%
\ blast\isanewline
\isacommand{ultimately}\isamarkupfalse%
\ \isacommand{show}\isamarkupfalse%
\ {\isacharquery}thesis\ \isacommand{by}\isamarkupfalse%
\ presburger\isanewline
\isacommand{qed}\isamarkupfalse%
%
\endisatagproof
{\isafoldproof}%
%
\isadelimproof
\isanewline
%
\endisadelimproof
\isacommand{corollary}\isamarkupfalse%
\ mm{\isadigit{2}}{\isadigit{5}}c{\isacharcolon}\ \isakeyword{assumes}\ {\isachardoublequoteopen}a\ {\isasymin}\ possibleAllocationsRel\ N\ G{\isachardoublequoteclose}\ \isakeyword{shows}\ \isanewline
{\isachardoublequoteopen}setsum\ {\isacharparenleft}linearCompletion{\isacharprime}\ bids\ N\ G{\isacharparenright}\ a\ {\isacharequal}\ setsum\ {\isacharparenleft}{\isacharparenleft}setsum\ bids{\isacharparenright}\ {\isasymcirc}\ omega{\isacharparenright}\ a{\isachardoublequoteclose}\ {\isacharparenleft}\isakeyword{is}\ {\isachardoublequoteopen}{\isacharquery}L{\isacharequal}{\isacharquery}R{\isachardoublequoteclose}{\isacharparenright}\ \isanewline
%
\isadelimproof
%
\endisadelimproof
%
\isatagproof
\isacommand{using}\isamarkupfalse%
\ assms\ mm{\isadigit{5}}{\isadigit{9}}h\ mm{\isadigit{2}}{\isadigit{5}}\ \isanewline
\isacommand{proof}\isamarkupfalse%
\ {\isacharminus}\isanewline
\isacommand{let}\isamarkupfalse%
\ {\isacharquery}inner{\isadigit{1}}{\isacharequal}{\isachardoublequoteopen}snd\ {\isasymcirc}\ {\isacharparenleft}partialCompletionOf\ bids{\isacharparenright}{\isachardoublequoteclose}\ \isacommand{let}\isamarkupfalse%
\ {\isacharquery}inner{\isadigit{2}}{\isacharequal}{\isachardoublequoteopen}{\isacharparenleft}setsum\ bids{\isacharparenright}\ {\isasymcirc}\ omega{\isachardoublequoteclose}\isanewline
\isacommand{let}\isamarkupfalse%
\ {\isacharquery}M{\isacharequal}{\isachardoublequoteopen}setsum\ {\isacharquery}inner{\isadigit{1}}\ a{\isachardoublequoteclose}\isanewline
\isacommand{have}\isamarkupfalse%
\ {\isachardoublequoteopen}{\isacharquery}L\ {\isacharequal}\ {\isacharquery}M{\isachardoublequoteclose}\ \isacommand{using}\isamarkupfalse%
\ assms\ \isacommand{by}\isamarkupfalse%
\ {\isacharparenleft}rule\ mm{\isadigit{5}}{\isadigit{9}}h{\isacharparenright}\isanewline
\isacommand{moreover}\isamarkupfalse%
\ \isacommand{have}\isamarkupfalse%
\ {\isachardoublequoteopen}{\isacharquery}inner{\isadigit{1}}\ {\isacharequal}\ {\isacharquery}inner{\isadigit{2}}{\isachardoublequoteclose}\ \isacommand{using}\isamarkupfalse%
\ mm{\isadigit{2}}{\isadigit{5}}\ assms\ \isacommand{by}\isamarkupfalse%
\ fastforce\isanewline
\isacommand{ultimately}\isamarkupfalse%
\ \isacommand{show}\isamarkupfalse%
\ {\isachardoublequoteopen}{\isacharquery}L\ {\isacharequal}\ {\isacharquery}R{\isachardoublequoteclose}\ \isacommand{using}\isamarkupfalse%
\ assms\ \isacommand{by}\isamarkupfalse%
\ metis\isanewline
\isacommand{qed}\isamarkupfalse%
%
\endisatagproof
{\isafoldproof}%
%
\isadelimproof
\isanewline
%
\endisadelimproof
\isanewline
\isacommand{corollary}\isamarkupfalse%
\ mm{\isadigit{2}}{\isadigit{5}}d{\isacharcolon}\ \isakeyword{assumes}\ {\isachardoublequoteopen}a\ {\isasymin}\ possibleAllocationsRel\ N\ G{\isachardoublequoteclose}\ \isakeyword{shows}\isanewline
{\isachardoublequoteopen}setsum\ {\isacharparenleft}linearCompletion{\isacharprime}\ bids\ N\ G{\isacharparenright}\ a\ {\isacharequal}\ setsum\ {\isacharparenleft}setsum\ bids{\isacharparenright}\ {\isacharparenleft}omega{\isacharbackquote}\ a{\isacharparenright}{\isachardoublequoteclose}\isanewline
%
\isadelimproof
%
\endisadelimproof
%
\isatagproof
\isacommand{using}\isamarkupfalse%
\ assms\ mm{\isadigit{2}}{\isadigit{5}}c\ setsum{\isachardot}reindex\ mm{\isadigit{3}}{\isadigit{2}}\ \isanewline
\isacommand{proof}\isamarkupfalse%
\ {\isacharminus}\isanewline
\isacommand{have}\isamarkupfalse%
\ {\isachardoublequoteopen}{\isacharbraceleft}{\isacharbraceright}\ {\isasymnotin}\ Range\ a{\isachardoublequoteclose}\ \isacommand{using}\isamarkupfalse%
\ assms\ \isacommand{by}\isamarkupfalse%
\ {\isacharparenleft}metis\ mm{\isadigit{4}}{\isadigit{5}}b{\isacharparenright}\isanewline
\isacommand{then}\isamarkupfalse%
\ \isacommand{have}\isamarkupfalse%
\ {\isachardoublequoteopen}inj{\isacharunderscore}on\ omega\ a{\isachardoublequoteclose}\ \isacommand{using}\isamarkupfalse%
\ mm{\isadigit{3}}{\isadigit{2}}\ \isacommand{by}\isamarkupfalse%
\ blast\isanewline
\isacommand{then}\isamarkupfalse%
\ \isacommand{have}\isamarkupfalse%
\ {\isachardoublequoteopen}setsum\ {\isacharparenleft}setsum\ bids{\isacharparenright}\ {\isacharparenleft}omega\ {\isacharbackquote}\ a{\isacharparenright}\ {\isacharequal}\ setsum\ {\isacharparenleft}{\isacharparenleft}setsum\ bids{\isacharparenright}\ {\isasymcirc}\ omega{\isacharparenright}\ a{\isachardoublequoteclose}\ \isanewline
\isacommand{by}\isamarkupfalse%
\ {\isacharparenleft}rule\ setsum{\isachardot}reindex{\isacharparenright}\isanewline
\isacommand{moreover}\isamarkupfalse%
\ \isacommand{have}\isamarkupfalse%
\ {\isachardoublequoteopen}setsum\ {\isacharparenleft}linearCompletion{\isacharprime}\ bids\ N\ G{\isacharparenright}\ a\ {\isacharequal}\ setsum\ {\isacharparenleft}{\isacharparenleft}setsum\ bids{\isacharparenright}\ {\isasymcirc}\ omega{\isacharparenright}\ a{\isachardoublequoteclose}\isanewline
\isacommand{using}\isamarkupfalse%
\ assms\ mm{\isadigit{2}}{\isadigit{5}}c\ \isacommand{by}\isamarkupfalse%
\ {\isacharparenleft}rule\ Extraction{\isachardot}exE{\isacharunderscore}realizer{\isacharparenright}\isanewline
\isacommand{ultimately}\isamarkupfalse%
\ \isacommand{show}\isamarkupfalse%
\ {\isacharquery}thesis\ \isacommand{by}\isamarkupfalse%
\ presburger\isanewline
\isacommand{qed}\isamarkupfalse%
%
\endisatagproof
{\isafoldproof}%
%
\isadelimproof
\isanewline
%
\endisadelimproof
\isanewline
\isacommand{lemma}\isamarkupfalse%
\ mm{\isadigit{6}}{\isadigit{7}}{\isacharcolon}\ \isakeyword{assumes}\ {\isachardoublequoteopen}finite\ {\isacharparenleft}snd\ pair{\isacharparenright}{\isachardoublequoteclose}\ \isakeyword{shows}\ {\isachardoublequoteopen}finite\ {\isacharparenleft}omega\ pair{\isacharparenright}{\isachardoublequoteclose}%
\isadelimproof
\ %
\endisadelimproof
%
\isatagproof
\isacommand{using}\isamarkupfalse%
\ assms\ \isanewline
\isacommand{by}\isamarkupfalse%
\ {\isacharparenleft}metis\ finite{\isachardot}emptyI\ finite{\isachardot}insertI\ finite{\isacharunderscore}SigmaI\ mm{\isadigit{4}}{\isadigit{0}}{\isacharparenright}%
\endisatagproof
{\isafoldproof}%
%
\isadelimproof
%
\endisadelimproof
\isanewline
\isacommand{corollary}\isamarkupfalse%
\ mm{\isadigit{6}}{\isadigit{7}}b{\isacharcolon}\ \isakeyword{assumes}\ {\isachardoublequoteopen}{\isasymforall}y{\isasymin}{\isacharparenleft}Range\ a{\isacharparenright}{\isachardot}\ finite\ y{\isachardoublequoteclose}\ \isakeyword{shows}\ {\isachardoublequoteopen}{\isasymforall}y{\isasymin}{\isacharparenleft}omega\ {\isacharbackquote}\ a{\isacharparenright}{\isachardot}\ finite\ y{\isachardoublequoteclose}\isanewline
%
\isadelimproof
%
\endisadelimproof
%
\isatagproof
\isacommand{using}\isamarkupfalse%
\ assms\ mm{\isadigit{6}}{\isadigit{7}}\ imageE\ mm{\isadigit{4}}{\isadigit{7}}\ \isacommand{by}\isamarkupfalse%
\ fast%
\endisatagproof
{\isafoldproof}%
%
\isadelimproof
\isanewline
%
\endisadelimproof
\isacommand{lemma}\isamarkupfalse%
\ \isakeyword{assumes}\ {\isachardoublequoteopen}a\ {\isasymin}\ possibleAllocationsRel\ N\ G{\isachardoublequoteclose}\ {\isachardoublequoteopen}finite\ G{\isachardoublequoteclose}\ \isakeyword{shows}\ {\isachardoublequoteopen}{\isasymforall}y\ {\isasymin}\ {\isacharparenleft}Range\ a{\isacharparenright}{\isachardot}\ finite\ y{\isachardoublequoteclose}\isanewline
%
\isadelimproof
%
\endisadelimproof
%
\isatagproof
\isacommand{using}\isamarkupfalse%
\ assms\ \isacommand{by}\isamarkupfalse%
\ {\isacharparenleft}metis\ mm{\isadigit{4}}{\isadigit{1}}{\isacharparenright}%
\endisatagproof
{\isafoldproof}%
%
\isadelimproof
\isanewline
%
\endisadelimproof
\isanewline
\isacommand{corollary}\isamarkupfalse%
\ mm{\isadigit{6}}{\isadigit{7}}c{\isacharcolon}\ \isakeyword{assumes}\ {\isachardoublequoteopen}a\ {\isasymin}\ possibleAllocationsRel\ N\ G{\isachardoublequoteclose}\ {\isachardoublequoteopen}finite\ G{\isachardoublequoteclose}\ \isakeyword{shows}\ {\isachardoublequoteopen}{\isasymforall}x{\isasymin}{\isacharparenleft}omega\ {\isacharbackquote}\ a{\isacharparenright}{\isachardot}\ finite\ x{\isachardoublequoteclose}\ \isanewline
%
\isadelimproof
%
\endisadelimproof
%
\isatagproof
\isacommand{using}\isamarkupfalse%
\ assms\ mm{\isadigit{6}}{\isadigit{7}}b\ mm{\isadigit{4}}{\isadigit{1}}\ \isacommand{by}\isamarkupfalse%
\ {\isacharparenleft}metis{\isacharparenleft}no{\isacharunderscore}types{\isacharparenright}{\isacharparenright}%
\endisatagproof
{\isafoldproof}%
%
\isadelimproof
\isanewline
%
\endisadelimproof
\isanewline
\isacommand{corollary}\isamarkupfalse%
\ mm{\isadigit{3}}{\isadigit{0}}b{\isacharcolon}\ \isakeyword{assumes}\ {\isachardoublequoteopen}a\ {\isasymin}\ possibleAllocationsRel\ N\ G{\isachardoublequoteclose}\ \isakeyword{shows}\ {\isachardoublequoteopen}is{\isacharunderscore}partition\ {\isacharparenleft}omega\ {\isacharbackquote}\ a{\isacharparenright}{\isachardoublequoteclose}\isanewline
%
\isadelimproof
%
\endisadelimproof
%
\isatagproof
\isacommand{using}\isamarkupfalse%
\ assms\ mm{\isadigit{3}}{\isadigit{0}}\ mm{\isadigit{4}}{\isadigit{5}}b\ image{\isacharunderscore}iff\ lll{\isadigit{8}}{\isadigit{1}}a\ \isanewline
\isacommand{proof}\isamarkupfalse%
\ {\isacharminus}\isanewline
\ \ \isacommand{have}\isamarkupfalse%
\ {\isachardoublequoteopen}runiq\ a{\isachardoublequoteclose}\ \isacommand{by}\isamarkupfalse%
\ {\isacharparenleft}metis\ {\isacharparenleft}no{\isacharunderscore}types{\isacharparenright}\ assms\ image{\isacharunderscore}iff\ lll{\isadigit{8}}{\isadigit{1}}a{\isacharparenright}\isanewline
\ \ \isacommand{moreover}\isamarkupfalse%
\ \isacommand{have}\isamarkupfalse%
\ {\isachardoublequoteopen}{\isacharbraceleft}{\isacharbraceright}\ {\isasymnotin}\ Range\ a{\isachardoublequoteclose}\ \isacommand{using}\isamarkupfalse%
\ assms\ mm{\isadigit{4}}{\isadigit{5}}b\ \isacommand{by}\isamarkupfalse%
\ blast\isanewline
\ \ \isacommand{ultimately}\isamarkupfalse%
\ \isacommand{show}\isamarkupfalse%
\ {\isacharquery}thesis\ \isacommand{using}\isamarkupfalse%
\ mm{\isadigit{3}}{\isadigit{0}}\ \isacommand{by}\isamarkupfalse%
\ blast\isanewline
\isacommand{qed}\isamarkupfalse%
%
\endisatagproof
{\isafoldproof}%
%
\isadelimproof
\isanewline
%
\endisadelimproof
\isanewline
\isacommand{lemma}\isamarkupfalse%
\ mm{\isadigit{6}}{\isadigit{8}}{\isacharcolon}\ \isakeyword{assumes}\ {\isachardoublequoteopen}a\ {\isasymin}\ possibleAllocationsRel\ N\ G{\isachardoublequoteclose}\ {\isachardoublequoteopen}finite\ G{\isachardoublequoteclose}\ \isakeyword{shows}\ \isanewline
{\isachardoublequoteopen}setsum\ {\isacharparenleft}setsum\ bids{\isacharparenright}\ {\isacharparenleft}omega{\isacharbackquote}\ a{\isacharparenright}\ {\isacharequal}\ setsum\ bids\ {\isacharparenleft}{\isasymUnion}\ {\isacharparenleft}omega\ {\isacharbackquote}\ a{\isacharparenright}{\isacharparenright}\ {\isachardoublequoteclose}\ \isanewline
%
\isadelimproof
%
\endisadelimproof
%
\isatagproof
\isacommand{using}\isamarkupfalse%
\ assms\ setsum{\isacharunderscore}Union{\isacharunderscore}disjoint{\isacharunderscore}{\isadigit{2}}\ mm{\isadigit{3}}{\isadigit{0}}b\ mm{\isadigit{6}}{\isadigit{7}}c\ \isacommand{by}\isamarkupfalse%
\ {\isacharparenleft}metis\ {\isacharparenleft}lifting{\isacharcomma}\ mono{\isacharunderscore}tags{\isacharparenright}{\isacharparenright}%
\endisatagproof
{\isafoldproof}%
%
\isadelimproof
\isanewline
%
\endisadelimproof
\isanewline
\isacommand{corollary}\isamarkupfalse%
\ mm{\isadigit{6}}{\isadigit{9}}{\isacharcolon}\ \isakeyword{assumes}\ {\isachardoublequoteopen}a\ {\isasymin}\ possibleAllocationsRel\ N\ G{\isachardoublequoteclose}\ {\isachardoublequoteopen}finite\ G{\isachardoublequoteclose}\ \isakeyword{shows}\ \isanewline
{\isachardoublequoteopen}setsum\ {\isacharparenleft}linearCompletion{\isacharprime}\ bids\ N\ G{\isacharparenright}\ a\ {\isacharequal}\ setsum\ bids\ {\isacharparenleft}pseudoAllocation\ a{\isacharparenright}{\isachardoublequoteclose}\ {\isacharparenleft}\isakeyword{is}\ {\isachardoublequoteopen}{\isacharquery}L\ {\isacharequal}\ {\isacharquery}R{\isachardoublequoteclose}{\isacharparenright}\isanewline
%
\isadelimproof
%
\endisadelimproof
%
\isatagproof
\isacommand{using}\isamarkupfalse%
\ assms\ mm{\isadigit{2}}{\isadigit{5}}d\ mm{\isadigit{6}}{\isadigit{8}}\ \isanewline
\isacommand{proof}\isamarkupfalse%
\ {\isacharminus}\isanewline
\isacommand{have}\isamarkupfalse%
\ {\isachardoublequoteopen}{\isacharquery}L\ {\isacharequal}\ setsum\ {\isacharparenleft}setsum\ bids{\isacharparenright}\ {\isacharparenleft}omega\ {\isacharbackquote}a{\isacharparenright}{\isachardoublequoteclose}\ \isacommand{using}\isamarkupfalse%
\ assms\ mm{\isadigit{2}}{\isadigit{5}}d\ \isacommand{by}\isamarkupfalse%
\ blast\isanewline
\isacommand{moreover}\isamarkupfalse%
\ \isacommand{have}\isamarkupfalse%
\ {\isachardoublequoteopen}{\isachardot}{\isachardot}{\isachardot}\ {\isacharequal}\ setsum\ bids\ {\isacharparenleft}{\isasymUnion}\ {\isacharparenleft}omega\ {\isacharbackquote}\ a{\isacharparenright}{\isacharparenright}{\isachardoublequoteclose}\ \isacommand{using}\isamarkupfalse%
\ assms\ mm{\isadigit{6}}{\isadigit{8}}\ \isacommand{by}\isamarkupfalse%
\ blast\isanewline
\isacommand{ultimately}\isamarkupfalse%
\ \isacommand{show}\isamarkupfalse%
\ {\isacharquery}thesis\ \isacommand{by}\isamarkupfalse%
\ presburger\isanewline
\isacommand{qed}\isamarkupfalse%
%
\endisatagproof
{\isafoldproof}%
%
\isadelimproof
\isanewline
%
\endisadelimproof
\isanewline
\isacommand{lemma}\isamarkupfalse%
\ mm{\isadigit{7}}{\isadigit{3}}{\isacharcolon}\ {\isachardoublequoteopen}Domain\ {\isacharparenleft}pseudoAllocation\ a{\isacharparenright}\ {\isasymsubseteq}\ Domain\ a{\isachardoublequoteclose}%
\isadelimproof
\ %
\endisadelimproof
%
\isatagproof
\isacommand{by}\isamarkupfalse%
\ auto%
\endisatagproof
{\isafoldproof}%
%
\isadelimproof
%
\endisadelimproof
\ \isanewline
\isacommand{corollary}\isamarkupfalse%
\ \isakeyword{assumes}\ {\isachardoublequoteopen}a\ {\isasymin}\ possibleAllocationsRel\ N\ G{\isachardoublequoteclose}\ \isakeyword{shows}\ {\isachardoublequoteopen}{\isasymUnion}\ Range\ a\ {\isacharequal}\ G{\isachardoublequoteclose}%
\isadelimproof
\ %
\endisadelimproof
%
\isatagproof
\isacommand{using}\isamarkupfalse%
\ assms\ lm{\isadigit{4}}{\isadigit{7}}\ \isanewline
is{\isacharunderscore}partition{\isacharunderscore}of{\isacharunderscore}def\ \isacommand{by}\isamarkupfalse%
\ metis%
\endisatagproof
{\isafoldproof}%
%
\isadelimproof
%
\endisadelimproof
\isanewline
\isacommand{corollary}\isamarkupfalse%
\ mm{\isadigit{7}}{\isadigit{2}}{\isacharcolon}\ \isakeyword{assumes}\ {\isachardoublequoteopen}a\ {\isasymin}\ possibleAllocationsRel\ N\ G{\isachardoublequoteclose}\ \isakeyword{shows}\ {\isachardoublequoteopen}Range\ {\isacharparenleft}pseudoAllocation\ a{\isacharparenright}\ {\isacharequal}\ finestpart\ G{\isachardoublequoteclose}\ \isanewline
%
\isadelimproof
%
\endisadelimproof
%
\isatagproof
\isacommand{using}\isamarkupfalse%
\ assms\ mm{\isadigit{5}}{\isadigit{5}}s\ lm{\isadigit{4}}{\isadigit{7}}\ is{\isacharunderscore}partition{\isacharunderscore}of{\isacharunderscore}def\ \isacommand{by}\isamarkupfalse%
\ metis%
\endisatagproof
{\isafoldproof}%
%
\isadelimproof
\isanewline
%
\endisadelimproof
\isacommand{corollary}\isamarkupfalse%
\ mm{\isadigit{7}}{\isadigit{3}}b{\isacharcolon}\ \isakeyword{assumes}\ {\isachardoublequoteopen}a\ {\isasymin}\ possibleAllocationsRel\ N\ G{\isachardoublequoteclose}\ \isakeyword{shows}\ {\isachardoublequoteopen}Domain\ {\isacharparenleft}pseudoAllocation\ a{\isacharparenright}\ {\isasymsubseteq}\ N\ {\isacharampersand}\ \isanewline
Range\ {\isacharparenleft}pseudoAllocation\ a{\isacharparenright}\ {\isacharequal}\ finestpart\ G{\isachardoublequoteclose}\ \isanewline
%
\isadelimproof
%
\endisadelimproof
%
\isatagproof
\isacommand{using}\isamarkupfalse%
\ assms\ mm{\isadigit{7}}{\isadigit{3}}\ lm{\isadigit{4}}{\isadigit{7}}\ mm{\isadigit{5}}{\isadigit{5}}s\ is{\isacharunderscore}partition{\isacharunderscore}of{\isacharunderscore}def\ subset{\isacharunderscore}trans\ \isacommand{by}\isamarkupfalse%
\ {\isacharparenleft}metis{\isacharparenleft}no{\isacharunderscore}types{\isacharparenright}{\isacharparenright}%
\endisatagproof
{\isafoldproof}%
%
\isadelimproof
\isanewline
%
\endisadelimproof
\isacommand{corollary}\isamarkupfalse%
\ mm{\isadigit{7}}{\isadigit{3}}c{\isacharcolon}\ \isakeyword{assumes}\ {\isachardoublequoteopen}a\ {\isasymin}\ possibleAllocationsRel\ N\ G{\isachardoublequoteclose}\ \isanewline
\isakeyword{shows}\ {\isachardoublequoteopen}pseudoAllocation\ a\ {\isasymsubseteq}\ N\ {\isasymtimes}\ finestpart\ G{\isachardoublequoteclose}%
\isadelimproof
\ %
\endisadelimproof
%
\isatagproof
\isacommand{using}\isamarkupfalse%
\ assms\ mm{\isadigit{7}}{\isadigit{3}}b\ \isanewline
\isacommand{proof}\isamarkupfalse%
\ {\isacharminus}\isanewline
\isacommand{let}\isamarkupfalse%
\ {\isacharquery}p{\isacharequal}pseudoAllocation\ \isacommand{let}\isamarkupfalse%
\ {\isacharquery}aa{\isacharequal}{\isachardoublequoteopen}{\isacharquery}p\ a{\isachardoublequoteclose}\ \isacommand{let}\isamarkupfalse%
\ {\isacharquery}d{\isacharequal}Domain\ \isacommand{let}\isamarkupfalse%
\ {\isacharquery}r{\isacharequal}Range\isanewline
\isacommand{have}\isamarkupfalse%
\ {\isachardoublequoteopen}{\isacharquery}d\ {\isacharquery}aa\ {\isasymsubseteq}\ N{\isachardoublequoteclose}\ \isacommand{using}\isamarkupfalse%
\ assms\ mm{\isadigit{7}}{\isadigit{3}}b\ \isacommand{by}\isamarkupfalse%
\ {\isacharparenleft}metis\ {\isacharparenleft}lifting{\isacharcomma}\ mono{\isacharunderscore}tags{\isacharparenright}{\isacharparenright}\isanewline
\isacommand{moreover}\isamarkupfalse%
\ \isacommand{have}\isamarkupfalse%
\ {\isachardoublequoteopen}{\isacharquery}r\ {\isacharquery}aa\ {\isasymsubseteq}\ finestpart\ G{\isachardoublequoteclose}\ \isacommand{using}\isamarkupfalse%
\ assms\ mm{\isadigit{7}}{\isadigit{3}}b\ \isacommand{by}\isamarkupfalse%
\ {\isacharparenleft}metis\ {\isacharparenleft}lifting{\isacharcomma}\ mono{\isacharunderscore}tags{\isacharparenright}\ equalityE{\isacharparenright}\isanewline
\isacommand{ultimately}\isamarkupfalse%
\ \isacommand{have}\isamarkupfalse%
\ {\isachardoublequoteopen}{\isacharquery}d\ {\isacharquery}aa\ {\isasymtimes}\ {\isacharparenleft}{\isacharquery}r\ {\isacharquery}aa{\isacharparenright}\ {\isasymsubseteq}\ N\ {\isasymtimes}\ finestpart\ G{\isachardoublequoteclose}\ \isacommand{by}\isamarkupfalse%
\ auto\isanewline
\isacommand{then}\isamarkupfalse%
\ \isacommand{show}\isamarkupfalse%
\ {\isachardoublequoteopen}{\isacharquery}aa\ {\isasymsubseteq}\ N\ {\isasymtimes}\ finestpart\ G{\isachardoublequoteclose}\ \isacommand{by}\isamarkupfalse%
\ auto\isanewline
\isacommand{qed}\isamarkupfalse%
%
\endisatagproof
{\isafoldproof}%
%
\isadelimproof
%
\endisadelimproof
\isanewline
\isanewline
\isanewline
\isanewline
\isanewline
\isanewline
\isanewline
\isanewline
\isanewline
\isanewline
\isanewline
\isanewline
\isanewline
\isanewline
\isanewline
\isanewline
\isanewline
\isanewline
\isanewline
\isanewline
\isanewline
\isanewline
\isanewline
\isanewline
\isanewline
\isanewline
\isanewline
\isanewline
\isanewline
\isanewline
\isanewline
\isanewline
\isanewline
\isanewline
\isanewline
\isacommand{abbreviation}\isamarkupfalse%
\ {\isachardoublequoteopen}mbc\ pseudo\ {\isacharequal}{\isacharequal}\ {\isacharbraceleft}{\isacharparenleft}x{\isacharcomma}\ {\isasymUnion}\ {\isacharparenleft}pseudo\ {\isacharbackquote}{\isacharbackquote}\ {\isacharbraceleft}x{\isacharbraceright}{\isacharparenright}{\isacharparenright}{\isacharbar}\ x{\isachardot}\ x\ {\isasymin}\ Domain\ pseudo{\isacharbraceright}{\isachardoublequoteclose}\isanewline
\ \isanewline
\isacommand{corollary}\isamarkupfalse%
\ \isakeyword{assumes}\ {\isachardoublequoteopen}{\isacharbraceleft}{\isacharbraceright}\ {\isasymnotin}\ Range\ X{\isachardoublequoteclose}\ \isakeyword{shows}\ {\isachardoublequoteopen}inj{\isacharunderscore}on\ {\isacharparenleft}image\ omega{\isacharparenright}\ {\isacharparenleft}Pow\ X{\isacharparenright}{\isachardoublequoteclose}%
\isadelimproof
\ %
\endisadelimproof
%
\isatagproof
\isacommand{using}\isamarkupfalse%
\ assms\ mm{\isadigit{7}}{\isadigit{4}}\ mm{\isadigit{3}}{\isadigit{2}}\ \isacommand{by}\isamarkupfalse%
\ blast%
\endisatagproof
{\isafoldproof}%
%
\isadelimproof
%
\endisadelimproof
\isanewline
\isanewline
\isacommand{lemma}\isamarkupfalse%
\ {\isachardoublequoteopen}pseudoAllocation\ {\isacharequal}\ Union\ {\isasymcirc}\ {\isacharparenleft}image\ omega{\isacharparenright}{\isachardoublequoteclose}%
\isadelimproof
\ %
\endisadelimproof
%
\isatagproof
\isacommand{by}\isamarkupfalse%
\ force%
\endisatagproof
{\isafoldproof}%
%
\isadelimproof
%
\endisadelimproof
\isanewline
\isacommand{lemma}\isamarkupfalse%
\ mm{\isadigit{7}}{\isadigit{5}}d{\isacharcolon}\ \isakeyword{assumes}\ {\isachardoublequoteopen}runiq\ a{\isachardoublequoteclose}\ {\isachardoublequoteopen}{\isacharbraceleft}{\isacharbraceright}\ {\isasymnotin}\ Range\ a{\isachardoublequoteclose}\ \isakeyword{shows}\ \isanewline
{\isachardoublequoteopen}a\ {\isacharequal}\ mbc\ {\isacharparenleft}pseudoAllocation\ a{\isacharparenright}{\isachardoublequoteclose}\isanewline
%
\isadelimproof
%
\endisadelimproof
%
\isatagproof
\isacommand{proof}\isamarkupfalse%
\ {\isacharminus}\isanewline
\isacommand{let}\isamarkupfalse%
\ {\isacharquery}p{\isacharequal}{\isachardoublequoteopen}{\isacharbraceleft}{\isacharparenleft}x{\isacharcomma}\ Y{\isacharparenright}{\isacharbar}\ Y\ x{\isachardot}\ Y\ {\isasymin}\ finestpart\ {\isacharparenleft}a{\isacharcomma}{\isacharcomma}x{\isacharparenright}\ {\isacharampersand}\ x\ {\isasymin}\ Domain\ a{\isacharbraceright}{\isachardoublequoteclose}\isanewline
\isacommand{let}\isamarkupfalse%
\ {\isacharquery}a{\isacharequal}{\isachardoublequoteopen}{\isacharbraceleft}{\isacharparenleft}x{\isacharcomma}\ {\isasymUnion}\ {\isacharparenleft}{\isacharquery}p\ {\isacharbackquote}{\isacharbackquote}\ {\isacharbraceleft}x{\isacharbraceright}{\isacharparenright}{\isacharparenright}{\isacharbar}\ x{\isachardot}\ x\ {\isasymin}\ Domain\ {\isacharquery}p{\isacharbraceright}{\isachardoublequoteclose}\ \isanewline
\isacommand{have}\isamarkupfalse%
\ {\isachardoublequoteopen}{\isasymforall}x\ {\isasymin}\ Domain\ a{\isachardot}\ a{\isacharcomma}{\isacharcomma}x\ {\isasymnoteq}\ {\isacharbraceleft}{\isacharbraceright}{\isachardoublequoteclose}\ \isacommand{by}\isamarkupfalse%
\ {\isacharparenleft}metis\ assms\ ll{\isadigit{1}}{\isadigit{4}}{\isacharparenright}\isanewline
\isacommand{then}\isamarkupfalse%
\ \isacommand{have}\isamarkupfalse%
\ {\isachardoublequoteopen}{\isasymforall}x\ {\isasymin}\ Domain\ a{\isachardot}\ finestpart\ {\isacharparenleft}a{\isacharcomma}{\isacharcomma}x{\isacharparenright}\ {\isasymnoteq}\ {\isacharbraceleft}{\isacharbraceright}{\isachardoublequoteclose}\ \isacommand{by}\isamarkupfalse%
\ {\isacharparenleft}metis\ mm{\isadigit{2}}{\isadigit{9}}{\isacharparenright}\ \isanewline
\isacommand{then}\isamarkupfalse%
\ \isacommand{have}\isamarkupfalse%
\ {\isachardoublequoteopen}Domain\ a\ {\isasymsubseteq}\ Domain\ {\isacharquery}p{\isachardoublequoteclose}\ \isacommand{by}\isamarkupfalse%
\ force\isanewline
\isacommand{moreover}\isamarkupfalse%
\ \isacommand{have}\isamarkupfalse%
\ {\isachardoublequoteopen}Domain\ a\ {\isasymsupseteq}\ Domain\ {\isacharquery}p{\isachardoublequoteclose}\ \isacommand{by}\isamarkupfalse%
\ fast\isanewline
\isacommand{ultimately}\isamarkupfalse%
\ \isacommand{have}\isamarkupfalse%
\ \isanewline
{\isadigit{1}}{\isacharcolon}\ {\isachardoublequoteopen}Domain\ a\ {\isacharequal}\ Domain\ {\isacharquery}p{\isachardoublequoteclose}\ \isacommand{by}\isamarkupfalse%
\ fast\isanewline
\isacommand{{\isacharbraceleft}}\isamarkupfalse%
\isanewline
\ \ \isacommand{fix}\isamarkupfalse%
\ z\ \isacommand{assume}\isamarkupfalse%
\ {\isachardoublequoteopen}z\ {\isasymin}\ {\isacharquery}a{\isachardoublequoteclose}\ \isanewline
\ \ \isacommand{then}\isamarkupfalse%
\ \isacommand{obtain}\isamarkupfalse%
\ x\ \isakeyword{where}\ \isanewline
\ \ {\isachardoublequoteopen}x\ {\isasymin}\ Domain\ {\isacharquery}p\ {\isacharampersand}\ z{\isacharequal}{\isacharparenleft}x\ {\isacharcomma}\ {\isasymUnion}\ {\isacharparenleft}{\isacharquery}p\ {\isacharbackquote}{\isacharbackquote}\ {\isacharbraceleft}x{\isacharbraceright}{\isacharparenright}{\isacharparenright}{\isachardoublequoteclose}\ \isacommand{by}\isamarkupfalse%
\ blast\isanewline
\ \ \isacommand{then}\isamarkupfalse%
\ \isacommand{have}\isamarkupfalse%
\ {\isachardoublequoteopen}x\ {\isasymin}\ Domain\ a\ {\isacharampersand}\ z{\isacharequal}{\isacharparenleft}x\ {\isacharcomma}\ {\isasymUnion}\ {\isacharparenleft}{\isacharquery}p\ {\isacharbackquote}{\isacharbackquote}\ {\isacharbraceleft}x{\isacharbraceright}{\isacharparenright}{\isacharparenright}{\isachardoublequoteclose}\ \isacommand{by}\isamarkupfalse%
\ fast\isanewline
\ \ \isacommand{then}\isamarkupfalse%
\ \isacommand{moreover}\isamarkupfalse%
\ \isacommand{have}\isamarkupfalse%
\ {\isachardoublequoteopen}{\isacharquery}p{\isacharbackquote}{\isacharbackquote}{\isacharbraceleft}x{\isacharbraceright}\ {\isacharequal}\ finestpart\ {\isacharparenleft}a{\isacharcomma}{\isacharcomma}x{\isacharparenright}{\isachardoublequoteclose}\ \isacommand{using}\isamarkupfalse%
\ assms\ \isacommand{by}\isamarkupfalse%
\ fastforce\isanewline
\ \ \isacommand{moreover}\isamarkupfalse%
\ \isacommand{have}\isamarkupfalse%
\ {\isachardoublequoteopen}{\isasymUnion}\ {\isacharparenleft}finestpart\ {\isacharparenleft}a{\isacharcomma}{\isacharcomma}x{\isacharparenright}{\isacharparenright}{\isacharequal}\ a{\isacharcomma}{\isacharcomma}x{\isachardoublequoteclose}\ \isacommand{by}\isamarkupfalse%
\ {\isacharparenleft}metis\ mm{\isadigit{7}}{\isadigit{5}}{\isacharparenright}\isanewline
\ \ \isacommand{ultimately}\isamarkupfalse%
\ \isacommand{have}\isamarkupfalse%
\ {\isachardoublequoteopen}z\ {\isasymin}\ a{\isachardoublequoteclose}\ \isacommand{by}\isamarkupfalse%
\ {\isacharparenleft}metis\ assms{\isacharparenleft}{\isadigit{1}}{\isacharparenright}\ eval{\isacharunderscore}runiq{\isacharunderscore}rel{\isacharparenright}\isanewline
\ \ \isacommand{{\isacharbraceright}}\isamarkupfalse%
\isanewline
\isacommand{then}\isamarkupfalse%
\ \isacommand{have}\isamarkupfalse%
\isanewline
{\isadigit{3}}{\isacharcolon}\ {\isachardoublequoteopen}{\isacharquery}a\ {\isasymsubseteq}\ a{\isachardoublequoteclose}\ \isacommand{by}\isamarkupfalse%
\ fast\isanewline
\isacommand{{\isacharbraceleft}}\isamarkupfalse%
\isanewline
\ \ \isacommand{fix}\isamarkupfalse%
\ z\ \isacommand{assume}\isamarkupfalse%
\ {\isadigit{0}}{\isacharcolon}\ {\isachardoublequoteopen}z\ {\isasymin}\ a{\isachardoublequoteclose}\ \isacommand{let}\isamarkupfalse%
\ {\isacharquery}x{\isacharequal}{\isachardoublequoteopen}fst\ z{\isachardoublequoteclose}\ \isacommand{let}\isamarkupfalse%
\ {\isacharquery}Y{\isacharequal}{\isachardoublequoteopen}a{\isacharcomma}{\isacharcomma}{\isacharquery}x{\isachardoublequoteclose}\ \isacommand{let}\isamarkupfalse%
\ {\isacharquery}YY{\isacharequal}{\isachardoublequoteopen}finestpart\ {\isacharquery}Y{\isachardoublequoteclose}\isanewline
\ \ \isacommand{have}\isamarkupfalse%
\ {\isachardoublequoteopen}z\ {\isasymin}\ a\ {\isacharampersand}\ {\isacharquery}x\ {\isasymin}\ Domain\ a{\isachardoublequoteclose}\ \isacommand{using}\isamarkupfalse%
\ {\isadigit{0}}\ \isacommand{by}\isamarkupfalse%
\ {\isacharparenleft}metis\ fst{\isacharunderscore}eq{\isacharunderscore}Domain\ rev{\isacharunderscore}image{\isacharunderscore}eqI{\isacharparenright}\ \isacommand{then}\isamarkupfalse%
\isanewline
\ \ \isacommand{have}\isamarkupfalse%
\ \isanewline
\ \ {\isadigit{2}}{\isacharcolon}{\isachardoublequoteopen}z\ {\isasymin}\ a\ {\isacharampersand}\ {\isacharquery}x\ {\isasymin}\ Domain\ {\isacharquery}p{\isachardoublequoteclose}\ \isacommand{using}\isamarkupfalse%
\ {\isadigit{1}}\ \isacommand{by}\isamarkupfalse%
\ presburger\ \ \isacommand{then}\isamarkupfalse%
\isanewline
\ \ \isacommand{have}\isamarkupfalse%
\ {\isachardoublequoteopen}{\isacharquery}p\ {\isacharbackquote}{\isacharbackquote}\ {\isacharbraceleft}{\isacharquery}x{\isacharbraceright}\ {\isacharequal}\ {\isacharquery}YY{\isachardoublequoteclose}\ \isacommand{by}\isamarkupfalse%
\ fastforce\isanewline
\ \ \isacommand{then}\isamarkupfalse%
\ \isacommand{have}\isamarkupfalse%
\ {\isachardoublequoteopen}{\isasymUnion}\ {\isacharparenleft}{\isacharquery}p\ {\isacharbackquote}{\isacharbackquote}\ {\isacharbraceleft}{\isacharquery}x{\isacharbraceright}{\isacharparenright}\ {\isacharequal}\ {\isacharquery}Y{\isachardoublequoteclose}\ \isacommand{by}\isamarkupfalse%
\ {\isacharparenleft}metis\ mm{\isadigit{7}}{\isadigit{5}}{\isacharparenright}\isanewline
\ \ \isacommand{moreover}\isamarkupfalse%
\ \isacommand{have}\isamarkupfalse%
\ {\isachardoublequoteopen}z\ {\isacharequal}\ {\isacharparenleft}{\isacharquery}x{\isacharcomma}\ {\isacharquery}Y{\isacharparenright}{\isachardoublequoteclose}\ \isacommand{using}\isamarkupfalse%
\ assms\ \isacommand{by}\isamarkupfalse%
\ {\isacharparenleft}metis\ {\isachardoublequoteopen}{\isadigit{0}}{\isachardoublequoteclose}\ mm{\isadigit{6}}{\isadigit{0}}\ surjective{\isacharunderscore}pairing{\isacharparenright}\isanewline
\ \ \isacommand{ultimately}\isamarkupfalse%
\ \isacommand{have}\isamarkupfalse%
\ {\isachardoublequoteopen}z\ {\isasymin}\ {\isacharquery}a{\isachardoublequoteclose}\ \isacommand{using}\isamarkupfalse%
\ {\isadigit{2}}\ \isacommand{by}\isamarkupfalse%
\ {\isacharparenleft}metis\ {\isacharparenleft}lifting{\isacharcomma}\ mono{\isacharunderscore}tags{\isacharparenright}\ mem{\isacharunderscore}Collect{\isacharunderscore}eq{\isacharparenright}\isanewline
\ \ \isacommand{{\isacharbraceright}}\isamarkupfalse%
\isanewline
\isacommand{then}\isamarkupfalse%
\ \isacommand{have}\isamarkupfalse%
\ {\isachardoublequoteopen}a\ {\isacharequal}\ {\isacharquery}a{\isachardoublequoteclose}\ \isacommand{using}\isamarkupfalse%
\ {\isadigit{3}}\ \isacommand{by}\isamarkupfalse%
\ blast\isanewline
\isacommand{moreover}\isamarkupfalse%
\ \isacommand{have}\isamarkupfalse%
\ {\isachardoublequoteopen}{\isacharquery}p\ {\isacharequal}\ pseudoAllocation\ a{\isachardoublequoteclose}\ \isacommand{using}\isamarkupfalse%
\ mm{\isadigit{5}}{\isadigit{5}}v\ assms\ \isacommand{by}\isamarkupfalse%
\ {\isacharparenleft}metis\ {\isacharparenleft}lifting{\isacharcomma}\ mono{\isacharunderscore}tags{\isacharparenright}{\isacharparenright}\isanewline
\isacommand{ultimately}\isamarkupfalse%
\ \isacommand{show}\isamarkupfalse%
\ {\isacharquery}thesis\ \isacommand{by}\isamarkupfalse%
\ auto\isanewline
\isacommand{qed}\isamarkupfalse%
%
\endisatagproof
{\isafoldproof}%
%
\isadelimproof
\isanewline
%
\endisadelimproof
\isacommand{corollary}\isamarkupfalse%
\ mm{\isadigit{7}}{\isadigit{5}}dd{\isacharcolon}\ \isakeyword{assumes}\ {\isachardoublequoteopen}a\ {\isasymin}\ runiqs\ {\isasyminter}\ Pow\ {\isacharparenleft}UNIV\ {\isasymtimes}\ {\isacharparenleft}UNIV\ {\isacharminus}\ {\isacharbraceleft}{\isacharbraceleft}{\isacharbraceright}{\isacharbraceright}{\isacharparenright}{\isacharparenright}{\isachardoublequoteclose}\ \isakeyword{shows}\isanewline
{\isachardoublequoteopen}{\isacharparenleft}mbc\ {\isasymcirc}\ pseudoAllocation{\isacharparenright}\ a\ {\isacharequal}\ id\ a{\isachardoublequoteclose}%
\isadelimproof
\ %
\endisadelimproof
%
\isatagproof
\isacommand{using}\isamarkupfalse%
\ assms\ mm{\isadigit{7}}{\isadigit{5}}d\ \isanewline
\isacommand{proof}\isamarkupfalse%
\ {\isacharminus}\isanewline
\isacommand{have}\isamarkupfalse%
\ {\isachardoublequoteopen}runiq\ a{\isachardoublequoteclose}\ \isacommand{using}\isamarkupfalse%
\ runiqs{\isacharunderscore}def\ assms\ \isacommand{by}\isamarkupfalse%
\ fast\isanewline
\isacommand{moreover}\isamarkupfalse%
\ \isacommand{have}\isamarkupfalse%
\ {\isachardoublequoteopen}{\isacharbraceleft}{\isacharbraceright}\ {\isasymnotin}\ Range\ a{\isachardoublequoteclose}\ \isacommand{using}\isamarkupfalse%
\ assms\ \isacommand{by}\isamarkupfalse%
\ blast\isanewline
\isacommand{ultimately}\isamarkupfalse%
\ \isacommand{show}\isamarkupfalse%
\ {\isacharquery}thesis\ \isacommand{using}\isamarkupfalse%
\ mm{\isadigit{7}}{\isadigit{5}}d\ \isacommand{by}\isamarkupfalse%
\ fastforce\isanewline
\isacommand{qed}\isamarkupfalse%
%
\endisatagproof
{\isafoldproof}%
%
\isadelimproof
%
\endisadelimproof
\isanewline
\isacommand{lemma}\isamarkupfalse%
\ mm{\isadigit{7}}{\isadigit{5}}e{\isacharcolon}\ {\isachardoublequoteopen}inj{\isacharunderscore}on\ {\isacharparenleft}mbc\ {\isasymcirc}\ pseudoAllocation{\isacharparenright}\ {\isacharparenleft}runiqs\ {\isasyminter}\ Pow\ {\isacharparenleft}UNIV\ {\isasymtimes}\ {\isacharparenleft}UNIV\ {\isacharminus}\ {\isacharbraceleft}{\isacharbraceleft}{\isacharbraceright}{\isacharbraceright}{\isacharparenright}{\isacharparenright}{\isacharparenright}{\isachardoublequoteclose}\ \isanewline
%
\isadelimproof
%
\endisadelimproof
%
\isatagproof
\isacommand{using}\isamarkupfalse%
\ assms\ mm{\isadigit{7}}{\isadigit{5}}dd\ inj{\isacharunderscore}on{\isacharunderscore}def\ inj{\isacharunderscore}on{\isacharunderscore}id\ \isanewline
\isacommand{proof}\isamarkupfalse%
\ {\isacharminus}\isanewline
\isacommand{let}\isamarkupfalse%
\ {\isacharquery}ne{\isacharequal}{\isachardoublequoteopen}Pow\ {\isacharparenleft}UNIV\ {\isasymtimes}\ {\isacharparenleft}UNIV\ {\isacharminus}\ {\isacharbraceleft}{\isacharbraceleft}{\isacharbraceright}{\isacharbraceright}{\isacharparenright}{\isacharparenright}{\isachardoublequoteclose}\ \isacommand{let}\isamarkupfalse%
\ {\isacharquery}X{\isacharequal}{\isachardoublequoteopen}runiqs\ {\isasyminter}\ {\isacharquery}ne{\isachardoublequoteclose}\ \isacommand{let}\isamarkupfalse%
\ {\isacharquery}f{\isacharequal}{\isachardoublequoteopen}mbc\ {\isasymcirc}\ pseudoAllocation{\isachardoublequoteclose}\isanewline
\isacommand{have}\isamarkupfalse%
\ {\isachardoublequoteopen}{\isasymforall}a{\isadigit{1}}\ {\isasymin}\ {\isacharquery}X{\isachardot}\ {\isasymforall}\ a{\isadigit{2}}\ {\isasymin}\ {\isacharquery}X{\isachardot}\ {\isacharquery}f\ a{\isadigit{1}}\ {\isacharequal}\ {\isacharquery}f\ a{\isadigit{2}}\ {\isasymlongrightarrow}\ id\ a{\isadigit{1}}\ {\isacharequal}\ id\ a{\isadigit{2}}{\isachardoublequoteclose}\ \isacommand{using}\isamarkupfalse%
\ mm{\isadigit{7}}{\isadigit{5}}dd\ \isacommand{by}\isamarkupfalse%
\ blast\ \isacommand{then}\isamarkupfalse%
\ \isanewline
\isacommand{have}\isamarkupfalse%
\ {\isachardoublequoteopen}{\isasymforall}a{\isadigit{1}}\ {\isasymin}\ {\isacharquery}X{\isachardot}\ {\isasymforall}\ a{\isadigit{2}}\ {\isasymin}\ {\isacharquery}X{\isachardot}\ {\isacharquery}f\ a{\isadigit{1}}\ {\isacharequal}\ {\isacharquery}f\ a{\isadigit{2}}\ {\isasymlongrightarrow}\ a{\isadigit{1}}\ {\isacharequal}\ a{\isadigit{2}}{\isachardoublequoteclose}\ \isacommand{by}\isamarkupfalse%
\ auto\isanewline
\isacommand{thus}\isamarkupfalse%
\ {\isacharquery}thesis\ \isacommand{unfolding}\isamarkupfalse%
\ inj{\isacharunderscore}on{\isacharunderscore}def\ \isacommand{by}\isamarkupfalse%
\ blast\isanewline
\isacommand{qed}\isamarkupfalse%
%
\endisatagproof
{\isafoldproof}%
%
\isadelimproof
\isanewline
%
\endisadelimproof
\isanewline
\isacommand{corollary}\isamarkupfalse%
\ mm{\isadigit{7}}{\isadigit{5}}g{\isacharcolon}\ {\isachardoublequoteopen}inj{\isacharunderscore}on\ pseudoAllocation\ {\isacharparenleft}runiqs\ {\isasyminter}\ Pow\ {\isacharparenleft}UNIV\ {\isasymtimes}\ {\isacharparenleft}UNIV\ {\isacharminus}\ {\isacharbraceleft}{\isacharbraceleft}{\isacharbraceright}{\isacharbraceright}{\isacharparenright}{\isacharparenright}{\isacharparenright}{\isachardoublequoteclose}\ \isanewline
%
\isadelimproof
%
\endisadelimproof
%
\isatagproof
\isacommand{using}\isamarkupfalse%
\ mm{\isadigit{7}}{\isadigit{5}}e\ inj{\isacharunderscore}on{\isacharunderscore}imageI{\isadigit{2}}\ \isacommand{by}\isamarkupfalse%
\ blast%
\endisatagproof
{\isafoldproof}%
%
\isadelimproof
\isanewline
%
\endisadelimproof
\isacommand{lemma}\isamarkupfalse%
\ mm{\isadigit{7}}{\isadigit{6}}{\isacharcolon}\ {\isachardoublequoteopen}injectionsUniverse\ {\isasymsubseteq}\ runiqs{\isachardoublequoteclose}%
\isadelimproof
\ %
\endisadelimproof
%
\isatagproof
\isacommand{using}\isamarkupfalse%
\ runiqs{\isacharunderscore}def\ Collect{\isacharunderscore}conj{\isacharunderscore}eq\ Int{\isacharunderscore}lower{\isadigit{1}}\ \isacommand{by}\isamarkupfalse%
\ metis%
\endisatagproof
{\isafoldproof}%
%
\isadelimproof
%
\endisadelimproof
\isanewline
\isacommand{lemma}\isamarkupfalse%
\ mm{\isadigit{7}}{\isadigit{7}}{\isacharcolon}\ {\isachardoublequoteopen}partitionValuedUniverse\ {\isasymsubseteq}\ Pow\ {\isacharparenleft}UNIV\ {\isasymtimes}\ {\isacharparenleft}UNIV\ {\isacharminus}\ {\isacharbraceleft}{\isacharbraceleft}{\isacharbraceright}{\isacharbraceright}{\isacharparenright}{\isacharparenright}{\isachardoublequoteclose}%
\isadelimproof
\ %
\endisadelimproof
%
\isatagproof
\isacommand{using}\isamarkupfalse%
\ assms\ is{\isacharunderscore}partition{\isacharunderscore}def\ \isacommand{by}\isamarkupfalse%
\ force%
\endisatagproof
{\isafoldproof}%
%
\isadelimproof
%
\endisadelimproof
\isanewline
\isacommand{corollary}\isamarkupfalse%
\ mm{\isadigit{7}}{\isadigit{5}}i{\isacharcolon}\ {\isachardoublequoteopen}allocationsUniverse\ {\isasymsubseteq}\ runiqs\ {\isasyminter}\ Pow\ {\isacharparenleft}UNIV\ {\isasymtimes}\ {\isacharparenleft}UNIV\ {\isacharminus}\ {\isacharbraceleft}{\isacharbraceleft}{\isacharbraceright}{\isacharbraceright}{\isacharparenright}{\isacharparenright}{\isachardoublequoteclose}%
\isadelimproof
\ %
\endisadelimproof
%
\isatagproof
\isacommand{using}\isamarkupfalse%
\ mm{\isadigit{7}}{\isadigit{6}}\ mm{\isadigit{7}}{\isadigit{7}}\ \isacommand{by}\isamarkupfalse%
\ auto%
\endisatagproof
{\isafoldproof}%
%
\isadelimproof
%
\endisadelimproof
\isanewline
\isacommand{corollary}\isamarkupfalse%
\ mm{\isadigit{7}}{\isadigit{5}}h{\isacharcolon}\ {\isachardoublequoteopen}inj{\isacharunderscore}on\ pseudoAllocation\ allocationsUniverse{\isachardoublequoteclose}%
\isadelimproof
\ %
\endisadelimproof
%
\isatagproof
\isacommand{using}\isamarkupfalse%
\ assms\ mm{\isadigit{7}}{\isadigit{5}}g\ mm{\isadigit{7}}{\isadigit{5}}i\ subset{\isacharunderscore}inj{\isacharunderscore}on\ \isacommand{by}\isamarkupfalse%
\ blast%
\endisatagproof
{\isafoldproof}%
%
\isadelimproof
%
\endisadelimproof
\isanewline
\isacommand{corollary}\isamarkupfalse%
\ mm{\isadigit{7}}{\isadigit{5}}j{\isacharcolon}\ {\isachardoublequoteopen}inj{\isacharunderscore}on\ pseudoAllocation\ {\isacharparenleft}possibleAllocationsRel\ N\ G{\isacharparenright}{\isachardoublequoteclose}\ \isanewline
%
\isadelimproof
%
\endisadelimproof
%
\isatagproof
\isacommand{proof}\isamarkupfalse%
\ {\isacharminus}\isanewline
\ \ \isacommand{have}\isamarkupfalse%
\ {\isachardoublequoteopen}possibleAllocationsRel\ N\ G\ {\isasymsubseteq}\ allocationsUniverse{\isachardoublequoteclose}\ \isacommand{by}\isamarkupfalse%
\ {\isacharparenleft}metis\ {\isacharparenleft}no{\isacharunderscore}types{\isacharparenright}\ lm{\isadigit{5}}{\isadigit{0}}{\isacharparenright}\isanewline
\ \ \isacommand{thus}\isamarkupfalse%
\ {\isachardoublequoteopen}inj{\isacharunderscore}on\ pseudoAllocation\ {\isacharparenleft}possibleAllocationsRel\ N\ G{\isacharparenright}{\isachardoublequoteclose}\ \isacommand{using}\isamarkupfalse%
\ mm{\isadigit{7}}{\isadigit{5}}h\ subset{\isacharunderscore}inj{\isacharunderscore}on\ \isacommand{by}\isamarkupfalse%
\ blast\isanewline
\isacommand{qed}\isamarkupfalse%
%
\endisatagproof
{\isafoldproof}%
%
\isadelimproof
\isanewline
%
\endisadelimproof
\isanewline
\isanewline
\isanewline
\isanewline
\isanewline
\isanewline
\isanewline
\isanewline
\isanewline
\isanewline
\isanewline
\isanewline
\isanewline
\isanewline
\isanewline
\isanewline
\isanewline
\isanewline
\isanewline
\isanewline
\isanewline
\isanewline
\isanewline
\isanewline
\isanewline
\isanewline
\isanewline
\isanewline
\isanewline
\isanewline
\isanewline
\isanewline
\isanewline
\isanewline
\isanewline
\isanewline
\isanewline
\isanewline
\isanewline
\isanewline
\isacommand{fun}\isamarkupfalse%
\ prova\ \isakeyword{where}\ {\isachardoublequoteopen}prova\ f\ X\ {\isadigit{0}}\ {\isacharequal}\ X{\isachardoublequoteclose}\ {\isacharbar}\ {\isachardoublequoteopen}prova\ f\ X\ {\isacharparenleft}Suc\ n{\isacharparenright}\ {\isacharequal}\ f\ n\ {\isacharparenleft}prova\ f\ X\ n{\isacharparenright}{\isachardoublequoteclose}\isanewline
\isanewline
\isacommand{fun}\isamarkupfalse%
\ prova{\isadigit{2}}\ \isakeyword{where}\ {\isachardoublequoteopen}prova{\isadigit{2}}\ f\ {\isadigit{0}}\ {\isacharequal}\ UNIV{\isachardoublequoteclose}\ {\isacharbar}{\isachardoublequoteopen}prova{\isadigit{2}}\ f\ {\isacharparenleft}Suc\ n{\isacharparenright}\ {\isacharequal}\ f\ n\ {\isacharparenleft}prova{\isadigit{2}}\ f\ n{\isacharparenright}{\isachardoublequoteclose}\isanewline
\isanewline
\isacommand{fun}\isamarkupfalse%
\ geniter\ \isakeyword{where}\ {\isachardoublequoteopen}geniter\ f\ {\isadigit{0}}\ {\isacharequal}\ f\ {\isadigit{0}}{\isachardoublequoteclose}\ {\isacharbar}\ {\isachardoublequoteopen}geniter\ f\ {\isacharparenleft}Suc\ n{\isacharparenright}{\isacharequal}{\isacharparenleft}f\ {\isacharparenleft}Suc\ n{\isacharparenright}{\isacharparenright}\ o\ {\isacharparenleft}geniter\ f\ n{\isacharparenright}{\isachardoublequoteclose}\isanewline
\isanewline
\isacommand{abbreviation}\isamarkupfalse%
\ {\isachardoublequoteopen}pseudodecreasing\ X\ Y\ {\isacharequal}{\isacharequal}\ card\ X\ {\isacharminus}\ {\isadigit{1}}\ {\isasymle}\ card\ Y\ {\isacharminus}\ {\isadigit{2}}{\isachardoublequoteclose}\isanewline
\isanewline
\isacommand{notation}\isamarkupfalse%
\ pseudodecreasing\ {\isacharparenleft}\isakeyword{infix}\ {\isachardoublequoteopen}{\isacharless}{\isachartilde}{\isachardoublequoteclose}\ {\isadigit{7}}{\isadigit{5}}{\isacharparenright}\isanewline
\isanewline
\isacommand{abbreviation}\isamarkupfalse%
\ {\isachardoublequoteopen}subList\ l\ xl\ {\isacharequal}{\isacharequal}\ map\ {\isacharparenleft}nth\ l{\isacharparenright}\ {\isacharparenleft}takeAll\ {\isacharparenleft}{\isacharpercent}x{\isachardot}\ x\ {\isasymle}\ size\ l{\isacharparenright}\ xl{\isacharparenright}{\isachardoublequoteclose}\isanewline
\isacommand{abbreviation}\isamarkupfalse%
\ {\isachardoublequoteopen}insertRightOf{\isadigit{2}}\ x\ l\ n\ {\isacharequal}{\isacharequal}\ {\isacharparenleft}subList\ l\ {\isacharparenleft}map\ nat\ {\isacharbrackleft}{\isadigit{0}}{\isachardot}{\isachardot}n{\isacharbrackright}{\isacharparenright}{\isacharparenright}\ {\isacharat}\ {\isacharbrackleft}x{\isacharbrackright}\ {\isacharat}\ \isanewline
{\isacharparenleft}subList\ l\ {\isacharparenleft}map\ nat\ {\isacharbrackleft}n{\isacharplus}{\isadigit{1}}{\isachardot}{\isachardot}size\ l\ {\isacharminus}\ {\isadigit{1}}{\isacharbrackright}{\isacharparenright}{\isacharparenright}{\isachardoublequoteclose}\isanewline
\isacommand{abbreviation}\isamarkupfalse%
\ {\isachardoublequoteopen}insertRightOf{\isadigit{3}}\ x\ l\ n\ {\isacharequal}{\isacharequal}\ insertRightOf{\isadigit{2}}\ x\ l\ {\isacharparenleft}Min\ {\isacharbraceleft}n{\isacharcomma}\ size\ l\ {\isacharminus}\ {\isadigit{1}}{\isacharbraceright}{\isacharparenright}{\isachardoublequoteclose}\isanewline
\isacommand{definition}\isamarkupfalse%
\ {\isachardoublequoteopen}insertRightOf\ x\ l\ n\ {\isacharequal}\ sublist\ l\ {\isacharbraceleft}{\isadigit{0}}{\isachardot}{\isachardot}{\isacharless}{\isadigit{1}}{\isacharplus}n{\isacharbraceright}\ {\isacharat}\ {\isacharbrackleft}x{\isacharbrackright}\ {\isacharat}\ sublist\ l\ {\isacharbraceleft}n{\isacharplus}{\isadigit{1}}{\isachardot}{\isachardot}{\isacharless}\ {\isadigit{1}}{\isacharplus}size\ l{\isacharbraceright}{\isachardoublequoteclose}\isanewline
\isacommand{lemma}\isamarkupfalse%
\ {\isachardoublequoteopen}set\ {\isacharparenleft}insertRightOf\ x\ l\ n{\isacharparenright}\ {\isacharequal}\ set\ {\isacharparenleft}sublist\ l\ {\isacharbraceleft}{\isadigit{0}}{\isachardot}{\isachardot}{\isacharless}{\isadigit{1}}{\isacharplus}n{\isacharbraceright}{\isacharparenright}\ {\isasymunion}\ {\isacharparenleft}set\ {\isacharbrackleft}x{\isacharbrackright}{\isacharparenright}\ {\isasymunion}\ \isanewline
set\ {\isacharparenleft}sublist\ l\ {\isacharbraceleft}n{\isacharplus}{\isadigit{1}}{\isachardot}{\isachardot}{\isacharless}{\isadigit{1}}{\isacharplus}size\ l{\isacharbraceright}{\isacharparenright}{\isachardoublequoteclose}%
\isadelimproof
\ %
\endisadelimproof
%
\isatagproof
\isacommand{using}\isamarkupfalse%
\ insertRightOf{\isacharunderscore}def\ \isanewline
\isacommand{by}\isamarkupfalse%
\ {\isacharparenleft}metis\ append{\isacharunderscore}assoc\ set{\isacharunderscore}append{\isacharparenright}%
\endisatagproof
{\isafoldproof}%
%
\isadelimproof
%
\endisadelimproof
\isanewline
\isacommand{lemma}\isamarkupfalse%
\ {\isachardoublequoteopen}set\ l{\isadigit{1}}\ {\isasymunion}\ set\ l{\isadigit{2}}\ {\isacharequal}\ set\ {\isacharparenleft}l{\isadigit{1}}\ {\isacharat}\ l{\isadigit{2}}{\isacharparenright}{\isachardoublequoteclose}%
\isadelimproof
\ %
\endisadelimproof
%
\isatagproof
\isacommand{by}\isamarkupfalse%
\ simp%
\endisatagproof
{\isafoldproof}%
%
\isadelimproof
%
\endisadelimproof
\isanewline
\isanewline
\isacommand{fun}\isamarkupfalse%
\ permOld{\isacharcolon}{\isacharcolon}{\isachardoublequoteopen}{\isacharprime}a\ list\ {\isacharequal}{\isachargreater}\ {\isacharparenleft}nat\ {\isasymtimes}\ {\isacharparenleft}{\isacharprime}a\ list{\isacharparenright}{\isacharparenright}\ set{\isachardoublequoteclose}\ \isakeyword{where}\ \isanewline
{\isachardoublequoteopen}permOld\ {\isacharbrackleft}{\isacharbrackright}\ {\isacharequal}\ {\isacharbraceleft}{\isacharbraceright}{\isachardoublequoteclose}\ {\isacharbar}\ {\isachardoublequoteopen}permOld\ {\isacharparenleft}x{\isacharhash}l{\isacharparenright}\ {\isacharequal}\ \isanewline
graph\ {\isacharbraceleft}fact\ {\isacharparenleft}size\ l{\isacharparenright}\ {\isachardot}{\isachardot}{\isacharless}\ {\isadigit{1}}{\isacharplus}fact\ {\isacharparenleft}{\isadigit{1}}\ {\isacharplus}\ {\isacharparenleft}size\ l{\isacharparenright}{\isacharparenright}{\isacharbraceright}\isanewline
{\isacharparenleft}{\isacharpercent}n{\isacharcolon}{\isacharcolon}nat{\isachardot}\ insertRightOf\ x\ {\isacharparenleft}permOld\ l{\isacharcomma}{\isacharcomma}{\isacharparenleft}n\ div\ size\ l{\isacharparenright}{\isacharparenright}\ {\isacharparenleft}n\ mod\ {\isacharparenleft}size\ l{\isacharparenright}{\isacharparenright}{\isacharparenright}\isanewline
{\isacharplus}{\isacharasterisk}\ {\isacharparenleft}permOld\ l{\isacharparenright}{\isachardoublequoteclose}\isanewline
\isanewline
\isacommand{fun}\isamarkupfalse%
\ permL\ \isakeyword{where}\ \isanewline
{\isachardoublequoteopen}permL\ {\isacharbrackleft}{\isacharbrackright}\ {\isacharequal}\ {\isacharparenleft}{\isacharpercent}n{\isachardot}\ {\isacharbrackleft}{\isacharbrackright}{\isacharparenright}{\isachardoublequoteclose}{\isacharbar}\isanewline
{\isachardoublequoteopen}permL\ {\isacharparenleft}x{\isacharhash}l{\isacharparenright}\ {\isacharequal}\ {\isacharparenleft}{\isacharpercent}n{\isachardot}\ \isanewline
if\ {\isacharparenleft}fact\ {\isacharparenleft}size\ l{\isacharparenright}\ {\isacharless}\ n\ {\isacharampersand}\ n\ {\isacharless}{\isacharequal}\ fact\ {\isacharparenleft}{\isadigit{1}}\ {\isacharplus}\ {\isacharparenleft}size\ l{\isacharparenright}{\isacharparenright}{\isacharparenright}\isanewline
then\isanewline
{\isacharparenleft}insertRightOf{\isadigit{3}}\ x\ {\isacharparenleft}permL\ l\ {\isacharparenleft}n\ div\ size\ l{\isacharparenright}{\isacharparenright}\ {\isacharparenleft}n\ mod\ {\isacharparenleft}size\ l{\isacharparenright}{\isacharparenright}{\isacharparenright}\isanewline
else\isanewline
{\isacharparenleft}x\ {\isacharhash}\ {\isacharparenleft}permL\ l\ n{\isacharparenright}{\isacharparenright}\isanewline
{\isacharparenright}{\isachardoublequoteclose}\isanewline
\isanewline
\isacommand{lemma}\isamarkupfalse%
\ mm{\isadigit{9}}{\isadigit{4}}{\isacharcolon}\ {\isachardoublequoteopen}possibleAllocationsAlg{\isadigit{2}}\ N\ G\ {\isacharequal}\ set\ {\isacharparenleft}possibleAllocationsAlg{\isadigit{3}}\ N\ G{\isacharparenright}{\isachardoublequoteclose}%
\isadelimproof
\ %
\endisadelimproof
%
\isatagproof
\isacommand{by}\isamarkupfalse%
\ auto%
\endisatagproof
{\isafoldproof}%
%
\isadelimproof
%
\endisadelimproof
\isanewline
\isacommand{lemma}\isamarkupfalse%
\ mm{\isadigit{9}}{\isadigit{5}}{\isacharcolon}\ \isakeyword{assumes}\ {\isachardoublequoteopen}card\ N\ {\isachargreater}\ {\isadigit{0}}{\isachardoublequoteclose}\ {\isachardoublequoteopen}distinct\ G{\isachardoublequoteclose}\ \isakeyword{shows}\ \isanewline
{\isachardoublequoteopen}winningAllocationsRel\ N\ {\isacharparenleft}set\ G{\isacharparenright}\ bids\ {\isasymsubseteq}\ set\ {\isacharparenleft}possibleAllocationsAlg{\isadigit{3}}\ N\ G{\isacharparenright}{\isachardoublequoteclose}\isanewline
%
\isadelimproof
%
\endisadelimproof
%
\isatagproof
\isacommand{using}\isamarkupfalse%
\ assms\ mm{\isadigit{9}}{\isadigit{4}}\ lm{\isadigit{0}}{\isadigit{3}}\ lm{\isadigit{7}}{\isadigit{0}}b\ \isacommand{by}\isamarkupfalse%
\ {\isacharparenleft}metis{\isacharparenleft}no{\isacharunderscore}types{\isacharparenright}{\isacharparenright}%
\endisatagproof
{\isafoldproof}%
%
\isadelimproof
\isanewline
%
\endisadelimproof
\isacommand{corollary}\isamarkupfalse%
\ mm{\isadigit{9}}{\isadigit{6}}{\isacharcolon}\ \isakeyword{assumes}\ \isanewline
{\isachardoublequoteopen}N\ {\isasymnoteq}\ {\isacharbraceleft}{\isacharbraceright}{\isachardoublequoteclose}\ {\isachardoublequoteopen}finite\ N{\isachardoublequoteclose}\ {\isachardoublequoteopen}distinct\ G{\isachardoublequoteclose}\ {\isachardoublequoteopen}set\ G\ {\isasymnoteq}\ {\isacharbraceleft}{\isacharbraceright}{\isachardoublequoteclose}\ \isakeyword{shows}\isanewline
{\isachardoublequoteopen}winningAllocationsRel\ N\ {\isacharparenleft}set\ G{\isacharparenright}\ bids\ {\isasyminter}\ set\ {\isacharparenleft}possibleAllocationsAlg{\isadigit{3}}\ N\ G{\isacharparenright}\ {\isasymnoteq}\ {\isacharbraceleft}{\isacharbraceright}{\isachardoublequoteclose}\isanewline
%
\isadelimproof
%
\endisadelimproof
%
\isatagproof
\isacommand{using}\isamarkupfalse%
\ assms\ mm{\isadigit{9}}{\isadigit{1}}\ mm{\isadigit{9}}{\isadigit{5}}\ \isanewline
\isacommand{proof}\isamarkupfalse%
\ {\isacharminus}\isanewline
\isacommand{let}\isamarkupfalse%
\ {\isacharquery}w{\isacharequal}winningAllocationsRel\ \isacommand{let}\isamarkupfalse%
\ {\isacharquery}a{\isacharequal}possibleAllocationsAlg{\isadigit{3}}\isanewline
\isacommand{let}\isamarkupfalse%
\ {\isacharquery}G{\isacharequal}{\isachardoublequoteopen}set\ G{\isachardoublequoteclose}\ \isacommand{have}\isamarkupfalse%
\ {\isachardoublequoteopen}card\ N\ {\isachargreater}\ {\isadigit{0}}{\isachardoublequoteclose}\ \isacommand{using}\isamarkupfalse%
\ assms\ \isacommand{by}\isamarkupfalse%
\ {\isacharparenleft}metis\ card{\isacharunderscore}gt{\isacharunderscore}{\isadigit{0}}{\isacharunderscore}iff{\isacharparenright}\isanewline
\isacommand{then}\isamarkupfalse%
\ \isacommand{have}\isamarkupfalse%
\ {\isachardoublequoteopen}{\isacharquery}w\ N\ {\isacharquery}G\ bids\ {\isasymsubseteq}\ set\ {\isacharparenleft}{\isacharquery}a\ N\ G{\isacharparenright}{\isachardoublequoteclose}\ \isacommand{using}\isamarkupfalse%
\ mm{\isadigit{9}}{\isadigit{5}}\ \isacommand{by}\isamarkupfalse%
\ {\isacharparenleft}metis\ assms{\isacharparenleft}{\isadigit{3}}{\isacharparenright}{\isacharparenright}\isanewline
\isacommand{then}\isamarkupfalse%
\ \isacommand{show}\isamarkupfalse%
\ {\isacharquery}thesis\ \isacommand{using}\isamarkupfalse%
\ assms\ mm{\isadigit{9}}{\isadigit{1}}\ \isacommand{by}\isamarkupfalse%
\ {\isacharparenleft}metis\ List{\isachardot}finite{\isacharunderscore}set\ le{\isacharunderscore}iff{\isacharunderscore}inf{\isacharparenright}\isanewline
\isacommand{qed}\isamarkupfalse%
%
\endisatagproof
{\isafoldproof}%
%
\isadelimproof
\isanewline
%
\endisadelimproof
\isacommand{lemma}\isamarkupfalse%
\ mm{\isadigit{9}}{\isadigit{7}}{\isacharcolon}\ {\isachardoublequoteopen}X\ {\isacharequal}\ {\isacharparenleft}{\isacharpercent}x{\isachardot}\ x\ {\isasymin}\ X{\isacharparenright}\ {\isacharminus}{\isacharbackquote}{\isacharbraceleft}True{\isacharbraceright}{\isachardoublequoteclose}%
\isadelimproof
\ %
\endisadelimproof
%
\isatagproof
\isacommand{by}\isamarkupfalse%
\ blast%
\endisatagproof
{\isafoldproof}%
%
\isadelimproof
%
\endisadelimproof
\isanewline
\isacommand{corollary}\isamarkupfalse%
\ mm{\isadigit{9}}{\isadigit{6}}b{\isacharcolon}\ \isakeyword{assumes}\ \isanewline
{\isachardoublequoteopen}N\ {\isasymnoteq}\ {\isacharbraceleft}{\isacharbraceright}{\isachardoublequoteclose}\ {\isachardoublequoteopen}finite\ N{\isachardoublequoteclose}\ {\isachardoublequoteopen}distinct\ G{\isachardoublequoteclose}\ {\isachardoublequoteopen}set\ G\ {\isasymnoteq}\ {\isacharbraceleft}{\isacharbraceright}{\isachardoublequoteclose}\ \isakeyword{shows}\isanewline
{\isachardoublequoteopen}{\isacharparenleft}{\isacharpercent}x{\isachardot}\ x{\isasymin}winningAllocationsRel\ N\ {\isacharparenleft}set\ G{\isacharparenright}\ bids{\isacharparenright}{\isacharminus}{\isacharbackquote}{\isacharbraceleft}True{\isacharbraceright}\ {\isasyminter}\ set\ {\isacharparenleft}possibleAllocationsAlg{\isadigit{3}}\ N\ G{\isacharparenright}\ {\isasymnoteq}\ {\isacharbraceleft}{\isacharbraceright}{\isachardoublequoteclose}\isanewline
%
\isadelimproof
%
\endisadelimproof
%
\isatagproof
\isacommand{using}\isamarkupfalse%
\ assms\ mm{\isadigit{9}}{\isadigit{6}}\ mm{\isadigit{9}}{\isadigit{7}}\ \isacommand{by}\isamarkupfalse%
\ metis%
\endisatagproof
{\isafoldproof}%
%
\isadelimproof
\isanewline
%
\endisadelimproof
\isacommand{lemma}\isamarkupfalse%
\ mm{\isadigit{8}}{\isadigit{4}}b{\isacharcolon}\ \isakeyword{assumes}\ {\isachardoublequoteopen}P\ {\isacharminus}{\isacharbackquote}\ {\isacharbraceleft}True{\isacharbraceright}\ {\isasyminter}\ set\ l\ {\isasymnoteq}\ {\isacharbraceleft}{\isacharbraceright}{\isachardoublequoteclose}\ \isakeyword{shows}\ {\isachardoublequoteopen}takeAll\ P\ l\ {\isasymnoteq}\ {\isacharbrackleft}{\isacharbrackright}{\isachardoublequoteclose}%
\isadelimproof
\ %
\endisadelimproof
%
\isatagproof
\isacommand{using}\isamarkupfalse%
\ assms\isanewline
mm{\isadigit{8}}{\isadigit{4}}g\ filterpositions{\isadigit{2}}{\isacharunderscore}def\ \isacommand{by}\isamarkupfalse%
\ {\isacharparenleft}metis\ Nil{\isacharunderscore}is{\isacharunderscore}map{\isacharunderscore}conv{\isacharparenright}%
\endisatagproof
{\isafoldproof}%
%
\isadelimproof
%
\endisadelimproof
\isanewline
\isacommand{corollary}\isamarkupfalse%
\ mm{\isadigit{8}}{\isadigit{4}}h{\isacharcolon}\ \isakeyword{assumes}\ {\isachardoublequoteopen}N\ {\isasymnoteq}\ {\isacharbraceleft}{\isacharbraceright}{\isachardoublequoteclose}\ {\isachardoublequoteopen}finite\ N{\isachardoublequoteclose}\ {\isachardoublequoteopen}distinct\ G{\isachardoublequoteclose}\ {\isachardoublequoteopen}set\ G\ {\isasymnoteq}\ {\isacharbraceleft}{\isacharbraceright}{\isachardoublequoteclose}\ \isakeyword{shows}\ \isanewline
{\isachardoublequoteopen}takeAll\ {\isacharparenleft}{\isacharpercent}x{\isachardot}\ x\ {\isasymin}\ winningAllocationsRel\ N\ {\isacharparenleft}set\ G{\isacharparenright}\ bids{\isacharparenright}\ {\isacharparenleft}possibleAllocationsAlg{\isadigit{3}}\ N\ G{\isacharparenright}\ {\isasymnoteq}\ {\isacharbrackleft}{\isacharbrackright}{\isachardoublequoteclose}\isanewline
%
\isadelimproof
%
\endisadelimproof
%
\isatagproof
\isacommand{using}\isamarkupfalse%
\ assms\ mm{\isadigit{8}}{\isadigit{4}}b\ mm{\isadigit{9}}{\isadigit{6}}b\ \isacommand{by}\isamarkupfalse%
\ metis%
\endisatagproof
{\isafoldproof}%
%
\isadelimproof
\isanewline
%
\endisadelimproof
\isanewline
\isacommand{corollary}\isamarkupfalse%
\ nn{\isadigit{0}}{\isadigit{5}}b{\isacharcolon}\ \isakeyword{assumes}\ {\isachardoublequoteopen}N\ {\isasymnoteq}\ {\isacharbraceleft}{\isacharbraceright}{\isachardoublequoteclose}\ {\isachardoublequoteopen}finite\ N{\isachardoublequoteclose}\ {\isachardoublequoteopen}distinct\ G{\isachardoublequoteclose}\ {\isachardoublequoteopen}set\ G\ {\isasymnoteq}\ {\isacharbraceleft}{\isacharbraceright}{\isachardoublequoteclose}\ \isakeyword{shows}\ \isanewline
{\isachardoublequoteopen}perm{\isadigit{2}}\ {\isacharparenleft}takeAll\ {\isacharparenleft}{\isacharpercent}x{\isachardot}\ x\ {\isasymin}\ winningAllocationsRel\ N\ {\isacharparenleft}set\ G{\isacharparenright}\ bids{\isacharparenright}\ {\isacharparenleft}possibleAllocationsAlg{\isadigit{3}}\ N\ G{\isacharparenright}{\isacharparenright}\ n\ {\isasymnoteq}\ {\isacharbrackleft}{\isacharbrackright}{\isachardoublequoteclose}\isanewline
%
\isadelimproof
%
\endisadelimproof
%
\isatagproof
\isacommand{using}\isamarkupfalse%
\ assms\ mm{\isadigit{8}}{\isadigit{3}}\ mm{\isadigit{8}}{\isadigit{4}}h\ \isacommand{by}\isamarkupfalse%
\ metis%
\endisatagproof
{\isafoldproof}%
%
\isadelimproof
\isanewline
%
\endisadelimproof
\isacommand{corollary}\isamarkupfalse%
\ mm{\isadigit{8}}{\isadigit{2}}{\isacharcolon}\ \isakeyword{assumes}\ {\isachardoublequoteopen}N\ {\isasymnoteq}\ {\isacharbraceleft}{\isacharbraceright}{\isachardoublequoteclose}\ {\isachardoublequoteopen}finite\ N{\isachardoublequoteclose}\ {\isachardoublequoteopen}distinct\ G{\isachardoublequoteclose}\ {\isachardoublequoteopen}set\ G\ {\isasymnoteq}\ {\isacharbraceleft}{\isacharbraceright}{\isachardoublequoteclose}\ \isakeyword{shows}\ \isanewline
{\isachardoublequoteopen}chosenAllocation{\isacharprime}\ N\ G\ bids\ random\ {\isasymin}\ winningAllocationsRel\ N\ {\isacharparenleft}set\ G{\isacharparenright}\ bids{\isachardoublequoteclose}\isanewline
%
\isadelimproof
%
\endisadelimproof
%
\isatagproof
\isacommand{using}\isamarkupfalse%
\ assms\ nn{\isadigit{0}}{\isadigit{5}}a\ nn{\isadigit{0}}{\isadigit{5}}b\ hd{\isacharunderscore}in{\isacharunderscore}set\ in{\isacharunderscore}mono\ Int{\isacharunderscore}def\ Int{\isacharunderscore}lower{\isadigit{1}}\ all{\isacharunderscore}not{\isacharunderscore}in{\isacharunderscore}conv\ image{\isacharunderscore}set\ nn{\isadigit{0}}{\isadigit{4}}\ nn{\isadigit{0}}{\isadigit{6}}c\ set{\isacharunderscore}empty\ subsetI\ subset{\isacharunderscore}trans\isanewline
\isacommand{proof}\isamarkupfalse%
\ {\isacharminus}\isanewline
\isacommand{let}\isamarkupfalse%
\ {\isacharquery}w{\isacharequal}winningAllocationsRel\ \isacommand{let}\isamarkupfalse%
\ {\isacharquery}p{\isacharequal}possibleAllocationsAlg{\isadigit{3}}\ \isacommand{let}\isamarkupfalse%
\ {\isacharquery}G{\isacharequal}{\isachardoublequoteopen}set\ G{\isachardoublequoteclose}\isanewline
\isacommand{let}\isamarkupfalse%
\ {\isacharquery}X{\isacharequal}{\isachardoublequoteopen}{\isacharquery}w\ N\ {\isacharquery}G\ bids{\isachardoublequoteclose}\ \isacommand{let}\isamarkupfalse%
\ {\isacharquery}l{\isacharequal}{\isachardoublequoteopen}perm{\isadigit{2}}\ {\isacharparenleft}takeAll\ {\isacharparenleft}{\isacharpercent}x{\isachardot}{\isacharparenleft}x{\isasymin}{\isacharquery}X{\isacharparenright}{\isacharparenright}\ {\isacharparenleft}{\isacharquery}p\ N\ G{\isacharparenright}{\isacharparenright}\ random{\isachardoublequoteclose}\isanewline
\isacommand{have}\isamarkupfalse%
\ {\isachardoublequoteopen}set\ {\isacharquery}l\ {\isasymsubseteq}\ {\isacharquery}X{\isachardoublequoteclose}\ \isacommand{using}\isamarkupfalse%
\ nn{\isadigit{0}}{\isadigit{5}}a\ \isacommand{by}\isamarkupfalse%
\ fast\isanewline
\isacommand{moreover}\isamarkupfalse%
\ \isacommand{have}\isamarkupfalse%
\ {\isachardoublequoteopen}{\isacharquery}l\ {\isasymnoteq}\ {\isacharbrackleft}{\isacharbrackright}{\isachardoublequoteclose}\ \isacommand{using}\isamarkupfalse%
\ assms\ nn{\isadigit{0}}{\isadigit{5}}b\ \isacommand{by}\isamarkupfalse%
\ blast\isanewline
\isacommand{ultimately}\isamarkupfalse%
\ \isacommand{show}\isamarkupfalse%
\ {\isacharquery}thesis\ \isacommand{by}\isamarkupfalse%
\ {\isacharparenleft}metis\ {\isacharparenleft}lifting{\isacharcomma}\ no{\isacharunderscore}types{\isacharparenright}\ hd{\isacharunderscore}in{\isacharunderscore}set\ in{\isacharunderscore}mono{\isacharparenright}\isanewline
\isacommand{qed}\isamarkupfalse%
%
\endisatagproof
{\isafoldproof}%
%
\isadelimproof
\isanewline
%
\endisadelimproof
\isanewline
\isacommand{lemma}\isamarkupfalse%
\ mm{\isadigit{4}}{\isadigit{9}}b{\isacharcolon}\ \isakeyword{assumes}\ {\isachardoublequoteopen}finite\ G{\isachardoublequoteclose}\ {\isachardoublequoteopen}a\ {\isasymin}\ possibleAllocationsRel\ N\ G{\isachardoublequoteclose}\ {\isachardoublequoteopen}aa\ {\isasymin}\ possibleAllocationsRel\ N\ G{\isachardoublequoteclose}\isanewline
\isakeyword{shows}\ {\isachardoublequoteopen}real{\isacharparenleft}setsum{\isacharparenleft}maxbid{\isacharprime}\ a\ N\ G{\isacharparenright}{\isacharparenleft}pseudoAllocation\ a{\isacharparenright}{\isacharparenright}\ {\isacharminus}\ setsum{\isacharparenleft}maxbid{\isacharprime}\ a\ N\ G{\isacharparenright}{\isacharparenleft}pseudoAllocation\ aa{\isacharparenright}\ \isanewline
{\isacharequal}\ real\ {\isacharparenleft}card\ G{\isacharparenright}\ {\isacharminus}\ card\ {\isacharparenleft}pseudoAllocation\ aa\ {\isasyminter}\ {\isacharparenleft}pseudoAllocation\ a{\isacharparenright}{\isacharparenright}{\isachardoublequoteclose}\isanewline
%
\isadelimproof
%
\endisadelimproof
%
\isatagproof
\isacommand{proof}\isamarkupfalse%
\ {\isacharminus}\isanewline
\isacommand{let}\isamarkupfalse%
\ {\isacharquery}p{\isacharequal}pseudoAllocation\ \isacommand{let}\isamarkupfalse%
\ {\isacharquery}f{\isacharequal}finestpart\ \isacommand{let}\isamarkupfalse%
\ {\isacharquery}m{\isacharequal}maxbid{\isacharprime}\ \isacommand{let}\isamarkupfalse%
\ {\isacharquery}B{\isacharequal}{\isachardoublequoteopen}{\isacharquery}m\ a\ N\ G{\isachardoublequoteclose}\ \isacommand{have}\isamarkupfalse%
\ \isanewline
{\isadigit{2}}{\isacharcolon}\ {\isachardoublequoteopen}{\isacharquery}p\ aa\ {\isasymsubseteq}\ N\ {\isasymtimes}\ {\isacharquery}f\ G{\isachardoublequoteclose}\ \isacommand{using}\isamarkupfalse%
\ assms\ mm{\isadigit{7}}{\isadigit{3}}c\ \isacommand{by}\isamarkupfalse%
\ {\isacharparenleft}metis\ {\isacharparenleft}lifting{\isacharcomma}\ mono{\isacharunderscore}tags{\isacharparenright}{\isacharparenright}\ \isacommand{then}\isamarkupfalse%
\ \isacommand{have}\isamarkupfalse%
\ \isanewline
{\isadigit{0}}{\isacharcolon}\ {\isachardoublequoteopen}{\isacharquery}p\ aa\ {\isasymsubseteq}\ {\isacharquery}p\ a\ {\isasymunion}\ {\isacharparenleft}N\ {\isasymtimes}\ {\isacharquery}f\ G{\isacharparenright}{\isachardoublequoteclose}\ \isacommand{by}\isamarkupfalse%
\ auto\ \isacommand{moreover}\isamarkupfalse%
\ \isacommand{have}\isamarkupfalse%
\ \isanewline
{\isadigit{1}}{\isacharcolon}\ {\isachardoublequoteopen}finite\ {\isacharparenleft}{\isacharquery}p\ aa{\isacharparenright}{\isachardoublequoteclose}\ \isacommand{using}\isamarkupfalse%
\ assms\ mm{\isadigit{4}}{\isadigit{8}}\ mm{\isadigit{5}}{\isadigit{4}}\ \isacommand{by}\isamarkupfalse%
\ blast\ \isacommand{ultimately}\isamarkupfalse%
\ \isacommand{have}\isamarkupfalse%
\ \isanewline
{\isachardoublequoteopen}real{\isacharparenleft}setsum\ {\isacharquery}B\ {\isacharparenleft}{\isacharquery}p\ a{\isacharparenright}{\isacharparenright}{\isacharminus}setsum\ {\isacharquery}B\ {\isacharparenleft}{\isacharquery}p\ aa{\isacharparenright}\ {\isacharequal}\ real{\isacharparenleft}card\ {\isacharparenleft}{\isacharquery}p\ a{\isacharparenright}{\isacharparenright}{\isacharminus}card{\isacharparenleft}{\isacharquery}p\ aa\ {\isasyminter}\ {\isacharparenleft}{\isacharquery}p\ a{\isacharparenright}{\isacharparenright}{\isachardoublequoteclose}\ \isanewline
\isacommand{using}\isamarkupfalse%
\ mm{\isadigit{2}}{\isadigit{8}}d\ \isacommand{by}\isamarkupfalse%
\ fast\isanewline
\isacommand{moreover}\isamarkupfalse%
\ \isacommand{have}\isamarkupfalse%
\ {\isachardoublequoteopen}{\isachardot}{\isachardot}{\isachardot}\ {\isacharequal}\ real\ {\isacharparenleft}card\ G{\isacharparenright}\ {\isacharminus}\ card\ {\isacharparenleft}{\isacharquery}p\ aa\ {\isasyminter}\ {\isacharparenleft}{\isacharquery}p\ a{\isacharparenright}{\isacharparenright}{\isachardoublequoteclose}\ \isacommand{using}\isamarkupfalse%
\ assms\ mm{\isadigit{4}}{\isadigit{8}}\ \isanewline
\isacommand{by}\isamarkupfalse%
\ {\isacharparenleft}metis\ {\isacharparenleft}lifting{\isacharcomma}\ mono{\isacharunderscore}tags{\isacharparenright}{\isacharparenright}\isanewline
\isacommand{ultimately}\isamarkupfalse%
\ \isacommand{show}\isamarkupfalse%
\ {\isacharquery}thesis\ \isacommand{by}\isamarkupfalse%
\ presburger\isanewline
\isacommand{qed}\isamarkupfalse%
%
\endisatagproof
{\isafoldproof}%
%
\isadelimproof
\isanewline
%
\endisadelimproof
\isanewline
\isacommand{lemma}\isamarkupfalse%
\ mm{\isadigit{6}}{\isadigit{6}}e{\isacharcolon}\ {\isachardoublequoteopen}LinearCompletion\ bids\ N\ G\ {\isacharequal}\ graph\ {\isacharparenleft}N\ {\isasymtimes}\ {\isacharparenleft}Pow\ G{\isacharminus}{\isacharbraceleft}{\isacharbraceleft}{\isacharbraceright}{\isacharbraceright}{\isacharparenright}{\isacharparenright}\ {\isacharparenleft}test\ bids{\isacharparenright}{\isachardoublequoteclose}\ \isanewline
%
\isadelimproof
%
\endisadelimproof
%
\isatagproof
\isacommand{unfolding}\isamarkupfalse%
\ graph{\isacharunderscore}def\ \isacommand{using}\isamarkupfalse%
\ mm{\isadigit{6}}{\isadigit{6}}\ \isacommand{by}\isamarkupfalse%
\ blast%
\endisatagproof
{\isafoldproof}%
%
\isadelimproof
\isanewline
%
\endisadelimproof
\isacommand{lemma}\isamarkupfalse%
\ ll{\isadigit{3}}{\isadigit{3}}b{\isacharcolon}\ \isakeyword{assumes}\ {\isachardoublequoteopen}x{\isasymin}X{\isachardoublequoteclose}\ \isakeyword{shows}\ {\isachardoublequoteopen}toFunction\ {\isacharparenleft}graph\ X\ f{\isacharparenright}\ x\ {\isacharequal}\ f\ x{\isachardoublequoteclose}%
\isadelimproof
\ %
\endisadelimproof
%
\isatagproof
\isacommand{using}\isamarkupfalse%
\ assms\ \isanewline
\isacommand{by}\isamarkupfalse%
\ {\isacharparenleft}metis\ ll{\isadigit{3}}{\isadigit{3}}\ toFunction{\isacharunderscore}def{\isacharparenright}%
\endisatagproof
{\isafoldproof}%
%
\isadelimproof
%
\endisadelimproof
\isanewline
\isacommand{corollary}\isamarkupfalse%
\ ll{\isadigit{3}}{\isadigit{3}}c{\isacharcolon}\ \isakeyword{assumes}\ {\isachardoublequoteopen}pair\ {\isasymin}\ N\ {\isasymtimes}\ {\isacharparenleft}Pow\ G{\isacharminus}{\isacharbraceleft}{\isacharbraceleft}{\isacharbraceright}{\isacharbraceright}{\isacharparenright}{\isachardoublequoteclose}\ \isakeyword{shows}\ {\isachardoublequoteopen}linearCompletion{\isacharprime}\ bids\ N\ G\ pair{\isacharequal}test\ bids\ pair{\isachardoublequoteclose}\isanewline
%
\isadelimproof
%
\endisadelimproof
%
\isatagproof
\isacommand{using}\isamarkupfalse%
\ assms\ ll{\isadigit{3}}{\isadigit{3}}b\ mm{\isadigit{6}}{\isadigit{6}}e\ \isacommand{by}\isamarkupfalse%
\ {\isacharparenleft}metis{\isacharparenleft}mono{\isacharunderscore}tags{\isacharparenright}{\isacharparenright}%
\endisatagproof
{\isafoldproof}%
%
\isadelimproof
\isanewline
%
\endisadelimproof
\isanewline
\isacommand{lemma}\isamarkupfalse%
\ lm{\isadigit{0}}{\isadigit{3}}{\isadigit{1}}{\isacharcolon}\ {\isachardoublequoteopen}test\ {\isacharparenleft}real\ {\isasymcirc}\ {\isacharparenleft}{\isacharparenleft}bids{\isacharcolon}{\isacharcolon}\ {\isacharunderscore}\ {\isacharequal}{\isachargreater}\ nat{\isacharparenright}{\isacharparenright}{\isacharparenright}\ pair\ {\isacharequal}\ real\ {\isacharparenleft}test\ bids\ pair{\isacharparenright}{\isachardoublequoteclose}\ {\isacharparenleft}\isakeyword{is}\ {\isachardoublequoteopen}{\isacharquery}L{\isacharequal}{\isacharquery}R{\isachardoublequoteclose}{\isacharparenright}\isanewline
%
\isadelimproof
%
\endisadelimproof
%
\isatagproof
\isacommand{by}\isamarkupfalse%
\ simp%
\endisatagproof
{\isafoldproof}%
%
\isadelimproof
\isanewline
%
\endisadelimproof
\isacommand{lemma}\isamarkupfalse%
\ lm{\isadigit{0}}{\isadigit{3}}{\isadigit{1}}b{\isacharcolon}\ \isakeyword{assumes}\ {\isachardoublequoteopen}pair\ {\isasymin}\ N\ {\isasymtimes}\ {\isacharparenleft}Pow\ G{\isacharminus}{\isacharbraceleft}{\isacharbraceleft}{\isacharbraceright}{\isacharbraceright}{\isacharparenright}{\isachardoublequoteclose}\ \isakeyword{shows}\ \isanewline
{\isachardoublequoteopen}linearCompletion{\isacharprime}\ {\isacharparenleft}real{\isasymcirc}{\isacharparenleft}bids{\isacharcolon}{\isacharcolon}\ {\isacharunderscore}\ {\isacharequal}{\isachargreater}\ nat{\isacharparenright}{\isacharparenright}\ N\ G\ pair\ {\isacharequal}\ real\ {\isacharparenleft}linearCompletion{\isacharprime}\ bids\ N\ G\ pair{\isacharparenright}{\isachardoublequoteclose}\ \isanewline
%
\isadelimproof
%
\endisadelimproof
%
\isatagproof
\isacommand{using}\isamarkupfalse%
\ assms\ ll{\isadigit{3}}{\isadigit{3}}c\ lm{\isadigit{0}}{\isadigit{3}}{\isadigit{1}}\ \isacommand{by}\isamarkupfalse%
\ {\isacharparenleft}metis{\isacharparenleft}no{\isacharunderscore}types{\isacharparenright}{\isacharparenright}%
\endisatagproof
{\isafoldproof}%
%
\isadelimproof
\isanewline
%
\endisadelimproof
\isacommand{corollary}\isamarkupfalse%
\ lm{\isadigit{0}}{\isadigit{3}}{\isadigit{1}}c{\isacharcolon}\ \isakeyword{assumes}\ {\isachardoublequoteopen}X\ {\isasymsubseteq}\ N\ {\isasymtimes}\ {\isacharparenleft}Pow\ G\ {\isacharminus}\ {\isacharbraceleft}{\isacharbraceleft}{\isacharbraceright}{\isacharbraceright}{\isacharparenright}{\isachardoublequoteclose}\ \isakeyword{shows}\ {\isachardoublequoteopen}{\isasymforall}pair\ {\isasymin}\ X{\isachardot}\ \isanewline
linearCompletion{\isacharprime}\ {\isacharparenleft}real\ {\isasymcirc}\ {\isacharparenleft}bids{\isacharcolon}{\isacharcolon}{\isacharunderscore}{\isacharequal}{\isachargreater}nat{\isacharparenright}{\isacharparenright}\ N\ G\ pair{\isacharequal}\ {\isacharparenleft}real\ {\isasymcirc}\ {\isacharparenleft}linearCompletion{\isacharprime}\ bids\ N\ G{\isacharparenright}{\isacharparenright}\ pair{\isachardoublequoteclose}\isanewline
%
\isadelimproof
%
\endisadelimproof
%
\isatagproof
\isacommand{using}\isamarkupfalse%
\ assms\ lm{\isadigit{0}}{\isadigit{3}}{\isadigit{1}}b\ \isanewline
\isacommand{proof}\isamarkupfalse%
\ {\isacharminus}\isanewline
\ \ \isacommand{{\isacharbraceleft}}\isamarkupfalse%
\ \isacommand{fix}\isamarkupfalse%
\ esk{\isadigit{4}}{\isadigit{8}}\isactrlsub {\isadigit{0}}\ {\isacharcolon}{\isacharcolon}\ {\isachardoublequoteopen}{\isacharprime}a\ {\isasymtimes}\ {\isacharprime}b\ set{\isachardoublequoteclose}\isanewline
\ \ \ \ \isacommand{{\isacharbraceleft}}\isamarkupfalse%
\ \isacommand{assume}\isamarkupfalse%
\ {\isachardoublequoteopen}esk{\isadigit{4}}{\isadigit{8}}\isactrlsub {\isadigit{0}}\ {\isasymin}\ N\ {\isasymtimes}\ {\isacharparenleft}Pow\ G\ {\isacharminus}\ {\isacharbraceleft}{\isacharbraceleft}{\isacharbraceright}{\isacharbraceright}{\isacharparenright}{\isachardoublequoteclose}\isanewline
\ \ \ \ \ \ \isacommand{hence}\isamarkupfalse%
\ {\isachardoublequoteopen}linearCompletion{\isacharprime}\ {\isacharparenleft}real\ {\isasymcirc}\ bids{\isacharparenright}\ N\ G\ esk{\isadigit{4}}{\isadigit{8}}\isactrlsub {\isadigit{0}}\ {\isacharequal}\ real\ {\isacharparenleft}linearCompletion{\isacharprime}\ bids\ N\ G\ esk{\isadigit{4}}{\isadigit{8}}\isactrlsub {\isadigit{0}}{\isacharparenright}{\isachardoublequoteclose}\ \isacommand{using}\isamarkupfalse%
\ lm{\isadigit{0}}{\isadigit{3}}{\isadigit{1}}b\ \isacommand{by}\isamarkupfalse%
\ blast\isanewline
\ \ \ \ \ \ \isacommand{hence}\isamarkupfalse%
\ {\isachardoublequoteopen}esk{\isadigit{4}}{\isadigit{8}}\isactrlsub {\isadigit{0}}\ {\isasymnotin}\ X\ {\isasymor}\ linearCompletion{\isacharprime}\ {\isacharparenleft}real\ {\isasymcirc}\ bids{\isacharparenright}\ N\ G\ esk{\isadigit{4}}{\isadigit{8}}\isactrlsub {\isadigit{0}}\ {\isacharequal}\ {\isacharparenleft}real\ {\isasymcirc}\ linearCompletion{\isacharprime}\ bids\ N\ G{\isacharparenright}\ esk{\isadigit{4}}{\isadigit{8}}\isactrlsub {\isadigit{0}}{\isachardoublequoteclose}\ \isacommand{by}\isamarkupfalse%
\ simp\ \isacommand{{\isacharbraceright}}\isamarkupfalse%
\isanewline
\ \ \ \ \isacommand{hence}\isamarkupfalse%
\ {\isachardoublequoteopen}esk{\isadigit{4}}{\isadigit{8}}\isactrlsub {\isadigit{0}}\ {\isasymnotin}\ X\ {\isasymor}\ linearCompletion{\isacharprime}\ {\isacharparenleft}real\ {\isasymcirc}\ bids{\isacharparenright}\ N\ G\ esk{\isadigit{4}}{\isadigit{8}}\isactrlsub {\isadigit{0}}\ {\isacharequal}\ {\isacharparenleft}real\ {\isasymcirc}\ linearCompletion{\isacharprime}\ bids\ N\ G{\isacharparenright}\ esk{\isadigit{4}}{\isadigit{8}}\isactrlsub {\isadigit{0}}{\isachardoublequoteclose}\ \isacommand{using}\isamarkupfalse%
\ assms\ \isacommand{by}\isamarkupfalse%
\ blast\ \isacommand{{\isacharbraceright}}\isamarkupfalse%
\isanewline
\ \ \isacommand{thus}\isamarkupfalse%
\ {\isachardoublequoteopen}{\isasymforall}pair{\isasymin}X{\isachardot}\ linearCompletion{\isacharprime}\ {\isacharparenleft}real\ {\isasymcirc}\ bids{\isacharparenright}\ N\ G\ pair\ {\isacharequal}\ {\isacharparenleft}real\ {\isasymcirc}\ linearCompletion{\isacharprime}\ bids\ N\ G{\isacharparenright}\ pair{\isachardoublequoteclose}\ \isacommand{by}\isamarkupfalse%
\ blast\isanewline
\isacommand{qed}\isamarkupfalse%
%
\endisatagproof
{\isafoldproof}%
%
\isadelimproof
\isanewline
%
\endisadelimproof
\isanewline
\isacommand{corollary}\isamarkupfalse%
\ lm{\isadigit{0}}{\isadigit{3}}{\isadigit{1}}e{\isacharcolon}\ \isakeyword{assumes}\ {\isachardoublequoteopen}aa\ {\isasymsubseteq}\ N\ {\isasymtimes}\ {\isacharparenleft}Pow\ G{\isacharminus}{\isacharbraceleft}{\isacharbraceleft}{\isacharbraceright}{\isacharbraceright}{\isacharparenright}{\isachardoublequoteclose}\ \isakeyword{shows}\isanewline
{\isachardoublequoteopen}setsum\ {\isacharparenleft}{\isacharparenleft}linearCompletion{\isacharprime}\ {\isacharparenleft}real\ {\isasymcirc}\ {\isacharparenleft}bids{\isacharcolon}{\isacharcolon}{\isacharunderscore}{\isacharequal}{\isachargreater}nat{\isacharparenright}{\isacharparenright}\ N\ G{\isacharparenright}{\isacharparenright}\ aa\ {\isacharequal}\ real\ {\isacharparenleft}setsum\ {\isacharparenleft}{\isacharparenleft}linearCompletion{\isacharprime}\ bids\ N\ G{\isacharparenright}{\isacharparenright}\ aa{\isacharparenright}{\isachardoublequoteclose}\ \isanewline
{\isacharparenleft}\isakeyword{is}\ {\isachardoublequoteopen}{\isacharquery}L{\isacharequal}{\isacharquery}R{\isachardoublequoteclose}{\isacharparenright}\isanewline
%
\isadelimproof
%
\endisadelimproof
%
\isatagproof
\isacommand{proof}\isamarkupfalse%
\ {\isacharminus}\isanewline
\isacommand{have}\isamarkupfalse%
\ {\isachardoublequoteopen}{\isasymforall}\ pair\ {\isasymin}\ aa{\isachardot}\ linearCompletion{\isacharprime}\ {\isacharparenleft}real\ {\isasymcirc}\ bids{\isacharparenright}\ N\ G\ pair\ {\isacharequal}\ {\isacharparenleft}real\ {\isasymcirc}\ {\isacharparenleft}linearCompletion{\isacharprime}\ bids\ N\ G{\isacharparenright}{\isacharparenright}\ pair{\isachardoublequoteclose}\isanewline
\isacommand{using}\isamarkupfalse%
\ assms\ \isacommand{by}\isamarkupfalse%
\ {\isacharparenleft}rule\ lm{\isadigit{0}}{\isadigit{3}}{\isadigit{1}}c{\isacharparenright}\isanewline
\isacommand{then}\isamarkupfalse%
\ \isacommand{have}\isamarkupfalse%
\ {\isachardoublequoteopen}{\isacharquery}L\ {\isacharequal}\ setsum\ {\isacharparenleft}real{\isasymcirc}{\isacharparenleft}linearCompletion{\isacharprime}\ bids\ N\ G{\isacharparenright}{\isacharparenright}\ aa{\isachardoublequoteclose}\ \isacommand{using}\isamarkupfalse%
\ setsum{\isachardot}cong\ \isacommand{by}\isamarkupfalse%
\ force\isanewline
\isacommand{then}\isamarkupfalse%
\ \isacommand{show}\isamarkupfalse%
\ {\isacharquery}thesis\ \isacommand{by}\isamarkupfalse%
\ simp\isanewline
\isacommand{qed}\isamarkupfalse%
%
\endisatagproof
{\isafoldproof}%
%
\isadelimproof
\isanewline
%
\endisadelimproof
\isanewline
\isacommand{corollary}\isamarkupfalse%
\ lm{\isadigit{0}}{\isadigit{3}}{\isadigit{1}}d{\isacharcolon}\ \isakeyword{assumes}\ {\isachardoublequoteopen}aa\ {\isasymin}\ possibleAllocationsRel\ N\ G{\isachardoublequoteclose}\ \isakeyword{shows}\isanewline
{\isachardoublequoteopen}setsum\ {\isacharparenleft}{\isacharparenleft}linearCompletion{\isacharprime}\ {\isacharparenleft}real\ {\isasymcirc}\ {\isacharparenleft}bids{\isacharcolon}{\isacharcolon}{\isacharunderscore}{\isacharequal}{\isachargreater}nat{\isacharparenright}{\isacharparenright}\ N\ G{\isacharparenright}{\isacharparenright}\ aa\ {\isacharequal}\ real\ {\isacharparenleft}setsum\ {\isacharparenleft}{\isacharparenleft}linearCompletion{\isacharprime}\ bids\ N\ G{\isacharparenright}{\isacharparenright}\ aa{\isacharparenright}{\isachardoublequoteclose}\ \isanewline
%
\isadelimproof
%
\endisadelimproof
%
\isatagproof
\isacommand{using}\isamarkupfalse%
\ assms\ lm{\isadigit{0}}{\isadigit{3}}{\isadigit{1}}e\ mm{\isadigit{6}}{\isadigit{3}}c\ \isacommand{by}\isamarkupfalse%
\ {\isacharparenleft}metis{\isacharparenleft}lifting{\isacharcomma}mono{\isacharunderscore}tags{\isacharparenright}{\isacharparenright}%
\endisatagproof
{\isafoldproof}%
%
\isadelimproof
\isanewline
%
\endisadelimproof
\isanewline
\isacommand{corollary}\isamarkupfalse%
\ mm{\isadigit{7}}{\isadigit{0}}b{\isacharcolon}\ \isanewline
\isakeyword{assumes}\ {\isachardoublequoteopen}finite\ G{\isachardoublequoteclose}\ {\isachardoublequoteopen}a\ {\isasymin}\ possibleAllocationsRel\ N\ G{\isachardoublequoteclose}\ {\isachardoublequoteopen}aa\ {\isasymin}\ possibleAllocationsRel\ N\ G{\isachardoublequoteclose}\isanewline
\isakeyword{shows}\ \isanewline
{\isachardoublequoteopen}real\ {\isacharparenleft}setsum\ {\isacharparenleft}tiebids{\isacharprime}\ a\ N\ G{\isacharparenright}\ a{\isacharparenright}\ {\isacharminus}\ setsum\ {\isacharparenleft}tiebids{\isacharprime}\ a\ N\ G{\isacharparenright}\ aa\ {\isacharequal}\ \isanewline
real\ {\isacharparenleft}card\ G{\isacharparenright}\ {\isacharminus}\ card\ {\isacharparenleft}pseudoAllocation\ aa\ {\isasyminter}\ {\isacharparenleft}pseudoAllocation\ a{\isacharparenright}{\isacharparenright}{\isachardoublequoteclose}\ {\isacharparenleft}\isakeyword{is}\ {\isachardoublequoteopen}{\isacharquery}L{\isacharequal}{\isacharquery}R{\isachardoublequoteclose}{\isacharparenright}\isanewline
%
\isadelimproof
%
\endisadelimproof
%
\isatagproof
\isacommand{proof}\isamarkupfalse%
\ {\isacharminus}\isanewline
\ \ \isacommand{let}\isamarkupfalse%
\ {\isacharquery}l{\isacharequal}linearCompletion{\isacharprime}\ \isacommand{let}\isamarkupfalse%
\ {\isacharquery}m{\isacharequal}maxbid{\isacharprime}\ \isacommand{let}\isamarkupfalse%
\ {\isacharquery}s{\isacharequal}setsum\ \isacommand{let}\isamarkupfalse%
\ {\isacharquery}p{\isacharequal}pseudoAllocation\isanewline
\ \ \isacommand{let}\isamarkupfalse%
\ {\isacharquery}bb{\isacharequal}{\isachardoublequoteopen}{\isacharquery}m\ a\ N\ G{\isachardoublequoteclose}\ \isacommand{let}\isamarkupfalse%
\ {\isacharquery}b{\isacharequal}{\isachardoublequoteopen}real\ {\isasymcirc}\ {\isacharparenleft}{\isacharquery}m\ a\ N\ G{\isacharparenright}{\isachardoublequoteclose}\ \ \isanewline
\ \ \isacommand{have}\isamarkupfalse%
\ {\isachardoublequoteopen}real\ {\isacharparenleft}{\isacharquery}s\ {\isacharquery}bb\ {\isacharparenleft}{\isacharquery}p\ a{\isacharparenright}{\isacharparenright}\ {\isacharminus}\ {\isacharparenleft}{\isacharquery}s\ {\isacharquery}bb\ {\isacharparenleft}{\isacharquery}p\ aa{\isacharparenright}{\isacharparenright}\ {\isacharequal}\ {\isacharquery}R{\isachardoublequoteclose}\ \isacommand{using}\isamarkupfalse%
\ assms\ mm{\isadigit{4}}{\isadigit{9}}b\ \isacommand{by}\isamarkupfalse%
\ blast\ \isanewline
\ \ \isacommand{then}\isamarkupfalse%
\ \isacommand{have}\isamarkupfalse%
\ {\isachardoublequoteopen}{\isacharquery}R\ {\isacharequal}\ real\ {\isacharparenleft}{\isacharquery}s\ {\isacharquery}bb\ {\isacharparenleft}{\isacharquery}p\ a{\isacharparenright}{\isacharparenright}\ {\isacharminus}\ {\isacharparenleft}{\isacharquery}s\ {\isacharquery}bb\ {\isacharparenleft}{\isacharquery}p\ aa{\isacharparenright}{\isacharparenright}{\isachardoublequoteclose}\ \isacommand{by}\isamarkupfalse%
\ presburger\isanewline
\ \ \isacommand{have}\isamarkupfalse%
\ {\isachardoublequoteopen}\ {\isacharquery}s\ {\isacharparenleft}{\isacharquery}l\ {\isacharquery}b\ N\ G{\isacharparenright}\ aa\ {\isacharequal}\ {\isacharquery}s\ {\isacharquery}b\ {\isacharparenleft}{\isacharquery}p\ aa{\isacharparenright}{\isachardoublequoteclose}\ \isacommand{using}\isamarkupfalse%
\ assms\ mm{\isadigit{6}}{\isadigit{9}}\ \isacommand{by}\isamarkupfalse%
\ blast\ \isacommand{moreover}\isamarkupfalse%
\ \isacommand{have}\isamarkupfalse%
\ \isanewline
\ \ {\isachardoublequoteopen}{\isachardot}{\isachardot}{\isachardot}\ {\isacharequal}\ {\isacharquery}s\ {\isacharquery}bb\ {\isacharparenleft}{\isacharquery}p\ aa{\isacharparenright}{\isachardoublequoteclose}\ \isacommand{by}\isamarkupfalse%
\ fastforce\ \isanewline
\isacommand{moreover}\isamarkupfalse%
\ \isacommand{have}\isamarkupfalse%
\ {\isachardoublequoteopen}{\isacharparenleft}{\isacharquery}s\ {\isacharparenleft}{\isacharquery}l\ {\isacharquery}b\ N\ G{\isacharparenright}\ aa{\isacharparenright}\ {\isacharequal}\ real\ {\isacharparenleft}{\isacharquery}s\ {\isacharparenleft}{\isacharquery}l\ {\isacharquery}bb\ N\ G{\isacharparenright}\ aa{\isacharparenright}{\isachardoublequoteclose}\ \isacommand{using}\isamarkupfalse%
\ assms{\isacharparenleft}{\isadigit{3}}{\isacharparenright}\ \isacommand{by}\isamarkupfalse%
\ {\isacharparenleft}rule\ lm{\isadigit{0}}{\isadigit{3}}{\isadigit{1}}d{\isacharparenright}\isanewline
\isanewline
\isacommand{ultimately}\isamarkupfalse%
\ \isacommand{have}\isamarkupfalse%
\ \isanewline
\ \ {\isadigit{1}}{\isacharcolon}\ {\isachardoublequoteopen}{\isacharquery}R\ {\isacharequal}\ real\ {\isacharparenleft}{\isacharquery}s\ {\isacharquery}bb\ {\isacharparenleft}{\isacharquery}p\ a{\isacharparenright}{\isacharparenright}\ {\isacharminus}\ {\isacharparenleft}{\isacharquery}s\ {\isacharparenleft}{\isacharquery}l\ {\isacharquery}bb\ N\ G{\isacharparenright}\ aa{\isacharparenright}{\isachardoublequoteclose}\ \isanewline
\isacommand{by}\isamarkupfalse%
\ {\isacharparenleft}metis\ {\isacharbackquoteopen}real\ {\isacharparenleft}card\ G{\isacharparenright}\ {\isacharminus}\ real\ {\isacharparenleft}card\ {\isacharparenleft}pseudoAllocation\ aa\ {\isasyminter}\ pseudoAllocation\ a{\isacharparenright}{\isacharparenright}\ {\isacharequal}\ real\ {\isacharparenleft}setsum\ {\isacharparenleft}pseudoAllocation\ a\ {\isacharless}{\isacharbar}\ {\isacharparenleft}N\ {\isasymtimes}\ finestpart\ G{\isacharparenright}{\isacharparenright}\ {\isacharparenleft}pseudoAllocation\ a{\isacharparenright}{\isacharparenright}\ {\isacharminus}\ real\ {\isacharparenleft}setsum\ {\isacharparenleft}pseudoAllocation\ a\ {\isacharless}{\isacharbar}\ {\isacharparenleft}N\ {\isasymtimes}\ finestpart\ G{\isacharparenright}{\isacharparenright}\ {\isacharparenleft}pseudoAllocation\ aa{\isacharparenright}{\isacharparenright}{\isacharbackquoteclose}{\isacharparenright}\isanewline
\ \ \isacommand{have}\isamarkupfalse%
\ {\isachardoublequoteopen}{\isacharquery}s\ {\isacharparenleft}{\isacharquery}l\ {\isacharquery}b\ N\ G{\isacharparenright}\ a{\isacharequal}{\isacharparenleft}{\isacharquery}s\ {\isacharquery}b\ {\isacharparenleft}{\isacharquery}p\ a{\isacharparenright}{\isacharparenright}{\isachardoublequoteclose}\ \isacommand{using}\isamarkupfalse%
\ assms\ mm{\isadigit{6}}{\isadigit{9}}\ \isacommand{by}\isamarkupfalse%
\ blast\isanewline
\ \ \isacommand{moreover}\isamarkupfalse%
\ \isacommand{have}\isamarkupfalse%
\ {\isachardoublequoteopen}{\isachardot}{\isachardot}{\isachardot}\ {\isacharequal}\ {\isacharquery}s\ {\isacharquery}bb\ {\isacharparenleft}{\isacharquery}p\ a{\isacharparenright}{\isachardoublequoteclose}\ \isacommand{by}\isamarkupfalse%
\ force\isanewline
\ \ \isacommand{moreover}\isamarkupfalse%
\ \isacommand{have}\isamarkupfalse%
\ {\isachardoublequoteopen}{\isachardot}{\isachardot}{\isachardot}\ {\isacharequal}\ real\ {\isacharparenleft}{\isacharquery}s\ {\isacharquery}bb\ {\isacharparenleft}{\isacharquery}p\ a{\isacharparenright}{\isacharparenright}{\isachardoublequoteclose}\ \isacommand{by}\isamarkupfalse%
\ fast\isanewline
\ \ \isacommand{moreover}\isamarkupfalse%
\ \isacommand{have}\isamarkupfalse%
\ {\isachardoublequoteopen}{\isacharquery}s\ {\isacharparenleft}{\isacharquery}l\ {\isacharquery}b\ N\ G{\isacharparenright}\ a\ {\isacharequal}\ real\ {\isacharparenleft}{\isacharquery}s\ {\isacharparenleft}{\isacharquery}l\ {\isacharquery}bb\ N\ G{\isacharparenright}\ a{\isacharparenright}{\isachardoublequoteclose}\ \isacommand{using}\isamarkupfalse%
\ assms{\isacharparenleft}{\isadigit{2}}{\isacharparenright}\ \isacommand{by}\isamarkupfalse%
\ {\isacharparenleft}rule\ lm{\isadigit{0}}{\isadigit{3}}{\isadigit{1}}d{\isacharparenright}\isanewline
\ \ \isacommand{ultimately}\isamarkupfalse%
\ \isacommand{have}\isamarkupfalse%
\ {\isachardoublequoteopen}{\isacharquery}s\ {\isacharparenleft}{\isacharquery}l\ {\isacharquery}bb\ N\ G{\isacharparenright}\ a\ {\isacharequal}\ real\ {\isacharparenleft}{\isacharquery}s\ {\isacharquery}bb\ {\isacharparenleft}{\isacharquery}p\ a{\isacharparenright}{\isacharparenright}{\isachardoublequoteclose}%
\endisatagproof
{\isafoldproof}%
%
\isadelimproof
%
\endisadelimproof
\ \isacommand{try{\isadigit{0}}}\isamarkupfalse%
\isanewline
%
\isadelimproof
%
\endisadelimproof
%
\isatagproof
\isacommand{by}\isamarkupfalse%
\ presburger\isanewline
\ \ \isacommand{thus}\isamarkupfalse%
\ {\isacharquery}thesis\ \isacommand{using}\isamarkupfalse%
\ {\isadigit{1}}\ \isacommand{by}\isamarkupfalse%
\ presburger\isanewline
\isacommand{qed}\isamarkupfalse%
%
\endisatagproof
{\isafoldproof}%
%
\isadelimproof
\isanewline
%
\endisadelimproof
\isanewline
\isacommand{corollary}\isamarkupfalse%
\ mm{\isadigit{7}}{\isadigit{0}}c{\isacharcolon}\ \isakeyword{assumes}\ {\isachardoublequoteopen}finite\ G{\isachardoublequoteclose}\ {\isachardoublequoteopen}a\ {\isasymin}\ possibleAllocationsRel\ N\ G{\isachardoublequoteclose}\ {\isachardoublequoteopen}aa\ {\isasymin}\ possibleAllocationsRel\ N\ G{\isachardoublequoteclose}\isanewline
{\isachardoublequoteopen}x{\isacharequal}real\ {\isacharparenleft}setsum\ {\isacharparenleft}tiebids{\isacharprime}\ a\ N\ G{\isacharparenright}\ a{\isacharparenright}\ {\isacharminus}\ setsum\ {\isacharparenleft}tiebids{\isacharprime}\ a\ N\ G{\isacharparenright}\ aa{\isachardoublequoteclose}\ \isakeyword{shows}\isanewline
{\isachardoublequoteopen}x\ {\isacharless}{\isacharequal}\ card\ G\ {\isacharampersand}\ x\ {\isasymge}\ {\isadigit{0}}\ {\isacharampersand}\ {\isacharparenleft}x{\isacharequal}{\isadigit{0}}\ {\isasymlongleftrightarrow}\ a\ {\isacharequal}\ aa{\isacharparenright}\ {\isacharampersand}\ {\isacharparenleft}aa\ {\isasymnoteq}\ a\ {\isasymlongrightarrow}\ setsum\ {\isacharparenleft}tiebids{\isacharprime}\ a\ N\ G{\isacharparenright}\ aa\ {\isacharless}\ setsum\ {\isacharparenleft}tiebids{\isacharprime}\ a\ N\ G{\isacharparenright}\ a{\isacharparenright}{\isachardoublequoteclose}\isanewline
%
\isadelimproof
%
\endisadelimproof
%
\isatagproof
\isacommand{proof}\isamarkupfalse%
\ {\isacharminus}\isanewline
\isacommand{let}\isamarkupfalse%
\ {\isacharquery}p{\isacharequal}pseudoAllocation\ \isacommand{have}\isamarkupfalse%
\ {\isachardoublequoteopen}real\ {\isacharparenleft}card\ G{\isacharparenright}\ {\isachargreater}{\isacharequal}\ real\ {\isacharparenleft}card\ G{\isacharparenright}\ {\isacharminus}\ card\ {\isacharparenleft}{\isacharquery}p\ aa\ {\isasyminter}\ {\isacharparenleft}{\isacharquery}p\ a{\isacharparenright}{\isacharparenright}{\isachardoublequoteclose}\ \isacommand{by}\isamarkupfalse%
\ force\isanewline
\isacommand{moreover}\isamarkupfalse%
\ \isacommand{have}\isamarkupfalse%
\ \isanewline
{\isachardoublequoteopen}real\ {\isacharparenleft}setsum\ {\isacharparenleft}tiebids{\isacharprime}\ a\ N\ G{\isacharparenright}\ a{\isacharparenright}\ {\isacharminus}\ setsum\ {\isacharparenleft}tiebids{\isacharprime}\ a\ N\ G{\isacharparenright}\ aa\ {\isacharequal}\ \isanewline
real\ {\isacharparenleft}card\ G{\isacharparenright}\ {\isacharminus}\ card\ {\isacharparenleft}pseudoAllocation\ aa\ {\isasyminter}\ {\isacharparenleft}pseudoAllocation\ a{\isacharparenright}{\isacharparenright}{\isachardoublequoteclose}\isanewline
\isacommand{using}\isamarkupfalse%
\ assms\ mm{\isadigit{7}}{\isadigit{0}}b\ \isacommand{by}\isamarkupfalse%
\ blast\ \isacommand{ultimately}\isamarkupfalse%
\ \isacommand{have}\isamarkupfalse%
\isanewline
{\isadigit{4}}{\isacharcolon}\ {\isachardoublequoteopen}x{\isacharequal}real{\isacharparenleft}card\ G{\isacharparenright}{\isacharminus}card{\isacharparenleft}pseudoAllocation\ aa{\isasyminter}{\isacharparenleft}pseudoAllocation\ a{\isacharparenright}{\isacharparenright}{\isachardoublequoteclose}\ \isacommand{using}\isamarkupfalse%
\ assms\ \isacommand{by}\isamarkupfalse%
\ force\ \isacommand{then}\isamarkupfalse%
\ \isacommand{have}\isamarkupfalse%
\isanewline
{\isadigit{1}}{\isacharcolon}\ {\isachardoublequoteopen}x\ {\isasymle}\ real\ {\isacharparenleft}card\ G{\isacharparenright}{\isachardoublequoteclose}\ \isacommand{using}\isamarkupfalse%
\ assms\ \isacommand{by}\isamarkupfalse%
\ linarith\ \isacommand{have}\isamarkupfalse%
\ \isanewline
{\isadigit{0}}{\isacharcolon}\ {\isachardoublequoteopen}card\ {\isacharparenleft}{\isacharquery}p\ aa{\isacharparenright}\ {\isacharequal}\ card\ G\ {\isacharampersand}\ card\ {\isacharparenleft}{\isacharquery}p\ a{\isacharparenright}\ {\isacharequal}\ card\ G{\isachardoublequoteclose}\ \isacommand{using}\isamarkupfalse%
\ assms\ mm{\isadigit{4}}{\isadigit{8}}\ \isacommand{by}\isamarkupfalse%
\ blast\ \isanewline
\isacommand{moreover}\isamarkupfalse%
\ \isacommand{have}\isamarkupfalse%
\ {\isachardoublequoteopen}finite\ {\isacharparenleft}{\isacharquery}p\ aa{\isacharparenright}\ {\isacharampersand}\ finite\ {\isacharparenleft}{\isacharquery}p\ a{\isacharparenright}{\isachardoublequoteclose}\ \isacommand{using}\isamarkupfalse%
\ assms\ mm{\isadigit{5}}{\isadigit{4}}\ \isacommand{by}\isamarkupfalse%
\ blast\ \isacommand{ultimately}\isamarkupfalse%
\isanewline
\isacommand{have}\isamarkupfalse%
\ {\isachardoublequoteopen}card\ {\isacharparenleft}{\isacharquery}p\ aa\ {\isasyminter}\ {\isacharquery}p\ a{\isacharparenright}\ {\isasymle}\ card\ G{\isachardoublequoteclose}\ \isacommand{using}\isamarkupfalse%
\ Int{\isacharunderscore}lower{\isadigit{2}}\ card{\isacharunderscore}mono\ \isacommand{by}\isamarkupfalse%
\ fastforce\ \isacommand{then}\isamarkupfalse%
\ \isacommand{have}\isamarkupfalse%
\ \isanewline
{\isadigit{2}}{\isacharcolon}\ {\isachardoublequoteopen}x\ {\isasymge}\ {\isadigit{0}}{\isachardoublequoteclose}\ \isacommand{using}\isamarkupfalse%
\ assms\ mm{\isadigit{7}}{\isadigit{0}}b\ {\isadigit{4}}\ \isacommand{by}\isamarkupfalse%
\ linarith\ \isanewline
\isacommand{have}\isamarkupfalse%
\ {\isachardoublequoteopen}card\ {\isacharparenleft}{\isacharquery}p\ aa\ {\isasyminter}\ {\isacharparenleft}{\isacharquery}p\ a{\isacharparenright}{\isacharparenright}\ {\isacharequal}\ card\ G\ {\isasymlongleftrightarrow}\ {\isacharparenleft}{\isacharquery}p\ aa\ {\isacharequal}\ {\isacharquery}p\ a{\isacharparenright}{\isachardoublequoteclose}\ \isanewline
\isacommand{using}\isamarkupfalse%
\ {\isadigit{0}}\ mm{\isadigit{5}}{\isadigit{6}}\ {\isadigit{4}}\ assms\ \isacommand{by}\isamarkupfalse%
\ {\isacharparenleft}metis\ {\isacharparenleft}lifting{\isacharcomma}\ mono{\isacharunderscore}tags{\isacharparenright}{\isacharparenright}\isanewline
\isacommand{moreover}\isamarkupfalse%
\ \isacommand{have}\isamarkupfalse%
\ {\isachardoublequoteopen}{\isacharquery}p\ aa\ {\isacharequal}\ {\isacharquery}p\ a\ {\isasymlongrightarrow}\ a\ {\isacharequal}\ aa{\isachardoublequoteclose}\ \isacommand{using}\isamarkupfalse%
\ assms\ mm{\isadigit{7}}{\isadigit{5}}j\ inj{\isacharunderscore}on{\isacharunderscore}def\ \isanewline
\isacommand{by}\isamarkupfalse%
\ {\isacharparenleft}metis\ {\isacharparenleft}lifting{\isacharcomma}\ mono{\isacharunderscore}tags{\isacharparenright}{\isacharparenright}\isanewline
\isacommand{ultimately}\isamarkupfalse%
\ \isacommand{have}\isamarkupfalse%
\ {\isachardoublequoteopen}card\ {\isacharparenleft}{\isacharquery}p\ aa\ {\isasyminter}\ {\isacharparenleft}{\isacharquery}p\ a{\isacharparenright}{\isacharparenright}\ {\isacharequal}\ card\ G\ {\isasymlongleftrightarrow}\ {\isacharparenleft}a{\isacharequal}aa{\isacharparenright}{\isachardoublequoteclose}\ \isacommand{by}\isamarkupfalse%
\ blast\isanewline
\isacommand{moreover}\isamarkupfalse%
\ \isacommand{have}\isamarkupfalse%
\ {\isachardoublequoteopen}x\ {\isacharequal}\ real\ {\isacharparenleft}card\ G{\isacharparenright}\ {\isacharminus}\ card\ {\isacharparenleft}{\isacharquery}p\ aa\ {\isasyminter}\ {\isacharparenleft}{\isacharquery}p\ a{\isacharparenright}{\isacharparenright}{\isachardoublequoteclose}\ \isacommand{using}\isamarkupfalse%
\ assms\ mm{\isadigit{7}}{\isadigit{0}}b\ \isacommand{by}\isamarkupfalse%
\ blast\isanewline
\isacommand{ultimately}\isamarkupfalse%
\ \isacommand{have}\isamarkupfalse%
\ \isanewline
{\isadigit{3}}{\isacharcolon}\ {\isachardoublequoteopen}x\ {\isacharequal}\ {\isadigit{0}}\ {\isasymlongleftrightarrow}\ {\isacharparenleft}a{\isacharequal}aa{\isacharparenright}{\isachardoublequoteclose}\ \isacommand{by}\isamarkupfalse%
\ linarith\ \isacommand{then}\isamarkupfalse%
\ \isacommand{have}\isamarkupfalse%
\ \isanewline
{\isachardoublequoteopen}aa\ {\isasymnoteq}\ a\ {\isasymlongrightarrow}\ setsum\ {\isacharparenleft}tiebids{\isacharprime}\ a\ N\ G{\isacharparenright}\ aa\ {\isacharless}\ real\ {\isacharparenleft}setsum\ {\isacharparenleft}tiebids{\isacharprime}\ a\ N\ G{\isacharparenright}\ a{\isacharparenright}{\isachardoublequoteclose}\ \isacommand{using}\isamarkupfalse%
\ {\isadigit{1}}\ {\isadigit{2}}\ assms\ \isanewline
\isacommand{by}\isamarkupfalse%
\ auto\isanewline
\isacommand{thus}\isamarkupfalse%
\ {\isacharquery}thesis\ \isacommand{using}\isamarkupfalse%
\ {\isadigit{1}}\ {\isadigit{2}}\ {\isadigit{3}}\ \isacommand{by}\isamarkupfalse%
\ force\isanewline
\isacommand{qed}\isamarkupfalse%
%
\endisatagproof
{\isafoldproof}%
%
\isadelimproof
\ \isanewline
%
\endisadelimproof
\isanewline
\isacommand{corollary}\isamarkupfalse%
\ mm{\isadigit{7}}{\isadigit{0}}d{\isacharcolon}\ \isakeyword{assumes}\ {\isachardoublequoteopen}finite\ G{\isachardoublequoteclose}\ {\isachardoublequoteopen}a\ {\isasymin}\ possibleAllocationsRel\ N\ G{\isachardoublequoteclose}\ {\isachardoublequoteopen}aa\ {\isasymin}\ possibleAllocationsRel\ N\ G{\isachardoublequoteclose}\isanewline
{\isachardoublequoteopen}aa\ {\isasymnoteq}\ a{\isachardoublequoteclose}\ \isakeyword{shows}\ {\isachardoublequoteopen}setsum\ {\isacharparenleft}tiebids{\isacharprime}\ a\ N\ G{\isacharparenright}\ aa\ {\isacharless}\ setsum\ {\isacharparenleft}tiebids{\isacharprime}\ a\ N\ G{\isacharparenright}\ a{\isachardoublequoteclose}%
\isadelimproof
\ %
\endisadelimproof
%
\isatagproof
\isacommand{using}\isamarkupfalse%
\ assms\ mm{\isadigit{7}}{\isadigit{0}}c\ \isacommand{by}\isamarkupfalse%
\ blast%
\endisatagproof
{\isafoldproof}%
%
\isadelimproof
%
\endisadelimproof
\isanewline
\isanewline
\isacommand{lemma}\isamarkupfalse%
\ mm{\isadigit{8}}{\isadigit{1}}{\isacharcolon}\ \isakeyword{assumes}\isanewline
{\isachardoublequoteopen}N\ {\isasymnoteq}\ {\isacharbraceleft}{\isacharbraceright}{\isachardoublequoteclose}\ {\isachardoublequoteopen}finite\ N{\isachardoublequoteclose}\ {\isachardoublequoteopen}distinct\ G{\isachardoublequoteclose}\ {\isachardoublequoteopen}set\ G\ {\isasymnoteq}\ {\isacharbraceleft}{\isacharbraceright}{\isachardoublequoteclose}\isanewline
{\isachardoublequoteopen}aa\ {\isasymin}\ {\isacharparenleft}possibleAllocationsRel\ N\ {\isacharparenleft}set\ G{\isacharparenright}{\isacharparenright}{\isacharminus}{\isacharbraceleft}chosenAllocation{\isacharprime}\ N\ G\ bids\ random{\isacharbraceright}{\isachardoublequoteclose}\ \isakeyword{shows}\ \isanewline
{\isachardoublequoteopen}setsum\ {\isacharparenleft}resolvingBid{\isacharprime}\ N\ G\ bids\ random{\isacharparenright}\ aa\ {\isacharless}\ setsum\ {\isacharparenleft}resolvingBid{\isacharprime}\ N\ G\ bids\ random{\isacharparenright}\ {\isacharparenleft}chosenAllocation{\isacharprime}\ N\ G\ bids\ random{\isacharparenright}{\isachardoublequoteclose}\ \isanewline
%
\isadelimproof
%
\endisadelimproof
%
\isatagproof
\isacommand{proof}\isamarkupfalse%
\ {\isacharminus}\isanewline
\isacommand{let}\isamarkupfalse%
\ {\isacharquery}a{\isacharequal}{\isachardoublequoteopen}chosenAllocation{\isacharprime}\ N\ G\ bids\ random{\isachardoublequoteclose}\ \isacommand{let}\isamarkupfalse%
\ {\isacharquery}p{\isacharequal}possibleAllocationsRel\ \isacommand{let}\isamarkupfalse%
\ {\isacharquery}G{\isacharequal}{\isachardoublequoteopen}set\ G{\isachardoublequoteclose}\isanewline
\isacommand{have}\isamarkupfalse%
\ {\isachardoublequoteopen}{\isacharquery}a\ {\isasymin}\ winningAllocationsRel\ N\ {\isacharparenleft}set\ G{\isacharparenright}\ bids{\isachardoublequoteclose}\ \isacommand{using}\isamarkupfalse%
\ assms\ mm{\isadigit{8}}{\isadigit{2}}\ \isacommand{by}\isamarkupfalse%
\ blast\isanewline
\isacommand{moreover}\isamarkupfalse%
\ \isacommand{have}\isamarkupfalse%
\ {\isachardoublequoteopen}winningAllocationsRel\ N\ {\isacharparenleft}set\ G{\isacharparenright}\ bids\ {\isasymsubseteq}\ {\isacharquery}p\ N\ {\isacharquery}G{\isachardoublequoteclose}\ \isacommand{using}\isamarkupfalse%
\ assms\ lm{\isadigit{0}}{\isadigit{3}}\ \isacommand{by}\isamarkupfalse%
\ metis\isanewline
\isacommand{ultimately}\isamarkupfalse%
\ \isacommand{have}\isamarkupfalse%
\ {\isachardoublequoteopen}{\isacharquery}a\ {\isasymin}\ {\isacharquery}p\ N\ {\isacharquery}G{\isachardoublequoteclose}\ \isacommand{using}\isamarkupfalse%
\ mm{\isadigit{8}}{\isadigit{2}}\ assms\ lm{\isadigit{0}}{\isadigit{3}}\ set{\isacharunderscore}rev{\isacharunderscore}mp\ \isacommand{by}\isamarkupfalse%
\ blast\isanewline
\isacommand{then}\isamarkupfalse%
\ \isacommand{show}\isamarkupfalse%
\ {\isacharquery}thesis\ \isacommand{using}\isamarkupfalse%
\ assms\ mm{\isadigit{7}}{\isadigit{0}}d\ \isacommand{by}\isamarkupfalse%
\ blast\ \isanewline
\isacommand{qed}\isamarkupfalse%
%
\endisatagproof
{\isafoldproof}%
%
\isadelimproof
\isanewline
%
\endisadelimproof
\isanewline
\isacommand{abbreviation}\isamarkupfalse%
\ {\isachardoublequoteopen}terminatingAuctionRel\ N\ G\ bids\ random\ {\isacharequal}{\isacharequal}\ \isanewline
argmax\ {\isacharparenleft}setsum\ {\isacharparenleft}resolvingBid{\isacharprime}\ N\ G\ bids\ random{\isacharparenright}{\isacharparenright}\ {\isacharparenleft}argmax\ {\isacharparenleft}setsum\ bids{\isacharparenright}\ {\isacharparenleft}possibleAllocationsRel\ N\ {\isacharparenleft}set\ G{\isacharparenright}{\isacharparenright}{\isacharparenright}{\isachardoublequoteclose}%
\begin{isamarkuptext}%
Termination theorem: it assures that the number of winning allocations is exactly one%
\end{isamarkuptext}%
\isamarkuptrue%
\isacommand{theorem}\isamarkupfalse%
\ mm{\isadigit{9}}{\isadigit{2}}{\isacharcolon}\ \isakeyword{assumes}\ \isanewline
{\isachardoublequoteopen}N\ {\isasymnoteq}\ {\isacharbraceleft}{\isacharbraceright}{\isachardoublequoteclose}\ {\isachardoublequoteopen}distinct\ G{\isachardoublequoteclose}\ {\isachardoublequoteopen}set\ G\ {\isasymnoteq}\ {\isacharbraceleft}{\isacharbraceright}{\isachardoublequoteclose}\ {\isachardoublequoteopen}finite\ N{\isachardoublequoteclose}\ \ \isanewline
\isakeyword{shows}\ {\isachardoublequoteopen}terminatingAuctionRel\ N\ G\ {\isacharparenleft}bids{\isacharparenright}\ random\ {\isacharequal}\ {\isacharbraceleft}chosenAllocation{\isacharprime}\ N\ G\ bids\ random{\isacharbraceright}{\isachardoublequoteclose}\isanewline
%
\isadelimproof
%
\endisadelimproof
%
\isatagproof
\isacommand{proof}\isamarkupfalse%
\ {\isacharminus}\isanewline
\isacommand{let}\isamarkupfalse%
\ {\isacharquery}p{\isacharequal}possibleAllocationsRel\ \isacommand{let}\isamarkupfalse%
\ {\isacharquery}G{\isacharequal}{\isachardoublequoteopen}set\ G{\isachardoublequoteclose}\ \isanewline
\isacommand{let}\isamarkupfalse%
\ {\isacharquery}X{\isacharequal}{\isachardoublequoteopen}argmax\ {\isacharparenleft}setsum\ bids{\isacharparenright}\ {\isacharparenleft}{\isacharquery}p\ N\ {\isacharquery}G{\isacharparenright}{\isachardoublequoteclose}\isanewline
\isacommand{let}\isamarkupfalse%
\ {\isacharquery}a{\isacharequal}{\isachardoublequoteopen}chosenAllocation{\isacharprime}\ N\ G\ bids\ random{\isachardoublequoteclose}\ \isacommand{let}\isamarkupfalse%
\ {\isacharquery}b{\isacharequal}{\isachardoublequoteopen}resolvingBid{\isacharprime}\ N\ G\ bids\ random{\isachardoublequoteclose}\isanewline
\isacommand{let}\isamarkupfalse%
\ {\isacharquery}f{\isacharequal}{\isachardoublequoteopen}setsum\ {\isacharquery}b{\isachardoublequoteclose}\ \isacommand{let}\isamarkupfalse%
\ {\isacharquery}ff{\isacharequal}{\isachardoublequoteopen}setsum\ {\isacharquery}b{\isachardoublequoteclose}\ \isanewline
\isacommand{let}\isamarkupfalse%
\ {\isacharquery}t{\isacharequal}terminatingAuctionRel\ \isacommand{have}\isamarkupfalse%
\ {\isachardoublequoteopen}{\isasymforall}aa{\isasymin}{\isacharparenleft}possibleAllocationsRel\ N\ {\isacharquery}G{\isacharparenright}{\isacharminus}{\isacharbraceleft}{\isacharquery}a{\isacharbraceright}{\isachardot}\ \ {\isacharquery}f\ aa\ {\isacharless}\ {\isacharquery}f\ {\isacharquery}a{\isachardoublequoteclose}\ \isanewline
\isacommand{using}\isamarkupfalse%
\ assms\ mm{\isadigit{8}}{\isadigit{1}}\ \isacommand{by}\isamarkupfalse%
\ blast\ \isacommand{then}\isamarkupfalse%
\ \isacommand{have}\isamarkupfalse%
\ \isanewline
{\isadigit{0}}{\isacharcolon}\ {\isachardoublequoteopen}{\isasymforall}aa\ {\isasymin}\ {\isacharquery}X{\isacharminus}{\isacharbraceleft}{\isacharquery}a{\isacharbraceright}{\isachardot}\ {\isacharquery}f\ aa\ {\isacharless}\ {\isacharquery}f\ {\isacharquery}a{\isachardoublequoteclose}\ \isacommand{using}\isamarkupfalse%
\ assms\ mm{\isadigit{8}}{\isadigit{1}}\ \isacommand{by}\isamarkupfalse%
\ auto\isanewline
\isacommand{have}\isamarkupfalse%
\ {\isachardoublequoteopen}finite\ N{\isachardoublequoteclose}\ \isacommand{using}\isamarkupfalse%
\ assms\ \isacommand{by}\isamarkupfalse%
\ simp\ \isacommand{then}\isamarkupfalse%
\ \isanewline
\isacommand{have}\isamarkupfalse%
\ {\isachardoublequoteopen}finite\ {\isacharparenleft}{\isacharquery}p\ N\ {\isacharquery}G{\isacharparenright}{\isachardoublequoteclose}\ \isacommand{using}\isamarkupfalse%
\ assms\ lm{\isadigit{5}}{\isadigit{9}}\ \isacommand{by}\isamarkupfalse%
\ {\isacharparenleft}metis\ List{\isachardot}finite{\isacharunderscore}set{\isacharparenright}\isanewline
\isacommand{then}\isamarkupfalse%
\ \isacommand{have}\isamarkupfalse%
\ {\isachardoublequoteopen}finite\ {\isacharquery}X{\isachardoublequoteclose}\ \isacommand{using}\isamarkupfalse%
\ assms\ \isacommand{by}\isamarkupfalse%
\ {\isacharparenleft}metis\ finite{\isacharunderscore}subset\ lm{\isadigit{0}}{\isadigit{3}}{\isacharparenright}\isanewline
\isacommand{moreover}\isamarkupfalse%
\ \isacommand{have}\isamarkupfalse%
\ {\isachardoublequoteopen}{\isacharquery}a\ {\isasymin}\ {\isacharquery}X{\isachardoublequoteclose}\ \isacommand{using}\isamarkupfalse%
\ mm{\isadigit{8}}{\isadigit{2}}\ assms\ \isacommand{by}\isamarkupfalse%
\ blast\isanewline
\isacommand{ultimately}\isamarkupfalse%
\ \isacommand{have}\isamarkupfalse%
\ \isanewline
{\isachardoublequoteopen}finite\ {\isacharquery}X\ {\isacharampersand}\ {\isacharquery}a\ {\isasymin}\ {\isacharquery}X\ {\isacharampersand}\ {\isacharparenleft}{\isasymforall}aa\ {\isasymin}\ {\isacharquery}X{\isacharminus}{\isacharbraceleft}{\isacharquery}a{\isacharbraceright}{\isachardot}\ {\isacharquery}f\ aa\ {\isacharless}\ {\isacharquery}f\ {\isacharquery}a{\isacharparenright}{\isachardoublequoteclose}\ \isacommand{using}\isamarkupfalse%
\ {\isadigit{0}}\ \isacommand{by}\isamarkupfalse%
\ force\isanewline
\isacommand{moreover}\isamarkupfalse%
\ \isacommand{have}\isamarkupfalse%
\ {\isachardoublequoteopen}{\isacharparenleft}finite\ {\isacharquery}X\ {\isacharampersand}\ {\isacharquery}a\ {\isasymin}\ {\isacharquery}X\ {\isacharampersand}\ {\isacharparenleft}{\isasymforall}aa\ {\isasymin}\ {\isacharquery}X{\isacharminus}{\isacharbraceleft}{\isacharquery}a{\isacharbraceright}{\isachardot}\ {\isacharquery}f\ aa\ {\isacharless}\ {\isacharquery}f\ {\isacharquery}a{\isacharparenright}{\isacharparenright}\ {\isasymlongrightarrow}\ argmax\ {\isacharquery}f\ {\isacharquery}X\ {\isacharequal}\ {\isacharbraceleft}{\isacharquery}a{\isacharbraceright}{\isachardoublequoteclose}\isanewline
\isacommand{by}\isamarkupfalse%
\ {\isacharparenleft}rule\ mm{\isadigit{8}}{\isadigit{0}}c{\isacharparenright}\isanewline
\isacommand{ultimately}\isamarkupfalse%
\ \isacommand{have}\isamarkupfalse%
\ {\isachardoublequoteopen}{\isacharbraceleft}{\isacharquery}a{\isacharbraceright}\ {\isacharequal}\ argmax\ {\isacharquery}f\ {\isacharquery}X{\isachardoublequoteclose}\ \isacommand{using}\isamarkupfalse%
\ mm{\isadigit{8}}{\isadigit{0}}\ \isacommand{by}\isamarkupfalse%
\ presburger\isanewline
\isacommand{moreover}\isamarkupfalse%
\ \isacommand{have}\isamarkupfalse%
\ {\isachardoublequoteopen}{\isachardot}{\isachardot}{\isachardot}\ {\isacharequal}\ {\isacharquery}t\ N\ G\ bids\ random{\isachardoublequoteclose}\ \isacommand{by}\isamarkupfalse%
\ simp\isanewline
\isacommand{ultimately}\isamarkupfalse%
\ \isacommand{show}\isamarkupfalse%
\ {\isacharquery}thesis\ \isacommand{by}\isamarkupfalse%
\ presburger\isanewline
\isacommand{qed}\isamarkupfalse%
%
\endisatagproof
{\isafoldproof}%
%
\isadelimproof
%
\endisadelimproof
%
\begin{isamarkuptext}%
A more computable adaptor from set-theoretical to HOL function, with fallback value%
\end{isamarkuptext}%
\isamarkuptrue%
\isacommand{abbreviation}\isamarkupfalse%
\ {\isachardoublequoteopen}toFunctionWithFallback{\isadigit{2}}\ R\ fallback\ {\isacharequal}{\isacharequal}\ {\isacharparenleft}{\isacharpercent}\ x{\isachardot}\ if\ {\isacharparenleft}x\ {\isasymin}\ Domain\ R{\isacharparenright}\ then\ {\isacharparenleft}R{\isacharcomma}{\isacharcomma}x{\isacharparenright}\ else\ fallback{\isacharparenright}{\isachardoublequoteclose}\isanewline
\isacommand{notation}\isamarkupfalse%
\ toFunctionWithFallback{\isadigit{2}}\ {\isacharparenleft}\isakeyword{infix}\ {\isachardoublequoteopen}Elsee{\isachardoublequoteclose}\ {\isadigit{7}}{\isadigit{5}}{\isacharparenright}%
\isamarkupsection{Combinatorial auction input examples%
}
\isamarkuptrue%
\isacommand{abbreviation}\isamarkupfalse%
\ {\isachardoublequoteopen}N{\isadigit{0}}{\isadigit{0}}\ {\isacharequal}{\isacharequal}\ {\isacharbraceleft}{\isadigit{1}}{\isacharcomma}{\isadigit{2}}{\isacharcolon}{\isacharcolon}nat{\isacharbraceright}{\isachardoublequoteclose}\isanewline
\isacommand{abbreviation}\isamarkupfalse%
\ {\isachardoublequoteopen}G{\isadigit{0}}{\isadigit{0}}\ {\isacharequal}{\isacharequal}\ {\isacharbrackleft}{\isadigit{1}}{\isadigit{1}}{\isacharcolon}{\isacharcolon}nat{\isacharcomma}\ {\isadigit{1}}{\isadigit{2}}{\isacharcomma}\ {\isadigit{1}}{\isadigit{3}}{\isacharbrackright}{\isachardoublequoteclose}\isanewline
\isacommand{abbreviation}\isamarkupfalse%
\ {\isachardoublequoteopen}A{\isadigit{0}}{\isadigit{0}}\ {\isacharequal}{\isacharequal}\ {\isacharbraceleft}{\isacharparenleft}{\isadigit{0}}{\isacharcomma}{\isacharbraceleft}{\isadigit{1}}{\isadigit{0}}{\isacharcomma}{\isadigit{1}}{\isadigit{1}}{\isacharcolon}{\isacharcolon}nat{\isacharbraceright}{\isacharparenright}{\isacharcomma}\ {\isacharparenleft}{\isadigit{1}}{\isacharcomma}{\isacharbraceleft}{\isadigit{1}}{\isadigit{2}}{\isacharcomma}{\isadigit{1}}{\isadigit{3}}{\isacharbraceright}{\isacharparenright}{\isacharbraceright}{\isachardoublequoteclose}\isanewline
\isacommand{abbreviation}\isamarkupfalse%
\ {\isachardoublequoteopen}b{\isadigit{0}}{\isadigit{0}}\ {\isacharequal}{\isacharequal}\ \isanewline
{\isacharbraceleft}\isanewline
{\isacharparenleft}{\isacharparenleft}{\isadigit{1}}{\isacharcolon}{\isacharcolon}int{\isacharcomma}{\isacharbraceleft}{\isadigit{1}}{\isadigit{1}}{\isacharbraceright}{\isacharparenright}{\isacharcomma}{\isadigit{3}}{\isacharparenright}{\isacharcomma}\isanewline
{\isacharparenleft}{\isacharparenleft}{\isadigit{1}}{\isacharcomma}{\isacharbraceleft}{\isadigit{1}}{\isadigit{2}}{\isacharbraceright}{\isacharparenright}{\isacharcomma}{\isadigit{0}}{\isacharparenright}{\isacharcomma}\isanewline
{\isacharparenleft}{\isacharparenleft}{\isadigit{1}}{\isacharcomma}{\isacharbraceleft}{\isadigit{1}}{\isadigit{1}}{\isacharcomma}{\isadigit{1}}{\isadigit{2}}{\isacharcolon}{\isacharcolon}nat{\isacharbraceright}{\isacharparenright}{\isacharcomma}{\isadigit{4}}{\isacharcolon}{\isacharcolon}price{\isacharparenright}{\isacharcomma}\isanewline
{\isacharparenleft}{\isacharparenleft}{\isadigit{2}}{\isacharcomma}{\isacharbraceleft}{\isadigit{1}}{\isadigit{1}}{\isacharbraceright}{\isacharparenright}{\isacharcomma}{\isadigit{2}}{\isacharparenright}{\isacharcomma}\isanewline
{\isacharparenleft}{\isacharparenleft}{\isadigit{2}}{\isacharcomma}{\isacharbraceleft}{\isadigit{1}}{\isadigit{2}}{\isacharbraceright}{\isacharparenright}{\isacharcomma}{\isadigit{2}}{\isacharparenright}{\isacharcomma}\isanewline
{\isacharparenleft}{\isacharparenleft}{\isadigit{2}}{\isacharcomma}{\isacharbraceleft}{\isadigit{1}}{\isadigit{1}}{\isacharcomma}{\isadigit{1}}{\isadigit{2}}{\isacharbraceright}{\isacharparenright}{\isacharcomma}{\isadigit{1}}{\isacharparenright}\isanewline
{\isacharbraceright}{\isachardoublequoteclose}\isanewline
%
\isadelimtheory
\isanewline
%
\endisadelimtheory
%
\isatagtheory
\isacommand{end}\isamarkupfalse%
%
\endisatagtheory
{\isafoldtheory}%
%
\isadelimtheory
%
\endisadelimtheory
\end{isabellebody}%
%%% Local Variables:
%%% mode: latex
%%% TeX-master: "root"
%%% End:


%
\begin{isabellebody}%
\def\isabellecontext{CombinatorialAuction}%
%
\isamarkupheader{VCG auction: definitions and theorems%
}
\isamarkuptrue%
%
\isadelimtheory
%
\endisadelimtheory
%
\isatagtheory
\isacommand{theory}\isamarkupfalse%
\ CombinatorialAuction\isanewline
\isanewline
\isakeyword{imports}\isanewline
\isanewline
UniformTieBreaking\isanewline
{\isachardoublequoteopen}{\isachartilde}{\isachartilde}{\isacharslash}src{\isacharslash}HOL{\isacharslash}Library{\isacharslash}Code{\isacharunderscore}Target{\isacharunderscore}Nat{\isachardoublequoteclose}\ \isanewline
{\isachardoublequoteopen}{\isachartilde}{\isachartilde}{\isacharslash}src{\isacharslash}HOL{\isacharslash}Library{\isacharslash}Code{\isacharunderscore}Target{\isacharunderscore}Int{\isachardoublequoteclose}\ \isanewline
\isanewline
{\isachardoublequoteopen}{\isachartilde}{\isachartilde}{\isacharslash}src{\isacharslash}HOL{\isacharslash}Library{\isacharslash}Code{\isacharunderscore}Numeral{\isachardoublequoteclose}\isanewline
\isanewline
\isakeyword{begin}%
\endisatagtheory
{\isafoldtheory}%
%
\isadelimtheory
%
\endisadelimtheory
%
\isamarkupsection{Definition of a VCG auction scheme, through the pair \isa{{\isacharparenleft}vcga{\isacharprime}{\isacharcomma}\ vcgp{\isacharparenright}}%
}
\isamarkuptrue%
\isacommand{type{\isacharunderscore}synonym}\isamarkupfalse%
\ bidvector{\isacharprime}\ {\isacharequal}\ {\isachardoublequoteopen}{\isacharparenleft}{\isacharparenleft}participant\ {\isasymtimes}\ goods{\isacharparenright}\ {\isasymtimes}\ price{\isacharparenright}\ set{\isachardoublequoteclose}\isanewline
\isacommand{abbreviation}\isamarkupfalse%
\ {\isachardoublequoteopen}participants\ b{\isacharprime}\ {\isacharequal}{\isacharequal}\ Domain\ {\isacharparenleft}Domain\ b{\isacharprime}{\isacharparenright}{\isachardoublequoteclose}\isanewline
\isacommand{abbreviation}\isamarkupfalse%
\ {\isachardoublequoteopen}seller\ {\isacharequal}{\isacharequal}\ {\isacharparenleft}{\isadigit{0}}{\isacharcolon}{\isacharcolon}integer{\isacharparenright}{\isachardoublequoteclose}\isanewline
\isacommand{abbreviation}\isamarkupfalse%
\ {\isachardoublequoteopen}allAllocations\ N\ G\ {\isacharequal}{\isacharequal}\ possibleAllocationsRel\ N\ G{\isachardoublequoteclose}\isanewline
\isacommand{abbreviation}\isamarkupfalse%
\ {\isachardoublequoteopen}allAllocations{\isacharprime}\ N\ {\isasymOmega}\ {\isacharequal}{\isacharequal}\ injectionsUniverse\ {\isasyminter}\ \isanewline
{\isacharbraceleft}a{\isachardot}\ Domain\ a\ {\isasymsubseteq}\ N\ {\isacharampersand}\ Range\ a\ {\isasymin}\ all{\isacharunderscore}partitions\ {\isasymOmega}{\isacharbraceright}{\isachardoublequoteclose}\ \isanewline
\isacommand{abbreviation}\isamarkupfalse%
\ {\isachardoublequoteopen}allAllocations{\isacharprime}{\isacharprime}\ N\ G\ {\isacharequal}{\isacharequal}\ allocationsUniverse{\isasyminter}{\isacharbraceleft}a{\isachardot}\ Domain\ a\ {\isasymsubseteq}\ N\ {\isacharampersand}\ {\isasymUnion}\ Range\ a\ {\isacharequal}\ G{\isacharbraceright}{\isachardoublequoteclose}\isanewline
\isacommand{lemma}\isamarkupfalse%
\ lm{\isadigit{2}}{\isadigit{8}}{\isacharcolon}\ {\isachardoublequoteopen}allAllocations\ N\ G{\isacharequal}allAllocations{\isacharprime}\ N\ G\ {\isacharampersand}\ \isanewline
allAllocations\ N\ G{\isacharequal}allAllocations{\isacharprime}{\isacharprime}\ N\ G{\isachardoublequoteclose}%
\isadelimproof
\ %
\endisadelimproof
%
\isatagproof
\isacommand{using}\isamarkupfalse%
\ lm{\isadigit{1}}{\isadigit{9}}\ nn{\isadigit{2}}{\isadigit{4}}\ \isacommand{by}\isamarkupfalse%
\ metis%
\endisatagproof
{\isafoldproof}%
%
\isadelimproof
%
\endisadelimproof
\isanewline
\isacommand{lemma}\isamarkupfalse%
\ lm{\isadigit{2}}{\isadigit{8}}b{\isacharcolon}\ \isanewline
{\isachardoublequoteopen}{\isacharparenleft}a\ {\isasymin}\ allAllocations{\isacharprime}{\isacharprime}\ N\ G{\isacharparenright}\ {\isacharequal}\ {\isacharparenleft}a\ {\isasymin}\ allocationsUniverse{\isacharampersand}\ Domain\ a\ {\isasymsubseteq}\ N\ {\isacharampersand}\ {\isasymUnion}\ Range\ a\ {\isacharequal}\ G{\isacharparenright}{\isachardoublequoteclose}%
\isadelimproof
\ %
\endisadelimproof
%
\isatagproof
\isacommand{by}\isamarkupfalse%
\ force%
\endisatagproof
{\isafoldproof}%
%
\isadelimproof
%
\endisadelimproof
\isanewline
\isacommand{abbreviation}\isamarkupfalse%
\ {\isachardoublequoteopen}soldAllocations\ N\ {\isasymOmega}\ {\isacharequal}{\isacharequal}\ \isanewline
{\isacharparenleft}Outside{\isacharprime}\ {\isacharbraceleft}seller{\isacharbraceright}{\isacharparenright}\ {\isacharbackquote}\ {\isacharparenleft}allAllocations\ {\isacharparenleft}N\ {\isasymunion}\ {\isacharbraceleft}seller{\isacharbraceright}{\isacharparenright}\ {\isasymOmega}{\isacharparenright}{\isachardoublequoteclose}\isanewline
\isacommand{abbreviation}\isamarkupfalse%
\ {\isachardoublequoteopen}soldAllocations{\isacharprime}\ N\ {\isasymOmega}\ {\isacharequal}{\isacharequal}\ \isanewline
{\isacharparenleft}Outside{\isacharprime}\ {\isacharbraceleft}seller{\isacharbraceright}{\isacharparenright}\ {\isacharbackquote}\ {\isacharparenleft}allAllocations{\isacharprime}\ {\isacharparenleft}N\ {\isasymunion}\ {\isacharbraceleft}seller{\isacharbraceright}{\isacharparenright}\ {\isasymOmega}{\isacharparenright}{\isachardoublequoteclose}\isanewline
\isacommand{abbreviation}\isamarkupfalse%
\ {\isachardoublequoteopen}soldAllocations{\isacharprime}{\isacharprime}\ N\ {\isasymOmega}\ {\isacharequal}{\isacharequal}\ \isanewline
{\isacharparenleft}Outside{\isacharprime}\ {\isacharbraceleft}seller{\isacharbraceright}{\isacharparenright}\ {\isacharbackquote}\ {\isacharparenleft}allAllocations{\isacharprime}{\isacharprime}\ {\isacharparenleft}N\ {\isasymunion}\ {\isacharbraceleft}seller{\isacharbraceright}{\isacharparenright}\ {\isasymOmega}{\isacharparenright}{\isachardoublequoteclose}\isanewline
\isacommand{lemma}\isamarkupfalse%
\ lm{\isadigit{2}}{\isadigit{8}}c{\isacharcolon}\ \isanewline
{\isachardoublequoteopen}soldAllocations\ N\ G\ {\isacharequal}\ soldAllocations{\isacharprime}\ N\ G\ {\isacharampersand}\ soldAllocations{\isacharprime}\ N\ G\ {\isacharequal}\ soldAllocations{\isacharprime}{\isacharprime}\ N\ G{\isachardoublequoteclose}\isanewline
%
\isadelimproof
%
\endisadelimproof
%
\isatagproof
\isacommand{using}\isamarkupfalse%
\ assms\ lm{\isadigit{2}}{\isadigit{8}}\ \isacommand{by}\isamarkupfalse%
\ metis%
\endisatagproof
{\isafoldproof}%
%
\isadelimproof
\isanewline
%
\endisadelimproof
\isacommand{corollary}\isamarkupfalse%
\ lm{\isadigit{2}}{\isadigit{8}}d{\isacharcolon}\ {\isachardoublequoteopen}soldAllocations\ {\isacharequal}\ soldAllocations{\isacharprime}\ {\isacharampersand}\ soldAllocations{\isacharprime}\ {\isacharequal}\ soldAllocations{\isacharprime}{\isacharprime}\isanewline
{\isacharampersand}\ soldAllocations\ {\isacharequal}\ soldAllocations{\isacharprime}{\isacharprime}{\isachardoublequoteclose}%
\isadelimproof
\ %
\endisadelimproof
%
\isatagproof
\isacommand{using}\isamarkupfalse%
\ lm{\isadigit{2}}{\isadigit{8}}c\ \isacommand{by}\isamarkupfalse%
\ metis%
\endisatagproof
{\isafoldproof}%
%
\isadelimproof
%
\endisadelimproof
\isanewline
\isacommand{lemma}\isamarkupfalse%
\ lm{\isadigit{3}}{\isadigit{2}}{\isacharcolon}\ {\isachardoublequoteopen}soldAllocations\ {\isacharparenleft}N{\isacharminus}{\isacharbraceleft}seller{\isacharbraceright}{\isacharparenright}\ G\ {\isasymsubseteq}\ soldAllocations\ N\ G{\isachardoublequoteclose}%
\isadelimproof
\ %
\endisadelimproof
%
\isatagproof
\isacommand{using}\isamarkupfalse%
\ Outside{\isacharunderscore}def\ \isacommand{by}\isamarkupfalse%
\ simp%
\endisatagproof
{\isafoldproof}%
%
\isadelimproof
%
\endisadelimproof
\isanewline
\isanewline
\isacommand{lemma}\isamarkupfalse%
\ lm{\isadigit{3}}{\isadigit{4}}{\isacharcolon}\ {\isachardoublequoteopen}{\isacharparenleft}a\ {\isasymin}\ allocationsUniverse{\isacharparenright}\ {\isacharequal}\ {\isacharparenleft}a\ {\isasymin}\ allAllocations{\isacharprime}{\isacharprime}\ {\isacharparenleft}Domain\ a{\isacharparenright}\ {\isacharparenleft}{\isasymUnion}\ Range\ a{\isacharparenright}{\isacharparenright}{\isachardoublequoteclose}\isanewline
%
\isadelimproof
%
\endisadelimproof
%
\isatagproof
\isacommand{by}\isamarkupfalse%
\ blast%
\endisatagproof
{\isafoldproof}%
%
\isadelimproof
\isanewline
%
\endisadelimproof
\isacommand{lemma}\isamarkupfalse%
\ lm{\isadigit{3}}{\isadigit{5}}{\isacharcolon}\ \isakeyword{assumes}\ {\isachardoublequoteopen}N{\isadigit{1}}\ {\isasymsubseteq}\ N{\isadigit{2}}{\isachardoublequoteclose}\ \isakeyword{shows}\ {\isachardoublequoteopen}allAllocations{\isacharprime}{\isacharprime}\ N{\isadigit{1}}\ G\ {\isasymsubseteq}\ allAllocations{\isacharprime}{\isacharprime}\ N{\isadigit{2}}\ G{\isachardoublequoteclose}\ \isanewline
%
\isadelimproof
%
\endisadelimproof
%
\isatagproof
\isacommand{using}\isamarkupfalse%
\ assms\ \isacommand{by}\isamarkupfalse%
\ auto%
\endisatagproof
{\isafoldproof}%
%
\isadelimproof
\isanewline
%
\endisadelimproof
\isacommand{lemma}\isamarkupfalse%
\ lm{\isadigit{3}}{\isadigit{6}}{\isacharcolon}\ \isakeyword{assumes}\ {\isachardoublequoteopen}a\ {\isasymin}\ allAllocations{\isacharprime}{\isacharprime}\ N\ G{\isachardoublequoteclose}\ \isakeyword{shows}\ {\isachardoublequoteopen}Domain\ {\isacharparenleft}a\ {\isacharminus}{\isacharminus}\ x{\isacharparenright}\ {\isasymsubseteq}\ N{\isacharminus}{\isacharbraceleft}x{\isacharbraceright}{\isachardoublequoteclose}\ \isanewline
%
\isadelimproof
%
\endisadelimproof
%
\isatagproof
\isacommand{using}\isamarkupfalse%
\ assms\ Outside{\isacharunderscore}def\ \isacommand{by}\isamarkupfalse%
\ fastforce%
\endisatagproof
{\isafoldproof}%
%
\isadelimproof
\isanewline
%
\endisadelimproof
\isacommand{lemma}\isamarkupfalse%
\ lm{\isadigit{3}}{\isadigit{7}}{\isacharcolon}\ \isakeyword{assumes}\ {\isachardoublequoteopen}a\ {\isasymin}\ soldAllocations\ N\ G{\isachardoublequoteclose}\ \isakeyword{shows}\ {\isachardoublequoteopen}a\ {\isasymin}\ allocationsUniverse{\isachardoublequoteclose}\isanewline
%
\isadelimproof
%
\endisadelimproof
%
\isatagproof
\isacommand{proof}\isamarkupfalse%
\ {\isacharminus}\isanewline
\isacommand{obtain}\isamarkupfalse%
\ aa\ \isakeyword{where}\ {\isachardoublequoteopen}a{\isacharequal}aa\ {\isacharminus}{\isacharminus}\ seller\ {\isacharampersand}\ aa\ {\isasymin}\ allAllocations\ {\isacharparenleft}N{\isasymunion}{\isacharbraceleft}seller{\isacharbraceright}{\isacharparenright}\ G{\isachardoublequoteclose}\isanewline
\isacommand{using}\isamarkupfalse%
\ assms\ \isacommand{by}\isamarkupfalse%
\ blast\isanewline
\isacommand{then}\isamarkupfalse%
\ \isacommand{have}\isamarkupfalse%
\ {\isachardoublequoteopen}a\ {\isasymsubseteq}\ aa\ {\isacharampersand}\ aa\ {\isasymin}\ allocationsUniverse{\isachardoublequoteclose}\ \isacommand{unfolding}\isamarkupfalse%
\ Outside{\isacharunderscore}def\ \isacommand{using}\isamarkupfalse%
\ nn{\isadigit{2}}{\isadigit{4}}b\ \isacommand{by}\isamarkupfalse%
\ blast\isanewline
\isacommand{then}\isamarkupfalse%
\ \isacommand{show}\isamarkupfalse%
\ {\isacharquery}thesis\ \isacommand{using}\isamarkupfalse%
\ lm{\isadigit{3}}{\isadigit{5}}b\ \isacommand{by}\isamarkupfalse%
\ blast\isanewline
\isacommand{qed}\isamarkupfalse%
%
\endisatagproof
{\isafoldproof}%
%
\isadelimproof
\isanewline
%
\endisadelimproof
\isacommand{lemma}\isamarkupfalse%
\ lm{\isadigit{3}}{\isadigit{8}}{\isacharcolon}\ \isakeyword{assumes}\ {\isachardoublequoteopen}a\ {\isasymin}\ soldAllocations\ N\ G{\isachardoublequoteclose}\ \isakeyword{shows}\ {\isachardoublequoteopen}a\ {\isasymin}\ allAllocations{\isacharprime}{\isacharprime}\ {\isacharparenleft}Domain\ a{\isacharparenright}\ {\isacharparenleft}{\isasymUnion}Range\ a{\isacharparenright}{\isachardoublequoteclose}\isanewline
%
\isadelimproof
%
\endisadelimproof
%
\isatagproof
\isacommand{proof}\isamarkupfalse%
\ {\isacharminus}\ \isacommand{show}\isamarkupfalse%
\ {\isacharquery}thesis\ \isacommand{using}\isamarkupfalse%
\ assms\ lm{\isadigit{3}}{\isadigit{7}}\ \isacommand{by}\isamarkupfalse%
\ blast\ \isacommand{qed}\isamarkupfalse%
%
\endisatagproof
{\isafoldproof}%
%
\isadelimproof
\isanewline
%
\endisadelimproof
\isacommand{lemma}\isamarkupfalse%
\ \ \isakeyword{assumes}\ {\isachardoublequoteopen}N{\isadigit{1}}\ {\isasymsubseteq}\ N{\isadigit{2}}{\isachardoublequoteclose}\ \isakeyword{shows}\ {\isachardoublequoteopen}soldAllocations{\isacharprime}{\isacharprime}\ N{\isadigit{1}}\ G\ {\isasymsubseteq}\ soldAllocations{\isacharprime}{\isacharprime}\ N{\isadigit{2}}\ G{\isachardoublequoteclose}\isanewline
%
\isadelimproof
%
\endisadelimproof
%
\isatagproof
\isacommand{using}\isamarkupfalse%
\ assms\ lm{\isadigit{3}}{\isadigit{5}}\ lm{\isadigit{3}}{\isadigit{6}}\ \ nn{\isadigit{2}}{\isadigit{4}}c\ lm{\isadigit{2}}{\isadigit{8}}b\ lm{\isadigit{2}}{\isadigit{8}}\ lm{\isadigit{3}}{\isadigit{4}}\ lm{\isadigit{3}}{\isadigit{8}}\ Outside{\isacharunderscore}def\ \isacommand{by}\isamarkupfalse%
\ blast%
\endisatagproof
{\isafoldproof}%
%
\isadelimproof
\isanewline
%
\endisadelimproof
\isanewline
\isacommand{lemma}\isamarkupfalse%
\ lll{\isadigit{5}}{\isadigit{9}}b{\isacharcolon}\ {\isachardoublequoteopen}runiq\ {\isacharparenleft}X{\isasymtimes}{\isacharbraceleft}y{\isacharbraceright}{\isacharparenright}{\isachardoublequoteclose}%
\isadelimproof
\ %
\endisadelimproof
%
\isatagproof
\isacommand{using}\isamarkupfalse%
\ rightUniqueTrivialCartes\ \isacommand{by}\isamarkupfalse%
\ {\isacharparenleft}metis\ trivial{\isacharunderscore}singleton{\isacharparenright}%
\endisatagproof
{\isafoldproof}%
%
\isadelimproof
%
\endisadelimproof
\isanewline
\isacommand{lemma}\isamarkupfalse%
\ lm{\isadigit{3}}{\isadigit{7}}b{\isacharcolon}\ {\isachardoublequoteopen}{\isacharbraceleft}x{\isacharbraceright}{\isasymtimes}{\isacharbraceleft}y{\isacharbraceright}\ {\isasymin}\ injectionsUniverse{\isachardoublequoteclose}%
\isadelimproof
\ %
\endisadelimproof
%
\isatagproof
\isacommand{using}\isamarkupfalse%
\ Universes{\isachardot}lm{\isadigit{3}}{\isadigit{7}}\ \isacommand{by}\isamarkupfalse%
\ fastforce%
\endisatagproof
{\isafoldproof}%
%
\isadelimproof
%
\endisadelimproof
\isanewline
\isacommand{lemma}\isamarkupfalse%
\ lm{\isadigit{4}}{\isadigit{0}}b{\isacharcolon}\ \isakeyword{assumes}\ {\isachardoublequoteopen}a\ {\isasymin}\ soldAllocations{\isacharprime}{\isacharprime}\ N\ G{\isachardoublequoteclose}\ \isakeyword{shows}\ {\isachardoublequoteopen}{\isasymUnion}\ Range\ a\ {\isasymsubseteq}\ G{\isachardoublequoteclose}%
\isadelimproof
\ %
\endisadelimproof
%
\isatagproof
\isacommand{using}\isamarkupfalse%
\ assms\ Outside{\isacharunderscore}def\ \isacommand{by}\isamarkupfalse%
\ blast%
\endisatagproof
{\isafoldproof}%
%
\isadelimproof
%
\endisadelimproof
\isanewline
\isacommand{lemma}\isamarkupfalse%
\ lm{\isadigit{4}}{\isadigit{1}}{\isacharcolon}\ {\isachardoublequoteopen}a\ {\isasymin}\ soldAllocations{\isacharprime}{\isacharprime}\ N\ G\ {\isacharequal}\ \isanewline
{\isacharparenleft}EX\ aa{\isachardot}\ aa\ {\isacharminus}{\isacharminus}\ {\isacharparenleft}seller{\isacharparenright}{\isacharequal}a\ {\isacharampersand}\ aa{\isasymin}allAllocations{\isacharprime}{\isacharprime}\ {\isacharparenleft}N\ {\isasymunion}\ {\isacharbraceleft}seller{\isacharbraceright}{\isacharparenright}\ G{\isacharparenright}{\isachardoublequoteclose}%
\isadelimproof
\ %
\endisadelimproof
%
\isatagproof
\isacommand{by}\isamarkupfalse%
\ blast%
\endisatagproof
{\isafoldproof}%
%
\isadelimproof
%
\endisadelimproof
\isanewline
\isanewline
\isacommand{lemma}\isamarkupfalse%
\ lm{\isadigit{1}}{\isadigit{8}}{\isacharcolon}\ {\isachardoublequoteopen}{\isacharparenleft}R\ {\isacharplus}{\isacharasterisk}\ {\isacharparenleft}{\isacharbraceleft}x{\isacharbraceright}{\isasymtimes}Y{\isacharparenright}{\isacharparenright}\ {\isacharminus}{\isacharminus}\ x\ {\isacharequal}\ R\ {\isacharminus}{\isacharminus}\ x{\isachardoublequoteclose}%
\isadelimproof
\ %
\endisadelimproof
%
\isatagproof
\isacommand{unfolding}\isamarkupfalse%
\ Outside{\isacharunderscore}def\ paste{\isacharunderscore}def\ \isacommand{by}\isamarkupfalse%
\ blast%
\endisatagproof
{\isafoldproof}%
%
\isadelimproof
%
\endisadelimproof
\isanewline
\isanewline
\isacommand{lemma}\isamarkupfalse%
\ lm{\isadigit{3}}{\isadigit{7}}e{\isacharcolon}\ \isakeyword{assumes}\ {\isachardoublequoteopen}a\ {\isasymin}\ allocationsUniverse{\isachardoublequoteclose}\ {\isachardoublequoteopen}Domain\ a\ {\isasymsubseteq}\ N{\isacharminus}{\isacharbraceleft}seller{\isacharbraceright}{\isachardoublequoteclose}\ {\isachardoublequoteopen}{\isasymUnion}\ Range\ a\ {\isasymsubseteq}\ G{\isachardoublequoteclose}\ \isakeyword{shows}\isanewline
{\isachardoublequoteopen}a\ {\isasymin}\ soldAllocations{\isacharprime}{\isacharprime}\ N\ G{\isachardoublequoteclose}%
\isadelimproof
\ %
\endisadelimproof
%
\isatagproof
\isacommand{using}\isamarkupfalse%
\ assms\ lm{\isadigit{4}}{\isadigit{1}}\ \isanewline
\isacommand{proof}\isamarkupfalse%
\ {\isacharminus}\isanewline
\isacommand{let}\isamarkupfalse%
\ {\isacharquery}i{\isacharequal}{\isachardoublequoteopen}seller{\isachardoublequoteclose}\ \isacommand{let}\isamarkupfalse%
\ {\isacharquery}Y{\isacharequal}{\isachardoublequoteopen}{\isacharbraceleft}G{\isacharminus}{\isasymUnion}\ Range\ a{\isacharbraceright}{\isacharminus}{\isacharbraceleft}{\isacharbraceleft}{\isacharbraceright}{\isacharbraceright}{\isachardoublequoteclose}\ \isacommand{let}\isamarkupfalse%
\ {\isacharquery}b{\isacharequal}{\isachardoublequoteopen}{\isacharbraceleft}{\isacharquery}i{\isacharbraceright}{\isasymtimes}{\isacharquery}Y{\isachardoublequoteclose}\ \isacommand{let}\isamarkupfalse%
\ {\isacharquery}aa{\isacharequal}{\isachardoublequoteopen}a{\isasymunion}{\isacharquery}b{\isachardoublequoteclose}\isanewline
\isacommand{let}\isamarkupfalse%
\ {\isacharquery}aa{\isacharprime}{\isacharequal}{\isachardoublequoteopen}a\ {\isacharplus}{\isacharasterisk}\ {\isacharquery}b{\isachardoublequoteclose}\ \isanewline
\isacommand{have}\isamarkupfalse%
\isanewline
{\isadigit{1}}{\isacharcolon}\ {\isachardoublequoteopen}a\ {\isasymin}\ allocationsUniverse{\isachardoublequoteclose}\ \isacommand{using}\isamarkupfalse%
\ assms{\isacharparenleft}{\isadigit{1}}{\isacharparenright}\ \isacommand{by}\isamarkupfalse%
\ fast\ \isanewline
\isacommand{have}\isamarkupfalse%
\ {\isachardoublequoteopen}{\isacharquery}b\ {\isasymsubseteq}\ {\isacharbraceleft}{\isacharparenleft}{\isacharquery}i{\isacharcomma}G{\isacharminus}{\isasymUnion}\ Range\ a{\isacharparenright}{\isacharbraceright}\ {\isacharminus}\ {\isacharbraceleft}{\isacharparenleft}{\isacharquery}i{\isacharcomma}\ {\isacharbraceleft}{\isacharbraceright}{\isacharparenright}{\isacharbraceright}{\isachardoublequoteclose}\ \isacommand{by}\isamarkupfalse%
\ fastforce\ \isacommand{then}\isamarkupfalse%
\ \isacommand{have}\isamarkupfalse%
\ \isanewline
{\isadigit{2}}{\isacharcolon}\ {\isachardoublequoteopen}{\isacharquery}b\ {\isasymin}\ allocationsUniverse{\isachardoublequoteclose}\ \isacommand{using}\isamarkupfalse%
\ allocationUniverseProperty\ lm{\isadigit{3}}{\isadigit{5}}b\ \isacommand{by}\isamarkupfalse%
\ {\isacharparenleft}metis{\isacharparenleft}no{\isacharunderscore}types{\isacharparenright}{\isacharparenright}\ \isacommand{have}\isamarkupfalse%
\ \isanewline
{\isadigit{3}}{\isacharcolon}\ {\isachardoublequoteopen}{\isasymUnion}\ Range\ a\ {\isasyminter}\ {\isasymUnion}\ {\isacharparenleft}Range\ {\isacharquery}b{\isacharparenright}\ {\isacharequal}\ {\isacharbraceleft}{\isacharbraceright}{\isachardoublequoteclose}\ \isacommand{by}\isamarkupfalse%
\ blast\ \isacommand{have}\isamarkupfalse%
\ \isanewline
{\isadigit{4}}{\isacharcolon}\ {\isachardoublequoteopen}Domain\ a\ {\isasyminter}\ Domain\ {\isacharquery}b\ {\isacharequal}{\isacharbraceleft}{\isacharbraceright}{\isachardoublequoteclose}\ \isacommand{using}\isamarkupfalse%
\ assms\ \isacommand{by}\isamarkupfalse%
\ fast\isanewline
\isacommand{have}\isamarkupfalse%
\ {\isachardoublequoteopen}{\isacharquery}aa\ {\isasymin}\ allocationsUniverse{\isachardoublequoteclose}\ \isacommand{using}\isamarkupfalse%
\ {\isadigit{1}}\ {\isadigit{2}}\ {\isadigit{3}}\ {\isadigit{4}}\ \isacommand{by}\isamarkupfalse%
\ {\isacharparenleft}rule\ lm{\isadigit{2}}{\isadigit{3}}{\isacharparenright}\isanewline
\isacommand{then}\isamarkupfalse%
\ \isacommand{have}\isamarkupfalse%
\ {\isachardoublequoteopen}{\isacharquery}aa\ {\isasymin}\ allAllocations{\isacharprime}{\isacharprime}\ {\isacharparenleft}Domain\ {\isacharquery}aa{\isacharparenright}\ \isanewline
{\isacharparenleft}{\isasymUnion}\ Range\ {\isacharquery}aa{\isacharparenright}{\isachardoublequoteclose}\ \isacommand{unfolding}\isamarkupfalse%
\ lm{\isadigit{3}}{\isadigit{4}}\ \isacommand{by}\isamarkupfalse%
\ metis\ \isacommand{then}\isamarkupfalse%
\ \isacommand{have}\isamarkupfalse%
\ \isanewline
{\isachardoublequoteopen}{\isacharquery}aa\ {\isasymin}\ allAllocations{\isacharprime}{\isacharprime}\ {\isacharparenleft}N{\isasymunion}{\isacharbraceleft}{\isacharquery}i{\isacharbraceright}{\isacharparenright}\ {\isacharparenleft}{\isasymUnion}\ Range\ {\isacharquery}aa{\isacharparenright}{\isachardoublequoteclose}\ \isacommand{using}\isamarkupfalse%
\ lm{\isadigit{3}}{\isadigit{5}}\ assms\ paste{\isacharunderscore}def\ \isacommand{by}\isamarkupfalse%
\ auto\isanewline
\isacommand{moreover}\isamarkupfalse%
\ \isacommand{have}\isamarkupfalse%
\ {\isachardoublequoteopen}Range\ {\isacharquery}aa\ {\isacharequal}\ Range\ a\ {\isasymunion}\ {\isacharquery}Y{\isachardoublequoteclose}\ \isacommand{by}\isamarkupfalse%
\ blast\ \isacommand{then}\isamarkupfalse%
\ \isacommand{moreover}\isamarkupfalse%
\ \isacommand{have}\isamarkupfalse%
\ \isanewline
{\isachardoublequoteopen}{\isasymUnion}\ Range\ {\isacharquery}aa\ {\isacharequal}\ G{\isachardoublequoteclose}\ \isacommand{using}\isamarkupfalse%
\ Un{\isacharunderscore}Diff{\isacharunderscore}cancel\ Un{\isacharunderscore}Diff{\isacharunderscore}cancel{\isadigit{2}}\ Union{\isacharunderscore}Un{\isacharunderscore}distrib\ Union{\isacharunderscore}empty\ Union{\isacharunderscore}insert\ \ \isanewline
\isacommand{by}\isamarkupfalse%
\ {\isacharparenleft}metis\ {\isacharparenleft}lifting{\isacharcomma}\ no{\isacharunderscore}types{\isacharparenright}\ assms{\isacharparenleft}{\isadigit{3}}{\isacharparenright}\ cSup{\isacharunderscore}singleton\ subset{\isacharunderscore}Un{\isacharunderscore}eq{\isacharparenright}\ \isacommand{moreover}\isamarkupfalse%
\ \isacommand{have}\isamarkupfalse%
\ \isanewline
{\isachardoublequoteopen}{\isacharquery}aa{\isacharprime}\ {\isacharequal}\ {\isacharquery}aa{\isachardoublequoteclose}\ \isacommand{using}\isamarkupfalse%
\ {\isadigit{4}}\ \isacommand{by}\isamarkupfalse%
\ {\isacharparenleft}rule\ paste{\isacharunderscore}disj{\isacharunderscore}domains{\isacharparenright}\isanewline
\isacommand{ultimately}\isamarkupfalse%
\ \isacommand{have}\isamarkupfalse%
\ {\isachardoublequoteopen}{\isacharquery}aa{\isacharprime}\ {\isasymin}\ allAllocations{\isacharprime}{\isacharprime}\ {\isacharparenleft}N{\isasymunion}{\isacharbraceleft}{\isacharquery}i{\isacharbraceright}{\isacharparenright}\ G{\isachardoublequoteclose}\ \isacommand{by}\isamarkupfalse%
\ simp\isanewline
\isacommand{moreover}\isamarkupfalse%
\ \isacommand{have}\isamarkupfalse%
\ {\isachardoublequoteopen}Domain\ {\isacharquery}b\ {\isasymsubseteq}\ {\isacharbraceleft}{\isacharquery}i{\isacharbraceright}{\isachardoublequoteclose}\ \isacommand{by}\isamarkupfalse%
\ fast\ \isanewline
\isacommand{have}\isamarkupfalse%
\ {\isachardoublequoteopen}{\isacharquery}aa{\isacharprime}\ {\isacharminus}{\isacharminus}\ {\isacharquery}i\ {\isacharequal}\ a\ {\isacharminus}{\isacharminus}\ {\isacharquery}i{\isachardoublequoteclose}\ \ \isacommand{by}\isamarkupfalse%
\ {\isacharparenleft}rule\ lm{\isadigit{1}}{\isadigit{8}}{\isacharparenright}\isanewline
\isacommand{moreover}\isamarkupfalse%
\ \isacommand{have}\isamarkupfalse%
\ {\isachardoublequoteopen}{\isachardot}{\isachardot}{\isachardot}\ {\isacharequal}\ a{\isachardoublequoteclose}\ \isacommand{using}\isamarkupfalse%
\ Outside{\isacharunderscore}def\ assms{\isacharparenleft}{\isadigit{2}}{\isacharparenright}\ \isacommand{by}\isamarkupfalse%
\ auto\ \isanewline
\isacommand{ultimately}\isamarkupfalse%
\ \isacommand{show}\isamarkupfalse%
\ {\isacharquery}thesis\ \isacommand{using}\isamarkupfalse%
\ lm{\isadigit{4}}{\isadigit{1}}\ \isacommand{by}\isamarkupfalse%
\ auto\isanewline
\isacommand{qed}\isamarkupfalse%
%
\endisatagproof
{\isafoldproof}%
%
\isadelimproof
%
\endisadelimproof
\isanewline
\isanewline
\isacommand{lemma}\isamarkupfalse%
\ lm{\isadigit{2}}{\isadigit{3}}{\isacharcolon}\ \isanewline
{\isachardoublequoteopen}a{\isasymin}allAllocations\ N\ {\isasymOmega}{\isacharequal}{\isacharparenleft}a{\isasymin}injectionsUniverse\ {\isacharampersand}\ Domain\ a{\isasymsubseteq}N\ {\isacharampersand}\ Range\ a{\isasymin}all{\isacharunderscore}partitions\ {\isasymOmega}{\isacharparenright}{\isachardoublequoteclose}\ \isanewline
%
\isadelimproof
%
\endisadelimproof
%
\isatagproof
\isacommand{by}\isamarkupfalse%
\ {\isacharparenleft}metis\ {\isacharparenleft}full{\isacharunderscore}types{\isacharparenright}\ lm{\isadigit{1}}{\isadigit{9}}c{\isacharparenright}%
\endisatagproof
{\isafoldproof}%
%
\isadelimproof
\isanewline
%
\endisadelimproof
\isanewline
\isacommand{lemma}\isamarkupfalse%
\ lm{\isadigit{3}}{\isadigit{7}}n{\isacharcolon}\ \isakeyword{assumes}\ {\isachardoublequoteopen}a\ {\isasymin}\ soldAllocations{\isacharprime}{\isacharprime}\ N\ G{\isachardoublequoteclose}\ \isakeyword{shows}\ {\isachardoublequoteopen}Domain\ a\ {\isasymsubseteq}\ N{\isacharminus}{\isacharbraceleft}seller{\isacharbraceright}\ {\isacharampersand}\ a\ {\isasymin}\ allocationsUniverse{\isachardoublequoteclose}\ \ \isanewline
%
\isadelimproof
%
\endisadelimproof
%
\isatagproof
\isacommand{proof}\isamarkupfalse%
\ {\isacharminus}\isanewline
\isacommand{let}\isamarkupfalse%
\ {\isacharquery}i{\isacharequal}{\isachardoublequoteopen}seller{\isachardoublequoteclose}\ \isacommand{obtain}\isamarkupfalse%
\ aa\ \isakeyword{where}\isanewline
{\isadigit{0}}{\isacharcolon}\ {\isachardoublequoteopen}a{\isacharequal}aa\ {\isacharminus}{\isacharminus}\ {\isacharquery}i\ {\isacharampersand}\ aa\ {\isasymin}\ allAllocations{\isacharprime}{\isacharprime}\ {\isacharparenleft}N\ {\isasymunion}\ {\isacharbraceleft}{\isacharquery}i{\isacharbraceright}{\isacharparenright}\ G{\isachardoublequoteclose}\ \isacommand{using}\isamarkupfalse%
\ assms{\isacharparenleft}{\isadigit{1}}{\isacharparenright}\ lm{\isadigit{4}}{\isadigit{1}}\ \isacommand{by}\isamarkupfalse%
\ blast\isanewline
\isacommand{then}\isamarkupfalse%
\ \isacommand{have}\isamarkupfalse%
\ {\isachardoublequoteopen}Domain\ aa\ {\isasymsubseteq}\ N\ {\isasymunion}\ {\isacharbraceleft}{\isacharquery}i{\isacharbraceright}{\isachardoublequoteclose}\ \isacommand{using}\isamarkupfalse%
\ lm{\isadigit{2}}{\isadigit{3}}\ \isacommand{by}\isamarkupfalse%
\ blast\isanewline
\isacommand{then}\isamarkupfalse%
\ \isacommand{have}\isamarkupfalse%
\ {\isachardoublequoteopen}Domain\ a\ {\isasymsubseteq}\ N\ {\isacharminus}\ {\isacharbraceleft}{\isacharquery}i{\isacharbraceright}{\isachardoublequoteclose}\ \isacommand{using}\isamarkupfalse%
\ {\isadigit{0}}\ Outside{\isacharunderscore}def\ \isacommand{by}\isamarkupfalse%
\ blast\isanewline
\isacommand{moreover}\isamarkupfalse%
\ \isacommand{have}\isamarkupfalse%
\ {\isachardoublequoteopen}a\ {\isasymin}\ soldAllocations\ N\ G{\isachardoublequoteclose}\ \isacommand{using}\isamarkupfalse%
\ assms\ lm{\isadigit{2}}{\isadigit{8}}d\ \isacommand{by}\isamarkupfalse%
\ metis\isanewline
\isacommand{then}\isamarkupfalse%
\ \isacommand{moreover}\isamarkupfalse%
\ \isacommand{have}\isamarkupfalse%
\ {\isachardoublequoteopen}a\ {\isasymin}\ allocationsUniverse{\isachardoublequoteclose}\ \isacommand{using}\isamarkupfalse%
\ lm{\isadigit{3}}{\isadigit{7}}\ \isacommand{by}\isamarkupfalse%
\ blast\isanewline
\isacommand{ultimately}\isamarkupfalse%
\ \isacommand{show}\isamarkupfalse%
\ {\isacharquery}thesis\ \isacommand{by}\isamarkupfalse%
\ blast\isanewline
\isacommand{qed}\isamarkupfalse%
%
\endisatagproof
{\isafoldproof}%
%
\isadelimproof
\isanewline
%
\endisadelimproof
\isanewline
\isacommand{corollary}\isamarkupfalse%
\ lm{\isadigit{3}}{\isadigit{7}}c{\isacharcolon}\ \isakeyword{assumes}\ {\isachardoublequoteopen}a\ {\isasymin}\ soldAllocations{\isacharprime}{\isacharprime}\ N\ G{\isachardoublequoteclose}\ \isakeyword{shows}\ \isanewline
{\isachardoublequoteopen}a\ {\isasymin}\ allocationsUniverse\ {\isacharampersand}\ Domain\ a\ {\isasymsubseteq}\ N{\isacharminus}{\isacharbraceleft}seller{\isacharbraceright}\ {\isacharampersand}\ {\isasymUnion}\ Range\ a\ {\isasymsubseteq}\ G{\isachardoublequoteclose}\isanewline
%
\isadelimproof
%
\endisadelimproof
%
\isatagproof
\isacommand{proof}\isamarkupfalse%
\ {\isacharminus}\isanewline
\isacommand{have}\isamarkupfalse%
\ {\isachardoublequoteopen}a\ {\isasymin}\ allocationsUniverse{\isachardoublequoteclose}\ \isacommand{using}\isamarkupfalse%
\ assms\ lm{\isadigit{3}}{\isadigit{7}}n\ \isacommand{by}\isamarkupfalse%
\ blast\isanewline
\isacommand{moreover}\isamarkupfalse%
\ \isacommand{have}\isamarkupfalse%
\ {\isachardoublequoteopen}Domain\ a\ {\isasymsubseteq}\ N{\isacharminus}{\isacharbraceleft}seller{\isacharbraceright}{\isachardoublequoteclose}\ \isacommand{using}\isamarkupfalse%
\ assms\ lm{\isadigit{3}}{\isadigit{7}}n\ \isacommand{by}\isamarkupfalse%
\ blast\isanewline
\isacommand{moreover}\isamarkupfalse%
\ \isacommand{have}\isamarkupfalse%
\ {\isachardoublequoteopen}{\isasymUnion}\ Range\ a\ {\isasymsubseteq}\ G{\isachardoublequoteclose}\ \isacommand{using}\isamarkupfalse%
\ assms\ lm{\isadigit{4}}{\isadigit{0}}b\ \isacommand{by}\isamarkupfalse%
\ blast\isanewline
\isacommand{ultimately}\isamarkupfalse%
\ \isacommand{show}\isamarkupfalse%
\ {\isacharquery}thesis\ \isacommand{by}\isamarkupfalse%
\ blast\isanewline
\isacommand{qed}\isamarkupfalse%
%
\endisatagproof
{\isafoldproof}%
%
\isadelimproof
\isanewline
%
\endisadelimproof
\isanewline
\isacommand{corollary}\isamarkupfalse%
\ lm{\isadigit{3}}{\isadigit{7}}d{\isacharcolon}\ \isanewline
{\isachardoublequoteopen}{\isacharparenleft}a{\isasymin}soldAllocations{\isacharprime}{\isacharprime}\ N\ G{\isacharparenright}{\isacharequal}{\isacharparenleft}a{\isasymin}allocationsUniverse{\isacharampersand}\ Domain\ a\ {\isasymsubseteq}\ N{\isacharminus}{\isacharbraceleft}seller{\isacharbraceright}\ {\isacharampersand}\ {\isasymUnion}\ Range\ a\ {\isasymsubseteq}\ G{\isacharparenright}{\isachardoublequoteclose}\ \isanewline
%
\isadelimproof
%
\endisadelimproof
%
\isatagproof
\isacommand{using}\isamarkupfalse%
\ lm{\isadigit{3}}{\isadigit{7}}c\ lm{\isadigit{3}}{\isadigit{7}}e\ \isacommand{by}\isamarkupfalse%
\ {\isacharparenleft}metis\ {\isacharparenleft}mono{\isacharunderscore}tags{\isacharparenright}{\isacharparenright}%
\endisatagproof
{\isafoldproof}%
%
\isadelimproof
\isanewline
%
\endisadelimproof
\isanewline
\isacommand{lemma}\isamarkupfalse%
\ lm{\isadigit{4}}{\isadigit{2}}{\isacharcolon}\ {\isachardoublequoteopen}{\isacharparenleft}a{\isasymin}allocationsUniverse{\isacharampersand}\ Domain\ a\ {\isasymsubseteq}\ N{\isacharminus}{\isacharbraceleft}seller{\isacharbraceright}\ {\isacharampersand}\ {\isasymUnion}\ Range\ a\ {\isasymsubseteq}\ G{\isacharparenright}\ {\isacharequal}\ \isanewline
{\isacharparenleft}a{\isasymin}allocationsUniverse{\isacharampersand}\ a{\isasymin}{\isacharbraceleft}aa{\isachardot}\ Domain\ aa\ {\isasymsubseteq}\ N{\isacharminus}{\isacharbraceleft}seller{\isacharbraceright}\ {\isacharampersand}\ {\isasymUnion}\ Range\ aa\ {\isasymsubseteq}\ G{\isacharbraceright}{\isacharparenright}{\isachardoublequoteclose}\ \isanewline
%
\isadelimproof
%
\endisadelimproof
%
\isatagproof
\isacommand{by}\isamarkupfalse%
\ {\isacharparenleft}metis\ {\isacharparenleft}lifting{\isacharcomma}\ no{\isacharunderscore}types{\isacharparenright}\ mem{\isacharunderscore}Collect{\isacharunderscore}eq{\isacharparenright}%
\endisatagproof
{\isafoldproof}%
%
\isadelimproof
\isanewline
%
\endisadelimproof
\isanewline
\isacommand{corollary}\isamarkupfalse%
\ lm{\isadigit{3}}{\isadigit{7}}f{\isacharcolon}\ {\isachardoublequoteopen}{\isacharparenleft}a{\isasymin}soldAllocations{\isacharprime}{\isacharprime}\ N\ G{\isacharparenright}{\isacharequal}\isanewline
{\isacharparenleft}a{\isasymin}allocationsUniverse{\isacharampersand}\ a{\isasymin}{\isacharbraceleft}aa{\isachardot}\ Domain\ aa\ {\isasymsubseteq}\ N{\isacharminus}{\isacharbraceleft}seller{\isacharbraceright}\ {\isacharampersand}\ {\isasymUnion}\ Range\ aa\ {\isasymsubseteq}\ G{\isacharbraceright}{\isacharparenright}{\isachardoublequoteclose}\ {\isacharparenleft}\isakeyword{is}\ {\isachardoublequoteopen}{\isacharquery}L\ {\isacharequal}\ {\isacharquery}R{\isachardoublequoteclose}{\isacharparenright}\ \isanewline
%
\isadelimproof
%
\endisadelimproof
%
\isatagproof
\isacommand{proof}\isamarkupfalse%
\ {\isacharminus}\isanewline
\ \ \isacommand{have}\isamarkupfalse%
\ {\isachardoublequoteopen}{\isacharquery}L\ {\isacharequal}\ {\isacharparenleft}a{\isasymin}allocationsUniverse{\isacharampersand}\ Domain\ a\ {\isasymsubseteq}\ N{\isacharminus}{\isacharbraceleft}seller{\isacharbraceright}\ {\isacharampersand}\ {\isasymUnion}\ Range\ a\ {\isasymsubseteq}\ G{\isacharparenright}{\isachardoublequoteclose}\ \isacommand{by}\isamarkupfalse%
\ {\isacharparenleft}rule\ lm{\isadigit{3}}{\isadigit{7}}d{\isacharparenright}\isanewline
\ \ \isacommand{moreover}\isamarkupfalse%
\ \isacommand{have}\isamarkupfalse%
\ {\isachardoublequoteopen}{\isachardot}{\isachardot}{\isachardot}\ {\isacharequal}\ {\isacharquery}R{\isachardoublequoteclose}\ \isacommand{by}\isamarkupfalse%
\ {\isacharparenleft}rule\ lm{\isadigit{4}}{\isadigit{2}}{\isacharparenright}\ \isacommand{ultimately}\isamarkupfalse%
\ \isacommand{show}\isamarkupfalse%
\ {\isacharquery}thesis\ \isacommand{by}\isamarkupfalse%
\ presburger\isanewline
\isacommand{qed}\isamarkupfalse%
%
\endisatagproof
{\isafoldproof}%
%
\isadelimproof
\isanewline
%
\endisadelimproof
\isanewline
\isacommand{corollary}\isamarkupfalse%
\ lm{\isadigit{3}}{\isadigit{7}}g{\isacharcolon}\ {\isachardoublequoteopen}a{\isasymin}soldAllocations{\isacharprime}{\isacharprime}\ N\ G{\isacharequal}\isanewline
{\isacharparenleft}a{\isasymin}\ {\isacharparenleft}allocationsUniverse\ {\isasyminter}\ {\isacharbraceleft}aa{\isachardot}\ Domain\ aa\ {\isasymsubseteq}\ N{\isacharminus}{\isacharbraceleft}seller{\isacharbraceright}\ {\isacharampersand}\ {\isasymUnion}\ Range\ aa\ {\isasymsubseteq}\ G{\isacharbraceright}{\isacharparenright}{\isacharparenright}{\isachardoublequoteclose}\ \isanewline
%
\isadelimproof
%
\endisadelimproof
%
\isatagproof
\isacommand{using}\isamarkupfalse%
\ lm{\isadigit{3}}{\isadigit{7}}f\ \isacommand{by}\isamarkupfalse%
\ {\isacharparenleft}metis\ {\isacharparenleft}mono{\isacharunderscore}tags{\isacharparenright}\ Int{\isacharunderscore}iff{\isacharparenright}%
\endisatagproof
{\isafoldproof}%
%
\isadelimproof
\isanewline
%
\endisadelimproof
\isanewline
\isacommand{abbreviation}\isamarkupfalse%
\ {\isachardoublequoteopen}soldAllocations{\isacharprime}{\isacharprime}{\isacharprime}\ N\ G\ {\isacharequal}{\isacharequal}\ \isanewline
allocationsUniverse{\isasyminter}{\isacharbraceleft}aa{\isachardot}\ Domain\ aa{\isasymsubseteq}N{\isacharminus}{\isacharbraceleft}seller{\isacharbraceright}\ {\isacharampersand}\ {\isasymUnion}Range\ aa{\isasymsubseteq}G{\isacharbraceright}{\isachardoublequoteclose}\isanewline
\isanewline
\isacommand{lemma}\isamarkupfalse%
\ lm{\isadigit{4}}{\isadigit{4}}{\isacharcolon}\ \isakeyword{assumes}\ {\isachardoublequoteopen}a\ {\isasymin}\ soldAllocations{\isacharprime}{\isacharprime}{\isacharprime}\ N\ G{\isachardoublequoteclose}\ \isakeyword{shows}\ {\isachardoublequoteopen}a\ {\isacharminus}{\isacharminus}\ n\ {\isasymin}\ soldAllocations{\isacharprime}{\isacharprime}{\isacharprime}\ {\isacharparenleft}N{\isacharminus}{\isacharbraceleft}n{\isacharbraceright}{\isacharparenright}\ G{\isachardoublequoteclose}\isanewline
%
\isadelimproof
%
\endisadelimproof
%
\isatagproof
\isacommand{proof}\isamarkupfalse%
\ {\isacharminus}\isanewline
\ \ \isacommand{let}\isamarkupfalse%
\ {\isacharquery}bb{\isacharequal}seller\ \isacommand{let}\isamarkupfalse%
\ {\isacharquery}d{\isacharequal}Domain\ \isacommand{let}\isamarkupfalse%
\ {\isacharquery}r{\isacharequal}Range\ \isacommand{let}\isamarkupfalse%
\ {\isacharquery}X{\isadigit{2}}{\isacharequal}{\isachardoublequoteopen}{\isacharbraceleft}aa{\isachardot}\ {\isacharquery}d\ aa{\isasymsubseteq}N{\isacharminus}{\isacharbraceleft}{\isacharquery}bb{\isacharbraceright}\ {\isacharampersand}\ {\isasymUnion}{\isacharquery}r\ aa{\isasymsubseteq}G{\isacharbraceright}{\isachardoublequoteclose}\ \isanewline
\ \ \isacommand{let}\isamarkupfalse%
\ {\isacharquery}X{\isadigit{1}}{\isacharequal}{\isachardoublequoteopen}{\isacharbraceleft}aa{\isachardot}\ {\isacharquery}d\ aa{\isasymsubseteq}N{\isacharminus}{\isacharbraceleft}n{\isacharbraceright}{\isacharminus}{\isacharbraceleft}{\isacharquery}bb{\isacharbraceright}\ {\isacharampersand}\ {\isasymUnion}{\isacharquery}r\ aa{\isasymsubseteq}G{\isacharbraceright}{\isachardoublequoteclose}\ \isanewline
\ \ \isacommand{have}\isamarkupfalse%
\ {\isachardoublequoteopen}a{\isasymin}{\isacharquery}X{\isadigit{2}}{\isachardoublequoteclose}\ \isacommand{using}\isamarkupfalse%
\ assms{\isacharparenleft}{\isadigit{1}}{\isacharparenright}\ \isacommand{by}\isamarkupfalse%
\ fast\ \isacommand{then}\isamarkupfalse%
\ \isacommand{have}\isamarkupfalse%
\ \isanewline
\ \ {\isadigit{0}}{\isacharcolon}\ {\isachardoublequoteopen}{\isacharquery}d\ a\ {\isasymsubseteq}\ N{\isacharminus}{\isacharbraceleft}{\isacharquery}bb{\isacharbraceright}\ {\isacharampersand}\ {\isasymUnion}{\isacharquery}r\ a\ {\isasymsubseteq}\ G{\isachardoublequoteclose}\ \isacommand{by}\isamarkupfalse%
\ blast\ \isacommand{then}\isamarkupfalse%
\ \isacommand{have}\isamarkupfalse%
\ {\isachardoublequoteopen}{\isacharquery}d\ {\isacharparenleft}a{\isacharminus}{\isacharminus}n{\isacharparenright}\ {\isasymsubseteq}\ N{\isacharminus}{\isacharbraceleft}{\isacharquery}bb{\isacharbraceright}{\isacharminus}{\isacharbraceleft}n{\isacharbraceright}{\isachardoublequoteclose}\ \isanewline
\ \ \isacommand{using}\isamarkupfalse%
\ outside{\isacharunderscore}reduces{\isacharunderscore}domain\ \isacommand{by}\isamarkupfalse%
\ {\isacharparenleft}metis\ Diff{\isacharunderscore}mono\ subset{\isacharunderscore}refl{\isacharparenright}\ \isacommand{moreover}\isamarkupfalse%
\ \isacommand{have}\isamarkupfalse%
\ \isanewline
\ \ {\isachardoublequoteopen}{\isachardot}{\isachardot}{\isachardot}\ {\isacharequal}\ N{\isacharminus}{\isacharbraceleft}n{\isacharbraceright}{\isacharminus}{\isacharbraceleft}{\isacharquery}bb{\isacharbraceright}{\isachardoublequoteclose}\ \isacommand{by}\isamarkupfalse%
\ fastforce\ \isacommand{ultimately}\isamarkupfalse%
\ \isacommand{have}\isamarkupfalse%
\ \isanewline
\ \ {\isachardoublequoteopen}{\isacharquery}d\ {\isacharparenleft}a{\isacharminus}{\isacharminus}n{\isacharparenright}\ {\isasymsubseteq}\ N{\isacharminus}{\isacharbraceleft}n{\isacharbraceright}{\isacharminus}{\isacharbraceleft}{\isacharquery}bb{\isacharbraceright}{\isachardoublequoteclose}\ \isacommand{by}\isamarkupfalse%
\ blast\ \isacommand{moreover}\isamarkupfalse%
\ \isacommand{have}\isamarkupfalse%
\ {\isachardoublequoteopen}{\isasymUnion}\ {\isacharquery}r\ {\isacharparenleft}a{\isacharminus}{\isacharminus}n{\isacharparenright}\ {\isasymsubseteq}\ G{\isachardoublequoteclose}\ \isanewline
\ \ \isacommand{unfolding}\isamarkupfalse%
\ Outside{\isacharunderscore}def\ \isacommand{using}\isamarkupfalse%
\ {\isadigit{0}}\ \isacommand{by}\isamarkupfalse%
\ blast\ \isacommand{ultimately}\isamarkupfalse%
\ \isacommand{have}\isamarkupfalse%
\ {\isachardoublequoteopen}a\ {\isacharminus}{\isacharminus}\ n\ {\isasymin}\ {\isacharquery}X{\isadigit{1}}{\isachardoublequoteclose}\ \isacommand{by}\isamarkupfalse%
\ fast\ \isacommand{moreover}\isamarkupfalse%
\ \isacommand{have}\isamarkupfalse%
\ \isanewline
\ \ {\isachardoublequoteopen}a{\isacharminus}{\isacharminus}n\ {\isasymin}\ allocationsUniverse{\isachardoublequoteclose}\ \isacommand{using}\isamarkupfalse%
\ assms{\isacharparenleft}{\isadigit{1}}{\isacharparenright}\ Int{\isacharunderscore}iff\ lm{\isadigit{3}}{\isadigit{5}}d\ \isacommand{by}\isamarkupfalse%
\ {\isacharparenleft}metis{\isacharparenleft}lifting{\isacharcomma}mono{\isacharunderscore}tags{\isacharparenright}{\isacharparenright}\ \isanewline
\ \ \isacommand{ultimately}\isamarkupfalse%
\ \isacommand{show}\isamarkupfalse%
\ {\isacharquery}thesis\ \isacommand{by}\isamarkupfalse%
\ blast\isanewline
\isacommand{qed}\isamarkupfalse%
%
\endisatagproof
{\isafoldproof}%
%
\isadelimproof
\isanewline
%
\endisadelimproof
\isanewline
\isacommand{corollary}\isamarkupfalse%
\ lm{\isadigit{3}}{\isadigit{7}}h{\isacharcolon}\ {\isachardoublequoteopen}soldAllocations{\isacharprime}{\isacharprime}\ N\ G{\isacharequal}soldAllocations{\isacharprime}{\isacharprime}{\isacharprime}\ N\ G{\isachardoublequoteclose}\isanewline
{\isacharparenleft}\isakeyword{is}\ {\isachardoublequoteopen}{\isacharquery}L{\isacharequal}{\isacharquery}R{\isachardoublequoteclose}{\isacharparenright}%
\isadelimproof
\ %
\endisadelimproof
%
\isatagproof
\isacommand{proof}\isamarkupfalse%
\ {\isacharminus}\ \isacommand{{\isacharbraceleft}}\isamarkupfalse%
\isacommand{fix}\isamarkupfalse%
\ a\ \isacommand{have}\isamarkupfalse%
\ {\isachardoublequoteopen}a\ {\isasymin}\ {\isacharquery}L\ {\isacharequal}\ {\isacharparenleft}a\ {\isasymin}\ {\isacharquery}R{\isacharparenright}{\isachardoublequoteclose}\ \isacommand{by}\isamarkupfalse%
\ {\isacharparenleft}rule\ lm{\isadigit{3}}{\isadigit{7}}g{\isacharparenright}\isacommand{{\isacharbraceright}}\isamarkupfalse%
\ \isacommand{thus}\isamarkupfalse%
\ {\isacharquery}thesis\ \isacommand{by}\isamarkupfalse%
\ blast\ \isacommand{qed}\isamarkupfalse%
%
\endisatagproof
{\isafoldproof}%
%
\isadelimproof
%
\endisadelimproof
\isanewline
\isanewline
\isacommand{lemma}\isamarkupfalse%
\ lm{\isadigit{2}}{\isadigit{8}}e{\isacharcolon}\ {\isachardoublequoteopen}soldAllocations{\isacharequal}soldAllocations{\isacharprime}\ {\isacharampersand}\ soldAllocations{\isacharprime}{\isacharequal}soldAllocations{\isacharprime}{\isacharprime}\ {\isacharampersand}\isanewline
soldAllocations{\isacharprime}{\isacharprime}{\isacharequal}soldAllocations{\isacharprime}{\isacharprime}{\isacharprime}{\isachardoublequoteclose}%
\isadelimproof
\ %
\endisadelimproof
%
\isatagproof
\isacommand{using}\isamarkupfalse%
\ lm{\isadigit{3}}{\isadigit{7}}h\ lm{\isadigit{2}}{\isadigit{8}}d\ \isacommand{by}\isamarkupfalse%
\ metis%
\endisatagproof
{\isafoldproof}%
%
\isadelimproof
%
\endisadelimproof
\isanewline
\isanewline
\isacommand{corollary}\isamarkupfalse%
\ lm{\isadigit{4}}{\isadigit{4}}b{\isacharcolon}\ \isakeyword{assumes}\ {\isachardoublequoteopen}a\ {\isasymin}\ soldAllocations\ N\ G{\isachardoublequoteclose}\ \isakeyword{shows}\ {\isachardoublequoteopen}a\ {\isacharminus}{\isacharminus}\ n\ {\isasymin}\ soldAllocations\ {\isacharparenleft}N{\isacharminus}{\isacharbraceleft}n{\isacharbraceright}{\isacharparenright}\ G{\isachardoublequoteclose}\isanewline
%
\isadelimproof
\isanewline
%
\endisadelimproof
%
\isatagproof
\isacommand{proof}\isamarkupfalse%
\ {\isacharminus}\ \isanewline
\isacommand{let}\isamarkupfalse%
\ {\isacharquery}A{\isacharprime}{\isacharequal}soldAllocations{\isacharprime}{\isacharprime}{\isacharprime}\ \isacommand{have}\isamarkupfalse%
\ {\isachardoublequoteopen}a\ {\isasymin}\ {\isacharquery}A{\isacharprime}\ N\ G{\isachardoublequoteclose}\ \isacommand{using}\isamarkupfalse%
\ assms\ lm{\isadigit{2}}{\isadigit{8}}e\ \isacommand{by}\isamarkupfalse%
\ metis\ \isacommand{then}\isamarkupfalse%
\isanewline
\isacommand{have}\isamarkupfalse%
\ {\isachardoublequoteopen}a\ {\isacharminus}{\isacharminus}\ n\ {\isasymin}\ {\isacharquery}A{\isacharprime}\ {\isacharparenleft}N{\isacharminus}{\isacharbraceleft}n{\isacharbraceright}{\isacharparenright}\ G{\isachardoublequoteclose}\ \isacommand{by}\isamarkupfalse%
\ {\isacharparenleft}rule\ lm{\isadigit{4}}{\isadigit{4}}{\isacharparenright}\ \isacommand{thus}\isamarkupfalse%
\ {\isacharquery}thesis\ \isacommand{using}\isamarkupfalse%
\ lm{\isadigit{2}}{\isadigit{8}}e\ \isacommand{by}\isamarkupfalse%
\ metis\ \isanewline
\isacommand{qed}\isamarkupfalse%
%
\endisatagproof
{\isafoldproof}%
%
\isadelimproof
\isanewline
%
\endisadelimproof
\isanewline
\isacommand{corollary}\isamarkupfalse%
\ lm{\isadigit{3}}{\isadigit{7}}i{\isacharcolon}\ \isakeyword{assumes}\ {\isachardoublequoteopen}G{\isadigit{1}}\ {\isasymsubseteq}\ G{\isadigit{2}}{\isachardoublequoteclose}\ \isakeyword{shows}\ {\isachardoublequoteopen}soldAllocations{\isacharprime}{\isacharprime}{\isacharprime}\ N\ G{\isadigit{1}}\ {\isasymsubseteq}\ soldAllocations{\isacharprime}{\isacharprime}{\isacharprime}\ N\ G{\isadigit{2}}{\isachardoublequoteclose}\isanewline
%
\isadelimproof
%
\endisadelimproof
%
\isatagproof
\isacommand{using}\isamarkupfalse%
\ assms\ \isacommand{by}\isamarkupfalse%
\ blast%
\endisatagproof
{\isafoldproof}%
%
\isadelimproof
\isanewline
%
\endisadelimproof
\isanewline
\isacommand{corollary}\isamarkupfalse%
\ lm{\isadigit{3}}{\isadigit{3}}{\isacharcolon}\ \isakeyword{assumes}\ {\isachardoublequoteopen}G{\isadigit{1}}\ {\isasymsubseteq}\ G{\isadigit{2}}{\isachardoublequoteclose}\ \isakeyword{shows}\ {\isachardoublequoteopen}soldAllocations{\isacharprime}{\isacharprime}\ N\ G{\isadigit{1}}\ {\isasymsubseteq}\ soldAllocations{\isacharprime}{\isacharprime}\ N\ G{\isadigit{2}}{\isachardoublequoteclose}\isanewline
%
\isadelimproof
%
\endisadelimproof
%
\isatagproof
\isacommand{using}\isamarkupfalse%
\ assms\ lm{\isadigit{3}}{\isadigit{7}}i\ lm{\isadigit{3}}{\isadigit{7}}h\ \isanewline
\isacommand{proof}\isamarkupfalse%
\ {\isacharminus}\isanewline
\isacommand{have}\isamarkupfalse%
\ {\isachardoublequoteopen}soldAllocations{\isacharprime}{\isacharprime}\ N\ G{\isadigit{1}}\ {\isacharequal}\ soldAllocations{\isacharprime}{\isacharprime}{\isacharprime}\ N\ G{\isadigit{1}}{\isachardoublequoteclose}\ \isacommand{by}\isamarkupfalse%
\ {\isacharparenleft}rule\ lm{\isadigit{3}}{\isadigit{7}}h{\isacharparenright}\isanewline
\isacommand{moreover}\isamarkupfalse%
\ \isacommand{have}\isamarkupfalse%
\ {\isachardoublequoteopen}{\isachardot}{\isachardot}{\isachardot}\ {\isasymsubseteq}\ soldAllocations{\isacharprime}{\isacharprime}{\isacharprime}\ N\ G{\isadigit{2}}{\isachardoublequoteclose}\ \isacommand{using}\isamarkupfalse%
\ assms{\isacharparenleft}{\isadigit{1}}{\isacharparenright}\ \isacommand{by}\isamarkupfalse%
\ {\isacharparenleft}rule\ lm{\isadigit{3}}{\isadigit{7}}i{\isacharparenright}\isanewline
\isacommand{moreover}\isamarkupfalse%
\ \isacommand{have}\isamarkupfalse%
\ {\isachardoublequoteopen}{\isachardot}{\isachardot}{\isachardot}\ {\isacharequal}\ soldAllocations{\isacharprime}{\isacharprime}\ N\ G{\isadigit{2}}{\isachardoublequoteclose}\ \isacommand{using}\isamarkupfalse%
\ lm{\isadigit{3}}{\isadigit{7}}h\ \isacommand{by}\isamarkupfalse%
\ metis\isanewline
\isacommand{ultimately}\isamarkupfalse%
\ \isacommand{show}\isamarkupfalse%
\ {\isacharquery}thesis\ \isacommand{by}\isamarkupfalse%
\ auto\isanewline
\isacommand{qed}\isamarkupfalse%
%
\endisatagproof
{\isafoldproof}%
%
\isadelimproof
\isanewline
%
\endisadelimproof
\isanewline
\isacommand{abbreviation}\isamarkupfalse%
\ {\isachardoublequoteopen}maximalAllocations{\isacharprime}{\isacharprime}\ N\ {\isasymOmega}\ b\ {\isacharequal}{\isacharequal}\ argmax\ {\isacharparenleft}setsum\ b{\isacharparenright}\ {\isacharparenleft}soldAllocations\ N\ {\isasymOmega}{\isacharparenright}{\isachardoublequoteclose}\isanewline
\isanewline
\isacommand{abbreviation}\isamarkupfalse%
\ {\isachardoublequoteopen}maximalStrictAllocations{\isacharprime}\ N\ G\ b{\isacharequal}{\isacharequal}\isanewline
argmax\ {\isacharparenleft}setsum\ b{\isacharparenright}\ {\isacharparenleft}allAllocations\ {\isacharparenleft}{\isacharbraceleft}seller{\isacharbraceright}{\isasymunion}N{\isacharparenright}\ G{\isacharparenright}{\isachardoublequoteclose}\isanewline
\isanewline
\isanewline
\isanewline
\isacommand{corollary}\isamarkupfalse%
\ lm{\isadigit{4}}{\isadigit{3}}d{\isacharcolon}\ \isakeyword{assumes}\ {\isachardoublequoteopen}a\ {\isasymin}\ allocationsUniverse{\isachardoublequoteclose}\ \isakeyword{shows}\ \isanewline
{\isachardoublequoteopen}{\isacharparenleft}a\ outside\ {\isacharparenleft}X{\isasymunion}{\isacharbraceleft}i{\isacharbraceright}{\isacharparenright}{\isacharparenright}\ {\isasymunion}\ {\isacharparenleft}{\isacharbraceleft}i{\isacharbraceright}{\isasymtimes}{\isacharparenleft}{\isacharbraceleft}{\isasymUnion}{\isacharparenleft}a{\isacharbackquote}{\isacharbackquote}{\isacharparenleft}X{\isasymunion}{\isacharbraceleft}i{\isacharbraceright}{\isacharparenright}{\isacharparenright}{\isacharbraceright}{\isacharminus}{\isacharbraceleft}{\isacharbraceleft}{\isacharbraceright}{\isacharbraceright}{\isacharparenright}{\isacharparenright}\ {\isasymin}\ allocationsUniverse{\isachardoublequoteclose}%
\isadelimproof
\ %
\endisadelimproof
%
\isatagproof
\isacommand{using}\isamarkupfalse%
\ assms\ lm{\isadigit{4}}{\isadigit{3}}b\ \isanewline
\isacommand{by}\isamarkupfalse%
\ fastforce%
\endisatagproof
{\isafoldproof}%
%
\isadelimproof
%
\endisadelimproof
\isanewline
\isanewline
\isanewline
\isanewline
\isacommand{abbreviation}\isamarkupfalse%
\ {\isachardoublequoteopen}randomBids{\isacharprime}\ N\ {\isasymOmega}\ b\ random{\isacharequal}{\isacharequal}resolvingBid{\isacharprime}\ {\isacharparenleft}N{\isasymunion}{\isacharbraceleft}seller{\isacharbraceright}{\isacharparenright}\ {\isasymOmega}\ b\ random{\isachardoublequoteclose}\isanewline
\isanewline
\isanewline
\isanewline
\isacommand{abbreviation}\isamarkupfalse%
\ {\isachardoublequoteopen}vcgas\ N\ G\ b\ r\ \ {\isacharequal}{\isacharequal}\ Outside{\isacharprime}\ {\isacharbraceleft}seller{\isacharbraceright}\ {\isacharbackquote}\ \isanewline
\ \ \ {\isacharparenleft}{\isacharparenleft}argmax{\isasymcirc}setsum{\isacharparenright}\ {\isacharparenleft}randomBids{\isacharprime}\ N\ G\ b\ r{\isacharparenright}\isanewline
\ \ \ \ {\isacharparenleft}{\isacharparenleft}argmax{\isasymcirc}setsum{\isacharparenright}\ b\ {\isacharparenleft}allAllocations\ {\isacharparenleft}N{\isasymunion}{\isacharbraceleft}seller{\isacharbraceright}{\isacharparenright}\ {\isacharparenleft}set\ G{\isacharparenright}{\isacharparenright}{\isacharparenright}{\isacharparenright}{\isachardoublequoteclose}\isanewline
\isacommand{abbreviation}\isamarkupfalse%
\ {\isachardoublequoteopen}vcga\ N\ G\ b\ r\ {\isacharequal}{\isacharequal}\ the{\isacharunderscore}elem\ {\isacharparenleft}vcgas\ N\ G\ b\ r{\isacharparenright}{\isachardoublequoteclose}\isanewline
\isacommand{abbreviation}\isamarkupfalse%
\ {\isachardoublequoteopen}vcga{\isacharprime}\ N\ G\ b\ r\ {\isacharequal}{\isacharequal}\ {\isacharparenleft}the{\isacharunderscore}elem\isanewline
{\isacharparenleft}argmax\ {\isacharparenleft}setsum\ {\isacharparenleft}randomBids{\isacharprime}\ N\ G\ b\ r{\isacharparenright}{\isacharparenright}\ {\isacharparenleft}maximalStrictAllocations{\isacharprime}\ N\ {\isacharparenleft}set\ G{\isacharparenright}\ b{\isacharparenright}{\isacharparenright}{\isacharparenright}\ {\isacharminus}{\isacharminus}\ seller{\isachardoublequoteclose}\isanewline
\isanewline
\isacommand{lemma}\isamarkupfalse%
\ lm{\isadigit{0}}{\isadigit{0}}{\isadigit{1}}{\isacharcolon}\ \isakeyword{assumes}\ \isanewline
{\isachardoublequoteopen}card\ {\isacharparenleft}{\isacharparenleft}argmax{\isasymcirc}setsum{\isacharparenright}\ {\isacharparenleft}randomBids{\isacharprime}\ N\ G\ b\ r{\isacharparenright}\ {\isacharparenleft}{\isacharparenleft}argmax{\isasymcirc}setsum{\isacharparenright}\ b\ {\isacharparenleft}allAllocations\ {\isacharparenleft}N{\isasymunion}{\isacharbraceleft}seller{\isacharbraceright}{\isacharparenright}\ {\isacharparenleft}set\ G{\isacharparenright}{\isacharparenright}{\isacharparenright}{\isacharparenright}{\isacharequal}{\isadigit{1}}{\isachardoublequoteclose}\ \isanewline
\isakeyword{shows}\ {\isachardoublequoteopen}vcga\ N\ G\ b\ r\ {\isacharequal}\ \isanewline
{\isacharparenleft}the{\isacharunderscore}elem\isanewline
{\isacharparenleft}{\isacharparenleft}argmax{\isasymcirc}setsum{\isacharparenright}\ {\isacharparenleft}randomBids{\isacharprime}\ N\ G\ b\ r{\isacharparenright}\ {\isacharparenleft}{\isacharparenleft}argmax{\isasymcirc}setsum{\isacharparenright}\ b\ {\isacharparenleft}allAllocations\ {\isacharparenleft}{\isacharbraceleft}seller{\isacharbraceright}{\isasymunion}N{\isacharparenright}\ {\isacharparenleft}set\ G{\isacharparenright}{\isacharparenright}{\isacharparenright}{\isacharparenright}{\isacharparenright}\ {\isacharminus}{\isacharminus}\ seller{\isachardoublequoteclose}\isanewline
%
\isadelimproof
%
\endisadelimproof
%
\isatagproof
\isacommand{using}\isamarkupfalse%
\ assms\ cardOneTheElem\ \isacommand{by}\isamarkupfalse%
\ auto%
\endisatagproof
{\isafoldproof}%
%
\isadelimproof
\isanewline
%
\endisadelimproof
\isanewline
\isacommand{corollary}\isamarkupfalse%
\ lm{\isadigit{0}}{\isadigit{0}}{\isadigit{1}}b{\isacharcolon}\ \isakeyword{assumes}\ \isanewline
{\isachardoublequoteopen}card\ {\isacharparenleft}{\isacharparenleft}argmax{\isasymcirc}setsum{\isacharparenright}\ {\isacharparenleft}randomBids{\isacharprime}\ N\ G\ b\ r{\isacharparenright}\ {\isacharparenleft}{\isacharparenleft}argmax{\isasymcirc}setsum{\isacharparenright}\ b\ {\isacharparenleft}allAllocations\ {\isacharparenleft}N{\isasymunion}{\isacharbraceleft}seller{\isacharbraceright}{\isacharparenright}\ {\isacharparenleft}set\ G{\isacharparenright}{\isacharparenright}{\isacharparenright}{\isacharparenright}{\isacharequal}{\isadigit{1}}{\isachardoublequoteclose}\isanewline
\isakeyword{shows}\ {\isachardoublequoteopen}vcga\ N\ G\ b\ r\ {\isacharequal}\ vcga{\isacharprime}\ N\ G\ b\ r{\isachardoublequoteclose}\ {\isacharparenleft}\isakeyword{is}\ {\isachardoublequoteopen}{\isacharquery}l{\isacharequal}{\isacharquery}r{\isachardoublequoteclose}{\isacharparenright}%
\isadelimproof
\ %
\endisadelimproof
%
\isatagproof
\isacommand{using}\isamarkupfalse%
\ assms\ lm{\isadigit{0}}{\isadigit{0}}{\isadigit{1}}\isanewline
\isacommand{proof}\isamarkupfalse%
\ {\isacharminus}\isanewline
\isacommand{have}\isamarkupfalse%
\ {\isachardoublequoteopen}{\isacharquery}l{\isacharequal}{\isacharparenleft}the{\isacharunderscore}elem\ {\isacharparenleft}{\isacharparenleft}argmax{\isasymcirc}setsum{\isacharparenright}\ {\isacharparenleft}randomBids{\isacharprime}\ N\ G\ b\ r{\isacharparenright}\ \isanewline
{\isacharparenleft}{\isacharparenleft}argmax{\isasymcirc}setsum{\isacharparenright}\ b\ {\isacharparenleft}allAllocations\ {\isacharparenleft}{\isacharbraceleft}seller{\isacharbraceright}{\isasymunion}N{\isacharparenright}\ {\isacharparenleft}set\ G{\isacharparenright}{\isacharparenright}{\isacharparenright}{\isacharparenright}{\isacharparenright}\ {\isacharminus}{\isacharminus}\ seller{\isachardoublequoteclose}\isanewline
\isacommand{using}\isamarkupfalse%
\ assms\ \isacommand{by}\isamarkupfalse%
\ {\isacharparenleft}rule\ lm{\isadigit{0}}{\isadigit{0}}{\isadigit{1}}{\isacharparenright}\ \isacommand{moreover}\isamarkupfalse%
\ \isacommand{have}\isamarkupfalse%
\ {\isachardoublequoteopen}{\isachardot}{\isachardot}{\isachardot}\ {\isacharequal}\ {\isacharquery}r{\isachardoublequoteclose}\ \isacommand{by}\isamarkupfalse%
\ force\ \isacommand{ultimately}\isamarkupfalse%
\ \isacommand{show}\isamarkupfalse%
\ {\isacharquery}thesis\ \isacommand{by}\isamarkupfalse%
\ blast\isanewline
\isacommand{qed}\isamarkupfalse%
%
\endisatagproof
{\isafoldproof}%
%
\isadelimproof
%
\endisadelimproof
\isanewline
\isanewline
\isacommand{lemma}\isamarkupfalse%
\ lm{\isadigit{9}}{\isadigit{2}}c{\isacharcolon}\ \isakeyword{assumes}\ {\isachardoublequoteopen}distinct\ G{\isachardoublequoteclose}\ {\isachardoublequoteopen}set\ G\ {\isasymnoteq}\ {\isacharbraceleft}{\isacharbraceright}{\isachardoublequoteclose}\ {\isachardoublequoteopen}finite\ N{\isachardoublequoteclose}\ \isakeyword{shows}\isanewline
{\isachardoublequoteopen}card\ {\isacharparenleft}\isanewline
{\isacharparenleft}argmax{\isasymcirc}setsum{\isacharparenright}\ {\isacharparenleft}randomBids{\isacharprime}\ N\ G\ bids\ random{\isacharparenright}\isanewline
\ \ \ \ {\isacharparenleft}{\isacharparenleft}argmax{\isasymcirc}setsum{\isacharparenright}\ bids\ {\isacharparenleft}allAllocations\ {\isacharparenleft}N{\isasymunion}{\isacharbraceleft}seller{\isacharbraceright}{\isacharparenright}\ {\isacharparenleft}set\ G{\isacharparenright}{\isacharparenright}{\isacharparenright}{\isacharparenright}{\isacharequal}{\isadigit{1}}{\isachardoublequoteclose}\ {\isacharparenleft}\isakeyword{is}\ {\isachardoublequoteopen}card\ {\isacharquery}l{\isacharequal}{\isacharunderscore}{\isachardoublequoteclose}{\isacharparenright}\isanewline
%
\isadelimproof
%
\endisadelimproof
%
\isatagproof
\isacommand{proof}\isamarkupfalse%
\ {\isacharminus}\ \isanewline
\ \ \isacommand{let}\isamarkupfalse%
\ {\isacharquery}N{\isacharequal}{\isachardoublequoteopen}N{\isasymunion}{\isacharbraceleft}seller{\isacharbraceright}{\isachardoublequoteclose}\ \isacommand{let}\isamarkupfalse%
\ {\isacharquery}b{\isacharprime}{\isacharequal}{\isachardoublequoteopen}randomBids{\isacharprime}\ N\ G\ bids\ random{\isachardoublequoteclose}\ \isacommand{let}\isamarkupfalse%
\ {\isacharquery}s{\isacharequal}setsum\ \isacommand{let}\isamarkupfalse%
\ {\isacharquery}a{\isacharequal}argmax\ \isacommand{let}\isamarkupfalse%
\ {\isacharquery}f{\isacharequal}{\isachardoublequoteopen}{\isacharquery}a\ {\isasymcirc}\ {\isacharquery}s{\isachardoublequoteclose}\isanewline
\ \ \isacommand{have}\isamarkupfalse%
\ {\isadigit{1}}{\isacharcolon}\ {\isachardoublequoteopen}{\isacharquery}N{\isasymnoteq}{\isacharbraceleft}{\isacharbraceright}{\isachardoublequoteclose}\ \isacommand{by}\isamarkupfalse%
\ auto\ \isacommand{have}\isamarkupfalse%
\ {\isadigit{4}}{\isacharcolon}\ {\isachardoublequoteopen}finite\ {\isacharquery}N{\isachardoublequoteclose}\ \isacommand{using}\isamarkupfalse%
\ assms{\isacharparenleft}{\isadigit{3}}{\isacharparenright}\ \isacommand{by}\isamarkupfalse%
\ simp\isanewline
\ \ \isacommand{have}\isamarkupfalse%
\ {\isachardoublequoteopen}{\isacharquery}a\ {\isacharparenleft}{\isacharquery}s\ {\isacharquery}b{\isacharprime}{\isacharparenright}\ {\isacharparenleft}{\isacharquery}a\ {\isacharparenleft}{\isacharquery}s\ bids{\isacharparenright}\ {\isacharparenleft}allAllocations\ {\isacharquery}N\ {\isacharparenleft}set\ G{\isacharparenright}{\isacharparenright}{\isacharparenright}{\isacharequal}\isanewline
\ \ {\isacharbraceleft}chosenAllocation{\isacharprime}\ {\isacharquery}N\ G\ bids\ random{\isacharbraceright}{\isachardoublequoteclose}\ {\isacharparenleft}\isakeyword{is}\ {\isachardoublequoteopen}{\isacharquery}L{\isacharequal}{\isacharquery}R{\isachardoublequoteclose}{\isacharparenright}\isanewline
\ \ \isacommand{using}\isamarkupfalse%
\ {\isadigit{1}}\ assms{\isacharparenleft}{\isadigit{1}}{\isacharparenright}\ assms{\isacharparenleft}{\isadigit{2}}{\isacharparenright}\ {\isadigit{4}}\ \isacommand{by}\isamarkupfalse%
\ {\isacharparenleft}rule\ lm{\isadigit{9}}{\isadigit{2}}{\isacharparenright}\isanewline
\ \ \isacommand{moreover}\isamarkupfalse%
\ \isacommand{have}\isamarkupfalse%
\ {\isachardoublequoteopen}{\isacharquery}L{\isacharequal}\ {\isacharquery}f\ {\isacharquery}b{\isacharprime}\ {\isacharparenleft}{\isacharquery}f\ bids\ {\isacharparenleft}allAllocations\ {\isacharquery}N\ {\isacharparenleft}set\ G{\isacharparenright}{\isacharparenright}{\isacharparenright}{\isachardoublequoteclose}\ \isacommand{by}\isamarkupfalse%
\ auto\isanewline
\ \ \isacommand{ultimately}\isamarkupfalse%
\ \isacommand{have}\isamarkupfalse%
\ {\isachardoublequoteopen}{\isacharquery}l{\isacharequal}{\isacharbraceleft}chosenAllocation{\isacharprime}\ {\isacharquery}N\ G\ bids\ random{\isacharbraceright}{\isachardoublequoteclose}\ \isacommand{by}\isamarkupfalse%
\ presburger\isanewline
\ \ \isacommand{moreover}\isamarkupfalse%
\ \isacommand{have}\isamarkupfalse%
\ {\isachardoublequoteopen}card\ {\isachardot}{\isachardot}{\isachardot}{\isacharequal}{\isadigit{1}}{\isachardoublequoteclose}\ \isacommand{by}\isamarkupfalse%
\ simp\ \isacommand{ultimately}\isamarkupfalse%
\ \isacommand{show}\isamarkupfalse%
\ {\isacharquery}thesis\ \isacommand{by}\isamarkupfalse%
\ presburger\ \isanewline
\isacommand{qed}\isamarkupfalse%
%
\endisatagproof
{\isafoldproof}%
%
\isadelimproof
\isanewline
%
\endisadelimproof
\isanewline
\isacommand{lemma}\isamarkupfalse%
\ lm{\isadigit{0}}{\isadigit{0}}{\isadigit{2}}{\isacharcolon}\ \isakeyword{assumes}\ {\isachardoublequoteopen}distinct\ G{\isachardoublequoteclose}\ {\isachardoublequoteopen}set\ G\ {\isasymnoteq}\ {\isacharbraceleft}{\isacharbraceright}{\isachardoublequoteclose}\ {\isachardoublequoteopen}finite\ N{\isachardoublequoteclose}\ \isakeyword{shows}\isanewline
{\isachardoublequoteopen}vcga\ N\ G\ b\ r\ {\isacharequal}\ vcga{\isacharprime}\ N\ G\ b\ r{\isachardoublequoteclose}\ {\isacharparenleft}\isakeyword{is}\ {\isachardoublequoteopen}{\isacharquery}l{\isacharequal}{\isacharquery}r{\isachardoublequoteclose}{\isacharparenright}%
\isadelimproof
\ %
\endisadelimproof
%
\isatagproof
\isacommand{using}\isamarkupfalse%
\ assms\ lm{\isadigit{0}}{\isadigit{0}}{\isadigit{1}}b\ lm{\isadigit{9}}{\isadigit{2}}c\ \isacommand{by}\isamarkupfalse%
\ blast%
\endisatagproof
{\isafoldproof}%
%
\isadelimproof
%
\endisadelimproof
\isanewline
\isanewline
\isacommand{theorem}\isamarkupfalse%
\ vcgaDefiniteness{\isacharcolon}\ \isakeyword{assumes}\ {\isachardoublequoteopen}distinct\ G{\isachardoublequoteclose}\ {\isachardoublequoteopen}set\ G\ {\isasymnoteq}\ {\isacharbraceleft}{\isacharbraceright}{\isachardoublequoteclose}\ {\isachardoublequoteopen}finite\ N{\isachardoublequoteclose}\ \isakeyword{shows}\isanewline
{\isachardoublequoteopen}card\ {\isacharparenleft}vcgas\ N\ G\ b\ r{\isacharparenright}{\isacharequal}{\isadigit{1}}{\isachardoublequoteclose}\isanewline
%
\isadelimproof
%
\endisadelimproof
%
\isatagproof
\isacommand{using}\isamarkupfalse%
\ assms\ lm{\isadigit{9}}{\isadigit{2}}c\ cardOneImageCardOne\ \isanewline
\isacommand{proof}\isamarkupfalse%
\ {\isacharminus}\isanewline
\isacommand{have}\isamarkupfalse%
\ {\isachardoublequoteopen}card\ {\isacharparenleft}{\isacharparenleft}argmax{\isasymcirc}setsum{\isacharparenright}\ {\isacharparenleft}randomBids{\isacharprime}\ N\ G\ b\ r{\isacharparenright}\ {\isacharparenleft}{\isacharparenleft}argmax{\isasymcirc}setsum{\isacharparenright}\ b\ {\isacharparenleft}allAllocations\ {\isacharparenleft}N{\isasymunion}{\isacharbraceleft}seller{\isacharbraceright}{\isacharparenright}\ {\isacharparenleft}set\ G{\isacharparenright}{\isacharparenright}{\isacharparenright}{\isacharparenright}{\isacharequal}{\isadigit{1}}{\isachardoublequoteclose}\ \isanewline
{\isacharparenleft}\isakeyword{is}\ {\isachardoublequoteopen}card\ {\isacharquery}X{\isacharequal}{\isacharunderscore}{\isachardoublequoteclose}{\isacharparenright}\ \isacommand{using}\isamarkupfalse%
\ assms\ lm{\isadigit{9}}{\isadigit{2}}c\ \isacommand{by}\isamarkupfalse%
\ blast\isanewline
\isacommand{moreover}\isamarkupfalse%
\ \isacommand{have}\isamarkupfalse%
\ {\isachardoublequoteopen}{\isacharparenleft}Outside{\isacharprime}{\isacharbraceleft}seller{\isacharbraceright}{\isacharparenright}\ {\isacharbackquote}\ {\isacharquery}X\ {\isacharequal}\ vcgas\ N\ G\ b\ r{\isachardoublequoteclose}\ \isacommand{by}\isamarkupfalse%
\ blast\isanewline
\isacommand{ultimately}\isamarkupfalse%
\ \isacommand{show}\isamarkupfalse%
\ {\isacharquery}thesis\ \isacommand{using}\isamarkupfalse%
\ cardOneImageCardOne\ \isacommand{by}\isamarkupfalse%
\ blast\isanewline
\isacommand{qed}\isamarkupfalse%
%
\endisatagproof
{\isafoldproof}%
%
\isadelimproof
\isanewline
%
\endisadelimproof
\isanewline
\isacommand{theorem}\isamarkupfalse%
\ vcgpDefiniteness{\isacharcolon}\ \isakeyword{assumes}\ {\isachardoublequoteopen}distinct\ G{\isachardoublequoteclose}\ {\isachardoublequoteopen}set\ G\ {\isasymnoteq}\ {\isacharbraceleft}{\isacharbraceright}{\isachardoublequoteclose}\ {\isachardoublequoteopen}finite\ N{\isachardoublequoteclose}\ \isakeyword{shows}\isanewline
{\isachardoublequoteopen}{\isasymexists}{\isacharbang}\ y{\isachardot}\ vcgap\ N\ G\ b\ r\ n{\isacharequal}y{\isachardoublequoteclose}%
\isadelimproof
\ %
\endisadelimproof
%
\isatagproof
\isacommand{using}\isamarkupfalse%
\ assms\ vcgaDefiniteness\ \isacommand{by}\isamarkupfalse%
\ simp%
\endisatagproof
{\isafoldproof}%
%
\isadelimproof
%
\endisadelimproof
%
\begin{isamarkuptext}%
Here we are showing that our way of randomizing using randomBids' actually breaks the tie:
we are left with a singleton after the tie-breaking step.%
\end{isamarkuptext}%
\isamarkuptrue%
\isacommand{lemma}\isamarkupfalse%
\ lm{\isadigit{9}}{\isadigit{2}}b{\isacharcolon}\ \isakeyword{assumes}\ {\isachardoublequoteopen}distinct\ G{\isachardoublequoteclose}\ {\isachardoublequoteopen}set\ G\ {\isasymnoteq}\ {\isacharbraceleft}{\isacharbraceright}{\isachardoublequoteclose}\ {\isachardoublequoteopen}finite\ N{\isachardoublequoteclose}\ \isakeyword{shows}\ \isanewline
{\isachardoublequoteopen}card\ {\isacharparenleft}argmax\ {\isacharparenleft}setsum\ {\isacharparenleft}randomBids{\isacharprime}\ N\ G\ b\ r{\isacharparenright}{\isacharparenright}\ {\isacharparenleft}maximalStrictAllocations{\isacharprime}\ N\ {\isacharparenleft}set\ G{\isacharparenright}\ b{\isacharparenright}{\isacharparenright}{\isacharequal}{\isadigit{1}}{\isachardoublequoteclose}\isanewline
{\isacharparenleft}\isakeyword{is}\ {\isachardoublequoteopen}card\ {\isacharquery}L{\isacharequal}{\isacharunderscore}{\isachardoublequoteclose}{\isacharparenright}\isanewline
%
\isadelimproof
\isanewline
%
\endisadelimproof
%
\isatagproof
\isacommand{proof}\isamarkupfalse%
\ {\isacharminus}\isanewline
\isacommand{let}\isamarkupfalse%
\ {\isacharquery}n{\isacharequal}{\isachardoublequoteopen}{\isacharbraceleft}seller{\isacharbraceright}{\isachardoublequoteclose}\ \isacommand{have}\isamarkupfalse%
\ \isanewline
{\isadigit{1}}{\isacharcolon}\ {\isachardoublequoteopen}{\isacharparenleft}{\isacharquery}n\ {\isasymunion}\ N{\isacharparenright}{\isasymnoteq}{\isacharbraceleft}{\isacharbraceright}{\isachardoublequoteclose}\ \isacommand{by}\isamarkupfalse%
\ simp\ \isacommand{have}\isamarkupfalse%
\ \isanewline
{\isadigit{4}}{\isacharcolon}\ {\isachardoublequoteopen}finite\ {\isacharparenleft}{\isacharquery}n{\isasymunion}N{\isacharparenright}{\isachardoublequoteclose}\ \isacommand{using}\isamarkupfalse%
\ assms{\isacharparenleft}{\isadigit{3}}{\isacharparenright}\ \isacommand{by}\isamarkupfalse%
\ fast\ \isacommand{have}\isamarkupfalse%
\ \isanewline
{\isachardoublequoteopen}terminatingAuctionRel\ {\isacharparenleft}{\isacharquery}n{\isasymunion}N{\isacharparenright}\ G\ b\ r\ {\isacharequal}\ {\isacharbraceleft}chosenAllocation{\isacharprime}\ {\isacharparenleft}{\isacharquery}n{\isasymunion}N{\isacharparenright}\ G\ b\ r{\isacharbraceright}{\isachardoublequoteclose}\ \isacommand{using}\isamarkupfalse%
\ {\isadigit{1}}\ assms{\isacharparenleft}{\isadigit{1}}{\isacharparenright}\ \isanewline
assms{\isacharparenleft}{\isadigit{2}}{\isacharparenright}\ {\isadigit{4}}\ \isacommand{by}\isamarkupfalse%
\ {\isacharparenleft}rule\ lm{\isadigit{9}}{\isadigit{2}}{\isacharparenright}\ \isacommand{moreover}\isamarkupfalse%
\ \isacommand{have}\isamarkupfalse%
\ {\isachardoublequoteopen}{\isacharquery}L\ {\isacharequal}\ terminatingAuctionRel\ {\isacharparenleft}{\isacharquery}n{\isasymunion}N{\isacharparenright}\ G\ b\ r{\isachardoublequoteclose}\ \isacommand{by}\isamarkupfalse%
\ auto\isanewline
\isacommand{ultimately}\isamarkupfalse%
\ \isacommand{show}\isamarkupfalse%
\ {\isacharquery}thesis\ \isacommand{by}\isamarkupfalse%
\ auto\isanewline
\isacommand{qed}\isamarkupfalse%
%
\endisatagproof
{\isafoldproof}%
%
\isadelimproof
\isanewline
%
\endisadelimproof
\isanewline
\isacommand{lemma}\isamarkupfalse%
\ {\isachardoublequoteopen}argmax\ {\isacharparenleft}setsum\ {\isacharparenleft}randomBids{\isacharprime}\ N\ G\ b\ r{\isacharparenright}{\isacharparenright}\ {\isacharparenleft}maximalStrictAllocations{\isacharprime}\ N\ {\isacharparenleft}set\ G{\isacharparenright}\ b{\isacharparenright}\ {\isasymsubseteq}\ \isanewline
maximalStrictAllocations{\isacharprime}\ N\ {\isacharparenleft}set\ G{\isacharparenright}\ b{\isachardoublequoteclose}%
\isadelimproof
\ %
\endisadelimproof
%
\isatagproof
\isacommand{by}\isamarkupfalse%
\ auto%
\endisatagproof
{\isafoldproof}%
%
\isadelimproof
%
\endisadelimproof
\isanewline
\isanewline
\isacommand{corollary}\isamarkupfalse%
\ lm{\isadigit{5}}{\isadigit{8}}{\isacharcolon}\ \isakeyword{assumes}\ {\isachardoublequoteopen}distinct\ G{\isachardoublequoteclose}\ {\isachardoublequoteopen}set\ G\ {\isasymnoteq}\ {\isacharbraceleft}{\isacharbraceright}{\isachardoublequoteclose}\ {\isachardoublequoteopen}finite\ N{\isachardoublequoteclose}\ \isakeyword{shows}\isanewline
{\isachardoublequoteopen}the{\isacharunderscore}elem\isanewline
{\isacharparenleft}argmax\ {\isacharparenleft}setsum\ {\isacharparenleft}randomBids{\isacharprime}\ N\ G\ b\ r{\isacharparenright}{\isacharparenright}\ {\isacharparenleft}maximalStrictAllocations{\isacharprime}\ N\ {\isacharparenleft}set\ G{\isacharparenright}\ b{\isacharparenright}{\isacharparenright}\ {\isasymin}\isanewline
{\isacharparenleft}maximalStrictAllocations{\isacharprime}\ N\ {\isacharparenleft}set\ G{\isacharparenright}\ b{\isacharparenright}{\isachardoublequoteclose}\ {\isacharparenleft}\isakeyword{is}\ {\isachardoublequoteopen}the{\isacharunderscore}elem\ {\isacharquery}X\ {\isasymin}\ {\isacharquery}R{\isachardoublequoteclose}{\isacharparenright}%
\isadelimproof
\ %
\endisadelimproof
%
\isatagproof
\isacommand{using}\isamarkupfalse%
\ assms\ lm{\isadigit{9}}{\isadigit{2}}b\ lm{\isadigit{5}}{\isadigit{7}}\isanewline
\isacommand{proof}\isamarkupfalse%
\ {\isacharminus}\isanewline
\isacommand{have}\isamarkupfalse%
\ {\isachardoublequoteopen}card\ {\isacharquery}X{\isacharequal}{\isadigit{1}}{\isachardoublequoteclose}\ \isacommand{using}\isamarkupfalse%
\ assms\ \isacommand{by}\isamarkupfalse%
\ {\isacharparenleft}rule\ lm{\isadigit{9}}{\isadigit{2}}b{\isacharparenright}\ \isacommand{moreover}\isamarkupfalse%
\ \isacommand{have}\isamarkupfalse%
\ {\isachardoublequoteopen}{\isacharquery}X\ {\isasymsubseteq}\ {\isacharquery}R{\isachardoublequoteclose}\ \isacommand{by}\isamarkupfalse%
\ auto\isanewline
\isacommand{ultimately}\isamarkupfalse%
\ \isacommand{show}\isamarkupfalse%
\ {\isacharquery}thesis\ \isacommand{using}\isamarkupfalse%
\ cardinalityOneTheElem\ \isacommand{by}\isamarkupfalse%
\ blast\isanewline
\isacommand{qed}\isamarkupfalse%
%
\endisatagproof
{\isafoldproof}%
%
\isadelimproof
%
\endisadelimproof
\isanewline
\isanewline
\isacommand{corollary}\isamarkupfalse%
\ lm{\isadigit{5}}{\isadigit{8}}b{\isacharcolon}\ \isakeyword{assumes}\ {\isachardoublequoteopen}distinct\ G{\isachardoublequoteclose}\ {\isachardoublequoteopen}set\ G\ {\isasymnoteq}\ {\isacharbraceleft}{\isacharbraceright}{\isachardoublequoteclose}\ {\isachardoublequoteopen}finite\ N{\isachardoublequoteclose}\ \isakeyword{shows}\ \isanewline
{\isachardoublequoteopen}vcga{\isacharprime}\ N\ G\ b\ r\ {\isasymin}\ {\isacharparenleft}Outside{\isacharprime}\ {\isacharbraceleft}seller{\isacharbraceright}{\isacharparenright}{\isacharbackquote}{\isacharparenleft}maximalStrictAllocations{\isacharprime}\ N\ {\isacharparenleft}set\ G{\isacharparenright}\ b{\isacharparenright}{\isachardoublequoteclose}\isanewline
%
\isadelimproof
%
\endisadelimproof
%
\isatagproof
\isacommand{using}\isamarkupfalse%
\ assms\ lm{\isadigit{5}}{\isadigit{8}}\ \isacommand{by}\isamarkupfalse%
\ blast%
\endisatagproof
{\isafoldproof}%
%
\isadelimproof
\isanewline
%
\endisadelimproof
\isanewline
\isacommand{lemma}\isamarkupfalse%
\ lm{\isadigit{6}}{\isadigit{2}}{\isacharcolon}\ {\isachardoublequoteopen}{\isacharparenleft}Outside{\isacharprime}\ {\isacharbraceleft}seller{\isacharbraceright}{\isacharparenright}{\isacharbackquote}{\isacharparenleft}maximalStrictAllocations{\isacharprime}\ N\ G\ b{\isacharparenright}\ {\isasymsubseteq}\ soldAllocations\ N\ G{\isachardoublequoteclose}\isanewline
%
\isadelimproof
%
\endisadelimproof
%
\isatagproof
\isacommand{using}\isamarkupfalse%
\ Outside{\isacharunderscore}def\ \isacommand{by}\isamarkupfalse%
\ force%
\endisatagproof
{\isafoldproof}%
%
\isadelimproof
\isanewline
%
\endisadelimproof
\isanewline
\isacommand{theorem}\isamarkupfalse%
\ lm{\isadigit{5}}{\isadigit{8}}d{\isacharcolon}\ \isakeyword{assumes}\ {\isachardoublequoteopen}distinct\ G{\isachardoublequoteclose}\ {\isachardoublequoteopen}set\ G\ {\isasymnoteq}\ {\isacharbraceleft}{\isacharbraceright}{\isachardoublequoteclose}\ {\isachardoublequoteopen}finite\ N{\isachardoublequoteclose}\ \isakeyword{shows}\ \isanewline
{\isachardoublequoteopen}vcga{\isacharprime}\ N\ G\ b\ r\ {\isasymin}\ soldAllocations\ N\ {\isacharparenleft}set\ G{\isacharparenright}{\isachardoublequoteclose}\ {\isacharparenleft}\isakeyword{is}\ {\isachardoublequoteopen}{\isacharquery}a\ {\isasymin}\ {\isacharquery}A{\isachardoublequoteclose}{\isacharparenright}%
\isadelimproof
\ %
\endisadelimproof
%
\isatagproof
\isacommand{using}\isamarkupfalse%
\ assms\ lm{\isadigit{5}}{\isadigit{8}}b\ lm{\isadigit{6}}{\isadigit{2}}\ \isanewline
\isacommand{proof}\isamarkupfalse%
\ {\isacharminus}\ \isacommand{have}\isamarkupfalse%
\ {\isachardoublequoteopen}{\isacharquery}a\ {\isasymin}\ {\isacharparenleft}Outside{\isacharprime}\ {\isacharbraceleft}seller{\isacharbraceright}{\isacharparenright}{\isacharbackquote}{\isacharparenleft}maximalStrictAllocations{\isacharprime}\ N\ {\isacharparenleft}set\ G{\isacharparenright}\ b{\isacharparenright}{\isachardoublequoteclose}\ \isanewline
\isacommand{using}\isamarkupfalse%
\ assms\ \isacommand{by}\isamarkupfalse%
\ {\isacharparenleft}rule\ lm{\isadigit{5}}{\isadigit{8}}b{\isacharparenright}\ \isacommand{thus}\isamarkupfalse%
\ {\isacharquery}thesis\ \isacommand{using}\isamarkupfalse%
\ lm{\isadigit{6}}{\isadigit{2}}\ \ \isacommand{by}\isamarkupfalse%
\ fastforce\ \isacommand{qed}\isamarkupfalse%
%
\endisatagproof
{\isafoldproof}%
%
\isadelimproof
%
\endisadelimproof
\isanewline
\isacommand{corollary}\isamarkupfalse%
\ lm{\isadigit{5}}{\isadigit{8}}f{\isacharcolon}\ \isakeyword{assumes}\ {\isachardoublequoteopen}distinct\ G{\isachardoublequoteclose}\ {\isachardoublequoteopen}set\ G\ {\isasymnoteq}\ {\isacharbraceleft}{\isacharbraceright}{\isachardoublequoteclose}\ {\isachardoublequoteopen}finite\ N{\isachardoublequoteclose}\ \isakeyword{shows}\ \isanewline
{\isachardoublequoteopen}vcga\ N\ G\ b\ r\ {\isasymin}\ soldAllocations\ N\ {\isacharparenleft}set\ G{\isacharparenright}{\isachardoublequoteclose}\ {\isacharparenleft}\isakeyword{is}\ {\isachardoublequoteopen}{\isacharunderscore}{\isasymin}{\isacharquery}r{\isachardoublequoteclose}{\isacharparenright}\ \isanewline
%
\isadelimproof
%
\endisadelimproof
%
\isatagproof
\isacommand{proof}\isamarkupfalse%
\ {\isacharminus}\ \isacommand{have}\isamarkupfalse%
\ {\isachardoublequoteopen}vcga{\isacharprime}\ N\ G\ b\ r\ {\isasymin}\ {\isacharquery}r{\isachardoublequoteclose}\ \isacommand{using}\isamarkupfalse%
\ assms\ \isacommand{by}\isamarkupfalse%
\ {\isacharparenleft}rule\ lm{\isadigit{5}}{\isadigit{8}}d{\isacharparenright}\ \isacommand{then}\isamarkupfalse%
\ \isacommand{show}\isamarkupfalse%
\ {\isacharquery}thesis\ \isacommand{using}\isamarkupfalse%
\ assms\ lm{\isadigit{0}}{\isadigit{0}}{\isadigit{2}}\ \isacommand{by}\isamarkupfalse%
\ blast\ \isacommand{qed}\isamarkupfalse%
%
\endisatagproof
{\isafoldproof}%
%
\isadelimproof
\isanewline
%
\endisadelimproof
\isanewline
\isacommand{corollary}\isamarkupfalse%
\ lm{\isadigit{5}}{\isadigit{9}}b{\isacharcolon}\ \isakeyword{assumes}\ {\isachardoublequoteopen}{\isasymforall}X{\isachardot}\ X\ {\isasymin}\ Range\ a\ {\isasymlongrightarrow}b\ {\isacharparenleft}seller{\isacharcomma}\ X{\isacharparenright}{\isacharequal}{\isadigit{0}}{\isachardoublequoteclose}\ {\isachardoublequoteopen}finite\ a{\isachardoublequoteclose}\ \isakeyword{shows}\ \isanewline
{\isachardoublequoteopen}setsum\ b\ a\ {\isacharequal}\ setsum\ b\ {\isacharparenleft}a{\isacharminus}{\isacharminus}seller{\isacharparenright}{\isachardoublequoteclose}\isanewline
%
\isadelimproof
%
\endisadelimproof
%
\isatagproof
\isacommand{proof}\isamarkupfalse%
\ {\isacharminus}\isanewline
\isacommand{let}\isamarkupfalse%
\ {\isacharquery}n{\isacharequal}seller\ \isacommand{have}\isamarkupfalse%
\ {\isachardoublequoteopen}finite\ {\isacharparenleft}a{\isacharbar}{\isacharbar}{\isacharbraceleft}{\isacharquery}n{\isacharbraceright}{\isacharparenright}{\isachardoublequoteclose}\ \isacommand{using}\isamarkupfalse%
\ assms\ restrict{\isacharunderscore}def\ \isacommand{by}\isamarkupfalse%
\ {\isacharparenleft}metis\ finite{\isacharunderscore}Int{\isacharparenright}\ \isanewline
\isacommand{moreover}\isamarkupfalse%
\ \isacommand{have}\isamarkupfalse%
\ {\isachardoublequoteopen}{\isasymforall}z\ {\isasymin}\ a{\isacharbar}{\isacharbar}{\isacharbraceleft}{\isacharquery}n{\isacharbraceright}{\isachardot}\ b\ z{\isacharequal}{\isadigit{0}}{\isachardoublequoteclose}\ \isacommand{using}\isamarkupfalse%
\ assms\ restrict{\isacharunderscore}def\ \isacommand{by}\isamarkupfalse%
\ fastforce\isanewline
\isacommand{ultimately}\isamarkupfalse%
\ \isacommand{have}\isamarkupfalse%
\ {\isachardoublequoteopen}setsum\ b\ {\isacharparenleft}a{\isacharbar}{\isacharbar}{\isacharbraceleft}{\isacharquery}n{\isacharbraceright}{\isacharparenright}\ {\isacharequal}\ {\isadigit{0}}{\isachardoublequoteclose}\ \isacommand{using}\isamarkupfalse%
\ assms\ \isacommand{by}\isamarkupfalse%
\ {\isacharparenleft}metis\ setsum{\isachardot}neutral{\isacharparenright}\isanewline
\isacommand{thus}\isamarkupfalse%
\ {\isacharquery}thesis\ \isacommand{using}\isamarkupfalse%
\ setsumOutside\ assms{\isacharparenleft}{\isadigit{2}}{\isacharparenright}\ \isacommand{by}\isamarkupfalse%
\ {\isacharparenleft}metis\ comm{\isacharunderscore}monoid{\isacharunderscore}add{\isacharunderscore}class{\isachardot}add{\isachardot}right{\isacharunderscore}neutral{\isacharparenright}\isanewline
\isacommand{qed}\isamarkupfalse%
%
\endisatagproof
{\isafoldproof}%
%
\isadelimproof
\isanewline
%
\endisadelimproof
\isanewline
\isacommand{corollary}\isamarkupfalse%
\ lm{\isadigit{5}}{\isadigit{9}}c{\isacharcolon}\ \isakeyword{assumes}\ {\isachardoublequoteopen}{\isasymforall}a{\isasymin}A{\isachardot}\ finite\ a\ {\isacharampersand}\ {\isacharparenleft}{\isasymforall}\ X{\isachardot}\ X{\isasymin}Range\ a\ {\isasymlongrightarrow}\ b\ {\isacharparenleft}seller{\isacharcomma}\ X{\isacharparenright}{\isacharequal}{\isadigit{0}}{\isacharparenright}{\isachardoublequoteclose}\isanewline
\isakeyword{shows}\ {\isachardoublequoteopen}{\isacharbraceleft}setsum\ b\ a{\isacharbar}\ a{\isachardot}\ a{\isasymin}A{\isacharbraceright}{\isacharequal}{\isacharbraceleft}setsum\ b\ {\isacharparenleft}a\ {\isacharminus}{\isacharminus}\ seller{\isacharparenright}{\isacharbar}\ a{\isachardot}\ a{\isasymin}A{\isacharbraceright}{\isachardoublequoteclose}%
\isadelimproof
\ %
\endisadelimproof
%
\isatagproof
\isacommand{using}\isamarkupfalse%
\ assms\ lm{\isadigit{5}}{\isadigit{9}}b\ \isanewline
\isacommand{by}\isamarkupfalse%
\ {\isacharparenleft}metis\ {\isacharparenleft}lifting{\isacharcomma}\ no{\isacharunderscore}types{\isacharparenright}{\isacharparenright}%
\endisatagproof
{\isafoldproof}%
%
\isadelimproof
%
\endisadelimproof
\isanewline
\isacommand{corollary}\isamarkupfalse%
\ lm{\isadigit{5}}{\isadigit{8}}c{\isacharcolon}\ \isakeyword{assumes}\ {\isachardoublequoteopen}distinct\ G{\isachardoublequoteclose}\ {\isachardoublequoteopen}set\ G\ {\isasymnoteq}\ {\isacharbraceleft}{\isacharbraceright}{\isachardoublequoteclose}\ {\isachardoublequoteopen}finite\ N{\isachardoublequoteclose}\ \isakeyword{shows}\isanewline
{\isachardoublequoteopen}EX\ a{\isachardot}\ {\isacharparenleft}{\isacharparenleft}a\ {\isasymin}\ {\isacharparenleft}maximalStrictAllocations{\isacharprime}\ N\ {\isacharparenleft}set\ G{\isacharparenright}\ b{\isacharparenright}{\isacharparenright}\ \isanewline
{\isacharampersand}\ {\isacharparenleft}vcga{\isacharprime}\ N\ G\ b\ r\ {\isacharequal}\ a\ {\isacharminus}{\isacharminus}\ seller{\isacharparenright}\ \isanewline
{\isacharampersand}\ {\isacharparenleft}a\ {\isasymin}\ argmax\ {\isacharparenleft}setsum\ b{\isacharparenright}\ {\isacharparenleft}allAllocations\ {\isacharparenleft}{\isacharbraceleft}seller{\isacharbraceright}{\isasymunion}N{\isacharparenright}\ {\isacharparenleft}set\ G{\isacharparenright}{\isacharparenright}{\isacharparenright}\isanewline
{\isacharparenright}{\isachardoublequoteclose}\ {\isacharparenleft}\isakeyword{is}\ {\isachardoublequoteopen}EX\ a{\isachardot}\ {\isacharunderscore}\ {\isacharampersand}\ {\isacharunderscore}\ {\isacharampersand}\ a\ {\isasymin}\ {\isacharquery}X{\isachardoublequoteclose}{\isacharparenright}\isanewline
%
\isadelimproof
%
\endisadelimproof
%
\isatagproof
\isacommand{using}\isamarkupfalse%
\ assms\ lm{\isadigit{5}}{\isadigit{8}}b\ argmax{\isacharunderscore}def\ \isacommand{by}\isamarkupfalse%
\ fast%
\endisatagproof
{\isafoldproof}%
%
\isadelimproof
\isanewline
%
\endisadelimproof
\isanewline
\isacommand{lemma}\isamarkupfalse%
\ \isakeyword{assumes}\ {\isachardoublequoteopen}distinct\ G{\isachardoublequoteclose}\ {\isachardoublequoteopen}set\ G\ {\isasymnoteq}\ {\isacharbraceleft}{\isacharbraceright}{\isachardoublequoteclose}\ {\isachardoublequoteopen}finite\ N{\isachardoublequoteclose}\ \isakeyword{shows}\ \isanewline
{\isachardoublequoteopen}{\isasymforall}aa{\isasymin}allAllocations\ {\isacharparenleft}{\isacharbraceleft}seller{\isacharbraceright}{\isasymunion}N{\isacharparenright}\ {\isacharparenleft}set\ G{\isacharparenright}{\isachardot}\ finite\ aa{\isachardoublequoteclose}\isanewline
%
\isadelimproof
%
\endisadelimproof
%
\isatagproof
\isacommand{using}\isamarkupfalse%
\ assms\ \isacommand{by}\isamarkupfalse%
\ {\isacharparenleft}metis\ List{\isachardot}finite{\isacharunderscore}set\ UniformTieBreaking{\isachardot}lm{\isadigit{4}}{\isadigit{4}}{\isacharparenright}%
\endisatagproof
{\isafoldproof}%
%
\isadelimproof
\isanewline
%
\endisadelimproof
\isanewline
\isacommand{lemma}\isamarkupfalse%
\ lm{\isadigit{6}}{\isadigit{1}}{\isacharcolon}\ \isakeyword{assumes}\ {\isachardoublequoteopen}distinct\ G{\isachardoublequoteclose}\ {\isachardoublequoteopen}set\ G\ {\isasymnoteq}\ {\isacharbraceleft}{\isacharbraceright}{\isachardoublequoteclose}\ {\isachardoublequoteopen}finite\ N{\isachardoublequoteclose}\ \isanewline
{\isachardoublequoteopen}{\isasymforall}aa{\isasymin}allAllocations\ {\isacharparenleft}{\isacharbraceleft}seller{\isacharbraceright}{\isasymunion}N{\isacharparenright}\ {\isacharparenleft}set\ G{\isacharparenright}{\isachardot}\ {\isasymforall}\ X\ {\isasymin}\ Range\ aa{\isachardot}\ b\ {\isacharparenleft}seller{\isacharcomma}\ X{\isacharparenright}{\isacharequal}{\isadigit{0}}{\isachardoublequoteclose}\isanewline
{\isacharparenleft}\isakeyword{is}\ {\isachardoublequoteopen}{\isasymforall}aa{\isasymin}{\isacharquery}X{\isachardot}\ {\isacharunderscore}{\isachardoublequoteclose}{\isacharparenright}\ \isakeyword{shows}\ {\isachardoublequoteopen}setsum\ b\ {\isacharparenleft}vcga{\isacharprime}\ N\ G\ b\ r{\isacharparenright}{\isacharequal}Max{\isacharbraceleft}setsum\ b\ aa{\isacharbar}\ aa{\isachardot}\ aa\ {\isasymin}\ soldAllocations\ N\ {\isacharparenleft}set\ G{\isacharparenright}{\isacharbraceright}{\isachardoublequoteclose}\isanewline
%
\isadelimproof
%
\endisadelimproof
%
\isatagproof
\isacommand{proof}\isamarkupfalse%
\ {\isacharminus}\isanewline
\isacommand{let}\isamarkupfalse%
\ {\isacharquery}n{\isacharequal}seller\ \isacommand{let}\isamarkupfalse%
\ {\isacharquery}s{\isacharequal}setsum\ \isacommand{let}\isamarkupfalse%
\ {\isacharquery}a{\isacharequal}{\isachardoublequoteopen}vcga{\isacharprime}\ N\ G\ b\ r{\isachardoublequoteclose}\ \isacommand{obtain}\isamarkupfalse%
\ a\ \isakeyword{where}\ \isanewline
{\isadigit{0}}{\isacharcolon}\ {\isachardoublequoteopen}a\ {\isasymin}\ maximalStrictAllocations{\isacharprime}\ N\ {\isacharparenleft}set\ G{\isacharparenright}\ b\ {\isacharampersand}\ {\isacharquery}a{\isacharequal}a{\isacharminus}{\isacharminus}{\isacharquery}n\ {\isacharampersand}\ \isanewline
{\isacharparenleft}a{\isasymin}argmax\ {\isacharparenleft}setsum\ b{\isacharparenright}\ {\isacharparenleft}allAllocations{\isacharparenleft}{\isacharbraceleft}seller{\isacharbraceright}{\isasymunion}N{\isacharparenright}{\isacharparenleft}set\ G{\isacharparenright}{\isacharparenright}{\isacharparenright}{\isachardoublequoteclose}{\isacharparenleft}\isakeyword{is}\ {\isachardoublequoteopen}{\isacharunderscore}\ {\isacharampersand}\ {\isacharquery}a{\isacharequal}{\isacharunderscore}\ {\isacharampersand}\ a{\isasymin}{\isacharquery}Z{\isachardoublequoteclose}{\isacharparenright}\isanewline
\isacommand{using}\isamarkupfalse%
\ assms{\isacharparenleft}{\isadigit{1}}{\isacharcomma}{\isadigit{2}}{\isacharcomma}{\isadigit{3}}{\isacharparenright}\ lm{\isadigit{5}}{\isadigit{8}}c\ \isacommand{by}\isamarkupfalse%
\ blast\ \isacommand{have}\isamarkupfalse%
\ \isanewline
{\isadigit{1}}{\isacharcolon}\ {\isachardoublequoteopen}{\isasymforall}aa{\isasymin}{\isacharquery}X{\isachardot}\ finite\ aa\ {\isacharampersand}\ {\isacharparenleft}{\isasymforall}\ X{\isachardot}\ X{\isasymin}Range\ aa\ {\isasymlongrightarrow}\ b\ {\isacharparenleft}{\isacharquery}n{\isacharcomma}\ X{\isacharparenright}{\isacharequal}{\isadigit{0}}{\isacharparenright}{\isachardoublequoteclose}\ \isacommand{using}\isamarkupfalse%
\ assms{\isacharparenleft}{\isadigit{4}}{\isacharparenright}\ \isanewline
List{\isachardot}finite{\isacharunderscore}set\ UniformTieBreaking{\isachardot}lm{\isadigit{4}}{\isadigit{4}}\ \isacommand{by}\isamarkupfalse%
\ metis\ \isacommand{have}\isamarkupfalse%
\ \isanewline
{\isadigit{2}}{\isacharcolon}\ {\isachardoublequoteopen}a\ {\isasymin}\ {\isacharquery}X{\isachardoublequoteclose}\ \isacommand{using}\isamarkupfalse%
\ {\isadigit{0}}\ \isacommand{by}\isamarkupfalse%
\ auto\ \isacommand{have}\isamarkupfalse%
\ {\isachardoublequoteopen}a\ {\isasymin}\ {\isacharquery}Z{\isachardoublequoteclose}\ \isacommand{using}\isamarkupfalse%
\ {\isadigit{0}}\ \isacommand{by}\isamarkupfalse%
\ fast\ \isanewline
\isacommand{then}\isamarkupfalse%
\ \isacommand{have}\isamarkupfalse%
\ {\isachardoublequoteopen}a\ {\isasymin}\ {\isacharquery}X{\isasyminter}{\isacharbraceleft}x{\isachardot}\ {\isacharquery}s\ b\ x\ {\isacharequal}\ Max\ {\isacharparenleft}{\isacharquery}s\ b\ {\isacharbackquote}\ {\isacharquery}X{\isacharparenright}{\isacharbraceright}{\isachardoublequoteclose}\ \isacommand{using}\isamarkupfalse%
\ lm{\isadigit{7}}{\isadigit{8}}\ \isacommand{by}\isamarkupfalse%
\ simp\isanewline
\isacommand{then}\isamarkupfalse%
\ \isacommand{have}\isamarkupfalse%
\ {\isachardoublequoteopen}a\ {\isasymin}\ {\isacharbraceleft}x{\isachardot}\ {\isacharquery}s\ b\ x\ {\isacharequal}\ Max\ {\isacharparenleft}{\isacharquery}s\ b\ {\isacharbackquote}\ {\isacharquery}X{\isacharparenright}{\isacharbraceright}{\isachardoublequoteclose}\ \isacommand{using}\isamarkupfalse%
\ lm{\isadigit{7}}{\isadigit{8}}\ \isacommand{by}\isamarkupfalse%
\ simp\isanewline
\isacommand{moreover}\isamarkupfalse%
\ \isacommand{have}\isamarkupfalse%
\ {\isachardoublequoteopen}{\isacharquery}s\ b\ {\isacharbackquote}\ {\isacharquery}X\ {\isacharequal}\ {\isacharbraceleft}{\isacharquery}s\ b\ aa{\isacharbar}\ aa{\isachardot}\ aa{\isasymin}{\isacharquery}X{\isacharbraceright}{\isachardoublequoteclose}\ \isacommand{by}\isamarkupfalse%
\ blast\isanewline
\isacommand{ultimately}\isamarkupfalse%
\ \isacommand{have}\isamarkupfalse%
\ {\isachardoublequoteopen}{\isacharquery}s\ b\ a\ {\isacharequal}\ Max\ {\isacharbraceleft}{\isacharquery}s\ b\ aa{\isacharbar}\ aa{\isachardot}\ aa{\isasymin}{\isacharquery}X{\isacharbraceright}{\isachardoublequoteclose}\ \isacommand{by}\isamarkupfalse%
\ auto\isanewline
\isacommand{moreover}\isamarkupfalse%
\ \isacommand{have}\isamarkupfalse%
\ {\isachardoublequoteopen}{\isacharbraceleft}{\isacharquery}s\ b\ aa{\isacharbar}\ aa{\isachardot}\ aa{\isasymin}{\isacharquery}X{\isacharbraceright}\ {\isacharequal}\ {\isacharbraceleft}{\isacharquery}s\ b\ {\isacharparenleft}aa{\isacharminus}{\isacharminus}{\isacharquery}n{\isacharparenright}{\isacharbar}\ aa{\isachardot}\ aa{\isasymin}{\isacharquery}X{\isacharbraceright}{\isachardoublequoteclose}\ \isacommand{using}\isamarkupfalse%
\ {\isadigit{1}}\ \isacommand{by}\isamarkupfalse%
\ {\isacharparenleft}rule\ lm{\isadigit{5}}{\isadigit{9}}c{\isacharparenright}\isanewline
\isacommand{moreover}\isamarkupfalse%
\ \isacommand{have}\isamarkupfalse%
\ {\isachardoublequoteopen}{\isachardot}{\isachardot}{\isachardot}\ {\isacharequal}\ {\isacharbraceleft}{\isacharquery}s\ b\ aa{\isacharbar}\ aa{\isachardot}\ aa\ {\isasymin}\ Outside{\isacharprime}\ {\isacharbraceleft}{\isacharquery}n{\isacharbraceright}{\isacharbackquote}{\isacharquery}X{\isacharbraceright}{\isachardoublequoteclose}\ \isacommand{by}\isamarkupfalse%
\ blast\isanewline
\isacommand{moreover}\isamarkupfalse%
\ \isacommand{have}\isamarkupfalse%
\ {\isachardoublequoteopen}{\isachardot}{\isachardot}{\isachardot}\ {\isacharequal}\ {\isacharbraceleft}{\isacharquery}s\ b\ aa{\isacharbar}\ aa{\isachardot}\ aa\ {\isasymin}\ soldAllocations\ N\ {\isacharparenleft}set\ G{\isacharparenright}{\isacharbraceright}{\isachardoublequoteclose}\ \isacommand{by}\isamarkupfalse%
\ simp\isanewline
\isacommand{ultimately}\isamarkupfalse%
\ \isacommand{have}\isamarkupfalse%
\ {\isachardoublequoteopen}Max\ {\isacharbraceleft}{\isacharquery}s\ b\ aa{\isacharbar}\ aa{\isachardot}\ aa\ {\isasymin}\ soldAllocations\ N\ {\isacharparenleft}set\ G{\isacharparenright}{\isacharbraceright}\ {\isacharequal}\ {\isacharquery}s\ b\ a{\isachardoublequoteclose}\ \isacommand{by}\isamarkupfalse%
\ presburger\isanewline
\isacommand{moreover}\isamarkupfalse%
\ \isacommand{have}\isamarkupfalse%
\ {\isachardoublequoteopen}{\isachardot}{\isachardot}{\isachardot}\ {\isacharequal}\ {\isacharquery}s\ b\ {\isacharparenleft}a{\isacharminus}{\isacharminus}{\isacharquery}n{\isacharparenright}{\isachardoublequoteclose}\ \isacommand{using}\isamarkupfalse%
\ {\isadigit{1}}\ {\isadigit{2}}\ lm{\isadigit{5}}{\isadigit{9}}b\ \isacommand{by}\isamarkupfalse%
\ {\isacharparenleft}metis\ {\isacharparenleft}lifting{\isacharcomma}\ no{\isacharunderscore}types{\isacharparenright}{\isacharparenright}\isanewline
\isacommand{ultimately}\isamarkupfalse%
\ \isacommand{show}\isamarkupfalse%
\ {\isachardoublequoteopen}{\isacharquery}s\ b\ {\isacharquery}a{\isacharequal}Max{\isacharbraceleft}{\isacharquery}s\ b\ aa{\isacharbar}\ aa{\isachardot}\ aa\ {\isasymin}\ soldAllocations\ N\ {\isacharparenleft}set\ G{\isacharparenright}{\isacharbraceright}{\isachardoublequoteclose}\ \isacommand{using}\isamarkupfalse%
\ {\isadigit{0}}\ \isacommand{by}\isamarkupfalse%
\ presburger\isanewline
\isacommand{qed}\isamarkupfalse%
%
\endisatagproof
{\isafoldproof}%
%
\isadelimproof
%
\endisadelimproof
%
\begin{isamarkuptext}%
Adequacy theorem: the allocation satisfies the standard pen-and-paper specification of a VCG auction.
See, for example, \cite[\S~1.2]{cramton}.%
\end{isamarkuptext}%
\isamarkuptrue%
\isacommand{theorem}\isamarkupfalse%
\ lm{\isadigit{6}}{\isadigit{1}}b{\isacharcolon}\ \isakeyword{assumes}\ {\isachardoublequoteopen}distinct\ G{\isachardoublequoteclose}\ {\isachardoublequoteopen}set\ G\ {\isasymnoteq}\ {\isacharbraceleft}{\isacharbraceright}{\isachardoublequoteclose}\ {\isachardoublequoteopen}finite\ N{\isachardoublequoteclose}\ {\isachardoublequoteopen}{\isasymforall}\ X{\isachardot}\ b\ {\isacharparenleft}seller{\isacharcomma}\ X{\isacharparenright}{\isacharequal}{\isadigit{0}}{\isachardoublequoteclose}\ \isanewline
\isakeyword{shows}\ {\isachardoublequoteopen}setsum\ b\ {\isacharparenleft}vcga{\isacharprime}\ N\ G\ b\ r{\isacharparenright}{\isacharequal}Max{\isacharbraceleft}setsum\ b\ aa{\isacharbar}\ aa{\isachardot}\ aa\ {\isasymin}\ soldAllocations\ N\ {\isacharparenleft}set\ G{\isacharparenright}{\isacharbraceright}{\isachardoublequoteclose}\isanewline
%
\isadelimproof
%
\endisadelimproof
%
\isatagproof
\isacommand{using}\isamarkupfalse%
\ assms\ lm{\isadigit{6}}{\isadigit{1}}\ \isacommand{by}\isamarkupfalse%
\ blast%
\endisatagproof
{\isafoldproof}%
%
\isadelimproof
\isanewline
%
\endisadelimproof
\isanewline
\isacommand{corollary}\isamarkupfalse%
\ lm{\isadigit{5}}{\isadigit{8}}e{\isacharcolon}\ \isakeyword{assumes}\ {\isachardoublequoteopen}distinct\ G{\isachardoublequoteclose}\ {\isachardoublequoteopen}set\ G\ {\isasymnoteq}\ {\isacharbraceleft}{\isacharbraceright}{\isachardoublequoteclose}\ {\isachardoublequoteopen}finite\ N{\isachardoublequoteclose}\ \isakeyword{shows}\isanewline
{\isachardoublequoteopen}vcga{\isacharprime}\ N\ G\ b\ r\ {\isasymin}\ allocationsUniverse\ {\isacharampersand}\ {\isasymUnion}\ Range\ {\isacharparenleft}vcga{\isacharprime}\ N\ G\ b\ r{\isacharparenright}\ {\isasymsubseteq}\ set\ G{\isachardoublequoteclose}%
\isadelimproof
\ %
\endisadelimproof
%
\isatagproof
\isacommand{using}\isamarkupfalse%
\ assms\ lm{\isadigit{5}}{\isadigit{8}}b\ \isanewline
\isacommand{proof}\isamarkupfalse%
\ {\isacharminus}\isanewline
\isacommand{let}\isamarkupfalse%
\ {\isacharquery}a{\isacharequal}{\isachardoublequoteopen}vcga{\isacharprime}\ N\ G\ b\ r{\isachardoublequoteclose}\ \isacommand{let}\isamarkupfalse%
\ {\isacharquery}n{\isacharequal}seller\isanewline
\isacommand{obtain}\isamarkupfalse%
\ a\ \isakeyword{where}\ \isanewline
{\isadigit{0}}{\isacharcolon}\ {\isachardoublequoteopen}{\isacharquery}a{\isacharequal}a\ {\isacharminus}{\isacharminus}\ seller\ {\isacharampersand}\ a\ {\isasymin}\ maximalStrictAllocations{\isacharprime}\ N\ {\isacharparenleft}set\ G{\isacharparenright}\ b{\isachardoublequoteclose}\isanewline
\isacommand{using}\isamarkupfalse%
\ assms\ lm{\isadigit{5}}{\isadigit{8}}b\ \isacommand{by}\isamarkupfalse%
\ blast\isanewline
\isacommand{then}\isamarkupfalse%
\ \isacommand{moreover}\isamarkupfalse%
\ \isacommand{have}\isamarkupfalse%
\ \isanewline
{\isadigit{1}}{\isacharcolon}\ {\isachardoublequoteopen}a\ {\isasymin}\ allAllocations\ {\isacharparenleft}{\isacharbraceleft}{\isacharquery}n{\isacharbraceright}{\isasymunion}N{\isacharparenright}\ {\isacharparenleft}set\ G{\isacharparenright}{\isachardoublequoteclose}\ \isacommand{by}\isamarkupfalse%
\ auto\isanewline
\isacommand{moreover}\isamarkupfalse%
\ \isacommand{have}\isamarkupfalse%
\ {\isachardoublequoteopen}maximalStrictAllocations{\isacharprime}\ N\ {\isacharparenleft}set\ G{\isacharparenright}\ b\ {\isasymsubseteq}\ allocationsUniverse{\isachardoublequoteclose}\ \isanewline
\isacommand{by}\isamarkupfalse%
\ {\isacharparenleft}metis\ {\isacharparenleft}lifting{\isacharcomma}\ mono{\isacharunderscore}tags{\isacharparenright}\ UniformTieBreaking{\isachardot}lm{\isadigit{0}}{\isadigit{3}}\ Universes{\isachardot}lm{\isadigit{5}}{\isadigit{0}}\ subset{\isacharunderscore}trans{\isacharparenright}\isanewline
\isacommand{ultimately}\isamarkupfalse%
\ \isacommand{moreover}\isamarkupfalse%
\ \isacommand{have}\isamarkupfalse%
\ {\isachardoublequoteopen}{\isacharquery}a{\isacharequal}a\ {\isacharminus}{\isacharminus}\ seller\ {\isacharampersand}\ a\ {\isasymin}\ allocationsUniverse{\isachardoublequoteclose}\ \isacommand{by}\isamarkupfalse%
\ blast\isanewline
\isacommand{then}\isamarkupfalse%
\ \isacommand{have}\isamarkupfalse%
\ {\isachardoublequoteopen}{\isacharquery}a\ {\isasymin}\ allocationsUniverse{\isachardoublequoteclose}\ \isacommand{using}\isamarkupfalse%
\ lm{\isadigit{3}}{\isadigit{5}}d\ \isacommand{by}\isamarkupfalse%
\ auto\isanewline
\isacommand{moreover}\isamarkupfalse%
\ \isacommand{have}\isamarkupfalse%
\ {\isachardoublequoteopen}{\isasymUnion}\ Range\ a{\isacharequal}\ set\ G{\isachardoublequoteclose}\ \isacommand{using}\isamarkupfalse%
\ nn{\isadigit{2}}{\isadigit{4}}c\ {\isadigit{1}}\ \isacommand{by}\isamarkupfalse%
\ metis\isanewline
\isacommand{then}\isamarkupfalse%
\ \isacommand{moreover}\isamarkupfalse%
\ \isacommand{have}\isamarkupfalse%
\ {\isachardoublequoteopen}{\isasymUnion}\ Range\ {\isacharquery}a\ {\isasymsubseteq}\ set\ G{\isachardoublequoteclose}\ \isacommand{using}\isamarkupfalse%
\ Outside{\isacharunderscore}def\ {\isadigit{0}}\ \isacommand{by}\isamarkupfalse%
\ fast\isanewline
\isacommand{ultimately}\isamarkupfalse%
\ \isacommand{show}\isamarkupfalse%
\ {\isacharquery}thesis\ \isacommand{using}\isamarkupfalse%
\ lm{\isadigit{3}}{\isadigit{5}}d\ Outside{\isacharunderscore}def\ \isacommand{by}\isamarkupfalse%
\ blast\isanewline
\isacommand{qed}\isamarkupfalse%
%
\endisatagproof
{\isafoldproof}%
%
\isadelimproof
%
\endisadelimproof
\isanewline
\isanewline
\isacommand{lemma}\isamarkupfalse%
\ {\isachardoublequoteopen}vcga{\isacharprime}\ N\ G\ b\ r\ {\isacharequal}\ the{\isacharunderscore}elem\ {\isacharparenleft}{\isacharparenleft}argmax\ {\isasymcirc}\ setsum{\isacharparenright}\ {\isacharparenleft}randomBids{\isacharprime}\ N\ G\ b\ r{\isacharparenright}\ \isanewline
{\isacharparenleft}{\isacharparenleft}argmax\ {\isasymcirc}\ setsum{\isacharparenright}\ b\ {\isacharparenleft}allAllocations\ {\isacharparenleft}{\isacharbraceleft}seller{\isacharbraceright}{\isasymunion}N{\isacharparenright}\ {\isacharparenleft}set\ G{\isacharparenright}{\isacharparenright}{\isacharparenright}{\isacharparenright}\ {\isacharminus}{\isacharminus}\ seller{\isachardoublequoteclose}%
\isadelimproof
\ %
\endisadelimproof
%
\isatagproof
\isacommand{by}\isamarkupfalse%
\ simp%
\endisatagproof
{\isafoldproof}%
%
\isadelimproof
%
\endisadelimproof
\isanewline
\isanewline
\isacommand{abbreviation}\isamarkupfalse%
\ {\isachardoublequoteopen}vcgp\ N\ G\ b\ r\ n\ {\isacharequal}{\isacharequal}\isanewline
Max\ {\isacharparenleft}setsum\ b\ {\isacharbackquote}\ {\isacharparenleft}soldAllocations\ {\isacharparenleft}N{\isacharminus}{\isacharbraceleft}n{\isacharbraceright}{\isacharparenright}\ {\isacharparenleft}set\ G{\isacharparenright}{\isacharparenright}{\isacharparenright}\ {\isacharminus}\ {\isacharparenleft}setsum\ b\ {\isacharparenleft}vcga\ N\ G\ b\ r\ {\isacharminus}{\isacharminus}\ n{\isacharparenright}{\isacharparenright}{\isachardoublequoteclose}\isanewline
\isanewline
\isacommand{lemma}\isamarkupfalse%
\ lm{\isadigit{6}}{\isadigit{3}}{\isacharcolon}\ \isakeyword{assumes}\ {\isachardoublequoteopen}x\ {\isasymin}\ X{\isachardoublequoteclose}\ {\isachardoublequoteopen}finite\ X{\isachardoublequoteclose}\ \isakeyword{shows}\ {\isachardoublequoteopen}Max\ {\isacharparenleft}f{\isacharbackquote}X{\isacharparenright}\ {\isachargreater}{\isacharequal}\ f\ x{\isachardoublequoteclose}\ {\isacharparenleft}\isakeyword{is}\ {\isachardoublequoteopen}{\isacharquery}L\ {\isachargreater}{\isacharequal}\ {\isacharquery}R{\isachardoublequoteclose}{\isacharparenright}%
\isadelimproof
\ %
\endisadelimproof
%
\isatagproof
\isacommand{using}\isamarkupfalse%
\ assms\ \isanewline
\isacommand{by}\isamarkupfalse%
\ {\isacharparenleft}metis\ {\isacharparenleft}hide{\isacharunderscore}lams{\isacharcomma}\ no{\isacharunderscore}types{\isacharparenright}\ Max{\isachardot}coboundedI\ finite{\isacharunderscore}imageI\ image{\isacharunderscore}eqI{\isacharparenright}%
\endisatagproof
{\isafoldproof}%
%
\isadelimproof
%
\endisadelimproof
\isanewline
\isanewline
\isacommand{lemma}\isamarkupfalse%
\ lm{\isadigit{5}}{\isadigit{9}}{\isacharcolon}\ \isakeyword{assumes}\ {\isachardoublequoteopen}finite\ N{\isachardoublequoteclose}\ {\isachardoublequoteopen}finite\ G{\isachardoublequoteclose}\ \isakeyword{shows}\ {\isachardoublequoteopen}finite\ {\isacharparenleft}soldAllocations\ N\ G{\isacharparenright}{\isachardoublequoteclose}%
\isadelimproof
\ %
\endisadelimproof
%
\isatagproof
\isacommand{using}\isamarkupfalse%
\ assms\ lm{\isadigit{5}}{\isadigit{9}}\isanewline
finite{\isachardot}emptyI\ finite{\isachardot}insertI\ finite{\isacharunderscore}UnI\ finite{\isacharunderscore}imageI\ \isacommand{by}\isamarkupfalse%
\ metis%
\endisatagproof
{\isafoldproof}%
%
\isadelimproof
%
\endisadelimproof
%
\begin{isamarkuptext}%
The price paid by any participant is non-negative.%
\end{isamarkuptext}%
\isamarkuptrue%
\isacommand{theorem}\isamarkupfalse%
\ NonnegPrices{\isacharcolon}\ \isakeyword{assumes}\ {\isachardoublequoteopen}distinct\ G{\isachardoublequoteclose}\ {\isachardoublequoteopen}set\ G\ {\isasymnoteq}\ {\isacharbraceleft}{\isacharbraceright}{\isachardoublequoteclose}\ {\isachardoublequoteopen}finite\ N{\isachardoublequoteclose}\ \isakeyword{shows}\ \isanewline
{\isachardoublequoteopen}vcgp\ N\ G\ b\ r\ n\ {\isachargreater}{\isacharequal}\ {\isacharparenleft}{\isadigit{0}}{\isacharcolon}{\isacharcolon}price{\isacharparenright}{\isachardoublequoteclose}\ \isanewline
%
\isadelimproof
%
\endisadelimproof
%
\isatagproof
\isacommand{proof}\isamarkupfalse%
\ {\isacharminus}\ \isanewline
\isacommand{let}\isamarkupfalse%
\ {\isacharquery}a{\isacharequal}{\isachardoublequoteopen}vcga\ N\ G\ b\ r{\isachardoublequoteclose}\ \isacommand{let}\isamarkupfalse%
\ {\isacharquery}A{\isacharequal}soldAllocations\ \isacommand{let}\isamarkupfalse%
\ {\isacharquery}f{\isacharequal}{\isachardoublequoteopen}setsum\ b{\isachardoublequoteclose}\ \isanewline
\isacommand{have}\isamarkupfalse%
\ {\isachardoublequoteopen}{\isacharquery}a\ {\isasymin}\ {\isacharquery}A\ N\ {\isacharparenleft}set\ G{\isacharparenright}{\isachardoublequoteclose}\ \isacommand{using}\isamarkupfalse%
\ assms\ \isacommand{by}\isamarkupfalse%
\ {\isacharparenleft}rule\ lm{\isadigit{5}}{\isadigit{8}}f{\isacharparenright}\isanewline
\isacommand{then}\isamarkupfalse%
\ \isacommand{have}\isamarkupfalse%
\ {\isachardoublequoteopen}{\isacharquery}a\ {\isacharminus}{\isacharminus}\ n\ {\isasymin}\ {\isacharquery}A\ {\isacharparenleft}N{\isacharminus}{\isacharbraceleft}n{\isacharbraceright}{\isacharparenright}\ {\isacharparenleft}set\ G{\isacharparenright}{\isachardoublequoteclose}\ \isacommand{by}\isamarkupfalse%
\ {\isacharparenleft}rule\ lm{\isadigit{4}}{\isadigit{4}}b{\isacharparenright}\isanewline
\isacommand{moreover}\isamarkupfalse%
\ \isacommand{have}\isamarkupfalse%
\ {\isachardoublequoteopen}finite\ {\isacharparenleft}{\isacharquery}A\ {\isacharparenleft}N{\isacharminus}{\isacharbraceleft}n{\isacharbraceright}{\isacharparenright}\ {\isacharparenleft}set\ G{\isacharparenright}{\isacharparenright}{\isachardoublequoteclose}\ \isacommand{using}\isamarkupfalse%
\ assms{\isacharparenleft}{\isadigit{3}}{\isacharparenright}\ lm{\isadigit{5}}{\isadigit{9}}\ finite{\isacharunderscore}set\ finite{\isacharunderscore}Diff\ \isacommand{by}\isamarkupfalse%
\ blast\isanewline
\isacommand{ultimately}\isamarkupfalse%
\ \isacommand{have}\isamarkupfalse%
\ {\isachardoublequoteopen}Max\ {\isacharparenleft}{\isacharquery}f{\isacharbackquote}{\isacharparenleft}{\isacharquery}A\ {\isacharparenleft}N{\isacharminus}{\isacharbraceleft}n{\isacharbraceright}{\isacharparenright}\ {\isacharparenleft}set\ G{\isacharparenright}{\isacharparenright}{\isacharparenright}\ {\isasymge}\ {\isacharquery}f\ {\isacharparenleft}{\isacharquery}a\ {\isacharminus}{\isacharminus}\ n{\isacharparenright}{\isachardoublequoteclose}\ {\isacharparenleft}\isakeyword{is}\ {\isachardoublequoteopen}{\isacharquery}L\ {\isachargreater}{\isacharequal}\ {\isacharquery}R{\isachardoublequoteclose}{\isacharparenright}\ \isacommand{by}\isamarkupfalse%
\ {\isacharparenleft}rule\ lm{\isadigit{6}}{\isadigit{3}}{\isacharparenright}\isanewline
\isacommand{then}\isamarkupfalse%
\ \isacommand{show}\isamarkupfalse%
\ {\isachardoublequoteopen}{\isacharquery}L\ {\isacharminus}\ {\isacharquery}R\ {\isachargreater}{\isacharequal}{\isadigit{0}}{\isachardoublequoteclose}\ \isacommand{by}\isamarkupfalse%
\ linarith\isanewline
\isacommand{qed}\isamarkupfalse%
%
\endisatagproof
{\isafoldproof}%
%
\isadelimproof
\isanewline
%
\endisadelimproof
\isanewline
\isacommand{lemma}\isamarkupfalse%
\ lm{\isadigit{1}}{\isadigit{9}}b{\isacharcolon}\ {\isachardoublequoteopen}allAllocations\ N\ G\ {\isacharequal}\ possibleAllocationsRel\ N\ G{\isachardoublequoteclose}%
\isadelimproof
\ %
\endisadelimproof
%
\isatagproof
\isacommand{using}\isamarkupfalse%
\ assms\ \isacommand{by}\isamarkupfalse%
\ {\isacharparenleft}metis\ lm{\isadigit{1}}{\isadigit{9}}{\isacharparenright}%
\endisatagproof
{\isafoldproof}%
%
\isadelimproof
%
\endisadelimproof
\isanewline
\isacommand{abbreviation}\isamarkupfalse%
\ {\isachardoublequoteopen}strictAllocationsUniverse\ {\isacharequal}{\isacharequal}\ allocationsUniverse{\isachardoublequoteclose}\isanewline
\isanewline
\isacommand{abbreviation}\isamarkupfalse%
\ {\isachardoublequoteopen}Goods\ bids\ {\isacharequal}{\isacharequal}\ {\isasymUnion}{\isacharparenleft}{\isacharparenleft}snd{\isasymcirc}fst{\isacharparenright}{\isacharbackquote}bids{\isacharparenright}{\isachardoublequoteclose}\isanewline
\isanewline
\isacommand{corollary}\isamarkupfalse%
\ lm{\isadigit{4}}{\isadigit{5}}{\isacharcolon}\ \isakeyword{assumes}\ {\isachardoublequoteopen}a\ {\isasymin}\ soldAllocations{\isacharprime}{\isacharprime}{\isacharprime}\ N\ G{\isachardoublequoteclose}\ \isakeyword{shows}\ {\isachardoublequoteopen}Range\ a\ {\isasymin}\ partitionsUniverse{\isachardoublequoteclose}\ \isanewline
%
\isadelimproof
%
\endisadelimproof
%
\isatagproof
\isacommand{using}\isamarkupfalse%
\ assms\ \isacommand{by}\isamarkupfalse%
\ {\isacharparenleft}metis\ {\isacharparenleft}lifting{\isacharcomma}\ mono{\isacharunderscore}tags{\isacharparenright}\ Int{\isacharunderscore}iff\ lm{\isadigit{2}}{\isadigit{2}}\ mem{\isacharunderscore}Collect{\isacharunderscore}eq{\isacharparenright}%
\endisatagproof
{\isafoldproof}%
%
\isadelimproof
\isanewline
%
\endisadelimproof
\isanewline
\isacommand{corollary}\isamarkupfalse%
\ lm{\isadigit{4}}{\isadigit{5}}a{\isacharcolon}\ \isakeyword{assumes}\ {\isachardoublequoteopen}a\ {\isasymin}\ soldAllocations\ N\ G{\isachardoublequoteclose}\ \isakeyword{shows}\ {\isachardoublequoteopen}Range\ a\ {\isasymin}\ partitionsUniverse{\isachardoublequoteclose}\isanewline
%
\isadelimproof
%
\endisadelimproof
%
\isatagproof
\isacommand{proof}\isamarkupfalse%
\ {\isacharminus}\ \isacommand{have}\isamarkupfalse%
\ {\isachardoublequoteopen}a\ {\isasymin}\ soldAllocations{\isacharprime}{\isacharprime}{\isacharprime}\ N\ G{\isachardoublequoteclose}\ \isacommand{using}\isamarkupfalse%
\ assms\ lm{\isadigit{2}}{\isadigit{8}}e\ \isacommand{by}\isamarkupfalse%
\ metis\ \isacommand{thus}\isamarkupfalse%
\ {\isacharquery}thesis\ \isacommand{by}\isamarkupfalse%
\ {\isacharparenleft}rule\ lm{\isadigit{4}}{\isadigit{5}}{\isacharparenright}\ \isacommand{qed}\isamarkupfalse%
%
\endisatagproof
{\isafoldproof}%
%
\isadelimproof
\isanewline
%
\endisadelimproof
\isanewline
\isacommand{corollary}\isamarkupfalse%
\ \isakeyword{assumes}\ \isanewline
{\isachardoublequoteopen}N\ {\isasymnoteq}\ {\isacharbraceleft}{\isacharbraceright}{\isachardoublequoteclose}\ {\isachardoublequoteopen}distinct\ G{\isachardoublequoteclose}\ {\isachardoublequoteopen}set\ G\ {\isasymnoteq}\ {\isacharbraceleft}{\isacharbraceright}{\isachardoublequoteclose}\ {\isachardoublequoteopen}finite\ N{\isachardoublequoteclose}\ \ \isanewline
\isakeyword{shows}\ {\isachardoublequoteopen}{\isacharparenleft}Outside{\isacharprime}\ {\isacharbraceleft}seller{\isacharbraceright}{\isacharparenright}\ {\isacharbackquote}\ {\isacharparenleft}terminatingAuctionRel\ N\ G\ {\isacharparenleft}bids{\isacharparenright}\ random{\isacharparenright}\ {\isacharequal}\ \isanewline
{\isacharbraceleft}chosenAllocation{\isacharprime}\ N\ G\ bids\ random\ {\isacharminus}{\isacharminus}\ {\isacharparenleft}seller{\isacharparenright}{\isacharbraceright}{\isachardoublequoteclose}\ {\isacharparenleft}\isakeyword{is}\ {\isachardoublequoteopen}{\isacharquery}L{\isacharequal}{\isacharquery}R{\isachardoublequoteclose}{\isacharparenright}%
\isadelimproof
\ %
\endisadelimproof
%
\isatagproof
\isacommand{using}\isamarkupfalse%
\ assms\ lm{\isadigit{9}}{\isadigit{2}}\ Outside{\isacharunderscore}def\ \isanewline
\isacommand{proof}\isamarkupfalse%
\ {\isacharminus}\isanewline
\isacommand{have}\isamarkupfalse%
\ {\isachardoublequoteopen}{\isacharquery}R\ {\isacharequal}\ Outside{\isacharprime}\ {\isacharbraceleft}seller{\isacharbraceright}\ {\isacharbackquote}\ {\isacharbraceleft}chosenAllocation{\isacharprime}\ N\ G\ bids\ random{\isacharbraceright}{\isachardoublequoteclose}\ \isacommand{using}\isamarkupfalse%
\ Outside{\isacharunderscore}def\ \isanewline
\isacommand{by}\isamarkupfalse%
\ blast\ \isanewline
\isacommand{moreover}\isamarkupfalse%
\ \isacommand{have}\isamarkupfalse%
\ {\isachardoublequoteopen}{\isachardot}{\isachardot}{\isachardot}\ {\isacharequal}\ {\isacharparenleft}Outside{\isacharprime}{\isacharbraceleft}seller{\isacharbraceright}{\isacharparenright}{\isacharbackquote}{\isacharparenleft}terminatingAuctionRel\ N\ G\ bids\ random{\isacharparenright}{\isachardoublequoteclose}\ \isacommand{using}\isamarkupfalse%
\ assms\ lm{\isadigit{9}}{\isadigit{2}}\ \isanewline
\isacommand{by}\isamarkupfalse%
\ blast\isanewline
\isacommand{ultimately}\isamarkupfalse%
\ \isacommand{show}\isamarkupfalse%
\ {\isacharquery}thesis\ \isacommand{by}\isamarkupfalse%
\ presburger\isanewline
\isacommand{qed}\isamarkupfalse%
%
\endisatagproof
{\isafoldproof}%
%
\isadelimproof
%
\endisadelimproof
\isanewline
\isanewline
\isanewline
\isacommand{lemma}\isamarkupfalse%
\ {\isachardoublequoteopen}terminatingAuctionRel\ N\ G\ b\ r\ {\isacharequal}\ \isanewline
{\isacharparenleft}{\isacharparenleft}argmax\ {\isacharparenleft}setsum\ {\isacharparenleft}resolvingBid{\isacharprime}\ N\ G\ b\ {\isacharparenleft}r{\isacharparenright}{\isacharparenright}{\isacharparenright}{\isacharparenright}\ {\isasymcirc}\ {\isacharparenleft}argmax\ {\isacharparenleft}setsum\ b{\isacharparenright}{\isacharparenright}{\isacharparenright}\isanewline
{\isacharparenleft}possibleAllocationsRel\ N\ {\isacharparenleft}set\ G{\isacharparenright}{\isacharparenright}{\isachardoublequoteclose}%
\isadelimproof
\ %
\endisadelimproof
%
\isatagproof
\isacommand{by}\isamarkupfalse%
\ force%
\endisatagproof
{\isafoldproof}%
%
\isadelimproof
%
\endisadelimproof
\isanewline
\isacommand{term}\isamarkupfalse%
\ {\isachardoublequoteopen}{\isacharparenleft}Union\ {\isasymcirc}\ {\isacharparenleft}argmax\ {\isacharparenleft}setsum\ {\isacharparenleft}resolvingBid{\isacharprime}\ N\ G\ b\ {\isacharparenleft}r{\isacharparenright}{\isacharparenright}{\isacharparenright}{\isacharparenright}\ {\isasymcirc}\ {\isacharparenleft}argmax\ {\isacharparenleft}setsum\ b{\isacharparenright}{\isacharparenright}{\isacharparenright}\isanewline
{\isacharparenleft}possibleAllocationsRel\ N\ {\isacharparenleft}set\ G{\isacharparenright}{\isacharparenright}{\isachardoublequoteclose}\isanewline
\isanewline
\isacommand{lemma}\isamarkupfalse%
\ {\isachardoublequoteopen}maximalStrictAllocations{\isacharprime}\ N\ G\ b{\isacharequal}winningAllocationsRel\ {\isacharparenleft}{\isacharbraceleft}seller{\isacharbraceright}\ {\isasymunion}\ N{\isacharparenright}\ G\ b{\isachardoublequoteclose}%
\isadelimproof
\ %
\endisadelimproof
%
\isatagproof
\isacommand{by}\isamarkupfalse%
\ fast%
\endisatagproof
{\isafoldproof}%
%
\isadelimproof
%
\endisadelimproof
\isanewline
\isanewline
\isacommand{lemma}\isamarkupfalse%
\ lm{\isadigit{6}}{\isadigit{4}}{\isacharcolon}\ \isakeyword{assumes}\ {\isachardoublequoteopen}a\ {\isasymin}\ allocationsUniverse{\isachardoublequoteclose}\ \isanewline
{\isachardoublequoteopen}n{\isadigit{1}}\ {\isasymin}\ Domain\ a{\isachardoublequoteclose}\ {\isachardoublequoteopen}n{\isadigit{2}}\ {\isasymin}\ Domain\ a{\isachardoublequoteclose}\isanewline
{\isachardoublequoteopen}n{\isadigit{1}}\ {\isasymnoteq}\ n{\isadigit{2}}{\isachardoublequoteclose}\ \isanewline
\isakeyword{shows}\ {\isachardoublequoteopen}a{\isacharcomma}{\isacharcomma}n{\isadigit{1}}\ {\isasyminter}\ a{\isacharcomma}{\isacharcomma}n{\isadigit{2}}{\isacharequal}{\isacharbraceleft}{\isacharbraceright}{\isachardoublequoteclose}%
\isadelimproof
\ %
\endisadelimproof
%
\isatagproof
\isacommand{using}\isamarkupfalse%
\ assms\ is{\isacharunderscore}non{\isacharunderscore}overlapping{\isacharunderscore}def\ lm{\isadigit{2}}{\isadigit{2}}\ mem{\isacharunderscore}Collect{\isacharunderscore}eq\ \isanewline
\isacommand{proof}\isamarkupfalse%
\ {\isacharminus}\ \isacommand{have}\isamarkupfalse%
\ {\isachardoublequoteopen}Range\ a\ {\isasymin}\ partitionsUniverse{\isachardoublequoteclose}\ \isacommand{using}\isamarkupfalse%
\ assms\ lm{\isadigit{2}}{\isadigit{2}}\ \isacommand{by}\isamarkupfalse%
\ blast\isanewline
\isacommand{moreover}\isamarkupfalse%
\ \isacommand{have}\isamarkupfalse%
\ {\isachardoublequoteopen}a\ {\isasymin}\ injectionsUniverse\ {\isacharampersand}\ a\ {\isasymin}\ partitionValuedUniverse{\isachardoublequoteclose}\ \isacommand{using}\isamarkupfalse%
\ assms\ \isacommand{by}\isamarkupfalse%
\ {\isacharparenleft}metis\ {\isacharparenleft}lifting{\isacharcomma}\ no{\isacharunderscore}types{\isacharparenright}\ IntD{\isadigit{1}}\ IntD{\isadigit{2}}{\isacharparenright}\isanewline
\isacommand{ultimately}\isamarkupfalse%
\ \isacommand{moreover}\isamarkupfalse%
\ \isacommand{have}\isamarkupfalse%
\ {\isachardoublequoteopen}a{\isacharcomma}{\isacharcomma}n{\isadigit{1}}\ {\isasymin}\ Range\ a{\isachardoublequoteclose}\ \isacommand{using}\isamarkupfalse%
\ assms\ \isanewline
\isacommand{by}\isamarkupfalse%
\ {\isacharparenleft}metis\ {\isacharparenleft}mono{\isacharunderscore}tags{\isacharparenright}\ eval{\isacharunderscore}runiq{\isacharunderscore}in{\isacharunderscore}Range\ mem{\isacharunderscore}Collect{\isacharunderscore}eq{\isacharparenright}\isanewline
\isacommand{ultimately}\isamarkupfalse%
\ \isacommand{moreover}\isamarkupfalse%
\ \isacommand{have}\isamarkupfalse%
\ {\isachardoublequoteopen}a{\isacharcomma}{\isacharcomma}n{\isadigit{1}}\ {\isasymnoteq}\ a{\isacharcomma}{\isacharcomma}n{\isadigit{2}}{\isachardoublequoteclose}\ \isacommand{using}\isamarkupfalse%
\ \isanewline
assms\ converse{\isachardot}intros\ eval{\isacharunderscore}runiq{\isacharunderscore}rel\ mem{\isacharunderscore}Collect{\isacharunderscore}eq\ runiq{\isacharunderscore}basic\ \isacommand{by}\isamarkupfalse%
\ {\isacharparenleft}metis\ {\isacharparenleft}lifting{\isacharcomma}\ no{\isacharunderscore}types{\isacharparenright}{\isacharparenright}\isanewline
\isacommand{ultimately}\isamarkupfalse%
\ \isacommand{show}\isamarkupfalse%
\ {\isacharquery}thesis\ \isacommand{using}\isamarkupfalse%
\ is{\isacharunderscore}non{\isacharunderscore}overlapping{\isacharunderscore}def\ \isacommand{by}\isamarkupfalse%
\ {\isacharparenleft}metis\ {\isacharparenleft}lifting{\isacharcomma}\ no{\isacharunderscore}types{\isacharparenright}\ assms{\isacharparenleft}{\isadigit{3}}{\isacharparenright}\ eval{\isacharunderscore}runiq{\isacharunderscore}in{\isacharunderscore}Range\ mem{\isacharunderscore}Collect{\isacharunderscore}eq{\isacharparenright}\isanewline
\isacommand{qed}\isamarkupfalse%
%
\endisatagproof
{\isafoldproof}%
%
\isadelimproof
%
\endisadelimproof
\isanewline
\isanewline
\isacommand{lemma}\isamarkupfalse%
\ lm{\isadigit{6}}{\isadigit{4}}c{\isacharcolon}\ \isakeyword{assumes}\ {\isachardoublequoteopen}a\ {\isasymin}\ allocationsUniverse{\isachardoublequoteclose}\ \isanewline
{\isachardoublequoteopen}n{\isadigit{1}}\ {\isasymin}\ Domain\ a{\isachardoublequoteclose}\ {\isachardoublequoteopen}n{\isadigit{2}}\ {\isasymin}\ Domain\ a{\isachardoublequoteclose}\isanewline
{\isachardoublequoteopen}n{\isadigit{1}}\ {\isasymnoteq}\ n{\isadigit{2}}{\isachardoublequoteclose}\ \isanewline
\isakeyword{shows}\ {\isachardoublequoteopen}a{\isacharcomma}{\isacharcomma}{\isacharcomma}n{\isadigit{1}}\ {\isasyminter}\ a{\isacharcomma}{\isacharcomma}{\isacharcomma}n{\isadigit{2}}{\isacharequal}{\isacharbraceleft}{\isacharbraceright}{\isachardoublequoteclose}%
\isadelimproof
\ %
\endisadelimproof
%
\isatagproof
\isacommand{using}\isamarkupfalse%
\ assms\ lm{\isadigit{6}}{\isadigit{4}}\ imageEquivalence\ \isacommand{by}\isamarkupfalse%
\ fastforce%
\endisatagproof
{\isafoldproof}%
%
\isadelimproof
%
\endisadelimproof
%
\begin{isamarkuptext}%
No good is assigned twice.%
\end{isamarkuptext}%
\isamarkuptrue%
\isacommand{theorem}\isamarkupfalse%
\ PairwiseDisjointAllocations{\isacharcolon}\isanewline
\isakeyword{fixes}\ n{\isadigit{1}}{\isacharcolon}{\isacharcolon}{\isachardoublequoteopen}participant{\isachardoublequoteclose}\ \isakeyword{fixes}\ G{\isacharcolon}{\isacharcolon}{\isachardoublequoteopen}goods\ list{\isachardoublequoteclose}\ \isakeyword{fixes}\ N{\isacharcolon}{\isacharcolon}{\isachardoublequoteopen}participant\ set{\isachardoublequoteclose}\isanewline
\isakeyword{assumes}\ {\isachardoublequoteopen}distinct\ G{\isachardoublequoteclose}\ {\isachardoublequoteopen}set\ G\ {\isasymnoteq}\ {\isacharbraceleft}{\isacharbraceright}{\isachardoublequoteclose}\ {\isachardoublequoteopen}finite\ N{\isachardoublequoteclose}\ \ \isanewline
\ \isanewline
{\isachardoublequoteopen}n{\isadigit{1}}\ {\isasymnoteq}\ n{\isadigit{2}}{\isachardoublequoteclose}\ \isanewline
\isakeyword{shows}\ {\isachardoublequoteopen}{\isacharparenleft}vcga{\isacharprime}\ N\ G\ b\ r{\isacharparenright}{\isacharcomma}{\isacharcomma}{\isacharcomma}n{\isadigit{1}}\ {\isasyminter}\ {\isacharparenleft}vcga{\isacharprime}\ N\ G\ b\ r{\isacharparenright}{\isacharcomma}{\isacharcomma}{\isacharcomma}n{\isadigit{2}}{\isacharequal}{\isacharbraceleft}{\isacharbraceright}{\isachardoublequoteclose}\ \ \isanewline
%
\isadelimproof
%
\endisadelimproof
%
\isatagproof
\isacommand{proof}\isamarkupfalse%
\ {\isacharminus}\isanewline
\isacommand{have}\isamarkupfalse%
\ {\isachardoublequoteopen}vcga{\isacharprime}\ N\ G\ b\ r\ {\isasymin}\ allocationsUniverse{\isachardoublequoteclose}\ \isacommand{using}\isamarkupfalse%
\ lm{\isadigit{5}}{\isadigit{8}}e\ assms\ \isacommand{by}\isamarkupfalse%
\ blast\isanewline
\isacommand{then}\isamarkupfalse%
\ \isacommand{show}\isamarkupfalse%
\ {\isacharquery}thesis\ \isacommand{using}\isamarkupfalse%
\ lm{\isadigit{6}}{\isadigit{4}}c\ assms\ \isacommand{by}\isamarkupfalse%
\ fast\isanewline
\isacommand{qed}\isamarkupfalse%
%
\endisatagproof
{\isafoldproof}%
%
\isadelimproof
\isanewline
%
\endisadelimproof
\isanewline
\isacommand{lemma}\isamarkupfalse%
\ \isakeyword{assumes}\ {\isachardoublequoteopen}R{\isacharcomma}{\isacharcomma}{\isacharcomma}x\ {\isasymnoteq}\ {\isacharbraceleft}{\isacharbraceright}{\isachardoublequoteclose}\ \isakeyword{shows}\ {\isachardoublequoteopen}x\ {\isasymin}\ Domain\ R{\isachardoublequoteclose}%
\isadelimproof
\ %
\endisadelimproof
%
\isatagproof
\isacommand{using}\isamarkupfalse%
\ assms\ \ \isanewline
\isacommand{proof}\isamarkupfalse%
\ {\isacharminus}\ \isacommand{have}\isamarkupfalse%
\ {\isachardoublequoteopen}{\isasymUnion}\ {\isacharparenleft}R{\isacharbackquote}{\isacharbackquote}{\isacharbraceleft}x{\isacharbraceright}{\isacharparenright}\ {\isasymnoteq}\ {\isacharbraceleft}{\isacharbraceright}{\isachardoublequoteclose}\ \isacommand{using}\isamarkupfalse%
\ assms{\isacharparenleft}{\isadigit{1}}{\isacharparenright}\ \isacommand{by}\isamarkupfalse%
\ fast\isanewline
\isacommand{then}\isamarkupfalse%
\ \isacommand{have}\isamarkupfalse%
\ {\isachardoublequoteopen}R{\isacharbackquote}{\isacharbackquote}{\isacharbraceleft}x{\isacharbraceright}\ {\isasymnoteq}\ {\isacharbraceleft}{\isacharbraceright}{\isachardoublequoteclose}\ \isacommand{by}\isamarkupfalse%
\ fast\ \isacommand{thus}\isamarkupfalse%
\ {\isacharquery}thesis\ \isacommand{by}\isamarkupfalse%
\ blast\ \isacommand{qed}\isamarkupfalse%
%
\endisatagproof
{\isafoldproof}%
%
\isadelimproof
%
\endisadelimproof
\isanewline
\isanewline
\isacommand{lemma}\isamarkupfalse%
\ \isakeyword{assumes}\ {\isachardoublequoteopen}runiq\ f{\isachardoublequoteclose}\ \isakeyword{and}\ {\isachardoublequoteopen}x\ {\isasymin}\ Domain\ f{\isachardoublequoteclose}\ \isakeyword{shows}\ {\isachardoublequoteopen}{\isacharparenleft}f\ {\isacharcomma}{\isacharcomma}\ x{\isacharparenright}\ {\isasymin}\ Range\ f{\isachardoublequoteclose}%
\isadelimproof
\ %
\endisadelimproof
%
\isatagproof
\isacommand{using}\isamarkupfalse%
\ assms\ \isanewline
\isacommand{by}\isamarkupfalse%
\ {\isacharparenleft}rule\ eval{\isacharunderscore}runiq{\isacharunderscore}in{\isacharunderscore}Range{\isacharparenright}%
\endisatagproof
{\isafoldproof}%
%
\isadelimproof
%
\endisadelimproof
%
\begin{isamarkuptext}%
Nothing outside the set of goods is allocated.%
\end{isamarkuptext}%
\isamarkuptrue%
\isacommand{theorem}\isamarkupfalse%
\ OnlyGoodsAllocated{\isacharcolon}\ \isakeyword{assumes}\ {\isachardoublequoteopen}distinct\ G{\isachardoublequoteclose}\ {\isachardoublequoteopen}set\ G\ {\isasymnoteq}\ {\isacharbraceleft}{\isacharbraceright}{\isachardoublequoteclose}\ {\isachardoublequoteopen}finite\ N{\isachardoublequoteclose}\ {\isachardoublequoteopen}g\ {\isasymin}\ {\isacharparenleft}vcga\ N\ G\ b\ r{\isacharparenright}{\isacharcomma}{\isacharcomma}{\isacharcomma}n{\isachardoublequoteclose}\ \isanewline
\isakeyword{shows}\ {\isachardoublequoteopen}g\ {\isasymin}\ set\ G{\isachardoublequoteclose}\isanewline
%
\isadelimproof
%
\endisadelimproof
%
\isatagproof
\isacommand{proof}\isamarkupfalse%
\ {\isacharminus}\ \isanewline
\isacommand{let}\isamarkupfalse%
\ {\isacharquery}a{\isacharequal}{\isachardoublequoteopen}vcga{\isacharprime}\ N\ G\ b\ r{\isachardoublequoteclose}\ \isacommand{have}\isamarkupfalse%
\ {\isachardoublequoteopen}{\isacharquery}a\ {\isasymin}\ allocationsUniverse{\isachardoublequoteclose}\ \isacommand{using}\isamarkupfalse%
\ assms{\isacharparenleft}{\isadigit{1}}{\isacharcomma}{\isadigit{2}}{\isacharcomma}{\isadigit{3}}{\isacharparenright}\ lm{\isadigit{5}}{\isadigit{8}}e\ \isacommand{by}\isamarkupfalse%
\ blast\isanewline
\isacommand{then}\isamarkupfalse%
\ \isacommand{have}\isamarkupfalse%
\ {\isachardoublequoteopen}runiq\ {\isacharquery}a{\isachardoublequoteclose}\ \isacommand{using}\isamarkupfalse%
\ assms{\isacharparenleft}{\isadigit{1}}{\isacharcomma}{\isadigit{2}}{\isacharcomma}{\isadigit{3}}{\isacharparenright}\ \isacommand{by}\isamarkupfalse%
\ blast\isanewline
\isacommand{moreover}\isamarkupfalse%
\ \isacommand{have}\isamarkupfalse%
\ {\isachardoublequoteopen}n\ {\isasymin}\ Domain\ {\isacharquery}a{\isachardoublequoteclose}\ \isacommand{using}\isamarkupfalse%
\ assms\ eval{\isacharunderscore}rel{\isacharunderscore}def\ lm{\isadigit{0}}{\isadigit{0}}{\isadigit{2}}\ \isacommand{by}\isamarkupfalse%
\ fast\isanewline
\isacommand{ultimately}\isamarkupfalse%
\ \isacommand{moreover}\isamarkupfalse%
\ \isacommand{have}\isamarkupfalse%
\ {\isachardoublequoteopen}{\isacharquery}a{\isacharcomma}{\isacharcomma}n\ {\isasymin}\ Range\ {\isacharquery}a{\isachardoublequoteclose}\ \isacommand{using}\isamarkupfalse%
\ eval{\isacharunderscore}runiq{\isacharunderscore}in{\isacharunderscore}Range\ \isacommand{by}\isamarkupfalse%
\ fast\ \isanewline
\isacommand{ultimately}\isamarkupfalse%
\ \isacommand{have}\isamarkupfalse%
\ {\isachardoublequoteopen}{\isacharquery}a{\isacharcomma}{\isacharcomma}{\isacharcomma}n\ {\isasymin}\ Range\ {\isacharquery}a{\isachardoublequoteclose}\ \isacommand{using}\isamarkupfalse%
\ imageEquivalence\ \isacommand{by}\isamarkupfalse%
\ fastforce\isanewline
\isacommand{then}\isamarkupfalse%
\ \isacommand{have}\isamarkupfalse%
\ {\isachardoublequoteopen}g\ {\isasymin}\ {\isasymUnion}\ Range\ {\isacharquery}a{\isachardoublequoteclose}\ \isacommand{using}\isamarkupfalse%
\ assms\ lm{\isadigit{0}}{\isadigit{0}}{\isadigit{2}}\ \isacommand{by}\isamarkupfalse%
\ blast\ \isanewline
\isacommand{moreover}\isamarkupfalse%
\ \isacommand{have}\isamarkupfalse%
\ {\isachardoublequoteopen}{\isasymUnion}\ Range\ {\isacharquery}a\ {\isasymsubseteq}\ set\ G{\isachardoublequoteclose}\ \isacommand{using}\isamarkupfalse%
\ assms{\isacharparenleft}{\isadigit{1}}{\isacharcomma}{\isadigit{2}}{\isacharcomma}{\isadigit{3}}{\isacharparenright}\ lm{\isadigit{5}}{\isadigit{8}}e\ \isacommand{by}\isamarkupfalse%
\ fast\isanewline
\isacommand{ultimately}\isamarkupfalse%
\ \isacommand{show}\isamarkupfalse%
\ {\isacharquery}thesis\ \isacommand{by}\isamarkupfalse%
\ blast\isanewline
\isacommand{qed}\isamarkupfalse%
%
\endisatagproof
{\isafoldproof}%
%
\isadelimproof
\isanewline
%
\endisadelimproof
\isanewline
\isacommand{definition}\isamarkupfalse%
\ {\isachardoublequoteopen}allStrictAllocations\ N\ G\ {\isacharequal}{\isacharequal}\ possibleAllocationsAlg\ N\ G{\isachardoublequoteclose}\isanewline
\isacommand{abbreviation}\isamarkupfalse%
\ {\isachardoublequoteopen}maximalStrictAllocations\ N\ G\ b{\isacharequal}{\isacharequal}\isanewline
argmax\ {\isacharparenleft}setsum\ b{\isacharparenright}\ {\isacharparenleft}set\ {\isacharparenleft}allStrictAllocations\ {\isacharparenleft}{\isacharbraceleft}seller{\isacharbraceright}{\isasymunion}N{\isacharparenright}\ G{\isacharparenright}{\isacharparenright}{\isachardoublequoteclose}\isanewline
\isanewline
\isacommand{definition}\isamarkupfalse%
\ {\isachardoublequoteopen}maximalStrictAllocations{\isadigit{2}}\ N\ G\ b{\isacharequal}\isanewline
argmax\ {\isacharparenleft}setsum\ b{\isacharparenright}\ {\isacharparenleft}set\ {\isacharparenleft}allStrictAllocations\ {\isacharparenleft}{\isacharbraceleft}seller{\isacharbraceright}{\isasymunion}N{\isacharparenright}\ G{\isacharparenright}{\isacharparenright}{\isachardoublequoteclose}\isanewline
\isanewline
\isacommand{definition}\isamarkupfalse%
\ {\isachardoublequoteopen}chosenAllocation\ N\ G\ b\ {\isacharparenleft}r{\isacharcolon}{\isacharcolon}integer{\isacharparenright}\ {\isacharequal}{\isacharequal}\ \isanewline
hd{\isacharparenleft}perm{\isadigit{2}}\ {\isacharparenleft}takeAll\ {\isacharparenleft}{\isacharpercent}x{\isachardot}\ x{\isasymin}\ {\isacharparenleft}argmax\ {\isasymcirc}\ setsum{\isacharparenright}\ b\ {\isacharparenleft}set\ {\isacharparenleft}allStrictAllocations\ N\ G{\isacharparenright}{\isacharparenright}{\isacharparenright}\ {\isacharparenleft}allStrictAllocations\ N\ G{\isacharparenright}{\isacharparenright}\ {\isacharparenleft}nat{\isacharunderscore}of{\isacharunderscore}integer\ r{\isacharparenright}{\isacharparenright}{\isachardoublequoteclose}\isanewline
\isanewline
\isacommand{definition}\isamarkupfalse%
\ {\isachardoublequoteopen}chosenAllocationEff\ N\ G\ b\ {\isacharparenleft}r{\isacharcolon}{\isacharcolon}integer{\isacharparenright}\ {\isacharequal}{\isacharequal}\ \isanewline
{\isacharparenleft}takeAll\ {\isacharparenleft}{\isacharpercent}x{\isachardot}\ x{\isasymin}\ {\isacharparenleft}argmax\ {\isasymcirc}\ setsum{\isacharparenright}\ b\ {\isacharparenleft}set\ {\isacharparenleft}allStrictAllocations\ N\ G{\isacharparenright}{\isacharparenright}{\isacharparenright}\ {\isacharparenleft}allStrictAllocations\ N\ G{\isacharparenright}\ {\isacharbang}\ {\isacharparenleft}nat{\isacharunderscore}of{\isacharunderscore}integer\ r{\isacharparenright}{\isacharparenright}{\isachardoublequoteclose}\isanewline
\isanewline
\isanewline
\isacommand{definition}\isamarkupfalse%
\ {\isachardoublequoteopen}maxbid\ a\ N\ G\ {\isacharequal}{\isacharequal}\ {\isacharparenleft}bidMaximizedBy\ a\ N\ G{\isacharparenright}\ Elsee\ {\isadigit{0}}{\isachardoublequoteclose}\isanewline
\isacommand{definition}\isamarkupfalse%
\ {\isachardoublequoteopen}summedBidVector\ bids\ N\ G\ {\isacharequal}{\isacharequal}\ {\isacharparenleft}summedBidVectorRel\ bids\ N\ G{\isacharparenright}\ Elsee\ {\isadigit{0}}{\isachardoublequoteclose}\isanewline
\isacommand{definition}\isamarkupfalse%
\ {\isachardoublequoteopen}tiebids\ a\ N\ G\ {\isacharequal}{\isacharequal}\ summedBidVector\ {\isacharparenleft}maxbid\ a\ N\ G{\isacharparenright}\ N\ G{\isachardoublequoteclose}\isanewline
\isacommand{definition}\isamarkupfalse%
\ {\isachardoublequoteopen}resolvingBid\ N\ G\ bids\ random\ {\isacharequal}{\isacharequal}\ tiebids\ {\isacharparenleft}chosenAllocation\ N\ G\ bids\ random{\isacharparenright}\ N\ {\isacharparenleft}set\ G{\isacharparenright}{\isachardoublequoteclose}\isanewline
\isacommand{definition}\isamarkupfalse%
\ {\isachardoublequoteopen}randomBids\ N\ {\isasymOmega}\ b\ random{\isacharequal}{\isacharequal}resolvingBid\ {\isacharparenleft}N{\isasymunion}{\isacharbraceleft}seller{\isacharbraceright}{\isacharparenright}\ {\isasymOmega}\ b\ random{\isachardoublequoteclose}\isanewline
\isacommand{definition}\isamarkupfalse%
\ {\isachardoublequoteopen}vcgaAlgWithoutLosers\ N\ G\ b\ r\ {\isacharequal}{\isacharequal}\ {\isacharparenleft}the{\isacharunderscore}elem\isanewline
{\isacharparenleft}argmax\ {\isacharparenleft}setsum\ {\isacharparenleft}randomBids\ N\ G\ b\ r{\isacharparenright}{\isacharparenright}\ {\isacharparenleft}maximalStrictAllocations\ N\ G\ b{\isacharparenright}{\isacharparenright}{\isacharparenright}\ {\isacharminus}{\isacharminus}\ seller{\isachardoublequoteclose}\isanewline
\isacommand{abbreviation}\isamarkupfalse%
\ {\isachardoublequoteopen}addLosers\ participantset\ allo{\isacharequal}{\isacharequal}{\isacharparenleft}participantset\ {\isasymtimes}\ {\isacharbraceleft}{\isacharbraceleft}{\isacharbraceright}{\isacharbraceright}{\isacharparenright}\ {\isacharplus}{\isacharasterisk}\ allo{\isachardoublequoteclose}\isanewline
\isacommand{definition}\isamarkupfalse%
\ {\isachardoublequoteopen}vcgaAlg\ N\ G\ b\ r\ {\isacharequal}\ addLosers\ N\ {\isacharparenleft}vcgaAlgWithoutLosers\ N\ G\ b\ r{\isacharparenright}{\isachardoublequoteclose}\isanewline
\isacommand{abbreviation}\isamarkupfalse%
\ {\isachardoublequoteopen}allAllocationsComp\ N\ {\isasymOmega}\ {\isacharequal}{\isacharequal}\ \isanewline
{\isacharparenleft}Outside{\isacharprime}\ {\isacharbraceleft}seller{\isacharbraceright}{\isacharparenright}\ {\isacharbackquote}\ set\ {\isacharparenleft}allStrictAllocations\ {\isacharparenleft}N\ {\isasymunion}\ {\isacharbraceleft}seller{\isacharbraceright}{\isacharparenright}\ {\isasymOmega}{\isacharparenright}{\isachardoublequoteclose}\isanewline
\isacommand{definition}\isamarkupfalse%
\ {\isachardoublequoteopen}vcgpAlg\ N\ G\ b\ r\ n\ {\isacharequal}\isanewline
Max\ {\isacharparenleft}setsum\ b\ {\isacharbackquote}\ {\isacharparenleft}allAllocationsComp\ {\isacharparenleft}N{\isacharminus}{\isacharbraceleft}n{\isacharbraceright}{\isacharparenright}\ G{\isacharparenright}{\isacharparenright}\ {\isacharminus}\ {\isacharparenleft}setsum\ b\ {\isacharparenleft}vcgaAlgWithoutLosers\ N\ G\ b\ r\ {\isacharminus}{\isacharminus}\ n{\isacharparenright}{\isacharparenright}{\isachardoublequoteclose}\isanewline
\isanewline
\isacommand{lemma}\isamarkupfalse%
\ lm{\isadigit{0}}{\isadigit{1}}{\isacharcolon}\ \isakeyword{assumes}\ {\isachardoublequoteopen}x\ {\isasymin}\ Domain\ f{\isachardoublequoteclose}\ \isakeyword{shows}\ {\isachardoublequoteopen}toFunction\ f\ x\ {\isacharequal}\ {\isacharparenleft}f\ Elsee\ {\isadigit{0}}{\isacharparenright}\ x{\isachardoublequoteclose}\isanewline
%
\isadelimproof
%
\endisadelimproof
%
\isatagproof
\isacommand{unfolding}\isamarkupfalse%
\ toFunctionWithFallback{\isadigit{2}}{\isacharunderscore}def\isanewline
\isacommand{by}\isamarkupfalse%
\ {\isacharparenleft}metis\ assms\ toFunction{\isacharunderscore}def{\isacharparenright}%
\endisatagproof
{\isafoldproof}%
%
\isadelimproof
\isanewline
%
\endisadelimproof
\isacommand{lemma}\isamarkupfalse%
\ lm{\isadigit{0}}{\isadigit{3}}{\isacharcolon}\ {\isachardoublequoteopen}Domain\ {\isacharparenleft}Y\ {\isasymtimes}\ {\isacharbraceleft}{\isadigit{0}}{\isacharcolon}{\isacharcolon}nat{\isacharbraceright}{\isacharparenright}\ {\isacharequal}\ Y\ {\isacharampersand}\ Domain\ {\isacharparenleft}X\ {\isasymtimes}\ {\isacharbraceleft}{\isadigit{1}}{\isacharbraceright}{\isacharparenright}{\isacharequal}X{\isachardoublequoteclose}%
\isadelimproof
\ %
\endisadelimproof
%
\isatagproof
\isacommand{by}\isamarkupfalse%
\ blast%
\endisatagproof
{\isafoldproof}%
%
\isadelimproof
%
\endisadelimproof
\isanewline
\isacommand{lemma}\isamarkupfalse%
\ lm{\isadigit{0}}{\isadigit{4}}{\isacharcolon}\ {\isachardoublequoteopen}Domain\ {\isacharparenleft}X\ {\isacharless}{\isacharbar}{\isacharbar}\ Y{\isacharparenright}\ {\isacharequal}\ X\ {\isasymunion}\ Y{\isachardoublequoteclose}%
\isadelimproof
\ %
\endisadelimproof
%
\isatagproof
\isacommand{using}\isamarkupfalse%
\ lm{\isadigit{0}}{\isadigit{3}}\ paste{\isacharunderscore}Domain\ sup{\isacharunderscore}commute\ \isacommand{by}\isamarkupfalse%
\ metis%
\endisatagproof
{\isafoldproof}%
%
\isadelimproof
%
\endisadelimproof
\isanewline
\isacommand{corollary}\isamarkupfalse%
\ lm{\isadigit{0}}{\isadigit{4}}b{\isacharcolon}\ {\isachardoublequoteopen}Domain\ {\isacharparenleft}bidMaximizedBy\ a\ N\ G{\isacharparenright}\ {\isacharequal}\ pseudoAllocation\ a\ {\isasymunion}\ N\ {\isasymtimes}\ {\isacharparenleft}finestpart\ G{\isacharparenright}{\isachardoublequoteclose}%
\isadelimproof
\ %
\endisadelimproof
%
\isatagproof
\isacommand{using}\isamarkupfalse%
\ lm{\isadigit{0}}{\isadigit{4}}\ \isanewline
\isacommand{by}\isamarkupfalse%
\ metis%
\endisatagproof
{\isafoldproof}%
%
\isadelimproof
%
\endisadelimproof
\isanewline
\isacommand{lemma}\isamarkupfalse%
\ lm{\isadigit{1}}{\isadigit{9}}{\isacharcolon}\ {\isachardoublequoteopen}{\isacharparenleft}pseudoAllocation\ a{\isacharparenright}\ {\isasymsubseteq}\ Domain\ {\isacharparenleft}bidMaximizedBy\ a\ N\ G{\isacharparenright}{\isachardoublequoteclose}%
\isadelimproof
\ %
\endisadelimproof
%
\isatagproof
\isacommand{by}\isamarkupfalse%
\ {\isacharparenleft}metis\ lm{\isadigit{0}}{\isadigit{4}}\ Un{\isacharunderscore}upper{\isadigit{1}}{\isacharparenright}%
\endisatagproof
{\isafoldproof}%
%
\isadelimproof
%
\endisadelimproof
\isanewline
\isanewline
\isacommand{lemma}\isamarkupfalse%
\ lm{\isadigit{0}}{\isadigit{2}}{\isacharcolon}\ \isakeyword{assumes}\ {\isachardoublequoteopen}x\ {\isasymin}\ {\isacharparenleft}N\ {\isasymtimes}\ {\isacharparenleft}Pow\ G\ {\isacharminus}\ {\isacharbraceleft}{\isacharbraceleft}{\isacharbraceright}{\isacharbraceright}{\isacharparenright}{\isacharparenright}{\isachardoublequoteclose}\ \isakeyword{shows}\ \isanewline
{\isachardoublequoteopen}summedBidVector{\isacharprime}\ b\ N\ G\ x{\isacharequal}summedBidVector\ b\ N\ G\ x{\isachardoublequoteclose}%
\isadelimproof
\ %
\endisadelimproof
%
\isatagproof
\isacommand{unfolding}\isamarkupfalse%
\ summedBidVector{\isacharunderscore}def\ \isanewline
\isacommand{using}\isamarkupfalse%
\ assms\ lm{\isadigit{0}}{\isadigit{1}}\ Domain{\isachardot}simps\ imageI\ \isacommand{by}\isamarkupfalse%
\ {\isacharparenleft}metis{\isacharparenleft}no{\isacharunderscore}types{\isacharcomma}lifting{\isacharparenright}{\isacharparenright}%
\endisatagproof
{\isafoldproof}%
%
\isadelimproof
%
\endisadelimproof
\isanewline
\isanewline
\isanewline
\isanewline
\isanewline
\isacommand{corollary}\isamarkupfalse%
\ lm{\isadigit{2}}{\isadigit{0}}{\isacharcolon}\ \isakeyword{assumes}\ {\isachardoublequoteopen}{\isasymforall}x\ {\isasymin}\ X{\isachardot}\ f\ x\ {\isacharequal}\ g\ x{\isachardoublequoteclose}\ \isakeyword{shows}\ {\isachardoublequoteopen}setsum\ f\ X\ {\isacharequal}\ setsum\ g\ X{\isachardoublequoteclose}\ \isanewline
%
\isadelimproof
%
\endisadelimproof
%
\isatagproof
\isacommand{using}\isamarkupfalse%
\ assms\ setsum{\isachardot}cong\ \isacommand{by}\isamarkupfalse%
\ auto%
\endisatagproof
{\isafoldproof}%
%
\isadelimproof
\isanewline
%
\endisadelimproof
\isanewline
\isacommand{lemma}\isamarkupfalse%
\ lm{\isadigit{0}}{\isadigit{6}}{\isacharcolon}\ \isakeyword{assumes}\ {\isachardoublequoteopen}fst\ pair\ {\isasymin}\ N{\isachardoublequoteclose}\ {\isachardoublequoteopen}snd\ pair\ {\isasymin}\ Pow\ G\ {\isacharminus}\ {\isacharbraceleft}{\isacharbraceleft}{\isacharbraceright}{\isacharbraceright}{\isachardoublequoteclose}\ \isakeyword{shows}\ {\isachardoublequoteopen}setsum\ {\isacharparenleft}{\isacharpercent}g{\isachardot}\isanewline
{\isacharparenleft}toFunction\ {\isacharparenleft}bidMaximizedBy\ a\ N\ G{\isacharparenright}{\isacharparenright}\isanewline
{\isacharparenleft}fst\ pair{\isacharcomma}\ g{\isacharparenright}{\isacharparenright}\ {\isacharparenleft}finestpart\ {\isacharparenleft}snd\ pair{\isacharparenright}{\isacharparenright}\ {\isacharequal}\isanewline
setsum\ {\isacharparenleft}{\isacharpercent}g{\isachardot}\ \isanewline
{\isacharparenleft}{\isacharparenleft}bidMaximizedBy\ a\ N\ G{\isacharparenright}\ Elsee\ {\isadigit{0}}{\isacharparenright}\isanewline
{\isacharparenleft}fst\ pair{\isacharcomma}\ g{\isacharparenright}{\isacharparenright}\ {\isacharparenleft}finestpart\ {\isacharparenleft}snd\ pair{\isacharparenright}{\isacharparenright}{\isachardoublequoteclose}\isanewline
%
\isadelimproof
%
\endisadelimproof
%
\isatagproof
\isacommand{using}\isamarkupfalse%
\ assms\ lm{\isadigit{0}}{\isadigit{1}}\ lm{\isadigit{0}}{\isadigit{5}}\ lm{\isadigit{0}}{\isadigit{4}}\ Un{\isacharunderscore}upper{\isadigit{1}}\ UnCI\ UnI{\isadigit{1}}\ setsum{\isachardot}cong\ finestpartSubset\ Diff{\isacharunderscore}iff\ Pow{\isacharunderscore}iff\ in{\isacharunderscore}mono\isanewline
\isacommand{proof}\isamarkupfalse%
\ {\isacharminus}\isanewline
\isacommand{let}\isamarkupfalse%
\ {\isacharquery}f{\isadigit{1}}{\isacharequal}{\isachardoublequoteopen}{\isacharpercent}g{\isachardot}{\isacharparenleft}toFunction\ {\isacharparenleft}bidMaximizedBy\ a\ N\ G{\isacharparenright}{\isacharparenright}{\isacharparenleft}fst\ pair{\isacharcomma}\ g{\isacharparenright}{\isachardoublequoteclose}\isanewline
\isacommand{let}\isamarkupfalse%
\ {\isacharquery}f{\isadigit{2}}{\isacharequal}{\isachardoublequoteopen}{\isacharpercent}g{\isachardot}{\isacharparenleft}{\isacharparenleft}bidMaximizedBy\ a\ N\ G{\isacharparenright}\ Elsee\ {\isadigit{0}}{\isacharparenright}{\isacharparenleft}fst\ pair{\isacharcomma}\ g{\isacharparenright}{\isachardoublequoteclose}\isanewline
\isacommand{{\isacharbraceleft}}\isamarkupfalse%
\ \isanewline
\ \ \isacommand{fix}\isamarkupfalse%
\ g\ \isacommand{assume}\isamarkupfalse%
\ {\isachardoublequoteopen}g\ {\isasymin}\ finestpart\ {\isacharparenleft}snd\ pair{\isacharparenright}{\isachardoublequoteclose}\ \isacommand{then}\isamarkupfalse%
\ \isacommand{have}\isamarkupfalse%
\ \isanewline
\ \ {\isadigit{0}}{\isacharcolon}\ {\isachardoublequoteopen}g\ {\isasymin}\ finestpart\ G{\isachardoublequoteclose}\ \isacommand{using}\isamarkupfalse%
\ assms\ finestpartSubset\ \ \isacommand{by}\isamarkupfalse%
\ {\isacharparenleft}metis\ Diff{\isacharunderscore}iff\ Pow{\isacharunderscore}iff\ in{\isacharunderscore}mono{\isacharparenright}\isanewline
\ \ \isacommand{have}\isamarkupfalse%
\ {\isachardoublequoteopen}{\isacharquery}f{\isadigit{1}}\ g\ {\isacharequal}\ {\isacharquery}f{\isadigit{2}}\ g{\isachardoublequoteclose}\ \isanewline
\ \ \isacommand{proof}\isamarkupfalse%
\ {\isacharminus}\isanewline
\ \ \ \ \isacommand{have}\isamarkupfalse%
\ {\isachardoublequoteopen}{\isasymAnd}x\isactrlsub {\isadigit{1}}\ x\isactrlsub {\isadigit{2}}{\isachardot}\ {\isacharparenleft}x\isactrlsub {\isadigit{1}}{\isacharcomma}\ g{\isacharparenright}\ {\isasymin}\ x\isactrlsub {\isadigit{2}}\ {\isasymtimes}\ finestpart\ G\ {\isasymor}\ x\isactrlsub {\isadigit{1}}\ {\isasymnotin}\ x\isactrlsub {\isadigit{2}}{\isachardoublequoteclose}\ \isacommand{by}\isamarkupfalse%
\ {\isacharparenleft}metis\ {\isadigit{0}}\ mem{\isacharunderscore}Sigma{\isacharunderscore}iff{\isacharparenright}\ \isanewline
\ \ \ \ \isacommand{then}\isamarkupfalse%
\ \isacommand{have}\isamarkupfalse%
\ {\isachardoublequoteopen}{\isacharparenleft}pseudoAllocation\ a\ {\isacharless}{\isacharbar}\ {\isacharparenleft}N\ {\isasymtimes}\ finestpart\ G{\isacharparenright}{\isacharparenright}\ {\isacharparenleft}fst\ pair{\isacharcomma}\ g{\isacharparenright}\ {\isacharequal}\ maxbid\ a\ N\ G\ {\isacharparenleft}fst\ pair{\isacharcomma}\ g{\isacharparenright}{\isachardoublequoteclose}\isanewline
\ \ \ \ \isacommand{unfolding}\isamarkupfalse%
\ toFunctionWithFallback{\isadigit{2}}{\isacharunderscore}def\ maxbid{\isacharunderscore}def\isanewline
\ \ \ \ \isacommand{by}\isamarkupfalse%
\ {\isacharparenleft}metis\ {\isacharparenleft}no{\isacharunderscore}types{\isacharparenright}\ lm{\isadigit{0}}{\isadigit{4}}\ UnCI\ assms{\isacharparenleft}{\isadigit{1}}{\isacharparenright}\ toFunction{\isacharunderscore}def{\isacharparenright}\isanewline
\ \ \ \ \isacommand{thus}\isamarkupfalse%
\ {\isacharquery}thesis\ \isacommand{unfolding}\isamarkupfalse%
\ maxbid{\isacharunderscore}def\ \isacommand{by}\isamarkupfalse%
\ blast\isanewline
\ \ \isacommand{qed}\isamarkupfalse%
\isanewline
\isacommand{{\isacharbraceright}}\isamarkupfalse%
\isanewline
\isacommand{thus}\isamarkupfalse%
\ {\isacharquery}thesis\ \isacommand{using}\isamarkupfalse%
\ setsum{\isachardot}cong\ \isacommand{by}\isamarkupfalse%
\ simp\isanewline
\isacommand{qed}\isamarkupfalse%
%
\endisatagproof
{\isafoldproof}%
%
\isadelimproof
\isanewline
%
\endisadelimproof
\isanewline
\isacommand{corollary}\isamarkupfalse%
\ lm{\isadigit{0}}{\isadigit{7}}{\isacharcolon}\ \isakeyword{assumes}\ {\isachardoublequoteopen}pair\ {\isasymin}\ N\ {\isasymtimes}\ {\isacharparenleft}Pow\ G\ {\isacharminus}\ {\isacharbraceleft}{\isacharbraceleft}{\isacharbraceright}{\isacharbraceright}{\isacharparenright}{\isachardoublequoteclose}\ \isakeyword{shows}\ \isanewline
{\isachardoublequoteopen}summedBid\ {\isacharparenleft}toFunction\ {\isacharparenleft}bidMaximizedBy\ a\ N\ G{\isacharparenright}{\isacharparenright}\ pair\ {\isacharequal}\ \isanewline
summedBid\ {\isacharparenleft}{\isacharparenleft}bidMaximizedBy\ a\ N\ G{\isacharparenright}\ Elsee\ {\isadigit{0}}{\isacharparenright}\ pair{\isachardoublequoteclose}%
\isadelimproof
\ %
\endisadelimproof
%
\isatagproof
\isacommand{using}\isamarkupfalse%
\ assms\ lm{\isadigit{0}}{\isadigit{6}}\ \isanewline
\isacommand{proof}\isamarkupfalse%
\ {\isacharminus}\ \isanewline
\isacommand{have}\isamarkupfalse%
\ {\isachardoublequoteopen}fst\ pair\ {\isasymin}\ N{\isachardoublequoteclose}\ \isacommand{using}\isamarkupfalse%
\ assms\ \isacommand{by}\isamarkupfalse%
\ force\ \isanewline
\isacommand{moreover}\isamarkupfalse%
\ \isacommand{have}\isamarkupfalse%
\ {\isachardoublequoteopen}snd\ pair\ {\isasymin}\ Pow\ G\ {\isacharminus}\ {\isacharbraceleft}{\isacharbraceleft}{\isacharbraceright}{\isacharbraceright}{\isachardoublequoteclose}\ \isacommand{using}\isamarkupfalse%
\ assms{\isacharparenleft}{\isadigit{1}}{\isacharparenright}\ \isacommand{by}\isamarkupfalse%
\ force\isanewline
\isacommand{ultimately}\isamarkupfalse%
\ \isacommand{show}\isamarkupfalse%
\ {\isacharquery}thesis\ \isacommand{using}\isamarkupfalse%
\ lm{\isadigit{0}}{\isadigit{6}}\ \isacommand{by}\isamarkupfalse%
\ blast\isanewline
\isacommand{qed}\isamarkupfalse%
%
\endisatagproof
{\isafoldproof}%
%
\isadelimproof
%
\endisadelimproof
\isanewline
\isanewline
\isacommand{lemma}\isamarkupfalse%
\ lm{\isadigit{0}}{\isadigit{8}}{\isacharcolon}\ \isakeyword{assumes}\ {\isachardoublequoteopen}{\isasymforall}x\ {\isasymin}\ X{\isachardot}\ f\ x\ {\isacharequal}\ g\ x{\isachardoublequoteclose}\ \isakeyword{shows}\ {\isachardoublequoteopen}f{\isacharbackquote}X{\isacharequal}g{\isacharbackquote}X{\isachardoublequoteclose}%
\isadelimproof
\ %
\endisadelimproof
%
\isatagproof
\isacommand{using}\isamarkupfalse%
\ assms\ \isacommand{by}\isamarkupfalse%
\ {\isacharparenleft}metis\ image{\isacharunderscore}cong{\isacharparenright}%
\endisatagproof
{\isafoldproof}%
%
\isadelimproof
%
\endisadelimproof
\isanewline
\isanewline
\isacommand{corollary}\isamarkupfalse%
\ lm{\isadigit{0}}{\isadigit{9}}{\isacharcolon}\ {\isachardoublequoteopen}{\isasymforall}\ pair\ {\isasymin}\ N\ {\isasymtimes}\ {\isacharparenleft}Pow\ G\ {\isacharminus}\ {\isacharbraceleft}{\isacharbraceleft}{\isacharbraceright}{\isacharbraceright}{\isacharparenright}{\isachardot}\ \ \isanewline
summedBid\ {\isacharparenleft}toFunction\ {\isacharparenleft}bidMaximizedBy\ a\ N\ G{\isacharparenright}{\isacharparenright}\ pair\ {\isacharequal}\ \isanewline
summedBid\ {\isacharparenleft}{\isacharparenleft}bidMaximizedBy\ a\ N\ G{\isacharparenright}\ Elsee\ {\isadigit{0}}{\isacharparenright}\ pair{\isachardoublequoteclose}%
\isadelimproof
\ %
\endisadelimproof
%
\isatagproof
\isacommand{using}\isamarkupfalse%
\ lm{\isadigit{0}}{\isadigit{7}}\ \isanewline
\isacommand{by}\isamarkupfalse%
\ blast%
\endisatagproof
{\isafoldproof}%
%
\isadelimproof
%
\endisadelimproof
\ \ \isanewline
\isanewline
\isacommand{corollary}\isamarkupfalse%
\ lm{\isadigit{1}}{\isadigit{0}}{\isacharcolon}\ \isanewline
{\isachardoublequoteopen}{\isacharparenleft}summedBid\ {\isacharparenleft}toFunction\ {\isacharparenleft}bidMaximizedBy\ a\ N\ G{\isacharparenright}{\isacharparenright}{\isacharparenright}\ {\isacharbackquote}\ {\isacharparenleft}N\ {\isasymtimes}\ {\isacharparenleft}Pow\ G\ {\isacharminus}\ {\isacharbraceleft}{\isacharbraceleft}{\isacharbraceright}{\isacharbraceright}{\isacharparenright}{\isacharparenright}{\isacharequal}\isanewline
{\isacharparenleft}summedBid\ {\isacharparenleft}{\isacharparenleft}bidMaximizedBy\ a\ N\ G{\isacharparenright}\ Elsee\ {\isadigit{0}}{\isacharparenright}{\isacharparenright}\ {\isacharbackquote}\ {\isacharparenleft}N\ {\isasymtimes}\ {\isacharparenleft}Pow\ G\ {\isacharminus}\ {\isacharbraceleft}{\isacharbraceleft}{\isacharbraceright}{\isacharbraceright}{\isacharparenright}{\isacharparenright}{\isachardoublequoteclose}\ {\isacharparenleft}\isakeyword{is}\ {\isachardoublequoteopen}{\isacharquery}f{\isadigit{1}}\ {\isacharbackquote}\ {\isacharquery}Z\ {\isacharequal}\ {\isacharquery}f{\isadigit{2}}\ {\isacharbackquote}\ {\isacharquery}Z{\isachardoublequoteclose}{\isacharparenright}\isanewline
%
\isadelimproof
%
\endisadelimproof
%
\isatagproof
\isacommand{proof}\isamarkupfalse%
\ {\isacharminus}\ \isanewline
\isacommand{have}\isamarkupfalse%
\ {\isachardoublequoteopen}{\isasymforall}\ z\ {\isasymin}\ {\isacharquery}Z{\isachardot}\ {\isacharquery}f{\isadigit{1}}\ z\ {\isacharequal}\ {\isacharquery}f{\isadigit{2}}\ z{\isachardoublequoteclose}\ \isacommand{by}\isamarkupfalse%
\ {\isacharparenleft}rule\ lm{\isadigit{0}}{\isadigit{9}}{\isacharparenright}\ \isacommand{thus}\isamarkupfalse%
\ {\isacharquery}thesis\ \isacommand{by}\isamarkupfalse%
\ {\isacharparenleft}rule\ lm{\isadigit{0}}{\isadigit{8}}{\isacharparenright}\isanewline
\isacommand{qed}\isamarkupfalse%
%
\endisatagproof
{\isafoldproof}%
%
\isadelimproof
\isanewline
%
\endisadelimproof
\isanewline
\isacommand{corollary}\isamarkupfalse%
\ lm{\isadigit{1}}{\isadigit{1}}{\isacharcolon}\ {\isachardoublequoteopen}summedBidVectorRel\ {\isacharparenleft}toFunction\ {\isacharparenleft}bidMaximizedBy\ a\ N\ G{\isacharparenright}{\isacharparenright}\ N\ G\ {\isacharequal}\isanewline
summedBidVectorRel\ {\isacharparenleft}{\isacharparenleft}bidMaximizedBy\ a\ N\ G{\isacharparenright}\ Elsee\ {\isadigit{0}}{\isacharparenright}\ N\ G{\isachardoublequoteclose}%
\isadelimproof
\ %
\endisadelimproof
%
\isatagproof
\isacommand{using}\isamarkupfalse%
\ lm{\isadigit{1}}{\isadigit{0}}\ \isacommand{by}\isamarkupfalse%
\ metis%
\endisatagproof
{\isafoldproof}%
%
\isadelimproof
%
\endisadelimproof
\isanewline
\isanewline
\isacommand{corollary}\isamarkupfalse%
\ lm{\isadigit{1}}{\isadigit{2}}{\isacharcolon}\ {\isachardoublequoteopen}summedBidVectorRel\ {\isacharparenleft}maxbid{\isacharprime}\ a\ N\ G{\isacharparenright}\ N\ G\ {\isacharequal}\ summedBidVectorRel\ {\isacharparenleft}maxbid\ a\ N\ G{\isacharparenright}\ N\ G{\isachardoublequoteclose}\isanewline
%
\isadelimproof
%
\endisadelimproof
%
\isatagproof
\isacommand{unfolding}\isamarkupfalse%
\ maxbid{\isacharunderscore}def\ \isacommand{using}\isamarkupfalse%
\ lm{\isadigit{1}}{\isadigit{1}}\ \isacommand{by}\isamarkupfalse%
\ metis%
\endisatagproof
{\isafoldproof}%
%
\isadelimproof
\isanewline
%
\endisadelimproof
\isanewline
\isacommand{lemma}\isamarkupfalse%
\ lm{\isadigit{1}}{\isadigit{3}}{\isacharcolon}\ \isakeyword{assumes}\ {\isachardoublequoteopen}x\ {\isasymin}\ {\isacharparenleft}N\ {\isasymtimes}\ {\isacharparenleft}Pow\ G\ {\isacharminus}\ {\isacharbraceleft}{\isacharbraceleft}{\isacharbraceright}{\isacharbraceright}{\isacharparenright}{\isacharparenright}{\isachardoublequoteclose}\ \isakeyword{shows}\ \isanewline
{\isachardoublequoteopen}summedBidVector{\isacharprime}\ {\isacharparenleft}maxbid{\isacharprime}\ a\ N\ G{\isacharparenright}\ N\ G\ x\ {\isacharequal}\ summedBidVector\ {\isacharparenleft}maxbid\ a\ N\ G{\isacharparenright}\ N\ G\ x{\isachardoublequoteclose}\isanewline
{\isacharparenleft}\isakeyword{is}\ {\isachardoublequoteopen}{\isacharquery}f{\isadigit{1}}\ {\isacharquery}g{\isadigit{1}}\ N\ G\ x\ {\isacharequal}\ {\isacharquery}f{\isadigit{2}}\ {\isacharquery}g{\isadigit{2}}\ N\ G\ x{\isachardoublequoteclose}{\isacharparenright}\isanewline
%
\isadelimproof
%
\endisadelimproof
%
\isatagproof
\isacommand{using}\isamarkupfalse%
\ assms\ lm{\isadigit{0}}{\isadigit{2}}\ lm{\isadigit{1}}{\isadigit{2}}\ \ \isanewline
\isacommand{proof}\isamarkupfalse%
\ {\isacharminus}\isanewline
\isacommand{let}\isamarkupfalse%
\ {\isacharquery}h{\isadigit{1}}{\isacharequal}{\isachardoublequoteopen}maxbid{\isacharprime}\ a\ N\ G{\isachardoublequoteclose}\ \isacommand{let}\isamarkupfalse%
\ {\isacharquery}h{\isadigit{2}}{\isacharequal}{\isachardoublequoteopen}maxbid\ a\ N\ G{\isachardoublequoteclose}\ \isacommand{let}\isamarkupfalse%
\ {\isacharquery}hh{\isadigit{1}}{\isacharequal}{\isachardoublequoteopen}real\ {\isasymcirc}\ {\isacharquery}h{\isadigit{1}}{\isachardoublequoteclose}\ \isacommand{let}\isamarkupfalse%
\ {\isacharquery}hh{\isadigit{2}}{\isacharequal}{\isachardoublequoteopen}real\ {\isasymcirc}\ {\isacharquery}h{\isadigit{2}}{\isachardoublequoteclose}\isanewline
\isacommand{have}\isamarkupfalse%
\ {\isachardoublequoteopen}summedBidVectorRel\ {\isacharquery}h{\isadigit{1}}\ N\ G\ {\isacharequal}\ summedBidVectorRel\ {\isacharquery}h{\isadigit{2}}\ N\ G{\isachardoublequoteclose}\ \isacommand{using}\isamarkupfalse%
\ lm{\isadigit{1}}{\isadigit{2}}\ \isacommand{by}\isamarkupfalse%
\ metis\ \isanewline
\isacommand{moreover}\isamarkupfalse%
\ \isacommand{have}\isamarkupfalse%
\ {\isachardoublequoteopen}summedBidVector\ {\isacharquery}h{\isadigit{2}}\ N\ G{\isacharequal}{\isacharparenleft}summedBidVectorRel\ {\isacharquery}h{\isadigit{2}}\ N\ G{\isacharparenright}\ Elsee\ {\isadigit{0}}{\isachardoublequoteclose}\isanewline
\isacommand{unfolding}\isamarkupfalse%
\ summedBidVector{\isacharunderscore}def\ \isacommand{by}\isamarkupfalse%
\ fast\isanewline
\isacommand{ultimately}\isamarkupfalse%
\ \isacommand{have}\isamarkupfalse%
\ {\isachardoublequoteopen}\ summedBidVector\ {\isacharquery}h{\isadigit{2}}\ N\ G{\isacharequal}summedBidVectorRel\ {\isacharquery}h{\isadigit{1}}\ N\ G\ Elsee\ {\isadigit{0}}{\isachardoublequoteclose}\ \isacommand{by}\isamarkupfalse%
\ presburger\isanewline
\isacommand{moreover}\isamarkupfalse%
\ \isacommand{have}\isamarkupfalse%
\ {\isachardoublequoteopen}{\isachardot}{\isachardot}{\isachardot}\ x\ {\isacharequal}\ {\isacharparenleft}toFunction\ {\isacharparenleft}summedBidVectorRel\ {\isacharquery}h{\isadigit{1}}\ N\ G{\isacharparenright}{\isacharparenright}\ x{\isachardoublequoteclose}\ \isacommand{using}\isamarkupfalse%
\ assms\ \isanewline
lm{\isadigit{0}}{\isadigit{1}}\ UniformTieBreaking{\isachardot}lm{\isadigit{6}}{\isadigit{4}}\ \isacommand{by}\isamarkupfalse%
\ {\isacharparenleft}metis\ {\isacharparenleft}mono{\isacharunderscore}tags{\isacharparenright}{\isacharparenright}\isanewline
\isacommand{ultimately}\isamarkupfalse%
\ \isacommand{have}\isamarkupfalse%
\ {\isachardoublequoteopen}summedBidVector\ {\isacharquery}h{\isadigit{2}}\ N\ G\ x\ {\isacharequal}\ {\isacharparenleft}toFunction\ {\isacharparenleft}summedBidVectorRel\ {\isacharquery}h{\isadigit{1}}\ N\ G{\isacharparenright}{\isacharparenright}\ x{\isachardoublequoteclose}\ \isanewline
\isacommand{by}\isamarkupfalse%
\ {\isacharparenleft}metis\ {\isacharparenleft}lifting{\isacharcomma}\ no{\isacharunderscore}types{\isacharparenright}{\isacharparenright}\isanewline
\isacommand{thus}\isamarkupfalse%
\ {\isacharquery}thesis\ \isacommand{by}\isamarkupfalse%
\ simp\isanewline
\isacommand{qed}\isamarkupfalse%
%
\endisatagproof
{\isafoldproof}%
%
\isadelimproof
\isanewline
%
\endisadelimproof
\isanewline
\isacommand{corollary}\isamarkupfalse%
\ lm{\isadigit{7}}{\isadigit{0}}c{\isacharcolon}\ \isakeyword{assumes}\ {\isachardoublequoteopen}card\ N\ {\isachargreater}\ {\isadigit{0}}{\isachardoublequoteclose}\ {\isachardoublequoteopen}distinct\ G{\isachardoublequoteclose}\ \isakeyword{shows}\ \isanewline
{\isachardoublequoteopen}possibleAllocationsRel\ N\ {\isacharparenleft}set\ G{\isacharparenright}\ {\isacharequal}\ set\ {\isacharparenleft}possibleAllocationsAlg\ N\ G{\isacharparenright}{\isachardoublequoteclose}\ \ \isanewline
%
\isadelimproof
%
\endisadelimproof
%
\isatagproof
\isacommand{using}\isamarkupfalse%
\ assms\ Universes{\isachardot}lm{\isadigit{7}}{\isadigit{0}}b\ \isacommand{by}\isamarkupfalse%
\ metis%
\endisatagproof
{\isafoldproof}%
%
\isadelimproof
\isanewline
%
\endisadelimproof
\isanewline
\isacommand{lemma}\isamarkupfalse%
\ lm{\isadigit{2}}{\isadigit{4}}{\isacharcolon}\ \isakeyword{assumes}\ {\isachardoublequoteopen}card\ A\ {\isachargreater}\ {\isadigit{0}}{\isachardoublequoteclose}\ {\isachardoublequoteopen}card\ B\ {\isachargreater}\ {\isadigit{0}}{\isachardoublequoteclose}\ \isakeyword{shows}\ {\isachardoublequoteopen}card\ {\isacharparenleft}A\ {\isasymunion}\ B{\isacharparenright}\ {\isachargreater}\ {\isadigit{0}}{\isachardoublequoteclose}\ \isanewline
%
\isadelimproof
%
\endisadelimproof
%
\isatagproof
\isacommand{using}\isamarkupfalse%
\ assms\ card{\isacharunderscore}gt{\isacharunderscore}{\isadigit{0}}{\isacharunderscore}iff\ finite{\isacharunderscore}Un\ sup{\isacharunderscore}eq{\isacharunderscore}bot{\isacharunderscore}iff\ \isacommand{by}\isamarkupfalse%
\ {\isacharparenleft}metis{\isacharparenleft}no{\isacharunderscore}types{\isacharparenright}{\isacharparenright}%
\endisatagproof
{\isafoldproof}%
%
\isadelimproof
\isanewline
%
\endisadelimproof
\isacommand{corollary}\isamarkupfalse%
\ lm{\isadigit{2}}{\isadigit{4}}b{\isacharcolon}\ \isakeyword{assumes}\ {\isachardoublequoteopen}card\ A\ {\isachargreater}\ {\isadigit{0}}{\isachardoublequoteclose}\ \isakeyword{shows}\ {\isachardoublequoteopen}card\ {\isacharparenleft}{\isacharbraceleft}a{\isacharbraceright}\ {\isasymunion}\ A{\isacharparenright}\ {\isachargreater}\ {\isadigit{0}}{\isachardoublequoteclose}%
\isadelimproof
\ %
\endisadelimproof
%
\isatagproof
\isacommand{using}\isamarkupfalse%
\ assms\ lm{\isadigit{2}}{\isadigit{4}}\ \isanewline
\isacommand{by}\isamarkupfalse%
\ {\isacharparenleft}metis\ card{\isacharunderscore}empty\ card{\isacharunderscore}insert{\isacharunderscore}disjoint\ empty{\isacharunderscore}iff\ finite{\isachardot}emptyI\ lessI{\isacharparenright}%
\endisatagproof
{\isafoldproof}%
%
\isadelimproof
%
\endisadelimproof
\isanewline
\isanewline
\isacommand{corollary}\isamarkupfalse%
\ \isakeyword{assumes}\ {\isachardoublequoteopen}card\ N\ {\isachargreater}\ {\isadigit{0}}{\isachardoublequoteclose}\ {\isachardoublequoteopen}distinct\ G{\isachardoublequoteclose}\ \isakeyword{shows}\isanewline
{\isachardoublequoteopen}maximalStrictAllocations{\isacharprime}\ N\ {\isacharparenleft}set\ G{\isacharparenright}\ b\ {\isacharequal}\ maximalStrictAllocations\ N\ G\ b{\isachardoublequoteclose}\isanewline
%
\isadelimproof
%
\endisadelimproof
%
\isatagproof
\isacommand{unfolding}\isamarkupfalse%
\ allStrictAllocations{\isacharunderscore}def\isanewline
\isacommand{using}\isamarkupfalse%
\ assms\ lm{\isadigit{7}}{\isadigit{0}}c\ lm{\isadigit{2}}{\isadigit{4}}b\ \isacommand{by}\isamarkupfalse%
\ {\isacharparenleft}metis{\isacharparenleft}no{\isacharunderscore}types{\isacharparenright}{\isacharparenright}%
\endisatagproof
{\isafoldproof}%
%
\isadelimproof
\isanewline
%
\endisadelimproof
\isanewline
\isanewline
\isanewline
\isacommand{corollary}\isamarkupfalse%
\ lm{\isadigit{7}}{\isadigit{0}}d{\isacharcolon}\ \isakeyword{assumes}\ {\isachardoublequoteopen}card\ N\ {\isachargreater}\ {\isadigit{0}}{\isachardoublequoteclose}\ {\isachardoublequoteopen}distinct\ G{\isachardoublequoteclose}\ \isakeyword{shows}\ \isanewline
{\isachardoublequoteopen}allAllocations\ N\ {\isacharparenleft}set\ G{\isacharparenright}\ {\isacharequal}\ set\ {\isacharparenleft}allStrictAllocations\ N\ G{\isacharparenright}{\isachardoublequoteclose}\ \isanewline
%
\isadelimproof
%
\endisadelimproof
%
\isatagproof
\isacommand{unfolding}\isamarkupfalse%
\ allStrictAllocations{\isacharunderscore}def\isanewline
\isacommand{using}\isamarkupfalse%
\ assms\ lm{\isadigit{7}}{\isadigit{0}}c\ \isacommand{by}\isamarkupfalse%
\ blast%
\endisatagproof
{\isafoldproof}%
%
\isadelimproof
\ \isanewline
%
\endisadelimproof
\isanewline
\isacommand{corollary}\isamarkupfalse%
\ lm{\isadigit{7}}{\isadigit{0}}f{\isacharcolon}\ \isakeyword{assumes}\ {\isachardoublequoteopen}card\ N\ {\isachargreater}\ {\isadigit{0}}{\isachardoublequoteclose}\ {\isachardoublequoteopen}distinct\ G{\isachardoublequoteclose}\ \isakeyword{shows}\ \isanewline
{\isachardoublequoteopen}winningAllocationsRel\ N\ {\isacharparenleft}set\ G{\isacharparenright}\ b\ {\isacharequal}\ \isanewline
{\isacharparenleft}argmax\ {\isasymcirc}\ setsum{\isacharparenright}\ b\ {\isacharparenleft}set\ {\isacharparenleft}allStrictAllocations\ N\ G{\isacharparenright}{\isacharparenright}{\isachardoublequoteclose}\isanewline
%
\isadelimproof
%
\endisadelimproof
%
\isatagproof
\isacommand{unfolding}\isamarkupfalse%
\ allStrictAllocations{\isacharunderscore}def\isanewline
\isacommand{using}\isamarkupfalse%
\ assms\ lm{\isadigit{7}}{\isadigit{0}}c\ \isacommand{by}\isamarkupfalse%
\ {\isacharparenleft}metis\ comp{\isacharunderscore}apply{\isacharparenright}%
\endisatagproof
{\isafoldproof}%
%
\isadelimproof
\isanewline
%
\endisadelimproof
\isanewline
\isacommand{corollary}\isamarkupfalse%
\ lm{\isadigit{7}}{\isadigit{0}}g{\isacharcolon}\ \isakeyword{assumes}\ {\isachardoublequoteopen}card\ N\ {\isachargreater}\ {\isadigit{0}}{\isachardoublequoteclose}\ {\isachardoublequoteopen}distinct\ G{\isachardoublequoteclose}\ \isakeyword{shows}\isanewline
{\isachardoublequoteopen}chosenAllocation{\isacharprime}\ N\ G\ b\ r\ {\isacharequal}\ chosenAllocation\ N\ G\ b\ r{\isachardoublequoteclose}\ \isanewline
%
\isadelimproof
%
\endisadelimproof
%
\isatagproof
\isacommand{unfolding}\isamarkupfalse%
\ chosenAllocation{\isacharunderscore}def\ \isacommand{using}\isamarkupfalse%
\ assms\ lm{\isadigit{7}}{\isadigit{0}}f\ allStrictAllocations{\isacharunderscore}def\ \isacommand{by}\isamarkupfalse%
\ {\isacharparenleft}metis{\isacharparenleft}no{\isacharunderscore}types{\isacharparenright}{\isacharparenright}%
\endisatagproof
{\isafoldproof}%
%
\isadelimproof
\ \isanewline
%
\endisadelimproof
\isacommand{corollary}\isamarkupfalse%
\ lm{\isadigit{1}}{\isadigit{3}}b{\isacharcolon}\ \isakeyword{assumes}\ {\isachardoublequoteopen}x\ {\isasymin}\ {\isacharparenleft}N\ {\isasymtimes}\ {\isacharparenleft}Pow\ G\ {\isacharminus}\ {\isacharbraceleft}{\isacharbraceleft}{\isacharbraceright}{\isacharbraceright}{\isacharparenright}{\isacharparenright}{\isachardoublequoteclose}\ \isakeyword{shows}\ {\isachardoublequoteopen}tiebids{\isacharprime}\ a\ N\ G\ x\ {\isacharequal}\ tiebids\ a\ N\ G\ x{\isachardoublequoteclose}\ {\isacharparenleft}\isakeyword{is}\ {\isachardoublequoteopen}{\isacharquery}L{\isacharequal}{\isacharunderscore}{\isachardoublequoteclose}{\isacharparenright}\ \isanewline
%
\isadelimproof
%
\endisadelimproof
%
\isatagproof
\isacommand{proof}\isamarkupfalse%
\ {\isacharminus}\ \isanewline
\isacommand{have}\isamarkupfalse%
\ {\isachardoublequoteopen}{\isacharquery}L\ {\isacharequal}\ summedBidVector{\isacharprime}\ {\isacharparenleft}maxbid{\isacharprime}\ a\ N\ G{\isacharparenright}\ N\ G\ x{\isachardoublequoteclose}\ \isacommand{by}\isamarkupfalse%
\ fast\ \isacommand{moreover}\isamarkupfalse%
\ \isacommand{have}\isamarkupfalse%
\ {\isachardoublequoteopen}{\isachardot}{\isachardot}{\isachardot}{\isacharequal}\ \isanewline
summedBidVector\ {\isacharparenleft}maxbid\ a\ N\ G{\isacharparenright}\ N\ G\ x{\isachardoublequoteclose}\ \isacommand{using}\isamarkupfalse%
\ assms\ \isacommand{by}\isamarkupfalse%
\ {\isacharparenleft}rule\ lm{\isadigit{1}}{\isadigit{3}}{\isacharparenright}\ \isacommand{ultimately}\isamarkupfalse%
\ \isacommand{show}\isamarkupfalse%
\ {\isacharquery}thesis\ \isanewline
\isacommand{unfolding}\isamarkupfalse%
\ tiebids{\isacharunderscore}def\ \isacommand{by}\isamarkupfalse%
\ fast\isanewline
\isacommand{qed}\isamarkupfalse%
%
\endisatagproof
{\isafoldproof}%
%
\isadelimproof
\ \isanewline
%
\endisadelimproof
\isanewline
\isacommand{lemma}\isamarkupfalse%
\ lm{\isadigit{1}}{\isadigit{4}}{\isacharcolon}\ \isakeyword{assumes}\ {\isachardoublequoteopen}card\ N\ {\isachargreater}\ {\isadigit{0}}{\isachardoublequoteclose}\ {\isachardoublequoteopen}distinct\ G{\isachardoublequoteclose}\ {\isachardoublequoteopen}a\ {\isasymsubseteq}\ {\isacharparenleft}N\ {\isasymtimes}\ {\isacharparenleft}Pow\ {\isacharparenleft}set\ G{\isacharparenright}\ {\isacharminus}\ {\isacharbraceleft}{\isacharbraceleft}{\isacharbraceright}{\isacharbraceright}{\isacharparenright}{\isacharparenright}{\isachardoublequoteclose}\ \isakeyword{shows}\isanewline
{\isachardoublequoteopen}setsum\ {\isacharparenleft}resolvingBid{\isacharprime}\ N\ G\ b\ r{\isacharparenright}\ a\ {\isacharequal}\ setsum\ {\isacharparenleft}resolvingBid\ N\ G\ b\ r{\isacharparenright}\ a{\isachardoublequoteclose}\ {\isacharparenleft}\isakeyword{is}\ {\isachardoublequoteopen}{\isacharquery}L{\isacharequal}{\isacharquery}R{\isachardoublequoteclose}{\isacharparenright}\ \isanewline
%
\isadelimproof
\isanewline
%
\endisadelimproof
%
\isatagproof
\isacommand{proof}\isamarkupfalse%
\ {\isacharminus}\ \isanewline
\isacommand{let}\isamarkupfalse%
\ {\isacharquery}c{\isacharprime}{\isacharequal}{\isachardoublequoteopen}chosenAllocation{\isacharprime}\ N\ G\ b\ r{\isachardoublequoteclose}\ \isacommand{let}\isamarkupfalse%
\ {\isacharquery}c{\isacharequal}{\isachardoublequoteopen}chosenAllocation\ N\ G\ b\ r{\isachardoublequoteclose}\ \isacommand{let}\isamarkupfalse%
\ {\isacharquery}r{\isacharprime}{\isacharequal}{\isachardoublequoteopen}resolvingBid{\isacharprime}\ N\ G\ b\ r{\isachardoublequoteclose}\isanewline
\isacommand{have}\isamarkupfalse%
\ {\isachardoublequoteopen}{\isacharquery}c{\isacharprime}\ {\isacharequal}\ {\isacharquery}c{\isachardoublequoteclose}\ \isacommand{using}\isamarkupfalse%
\ assms{\isacharparenleft}{\isadigit{1}}{\isacharcomma}{\isadigit{2}}{\isacharparenright}\ \isacommand{by}\isamarkupfalse%
\ {\isacharparenleft}rule\ lm{\isadigit{7}}{\isadigit{0}}g{\isacharparenright}\ \isacommand{then}\isamarkupfalse%
\isanewline
\isacommand{have}\isamarkupfalse%
\ {\isachardoublequoteopen}{\isacharquery}r{\isacharprime}\ {\isacharequal}\ tiebids{\isacharprime}\ {\isacharquery}c\ N\ {\isacharparenleft}set\ G{\isacharparenright}{\isachardoublequoteclose}\ \isacommand{by}\isamarkupfalse%
\ presburger\isanewline
\isacommand{moreover}\isamarkupfalse%
\ \isacommand{have}\isamarkupfalse%
\ {\isachardoublequoteopen}{\isasymforall}x\ {\isasymin}\ a{\isachardot}\ tiebids{\isacharprime}\ {\isacharquery}c\ N\ {\isacharparenleft}set\ G{\isacharparenright}\ x\ {\isacharequal}\ tiebids\ {\isacharquery}c\ N\ {\isacharparenleft}set\ G{\isacharparenright}\ x{\isachardoublequoteclose}\ \isacommand{using}\isamarkupfalse%
\ assms{\isacharparenleft}{\isadigit{3}}{\isacharparenright}\ lm{\isadigit{1}}{\isadigit{3}}b\ \isacommand{by}\isamarkupfalse%
\ blast\isanewline
\isacommand{ultimately}\isamarkupfalse%
\ \isacommand{have}\isamarkupfalse%
\ {\isachardoublequoteopen}{\isasymforall}x\ {\isasymin}\ a{\isachardot}\ {\isacharquery}r{\isacharprime}\ x\ {\isacharequal}\ resolvingBid\ N\ G\ b\ r\ x{\isachardoublequoteclose}\ \isacommand{unfolding}\isamarkupfalse%
\ resolvingBid{\isacharunderscore}def\ \isacommand{by}\isamarkupfalse%
\ presburger\isanewline
\isacommand{thus}\isamarkupfalse%
\ {\isacharquery}thesis\ \isacommand{using}\isamarkupfalse%
\ setsum{\isachardot}cong\ \isacommand{by}\isamarkupfalse%
\ simp\isanewline
\isacommand{qed}\isamarkupfalse%
%
\endisatagproof
{\isafoldproof}%
%
\isadelimproof
\isanewline
%
\endisadelimproof
\isacommand{lemma}\isamarkupfalse%
\ lm{\isadigit{1}}{\isadigit{5}}{\isacharcolon}\ {\isachardoublequoteopen}allAllocations\ N\ G\ {\isasymsubseteq}\ Pow\ {\isacharparenleft}N\ {\isasymtimes}\ {\isacharparenleft}Pow\ G\ {\isacharminus}\ {\isacharbraceleft}{\isacharbraceleft}{\isacharbraceright}{\isacharbraceright}{\isacharparenright}{\isacharparenright}{\isachardoublequoteclose}%
\isadelimproof
\ %
\endisadelimproof
%
\isatagproof
\isacommand{by}\isamarkupfalse%
\ {\isacharparenleft}metis\ PowI\ lm{\isadigit{4}}{\isadigit{0}}c\ subsetI{\isacharparenright}%
\endisatagproof
{\isafoldproof}%
%
\isadelimproof
%
\endisadelimproof
\isanewline
\isacommand{corollary}\isamarkupfalse%
\ lm{\isadigit{1}}{\isadigit{4}}b{\isacharcolon}\ \isakeyword{assumes}\ {\isachardoublequoteopen}card\ N\ {\isachargreater}\ {\isadigit{0}}{\isachardoublequoteclose}\ {\isachardoublequoteopen}distinct\ G{\isachardoublequoteclose}\ {\isachardoublequoteopen}a\ {\isasymin}\ allAllocations\ N\ {\isacharparenleft}set\ G{\isacharparenright}{\isachardoublequoteclose}\ \isanewline
\isakeyword{shows}\ {\isachardoublequoteopen}setsum\ {\isacharparenleft}resolvingBid{\isacharprime}\ N\ G\ b\ r{\isacharparenright}\ a\ {\isacharequal}\ setsum\ {\isacharparenleft}resolvingBid\ N\ G\ b\ r{\isacharparenright}\ a{\isachardoublequoteclose}\isanewline
%
\isadelimproof
%
\endisadelimproof
%
\isatagproof
\isacommand{proof}\isamarkupfalse%
\ {\isacharminus}\isanewline
\isacommand{have}\isamarkupfalse%
\ {\isachardoublequoteopen}a\ {\isasymsubseteq}\ N\ {\isasymtimes}\ {\isacharparenleft}Pow\ {\isacharparenleft}set\ G{\isacharparenright}\ {\isacharminus}\ {\isacharbraceleft}{\isacharbraceleft}{\isacharbraceright}{\isacharbraceright}{\isacharparenright}{\isachardoublequoteclose}\ \isacommand{using}\isamarkupfalse%
\ assms{\isacharparenleft}{\isadigit{3}}{\isacharparenright}\ lm{\isadigit{1}}{\isadigit{5}}\ \isacommand{by}\isamarkupfalse%
\ blast\ \isanewline
\isacommand{thus}\isamarkupfalse%
\ {\isacharquery}thesis\ \isacommand{using}\isamarkupfalse%
\ assms{\isacharparenleft}{\isadigit{1}}{\isacharcomma}{\isadigit{2}}{\isacharparenright}\ lm{\isadigit{1}}{\isadigit{4}}\ \isacommand{by}\isamarkupfalse%
\ blast\isanewline
\isacommand{qed}\isamarkupfalse%
%
\endisatagproof
{\isafoldproof}%
%
\isadelimproof
\isanewline
%
\endisadelimproof
\isanewline
\isacommand{corollary}\isamarkupfalse%
\ lm{\isadigit{1}}{\isadigit{4}}c{\isacharcolon}\ \isakeyword{assumes}\ {\isachardoublequoteopen}finite\ N{\isachardoublequoteclose}\ {\isachardoublequoteopen}distinct\ G{\isachardoublequoteclose}\ {\isachardoublequoteopen}a\ {\isasymin}\ maximalStrictAllocations{\isacharprime}\ N\ {\isacharparenleft}set\ G{\isacharparenright}\ b{\isachardoublequoteclose}\ \isanewline
\isakeyword{shows}\ {\isachardoublequoteopen}setsum\ {\isacharparenleft}randomBids{\isacharprime}\ N\ G\ b\ r{\isacharparenright}\ a\ {\isacharequal}\ setsum\ {\isacharparenleft}randomBids\ N\ G\ b\ r{\isacharparenright}\ a{\isachardoublequoteclose}\isanewline
%
\isadelimproof
%
\endisadelimproof
%
\isatagproof
\isacommand{proof}\isamarkupfalse%
\ {\isacharminus}\ \isanewline
\isacommand{have}\isamarkupfalse%
\ {\isachardoublequoteopen}card\ {\isacharparenleft}N{\isasymunion}{\isacharbraceleft}seller{\isacharbraceright}{\isacharparenright}{\isachargreater}{\isadigit{0}}{\isachardoublequoteclose}\ \isacommand{using}\isamarkupfalse%
\ assms{\isacharparenleft}{\isadigit{1}}{\isacharparenright}\ sup{\isacharunderscore}eq{\isacharunderscore}bot{\isacharunderscore}iff\ insert{\isacharunderscore}not{\isacharunderscore}empty\isanewline
\isacommand{by}\isamarkupfalse%
\ {\isacharparenleft}metis\ card{\isacharunderscore}gt{\isacharunderscore}{\isadigit{0}}{\isacharunderscore}iff\ finite{\isachardot}emptyI\ finite{\isachardot}insertI\ finite{\isacharunderscore}UnI{\isacharparenright}\isanewline
\isacommand{moreover}\isamarkupfalse%
\ \isacommand{have}\isamarkupfalse%
\ {\isachardoublequoteopen}distinct\ G{\isachardoublequoteclose}\ \isacommand{using}\isamarkupfalse%
\ assms{\isacharparenleft}{\isadigit{2}}{\isacharparenright}\ \isacommand{by}\isamarkupfalse%
\ simp\isanewline
\isacommand{moreover}\isamarkupfalse%
\ \isacommand{have}\isamarkupfalse%
\ {\isachardoublequoteopen}a\ {\isasymin}\ allAllocations\ {\isacharparenleft}N{\isasymunion}{\isacharbraceleft}seller{\isacharbraceright}{\isacharparenright}\ {\isacharparenleft}set\ G{\isacharparenright}{\isachardoublequoteclose}\ \isacommand{using}\isamarkupfalse%
\ assms{\isacharparenleft}{\isadigit{3}}{\isacharparenright}\ \isacommand{by}\isamarkupfalse%
\ fastforce\isanewline
\isacommand{ultimately}\isamarkupfalse%
\ \isacommand{show}\isamarkupfalse%
\ {\isacharquery}thesis\ \isacommand{unfolding}\isamarkupfalse%
\ randomBids{\isacharunderscore}def\ \isacommand{by}\isamarkupfalse%
\ {\isacharparenleft}rule\ lm{\isadigit{1}}{\isadigit{4}}b{\isacharparenright}\isanewline
\isacommand{qed}\isamarkupfalse%
%
\endisatagproof
{\isafoldproof}%
%
\isadelimproof
\isanewline
%
\endisadelimproof
\isanewline
\isacommand{lemma}\isamarkupfalse%
\ lm{\isadigit{1}}{\isadigit{6}}{\isacharcolon}\ \isakeyword{assumes}\ {\isachardoublequoteopen}{\isasymforall}x{\isasymin}X{\isachardot}\ f\ x\ {\isacharequal}\ g\ x{\isachardoublequoteclose}\ \isakeyword{shows}\ {\isachardoublequoteopen}argmax\ f\ X{\isacharequal}argmax\ g\ X{\isachardoublequoteclose}\ \isanewline
%
\isadelimproof
%
\endisadelimproof
%
\isatagproof
\isacommand{using}\isamarkupfalse%
\ assms\ argmaxLemma\ Collect{\isacharunderscore}cong\ image{\isacharunderscore}cong\ \isanewline
\isacommand{by}\isamarkupfalse%
\ {\isacharparenleft}metis{\isacharparenleft}no{\isacharunderscore}types{\isacharcomma}lifting{\isacharparenright}{\isacharparenright}%
\endisatagproof
{\isafoldproof}%
%
\isadelimproof
\isanewline
%
\endisadelimproof
\isanewline
\isanewline
\isanewline
\isacommand{corollary}\isamarkupfalse%
\ lm{\isadigit{9}}{\isadigit{2}}e{\isacharcolon}\ \isakeyword{assumes}\ {\isachardoublequoteopen}distinct\ G{\isachardoublequoteclose}\ {\isachardoublequoteopen}set\ G\ {\isasymnoteq}\ {\isacharbraceleft}{\isacharbraceright}{\isachardoublequoteclose}\ {\isachardoublequoteopen}finite\ N{\isachardoublequoteclose}\ \isakeyword{shows}\ \isanewline
{\isachardoublequoteopen}{\isadigit{1}}{\isacharequal}card\ {\isacharparenleft}argmax\ {\isacharparenleft}setsum\ {\isacharparenleft}randomBids\ N\ G\ b\ r{\isacharparenright}{\isacharparenright}\ {\isacharparenleft}maximalStrictAllocations{\isacharprime}\ N\ {\isacharparenleft}set\ G{\isacharparenright}\ b{\isacharparenright}{\isacharparenright}{\isachardoublequoteclose}\isanewline
%
\isadelimproof
%
\endisadelimproof
%
\isatagproof
\isacommand{using}\isamarkupfalse%
\ assms\ lm{\isadigit{9}}{\isadigit{2}}b\ lm{\isadigit{1}}{\isadigit{4}}c\ \isanewline
\isacommand{proof}\isamarkupfalse%
\ {\isacharminus}\isanewline
\isacommand{have}\isamarkupfalse%
\ {\isachardoublequoteopen}{\isasymforall}\ a\ {\isasymin}\ maximalStrictAllocations{\isacharprime}\ N\ {\isacharparenleft}set\ G{\isacharparenright}\ b{\isachardot}\ \isanewline
setsum\ {\isacharparenleft}randomBids{\isacharprime}\ N\ G\ b\ r{\isacharparenright}\ a\ {\isacharequal}\ setsum\ {\isacharparenleft}randomBids\ N\ G\ b\ r{\isacharparenright}\ a{\isachardoublequoteclose}\ \isacommand{using}\isamarkupfalse%
\ assms{\isacharparenleft}{\isadigit{3}}{\isacharcomma}{\isadigit{1}}{\isacharparenright}\ lm{\isadigit{1}}{\isadigit{4}}c\ \isacommand{by}\isamarkupfalse%
\ blast\isanewline
\isacommand{then}\isamarkupfalse%
\ \isacommand{have}\isamarkupfalse%
\ {\isachardoublequoteopen}argmax\ {\isacharparenleft}setsum\ {\isacharparenleft}randomBids\ N\ G\ b\ r{\isacharparenright}{\isacharparenright}\ {\isacharparenleft}maximalStrictAllocations{\isacharprime}\ N\ {\isacharparenleft}set\ G{\isacharparenright}\ b{\isacharparenright}\ {\isacharequal}\isanewline
argmax\ {\isacharparenleft}setsum\ {\isacharparenleft}randomBids{\isacharprime}\ N\ G\ b\ r{\isacharparenright}{\isacharparenright}\ {\isacharparenleft}maximalStrictAllocations{\isacharprime}\ N\ {\isacharparenleft}set\ G{\isacharparenright}\ b{\isacharparenright}{\isachardoublequoteclose}\ \isacommand{using}\isamarkupfalse%
\ lm{\isadigit{1}}{\isadigit{6}}\ \isacommand{by}\isamarkupfalse%
\ blast\isanewline
\isacommand{moreover}\isamarkupfalse%
\ \isacommand{have}\isamarkupfalse%
\ {\isachardoublequoteopen}card\ {\isachardot}{\isachardot}{\isachardot}\ {\isacharequal}\ {\isadigit{1}}{\isachardoublequoteclose}\ \isacommand{using}\isamarkupfalse%
\ assms\ \isacommand{by}\isamarkupfalse%
\ {\isacharparenleft}rule\ lm{\isadigit{9}}{\isadigit{2}}b{\isacharparenright}\isanewline
\isacommand{ultimately}\isamarkupfalse%
\ \isacommand{show}\isamarkupfalse%
\ {\isacharquery}thesis\ \isacommand{by}\isamarkupfalse%
\ presburger\isanewline
\isacommand{qed}\isamarkupfalse%
%
\endisatagproof
{\isafoldproof}%
%
\isadelimproof
\isanewline
%
\endisadelimproof
\isacommand{corollary}\isamarkupfalse%
\ lm{\isadigit{7}}{\isadigit{0}}e{\isacharcolon}\ \isakeyword{assumes}\ {\isachardoublequoteopen}finite\ N{\isachardoublequoteclose}\ {\isachardoublequoteopen}distinct\ G{\isachardoublequoteclose}\ \isakeyword{shows}\isanewline
{\isachardoublequoteopen}maximalStrictAllocations{\isacharprime}\ N\ {\isacharparenleft}set\ G{\isacharparenright}\ b{\isacharequal}maximalStrictAllocations\ N\ G\ b{\isachardoublequoteclose}\ \isanewline
%
\isadelimproof
%
\endisadelimproof
%
\isatagproof
\isacommand{proof}\isamarkupfalse%
\ {\isacharminus}\isanewline
\isacommand{let}\isamarkupfalse%
\ {\isacharquery}N{\isacharequal}{\isachardoublequoteopen}{\isacharbraceleft}seller{\isacharbraceright}\ {\isasymunion}\ N{\isachardoublequoteclose}\ \isanewline
\isacommand{have}\isamarkupfalse%
\ {\isachardoublequoteopen}card\ {\isacharquery}N{\isachargreater}{\isadigit{0}}{\isachardoublequoteclose}\ \isacommand{using}\isamarkupfalse%
\ assms{\isacharparenleft}{\isadigit{1}}{\isacharparenright}\ \isacommand{by}\isamarkupfalse%
\ {\isacharparenleft}metis\ {\isacharparenleft}full{\isacharunderscore}types{\isacharparenright}\ card{\isacharunderscore}gt{\isacharunderscore}{\isadigit{0}}{\isacharunderscore}iff\ finite{\isacharunderscore}insert\ insert{\isacharunderscore}is{\isacharunderscore}Un\ insert{\isacharunderscore}not{\isacharunderscore}empty{\isacharparenright}\isanewline
\isacommand{thus}\isamarkupfalse%
\ {\isacharquery}thesis\ \isacommand{using}\isamarkupfalse%
\ assms{\isacharparenleft}{\isadigit{2}}{\isacharparenright}\ lm{\isadigit{7}}{\isadigit{0}}d\ \isacommand{by}\isamarkupfalse%
\ metis\isanewline
\isacommand{qed}\isamarkupfalse%
%
\endisatagproof
{\isafoldproof}%
%
\isadelimproof
\isanewline
%
\endisadelimproof
\isacommand{corollary}\isamarkupfalse%
\ \isakeyword{assumes}\ {\isachardoublequoteopen}distinct\ G{\isachardoublequoteclose}\ {\isachardoublequoteopen}set\ G\ {\isasymnoteq}\ {\isacharbraceleft}{\isacharbraceright}{\isachardoublequoteclose}\ {\isachardoublequoteopen}finite\ N{\isachardoublequoteclose}\ \isakeyword{shows}\ \isanewline
{\isachardoublequoteopen}{\isadigit{1}}{\isacharequal}card\ {\isacharparenleft}argmax\ {\isacharparenleft}setsum\ {\isacharparenleft}randomBids\ N\ G\ b\ r{\isacharparenright}{\isacharparenright}\ {\isacharparenleft}maximalStrictAllocations\ N\ G\ b{\isacharparenright}{\isacharparenright}{\isachardoublequoteclose}\isanewline
%
\isadelimproof
%
\endisadelimproof
%
\isatagproof
\isacommand{proof}\isamarkupfalse%
\ {\isacharminus}\ \isanewline
\isacommand{have}\isamarkupfalse%
\ {\isachardoublequoteopen}{\isadigit{1}}{\isacharequal}card\ {\isacharparenleft}argmax\ {\isacharparenleft}setsum\ {\isacharparenleft}randomBids\ N\ G\ b\ r{\isacharparenright}{\isacharparenright}\ {\isacharparenleft}maximalStrictAllocations{\isacharprime}\ N\ {\isacharparenleft}set\ G{\isacharparenright}\ b{\isacharparenright}{\isacharparenright}{\isachardoublequoteclose}\isanewline
\isacommand{using}\isamarkupfalse%
\ assms\ \isacommand{by}\isamarkupfalse%
\ {\isacharparenleft}rule\ lm{\isadigit{9}}{\isadigit{2}}e{\isacharparenright}\isanewline
\isacommand{moreover}\isamarkupfalse%
\ \isacommand{have}\isamarkupfalse%
\ {\isachardoublequoteopen}maximalStrictAllocations{\isacharprime}\ N\ {\isacharparenleft}set\ G{\isacharparenright}\ b\ {\isacharequal}\ maximalStrictAllocations\ N\ G\ b{\isachardoublequoteclose}\ \isanewline
\isacommand{using}\isamarkupfalse%
\ assms{\isacharparenleft}{\isadigit{3}}{\isacharcomma}{\isadigit{1}}{\isacharparenright}\ \isacommand{by}\isamarkupfalse%
\ {\isacharparenleft}rule\ lm{\isadigit{7}}{\isadigit{0}}e{\isacharparenright}\ \isacommand{ultimately}\isamarkupfalse%
\ \isacommand{show}\isamarkupfalse%
\ {\isacharquery}thesis\ \isacommand{by}\isamarkupfalse%
\ metis\isanewline
\isacommand{qed}\isamarkupfalse%
%
\endisatagproof
{\isafoldproof}%
%
\isadelimproof
\isanewline
%
\endisadelimproof
\isanewline
\isacommand{lemma}\isamarkupfalse%
\ {\isachardoublequoteopen}maximalStrictAllocations{\isacharprime}\ N\ {\isacharparenleft}set\ G{\isacharparenright}\ b\ {\isasymsubseteq}\ Pow\ {\isacharparenleft}{\isacharparenleft}{\isacharbraceleft}seller{\isacharbraceright}{\isasymunion}N{\isacharparenright}\ {\isasymtimes}\ {\isacharparenleft}Pow\ {\isacharparenleft}set\ G{\isacharparenright}\ {\isacharminus}\ {\isacharbraceleft}{\isacharbraceleft}{\isacharbraceright}{\isacharbraceright}{\isacharparenright}{\isacharparenright}{\isachardoublequoteclose}\isanewline
%
\isadelimproof
%
\endisadelimproof
%
\isatagproof
\isacommand{using}\isamarkupfalse%
\ lm{\isadigit{1}}{\isadigit{5}}\ UniformTieBreaking{\isachardot}lm{\isadigit{0}}{\isadigit{3}}\ subset{\isacharunderscore}trans\ \isacommand{by}\isamarkupfalse%
\ {\isacharparenleft}metis\ {\isacharparenleft}no{\isacharunderscore}types{\isacharparenright}{\isacharparenright}%
\endisatagproof
{\isafoldproof}%
%
\isadelimproof
\isanewline
%
\endisadelimproof
\isanewline
\isacommand{lemma}\isamarkupfalse%
\ lm{\isadigit{1}}{\isadigit{7}}{\isacharcolon}\ {\isachardoublequoteopen}{\isacharparenleft}image\ converse{\isacharparenright}\ {\isacharparenleft}Union\ X{\isacharparenright}{\isacharequal}Union\ {\isacharparenleft}{\isacharparenleft}image\ converse{\isacharparenright}\ {\isacharbackquote}\ X{\isacharparenright}{\isachardoublequoteclose}%
\isadelimproof
\ %
\endisadelimproof
%
\isatagproof
\isacommand{by}\isamarkupfalse%
\ auto%
\endisatagproof
{\isafoldproof}%
%
\isadelimproof
%
\endisadelimproof
\isanewline
\isanewline
\isacommand{lemma}\isamarkupfalse%
\ {\isachardoublequoteopen}possibleAllocationsRel\ N\ G\ {\isacharequal}\isanewline
Union\ {\isacharbraceleft}converse{\isacharbackquote}{\isacharparenleft}injections\ Y\ N{\isacharparenright}{\isacharbar}\ Y{\isachardot}\ Y\ {\isasymin}\ all{\isacharunderscore}partitions\ G{\isacharbraceright}{\isachardoublequoteclose}\ \isanewline
%
\isadelimproof
%
\endisadelimproof
%
\isatagproof
\isacommand{by}\isamarkupfalse%
\ auto%
\endisatagproof
{\isafoldproof}%
%
\isadelimproof
\isanewline
%
\endisadelimproof
\isanewline
\isacommand{lemma}\isamarkupfalse%
\ {\isachardoublequoteopen}allAllocations\ N\ {\isasymOmega}\ {\isacharequal}\ Union{\isacharbraceleft}{\isacharbraceleft}a{\isacharcircum}{\isacharminus}{\isadigit{1}}{\isacharbar}a{\isachardot}\ a{\isasymin}injections\ Y\ N{\isacharbraceright}{\isacharbar}Y{\isachardot}\ Y{\isasymin}all{\isacharunderscore}partitions\ {\isasymOmega}{\isacharbraceright}{\isachardoublequoteclose}%
\isadelimproof
\ %
\endisadelimproof
%
\isatagproof
\isacommand{by}\isamarkupfalse%
\ auto%
\endisatagproof
{\isafoldproof}%
%
\isadelimproof
%
\endisadelimproof
\isanewline
\isacommand{term}\isamarkupfalse%
\ {\isachardoublequoteopen}{\isacharparenleft}{\isasymSum}i{\isasymin}X{\isachardot}\ f\ i{\isacharparenright}{\isachardoublequoteclose}\isanewline
\isacommand{term}\isamarkupfalse%
\ {\isachardoublequoteopen}{\isacharparenleft}{\isasymUnion}i{\isasymin}X{\isachardot}\ x\ i{\isacharparenright}{\isachardoublequoteclose}\isanewline
\isacommand{abbreviation}\isamarkupfalse%
\ {\isachardoublequoteopen}endowment\ a\ n\ {\isacharequal}{\isacharequal}\ a{\isacharcomma}{\isacharcomma}{\isacharcomma}n{\isachardoublequoteclose}\isanewline
\isacommand{abbreviation}\isamarkupfalse%
\ {\isachardoublequoteopen}vcgEndowment\ N\ G\ b\ r\ n{\isacharequal}{\isacharequal}endowment\ {\isacharparenleft}vcga\ N\ G\ b\ r{\isacharparenright}\ n{\isachardoublequoteclose}\isanewline
\isanewline
\isanewline
\isanewline
\isacommand{abbreviation}\isamarkupfalse%
\ {\isachardoublequoteopen}firstPriceP\ N\ {\isasymOmega}\ b\ r\ n\ {\isacharequal}{\isacharequal}\isanewline
b\ {\isacharparenleft}n{\isacharcomma}\ winningAllocationAlg\ N\ {\isasymOmega}\ r\ b{\isacharcomma}{\isacharcomma}\ n{\isacharparenright}{\isachardoublequoteclose}\isanewline
\isanewline
\isacommand{lemma}\isamarkupfalse%
\ \isakeyword{assumes}\ {\isachardoublequoteopen}{\isasymforall}\ X{\isachardot}\ b\ {\isacharparenleft}n{\isacharcomma}\ X{\isacharparenright}\ {\isasymge}\ {\isadigit{0}}{\isachardoublequoteclose}\ \isakeyword{shows}\isanewline
{\isachardoublequoteopen}firstPriceP\ N\ {\isasymOmega}\ b\ r\ n\ {\isasymge}\ {\isadigit{0}}{\isachardoublequoteclose}%
\isadelimproof
\ %
\endisadelimproof
%
\isatagproof
\isacommand{using}\isamarkupfalse%
\ assms\ \isacommand{by}\isamarkupfalse%
\ blast%
\endisatagproof
{\isafoldproof}%
%
\isadelimproof
%
\endisadelimproof
\isanewline
\isanewline
\isanewline
\isanewline
\isanewline
\isanewline
\isanewline
\isacommand{abbreviation}\isamarkupfalse%
\ {\isachardoublequoteopen}goods\ {\isacharequal}{\isacharequal}\ sorted{\isacharunderscore}list{\isacharunderscore}of{\isacharunderscore}set\ o\ Union\ o\ Range\ o\ Domain{\isachardoublequoteclose}\isanewline
\isanewline
\isanewline
\isanewline
\isacommand{abbreviation}\isamarkupfalse%
\ {\isachardoublequoteopen}allocationPrettyPrint{\isadigit{2}}\ a\ {\isacharequal}{\isacharequal}\ map\ {\isacharparenleft}{\isacharpercent}x{\isachardot}\ {\isacharparenleft}x{\isacharcomma}\ sorted{\isacharunderscore}list{\isacharunderscore}of{\isacharunderscore}set{\isacharparenleft}a{\isacharcomma}{\isacharcomma}x{\isacharparenright}{\isacharparenright}{\isacharparenright}\ {\isacharparenleft}{\isacharparenleft}sorted{\isacharunderscore}list{\isacharunderscore}of{\isacharunderscore}set\ {\isasymcirc}\ Domain{\isacharparenright}\ a{\isacharparenright}{\isachardoublequoteclose}\isanewline
\isacommand{definition}\isamarkupfalse%
\ {\isachardoublequoteopen}helper\ {\isacharparenleft}list{\isacharparenright}\ {\isacharequal}{\isacharequal}\ {\isacharparenleft}{\isacharparenleft}{\isacharparenleft}hd{\isasymcirc}hd{\isacharparenright}\ list{\isacharcomma}\ set\ {\isacharparenleft}list{\isacharbang}{\isadigit{1}}{\isacharparenright}{\isacharparenright}{\isacharcomma}\ hd{\isacharparenleft}list{\isacharbang}{\isadigit{2}}{\isacharparenright}{\isacharparenright}{\isachardoublequoteclose}\isanewline
\isacommand{definition}\isamarkupfalse%
\ {\isachardoublequoteopen}listBid{\isadigit{2}}funcBid\ listBid\ {\isacharequal}\ {\isacharparenleft}helper{\isacharbackquote}{\isacharparenleft}set\ listBid{\isacharparenright}{\isacharparenright}\ Elsee\ {\isacharparenleft}{\isadigit{0}}{\isacharcolon}{\isacharcolon}integer{\isacharparenright}{\isachardoublequoteclose}\isanewline
\isanewline
\isacommand{abbreviation}\isamarkupfalse%
\ {\isachardoublequoteopen}singleBidConverter\ x\ {\isacharequal}{\isacharequal}\ {\isacharparenleft}{\isacharparenleft}fst\ x{\isacharcomma}\ set\ {\isacharparenleft}{\isacharparenleft}fst\ o\ snd{\isacharparenright}\ x{\isacharparenright}{\isacharparenright}{\isacharcomma}\ {\isacharparenleft}snd\ o\ snd{\isacharparenright}\ x{\isacharparenright}{\isachardoublequoteclose}\isanewline
\isacommand{definition}\isamarkupfalse%
\ {\isachardoublequoteopen}Bid{\isadigit{2}}funcBid\ b\ {\isacharequal}\ set\ {\isacharparenleft}map\ singleBidConverter\ b{\isacharparenright}\ Elsee\ {\isacharparenleft}{\isadigit{0}}{\isacharcolon}{\isacharcolon}integer{\isacharparenright}{\isachardoublequoteclose}\isanewline
\isanewline
\isacommand{abbreviation}\isamarkupfalse%
\ {\isachardoublequoteopen}participantsSet\ b\ {\isacharequal}{\isacharequal}\ fst\ {\isacharbackquote}\ {\isacharparenleft}set\ b{\isacharparenright}{\isachardoublequoteclose}\isanewline
\isacommand{abbreviation}\isamarkupfalse%
\ {\isachardoublequoteopen}goodsList{\isadigit{2}}\ b\ {\isacharequal}{\isacharequal}\ sorted{\isacharunderscore}list{\isacharunderscore}of{\isacharunderscore}set\ {\isacharparenleft}Union\ {\isacharparenleft}{\isacharparenleft}set\ o\ fst\ o\ snd{\isacharparenright}\ {\isacharbackquote}{\isacharparenleft}set\ b{\isacharparenright}{\isacharparenright}{\isacharparenright}{\isachardoublequoteclose}\isanewline
\isanewline
\isacommand{definition}\isamarkupfalse%
\ {\isachardoublequoteopen}allocation\ b\ r\ {\isacharequal}\ {\isacharbraceleft}allocationPrettyPrint{\isadigit{2}}\ \isanewline
{\isacharparenleft}vcgaAlg\ {\isacharparenleft}{\isacharparenleft}participantsSet\ b{\isacharparenright}{\isacharparenright}\ {\isacharparenleft}goodsList{\isadigit{2}}\ b{\isacharparenright}\ {\isacharparenleft}Bid{\isadigit{2}}funcBid\ b{\isacharparenright}\ r{\isacharparenright}\isanewline
{\isacharbraceright}{\isachardoublequoteclose}\isanewline
\isanewline
\isacommand{definition}\isamarkupfalse%
\ {\isachardoublequoteopen}payments\ b\ r\ {\isacharequal}\ vcgpAlg\ {\isacharparenleft}{\isacharparenleft}participantsSet\ b{\isacharparenright}{\isacharparenright}\ {\isacharparenleft}goodsList{\isadigit{2}}\ b{\isacharparenright}\ {\isacharparenleft}Bid{\isadigit{2}}funcBid\ b{\isacharparenright}\ r{\isachardoublequoteclose}\isanewline
\isacommand{export{\isacharunderscore}code}\isamarkupfalse%
\ allocation\ payments\ chosenAllocationEff\ \isakeyword{in}\ Scala\ \isakeyword{module{\isacharunderscore}name}\ VCG\ \isakeyword{file}\ {\isachardoublequoteopen}{\isacharslash}dev{\isacharslash}shm{\isacharslash}VCG{\isachardot}scala{\isachardoublequoteclose}\isanewline
\isanewline
\isanewline
\isanewline
\isanewline
\isanewline
\isacommand{abbreviation}\isamarkupfalse%
\ {\isachardoublequoteopen}b{\isadigit{0}}{\isadigit{1}}\ {\isacharequal}{\isacharequal}\ \isanewline
{\isacharbraceleft}\isanewline
{\isacharparenleft}{\isacharparenleft}{\isadigit{1}}{\isacharcolon}{\isacharcolon}integer{\isacharcomma}{\isacharbraceleft}{\isadigit{1}}{\isadigit{1}}{\isacharcolon}{\isacharcolon}integer{\isacharcomma}\ {\isadigit{1}}{\isadigit{2}}{\isacharcomma}\ {\isadigit{1}}{\isadigit{3}}{\isacharbraceright}{\isacharparenright}{\isacharcomma}{\isadigit{2}}{\isadigit{0}}{\isacharcolon}{\isacharcolon}integer{\isacharparenright}{\isacharcomma}\isanewline
{\isacharparenleft}{\isacharparenleft}{\isadigit{1}}{\isacharcomma}{\isacharbraceleft}{\isadigit{1}}{\isadigit{1}}{\isacharcomma}{\isadigit{1}}{\isadigit{2}}{\isacharbraceright}{\isacharparenright}{\isacharcomma}{\isadigit{1}}{\isadigit{8}}{\isacharparenright}{\isacharcomma}\isanewline
{\isacharparenleft}{\isacharparenleft}{\isadigit{2}}{\isacharcomma}{\isacharbraceleft}{\isadigit{1}}{\isadigit{1}}{\isacharbraceright}{\isacharparenright}{\isacharcomma}{\isadigit{1}}{\isadigit{0}}{\isacharparenright}{\isacharcomma}\isanewline
{\isacharparenleft}{\isacharparenleft}{\isadigit{2}}{\isacharcomma}{\isacharbraceleft}{\isadigit{1}}{\isadigit{2}}{\isacharbraceright}{\isacharparenright}{\isacharcomma}{\isadigit{1}}{\isadigit{5}}{\isacharparenright}{\isacharcomma}\isanewline
{\isacharparenleft}{\isacharparenleft}{\isadigit{2}}{\isacharcomma}{\isacharbraceleft}{\isadigit{1}}{\isadigit{2}}{\isacharcomma}{\isadigit{1}}{\isadigit{3}}{\isacharbraceright}{\isacharparenright}{\isacharcomma}{\isadigit{1}}{\isadigit{8}}{\isacharparenright}{\isacharcomma}\isanewline
{\isacharparenleft}{\isacharparenleft}{\isadigit{3}}{\isacharcomma}{\isacharbraceleft}{\isadigit{1}}{\isadigit{1}}{\isacharbraceright}{\isacharparenright}{\isacharcomma}{\isadigit{2}}{\isacharparenright}{\isacharcomma}\isanewline
{\isacharparenleft}{\isacharparenleft}{\isadigit{3}}{\isacharcomma}{\isacharbraceleft}{\isadigit{1}}{\isadigit{1}}{\isacharcomma}{\isadigit{1}}{\isadigit{2}}{\isacharbraceright}{\isacharparenright}{\isacharcomma}{\isadigit{1}}{\isadigit{2}}{\isacharparenright}{\isacharcomma}\isanewline
{\isacharparenleft}{\isacharparenleft}{\isadigit{3}}{\isacharcomma}{\isacharbraceleft}{\isadigit{1}}{\isadigit{1}}{\isacharcomma}{\isadigit{1}}{\isadigit{3}}{\isacharbraceright}{\isacharparenright}{\isacharcomma}{\isadigit{1}}{\isadigit{7}}{\isacharparenright}{\isacharcomma}\isanewline
{\isacharparenleft}{\isacharparenleft}{\isadigit{3}}{\isacharcomma}{\isacharbraceleft}{\isadigit{1}}{\isadigit{2}}{\isacharcomma}{\isadigit{1}}{\isadigit{3}}{\isacharbraceright}{\isacharparenright}{\isacharcomma}{\isadigit{1}}{\isadigit{8}}{\isacharparenright}{\isacharcomma}\isanewline
{\isacharparenleft}{\isacharparenleft}{\isadigit{3}}{\isacharcomma}{\isacharbraceleft}{\isadigit{1}}{\isadigit{1}}{\isacharcomma}{\isadigit{1}}{\isadigit{2}}{\isacharcomma}{\isadigit{1}}{\isadigit{3}}{\isacharbraceright}{\isacharparenright}{\isacharcomma}{\isadigit{1}}{\isadigit{9}}{\isacharparenright}{\isacharcomma}\isanewline
{\isacharparenleft}{\isacharparenleft}{\isadigit{4}}{\isacharcomma}{\isacharbraceleft}{\isadigit{1}}{\isadigit{1}}{\isacharcomma}{\isadigit{1}}{\isadigit{2}}{\isacharcomma}{\isadigit{1}}{\isadigit{3}}{\isacharcomma}{\isadigit{1}}{\isadigit{4}}{\isacharcomma}{\isadigit{1}}{\isadigit{5}}{\isacharcomma}{\isadigit{1}}{\isadigit{6}}{\isacharbraceright}{\isacharparenright}{\isacharcomma}{\isadigit{1}}{\isadigit{9}}{\isacharparenright}\isanewline
{\isacharbraceright}{\isachardoublequoteclose}\isanewline
\isacommand{value}\isamarkupfalse%
\ {\isachardoublequoteopen}participants\ b{\isadigit{0}}{\isadigit{1}}{\isachardoublequoteclose}\isanewline
\isanewline
\isanewline
%
\isadelimtheory
\isanewline
%
\endisadelimtheory
%
\isatagtheory
\isacommand{end}\isamarkupfalse%
%
\endisatagtheory
{\isafoldtheory}%
%
\isadelimtheory
\isanewline
%
\endisadelimtheory
\isanewline
\end{isabellebody}%
%%% Local Variables:
%%% mode: latex
%%% TeX-master: "root"
%%% End:


%%% Local Variables:
%%% mode: latex
%%% TeX-master: "root"
%%% End:


% optional bibliography
\bibliographystyle{abbrv}
\bibliography{root}

\end{document}

%%% Local Variables:
%%% mode: latex
%%% TeX-master: t
%%% End:
